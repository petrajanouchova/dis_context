\environment env_dis
\startcomponent Apendix1
\chapter{Apendix 1 - tabulky}
\setuplocalinterlinespace[line=2.0ex]


\placetable[3_01]{Charakteristika komplexní společnosti dle Tainter (1988): kmenově uspořádaná společnost vs. státní organizace.}
{\bTABLE
\bTR \bTD Charakteristika společnosti dle Tainter (1988, 22-31) \eTD
\bTD Kmenově uspořádaná společnost ({\em Big Man society, chiefdom}) \eTD \bTD Státní uspořádání ({\em Early state}) \eTD \eTR
\bTR \bTD Rozložení moci \eTD
\bTD Úzká vrstva aristokratů, v čele s jedním panovníkem, dědičnost moci založena na příbuzenských vztazích \eTD \bTD Profesionalizovaná vrstva vládnoucích jedinců; centralizace moci \eTD \eTR
\bTR \bTD Legitimizace moci \eTD
\bTD Charisma, autorita a společenský status vládnoucího jedince, schopnost zajistit si následovníky; nestabilní \eTD \bTD Stabilní území, byrokratický aparát, organizované vojenské jednotky, centrální státní ideologie; relativně stabilní \eTD \eTR
\bTR \bTD Struktura společnosti \eTD
\bTD Homogenní společnost, bez výrazných společenských a ekonomických rozdílů, velký důraz na existující příbuzenské vztahy; kulturně a etnicky konzervativní společnost \eTD \bTD Heterogenní společnost, ekonomická a společenská specializace, stratifikace společnosti, nerovnoměrné rozložení moci a ekonomických prostředků; kulturní a etnická heterogeneita společnosti \eTD \eTR
\bTR \bTD Kulturní charakteristika \eTD
\bTD Ideologie slouží k legitimizaci moci vládnoucí vrstvy; náboženství není centrálně organizováno \eTD \bTD Jednotná státní ideologie a náboženství, sounáležitost na základě územního principu \eTD \eTR
\eTABLE}





























\stopcomponent