{\em 2017 Mgr. Petra Janouchová}

{\bf Univerzita Karlova}

{\bf Filozofická fakulta}

Ústav řeckých a latinských studií

studijní program: Historické vědy

studijní obor: Dějiny antického starověku

Disertační práce

{\em Hellénizace antické Thrákie ve světle epigrafických nálezů}

{\em Hellenization of Ancient Thrace based on epigraphic evidence}

{\em PhDr. Jan Souček, CSc.}

{\em 2017 Mgr. Petra Janouchová}

{\bf Prohlášení:}

Prohlašuji, že jsem disertační práci napsal/a samostatně s využitím pouze uvedených a řádně citovaných pramenů a literatury a že práce nebyla využita v rámci jiného vysokoškolského studia či k získání jiného nebo stejného titulu.

V Praze, dne

\ldots{}\ldots{}\ldots{}\ldots{}\ldots{}\ldots{}\ldots{}\ldots{}\ldots{}\ldots{}\ldots{}...

Jméno a příjmení

{\bf Abstrakt:}

Z území antické Thrákie z oblasti jihovýchodního Balkánu pochází více než 4600 převážně řecky psaných nápisů. Tyto nápisy poskytují jedinečný zdroj demografických a sociologických informací o tehdejší populaci, umožňující hodnotit případnou proměnu vzorců chování v reakci na mezikulturní kontakt a vývoj společenské organizace. Řecky psané nápisy bývají považovány za jeden ze základních projevů hellénizace obyvatelstva antické Thrákie, tedy postupného a nevratného procesu adopce řecké kultury a identity. V této práci hodnotím na základě časoprostorové analýzy dochovaných nápisů relevanci hellénizace jako výchozího interpretačního rámce pro studium antické společnosti. Zároveň s tím uplatňuji alternativní přístup, který respektuje jednak specifika epigrafického materiálu, ale i poznatky současného bádání v oblasti mezikulturního kontaktu. Tento metodologický přístup umožňuje podrobně hodnotit produkci nápisů nejen v průřezu staletími, ale i navzájem srovnávat jednotlivé regiony a zapojení tamní populace. Z časoprostorové analýzy nápisů je zjevné, že rozvoj epigrafické produkce v Thrákii nemůže být spojován pouze s kulturním a politickým vlivem řeckých komunit, ale z velké části jde o jev úzce spojený s narůstající komplexitou politické organizace společnosti a následnými proměnami vzorců chování tehdejší populace. Tento jev je patrný zejména v době římské, kdy dochází k významnému rozvoji epigrafické produkce na celém území Thrákie, nikoliv pouze v okolí původně řeckých kolonií, a ve zvýšené míře i k zapojení thrácké populace do procesu publikace nápisů. Použitá metodologie je do velké míry inovativní kombinací řady moderních přístupů z příbuzných disciplín, avšak za zachování tradičních principů epigrafické práce. Zapojení digitální technologie umožňuje studovat nápisy z nové perspektivy a umístit je do regionálního kontextu.

{\bf Absctract:}

More than 4600 inscriptions in the Greek language come from Thrace, the area located in the Southeastern Balkan Peninsula. These inscriptions provide socio-demographic data, allowing the study of changing behavioural patterns in reaction to cross-cultural interactions. Traditionally, one of the essential indications of the influence of the Greek culture on the population of ancient Thrace was the practice of commissioning inscriptions in the Greek language. By using quantitative and systematic analysis, the inscriptions can be studied from a new perspective that places them into broader regional context. I use this methodology to assess the concept of Hellenization as one of the possible interpretative frameworks for the study of ancient society. Using a spatiotemporal analysis of inscriptions, this research shows that epigraphic production cannot be solely linked with the cultural and political influence of Greek speaking communities. However, the phenomenon of epigraphic production is closely connected to the growth of social complexity and consequent changes in the behavioural patterns of the population. The growth in social complexity is followed by an increase of epigraphic production of public and private character alike; while at the time of socio-economic crisis and political unrest, the production of inscriptions significantly drops. The sudden change in the character of epigraphic production is obvious in the Roman period, where the production substantially intensifies as a result of growing social complexity. Moreover, the Thracian population becomes more involved in the whole process of commissioning inscriptions as a result of their involvement in the civic and military service. The spatiotemporal analysis of inscriptions allows the discussion of the societal function of the epigraphic production over time, places them into broader regional context, and evaluates the degree of involvement of the population in the practice of commissioning of inscriptions. This research combines the specifics of epigraphic evidence and the current scholarship on crosscultural contact. This innovative methodological approach combines a range of modern theoretical concepts from related disciplines, such as archaeology and anthropology, while maintaining the epigraphic discipline best practices.

{\bf Klíčová slova:} nápisy; epigrafika; Thrákie; hellénizace; kulturní kontakt; digital humanities

{\bf Keywords:} inscriptions; epigraphy; Thrace; hellenization; cultural contact; digital humanities

{\bf Obsah}

\chapter{Úvod}
\section[představení-tématu]{Představení tématu}

Oblast antické Thrákie sloužila jako důležitá spojnice na pomezí Evropy a Asie, díky čemuž hrála důležitou roli ve vývoji historických událostí formujících antickou společnost. V průběhu několika století se stala svědkem řecké kolonizace, makedonské expanze a nadvlády, a konečně mocenského vzestupu a krize římské říše. Thrácká společnost byla v přímém kontaktu s různorodými světy řeckých {\em poleis}, hellénistických vládců i římské říše, a to na bázi diplomatické, obchodní i vojenské. Vzájemná interakce obyvatel Thrákie různého kulturního původu ovlivňovala každodenní život aristokracie i běžných obyvatel, což vedlo k postupným změnám nejen ve struktuře společnosti, ale i v ideových a materiálních projevech tamní kultury.

V 19. a 20. století byla Thrákie v duchu tehdy oblíbených akulturačních teorií považována za sféru vlivu řecké kultury, což mělo za následek nevyhnutelnou {\em hellénizaci} místního obyvatelstva. Hellénizační přístup sloužil v té době jako univerzální vysvětlení kulturních změn, k nimž docházelo při kontaktu místního obyvatelstva s řeckým světem, který byl považován za civilizačně vyspělejší. Neřecké komunity pak v rámci postupného a nevratného procesu hellénizace přijímaly novou materiální kulturu a ideologické koncepty na úkor vlastní kultury i identity vlastní a přijetí identity pořečtěné (Dietler 2005; Vranič 2014). Stejný přístup byl aplikován i na projevy epigrafické produkce, aniž by byla vzata v potaz specifika epigrafického materiálu a zhodnocena výpovědní hodnota nápisů v rámci dané problematiky. Užívání řeckého písma a vydávání nápisů v řeckém jazyce bylo automaticky považováno za jeden z typických znaků pořečtění thráckého obyvatelstva a přijmutí řecké kultury za vlastní (Mihailov 1977, 343-344; Sharankov 2011, 135-145).

\section[cíle]{Cíle}

Tato práce si klade za cíl zhodnotit na základě studia dochovaných epigrafických pramenů, zda je na počátku 21. století koncept {\em hellénizace} vhodným interpretačním rámcem vysvětlujícím vznik a rozvoj epigrafické produkce na území Thrákie, či je možné ho nahradit novými metodologickými přístupy. Tato práce si klade za cíl kriticky zhodnotit, nakolik je možné i nadále používat mnohdy jednostranně zaměřený přístup hellénizace ke studiu epigrafické produkce z území Thrákie, a nakolik se jedná v dnešní době již o překonaný teoretický koncept.

Dalším cílem této práce je charakterizovat epigrafickou produkci z území Thrákie v co možná nejucelenější podobě a zhodnotit jaký význam měla nápisná kultura v tehdejší společnosti. Jak je známo, přístup společnosti k epigrafické produkci se vyvíjel nejen v závislosti na čase s ohledem na tehdejší politické události, ale významnou roli hrály i zeměpisné a demografické faktory, což se odráží i na uspořádání této práce. Jedním z cílů této studie je vytvořit chronologický průřez společností antické Thrákie od 6. st. př. n. l. do 5. st. n. l. a zhodnotit, jakou roli v ní zaujímal zvyk publikovat nápisy, v jakých komunitách se úspěšně prosadil a v jakých nedošlo k jeho rozvoji a proč. Epigrafické prameny slouží jako primární zdroj informací k diskuzi o proměňujících se demografických trendech, kulturních zvyklostech a složení společnosti antické Thrákie. Dochované nápisy umožňují mapovat kulturní změny a projevy těchto změn v epigrafické kultuře, s cílem postihnout důvody, proč k těmto změnám docházelo a jejich dopad na tehdejší společnost. V neposlední řadě mě zajímá, zda je možné spojovat epigrafickou produkci s projevy vlivu řeckých obcí nehledě na časové období, či zda je nárůst epigrafické produkce spojen spíše s měnící se úrovní společenské komplexity a přístupem tehdejší politické autority.

\section[hypotézy-a-zvolená-metoda]{Hypotézy a zvolená metoda}

V disertační práci se snažím prokázat, zda došlo k rozšíření epigrafické produkce v Thrákii v souvislosti s aktivitami politických autorit, spíše než jako projev kulturního vlivu řecké společnosti a jako důsledek setkávání thrácké a řecké kultury. Ve snaze o objektivní přístup vůči kontaktům řecko-římské a místní thrácké kultury přistupuji k mapování změn společnosti antické Thrákie na základě zasazení do časového a místního kontextu jednotlivých komunit a zhodnocení konkrétních funkcí, které nápisy v tehdejší společnosti zastávaly. Zajímají mě především změny politického uspořádání, složení struktury populace a projevy kulturních zvyklostí v závislosti na probíhajících mezikulturních kontaktech a vzájemném ovlivňování thrácké a řecké, případně makedonské a římské společnosti.

Jako hlavní pramen používám dochované a publikované nápisy, pocházející z oblasti Thrákie. Nápisy využívám jako primární zdroj informací zejména pro širokou škálu součástí tehdejší společnosti, kterou pokrývají, počínaje pohřebními zvyklostmi, přes náboženství, etnické složení populace až po politické uspořádání komunit. Charakter nápisů umožňuje jednotlivé exempláře poměrně dobře zařadit do konkrétního historického rámce a určit jejich provenienci. S tím úzce souvisí fakt, že nápisy představují jeden z nejlépe zdokumentovaných historických pramenů z dané oblasti, pokrývající více než tisíc let vývoje společnosti. Díky relativně velkému počtu dochovaných nápisů je tak možné provádět relevantní kvantitativní a kvalitativní analýzy za účelem sledování proměn tehdejší společnosti.

Za tímto účelem jsem shromáždila přes 4600 nápisů nalezených na území Thrákie v databázi {\em Hellenization of Ancient Thrace}, která je výchozím pramenným souborem této práce. Struktura databáze respektuje tradiční principy epigrafické disciplíny doplněné o moderní metodologické přístupy a technologie, zefektivňující analýzu velkého množství dat. Analýzy nápisů provádím s pomocí moderní mapovací a statistické technologie a metod, dosud používaných zejména v archeologickém prostředí. Výsledky srovnávám s dochovanými literárními a archeologickými prameny ve snaze dotvořit celkový obraz proměn společnosti antické Thrákie.

\section[význam-studie-v-rámci-zvolené-problematiky]{Význam studie v rámci zvolené problematiky}

Komplexní práce, která by hodnotila epigrafickou produkci antické Thrákie v celé její šíři a zároveň reflektovala teoretické přístupy, v současné době chybí. Podobné souhrnné práce vznikaly zejména v 80. létech 20. století (Mihailov 1977; Samsaris 1980; Samsaris 1989; Loukopoulou 1989), nicméně od té doby došlo nejen k proměně metodologického přístupu k materiálu, ale zejména k objevení stovek dalších nápisů a k vydání zhruba 1500 dosud nepublikovaných nápisů. Tato práce se snaží tento nedostatek napravit a pojednává o stavu dochované epigrafické produkce publikované do r. 2016 v co možná nejúplnější podobě, reflektující vývoj současné metodologie.

Široký geografický záběr této práce a snaha sjednotit zdroje různé povahy je v současné době ojedinělý. Jednotlivé epigrafické korpusy, které současná práce kombinuje, jsou především založeny na principu moderních národů a málokterý zdroj kombinuje nápisy pocházející z území dnešního Bulharska, Řecka a Turecka, na jejichž území se antická Thrákie rozkládala. Tyto korpusy jsou nejen omezeny na území moderních států, ale často jsou psány různými jazyky, což znesnadňuje přístup k datům a omezuje výzkum nápisů na menší oblasti a regiony (např. Mihailov 1977; Loukopoulou 1989). Tato práce se snaží reagovat na tuto roztříštěnost zdrojů, nejednotnost metodologických přístupů a omezenou dostupnost v digitální formě. Základem práce je převedení nápisů do jednotného digitálního formátu, který je vhodný pro další zpracování, a jejich uchování v podobě databáze. Sjednocením metodologických přístupů editorů jednotlivých korpusů došlo k převedení různorodých dat do koherentní a navzájem srovnatelné formy, což dříve nebylo možné. Zároveň další výhodou je jejich převedení do elektronické podoby, což usnadňuje budoucí využití celého souboru pro další projekty.

Zcela v duchu moderních principů vědecké práce jsou analyzovaná data volně dostupná na internetu a kdokoliv je může využít v rámci svého projektu, navázat na ně, či naopak zhodnotit platnost zde uváděných závěrů a interpretací.\footnote{\useURL[url1][https://github.com/petrajanouchova/hat_project][][{\em https://github.com/petrajanouchova/hat_project}]\from[url1]} Užitá metodologie je kombinací několika přístupů z oblasti epigrafiky, archeologie, sociologie a antropologie, a jejím hlavním cílem je sledovat proměny společnosti na základě systematického studia dochovaných nápisů. V současné práci je tento přístup použit na oblast Thrákie, nicméně stejně tak může být aplikován i na jiná místa, kde docházelo k setkávání dvou či více kultur, a odkud se dochoval dostatečný počet epigrafických památek.

\section[stručné-nastínění-obsahu-práce]{Stručné nastínění obsahu práce}

V úvodní kapitole seznamuji čtenáře s geografickým rozsahem území Thrákie a charakterem jejich obyvatel tak, jak o nich hovoří dochované antické literární prameny, ale i moderní sekundární literatura. V druhé kapitole pojednávám o teoretických přístupech problematiky mezikulturních kontaktů a jejich projevů v materiální kultuře z pohledu archeologie a historických věd. Hlavní pozornost věnuji revizi hellénizačního přístupu, jako jednoho v minulosti z nejčastěji používaných teoretických konceptů. Zároveň probírám současné teoretické přístupy zabývající se mezikulturním kontaktem a snažím se nabídnout alternativní teoretický směr, který by se věnoval všem zúčastněným stranám stejnou měrou. Ve třetí kapitole se zabývám specifiky epigrafického materiálu, s důrazem na teoretické zhodnocení přínosů nápisů pro studium antické společnosti. Dále zde podrobněji představuji základní principy, s nimiž přistupuji ke studiu epigrafického materiálu. Ve čtvrté kapitole se zabývám použitou metodologií a organizací práce, která vysvětluje uspořádání a obsah následujících analytických kapitol. Pátá kapitola nahlíží na analyzovaný soubor nápisů jako na celek a představuje jeho základní charakteristické rysy. Šestá kapitola představuje chronologický přehled datovaných nápisů a předkládá detailní náhled na epigrafickou produkci v jednotlivých stoletích. V sedmé kapitole se zabývám rozmístěním nápisů v krajině a vztahem mezi nárůstem společensko-politické organizace v Thrákii. V poslední osmé kapitole shrnuji výsledky současné práce a nastiňuji další možné směřování zvolené problematiky. Na text práce samotné navazují přílohy, které obsahují detailní informace k organizaci databáze, a zejména pak tabulky a grafy s výsledky analýz a soubor map časoprostorového rozmístění nápisů.

\section[literární-a-historiografické-prameny]{Literární a historiografické prameny}

Literární zmínky o Thrákii a Thrácích jsou však relativně hojné, vzhledem k tomu, že se jednalo o obyvatele nepříliš vzdáleného území, v nichž měli Řekové a později Římané eminentní zájmy, a mnoho Thráků žilo a sloužilo v řeckých městech (Xydopoulos 2010; Janouchová 2013) a v římské armádě (Dana 2013; Boyanov 2008; Boyanov 2012, 251-269).Literární prameny ve většině případů nezaměřují výhradně na popis Thrákie a jejích obyvatel, ale zmiňují se o oblasti spíše v rámci širšího výkladu, pro ilustraci celkového děje. I přes svou limitovanou výpovědní hodnotu měly literární prameny zásadní formativní vliv na tzv. hellénizační interpretační rámec, který sloužil jako nejčastěji používané vysvětlení společenských změn při kontaktu s řeckou kulturou (Dietler 2005, 33-47; Jones 1997, 33). Z tohoto důvodu věnuji obrazu Thrákie a Thráků v literárních pramenech poměrně velkou pozornost, ale stejně tak i charakteru těchto pramenů a jejich možné výpovědní hodnotě pro studium mezikulturních kontaktů.

\subsection[informativní-hodnota-literárních-pramenů]{Informativní hodnota literárních pramenů}

Literární historiografické prameny jsou důležitým zdrojem informacím, ale v případě Thrákie bohužel zdrojem jednostranným. Neexistence thrácké literární tradice má za následek, že veškeré dostupné historické a literární prameny pochází z nitra řecké či případně pozdější římské kultury, což do jisté míry omezuje jejich výpovědní hodnotu s ohledem na thrácké obyvatelstvo. Antická literární díla byla zpravidla psaná autory žijícími mimo danou oblast, kteří se velkou část informací dozvídali přes prostředníky, z doslechu, či dokonce zprostředkovaně s odstupem několika století, což snižuje míru autenticity jejich sdělení. Mnohá literární díla měla navíc své zcela specifické zaměření a jejich obsah byl motivován politickými či jinými cíli, a ne vždy reflektoval nestranně pojatou historickou skutečnost. Proto je k dochovaným literárním pramenům, které se zabývají Thrákií, vhodné přistupovat s jistou mírou kriticismu a nebrat je za popis skutečné historické reality, ale spíše za literární dílo, jehož autor neměl ve většině případů za cíl nestranný popis událostí a tehdejší společnosti (Derks 2009, 240).

Dalším neoddiskutovatelným faktem je i značný hellénocentrismus dochovaných literárních památek, který se částečně přenáší z raných řeckých pramenů i do pozdější latinsky psané literatury. Fakt, že tato díla byla vytvořena pro řecky či latinsky mluvící publikum, nikoliv pro Thráky samotné, ovlivnil jejich obsah a jistou míru přizpůsobení kultuře posluchače ({\em interpretatio graeca} na poli náboženských představ). Dalším omezením literárních pramenů je zaměření jejich výkladu na horní vrstvy společnosti, a to zejména na politické, náboženské a vojenské elity, a naopak pouze okrajové zmínky o běžné populaci. O mezikulturních kontaktech na úrovni každodenního života běžných lidí se tedy mnoho nedozvídáme ani z literárních pramenů a musíme tento druh informací hledat mezi řádky či se obracet na jiné druhy pramenů, jako jsou archeologické památky a nápisy.

\subsection[thrákie-v-literárních-pramenech]{Thrákie v literárních pramenech}

Thrákie patří k relativně dobře popsaným regionům antického světa, alespoň co se týká pobřežních oblastí, které byly v archaické době osídleny řecky mluvícími obyvateli.\footnote{Velmi užitečné informace o thráckém pobřeží, řeckých koloniích a obyvatelích poskytuje autor 4. st. př. n. l. Pseudo-Skylax ve svém Periplu, kap. 67, 1-10 (Shipley 2011, 69-71). Pseudo-Skylax uvádí, že po moři je možné obeplout thrácké pobřeží od řeky Strýmónu až k Séstu za dva dny a dvě noci, ze Séstu do Pontu pak za další dva dny a dvě noci. Odtud pak k řece Istru za tři dny a tři noci. Celkem je tedy dle něj možné obeplout Thrákii po moři za osm dní (67.10), {[}ač prostý součet je dní sedm, poznámka P. J.{]}. Dalším autorem popisujícím především řecká města na pobřeží je Pseudo-Skymnos (646-746), jehož dílo bývá datováno do 1. st. př. n. l. (Kazarow 1949, 143). Důležitým zdrojem z 1. st. n. l. je Klaudios Ptolemaios, který udává rozměry a uspořádání provincií {\em Thracia} a {\em Moesia Inferior} ({\em Geogr}. 3.10-11). Obecně avšak autoři římské doby udávají více detailů jako například vzdálenosti mezi městy (Arrián {\em Peripl. Ponti Euxini} 21-24).} Konkrétní hranice Thrákie se měnily v závislosti na politické situaci a na vnímání daného autora, nicméně pohoří a vodní plochy stanovily poměrně stabilní přírodní hranice. Tradičně dle antických pramenů území Thrákie začínalo již za makedonskou Piérií, ze západu bylo pak ohraničena řekami Strýmónem a Néstem, z jihu Egejským mořem, Marmarským mořem a Helléspontem, z východu pak Černým mořem. Severní hranice Thrákie jsou méně jasně geograficky ohraničené, ale většinou se za severní hranici považuje řeka {\em Istros} (dn. Dunaj) a pohoří {\em Haimos} (dn. Stara Planina, téže Balkán). Horské masivy Rodopy, Pirin a Rila tvořily relativně neprostupnou bariéru, oddělující pobřežní od vnitrozemské Thrákie, a tedy Řeky od Thráků. Spojnicí vnitrozemí se Středozemím tvořily zejména řeky {\em Hebros} (dn. Marica-Evros), {\em Tonzos} (dn. Tundža-Tonzos), {\em Strýmón} (dn. Struma-Strymónas) a {\em Néstos} (dn. Mesta-Néstos) a několik cest skrze horské průsmyky (Sears 2013, 7-8; Bouzek a Graninger 2015, 12-19; Theodossiev 2011, 2-4; Theodossiev 2014, 157).\footnote{Někteří autoři zařazují do území obývané Thráky oblasti Bíthýnie v Malé Asii, ostrovy Thasos a Samothráké, a území na severu sahající až k řece Morava (Theodossiev 2014, 157).}

Thrákie je v literárních pramenech z archaické a klasické doby známá pro své nerostné bohatství, úrodnou zemědělskou půdu, nevlídné a kruté zimy, vodnaté řeky, vysoké hory a široké planiny, na nichž se dařilo chovu koní, které zmiňuje již Homér (Hom. {\em Il}. 13.1-16; 10.484; Sears 2013, 31; Tsiafakis 2000). Již od dob Archilocha je Thrákie proslulá svými silným vínem a existencí bohatých zlatými a stříbrných dolů (Diehl frg. 2 a 51), k nimž se jako první z Řeků pokoušeli dostat kolonisté z ostrova Paros.\footnote{V pozdějších dobách se o zdroj nerostných surovin začali zajímat i Athéňané (např. Peisistratos, Hdt. 1.64.1; Isaac 1986; 14-15; Lavelle 1992, 14-22).} Hérodotos také zmiňuje i nadbytek dřeva vhodného ke stavbě lodí, jednu z hlavních komodit antického starověku (Hdt. 5.23.2). Celkově je možné říci, že Thrákie byla známá jako plodná, ale poměrně drsná krajina, a podobně řecké prameny líčily i její obyvatele.

\subsection[charakteristika-obyvatel-thrákie-v-literárních-pramenech]{Charakteristika obyvatel Thrákie v literárních pramenech}

První zmínky o Thrácích jako o spojencích pocházejí již z homérských eposů, kde thráčtí králové vystupují po boku řeckých panovníků.\footnote{Nejslavnější polo-mýtický král Rhésos byl znám díky svým pověstným bílým koním a nádherné zbroji (Hom. {\em Il}. 10.435, 495), které se lstí podaří získat Odysseovi.} V archaické literatuře se zmínky o Thrákii týkají spíše její geografie a o jejích obyvatelích se mluví vždy v souvislosti s krajinou, kterou obývají.\footnote{Do této kategorie spadá např. Zmínky v homérských eposech, homérských hymnech, Hésiodovi, Hekatáiovi, Archilochovi, Simonidovi a v díle dalších archaických básníků (Xydopoulos 2004, 18-20).} Od klasické doby se pak setkáváme se dvěma hlavními tématy prolínající se téměř všemi dochovanými literárními prameny: Thrákové jsou vnímáni jednak jako blízcí sousedé a často i spojenci, kteří se podobají společenským uspořádáním a zvyklostmi dávné řecké minulosti, a dále jako krutí a bojovní válečníci, s nimiž není radno přijít do sporu (Xydopoulos 2005, 600; Sears 2013, 148). Z těchto důvodů často hráli nezanedbatelnou roli v politickém vývoji severního Egejdy jako spojenci Athén. Zejména v 5. a 4. st. př. n. l. byli panovníci z thráckého kmen Odrysů vnímáni jako mocní spojenci, a to pro velikost jejich vojska, bohatství, kterým disponovali, a výhodnou geografickou pozici jimi ovládané části Thrákie.

\subsection[společenská-organizace-thráků-z-pohledu-vnějšího-pozorovatele]{Společenská organizace Thráků z pohledu vnějšího pozorovatele}

Dle Strabóna (7.7.4) bylo vnitrozemí směrem na východ od řeky Strýmónu obýváno thráckými kmeny, zatímco pobřeží bylo osídleno řecky mluvícími obyvateli, žijícími v městech. Hlavním zdrojem obživy bylo pro Thráky zemědělství, chov koní a dobytka, válečné výpravy a do omezené míry zde fungoval i obchod a nepeněžní výměna se Středomořím (Strabo 7.3.7).

Hérodotos (5.3) uvádí, že Thrákové jsou hned po Indech druhý nejpočetnější národ v tehdy známém světě.\footnote{Ač Hérodotos používá slovo národ či kmen (ἔθνος) a Thrákové byli v řeckém prostředí vnímáni a popisováni jako jedno etnikum, ve skutečnosti se jednalo o mnoho kmenů, které spadaly pod jednotné označení „thrácké”.} Sám Hérodotos uznává, že Thrákové jsou nejednotní, a tím přichází tak o značnou strategickou výhodu, jakou jim velké množství obyvatel přináší. Spojuje je však geografická blízkost obývaných území, a až na výjimky, podobné zvyklosti. Pozdější zdroj Plinius Starší uvádí jména 36 kmenů s relativní polohou oblasti, kterou obývají (Plin. {\em H. N.} 4.11), ale zároveň mluví o existenci 50 stratégií, tedy administrativních jednotek, které pravděpodobně vycházely z kmenového principu (Theodossiev 2011, 8).\footnote{Plin. H. N. 4.11. 40: {\em Thracia sequitur, inter validissimas Europae gentes, in strategias L divisa}.} Strabón uvádí pouze 22 kmenů pro celou Thrákii (7, frg. 48; 7.3.2), avšak dá se předpokládat, že jich celkem bylo až dvakrát tolik. Fol a Spiridonov (1983, 21-61) celkem sesbírali na 50 jmen kmenů, které je možné označit jako thrácké a které se vyskytovaly v písemných pramenech do poloviny 3. st. př. n. l. Některé z kmenů prameny zmiňují jen jednou či ojediněle, o jiných víme poměrně hodně detailů, jako např. o Odrysech, Getech, Sapaích atp.\footnote{Sears (2013, 9-13) uvádí nejdůležitější thrácké kmeny vyskytující se v řeckých literárních pramenech: Apsinthiové, Bessové, Bisaltové, Bíthýnové, Diové, Dolonkové, Édónové, Mygdóni, Thýnové, Odomanti, Odrysové a Satrové.}

Jednotlivým kmenům vládnou kmenoví vůdci či dle řeckých pramenů králové, kteří ve svých rukou soustředili jak politickou, tak ekonomickou moc Archibald (2015, 912).\footnote{Thúkýdidés a Hérodotos používají termín οἱ βασιλέες, králové (Thuc. 2.97; Hdt. 5.7; 6.34); Diodóros pak při popisu odryského panovníka Sitalka používá termín ὁ τῶν Θρᾳκῶν βασιλεὺς, král Thráků (D. S. 12.50).} S řecky mluvícími městy udržovali kontakty právě tito thráčtí aristokraté, kteří pocházeli z kmenů žijících v blízkosti řeckých sídlišť na pobřeží. Zásadní roli mezi thráckou aristokracií hrálo společenské postavení a status, který si aristokraté pečlivě budovali. Hérodotos nás informuje o rozšířené praxi mnohoženství, které bylo známkou společenské prestiže, protože několik žen si mohl dovolit jen bohatý člověk (Hdt. 5.5; Strabo 7.3.4). Další známkou společenského postavení byl zvyk přijímat, a nikoliv dávat bohaté dary, který popisuje Xenofón (Xen. {\em Anab.} 7.3.18-20). Čím vyšší bylo společenské postavení obdarovaného, tím větší a cennější dar se očekával. Úspěšný aristokrat si takto mohl opatřit poměrně velké bohatství, jako to Thúkýdidés popisuje u odryského panovníka Sitalka (2.97.3; D. S. 12.50). Bohatství pak Thrákové vystavovali na odiv v rámci společenských setkání, ale i zhotovováním nákladných hrobek a konáním pohřebních rituálů, které jsou u nich v oblibě (Hdt. 5.8).

Hérodotos dělí thrácké kmeny dle vztahu k Řekům a řecké kultuře jednak na civilizované kmeny, které jsou v kontaktu s řeckou komunitou a jsou do velké míry provázány společnými zájmy a žijí v dosahu pobřeží a řeckých kolonií,\footnote{Hdt. 7.110 a 7.115; Hdt. 6.34 např. kmen Dolonků.} a dále na kmeny žijící v horách, které jsou na Řecích nezávislé či jsou vůči nim nepřátelsky naladěné, mají bojovný charakter a udržují si své tradiční zvyky.\footnote{Hdt. 7.111 a 7.116, např. kmen Satrů či Hdt. 6.36 a 9.119 kmen Apsinthiů, Hdt. 5.124-126 kmen Édónů, Hdt. 6.45 kmen Brygů. V lidové tradici a mýtech je Thrákie je taktéž vnímána jako domov mýtického pěvce Orfea, a místo, kde se v odlehlých horách odehrávají divoké bakchické rituály (Pindar frg. 126.9; Paus. 6.20.18; Archibald 1999, 460).} Postoj thráckých aristokratů vůči řecké komunitě se různí, od nepřátelských kontaktů až po snahu některých jedinců o adopci řeckého stylu života a vzdělání.\footnote{Hdt. 4.78-80: Hérodotos popisuje situaci u Skýthů, přímých sousedů Thráků, u nichž mohla situace vypadat analogicky. Autor mluví o skýthském princi Skýlovi, který pocházel ze smíšeného svazku a uměl mluvit a psát řecky a byl nakloněn řeckému způsobu života, např. nosil řecký oděv, vyznával řecká božstva, a nakonec se usídlil v řeckém Borysthénu.} Ač jsou tito Thrákové řeckými prameny popisováni jako „hellénizovaní”, i nadále si udržují své typické zvyky a náboženství (Hdt. 5. 3-8).\footnote{Známá pasáž z Hérodota (Hdt 5. 7) je zářným příkladem: Hérodotos tvrdí, že Thrákové uctívají Area, Dionýsa, Artemidu a aristokraté ještě navíc Herma. Zůstává nadále otázkou, nakolik řecké prameny přizpůsobovaly reálie řeckému posluchači ({\em interpretatio Graeca}) a nakolik skutečně reflektovaly realitu. Nicméně srovnání s dostupnými archeologickými prameny poukazuje na uchování tradičního charakteru náboženství za užití řecké nomenklatury (Janouchová 2013a; Janouchová 2013b). Více k tomuto tématu v kapitole 6.} Ke kontaktům a prolínání kulturních zvyklostí mezi Thráky a Řeky docházelo i dle Helláníka (frg. 71a), který nazývá původně thrácké obyvatele polovičními Řeky, avšak tento popis se týká pouze thráckých kmenů žijících na pobřeží v těsné blízkosti řeckých měst.\footnote{Termín μιξέλληνες Helláníkos používá k popisu obyvatel thráckého pobřeží známého jako Kolpos Melas, mezi Thráckým Chersonésem a pevninou.}

Politické vztahy a vzájemné postavení řeckých měst a thráckých kmenů literární prameny přímo nezmiňují, nicméně Thúkýdidés nepřímo informuje o jistém druhu finanční nadřazenosti Thráků nad Řeky žijícími v Thrákii, když poukazuje na nemalé finanční částky, které kmen Odrysů vybíral právě od Řeků žijících na území Thrákie (Thuc. 2. 97). Kmen Odrysů si získal výsadní postavení v řecké historiografii, zejména pro svou důležitost v rámci athénské zahraniční politiky v 5. a ve 4. st. př. n. l., a tudíž nelze vztahovat vyjádření týkající se Odrysů obecně na všechny thrácké kmeny. Je však jisté, že Thrákové po většinu své historie netvořili jednotný stát, který by zahrnoval celou oblast Thrákie, ale spíše se jednalo o kmeny s podobnými zvyky a příbuznými dialekty (Theodossiev 2011, 2), které se dostávaly do kontaktu s řeckou komunitou.

Thrákie a thráčtí kmenoví vůdci byli pověstní svou nejednotností a odporem vůči jakékoliv autorit, tedy i vůči vlastním kmenovým vůdcům. Pokud však vysoce postavení Thrákové vycítili politickou příležitost, dokázali využít situace za účelem naplnění vlastních mocenských ambicí, a to i za cenu ztráty autonomie v dlouhodobém měřítku (Haynes 2011, 7). To je případ thráckých panovníků z 1. st. př. n. l. a 1. st. n. l., kteří se vzdali politické autonomie a stali se vazaly Říma výměnou za vlastní prospěch (Tac. {\em Ann}. 4.46-47; Lozanov 2015, 75).

\subsection[literární-topos-nevyzpytatelní-spojenci-a-nebezpeční-nepřátelé]{Literární topos: nevyzpytatelní spojenci a nebezpeční nepřátelé}

Obraz Thráků v řecké literatuře se do značné míry vyvíjí v závislosti na současné politické situaci, avšak i přesto jsou mnohými autory chápáni jako blízcí a mocní sousedé, kteří však nedosahují stejné civilizační a kulturní úrovně jako Řekové, viz dříve. Ač se může zdát, že tento motiv se prolíná napříč celou řecky psanou literaturou, ne vždy se muselo nutně jednat o pouhý popis skutečnosti, ale spíše o literární {\em topos}, jakýsi všeobecně přijímaný názor, zvolený záměrně autorem daného díla. Tento dualismus se do velké míry přenesl i do literatury pozdějšího období a stal se tak typickým stereotypem zobrazování Thráků v literatuře, ale i v například v malířském umění a ikonografii (Tsiafakis 2000, 388-389). V moderní době se tento dualistický popis Thráků stal součástí interpretací vycházejících z hellénizačního teoretického přístupu.

\subsubsection[thrákové-a-literární-žánr-historiografie-5.-a-4.-st.-př.-n.-l.]{Thrákové a literární žánr historiografie 5. a 4. st. př. n. l.}

Hérodotos popisuje Thráky jako skupinu lidí relativně podobnou řeckému posluchači a záměrně vyzdvihuje zajímavosti a jasné odlišnosti jako je užívaný pohřební ritus, přítomnost tetování, zvláštnosti ve výchově potomků, postavení žen ve společnosti, víra v nesmrtelnost (Hdt. 5.4-6). Jedním z cílů Hérodotova vyprávění bylo posluchače pobavit a zaujmout novými a zvláštními fakty (Asheri 1990, 162). Vzhledem k tomu, že v Athénách, kde Hérodotos také působil, a i v jiných městech žilo poměrně velké množství thráckého obyvatelstva, nemusel své posluchače seznamovat se základními skutečnostmi o Thrácích, jako tomu třeba bylo v případě Aithiopů či jiných národů z vzdálenějších a exotičtějších končin (Sears 2013, 149).\footnote{Jedna z nejznámějších zmínek o thráckém obyvatelstvu v Athénách pochází z díla samotného Platóna. Platón na začátku Ústavy popisuje epizodu, kdy se Sókratés šel podívat do Peiraea na noční slavnost {\em Bendideí}, pořádanou místními Thráky (Plat. {\em Resp}. 1.327a). Bendis byla původně thrácká bohyně, která byla uvedena do oficiálního athénského pantheonu pravděpodobně v souvislosti se spojenectvím s Odrysy v r. 431 př. n. l. Její slavnosti se ve 4. st. př. n. l. každoročně odehrávaly v athénském přístavu za hojné účasti Thráků, kteří zde trvale žili jako {\em metoikové}, či jako pomocníci v domácnosti, případně otroci ve stříbrných dolech (Archibald 1999, 457-460; Janouchová 2013, 98; Sears 2013, 149-157).} Proto je Hérodotovo líčení Thráků a Thrákie značně selektivní a útržkovité, protože se u posluchačstva předpokládala předchozí znalost tématu (Asheri 1990, 162).

Thúkýdidés, jehož rodina pravděpodobně pocházela z Thrákie, je považován za věrohodný zdroj informací, avšak i on podává svůj výklad za účelem osvětlit politické dění a průběh peloponnéské války.\footnote{Sears 2013, 14: Thúkýdidés, syn Olorův, jehož rodina pocházela z Thrákie (Thuc. 4.104), se sám považoval za Athéňana (Thuc. 1.1). Sám přiznává, že jeho rodina měla právo dolovat drahé kovy v Thrákii (Thuc. 4.105), a tudíž byl obeznámen se situací v Thrákii detailněji, než většina Athéňanů.} Když Thúkýdidés mluví o obyvatelích Thrákie, používá společné pojmenování Thrákové, ale také rozeznává jednotlivé kmeny a jejich odlišný politický status, tj. autonomní kmeny vs. závislé kmeny, jejich postoj vůči Řekům a oblast, kterou obývají (Thuc. 2.96). Ve většině případů Thúkýdidés mluví specificky o aristokratech z kmene Odrysů, kteří se stali athénskými spojenci v roce 431 př. n. l. (2.29, 2.95-97). Ve svém líčení zdůrazňuje svrchovanost Odrysů nad ostatními thráckými kmeny, velikost a důležitost oblasti, které vládnou. Thúkýdidés zde specificky používá termín říše, aby poukázal na její významnost (ἀρχή, Thuc. 2.97). O Odrysech nevyjadřuje jako o barbarech, ale poukazuje na jejich bohatství a moc, zejména aby ospravedlnil důvody vedoucí k uzavření spojenectví v roce 431 př. n. l.\footnote{V jiném kontextu se zmiňuje o blíže nespecifikovaných kmenech obývajících Rodopy, kteří byli podle něj nejbojovnější ze svobodných thráckých kmenů, nespadali pod odryskou říši (Thuc. 2.98.4), avšak přišli Odrysům na pomoc v nouzi. Dodává tak vážnost odryské říši i v rámci nezávislých thráckých kmenů a poukazuje na sílu thráckého vojska.}

V případech kdy Thúkýdidés popisuje Thráky jako hrubé a bojovné, je to vždy, aby podtrhl jejich kvality coby válečníků či aby zdůraznil vážnost konkrétních historických situací. Thúkýdidés nazývá Thráky barbarskými a krvelačnými pouze ve specifických kontextech, kde thráčtí vojáci sehráli roli v rámci peloponnéské války. V jedné z těchto epizod popisuje thrácké žoldnéře z kmene Diů jako jedny z nejkrvelačnějších barbarů, avšak vztahuje tuto charakteristiku obecně na celé etnikum (Thuc. 7.29).\footnote{Jedná se o epizodu související s athénskou výpravou na Sicílii, kdy Athéňané povolali thrácké žoldnéře, aby se zúčastnili výpravy. Ti však přijeli do Athén příliš pozdě a byli posláni zpět bez jakékoliv finanční náhrady. Athénský generál Dieitrefés a thráčtí vojáci se rozhodli opatřit si slíbené peníze vypleněním boiótského Mykaléssu a povražděním jeho obyvatel a vypleněním chrámů. Tato akce byla vnímána velmi negativně, zejména kvůli krutosti počínání Thráků, ale i vzhledem k faktu, že vedení se ujal athénský generál, který Thráky využil k vlastnímu finančnímu zisku (Kallet 2002, 140-146).} Ve většině případů Thrákové vystupují v Thúkýdidově díle jako sousedé a spojenci, jejichž vojenské schopnosti by Athéňané rádi využili ve svůj prospěch, avšak divoká povaha a bojovnost Thráků v tom často brání.

Další historik Xenofón v šesté a sedmé knize {\em Anabasis} popisuje své vlastní zkušenosti s Thráky během tažení perského prince Kýra mladšího a služby pro thráckého panovníka Seutha II.\footnote{Xenofón, spolu s dalšími Řeky působil jako najatý žoldnéř ve službách Kýra, a když Kýros zemřel, Řekové se dali na ústup zpět do Evropy. Xenofón se sám cestou dostal do služeb thráckého prince Seutha, pozdějšího odryského panovníka Seutha II.} Xenofón popisuje Thráky jako skvělé válečníky a při mnoha příležitostech zdůrazňuje jejich divoký a nespoutaný charakter. Zejména se vyjadřuje o Seuthovi jako o krutém panovníkovi, který se v rámci udržení své autority neváhal uchýlit k pálení vesnic a zabíjení lidí (Xen. {\em Anab.} 7.4.1; 7.4.6). Ač s ním Xenofón uzavřel dohodu a přísahal na vzájemné přátelství, to však nemělo dlouhého trvání (Xen. {\em Anab.} 7.3.30). Konkrétně o Seuthovi Xenofón vždy hovoří jako o Thrákovi, nikoliv o příslušníkovi kmene Odrysů, ač si byl vědom i existence kmenové organizace mezi Thráky (Xen. {\em Anab.} 7.2.22). Specifika thráckých zvyklostí a společenského uspořádání zdůrazňoval zejména v případech, kdy se odlišovaly od řeckých či kdy se jejich neznalostí sám dostal do potíží (Xen. {\em Anab.} 7.3.19-21). Matthew Sears navrhuje (2013, 115), že se Xenofón takto ostře vymezoval vůči Seuthovi a ostatním Thrákům, aby vyvrátil případné domněnky o své náklonnosti vůči Thrákům a zejména jejich bohatství a zároveň aby ospravedlnil své činy v očích řeckého čtenáře. Seuthés totiž nabídl Xenofóntovi za své služby nejen peněžní odměnu, sídlo v Bisanthé na thráckém pobřeží, ale i svou vlastní dceru za ženu, což Xenofón poměrně ochotně přijal (Xen. {\em Anab.} 7.2.38). Po nějakém čase se ale přišlo na to, že Seuthés dle Xenofónta neplatil vojákům slíbené odměny, a tudíž k nim došlo k rozepři a ukončení spolupráce a přátelství (Xen. {\em Anab.} 7.7.48-54). Xenofón tak mohl, ve snaze se od Seutha distancovat, hodnotit některé Seuthovy akce negativněji, než by si bývaly zasloužily.

\subsubsection[thrákové-a-attická-komedie-a-řečnictví-5.-a-4.-st.-př.-n.-l.]{Thrákové a attická komedie a řečnictví 5. a 4. st. př. n. l.}

Attická komedie jednak poskytuje kritický obraz politických událostí, v nichž Thrákie hrála důležitou roli, a jednak přináší informace o tom, jak Athéňané obecně vnímali Thráky. Jak Matthew Sears (2013, 18) však správně podotýká, historická skutečnost je v komediích často překroucená tak, aby pobavila a lépe pasovala do děje, a není tudíž vhodné informace brát jako zcela odpovídající historické skutečnosti. Příkladem takového komentáře k aktuální politické situaci může být Aristofanova hra {\em Acharňané}, kde zmiňuje nedávno uzavřené spojenectví s Odrysy v roce 431 př. n. l. (Aristoph. {\em Ach.} 134-173).\footnote{Tato událost je známá i z jiných zdrojů (Thúkýdidés, epigrafické prameny, Janouchová 2013), a je možné ji tak považovat za věrohodný historický rámec, na němž Aristofanés založil epizodu týkající se Thrákie a Thráků.} Aristofanés v textu používá obecné pojmenování Thrákové pro popis etnika jako takového, ale rozlišuje i mezi jednotlivými thráckými kmeny. Thráky však v přímém protikladu k Athéňanům nazývá barbary a vyzdvihuje jejich negativní vlastnosti. Žoldnéři z kmene Odomantů jsou podle něj jedni z nejbojovnějších a nazývá je zhoubou pro Athény, požírající athénské bohatství.\footnote{Aristoph. Ach. 153: μαχιμώτατον Θρᾳκῶν ἔθνος.}

Poprvé se Thrákové objevují pod označením {\em barbaroi} v negativním slova smyslu až v attickém řečnictví, a to z politických důvodů (Xydopoulos 2004, 18-20). Ioannis Xydopoulos správně podotýká, že obraz Thráků jako barbarů v literatuře klasické doby velmi záležel na konkrétní politické situaci v Athénách a na politickém cíli, který si daný autor vytkl (Xydopoulos 2010, 214-221). Attické řečnictví 5. a 4. st. př. n. l. pak zejména poukazovalo na kulturně-společenskou převahu řeckých obcí a využívalo proti-thrácké a obecně proti-barbarské rétoriky pro politické účely.\footnote{Např. Ísokratés {\em Panegyrikos} 50.}

Démosthenés se vyjadřuje o Thrákii v souvislosti s tažením Filipa Makedonského do Thrákie. Varuje před nebezpečím v podobě makedonského vpádu a Thrákii používá jako odstrašující případ (8.43-45). O Thrákii hovoří o jako relativně chudé, zemědělské oblasti s obtížnými klimatickými podmínkami, a nejednotným vedením v podobě thráckých „králů” (23.8-11), kterou se i přesto Filip rozhodl získat, a je tedy pouze otázkou času, kdy se obrátí proti bohatým Athénám. Thráky popisuje většinou jako spojence Athén (18.25-27). Mluví o nich jako o barbarech, když popisuje zradu Kotya vůči Ífikratovi (23.130-132), nicméně je nepovažuje za krutější než samotné Řeky (23.169-170). Spojence z řad Thráků považuje za morálně na vyšší úrovni, než nepřátele Athén z řad Řeků, a to zejména v jednání se zajatci. Opačné názory zastává Démosthénův oponent Aischínés, který ve snaze vyvrátit Démosthenovu argumentaci zpochybňuje Thráky jako spojence Athéňanů a poukazuje na jejich slabé stránky a na jejich nespolehlivost coby spojenců, proto je jeho líčení Thráků velmi negativní (2. 81-86, 89-93).

\subsubsection[thrákové-a-pozdní-literární-zdroje]{Thrákové a pozdní literární zdroje}

Zhruba od poloviny 4. st. př. n. l. se Thrákie a Thrákové téměř vytrácí z literárních pramenů a objevují se znovu opět v 1. st. př. n. l. v řecky, tak i latinsky psaných pramenech. Prameny z římské doby (1. až 4. st. n. l.) mluví o Thrákii v rámci vývoje politické situace v římské říši či pro ilustraci pozadí významných událostí odehrávajících se právě v Thrákii či bezprostředním okolí.\footnote{Např. Tacitus 2.64-7; 3.38; 4.46-51; Suetonius {\em Aug}. 3-4; {\em Tib}. 37; Paus. 10.19.5-12; Dio. Cass. 51.23.2-27.2. Z pozdních autorů také Ammianus Marcellinus 22.8; 27.4. Pro kompletní přehled pramenů zmiňující Thrákii viz Katzarow 1949; Velkov {\em et al.} 1981.} Thrákové jsou většinou vnímáni jako odbojní poddaní římské říše, kteří se čas od času vzbouří. Thrákie je součástí římského impéria, která je nicméně po většinu času na pokraji politického zájmu. Literární autoři tedy v duchu této tradice většinou necítí potřebu vytvářet individuální pojednání o Thrákii a v mnohém vychází z předcházející literární tvorby. Zejména geografické a etnografické popisy Thrákie autorů působících v římské době do velké míry vychází z dříve píšících řeckých autorů jako je Hérodotos, Thúkýdidés, Xenofón, či Eforos.

Poměrně obsáhlé pojednání o zeměpise Thrákie a jejích obyvatelích sepsal v 1. st. n. l. Strabón (Strabo 7.3; 7.5-7.7). Strabón se věnuje obecnému popisu thráckých kmenů, zejména pak však Getů, jejich zvyků, a uvádí několik stručných výňatků z historie, vždy v souvislosti s historií řeckou. Další systematické, avšak stručné, pojednání o obyvatelích římské Thrákie poskytuje Pomponius Mela, taktéž autor 1. st. n. l. (2.2.16-33). Mela však do velké míry vycházel z Hérodota, ale i přesto se však jedná o jeden z nejucelenějších etnografických popisů římské Thrákie (Dimova 2014, 36). Dále se k Thrákii systematicky vyjadřuje v 1. st. n. l. i Plinius Starší (4. 11. 40-50), který částečně vychází právě z díla Pomponia Mely (Romer 1998, 27). Jeho deset odstavců dlouhý popis je součástí jeho monumentálního díla {\em Naturalis Historia} a, i přes malý rozsah, patří k nejinfomativnějším zdrojům o geografii a tehdejším uspořádání Thrákie. Dalším významným zdrojem je Diodóros Sicilský, který sepsal své historické dílo {\em Bibliothéké historiké} v 1. st. př. n. l., přičemž místy čerpal z dřívějších řecky píšících autorů, zejména historika Efora pro události 5. st. př. n. l.\footnote{I přesto je dílo Diodóra velmi cenným historickým pramenem, vzhledem k tomu, že mnohé události popisuje jako jediný dochovaný zdroj. Diodóros epizodicky zmiňuje historické události, kde Thrákové vystupují v rámci širšího kontextu, jako např. v době uzavření spojenectví Odrysů s Athénami (D. S. 12.50-51), Alkibiadovo spojenectví s odryskými panovníky (D. S. 13. 105), Xenofóntovo tažení skrz jihovýchodní Thrákii (D. S. 14.37), thrácké spojenectví s Thrasybúlem (D. S. 14.94), útok Thráků na Abdéru (D. S. 15.36). Jen ve fragmentech se dochoval popis makedonského tažení do Thrákie a následné vlády diadocha Lýsimacha, včetně jeho sporů s Thráky a jeho zajetí thráckým vládcem Dromichaitem (18.3; 18.14; 21.11-12).} Diodóros hovoří o Thrácích jako o spojencích - tehdy figurují jako rovnocenní partneři, kteří však mají odlišné kulturní zvyky a jsou na primitivnějším stupni vývoje společnosti (D. S. 14. 83; 15.36; 18.14), anebo jako nepřátelé a krvelační barbaři. V jeho díle se tak setkávají obě dvě tradice stereotypního popisu Thráků, jejichž užití závisí na konkrétní situaci.

Podobně jako dřívější řecké zdroje, i prameny doby římské se věnují spíše životu elit a jejich rodin, nikoliv běžných Thráků. Obyvatelstvo autoři charakterizují zcela v duchu stereotypního zobrazení jako drsné, divoké a bojovné (Pom. Mela 2.2.18; Strabo 7.3.7), a to obyvatelstvo nejen současné, ale i minulé (Arr. {\em Anab}. 1.3-6; Haynes 2011, 7-8). Thrákové jsou i v době římské vnímání jako dobří válečníci a vojáci, a to pravděpodobně vzhledem k faktu, že jich velký počet sloužil v římské armádě přímo v Thrákii či i mimo ni (Boyanov 2008; 2012, 251-269; Dana 2013, 219-269). Pojem barbaři se objevuje zejména ve spojení s vojenskými akcemi Thráků proti Římanům, ale nikoliv jako obecný pojem, který by charakterizoval všechny Thráky (Tac. {\em Ann.} 4.51).

\subsection[thrákie-a-thrákové-v-literárních-pramenech-shrnutí]{Thrákie a Thrákové v literárních pramenech: shrnutí}

Literární prameny se vyjadřují o Thrácích poměrně stereotypně po celou dobu antiky. Ač Thrákie nikdy nestála v popředí zájmu autorů, i přesto se objevovala poměrně často v narážkách, vedlejších příbězích či jako ilustrace širšího společensko-historického rámce. Nicméně z dochovaných četných zmínek o Thrákii a Thrácích vyplývá, že ke vzájemným kontaktům docházelo poměrně běžně, a to jak na úrovni diplomatických styků, tak i na úrovni obchodních kontaktů, a každodenních styků. K vzájemnému ovlivňování docházelo např. i na poli náboženství a kulturních zvyklostí, ač informace o vzájemném vlivu jsou zprostředkované skrze médium literární tvorby a často s odstupem několika století.

Historiografické prameny 5. a 4. st. př. n. l. daly základ všem pozdějším dílům. Právě z této doby pochází i dva základní stereotypy pevně svázané s charakterizací Thráků. Jako červená nit se vine literárními prameny zvěst o thrácké statečnosti, zálibě v boji a často až hrubosti, která byla jistě i v mnoha případech zasloužená. K Thrákům autoři přistupují dle aktuální politické situace a popisují je tak, aby dosáhli kýženého efektu. V dobách spojenectví jsou Thrákové líčeni jako rovní partneři, z trochu odlišnými zvyklostmi a vírou. V dobách, a v situacích, kdy jsou vnímáni jako nepřátelé, literární prameny mluví o barbarech, divokých a krutých stvořeních, kteří jsou kulturně primitivnější v porovnání s Řeky a Římany. Ani tento fakt však nebrání, aby Thrákové hráli roli rovnocenných partnerů a obávaných protivníku, a to jak v řeckém kulturním okruhu, tak i v prostředí římské říše.

Literární prameny rozlišují jednotlivé kmeny a uznávají odlišnost společenského uspořádání Thrákie. Řečtí autoři právě v Thrácích vidí svůj vlastní předobraz polomýtické minulosti, kdy společnost ovládali kmenoví vůdcové a hlavním měřítkem úspěšnosti byla společenská prestiž. Kontakty a kulturní výměnu s Thráky řecké prameny do určité míry reflektují, ale nemluví však o běžných obyvatelích, ale téměř výhradně o politických elitách. Navíc je popis thrácké společnosti značně nesystematický a selektivní, často přizpůsobený řeckému publiku. Proto je vhodné k literárním pramenům přistupovat obezřetně, pečlivě zvažovat kontext jejich vzniku a nepovažovat je automaticky za přesný obraz tehdejší společnosti. Pro doplnění a rozšíření je vhodné se obrátit na archeologické a epigrafické prameny, které mají tu výhodu, že pochází přímo od obyvatel Thrákie, a nikoliv od vnějších pozorovatelů. Jejich výpovědní hodnota není zprostředkovaná sítem řecké a římské literární tradice a dává nám tak nahlédnout do nitra komunity samotné.

\section[historický-nástin-vývoje-regionu]{Historický nástin vývoje regionu}

Území Thrákie bylo v 1. tis. př. n. l. osídleno thráckými kmeny, jejichž jména známe často jen z řeckých literárních zdrojů. Thrákové bohužel nezanechali písemné prameny a z thráčtiny se do dnešní doby zachovaly jen místní názvy, osobní jména a několik nápisů nejasného charakteru (Dana 2015, 244-245). Proto se při výkladu thráckých dějin z této doby musíme spoléhat převážně na interpretaci archeologických pramenů a na občasné zmínky v literárních pramenech.

\subsection[thrákie-v-1.-tis.-př.-n.-l.]{Thrákie v 1. tis. př. n. l.}

Materiální kultura první poloviny 1. tis. př. n. l. vykazuje velké množství lokálních variant, které poukazují na nejednotnost thráckého etnika, což odpovídá spíše regionálnímu rozdělení dle jednotlivých kmenových uskupení (Kostoglou 2010, 180-185; Graninger 2015, 25). Členění území dle kmenovému principu odpovídá i absence sídlišť městského typu, a naopak velké množství decentralizovaných menších sídel se zaměřením na zemědělskou produkci, doplněné opevněnými sídly kmenových vůdců a opevněných strážních sídel v horských oblastech (Theodossiev 2011, 15-17; Popov 2015, 111-114). Pro thráckou materiální kulturu jsou typické náhrobní mohyly, z hlíny navršené stavby překrývající hrobky, které zpravidla obsahovaly bohatou pohřební výbavu a usuzuje se, že patřily thráckým aristokratům (Theodossiev 2011, 20-21). Dalšími typickými stavbami jsou dolmeny a svatyně tesané do skal, které se nacházejí zejména v hornatých a hůře přístupných částech jihovýchodní Thrákie, a jejichž funkce souvisela pravděpodobně s rituálními aktivitami thráckého obyvatelstva (Nekhrizov 2015, 126-141).

Oblasti obývané Thráky se staly předmětem zájmu řeckých obcí zejména od 7. st. př. n. l., nicméně jak dokazují archeologické nálezy poslední doby, docházelo ke kontaktu s okolními oblastmi již v době před vznikem prvních řeckých kolonií na území Thrákie (Zahrnt 2015, 36). Tyto kontakty byly pravděpodobně obchodní povahy, ale bohužel archeologické doklady nejsou zcela jednoznačné, vzhledem k jejich mnoha možným interpretacím.

\subsection[řecká-kolonizace-thráckého-pobřeží]{Řecká kolonizace thráckého pobřeží}

První doložená řecká osídlení trvalého charakteru se na thráckém pobřeží objevila v průběhu 7. st. n. l. a během dvou následujících století se řecké kolonie rozšířily téměř na celé pobřeží Egejského, Marmarského a Černého moře (Isaac 1986; Archibald 1998, 32-47). Kolonizační aktivity vycházely vždy z iniciativy jednotlivých obcí či ze spojeného úsilí několika obcí, ale v žádném případě se nejednalo o koordinovanou aktivitu, kterou by řídila jedna politická autorita na celém území pobřežní Thrákie.

Jako hlavní důvody řecké kolonizace literární prameny uvádějí bohatství Thrákie, zejména nerostné zdroje a úrodnou půdu (Hom. {\em Il}. 13.1-16; 10.484; Archilochos, Diehl frg. 2 a 51; Hdt. 1.64.1; 5.23.2; Isaac 186, 282-285; Tiverios 2008, 80). Řecká města byla zakládána v blízkosti mořského pobřeží, často nedaleko vodních toků a v blízkosti zdrojů nerostných surovin. Koncem 6. a v průběhu 5. st. př. n. l. řecká města začala razit zlaté a zejména stříbrné mince, na něž získávala materiál těžbou v místních dolech. Poměrně záhy začali mince razit i jednotliví thráčtí panovníci (Archibald 2015, 912). Mince byly raženy dle řeckých mincovních standardů, což usnadnilo kontakty ekonomického charakteru mezi řeckými městy a thráckými aristokraty (Tiverios 2008, 128). Aby dále podpořily obchodní aktivity, řecké kolonie zakládaly svá vlastní osídlení, {\em emporia}, jejichž primární funkcí bylo zajišťovat obchod s Thráky, případně obstarávat zemědělské produkty či se starat o těžbu nerostných surovin (Tiverios 2008, 86-91). Většinou se tato {\em emporia} nacházela v okolí nerostných zdrojů a řek. Nicméně osídlování neprobíhalo pouze na pobřeží, ale máme i archeologické důkazy o vznikajících obchodních stanicích ve vnitrozemí, které umožňovaly přímý kontakt a usnadňovaly obchod s pobřežními oblastmi, např. na řece Hebru ležící {\em emporion} Pistiros (Bouzek et al. 1996; 2002; 2007; 2010; 2013; 2016; Bouzek a Domaradzka 2011). Vnitrozemská Thrákie tak díky těmto stanicím snáze získávala předměty řecké provenience, jako např. keramika, transportní amfory, víno, olej, a to vše pravděpodobně výměnou za nerostné suroviny, jichž bylo v Thrákii dostatek.

Povaha kontaktů nově příchozího a původního obyvatelstva se velmi lišila dle konkrétní situace. V souvislosti se zakládáním nových sídlišť a obchodních kontaktů pravděpodobně docházelo i ke sporům o území a životní prostor. Řečtí kolonizátoři se často usazovali v místech již existujících thráckých osídlení, což jistě vyvolávalo nevoli thráckého obyvatelstva. Z řecky psané literatury víme konfliktech mezi Thráky a řeckými kolonizátory z Paru, jak je zmiňuje např. řecký básník Archilochos (Diehls frg. 2, 6 a 51) či o neúspěchu první kolonie v Abdéře způsobeném jak nepřátelsky nalazenými Thráky, tak i nepříznivými zeměpisnými podmínkami a šířením nemocí (Hdt. 1.168; Graham 1992, 46-48; Loukopoulou 1989, 185-190). Na jiných místech dochované archeologické nálezy nicméně nasvědčují spíše poklidnému soužití, či dokonce ekonomické kooperaci mezi Thráky a Řeky žijícími na pobřeží (Archibald 1998, 47, 73; Ilieva 2011, 25-43). V některých případech mohlo dokonce docházet k soužití, která se nevyhnutelně projevilo i na archeologickém materiálu. Epigrafická produkce, která by dokumentovala povahu raných kontaktů mezi Thráky a Řeky, buď vůbec nevznikla, nebo se do dnešní nedochovala, a veškeré nápisy pochází až z doby stabilních a fungujících řeckých komunit na thráckém území, tedy z doby několik desítek let od založení kolonií.\footnote{Podrobněji v kapitole 6.}

\subsubsection[geografický-rozsah-řecké-kolonizace]{Geografický rozsah řecké kolonizace}

Řecká kolonizace představovala aktivitu jednotlivých obcí, čemuž odpovídá i rozdělení sfér vlivu a silné regionální vazby nově vniklých osídlení. V egejské oblasti docházelo zejména k osídlení z přilehlého Thasu a Samothráké\footnote{Mezi nejaktivnější kolonizátory v egejské oblasti patřili osadníci z ostrova Paros a Thasos, kteří se zaměřili především na protilehlé thrácké pobřeží v okolí řeky Strýmón, kde se nacházely zdroje nerostných surovin (Zahrnt 2015, 36). Mezi nejvýznamnější thaské kolonie patří Neápolis, Oisímé a Strýmé-Molyvoti, které zajišťovaly jak obchodní výměnu, ale umožňovaly přístup k úrodné thrácké půdě (Tiverios 2008, 80-85).} a dále z iónských a aiolských maloasijských obcí. Oproti tomu v oblasti Propontidy a Černého moře zásadní roli zaujímaly kolonie dórské Megary a iónského Mílétu a Samu.

Nejvýznamnějším příkladem maloasijské iónské kolonizace je založení Abdéry na egejském pobřeží poblíž řeky Néstos. Abdéra byla založena zhruba v polovině 7. st. př. n. l. byla z maloasijských Klazomen, jejíž obyvatelé se potýkali s odporem místních thráckých kmenů a nemocemi, díky čemuž byla Abdéra pravděpodobně opuštěna. V polovině 6. st. př. n. l. došlo k její opětné kolonizaci z~Teu, a nakonec se jim podařilo Thráky přemoci a z Abdéry se později stala jedno z nejvýznamnějších ekonomických a kulturních center egejské Thrákie a jedním z prvních producentů nápisů (Hdt. 1.168; Isaac 1986, 73-89; Graham 1992, 46-53; Loukopoulou 2004, 872-875; Tiverios 2008, 91-99).\footnote{Dle odhadů mohla populace Abdéry dosahovat až 100 000. Jednalo se tak o jedno z největších měst z oblasti egejské Thrákie (Loukopoulou 2004, 873). Dimitri Samsaris (1980, 166-167) řadí Abdéru na stejnou úroveň jako města Amfipolis, Maróneiu, Byzantion, Perinthos a Thasos, tedy s populací o velikosti řádově několika desítek tisíc obyvatel.} Zhruba od r. 540 př. n. l. Abdéra razila vlastní mince z lokálních zdrojů stříbra, k nimž měla pravděpodobně neomezený přístup. Nálezy abdérských mincí v depotech mincí po celém Středomoří a na území Thrákie dosvědčují, že mince byly určeny jak pro dálkový, tak i pro místní obchod (May 1996, 1-2, 16-17, 59-66; Sheedy 2013, 40-41; Paunov 2015, 267). Vzhledem ke svému důležitému postavení v průběhu 5. st. př. n. l. hrála Abdéra poměrně zásadní roli při zprostředkování diplomatických kontaktů mezi Athénami a odryským panovníkem, které vyústily v uzavření spojenectví v roce 431 př. n. l. (Thuc. 2.29; 2.95-101; Janouchová 2013, 96-97; Sheedy 2013, 46). Abdéra hrála důležitou roli nejen při zprostředkování diplomatických kontaktů mezi Řeky a Thráky, ale na jejím území a v blízkém okolí docházelo i ke kontaktům běžné populace a prolínání náboženských představ.\footnote{Příkladem setkávání dvou tradic je kult řeckého Apollóna, který nese místní thrácké přízvisko {\em Derénos}. Tento kult je doložen nejen z Abdéry, ale v římské době se dokonce objevil v thráckém vnitrozemí (Graham 1992, 67-68; Janouchová 2016, 93-95). Abdéra tedy hrála poměrně zásadní roli zprostředkovávání kontaktů mezi Řeky a Thráky jak na politické, ekonomické, tak kulturní úrovni.}

Neméně významným regionálním centrem iónského původu byla Maróneia, kterou založil v polovině 7. st. př. n. l. Chios, pravděpodobně na místě dřívějšího thráckého osídlení. Maróneia a okolní oblast hory Ismaros se stala známou pro své víno a díky strategicky situovanému přístavu si zajistila výhodné ekonomické postavení a četné obchodní kontakty jak se Středozemím, tak s vnitrozemskými Thráky (Archilochos, Diehl frg. 2; Isaac 1986, 111-115; Tiverios 2008, 99-104).

Území na egejském pobřeží mezi řekami Néstos a Hebros spadalo do sféry vlivu Samothráké, která byla sama pravděpodobně kolonií iónského Samu a neznámého aiolského města. V 6. st. př. n. l. byla na protilehlé thrácké pevnině ze Samothráké založena města Drys a Zóné-Mesámbria.\footnote{Lokalita Mesámbria, či Mesémbria se vyskytuje jak na pobřeží Egejského, tak Černého moře. Mesámbria v egejské oblasti je známá z literárních zdrojů (Hdt. 7.108.2, Steph. Byz. 446.19-21) a bývá ztotožňována s další lokalitou Zóné, Orthagoria, či Drys (Loukopoulou 2004, 880). Lokalita Mesámbria z oblasti Černého moře je taktéž známá z literárních zdrojů (Hdt. 4.93, 6.33.2) a dnes se nachází pod moderním městem Nesebar v Bulharsku.} Mezi Thráky a Řeky v této oblasti docházelo k četným kontaktům, které dle dostupných archeologických pramenů nebyly násilného charakteru a měly za následek jak mísení obyvatelstva a náboženských zvyklostí, ale i například technologií zpracování kovu a výroby keramických nádob (Ilieva 2007, 221; 2011, 36-38; Tiverios 2008, 107-118; Kostoglou 2010, 180-185).

Východně od této oblasti ležel aiolský Ainos, který byl pravděpodobně založen na místě dřívějšího thráckého osídlení. Ainos se nacházel v ústí řeky Hebros, která spojovala egejské pobřeží s thráckým vnitrozemím, a stal se tak ekonomickým a kulturním centrem pro celý region s napojením na thráckou dynastii Odrysů (Isaac 1986, 140-156; Tiverios 2008, 118-120).

Oblast Marmarského moře se stala středem zájmu dórské Megary a iónského Samu. Megara založila již v 7. st. př. n. l. města Sélymbrii a Byzantion, která sehrála velmi důležitou roli v kulturně-politickém vývoji dalších století. Město Byzantion těžilo zejména ze své výhodné pozice, která mu umožňovala kontrolovat bosporskou úžinu, ale i přesto se nevyhnulo občasným útokům okolních thráckých kmenů (Isaac 1986, 230-231). Iónský Samos pak na začátku 6. st. př. n. l. na místě existujících thráckých osad založil Perinthos a Bisanthé, z nichž zejména Perinthos hrál velmi důležitou roli v pozdějších stoletích, kdy se stal prvním sídlem místodržícího římské provincie {\em Thracia} (Isaac 1986, 198-201).\footnote{Isaac 1986, 205: ve 3. či 4. st. n. l. byl Perinthos přejmenován na Hérakleiu. V epigrafických pramenech se vyskytují obě dvě jména, a proto se držím pojmenování Perinthos (Hérakleia), a od konce 3. st. n. l. Hérakleia (Perinthos).}

Oblast Thráckého Chersonésu byla pro svou strategickou pozici a úrodnou půdu osídlena nejpozději na konci 7. st. př. n. l. aiolskými osadníky. Jména nejvýznamnějších měst z této oblasti jsou Séstos, Alopekonnésos či Madytos (Loukopoulou 2004, 900). V 6. st. př. n. l. se o oblast Thráckého Chersonésu velmi zajímaly Athény, které sem na určitou dobu dosadily vojenské posádky a snažily se jak silou, tak spoluprací s místními thráckými kmeny ovládnout toto strategické území (Tiverios 121-124).

Na pobřeží Černého moře založil řecký Mílétos celou řadu kolonií již v 6. st. př. n. l., z nichž se poměrně záhy stala regionální ekonomická a kulturní centra, jako v případě Apollónie Pontské a později Odéssu. Nejvýznamnější dórskou kolonií v této oblasti se stala Mesámbria založená Megarou, která zaujímala výsadní postavení ve 3. st. př. n. l.\footnote{O thráckém původu Mesámbrie svědčí přípona -{\em bria}, označujícící osídlení v thráčtině (Venedikov 1997,76).} Černomořské kolonie byly pravděpodobně v úzkém kontaktu s Thráky, jak dokládají nejen thrácká jména vyskytující se na pohřebních stélách, ale i epigraficky potvrzené diplomatické kontakty mezi městy a Odrysy či přítomnost thrácké keramiky a kovových předmětů thrácké provenience na území řeckých měst.\footnote{Náboženství a kulty v černomořských koloniích do velké míry vycházely z thráckých vzorů, nicméně první doklady o tomto náboženském synkretismu máme až z hellénismu a římské doby (Isaac 1986, 241-257).}

Povaha kontaktů mezi Řeky a Thráky záležela vždy na konkrétní situaci a rozhodně nelze z dochovaných materiálních dokladů vyvozovat, že Thrákové k příchozím Řekům zaujímali automaticky nepřátelský postoj, jak by mohly naznačovat literární zmínky u básníka Archilocha.\footnote{Zejména Diehls frg. 2, 6 a 51.} Prvotní kontakty mohly být nepřátelské, zejména proto, že Řekové se většinou snažili založit kolonii v místě existujícího thráckého osídlení. Pravděpodobně také docházelo ke konfliktům o vlastnictví zemědělské půdy a zdrojů nerostných surovin, což je však velmi špatně doložitelné v materiálních pramenech. Pokud jde o období následující po primární kolonizaci, archeologické studie keramické a metalurgické produkce napovídají, že se vzájemné kontakty Řeků a Thráků nesly v duchu ekonomické spolupráce a relativně poklidného soužití, alespoň v nejbližším okolí řeckých měst (Ilieva 2007, 221; 2011, 36-38; Kostoglou 2010, 180-185).

\subsection[peršané-a-vznik-dynastie-odrysů]{Peršané a vznik dynastie Odrysů}

V 6. a 5. st. př. n. l. se část Thrákie se na několik desítek let stala na součástí perské říše (Archibald 1998, 79-90; Boteva 2011, 738-749).\footnote{První zmínky o perské přítomnosti v Thrákii pocházejí z doby Dáreiova tažení proti Skýthům v roce 513 př. n. l. (Hdt. 4.83ff; 5.1-5.2). Po řecko-perských válkách byli Peršané postupně z Thrákie vytlačováni a poslední doklady o jejich přítomnosti pocházejí z území Thráckého Chersonésu z poloviny šedesátých let 5. st. př. n. l. (Plut. {\em Kim}. 14.1).} Bohužel nemáme přesné informace o tom, zda Peršané ovládli celou Thrákii a jakým způsobem řídili ovládnuté území (Zahrnt 2015, 38).\footnote{Tj. zda se jednalo o satrapii v pravém slova smyslu.} Hérodotos dokládá rozdělení ovládnutého území do několika oblastí, které byly spravovány ustanovenými místodržícími (Hdt. 7.106).

Perská přítomnost se projevila na proměňující se materiální kultuře a zvyklostech: Zofia Archibald však naznačuje, že právě vliv Peršanů mohl být hlavní příčinou objevení tzv. mincí thráckých králů na počátku 5. st. př. n. l. a zavedení nových technologických postupů při zpracování kovů (1998, 89-90). S obdobím perské nadvlády se taktéž spojuje zavádění nových technologií zpracování kovu a následné rozšíření luxusních nádob z drahých kovů. Ty se mnohdy staly součástí pohřební výbavy thrácké aristokracie v 5. a 4. st. př. n. l. a některé z nádob nesly nápisy odkazující na majitele (Archibald 1998, 85; Theodossiev 2011, 6; Loukopoulou 2008, 148).\footnote{K jednotlivým nádobám v kapitole 6 a sekcím věnovaným 5. a 4. st. př. n. l.} Peršané měli dále vliv na zvyky spojené s vytvářením nápisů a předmětů nesoucí nápisy. \footnote{Jak se dozvídáme z Hérodotova vyprávění, Peršané na území Thrákie vztyčili několik mramorových stél. Jednalo se o dvě stély s asyrským a řeckým písmem, zaznamenávající členy Dáreiovy výpravy, umístěné na evropské straně Bosporu a u pramenů řeky Teáru ve vnitrozemské Thrákii (Hdt. 4.87, 4.91). Hérodotos rovněž popisuje, že Byzantští si dvě zmíněné stély z Bosporu odvezli a sekundárně je použili v chrámě Artemidy {\em Orthósie}.} Ač se nám dochovaly starší nápisy přímo z řeckých měst na pobřeží Thrákie, tak se jedná o první reflexi epigrafické aktivity na území Thrákie a první zmínku o zvyku veřejně vystavovat nápisy tesané do kamene. Překvapivé je, že iniciátory vztyčení stél byli Peršané, a nikoliv Řekové, avšak i přesto dle slov Hérodota vztyčil Dáreios jednu ze stél psanou v řečtině, aby textu rozumělo i místní obyvatelstvo.

Po vyhnání Peršanů z oblasti se většina řeckých měst na pobřeží egejského moře připojila k athénskému námořnímu spolku, kam začala přispívat nemalou měrou. Podíl měst z thrácké oblasti činil zhruba jednu čtvrtinu všech příjmů spolku (Meritt, Wade-Gery a McGregor 1950, 52-57). Vzhledem ke strategické pozici a bohatství zdrojů nerostných surovin se Athény o oblast severní Egejdy začaly zajímat v prvních letech po vyhnání Peršanů z oblasti. Ekonomické a politické zájmy Athén v oblasti Thrákie vyústily nejen umístěním vojenských posádek na Thráckém Chersonésu, ale i opakovanou snahou založit vlastní kolonii na území egejské Thrákie. Zakládání nových kolonií v případě Brey či Amfipole narazilo na odpor místních obyvatel a bylo ztíženo probíhajícími konflikty znepřátelených řeckých obcí (IG I\high{3} 1, 46; Thuc. 1.98, 1.100, 4.102; Meiggs 1972, 158-159, 195).

Téměř současně ústupem perské nadvlády se zhruba před polovinou 5. st. př. n. l. na jihovýchodě Thrákie v okolí řek Hebros a Tonzos objevila první autonomní politická organizace, známá z řeckých historických pramenů jako odryský stát, či království.\footnote{Zofia Archibald (1998, 93) tvrdí, že se jednalo o stát se všemi jeho atributy: existovala zde společenská stratifikace, aristokracie ve svých rukách soustředila veškerou politickou moc a kontrolu nad přerozdělováním prostředků, docházelo zde k specializaci činností, a docházelo ke stavbám výstavných rezidencí a monumentálních pohřebních mohyl a v omezené míře se zde vyskytovaly i písemné památky. Dle dostupných výsledků archeologických pozemních sběrů z nedávné doby je však spíše pravděpodobné, že se jednalo o mezistupeň mezi kmenovým uspořádáním a raným státem, kde velkou roli hrála rodově uspořádaná aristokracie a specializace práce byla spíše záležitostí přítomnosti cizích umělců a specialistů, než reflexe skutečné stratifikace odryské společnosti a existence institucionálního uspořádání dlouhodobého charakteru (Sobotkova 2013, 133-142).} Odrysové si v průběhu 5. st. př. n. l. vybudovali takovou pozici, že se stali nejen spojenci Athén, ale dokonce si mohli dovolit od nich vybírat pravidelné peněžní i nepeněžní dávky ve výši až 800 talentů (Thuc. 2.29, 2.97; Archibald 1998, 145). Tyto platby mohly sloužit výměnou za ochranu řeckých měst na pobřeží i ve vnitrozemí, či jako jistá forma cla, poplatek za možnost obchodovat s thráckým vnitrozemím (Zahrnt 2015, 42).

Mezi nejznámější panovníky patří Sitalkés, za jehož vlády došlo jak k~ekonomické, tak územní expanzi odryské říše. V roce 431 př. n. l. dokonce uzavřel spojeneckou smlouvu s Athénami, kterou stvrdil významnou pozici odryského území (Thuc. 2.29; Archibald 1998, 118-120; Zahrnt 2015, 41-42). Jak víme z literárních pramenů, okolo r. 400 př. n. l. si odryský panovník Seuthés II. najal do svých služeb historika Xenofónta a další řecké žoldnéře, aby mu pomohli získat zpět území a výsadní pozici v rámci odryské aristokracie (Xen. {\em Anab.} 6-7; Zahrnt 2015, 43). Odrysové si udrželi výsadní pozici i po konci peloponnéské války a v první polovině 4. st. př. n. l., kdy i nadále vystupovali jako mocní spojenci Athén a okolních kmenů. V roce 383 př. n. l. se k moci dostal panovník Kotys I., který odryskou říši upevnil obratnou diplomatickou politikou. Pravděpodobně kvůli narůstající politické pozici byl v roce 359 př. n. l. zavražděn. Území Odrysů se rozpadlo na několik menších regionů a fakticky tak ztratilo svou výsadní pozici a stabilitu (Dem. 23.8; Archibald 1998, 218-222; Zahrnt 2015, 44-45). Cesta do Thrákie se tak uvolnila Athénám, ale i nejbližšímu rivalovi a sousedovi, Makedonii.

\subsection[makedonská-kolonizace-a-období-hellénismu]{Makedonská kolonizace a období hellénismu}

Již před polovinou 4. st. př. n. l. Filip II. Makedonský podnikl několik výprav na pobřeží egejské Thrákie, kde ovládl řecká města, zmocnil se zdrojů nerostných surovin a umístil zde vojenské posádky a o několik let později se vypravil do vnitrozemí Thrákie, aby si podrobil nebezpečné a výbojné thrácké kmeny a zabezpečil oblast bezprostředně sousedící s Makedonií (Worthington 2015, 76). Filip tak využil ve svůj prospěch nejednotnosti thráckých kmenů, k níž došlo po smrti Kotya I. a nakonec se v roce 340 př. n. l. zmocnil větší části území vnitrozemské a pobřežní Thrákie.\footnote{Makedonci podnikli výpravy nejen proti Odrysům, ale i proti Bessům či Maidům v údolí Strýmónu a nějakou dobu neúspěšně obléhali Perinthos a Byzantion a válčili s kmenem Triballů.} Thrákie se tak stala nedílnou součástí pozdější makedonské říše a odryští králové byli dočasně poraženi. Bohužel nemáme přesné informace o rozsahu makedonské moci v Thrákii, ale předpokládá se, že zhruba od roku 340 př. n. l. většina území plně spadala pod makedonskou administrativu, systém výběru daní a docházelo i k verbování thrácké populace do makedonské armády. Na území dohlížel ustanovený makedonský vojevůdce, kterým se v roce 323 př. n. l. stal Lýsimachos ({\em stratégos}; D. S. 17.62.5; Arr. {\em Anab.} 1.25.2; Archibald 1998, 231-239; Delev 2015, 49-53; Worthington 2015, 76).

Během pobytu v thráckém vnitrozemí Filip II. založil několik vojensky zaměřených osídlení na místě existujících thráckých sídlišť (Dem. 8.44, 10.15). Tato místa se postupem času rozrostla na sídla městského charakteru, jak je patrné na příkladě Kabylé nebo původní thráckého sídla {\em Pulpudeva}, která je od této doby známá pod jménem Filippopolis. Podobným způsobem vznikla i Hérakleia Sintská na řece Strýmón, kdy se původně obchodní osada s vojenskou posádkou rozrostla až do podoby hellénistického města (Nankov 2015, 26-27). V těchto prvních sídlech městského charakteru v thráckém vnitrozemí spolu pravděpodobně žili jak nově příchozí Makedonci, tak původní Thrákové. Jedinečný charakter těchto míst umožnil setkávání několika kultur na každodenní bázi, což mělo za následek i rozšíření epigrafické produkce do thráckého vnitrozemí.

Z literárních zdrojů víme o několika pokusech o osamostatnění se z řad makedonských vojevůdců, které však byly vesměs potlačeny (tzv. Memnónova revolta v roce 331/0 př. n. l., D. S. 17.62-63). Nicméně ani thráčtí aristokraté se nehodlali smířit s makedonskou nadvládou a docházelo ke konfliktům mezi Makedonci a Odrysy.\footnote{Příkladem thráckého odboje je konflikt mezi makedonským Lýsimachem a thráckým Seuthem III., který pravděpodobně vyústil v určitou nezávislost Seuthova postavení (D. S. 18.14.2-4; Tacheva 2000, 12-15; Delev 2015, 53-55). Lýsimachos nicméně i přes Seuthovo nezávislé postavení ovládal velkou část Thrákie a na Thráckém Chersonésu založil město Lýsimacheia (Jones 1971, 5).}

Odryský panovník Seuthés III., který žil na přelomu 4. a 3. st. př. n. l., je známý především díky objevu výstavné rezidence na řece Tonzos, která nesla jeho jméno, Seuthopolis, a díky objevu monumentálních mohylových hrobek v Kazanlackém údolí, patřícím Seuthovi a jeho rodině (Dimitrov, Čičikova a Alexieva 1978; Dimitrova 2015; Delev 2015, 53-54). Dle nalezeného archeologického a epigrafického materiálu se zdá, že Seuthopolis byla osídlena jak thráckým, tak řeckým či makedonským obyvatelstvem, které se po dobu existence rezidence podílelo na vzniku unikátní kombinace kultur a zvyklostí. Seuthopolis, ač se nachází uprostřed thráckého vnitrozemí, nesla všechny charakteristiky typické pro hellénistická sídla tehdejší doby, počínaje užitou architekturou s domy typu {\em pastas} a {\em prostas}, nalezenou keramikou řeckého a místního původu, precizně zhotovenou toreutikou, sochařskou a dekorativní výzdobou a zejména unikátními epigrafickými nálezy (Tacheva 2000, 25-35). Krátký časový horizont existence Seuthopole nicméně potvrdil, že tamní unikátní společnost plně závisela na osobě panovníka Seutha III. a po jeho smrti na počátku 3. st. př. n. l. došlo k poměrně rychlému úpadku aktivit, vymizení jak materiální, tak epigrafické produkce a zániku tohoto jedinečného příkladu hellénistické kultury v samém středu thráckého území (Nankov 2012, 120; Janouchová 2017, v tisku).

Ve 3. st. př. n. l. byla Thrákie svědkem válečných konfliktů diadochů, kteří se snažili ovládnout tuto významnou spojnici mezi Evropou a Asií. Asi nejvýznamnějším byl konflikt mezi makedonským Lýsimachem, původně místodržícím v Thrákii, který se stal vládcem evropské části makedonské říše a samozvaným králem Thrákie, a Seleukem I., jemuž v roce 281 př. n. l. nakonec Lýsimachos podlehl a Thrákie se stala součástí seleukovské říše (Samsaris 1980, 33-34). V následujících letech pokračovaly konflikty následovnických rodů, které z větší či menší části zahrnovaly i území Thrákie. Z této doby pocházejí mince seleukovských panovníků ražené na území Thrákie, např. stříbrné mince z Kabylé, ale i drobné seleukovské mince nesoucí kontramarky Kabylé, určené pro lokální trh (Draganov 1991, 198-208; Draganov 1993, 87-99). Řecká města na pobřeží se v polovině 3. st. př. n. l. dostala do vlivu Ptolemaiovců, což vysvětluje výskyt původně egyptských božstev a motivů ve figurálním umění a na nápisech (Delev 2015, 61).\footnote{Jones 1971, 6: Pod vliv Ptolemaiovců se dostal Ainos, Maróneia a část Chersonésu s Lýsimacheiou.}

V průběhu 3. st. př. n. l. do Thrákie v několika vlnách vpadly keltské kmeny, což s sebou neslo významné změny ve fungování mnoha významných osídlení či dokonce jejich zánik, jako v případě Seuthopole, či {\em emporia} Pistiros. Keltové se na území Thrákie na určitou dobu usadili, a dokonce si postavili nové hlavní město Tylis, které existovalo až do roku 213 př. n. l. (Theodossiev 2011, 15). Přítomnost Keltů měla vliv na thrácké umění, zejména na toreutiku, kde je možné sledovat výskyt nových motivů, ale i technologií. Keltové dokonce razili vlastní mince a využívali k tomu již existující infrastrukturu, čehož jsou důkazem např. mince keltského panovníka z Kabylé, Mesámbrie či Odéssu (Draganov 1993, 107).

Nedostatek pramenů neumožňuje podrobně rekonstruovat následující vývoj, a ve výkladu zůstává mnoho bílých míst. Thrákie zůstala ve sféře vlivu Makedonie až do poloviny 2. st. př. n. l. I nadále docházelo k politickým konfliktům na území Thrákie a ve větší míře zde operovaly makedonské armády, nicméně makedonská administrativa fungovala bez větších změn. Během 2. st. př. n. l. se začal také pomalu stupňovat politický vliv Říma na dění v Thrákii, který se zintenzivnil po roce 148-146 př. n. l., kdy došlo k ovládnutí Makedonie Římem (Eckstein 2010, 248). Thráčtí Odrysové využili situace a stali se spojenci a podporovatelé Říma (T. Liv. 45.42.6-12; Delev 2015, 63-68). Řecká města na pobřeží zčásti spadala pod římskou provincii {\em Macedonia}, stejně tak jako vnitrozemská Hérakleia Sintská, a z části si udržela autonomní pozici, jako např. Abdéra, Maróneia a Ainos (T. Liv. 45.29.5-7).\footnote{Jones 1971, 7: města Ainos a Maróneia spolu s pobřežními oblastmi se opět stala součástí Makedonie. V roce 168 př. n. l. se obě města stala autonomními politickými autoritami.} Byzantion v roce 148 př. n. l. uzavřel spojeneckou smlouvu, což vedlo k nárůstu politické moci Říma v regionu a zároveň i rozvoji Byzantia (Tac. {\em Ann.} 12.62-63; Jones 1971, 7). Římská přítomnost v regionu byla patrná zejména podél cesty {\em Via Egnatia}, která spojovala oblast západního a východního balkánského poloostrova. Oblast Thráckého Chersonésu v té době spadala pod pergamské království a byla spravována vojenskými veliteli ({\em stratégy}), jak dokazují dochované nápisy (Delev 2015, 68). Se zánikem pergamského království se v r. 133 př. n. l. Thrácký Chersonésos stal součástí římské provincie {\em Asia}. V první polovině 1. st. př. n. l. se část Thráků a řeckých měst přidala na stranu pontského krále Mithridata VI., což mělo za následek římské vojenské akce na území Thrákie ve snaze zajistit stabilitu v oblasti a eliminovat případný odpor místních kmenů (Lozanov 2015, 76-77). V této době se tak vytvořil prostor pro iniciativní jedince z řad thrácké aristokracie, kteří za svou podporu Říma získali výsadní postavení.

\subsection[období-vazalských-králů]{Období vazalských králů}

V 1. st. př. n. l. se prosadilo několik thráckých aristokratických dynastií, které získaly své postavení zejména spoluprací s Římem a došlo k vytvoření tzv. vazalských království, podobně jako např. v oblasti Británie či Judeje. Thrákové poskytovali Římu zejména vojenskou a materiální pomoc a výměnou se jim dostalo relativní autonomie a osobních výhod.\footnote{David Braund (1984, 23-29) uvádí, že vazalští králové často obdrželi jak římské občanství a jiné posty, za které byli ochotni zaplatit nemalé finanční částky. Spolu s poctami často dostávali i luxusní dary, jako oděvy a znaky moci, které zvyšovaly jejich společenskou prestiž a upevňovaly postavení krále.} Mezi nejvýznamnější kmeny této doby patří Odrysové, Sapaiové a Astové (Lozanov 2015, 78).

Odryští králové byli u moci přibližně do poloviny 1. st. př. n. l. (Manov 2002, 627-631) a z historických pramenů víme, že po roce 42 př. n. l. thrácký Rhéskúporis I. založil sapajskou dynastii se sídlem v Bizyi v jihovýchodní Thrákii a vládl v letech 48 - 41 př. n. l. (Strabo 7, frg. 47; 12.3.29; Jones 1971, 9-10). Bezprostředně po něm není následnická linie zcela jasná, nicméně se sapajští králové udrželi u moci až do r. 46 n. l.\footnote{Lozanov 2015, (78-80); Manov 2002 (627-631): přesné roky vlády jednotlivých králů jsou i nadále předmětem akademické debaty. Z literárních pramenů známe jména a přibližnou dobu vlády následujících panovníků: Rhéskúporis I (48-42 př. n. l.), Kotys (42-18? př. n. l.), Rhéskúporis II. (18-13? př. n. l.), Rhoimetalkás I. (13 př. n. l. - 11 n. l.), Kotys (12-19 n. l.), Rhoimetalkás II. (18-38 n. l.) a Rhoimetalkás III (38-45 či 46 n. l.).} Posledním thráckým králem se stal Rhoimetalkás III., který vládl v letech 38 až 46 n. l. Ač byla dynastie Sapaiů teoreticky nezávislá, v praxi se jednalo o panovníky dosazené k moci Římem, který sledoval své vlastní mocenské zájmy. Z epigrafických a literárních zdrojů víme, že území Thrákie bylo rozděleno na samosprávní jednotky, tzv. stratégie, které spravovali stratégové, původem thráčtí aristokraté, či loajální Řekové. Rozdělení do stratégií mělo primárně usnadňovat administrativu a sloužilo i pro zvýšení vojenské kontrola území či verbování jednotek (Lozanov 2015, 78-79). Za vlády sapajské dynastie mnoho thráckých mužů vstoupilo do římské armády jako příslušníci pomocných jednotek (Tac. {\em Ann.} 4.47). Nejen, že Thrákové v této době sloužili v římské armádě, ale římská armáda operovala na území Thrákie.\footnote{Z literárních pramenů se dozvídáme, že místodržící v Makedonii Marcus Lucullus v roce 72/1 př. n. l. porazil kmen Bessů a dobyl nejen Kabylé, ale zmocnil se i měst na pobřeží Černého moře a založil zde vojenské posádky (Eutrop. {\em Breviarium} 6.10).}

Postupný nárůst moci Říma měl za následek transformaci vazalských království do podoby římské provincie. Sever území okolo řeky Istros se stal součástí římské říše pravděpodobně okolo r. 15. n. l. jako provincie {\em Moesia Inferior}. O 30 let později po smrti Rhoimetalka III. v roce 45 či 46 n. l. Řím využil příležitosti a chaosem zmítané území vnitrozemské Thrákie přeměnil na provincii {\em Thracia}, která tak plně spadala pod autoritu římského císaře (Lozanov 2015, 76-80).

\subsection[římské-provincie-thracia-a-moesia-inferior]{Římské provincie Thracia a Moesia Inferior}

Bezprostředně po vzniku nových provincií {\em Moesia Inferior} a {\em Thracia} nedošlo k zásadním proměnám společenského uspořádání a administrativy, ale spíše došlo k navázání na již existující infrastrukturu vytvořenou vazalskými panovníky. Systém stratégií přetrval pravděpodobně až do začátku 2. st. n. l. a výsadní pozici si udržovala i nadále thrácká aristokracie, která se dokázala adaptovat na nové podmínky.\footnote{Role stratégií zůstala pravděpodobně stejná, avšak s postupem času se snižoval jejich počet z původních 50 na 14 (Plin. {\em H. N.} 4.11.40; Ptolem. {\em Geogr.} 3.11.6; Jones 1971, 10-15).} Odměnou za jejich loajální služby bylo udělení římského občanství a vysoká pozice v provinciálním aparátu (Lozanov 2015, 80-82). Některé z řeckých měst si udržela nezávislost na systému stratégií a pravděpodobně si uchovala politickou a ekonomickou autonomii. Plinius zmiňuje specificky Abdéru, Maróneiu, Ainos a Byzantion (Pliny {\em H. N.} 4.42-43). Naopak Anchialos a Perinthos se stala hlavními městy stratégií, ale byla jim ponechána určitá autonomie (Jones 1971, 15).

Provincie {\em Thracia} byla územím bez trvale usídlené legie ({\em provincia inermis}) a existovalo zde pouze několik pravidelných jednotek, skládajících se mimo jiné i z místního obyvatelstva, jako např. tábor v Kabylé či Germaneii. Oproti tomu {\em Moesia Inferior} byla provincií se silnou vojenskou přítomností, zejména podél řeky Dunaje, kde byla vytvořena soustava vojenských táborů a opevnění, což mělo vliv i na civilní obyvatelstvo (Haynes 2011, 7-9; Lozanov 2015, 80-82).

Předpokládá se také, že Thrákové se aktivně účastnili služby v římské armádě již v průběhu 1. st. př. n. l. a jejich nábor měl mít na starosti stratégos, který jednal znal nejlépe místní poměry, ale zároveň prokázal svou věrnost Římu (Lozanov 2015, 79-81). Z literárních a epigrafických pramenů pocházejících z celé římské říše víme, že již v 1. st. n. l. existovaly pomocné vojenské jednotky ({\em auxilia}), které nesly ve jméně thrácký původ. Tyto vojenské jednotky se skládaly jak z jezdců ({\em alae}), tak z pěšího vojska ({\em cohortes}) a sloužily po celém území římské říše (Jarrett 1969, 215). Jména pomocných jednotek nesla typicky odkaz na thrácký původ, např. {\em alae Thracorum}, či {\em alae Bessorum}.\footnote{Kmen Bessů byl používán nikoliv proto, že by všichni členové jednotky pocházeli z kmene Bessů, ale jako zástupné obecné pojmenování původu vojáků. Tento jev byl součástí verbovací strategie římských pomocných jednotek a je pozorovatelný i na jiných místech římského impéria, jako např. v Batávii (Derks 2009, 239-270). Thrákové naverbovaní do římské armády byli zařazeni do jednotky, která primárně nemusela odpovídat jejich kmenové příslušnosti, ale v očích římské administrativy byl zvolen jeden kmen, podle nějž se jednotky jmenovaly. Thrákové byli nuceni se vstupem do římského vojska adoptovat novou identitu, která se zcela nemusela shodovat s jejich původem. Thrákové však na svůj původ nezanevřeli zcela a na vojenských diplomech často uvádí jednak svůj etnický původ, ale i konkrétní obci či vesnici, z níž pocházejí, což spolu s uchováním thráckých osobních jmen poukazuje na silný tradicionalismus (Dana 2013, 246).} Thrákové se stali jedním z nejpočetnějších etnik římské armády a po odsloužení 25 let vojenské služby se často vraceli jako veteráni zpět do vlasti, zakládali nová sídliště a podíleli se na správě provincie. Ke konci 1. st. n. l. byla zakládána nová města, kde byli umísťováni veteráni římské armády, jako např. {\em Colonia Flavia Pacis Deultensium}, známá jako Deultum, či {\em Colonia Claudia Aprensis}, známá jako Apros (Lozanov 2015, 85). Veteráni tak měli zajišťovat bezpečí v nejbližším okolí strategicky umístěných měst výměnou za pozemky. Tento trend byl ještě patrnější severně od pohoří Haimos v {\em Moesii Inferior}, kde v okolí vojenských táborů vznikala civilní osídlení, která zásobovala početné římské vojsko. Ve stejné době se zde rozvinul systém {\em vill}, tedy venkovských usedlostí zaměřených primárně na zemědělskou produkci, které se nacházely v blízkosti vojenských táborů, avšak dostatečně daleko od hranic.

\subsection[společenské-reformy-2.-st.-n.-l.]{Společenské reformy 2. st. n. l.}

Na začátku 2. st. n. l. za vlády Trajána a Hadriána došlo k poměrně zásadním reformám provinciální administrativy v Thrákii, které měly vliv na celou společnost. Nejen, že došlo ke změně statutu provincie na tzv. pretoriánskou provincii, ale i k organizačním změnám, které vedly k posílení politické moci místní samosprávy. To vedlo ve 2. st. n. l. ke zintenzivnění stavební aktivity, zakládání nových měst, či jejich rozšiřování a výstavbě veřejné infrastruktury, jako např. lázní, akvaduktů, divadel či cest.\footnote{Jména nově vzniklých či obnovených měst dokazují šíři těchto stavebních aktivit: Augusta Traiana, Trainúpolis, Hadriánúpolis, Plotinúpolis, {\em Ulpia Nicopolis ad Istrum}, {\em Ulpia Nicopolis ad Nestum}, {\em Ulpia Marcianopolis} (Jones 1971, 18-19).} Stejně tak narostl význam městských samospráv, jimž bylo uděleno více pravomocí na úkor rodové aristokracie. V Thrákii, ale např. i v sousední Bíthýnii se setkáváme s novou společenskou skupinou městských elit, které pocházely z místního prostředí, avšak dokázaly se plně adaptovat na novou společenskou strukturu (Fernoux 2004, 415-511). V této době také vznikaly nové úřady a politická uskupení, jako např. {\em koinon tón Thraikón}, což bylo uskupení vnitrozemských měst se sídlem ve Filippopoli (Lozanov 2015, 82-83).

Sídlo místodržícího se přesunulo z Perinthu na pobřeží do vnitrozemské Filippopole. Původně řecká města na pobřeží postupně ztrácela svůj autonomní status a stávala se součástí římské říše, a to včetně povinnosti platit daně. Města razila vlastní mince s vyobrazením císaře na aversu a charakteristickým symbolem daného města na reversu. Města byla navzájem propojena sítí silnic, které primárně sloužily pro přesuny armády, na jejichž stavbě a údržbě se podílela městská samospráva do jejíhož teritoria cesta spadala (Madzharov 2009, 29-30). Od konce 2. st. n. l. vznikaly podél cest stanice, které zajišťovaly bezpečnost a zásobování. K tomuto účelu byla ve vnitrozemí stavěna {\em emporia}, jejichž hlavním úkolem bylo zprostředkování obchodu mezi zemědělsky zaměřeným venkovem a městskými centry (Lozanov 2015, 84-85).\footnote{Příkladem může být např. {\em emporion} Pizos, či Discoduraterae, o nichž podobněji pojednávám v kapitole 6.}

Od 2. st. n. l. mohlo thrácké obyvatelstvo sloužit nejen v pomocných vojenských jednotkách, ale i vstupovat do legií, jak dosvědčují epigrafické památky z celého území římské říše (Samsaris 1980, 38-39). V průběhu 2. st. n. l. také došlo k výraznějšímu přesunu obyvatelstva z oblasti Bíthýnie do vnitrozemské Thrákie, jak dosvědčují mnohé nápisy a monumenty, což mělo za následek i proměny nejen ve složení obyvatelstva ale i nárůst počtu nových kultů východní provenience (Delchev 2013, 15-19; Raycheva a Delchev 2016; Lozanov 2015, 82; Tacheva-Hitova 1983).

\subsection[krize-3.-a-4.-st.-n.-l.-a-transformace-společnosti]{Krize 3. a 4. st. n. l. a transformace společnosti}

Ve 3. st. n. l. musela římská říše čelit nájezdům nepřátelských kmenů, a nejinak tomu bylo i v Thrákii, která byla v druhé polovině století opakovaně stižena nájezdy Gótů. Za císaře Aureliána docházelo k poničení měst a infrastruktury, vyvrcholila vleklá ekonomická krize a došlo ke snížení počtu obyvatelstva. Zásluhou reforem a posílení vojenské přítomnosti na severních hranicích dochází k určitému oživení na konci 3. st. n. l., nicméně i přes tato opatření nájezdy Gótů probíhaly i v následujících staletích (Velkov 1977, 21-29; Poulter 2007, 29-36).

Koncem 3. st. n. l. docházelo za císaře Aureliána, ale zejména později za Diokleciána a Konstantina k reformám uspořádání římské říše a tyto změny se nevyhnuly ani Thrákii. Území Thrákie spadalo jednak pod nově vytvořenou diecézi {\em Thracia} a severovýchodní Thrákie s okolím města Serdica pod diecézi {\em Moesia} a od 4. st. n. l. pod diecézi {\em Dacia}.\footnote{Diecéze {\em Thracia} se dále dělila na šest provincií, z nichž každá měla své hlavní město: {\em Europa} s hl. m. Eudoxiopolí, dříve Sélymbrií, {\em Rhodope} s Ainem, {\em Haemimontus} s Hadriánopolí, {\em Moesia Inferior} s Marcianopolí, {\em Scythia Minor} s Tomidou a {\em Thracia} s Filippopolí (Velkov 1977, 61-62). Hlavou diecéze {\em Thracia} se stal pověřený {\em vicarius Thraciae} se sídlem v Konstantinopoli, který měl na starosti zejména civilní správu Thrákie.} Hlavním městem říše se v roce 330 n. l. stala Konstantinopol, bývalé Byzantion, což Thrákii posunulo z okrajové části římského impéria rovnou do jejího středu. S přesunutím hlavního města souvisí i zvýšení stavebních aktivit a udržování rozsáhlé sítě cest, které sloužily k přepravě zboží, ale zejména vojsk (Madzharov 2009, 63-65). Vojenské velení mělo propracovanou hierarchii, rozdělenou dle zeměpisného principu na jednotlivé regiony. Většina armádních jednotek byla umístěna podél Dunaje a také ve městech na pobřeží, nicméně vojenské tábory byly umístěny i ve vnitrozemí, avšak v menším počtu (Velkov 1977, 63-68).

Již v průběhu 3. a zejména na počátku 4. st. n. l. začíná veřejný život ovlivňovat rostoucí oblíbenost křesťanství. Nejvýznamnější křesťanské komunity byly ve Filippopoli, Konstantinopoli (býv. Byzantiu), Hérakleii (býv. Perinthu), Traianúpoli, Augustě Traianě a Odéssu. V Konstantinopoli bylo sídlo arcibiskupa a biskupství byla ve většině velkých městských center. Z tohoto důvodu docházelo k šíření křesťanské víry rychleji ve městech, než na thráckém venkově (Dumanov 2015, 92-94). V této době dochází také na okrajích měst ke stavbě kostelů a bazilik, zatímco na venkově se křesťanská architektura prosazuje až na počátku 6. st. n. l.

Byrokratický aparát koncem 3. st. n. l. a zejména v průběhu 4. st. n. l. narostl a stal se poměrně nákladným na udržování, což vyústilo v růst daňových povinností jednotlivých měst. V provinciích měli největší moc místodržící, kteří zpravidla sídlili v hlavním městě a obklopovali se početným úřednickým aparátem. Důležitým orgánem se staly městské sněmy ({\em concilium}, {\em koinon}) se zástupci městských samospráv, kde se projednávaly otázky regionálního charakteru bezprostředně spojené s chodem provincie. Postupem času tyto sněmy nabyly důležitosti a staly se prostředníkem mezi samosprávou jednotlivých měst a autoritou císaře. Menší osídlení, jako např. vesnice či venkovské usedlosti, administrativně a fiskálně spadala pod autoritu města, na jehož území se nacházela. V této době také dochází k přerozdělení půdy: území měst se zmenšují, a naopak se zvětšují soukromé pozemky bohatých jednotlivců, či území patřící církvi a armádě (Velkov 1977, 70-76). Tento trend omezení autority měst a nárůst moci bohatých soukromých osob, biskupů a vysokých vojenských úředníků je patrný již od 4. st. n. l. a trvá i v následujících stoletích. Vytváří se tak nová elitní vrstva, která disponuje mocí a ekonomickým kapitálem. Na této relativně malé skupině lidí nicméně leží i většinová daňová povinnost, která se s rostoucími náklady na udržení byrokratického aparátu a armády neustále zvyšovala, což vedlo k prohlubování ekonomické krize (Poulter 2007, 15-16; Dumanov 2015, 100).

Nájezdy Ostrogótů a Vizigótů na severní hranicích Thrákie pokračují i v průběhu 4. st. n. l. V roce 376 n. l. se dokonce část Vizigótů usídlila na území Thrákie, a to se svolením císaře Valenta, za podmínek že budou dodržovat římské právo a vyznávat křesťanskou víru. V té době však došlo i k příchodu dalších kmenů, což vyvolalo poměrně dramatický konflikt, v němž se místní venkovské obyvatelstvo postavilo na stranu Gótů a společně porazili římské vojsko v bitvě u Marcianopole a v roce 378 n. l. u Hadriánopole (Velkov 1977, 34-35). Tato porážka měla za následek destabilizaci regionu, přerušení dosavadních pořádků a ekonomickou krizi, způsobenou vleklými kampaněmi a drancováním země.

V polovině 5. st. n. l. došlo ještě k ničivějším nájezdům Ostrogótů a Hunů, které měly za důsledek pokračující ekonomický a společenský propad. To mělo za následek nejen omezení zemědělské a řemeslné produkce, ale především i pokles počtu obyvatel. Na přelomu 5. a 6. st. n. l. byla Thrákie napadena kmeny Bulharů, což vedlo k pokračují ekonomické a demografické krizi, zániku sídel a celkové transformaci společnosti směrem decentralizované vládě lokálních center (Dumanov 2015, 98-99). Tato nová centra byla menší, avšak silně opevněná a sloužila jako {\em refugium} pro okolní obyvatelstvo. Panovníci se snažili za velkých finančních nákladů udržet dunajské hranice, což však ještě více prohlubovalo ekonomickou krizi. Završením byly pak nájezdy Avarů v průběhu 6. st. n. l., vedoucí k rozpadu společenského pořádku a politické autority ve formě státu a k přerodu k novému uspořádání a pozdějšímu vytvoření bulharského království (Velkov 1977, 40-59).

\subsection[thrákie-v-průběhu-dějin-shrnutí]{Thrákie v průběhu dějin: shrnutí}

Oblast Thrákie představovala důležitou spojnici mezi západem a východem, kde se střetávaly mocenské zájmy všech velmocí a v antice byla tato oblast svědkem mnohých historických zvratů. Politická intervence Řeků, Makedonců ale i Římanů s sebou však nesla i kulturní společenské změny, které měly dlouhodobé důsledky na obyvatele Thrákie, a to jak v pozitivním, tak negativním smyslu. Epigrafické památky představují zrcadlo tohoto politicko-kulturního vývoje a vypovídají mnohem více o tehdejší společnosti, než se může jevit na první pohled.

\chapter{Mezikulturní kontakty v antickém světě}
Thrákie tvořila spojnici mezi Asií a Evropou a jak dokládají archeologické nálezy, docházelo na tomto území k přesunům obyvatelstva v obou směrech dávno před objevením prvních písemných památek. V souvislosti s řeckou kolonizací regionu a následnou přítomností řecky mluvícího obyvatelstva se badatelé věnovali ve zvýšené míře právě kontaktům řeckého a místního thráckého obyvatelstva, avšak zejména z perspektivy příchozích Řeků. Tyto kontakty bývaly v minulosti hodnoceny jednotným interpretačním modelem hellénizace, o němž se dnes soudí, že je příliš jednostranně zaměřen a dostatečně nepostihuje mnohoznačnou realitu mezikulturních kontaktů. V rámci hledání nového interpretačního rámce došlo v posledních několika desetiletích ke změnám v přístupu k materiálu. Archeologové ve spolupráci s kolegy z příbuzných disciplín vyvinuli nové teoretické přístupy, zkoumající mezikulturní kontakty společností na základě dochované materiální kultury. Hlavním cílem této kapitoly je poskytnout přehled teoretických přístupů zabývajících se mezikulturním kontaktem a zhodnotit jejich dopad na současnou podobu archeologie, historie a v neposlední řadě i epigrafiky. Především se zaměřuji na tzv. koloniální přístupy v čele s akulturací a hellénizací, které měly největší vliv na interpretaci archeologického materiálu, a dále na tzv. post-koloniální přístupy a jejich obecný dopad na současné historické bádání. Závěrem hodlám představit zvolený teoretický přístup, z nějž do velké míry vycházím při interpretaci pramenů ve této práci.

\section[koloniální-teoretické-modely]{Koloniální teoretické modely}

Takzvané koloniální teoretické modely vycházejí z tradice evropského kolonialismu 19. a první poloviny 20. století, který vnímá západní kulturu jako kulturní a společenskou normu vzešlou z antické tradice (Dietler 2005, 33-47; Jones 1997, 33). Národy a společnosti, které se od této normy odlišují, jsou vnímány jako podřadné, a jako objekt vhodný k civilizování západní řecko-římskou kulturou.

Nejdříve se zaměřím na koncepty akulturace a hellénizace, které, ač koncepty 19. století, hluboce rezonovaly v archeologických a antropologických publikacích po většinu 20. století, a v případě Thrákie mají ze specifických důvodů vliv na interpretaci materiálu i dodnes. I přes svůj velký dopad na současné bádání mají tyto přístupy celou řadu nedostatků. Hlavními body kritiky je jejich jednostranné zaměření a interpretace mezikulturních kontaktů, dále pak monolitická neměnnost a opomíjení často bohatých kontextů a situací vznikajících z reality společenských a kulturních interakcí (Dietler 2005, 33-47).

\subsection[akulturace]{Akulturace}

Pojem akulturace nikdy nepředstavoval jeden ucelený model, ale spíše teoretický směr vysvětlující a popisující proces kulturní změny a důsledky kontaktu odlišných kultur, a to jak společensky, kulturně, tak materiálně (Redfield {\em et al}. 1936, 149)\footnote{Redfield {\em et al.} 1936, 149: „{\em Acculturation compherehends those phenomena which result when groups of individuals having different cultures come into continuous first-hand contact, with subsequent changes in the original patterns of either or both groups}”.}. Akulturace se stala jedním z hlavních interpretačních přístupů pojednávající o kulturních změnách v první polovině 20. století, tedy doby silně ovlivněném evropským a americkým kolonialismem a imigrací (Cusick 1998, 24). Akulturace našla své uplatnění v psychologii, sociologii, či archeologii, vždy se ale jednalo o vysvětlení procesu změny u jedince či ve společnosti na základě kontaktu s jinou kulturou.

Akulturační teorie předpokládá adopci či úplné nahrazení kulturou s propracovanější společenskou strukturou v prostředí méně vyvinuté kultury. Evropská, či západní, kultura, a tedy i kultura starověkého Řecka a Říma, bývá v rámci akulturačních teorií chápána jako nadřazená kultuře původní, a ta je naopak chápána pouze jako pasivní příjemce a kultura méně vyspělá. Reakce na mezikulturní kontakt bývají v rámci akulturační teorie interpretovány různě, počínaje úplnou adopcí nové kultury a odvržením staré, adaptací a splynutím s původní kulturou, úplným odmítnutím, či tzv. {\em code-switching}, čili používáním jedné či druhé kulturní normy dle aktuální situace (Cusick 1998, 29).\footnote{V publikaci {\em Memorandum for the Study of Acculturation} z roku 1936 (Redfield {\em et al.} 1936, 149-152) autoři sumarizují tři možné druhy reakcí přijímající kultury jako následující: a) přijetí nové kultury za částečné či úplné ztráty kultury původní, b) adaptaci nové kultury a spojení s kulturou původní, c) reakci a odmítnutí nové kultury.}

Autoři zmíněného {\em Memoranda} (Redfield {\em et al.} 1936, 150-151) poukazují na možné způsoby studia akulturace. Jako první bod uvádějí určení druhu a charakteru kontaktů mezi oběma kulturami\footnote{Zda kontakty probíhají mezi celými skupinami, či vybranými jednotlivci se speciálním posláním, zda jsou kontakty přátelské či nepřátelské, zda probíhají mezi skupinami stejné velikosti, na stejném stupni společenské komplexity, zda probíhají na území obývané jednou ze stran, či na zcela novém území.}, dále analýzu situací, v nichž ke kontaktům dochází\footnote{Zda byly elementy nové kultury představeny silou, či se jednalo o dobrovolnou akci, zda mezi oběma stranami panovala společenská nerovnost, a pokud existovala, zda se jednalo o politickou či společenskou dominanci jedné strany.}, zhodnocení samotného procesu akulturace\footnote{Jaké prvky společnosti či jedince se proměňují ve vztahu k druhu kontaktu a vázané na konkrétní situaci; dále k jakým změnám dochází pod nátlakem a jaké prvky jsou odmítnuty a z jakého důvodu. Dalším bodem je oboustranné zhodnocení praktických výhod akulturace, např. ekonomického prospěchu či politické dominance, a jiných motivů, jako je např. získaní společenské prestiže, či návaznost na prvky v kultuře již dříve existující.} a nakonec integraci nových kulturních prvků do původní kultury.\footnote{Autoři hodnotí dobu, jaká uplynula od představení prvku, obtížnost přijetí nového prvku v rámci původní kultury, dále míru nutnosti přizpůsobení původní kultury novému uspořádání.} Autoři v {\em Memorandu} (Redfield {\em et al.} 1936) nastínili základní metodologii a přístupy ke studiu změn ve společnosti, které jsou přínosné i v dnešní době, avšak i přesto zůstává akulturační model příliš obecný a jednostranně zaměřený pouze na změnu kultury příjemce. Mezikulturní kontakty probíhají však oběma směry a dochází ke změnám v kultuře obou zúčastněných stran, nikoliv pouze v kultuře s méně propracovaným systémem ideových hodnot. Tento jednostranný a částečně diskriminující přístup byl kritizován celou řadou vědců pro své zjednodušování skutečnosti a opomíjení skutečného stavu, kde se kultury ovlivňují navzájem, a proto se dnes od jeho užívání upouští (Cusick 1998, 23).

\subsubsection[akulturace-v-archeologii-a-historických-vědách]{Akulturace v archeologii a historických vědách}

Uplatnění akulturačního přístupu v archeologii a historických vědách se do velké míry zaobírá pouze adopcí a adaptací materiální kultury. Přítomnost materiálních projevů typických pro jednu kulturu mimo obvyklé území bývá pak často vysvětlována právě akulturací tamního obyvatelstva; jinými slovy čím větší množství cizího materiálu je nalezeno, tím větší bývá přisuzován akulturační vliv příchozí kultury na tamní obyvatelstvo. Často bývají změny v materiální kultuře vykládány pouze jako jevy vedoucí ke změně identity osob, ač existují i jiná možná vysvětlení, jako například změna trendu a poptávky nebo změna technologie (Cusick 1998, 31). Materiální památky samy o sobě většinou neposkytují dostatek informací k odlišení akulturace, tedy adopcí či adaptací cizí kultury, a pouhého používání předmětů pocházejících z jiné kultury, a proto se upouští od užívání akulturace jako interpretačního rámce kulturního kontaktu (Jones 1997, 29-39).\footnote{Terminologie používaná pro vysvětlení akulturačních procesů se stala natolik běžnou, že je mnohdy užívána bez přímé souvislosti s akulturačním přístupem či bez reflexe a znalosti problematiky.}

\subsection[hellénizace]{Hellénizace}

Teoretický koncept hellénizace vychází z akulturačního přístupu, který aplikuje specificky na kontakty s řeckým světem (Boardman 1980; Tsetskladze 2006). Akulturace i hellénizace sdílejí velké množství charakteristik, jako je nerovné rozdělení moci a kulturní dominance mezi hlavními aktéry a dále jednosměrnost kulturní výměny. Základním procesem kulturní změny u obou přístupů je difuze, kdy se materiální objekty, myšlenky a koncepty šíří od společensky vyvinutější společnosti k méně vyvinuté kultuře. Ač neexistuje jednotná definice hellénizace, většina autorů se shodne, že se jedná kontinuální progresivní proces adopce řeckého jazyka, materiální i nehmotné kultury, způsobu života, a v neposlední řadě řecké identity a příslušnosti k řecké komunitě neřeckým obyvatelstvem.\footnote{Vranič 2014, 33: „{\em It is widely believed that the ancient “Hellenization” process, traditionally perceived as a simple and unilateral “spreading of Greek influences", without any recognition of reciprocity, resistance, and non-Greek agency in the Mediterranean, begins with the initial colonial encounters in the Archaic period}.”} Hellénizace je vnímána jako nevratný proces se stoupající tendencí, tj. pokud je někdo hellénizován, nemůže se „odhellénizovat” a navrátit se k původnímu stavu (Vlassopoulos 2013, 9).\footnote{ Termín je odvozen od řeckého slovesa „Ἑλληνίζω”, což primárně znamená mluvit řeckým jazykem (LSJ). V přeneseném slova smyslu a v pasivní formě se používal tento termín již v antice, a to ve smyslu pořečťovat, činit řeckým ve smyslu jazyka a kultury. Opačný význam neslo sloveso „βαρβᾰρίζω”, které mohlo znamenat jak „mluvit jiným jazykem, než řecky”, tak i „chovat se jinak, než je obvyklé pro Řeka”, případně „stranit Neřekům”. Původní význam termínu „βάρβᾰρος” byl „člověk mluvící jinou řečí než řečtina” a jeho použití nemělo negativní nádech (Malkin 2004, 345). Pejorativní význam získalo slovo až druhotně v kontextu řecko-perských válek. Moderní použití slova barbar, či barbarský se odvozuje právě od negativních zkušeností Řeků s multikulturní a multinárodnostní perskou říší v průběhu 5. století př. n. l.}

Nejstarší definice hellénizace pochází od Hérodota, řeckého autora 5. st. př. n. l. který se pokusil se definovat podstatu {\em hellénicity} na základě reflexe společných rysů, které spolu Řekové sdíleli.\footnote{Hdt. 8.144.2: τὸ Ἑλληνικὸν.} Na základě srovnání mnoha řeckých komunit té doby Hérodotos formuloval čtyři kritéria příslušnosti k řecké komunitě, která se do velké míry shodují s definicí hellénizace tak, jak jí chápou moderní badatelé. Společnými rysy, dle Hérodota, byla sdílená minulost a fakt, že všichni Řekové mají společné předky a pojí je krevní pouto. Dalším charakteristickými rysy byly společná řeč, a fakt, že vedou podobný způsob života a vyznávají stejnou víru (8.144.2).\footnote{V současnosti je tato definice považována za příliš zobecňující, jednostranný pohled, který opomíjí variabilitu etnické identity v antickém světě (Zacharia 2008, 34-36; Janouchová 2016). Podobná kritika se objevila nejen vůči Hérodotově pojetí řecké příslušnosti, ale i vůči modernímu teoretickému konceptu hellénizace, jak byl chápán v 19. a 20. st. n. l.} Jakékoliv odchýlení od kulturní normy, ať už kulturní, tak morální, je prezentováno v ostrém kontrastu „těch druhých” (Hartog 1988 {[}1980{]}, 212-259; Vlassopoulos 2013b, 56-57). Hérodotův dualismus Řeků a „těch druhých” barbarů se ve velké míře odráží i v moderní sekundární literatuře zabývající se neřecky mluvícími národy antického starověku (Harrison 2002; Zacharia 2008). Proto i západní akademický svět a interpretace materiálu vycházející ze studia řecky psané literatury, jsou svou podstatou hellénocentrické a hellénofilské a přijímají hellénizaci jako jediné možné východisko kontaktu řecké a neřecké populace.\footnote{Jak je možné si povšimnout v rámci úvodní kapitoly, řečtí autoři, jako Hérodotos, Thúkýdidés či Xenofón, se nezaměřovali specificky na popis neřecky mluvících národů, ale zmiňovali se o nich v souvislosti s vysvětlením širších historických skutečností a pak zejména v komparaci s popisem řeckého etnika. Cílem mnoha těchto prací nebylo zaznamenávat etnografickou skutečnost o neřeckých národech, ale spíše bavit řeckého posluchače výčtem kuriozit a neobvyklých norem chování (Hartog 1988 {[}1980{]}, 230-237). Pro lepší ilustraci slouží například Hérodotovo dílo, jehož jednotlivé části, {\em logoi,} jsou sbírkou etnografických zajímavostí a faktů o neřeckých národech tak, jak byly prezentovány řeckému posluchači.}

Moderní teoretický koncept hellénizace je založen do značné míry na studiu dochovaných literárních pramenů, převážně psaných řecky, určených pro řecké publikum. Michael Dietler správně podotkl, že západní kultura vychází z hellénofilství 19. století, a je do značné míry postavena na studiu těchto textů, což by vysvětlovalo popularitu konceptu hellénizace v 19. a 20. století (Dietler 2005, 35-51). Řecká kultura byla považována v antice, tak i v badatelských kruzích za kulturní normu, a cokoliv odlišného bylo považováno za podřadné, barbarské v moderním slova smyslu.

Zcela v duchu akulturačních teorií i hellénizační přístup předpokládá kulturní nadřazenost jedné společnosti nad druhou (Stein 2005, 16). Neřecká společnost automaticky považována za kulturu vystav absorbující výdobytky kolonizující řecké společnosti, jako je společenská organizace či propracovaný náboženský systém (Whitehouse a Wilkins 1989, 103; Owen 2005, 12-13; Dietler 2005, 55-61).\footnote{Whitehouse a Wilkins 1995, 103: „{\em Equally invidious is the strongly pro-Greek prejudice of most scholars, which leads them to regard all things Greek as inherently superior. It follows that Greekness is seen as something that other societies will acquire through simple exposure---like measles (but nicer!)}.”} Jak akulturační, tak hellénizační přístup většinou ignoruje podíl místního neřeckého obyvatelstva na vytváření společné kultury, případně neuznává, že k procesu kulturní výměny mohlo docházet a často i docházelo z obou stran (Antonaccio 2001, 126). Místní populace je vnímána jako kulturně vyprázdněné území, jehož obyvatelé jsou příchozími Řeky civilizováni na vyšší úroveň a pasivně přijímají novou kulturu.

Whitehouse a Wilkins (2005, 103) udávají, že koncept hellénizace je možné použít jako interpretační rámec například pro srovnávání stylistických změn ve výrobě keramiky, či architektury, ale považují ho za zcela nevhodný pro studium změn ve struktuře společnosti. Hellénizační přístup bývá aplikován {\em en bloc} na celou populaci, bez reflexe lokálních variant, území větší či menší rezistence, či naopak otevřenosti nově příchozí kultuře. Hellénizace zcela opomíjí místní společenský, politický a ekonomický kontext, který hrál zásadní roli při kontaktu s novou kulturou. Kritik Hellénizace Michael Dietler (1997, 296) dokonce podotýká, že model sám osobně nenese interpretační hodnotu, pouze dokumentuje a popisuje situaci bez skutečného porozumění mezikulturních kontaktů a jejich důsledku pro všechny zúčastněné komunity. Tím se stává příliš obecným a jednostranně zaměřeným modelem i pro studium epigrafických památek jakožto produktů dané společnosti.

\subsubsection[srovnání-teoretických-konceptů-hellénizace-a-romanizace]{Srovnání teoretických konceptů hellénizace a romanizace}

Pro lepší pochopení vývoje teoretického konceptu hellénizace a jeho velké obliby a rozšíření mezi badateli je nutné stručně pohovořit o teoretickém pozadí, z nějž hellénizace vychází. Koncept hellénizace se totiž do velké míry inspiroval tematicky spřízněnou romanizací, tedy velmi oblíbeným teoretickým přístupem 19. a počátku 20. století, který si nicméně udržel své příznivce i do současné doby (Woolf 1994, 116; Freeman 1997, 27-50; Millett 1990). Koncept romanizace se zabývá taktéž adopcí materiální kultury, jazyka, zvyklostí, způsobu života a přijetí nové identity původními obyvateli nově získaných území římského impéria (Woolf 1997, 339; Jones 1997, 29-39). K rozvoji a oblibě konceptu romanizace přispěl nejen velký rozsah římské říše, ale i mnohonárodnostní složení jeho obyvatel a nutnost teoretického přístupu, který by vysvětloval velmi podobné změny v materiální kultuře obyvatel nově získaných území v přímé souvislosti s působením římského státního aparátu a zvýšenou vojenskou přítomností na daném území.

Společným rysem hellénizace a romanizace je jejich interpretace kulturní změny jako nevyhnutelné a přirozené reakce původních kultur na setkání s kulturně vyspělejšími civilizacemi (Champion 1995, 15; Dietler 1997, 336-337). Oba dva směry jsou primárně zaměřené na dominanci řecké a římské kultury nad pasivními kulturami původního obyvatelstva, jehož reakce na kontaktní situace jsou obecně v literatuře přehlíženy a recipročním změnám v řecké a římské kultuře není věnován dostatečný prostor. Oba dva směry jsou striktně dvoudimenzionální a soustředí se na kontakty mezi Řeky a Neřeky, Římany a Neřímany, kde řeckou, respektive římskou stranu považují kulturně, politicky i ekonomicky dominantní. Tento striktní dualismus a černobílé vidění obou přístupů je v posledních dvaceti letech kritizováno jako příliš zobecňující, opomíjející historickou realitu a jemné nuance a variace mezikulturních vztahů (Dietler 2005, 55-57; Mihailovič a Jankovič 2014, xv). Kritice čelí i omezenost hlavních proudů obou dvou směrů počátku 20. století na interakce převážně společenských elit a opomíjení nižších vrstev společnosti (Hingley 1997, 83). Tento rys vychází z faktu, že primární pramenem studia obou směrů byly zejména písemné prameny, které zaznamenávají právě elitní vrstvy společnosti.

V neposlední řadě se setkáváme s nejednotným metodologickým přístupem k interpretaci materiálu. Práce zabývající se hellénizací pojímají tento termín poměrně vágně, či nemají potřebu definovat teoretický přístup k hellénizaci s předpokladem, že termín nese vysvětlení sám o sobě (Dominguez 1999, 324; Vlassopoulos 2013, 9). Práce zabývající se romanizací většinou předkládají detailnější metodologii, nicméně proces sám o sobě je taktéž chápán jako dostačující vysvětlení kulturní změny (Mattingly 1997, 9-17). Ač oba dva koncepty spojuje do velké míry snaha postihnout změnu kultur v reakci na mezikulturní kontakt, prostředí, jež romanizace popisuje se od řeckého prostředí liší poměrně radikálně. Pojetí romanizace do velké míry ovlivnil charakter římského impéria, velký rozsah území a mnohonárodnostní složení, propracovaný státní aparát a dobře organizovaná armáda, což jsou prvky vyskytující se v řecké kultuře v menší míře, v nejednotné formě či dokonce vůbec. V řeckém prostředí totiž, na rozdíl od římské říše, neexistovala jednotná politická autorita, jednotná materiální kultura, a dokonce ani jednotný jazyk či kolektivní identita, což jsou zásadní podmínky pro úspěšnou aplikaci hellénizačního modelu v praxi. S tím souvisí i odlišné zaměření obou přístupů. Zatímco hellénizace se zaměřuje spíše na změny v kultuře a identitě obyvatel, romanizace se soustředí především na společensko-politické a ekonomické důsledky v různých regionech římského impéria s ohledem na projevy v tamní materiální kultuře (Freeman 1997, 27).

\subsubsection[nejednotný-charakter-řecké-kultury-a-identity]{Nejednotný charakter řecké kultury a identity}

Hellénizační model je založený na předpokladu, že neřecká kultura absorbuje a asimiluje řeckou materiální kulturu, řecký jazyk, způsob života a s tím spojenou i řeckou identitu (Owen 2005, 12-13; Zacharia 2008, 21). Aby tento model mohl fungovat, nutným předpokladem je nejen jednotný charakter řecké materiální kultury, ale také jednotný a statický charakter řecké identity, který je navíc úzce spojený s materiální kulturou. Archeologický výzkum v mnohém vyvrátil tvrzení, že je možné přímo spojovat etnickou identitu s materiální kulturou (Jones 1997, 106-127; Antonaccio 2001, 129). Jinými slovy, rozšíření tradičních archeologických kultur neodpovídá rozmístění jednotlivých etnických skupin a produkce určitého typu keramiky či dekorace je vázána spíše na individuální vkus a použitou technologii, než na etnicitu (Jones 1997, 128-129). Archeologické výzkumy navíc dokládají, že v antice neexistovala jednotná a normativní řecká materiální kultura, ale jednalo se o lokální varianty, které se více či méně navzájem inspirovaly. Zpochybněn byl i argument na provázanost jednotlivých zvyklostí čistě s řeckou etnicitou, což mimo jiné dokazuje i absence jednotného pohřebního ritu v řeckých koloniích na pobřeží Černého moře (Damyanov 2012). Dříve se usuzovalo, že se řecké etnikum vyznačovalo užitím kremace a místní Thrácké obyvatelstvo pohřbívalo inhumací, avšak paralelní existence obou ritů znemožnila etnickou identifikaci pouze na základě pohřebních zvyklostí. Stejně tak přítomnost objektů řecké provenience nedokazuje, že místo bylo obýváno Řeky, či obyvatelstvem, které se používáním řeckých předmětů stalo „hellénizovaným” a přijalo řeckou identitu (Vranič 2014, 41).\footnote{V nadsázce je možné tento přístup aplikovat i na současný svět: pokud by provenience používaných předmětů určovala etnicitu, více jak polovina moderního světa by byla obývána Číňany, vzhledem k tamní masové produkci předmětů denní potřeby.}

Druhým problematickým bodem je otázka řecké identity a její jednotnosti. François Hartog (1988 {[}1980{]})\footnote{1980 původní vydání, 1988 překlad do angličtiny.} a Edith Hall (1989) rozpoutali relativně bouřlivou akademickou debatu na téma řecké identity v antice a jejích projevech v kontaktu s jinými etnickými skupinami. Badatelé dnes všeobecně přijímají, že tzv. hellénicita, čili jednotná řecká etnicita, je konstruktem moderní společnosti a nejedná se o vrozenou vlastnost (tzv. {\em primordial ethnicity}). Naopak, většinově se dnes kloní k názoru, že etnicita tvořila a tvoří jen jednu ze složek identity, jejíž projevy se mohly mění dle aktuálních podmínek (tzv. {\em situational ethnicity}; Barth 1969; Hall 1997, 17-33; Jones 1997, 13; Demetriou 2012, 14; Mackil 2014, 270-271; Diaz-Andreu 2015, 102).

Namísto jednotné řecké etnicity je tak vhodné spíše mluvit o vědom řecké kolektivní identity, tzv. panhellénismu, která vznikala za velmi specifických podmínek a měla omezené trvání.\footnote{Vzhledem ke geografické členitosti řeckého území je velmi obtížné mluvit o řeckém národu v moderním slova smyslu (Gellner 1983). Řecká komunita neměla jednotné politické ani kulturní centrum a přírodní podmínky a charakter individuálních řeckých obcí nenahrával vytvoření panhellénské identity, která by si uchovala permanentní charakter. Panhellénská identita se objevovala v řecké kultuře spíše epizodicky, jako reakce na setkávání s jinými kulturami, či jako důsledek všeřeckých náboženských slavností.} Dle antropologické teorie dochází k formování příslušnosti k určité skupině dvěma možnými způsoby: první tzv. opoziční identita se formuluje při setkání s jinou skupinou, např. v průběhu válečných konfliktů, diplomatického vyjednávání, či obchodních transakcí (Barth 1969; Malkin 2014). Druhá, tzv. aggregační identita vzniká uvnitř komunity afiliací s dalšími jejími členy, např. v průběhu panhellénských slavností, či v prostředí multikulturní společnosti řeckých {\em emporií} a {\em apoikií} (Morgan 2001; Demetriou 2012; Vlassopoulos 2013).

Panhellénská identita hrála v řecké společnosti relativně marginální roli, větší důležitost zaujímala spíše přináležitost s bezprostředním regionem, či mateřskou obcí, čemuž nasvědčuje i politické uspořádání řecké komunity. Irad Malkin (2011) správně přirovnal tehdejší situaci k „malým světům”, autonomním jednotkám, navzájem propojeným skrz moře a mořeplavbu. I přes tento fakt propojenosti Středozemního a Černého moře si řecké komunity uchovávaly lokální identity, které měly zásadní formativní charakter na vědomí kolektivní příslušnosti tehdejších obyvatel. Dokonce i řecký jazyk, jakožto jeden z prvků, které mají být společné všem Řekům, byl roztříštěn na lokální dialekty.\footnote{Nicméně na rozdíl od jazyků používaných místní populací, byly tyto lokální dialekty více či méně navzájem srozumitelné, ač tyto rozdíly reflektují sami mluvčí (Morpurgo Davies 2002, 161-168).}

Jak jsem již naznačila, úspěšné uplatnění hellénizačního modelu naráží na své limity z několika důvodů, jako například absence jednotné materiální kultury, která by se dala definovat jako řecká. Dále je to mnohotvárnost a proměnlivost řecké identity, která se lišila nejen v závislosti na čase, ale i místně dle aktuální historicko-politické situace (Woolf 1994, 118). Globální použití hellénizačního modelu opomíjí celou řadu společenských interakcí a ignoruje aktivní roli neřecky mluvícího obyvatelstva v rámci celého procesu a nahlíží na neřeckou kulturu jako na nižší úrovni a primitivnější.

Ač se mnozí badatelé staví {\em a priori} negativně k hellénizaci s tím, že se jedná o překonaný model, který nenabízí nový úhel pohledu, a naopak mnohé skutečnosti zjednodušuje, i přesto je hellénizační rétorika natolik zakotvená v moderní společnosti, že je téměř nemožné tento model zcela ignorovat. Východiskem z dané situace může být přístup sledující jak celospolečenský kontext, tak i faktory a možné motivace ovlivňující mezikulturní kontakt a jeho důsledky, podobně jak navrhuje text {\em Memoranda pro studium akulturace} (Redfield {\em et al.} 1936). Tento alternativní přístup se v prvé řadě musí oprostit od kulturních předsudků a přistupovat k všem zúčastněným stranám nezaujatě, nikoliv pouze z pozice řecké společnosti, jak bylo běžné zcela v duchu tradičního pojetí hellénizace. Podobný trend začíná rezonovat i teoretickými archeologickými pracemi posledních 20 let (Dietler 1997; 2005; Mihailovič a Jankovič 2014, xv-xvi).

\subsubsection[hellénizace-v-thrákii]{Hellénizace v Thrákii}

Ačkoliv jsou akulturační modely dnes považovány za vágní a zastaralé, je nutné vzít v potaz, že koncept hellénizace ovlivnil velkou řadu badatelů ve 20. století a nadále si udržuje určitý vliv v široké veřejnosti a politických hnutích s národnostní tematikou, převážně v balkánských zemích s pozůstatky antických kultur. Určitá forma dualismu a předsudky o kulturní dominanci řecko-římské kultury se nevyhýbají ani badatelům zabývajícím se kontakty Thráků a Řeků (Asheri 1990; Xydopoulos 2004; 2007).

V průběhu 20. století byl koncept hellénizace často využíván k politickým účelům, a to například k ospravedlnění přináležitosti ke vznikajícím panevropským uskupením s poukazem na historii daných zemí a jejich propojenost s hellénským kulturním dědictvím Evropy (Owen 2005, 14; 18-21; Vranič 2014a, 39-42). Zároveň s tím ale na Balkáně existoval trend vyzdvihující předřecké kořeny moderních národů, které do jisté míry romantizoval a heroizoval v duchu moderního nacionalismu. Došlo tak k velmi zajímavé kombinaci těchto nacionalistických směrů poloviny 20. století, které přisuzovaly Thrákům poměrně aktivní roli v procesu hellénizace, a Řeky zmiňovaly mnohdy pouze okrajově (Vranič 2014b, 166-167; Fol a Marazov 1977; Fol 1997, 73).

Hellénizační model byl do nedávné minulosti používán mnoha badateli k popsání kontaktů mezi Řeky a neřecky mluvícími území známého jako Thrákie a důsledků těchto kontaktů (Danov 1976; Samsaris 1980; Boardman 1980; Fol 1997; Bouzek {\em et al.} 1996; 2002; 2007; Archibald 1998; Tsetshladze 2006; Tiverios 2008; Sharankov 2011).\footnote{Pro přehled užití hellénizačního přístupu v balkánské archeologii Vranič (2012, 29-31) a Vranič (2014b, 161-163).} Řekové jsou v rámci hellénizačního přístupu chápáni jako kolonizující nadřazený národ, který přináší kulturu a civilizaci primitivnějším Thrákům (Owens 2005, 13). Hellénizace byla ztotožňována s nárůstem společenské komplexity, či jinými slovy příchodem civilizace, a zaměňována za jevy jako urbanizace, centralizace politické moci apod. Přítomnost řecké materiální kultury, nebo její imitace, byla v archeologických kruzích často interpretována jako důkaz hellénizace obyvatelstva, jejíž intenzita byla určována množstvím nalezených předmětů řecké provenience (Vranič 2014a, 36-37). Převládajícím motivem hellénizace byly ekonomické faktory a vzájemný obchod mezi Řeky a Thráky (Vranič 2012, 31-32).

Primárním interpretačním rámcem mnoha prací jsou řecké literární zdroje, které popisují první kontakty s Thráky, které líčí jako násilné a bojovné obyvatele úrodného území (např. Archilochos, Diehls frg. 2, 6 a 51; Owen 2005, 19).\footnote{Podrobněji o charakteru literárních pramenů pojednávám v kapitole 1.} Thrákové jsou následně prezentováni jako divoká stvoření s méně vyvinutým náboženským systémem, kteří ale ochotně přijímají řeckou materiální kulturu a spolu s ní i řeckou identitu (Asheri 1990, 133-150). Jak ale poukazuji v kapitole 1, obraz Thráků v literárních pramenech je do značné míry ovlivněn literárním záměrem konkrétního autora a tehdejší politickou situací, a tudíž je nutné k informacím v nich dochovaným přistupovat nanejvýš kriticky a srovnávat s dalšími dostupnými prameny. Například archeologické práce posledních let dokazují, že první kontakty mezi Řeky a Thráky nebyly pouze násilné, ale že se jednalo o daleko komplexnější systém vztahů, než jak je popisují právě řecké literární zdroje. V některých případech bylo dokonce zaznamenáno soužití těchto dvou skupin pohromadě, bez archeologicky doložených známek násilí (Ilieva 2011; Damyanov 2012).

Mnozí historikové a archeologové považují vrchol hellénizace Balkánu dobu hellénismu, kdy došlo k intenzifikaci kontaktů mezi Řeky a Thráky, a tedy i k navýšení počtu kulturních výpůjček. Hlavní podíl na změnách společenské struktury měly dle Papazoglu (1980) místní elity, které se vědomě snažily imitovat hellénistické vladaře. Tato tendence je stále do jisté míry patrná i v současné literatuře, avšak badatelé se snaží reflektovat postkoloniální teoretické přístupy (Nankov 2008; 2009; 2012; Vranič 2014a; Vranič 2014b). Dle Nankova (2012) zásadní roli ve změně společnosti hráli navrátivší se vojáci, sloužící na dvorech a ve vojscích hellénistických vladařů, kteří se později usídlili v Thrákii (Hérakleia Sintská, Seuthopolis), v kombinaci s ekonomicko-politicky motivovanými snahami místních elit. Bouzek (Bouzek {\em et al.} 1996; 2002; 2007) a Archibald (1998; 2011) pak poukazují na aktivní zapojení řeckých obchodních kontaktů a Řeků sídlících přímo v Thrákii již od 5. st. př. n. l. a přisuzující hlavní roli na změně společenské struktury ekonomickým faktorům.

Hellénizace v thráckém prostředí je tradičně vnímána jako vliv řecké materiální a myšlenkové kultury na thrácké obyvatelstvo a s tím související proměny projevů materiální kultury, i uspořádání společnosti. Tento vliv je povětšinou jednostranně zaměřen a dochází k ovlivňování ze strany řecké kultury směrem ke kultuře thrácké. Práce zabývající se hellénizací v Thrákii se většinou zaměřují na dobu hellénismu a nereflektují návaznost na předcházející či následující období. Řecká i thrácká kultura jsou vnímány jako monolitické celky, bez reflexe regionálních variant, lokálních identit, či variability druhů interakcí. Thrácká společnost je považována, zcela v duchu řeckých literárních pramenů, za méně vyvinutou, a celkově je jí věnováno méně pozornosti.

\subsection[teorie-světových-systémů-a-její-návaznost-na-hellénizaci]{Teorie světových systémů a její návaznost na hellénizaci}

Jako alternativní teoretický směr, který do velké míry vycházel z podobných principů jako hellénizace, vznikla v 70. létech 20. století {\em teorie světových systémů} ({\em world-systems theory}). Teorie světových systémů se snažila vysvětlit změny ve společnosti na základě ekonomické teorie Immanuella Wallersteina (1974). Wallerstein s její pomocí vysvětloval rozdělení ekonomické síly a moci, pracovní síly, ekonomického potenciálu a pohybu zboží v moderním světě. Tato původně čistě ekonomická teorie se stala velmi oblíbenou mezi archeology, zabývajícími se mezikulturními kontakty a distribucí materiální kultury (Frankenstein a Rowlands 1978; Rowlands, Larsen a Kristiansen 1987; Bintliff 1996; Stein 1998; Champion 2005; Harding 2013).\footnote{Tato teorie se stala v archeologii též známá pod pojmem „centrum - periferie” ({\em core-periphery}) a sloužila primárně jako vysvětlení pohybu materiální kultury a dynamiky vzájemných kontaktů společností.}

Základním předpokladem teorie světových systémů je propojenost regionálního a nadregionálního ekonomického systému, jehož jednotlivé části se ovlivňují mezi sebou. Světové ekonomické systémy se skládají z center, periferií a semi-perifierií, přičemž v centrech se koncentruje většina ekonomické síly a politické moci. Tato moc je zajištěna existencí vojska, které zajišťuje hladký chod celého systému a dodává centru potřebnou autoritu. Z periferie do centra proudí pracovní síla a surový materiál, který je pak zpracován specialisty žijícími v centru (Rowlands 1987, 4). Periferie je většinou v rámci teorie vnímána jako pasivní součást systému, která dodává potřebný materiál a levnou pracovní sílu, a je oproti centru ekonomicky i kulturně na nižší úrovni. Centra jsou oproti periferiím považována za společnost vyvinutější na škále společenské komplexity, jako např. stát vs. předstátní uskupení (Stein 1998, 223-225). Systém center a periferií je tak poměrně nestabilní a založený na nerovné distribuci ekonomického kapitálu a moci. Systém center a periferií se vyvíjí v závislosti na vnějších a vnitřních okolnostech: centrum může postupem času degradovat na pouhou periferii, a naopak periferie se může proměnit na semi-periferii, či dokonce na centrum (Frankenstein a Rowlands 1978, 80-81).

V rámci archeologické aplikace je periferie na předstátní úrovni vedena skupinou mezi sebou soupeřících aristokratů, kteří organizují výměnu surového materiálu mezi periferií a centrem výměnou za vlastní prospěch a protislužby (Frankenstein a Rowlands 1978, 76).\footnote{Elity zajišťovaly přísun materiálu do centra, což v tomto případě představuje Středomoří, a zároveň si v periferii udržovaly výsadní postavení, které demonstrovaly právě vlastnictvím luxusních nádob pocházejícím z centra. Za odměnu elitám centrum poskytuje prestižní předměty zhotovené specialisty v centru. Většinou se mohlo jednat o nádoby, šperky či zbraně z drahého kovu, které aristokratům v rámci periferie sloužily jako prestižní předměty, poukazující na jejich vysoké postavení a společenský status (Frankenstein a Rowlands 1978, 76-77; Rowlands 1987, 5). Pouze pokud si elity zajistily kontrolu nad výměnou zboží s centrem, mohly úspěšně kontrolovat redistribuci importovaného luxusního zboží. Distribucí luxusního zboží si náčelníci zajišťovali dostatečný počet následovníků, a zavazovali si tak jejich přízeň a podporu do budoucnosti (Sahlins 1963, 288-296; Whitley 1991, 349-350).} Archeologové tak za využití ekonomické teorie vysvětlují na tomto koloběhu materiálu a protislužeb přítomnost luxusních nádob v kmenově řízených společnostech v jihozápadním Německu rané doby železné.

Michael Dietler kritizoval tento model pro přílišné zjednodušování reality a popření aktivní role původního obyvatelstva (2005, 58-61). Dietler porovnával teorii světových systémů s hellénizačním modelem, podobně jako u hellénizace, je centrum prezentováno jako ekonomicky a kulturně jednotné a nadřazené periferii. Dalším bodem Dietlerovy kritiky je zaměření teorie pouze na ekonomické vysvětlení kontaktů a cirkulaci a rozmístění kapitálu, a nikoliv na kulturní a společenské následky vycházející ze setkání s cizí kulturou. Primárním prostředkem kontaktu je u teorie světových systémů obchod, což však nezahrnuje komplexnost kontaktů a složitost vztahů k nimž mohlo docházet.\footnote{Jen pro příklad Colin Renfrew (1986, 8) uvádí jako možné druh mezikulturních interakcí vojenské konflikty, diplomatické vztahy, průzkum nových oblastí, migraci, konkurenčního soupeření apod.} Teorie světových systémů pouze málokdy zahrnuje i jiné než obchodní motivace a způsoby šíření luxusní nádob, jako je například výměna darů v rámci budování sítě často nadregionálních společenských vztahů a kontaktů (Mauss 1966).\footnote{Tzv. {\em gift-giving society}, více Mauss (1966).}

Hlavním přínosem použití teorie světových systémů v archeologii bylo kladení většího důrazu na vzájemnou propojenost regionů, a zároveň snaha o vysvětlení nerovnoměrného rozdělení kapitálu na úrovni regionů (Harding 2013, 379). Teorie světových systémů byla poměrně úspěšně aplikována na větší regionální a nadregionální celky, kde bylo možné definovat jedno ekonomické a politické centrum.\footnote{V případě starověkého světa se jednalo o říše rozkládající se na velké ploše jako např. Řím, Mezopotámie, Egypt, či pozdější Mayská říše (Rowlands 1987, 5).} Oproti tomu aplikace teorie světových systémů na řecký svět v sobě nese několik problémů: řecky mluvící svět nebyl po většinu své existence centralizovaný a nespadal pod jedno ekonomické a politické centrum (Stein 1998, 226).\footnote{John Bintliff se pokusil aplikovat teorii světových systémů na regionální celky v rámci antického Řecka (1996), avšak jeho model zahrnoval poměrně obecné ekonomické a produkční trendy a celková použitelnost tohoto modelu pro studium mezikulturních kontaktů byla velmi omezená.}

I přes na svou dobu inovativní postoj bývá teorie světových systémů kritizována pro svou aplikaci moderní ekonomické teorie na antickou ekonomiku, která jednak nedosahovala měřítek moderních ekonomik, ale pravděpodobně se řídila i jinými pravidly a měla odlišnou dynamiku, než moderní společnost (Hodos 2006, 5-7).

\subsubsection[teorie-světových-systémů-v-thrákii]{Teorie světových systémů v Thrákii}

Ekonomicky motivované interpretace archeologického materiálu v sobě spojují jednak teorii světového systému, ale nesou i prvky hellénizačního přístupu. Hlavní motivací mezikulturního kontaktu je obchod a výměna materiálu, případně absence surového materiálu v řeckých městech a jeho nadbytek v oblastech obývaných Thráky (Vranič 2012, 32-36). Oblasti Thrákie bývá v nadregionálních studiích zabývajících se teorii světových systémů věnováno poměrně málo prostoru, většinou jako zmínka na okraji, či podpůrný argument. V rámci teorie světových systémů je Thrákie chápána jako periferie či jako semi-periferie, v závislosti na měnících se podmínkách a politické situaci, a řecká města na thráckém pobřeží jako centrum (Randsborg 1994, 99-104). Thrákie bývá ve shodě s literárními prameny vnímána jako zdroj surových materiálů, zejména stříbra, zlata a surového dřeva na stavbu lodí, otroků a námezdných vojáků (Hdt. 1.64.1; 5.53.2; Sears 2013, 31; Tsiafakis 2000; Isaac 1986; 14-15; Lavelle 1992, 14-22).\footnote{Podrobněji v kapitole 1.} Lokální studie zaměřující se pouze na Thrákii interpretují hellénizovaná města a obchodní sídliště ({\em emporia}) jako lokální ekonomická centra, kde hellénizované elity či přímo řečtí obchodníci shromažďují a následně dodávají potřebné suroviny z thrácké periferie do řeckého světa (Bouzek 1996; 2002; 2007; Bouzek a Graninger 2015).

Přítomnost luxusních předmětů v thráckém vnitrozemí je vysvětlována jako důsledek výměny zboží a protislužeb mezi centrem a aristokraty z periferie. Z literárních pramenů jsou známy případy, kdy thráčtí aristokraté zprostředkovávali kontakt s řeckými městy, jako například jistý Nymfodóros z Abdéry, který se stal prostředníkem mezi Athénami a thráckými Odrysy (Hdt. 7.137; Thuc. 2.29; Sears 2013, 27). Můžeme jen odhadovat, že odměnou jim byly luxusní předměty řecké provenience, či služby řeckých specialistů, kteří pro ně pracovali přímo v Thrákii.\footnote{Archeologické výzkumy odhalují, že si thrácká aristokracie natolik považovala předmětů řecké provenience, že je ukládala do svých, mnohdy výstavních hrobek (Theodossiev 2011, 21-25).} Fenomén bohatých válečnických hrobek s množstvím importovaných předmětů, na nichž byly navršeny monumentální mohyly, je v duchu teorie světových systémů považován za jeden ze znaků kompetitivního charakteru thrácké společnosti, která byla ovládána místními kmenovými náčelníky. Tento fenomén se objevuje v Evropě v době železné, nejen v Thrákii, ale na dalších místech Evropy (Randsborg 1994, 102).

Pokud se podíváme na detailněji na povahu kontaktů mezi Řeky a Thráky, teorie světových systémů nemůže sloužit jako univerzální vysvětlení mezikulturních interakcí. Pro ilustraci uvádím rozložení ekonomické a politické moci v 5. st. př. n. l., které se zcela vymyká binárnímu rozdělení na centrum a periferii. O politické a ekonomické dominanci Řeků nad Thráky není možné mluvit zcela jednoznačně, zejména, když literární prameny vyjadřují spíše opak. Thúkýdidés si všímá, že řecká města platila thráckému králi nemalé obnosy a s největší pravděpodobností tudíž nad ním neměla ekonomickou ani politickou moc, ale spíše naopak (Thuc. 2.97; Graham 1992, 61-62). V neposlední řadě měli Thrákové nejen přístup k nerostnému bohatství, ale literární prameny na mnoha místech poukazují na fakt, že Thrákie měla silné a početné vojsko, které pravděpodobně svým počtem převyšovalo vojska jakéhokoliv z řeckých měst na pobřeží Thrákie.\footnote{Thúkýdidés zmiňuje, že v roce 431 př. n. l. měla armáda krále Seutha až 150 000 mužů (Thuc. 2.98).} Podobně tedy jako v případě hellénizace, ani teorie světových systémů nepostihuje mnohdy jemné nuance mezikulturního styku Thráků a Řeků.

\section[přehled-nejdůležitějších-směrů-post-koloniální-modely]{Přehled nejdůležitějších směrů: post-koloniální modely}

V reakci na politický vývoj 20. století a snahu o vyrovnání se s rozpadem evropských koloniálních velmocí se i v akademickém světě objevila celá řada nových teorií a směrů. Viděno novou optikou, dříve aplikované akulturační teorie nevystihovaly komplexnost studií zabývajících se mezikulturními vztahy, a tak bylo nutné hledat nové alternativní přístupy.\footnote{Postkoloniální teoretické přístupy jsou charakterizovány svou vzájemnou provázaností a mnohostranností. Není tedy možné určit jeden převládající teoretický směr, ale jedná se spíše o amalgám několika přístupů, jehož složení se liší dle konkrétního badatele a jeho záměru. Následující výčet charakterizuje pouze nejvýznamnější z hlavních směrů současnosti. Odbornou literaturou poslední doby rezonuje proces formování místní identity jako reakce na vliv nadregionálních uskupení, jako je římské impérium (Mattingly 2010). Podílem římské říše na globalizaci společnosti a následnými projevy na místní kulturu se zabývá směr taktéž známý jako tzv. {\em glokalizace} (Pitts 2008). Dále sem patří směry zabývající se prolínání kultur a vytváření nových forem mezikomunitní komunikace a společenského uspořádání jako je hybridizace, kreolizace (Liebmann 2013) či {\em middle-ground theory} (White 1991; Antonaccio 2013). Formováním nových identit na základě kontaktů s jinými kulturami a reprezentací nově vzniklých identit v materiální kultuře se zabývá celá řada badatelů (Hodos 2006; 2010), stejně tak kontextualizací společenských role v rámci původní a nové kultury (Appadurai 1986; Dietler 1997).} Celková nespokojenost s teoretickými přístupy, které se snažily vysvětlit historické procesy jedním možným modelem, vyústila v rozvoj několika paralelních, navzájem se prolínajících teoretických směrů, jejichž zastánci se ve velké míře inspirovali v současné antropologické teorii, sociologii a v exaktních vědách. Hlavními charakteristikami těchto směrů je důraz, který kladou na multidimenzionalitu a variabilitu mezikulturních kontaktů. Více prostoru badatelé věnují národům a společnostem, které byly dříve vnímány pouze v dualistické opozici kolonizátor - kolonizovaný, a jejich různým reakcím na kontakty s jinými kulturami (Dommelen 1998, 25-26; Silliman 2013, 495).\footnote{Silliman (2013, 495) popisuje jeden z postkoloniálních směrů, nicméně jeho definice může být použita jako definice celého teoretického směru: „{\em Hybridity in a postcolonial sense tends to be a direct critique of previous versions of colonial theory that considered the effects of colonialism on indigenous people to be those of assimilation, acculturation, or even the more neutrally termed culture change. Hybridity offers a counterclaim of cultural creativity and agency, and it lends more subversion, nuance, and ambiguity than traditional assessments of the effects of colonialism}.”} Vznikly tak zcela nové a unikátní modely a přístupy k archeologickému materiálu, zaměřující se jednak na dynamiku celého procesu, ale i na důsledky kulturních interakcí všech zúčastněných stran. Dřívější čistě ekonomická vysvětlení, či motivy kulturní převahy nejsou zcela zavrženy, ale figurují pouze jako jedno z možných paralelních vysvětlení.

\subsection[lokalizace-a-zaměření-na-místní-obyvatelstvo]{Lokalizace a zaměření na místní obyvatelstvo}

Postkoloniální teoretické přístupy podtrhují fluiditu mezikulturní výměny, hybnou sílu procesů a aktivní roli místní komunity v reakci na mezikulturní kontakty. Nově popsaným fenoménem je vznik tzv. smíšených společností, kdy nově vzniklá kultura v sobě nese prvky obou původních společností, která je neustále obnovována na základě vzájemných interakcí všech zúčastněných stran (White 1991; 2011). Nový pohled na mísení kultur tak nabízí zcela nové interpretace na interagující společnosti - obě kultury jsou nahlíženy jako sobě rovné, chybí prvek dominance jedné z nich a zcela zásadní je zde kreativní prvek, tedy vytváření nové kultury namísto přejímaní kultury dominantní společnosti.

Badatelé zabývající se mezikulturními vztahy hledali nové přístupy k popsání mnohdy velmi komplikovaných situací. Jedním z nejvlivnější přístupů je tzv. {\em middle-ground theory} amerického historika Richarda Whitea (1991; 2011). White, jakožto historik zabývající se kolonizací západní části amerického kontinentu a vzájemnými vztahy mezi Indiány a kolonizujícími bělochy, rozhodně netušil, jak dalekosáhlé důsledky a jak velké uplatnění jeho dílo bude mít i na poli středomořské archeologie. Základní myšlenkou je teorie o místě setkávání dvou kultur, které však nepatří ani do jedné z nich (tzv. {\em middle-ground}). White pojímal {\em middle-ground} jako reálné místo kontaktů, ale i přeneseně jako {\em ad hoc} vzniklý symbolický prostor s prvky z obou zúčastněných kultur, který byl ale plný vzájemných nedorozumění a nových významů (White 2011, xii).\footnote{V původním whiteovském použití {\em middle ground} vzniká jako reakce na obchodní kontakty mezi původními kmeny a příchozími francouzskými kolonizátory, které byly mnohdy plné násilí, vzájemného neporozumění, ale zároveň absence převahy jedné ze zúčastněných stran. Nicméně dokud se obě strany vzájemně potřebovaly, ať už z čistě obchodního hlediska, snažily se dosáhnout určité shody. Tím vznikl poměrně křehký stav, kde se obě strany snažily balancovat vzájemné vztahy a zároveň udržet života schopnou komunitu, např. zajištěním půdy, obživy atp.} {\em Middle-ground} je chápáno jako dočasný fenomén, který zaniká, pokud se jedna ze zúčastněných stran přestane podílet na společenské interakci, či získá dominanci nad druhou stranou (Bayman 2010, 132). White tedy formování {\em middle-ground} chápal jako neustálý proces vytváření symbolického systému porozumění, typický pro daný čas a dané místo, který nelze dost dobře aplikovat na jiné situace (White 2011, xiii). Ač je tento teoretický přístup možné aplikovat pouze na velmi malé množství situací, otevírá zcela nový prostor k interpretacím a poukazuje na aktivní roli obou zúčastněných stran a na specifika mezikulturní výměny.

„{\em Middle-ground theory}” se stala velice oblíbeným interpretačním rámcem i pro prostředí středomořské archeologie a historie (Malkin 1998; 2011; Woolf 2009; Antonaccio 2013). V kontextu Středomoří badatelé interpretují {\em middle-ground} jako novou kulturu, která sice pochází z obou původních kultur, a uchovává si dlouhodobý, avšak proměnlivý, charakter (Malkin 1998; 2011).\footnote{Malkin aplikuje {\em middle-ground} v kombinaci s tzv. {\em network theory}, tedy teorii o decentralizované řecké společnosti, kde k hlavním interakcím dochází v místech setkávání - přístavech a nadregionálních svatyních (2011, 45-48). Pro Malkina je každé takovéto místo setkávání (angl. {\em node}, {\em cluster}) zároveň místem, kde konstantně dochází k formování nové kultury, tedy {\em middle-ground}. Malkinův koncept má spíše popisný než interpretační charakter, avšak i tak výrazně ovlivnil současnou akademickou debatu. Důraz se tak začal klást nejen na propojenost Středozemního prostoru (Malkin {\em et al.} 2009; Constantakopoulou 2007; Archibald 2013, 96-97), ale i na jeho decentralizaci a roli lokálních komunit.} Irad Malkin interpretuje {\em middle-ground} jako nově vznikající kulturu v místech kontaktu, tj. například v koloniích a jejich bezprostředním okolí, či v emporiích, kde docházelo k setkávání velkého množství skupin z odlišných kultur. Malkin, a po jeho vzoru i další badatelé, se uchýlili k tomuto modelu, protože stírá vzájemné odlišnosti a protiklady typu Řek vs. barbar, ale zároveň nechává dostatek prostoru pro různé druhy interakcí a vysvětluje vznik nových smíšených kultur (Antonaccio 2013, 239).\footnote{Alternativní pohled představuje Greg Woolf (2009, 224), který chápe {\em middle-ground} jako svět prezentovaný antickými geografy a etnografy, v čele s Hérodotem. Je to svět, který není ani jedním ze dvou světů a často vzniká ze vzájemného neporozumění, vytržení z původního kontextu. Jedná se o zcela odlišnou interpretaci termínu, než jak ho navrhoval White, nicméně Woolfovo pojetí vrhá nový úhel pohledu na naše vnímání antických etnografických textů a jejich relevanci pro popis dané kultury.}

Použitelnost konceptu {\em middle-ground} je v kontextu řecké kolonizace omezená jen na určitý druh situací a nedá se obecně aplikovat na veškeré mezikulturní kontakty. Zásadní přínos {\em middle-ground theory} je odklon o dřívějších binárních modelů, které předpokládaly jednosměrnou výměnu s jasnými výsledky v přijímající kultuře. Poukázání na fakt, že reakce místního obyvatelstva nebyla vždy jasně daná, a na jeho aktivní účast, zcela pozměnila paradigma současného přístupu k mezikulturním kontaktům.

\subsection[kreativní-síla-mezikulturních-kontaktů]{Kreativní síla mezikulturních kontaktů}

Dalším společným bodem postkoloniálních teoretických přístupů je zaměření na kreativní sílu mezikulturních kontaktů a vzájemné ovlivňování všech zúčastněných stran. Interakce mezi jednotlivými společnostmi jsou pojímány v kontextu dané situace a je dán větší prostor variabilitě motivů, které vedly ke setkávání kultur. Prolínání kultur je vnímáno jako dynamický proces, který má celou řadu projevů v obou zúčastněných společnostech. V průběhu posledních 20 let vzniklo několik konceptů, využívajících poznatky z přírodních věd, jejichž aplikace se stala velmi oblíbenou, až téměř módní záležitostí (VanValkenburgh 2013, 301-302). Hybridizace\footnote{Jednotný přístup k {\em hybridizaci}, tedy vytváření hybridní materiální kultury a identit, neexistuje (Card 2013, 1-2). V rámci současného diskurzu je termín {\em hybridizace} používán pro popis důsledků setkávání dvou či více kultur a je vnímán jako neustálý proces vytváření nových významů v rámci se setkávajících kultur, často probíhající na úrovni jednotlivce (Bhabha 1994, 33-39). Termín se stal velmi oblíbeným mezi archeology, protože poskytuje prostředek analýzy materiální kultury, která je mnohdy směsí mnoha kulturních vlivů, ve formě prolínání stylistických a technologických trendů. Hlavní body kritiky {\em hybridizace} spočívají v silných biologických konotacích, které termín vyvolává, dále pak nadužívání termínu bez patřičného teoretického pozadí (Silliman 2013, 493), či jeho obsahová vyprázdněnost (Card 2013, 1-2).}, kreolizace\footnote{{\em Kreolizace} má svůj původ v lingvistice, kde popisuje proces kontaktu dvou sociolingvistických skupin. Ze specifických interakcí dvou skupin mluvícími rozdílnými jazyky vznikají zcela nová mínění a forma jazyka, zahrnující nejen slovní zásobu, ale i novou strukturu (Stewart 2007, 2; Jourdan 2015, 117). Typickým produktem kreolizace je nový jazyk, který čerpá prvky z různých jazyků ({\em creole} a {\em pidgin}), avšak udržuje si zcela unikátní charakter (Jourdan 2015, 118). V rámci sociologie a antropologie je pak {\em kreolizace} vnímána jako proces vytváření nových významů kdy jedna strana má dominantní pozici, například v rámci nuceného přesídlení obyvatelstva či v diaspoře (Liebmann 2013, 40). V mnoha ohledech je {\em kreolizace} velmi podobná {\em hybridizaci}, zejména s ohledem na vytváření nových významů a forem kultury. Hlavním rozdílem je zaměření {\em hybridizace} na výsledek kulturní výměny a na projevy v dané kultuře, zatímco {\em kreolizace} klade větší důraz na samotný proces změny (Jourdan 2015, 119).}, synkretismus\footnote{{\em Synkretismus} se jako termín poprvé objevil již v antice, kde označoval seskupení krétských komunit, které se často dostávaly do vzájemného konfliktu, ale v době nebezpečí se dokázali sjednotit. (Plut. {\em Mor.} 478a-490b.) V moderním pojetí je pojem používaný pro sloučení prvků z několika náboženství v rámci jednoho náboženského systému. V rámci studia mezikulturních kontaktů se {\em synkretismus} taktéž zaměřuje zejména na změny v rámci náboženství (Drooger 2015, 881-882). V rámci moderní antropologie získal termín pejorativní význam, kdy popisoval změnu jako nechtěný důsledek kulturních interakcí, a tak došlo k jeho postupnému vymizení z odborné literatury (Liebmann 2013, 28).}, bricolage\footnote{Termín {\em bricolage} se poprvé objevil v díle Levi-Strausse, jako fenomén popisující kreativní přeměnu kulturních prvků způsobenou aktivitou jednotlivců v rámci jedné kultury (Liebmann 2013, 29). Jean Comaroff (1985) rozšířila použití {\em bricolage} i na mezikulturní vztahy a koloniální kontext. {\em Bricolage} se soustředí spíše na vztah společenských struktur a jejich vliv na vytváření nové kultury, namísto archeology oblíbené hybné síly kulturní změny (Liebmann 2013, 29-30).} a mísení kultur představují jen výčet pojmů, používaných k popsání určitého druhu kontaktu dvou kultur, kdy dochází k adopci a adaptaci určitých prvků, často za neustálého vytváření a re-formulování kultury nové. Každý z těchto zmíněných termínů, a s nimi souvisejících směrů, se zaměřuje na jiný aspekt mísení kultur, avšak mnohé mají společného: důraz na aktivní zapojení zúčastněných společností, připisování kreativní role interagujícím skupinám a zdůrazňování podílu všech zúčastněných stran při vytváření nové kultury a identity.

Zásadním bodem kritiky těchto „kreativních směrů” je nemožnost určit původ konkrétních kultur, z čehož pramení neschopnost analyzovat jejich vzájemné ovlivňování (Jourdan 2015, 119). Dle Jourdana je kultura sama o sobě neustále se proměňující systém znaků a symbolů, který reaguje na vnější podněty a vyvíjí se, a tudíž je velmi obtížné zpětně vysledovat vzájemné ovlivňování kultur.

Hlavním zastáncem kreativní role lokální komunity v rámci Středozemí je Peter Van Dommelen, který na příkladu kolonizace Sardinie dokazuje, že dlouhodobým působením a vzájemnými kontakty několika kultur docházelo k vytváření místních hybridních materiálních kultur (1998; 2005; Dommelen a Knapp 2010). Materiální kultura dle Dommelena projevuje velkou míru mísení prvků, které jsou používány v novém kontextu a jsou vědomou volbou členů místní komunity. Van Dommelen (2005, 134-138) se primárně dívá na proměny různých součástí materiální kultury, jako je například architektura, keramická produkce, ikonografie soch, související použité technologie, a na materiální projevy rituálů, jako jsou např. dedikační předměty, či svatyně. Velký prostor dává Van Dommelen i procesu formování lokálních identit a aktivnímu odporu místního obyvatelstva v rámci selektivního přijímání konceptů a strukturálních prvků jiných kultur (Van Dommelen 1998, 214-216).\footnote{Van Dommelen 2005, 117: „{\em Cultural hybridity is a concept that has been propagated particularly by Bhabha as a means to capture the “in-betweenness” of people and their actions in colonial situations and to signal that it is often a mixture of differences and similarities that relates many people to both colonial and indigenous backgrounds without equating them entirely with either} (Bhabha 1985)”.} V neposlední řadě se zaměřuje nejen na reakce původního obyvatelstva, ale i nově kriticky hodnotí role příchozí komunity v rámci nového přístupu a bez předsudků koloniálních modelů (Van Dommelen 2005, 118).

\subsection[kontextualizace-mezikulturních-kontaktů]{Kontextualizace mezikulturních kontaktů}

Směrem, který má velký ohlas v současných pracích zabývajícími se mezikulturními kontakty, je přístup navrhovaný Michaelem Dietlerem (1997; 1998; 2005). Zatímco Dietler odmítá monolitické modely jako hellénizace či teorie světových systémů pro svou přílišnou obecnost, Dietler se namísto velkých dualistních modelů zaměřuje na formování lokální identity a různé reakce obyvatelstva na kontaktní situace. Dietlerovou inovací je zdůraznění nutnosti sledovat daný fenomén v původním kontextu, s přihlédnutím k roli, jakou daný prvek měl v původní kultuře, a jaké místo zaujímá v kultuře nově formované. Dietler tak navazuje na antropologický směr přisuzující materiální kultuře aktivní roli při formování společenských struktur, jinak též znám jako tzv. {\em social life of things} (Appadurai 1986).

Dietlera zajímají zejména důsledky prvotních kontaktů a jejich projevy na materiální kulturu (1998, 218-219).\footnote{Dietler 1998, 218: „{\em Examination of the initial phase of the colonial encounter in Iron Age France is important precisely because it holds the promise of revealing the specific historical processes that resulted in the entanglement of indigenous and colonial societies and how the early experience of interaction established the cultural and social conditions from which other, often unanticipated, kinds of colonial relationship developed.}” Michael Dietler na příkladu archeologického materiálu rané doby železné z oblasti jižní Galie a Germánie podél toku řeky Rhôny sleduje projevy mezikulturní výměny na materiální kulturu. Na tomto území docházelo k setkávání velkého množství skupin, zahrnující např. Etrusky, Féničany, Řeky, Kelty a v neposlední řadě Římany.} Dietlerova interpretace materiálu může být charakterizována jako přístup „zdola”, tedy přistupující přímo k interpretaci materiálních pramenů, oproti obecně pojatým modelům, které k materiálu přistupují „shora”, tedy čtením dochovaných literárních pramenů. Hlavním zkoumaným materiálem je pro Deitlera keramika, zejména keramika používaná ke konzumaci alkoholu v aristokratických kontextech.\footnote{Téma konzumace se ostatně prolíná i Dietlerovou metodologií: Dietler sám používá termín {\em consumption} pro popis kontextu v jakém materiální kultura splňuje své poslání a je využívána ke svému účelu čili jakou roli zaujímá v rámci zkoumané společnosti, případně jak se vyvíjí s měnícími se společensko-politickými poměry (1998, 219-221).} Volba konkrétního stylu, materiálu, symbolického systému, a tedy i jejich případná změna, je pojímána jako vědomý akt, volba na úrovni jednotlivce, či místní komunity, které však může být motivována celospolečenskými jevy. Změna je podmíněna různými motivy, jejichž objasnění není vždy snadné, či dokonce možné. Příkladem může být nádoba považovaná v jedné kultuře za téměř bezcennou, zatímco v kultuře druhé za exotický předmět, jemuž je přisuzována zvláštní hodnota (Appadurai 1986, 6-16). Role, jakou předmět v kultuře hraje, je do velké míry jeho společenskými a symbolickými konotacemi. Aby mohl být předmět vnímán jako hodnotný, a tedy i vhodný ke „konzumaci” v rámci dané komunity, musí splňovat požadavky dané konkrétním společensko-politickým uspořádáním a systémem hodnot. Tento systém hodnot bývá často interpretován jako vkus, či materializovaný projev životního stylu ({\em habitus} v bourdieovské teorii; Bourdieu 1977).

{\em Habitus}, tak jak ho definuje Bourdieu (1977), je souhrnem všech dispozic a předpokladů, které utvářejí životní styl a světonázor každého daného jedince. {\em Habitus} se nevědomě formuje v závislosti na společenských konvencích a struktuře společnosti, které člověka obklopuje, či obklopovala v minulosti. Nevyhnutelným projevem {\em habitu} každého člověka je materiální svět, který si sám vědomě okolo sebe vytváří v závislosti na vkusu a osobních preferencích (Bourdieu 1984, 173-5). Lidé pocházející z podobného prostředí, mají zpravidla podobné zvyklosti a podobá se i materiální kultura, kterou se obklopují. Jakmile dojde ke změně jedné či více z okolností (kontextů), které formují {\em habitus}, musí časem dojít i k proměně materiální stopy, kterou po sobě člověk zanechává (Sapiro 2015, 487). Může se jednat o změnu ekonomických podmínek, změnu společenského statutu, či se jedná o reakci na kontakt s cizí materiální kulturou, či zavedení nové technologie apod. Jinými slovy, plošné změny ve společenském uspořádání by se měly taktéž projevit na materiální kultuře a její produkci.

Kontextualizace, jak ji navrhuje Dietler, představuje poměrně komplexní analytický model, který vyžaduje systematické zhodnocení jak lokálních kontextů, definice role importovaných předmětů, tak i kvantitativní zhodnocení materiální kultury a její a časoprostorové rozmístění (Dietler 1998, 221). Stejný přístup může být velice dobře použit i na studium nápisů a z nich plynoucích společenských změn. Nápisy, v daleko větší míře než keramické nádoby, poskytují informace o struktuře společnosti, místních identitách a proměnách vkusu dané komunity.

\section[alternativní-přístup-k-mezikulturnímu-kontaktu]{Alternativní přístup k mezikulturnímu kontaktu}

Archeologové, stejně jako historikové a epigrafici, do velké míry čelí omezenosti pramenů samotných: za prvé, nelze archeologicky prozkoumat naprosto vše, a proto se téměř vždy se jedná o generalizace založené na částečných znalostech minulé skutečnosti. Dalším problémem je náhodné dochování archeologických či epigrafických pramenů, s nímž je nutné se vyrovnat. Jinými slovy obraz minulosti, který archeologie skládá dohromady, nemusí být reprezentativní, ani přesný. Představené metody jsou snahou se s těmito nedostatky vyrovnat a zároveň reflektují i soudobé teoretické trendy a vývoj moderní společnosti, které v mnohém ovlivnily konečnou podobu diskutovaných teoretických směrů. Proto ani sebelepší teoretický přístup nemůže postihnout minulost v její úplnosti, ale může alespoň přiblížit jeden z aspektů lidských dějin.

Současné teoretické přístupy nabízejí jednak analytický přístup hodnotící kontaktní situace bez kulturně podmíněných předsudků, na rozdíl od přístupů založených na tradičně pojímaných akulturačních modelech. Velký přínos pro studium společnosti a dynamiky jejího fungování má tzv. postkoloniální teoretický rámec, který se štěpí do několika směrů zaměřujících se na konkrétní aspekt kulturních interakcí a změn.

Možným východiskem plurality přístupů k mezikulturnímu kontaktu a nejednoznačnosti materiální kultury je kombinace několika metod, které v sobě kombinují kritické zhodnocení rolí obou zúčastněných společností v daných kontextech, specifických pro jednu či druhou kulturu, spolu se systematickým studiem dochovaných materiálních pramenů (Dietler 1998; 2010). Přítomnost řeckých importů v thráckém vnitrozemí již není automaticky vysvětlována jako jasný znak hellénizace thráckého obyvatelstva a kulturní převahy řeckých komunit, ale badatelé kriticky přihlíží k společensko-historickým kontextům, které by vysvětlili přítomnost materiální kultury tradičně klasifikované jako „řecké” v neřeckém prostředí (Nankov 2012). Upouští se od přiřazovaní jednotlivých materiálních kultur a pohřebních rituálů jako určujících znaků etnické či jinak „kulturní” příslušnosti (Damyanov 2012), a autoři začínají plně uznávat komplexnost mezikulturních a mezikomunitních kontaktů.

V posledních dvou desetiletích se někteří badatelé odklonili od tradičních koloniálních přístupů, zdůrazňujících polaritu kulturních kontaktů, směrem k pluralitě možných interakcí obyvatel Thrákie různého kulturního původu. Zcela v duchu postkoloniálních přístupů badatelé přistupují k analýze materiálu bez kulturních předsudků (Dietler 2005). V reakci na postkoloniální přístupy přestávají literární prameny sloužit jako jediný interpretační rámec, podle nějž se archeologické prameny musí nutně řídit, ale jsou konzultovány spíše jako ilustrativní, sekundární zdroj informací, který navíc může v mnohém s archeologickými prameny nesouhlasit (Ilieva 2011, 25-43). Společenské uspořádání antické thrácké společnosti a případné změny v interní struktuře společnosti v reakci na kontakty s vnějším světem, ať už řeckým, makedonským, či později římským, se stávají centrálním tématem některých archeologických projektů, ač zatím pouze v regionálním měřítku (Sobotková 2013; Archibald 2015, 393-395). Na tyto a podobné projekty chci navázat v současné práci analýzou epigrafických pramenů, které, ač jsou produktem stejné společnosti, která produkovala archeologicky zkoumaný materiál, bývají mnohdy studovány odděleně, což vede ke ztrátě znatelné části jejich výpovědní hodnoty. Proto kulturně nezatížené zasazení dochovaných epigrafických památek do širšího archeologického a historického může být klíčem k pochopení dynamiky společnosti antické Thrákie.

\chapter{Nápisy a jejich hodnota pro studium antické společnosti}
V této kapitole se zabývám výpovědní hodnotou a relevancí informací získaných z epigrafických památek a jejich využití ke studiu antické společnosti. Ton Derks velice příhodně poznamenal, že nápisy jsou často přehlížený druh historického pramene, jehož unikátní výpovědní hodnota bývá často využívána jen jako doplněk historického narativu, či datační materiál, což vede k opomíjení jejich plného potenciálu (Derks 2009, 240). Při studiu nápisů vycházím z přesvědčení, že epigrafické památky jako produkt společnosti a jejich členů nesou nenahraditelné informace jak o jedincích, kteří se podíleli na vydávání nápisů, ale i o jejich proměňujícím se vkusu, komunikačních strategií a postoji k jiným kulturám. Jsem přesvědčena, že studiem nápisů je možné získat jedinečné informace, jak o soukromém životě jednotlivců, tak i o kulturně-společenském uspořádání a jeho proměnách v závislosti na čase a místě, za předpokladu, že jsou respektována specifika epigrafického materiálu.

\section[specifika-epigrafických-pramenů]{Specifika epigrafických pramenů}

Nápisy jakožto historický pramen se nacházejí na pomezí psaného pramenu a zároveň archeologického monumentu, což sebou nese mnoho specifických problémů a netradičních interpretačních přístupů, o nichž promluvím podrobněji v následujících sekcích.

\subsection[nápis-vs.-literární-pramen]{Nápis vs. literární pramen}

Nápisy stojí na pomezí literárního a archeologického pramenu a jako takové si uchovávají vlastnosti typické pro obě dvě skupiny. Nápisy a archeologické prameny jsou často chápány jako přímý pramen pocházejícím přímo od členů společnosti, která je vytvořila, zatímco literární prameny mohou pocházet od vnějších pozorovatelů (Hansen 2001, 331-343).\footnote{Přímý pramen má de Hansena stejnou formu a nese stejné informace jako v době svého vzniku, čímž se liší od nepřímých zdrojů, které představují literární prameny. Nepřímé prameny se dochovaly ve formě manuskriptů a byly v průběhu staletí již mnohokrát pozměňovány a upravovány. Hansenovo dělení na základě přímosti a nepřímosti dochování sdělené informace není dostačující. Sám uznává, že nepostihuje např. rozdíl mezi literárními a dokumentárními papyry, ale nenabízí přesvědčivou alternativu.}

Analogicky se v oblasti antropologie a lingvistiky setkáváme se dvěma pojmy, rozlišujícími zdroje na základě jejich původu a postoje autora (Pike 1954, 8-28; Cohen 2000, 5). První z nich, zdroj emický popisuje subjekt z pozice vnitřního pozorovatele, v případě komunity jejího člena, který je obeznámen s kulturou, situací a disponuje nenahraditelnými informacemi o fungování společnosti. Oproti tomu etický zdroj je mnohdy nezúčastněný literární tvůrce popisující subjekt z pozice vnějšího pozorovatele. Výhodou etického zdroje je větší nadhled a možnost srovnání s podobnými subjekty. Na druhou stranu je to přílišný odstup a neznalost prostředí, které mohou vézt k nepochopení situace a v konečném důsledku i ke značně zkreslenému svědectví. Literární prameny, zejména v případě Thrákie, naopak nabízejí pohled na komunitu z venku, často z velmi odlišné perspektivy, která mnohdy vypovídá více o kultuře autora, než o kultuře popisovaného subjektu. Nicméně i přesto literární prameny představují další cenný druh informací o starověkém světě a postihují historické události, které samotné nápisy většinou nezaznamenávají (Bodel 2001, 41-45). Nápisy jakožto emický zdroj oproti tomu obsahují cenné a detailní informace o struktuře tehdejší společnosti, jejích proměnách a vzájemném prolínání kultur, které zaznamenali přímo členové dané komunity a je možné na ně nahlížet jako na emický zdroj infromací (McLean 2002, 2). Z toho plyne, že nápisy, jakožto emický zdroj informací, zprostředkovávají přímý pohled do společnosti, která je vytvořila. Představují tak nenahraditelný zdroj informací o vnitřním uspořádání komunit, identitě jedinců, povědomí o kolektivní sounáležitosti a o dynamice společenských vztahů, s nimiž pracuji zejména v kapitole 6 a 7.

\subsection[statický-a-selektivní-obraz-epigrafické-společnosti]{Statický a selektivní obraz epigrafické společnosti}

Ač epigrafické texty představují primární zdroj informací, jedná se o zdroj značně selektivní a statický. Nápisy jako produktem společnosti daného časového a místního kontextu, který se stává minulostí již v okamžiku publikování, a neslouží jako popis antické společnosti platný pro celou dobu jejího trvání.

Texty nápisů se přímo nevyjadřují ke všem oblastem lidského života, ale drží se zavedených kategorií, jako jsou například funerální či dedikační texty, které mohou být do jisté míry stylizované a navzájem se prolínat (Bodel 2001, 46-47). Některé oblasti společenského života, např. styky s jinými kulturami, se na nápisech objevují jen výjimečně ve formě přímé zmínky.\footnote{To však neznamená, že by ke kontaktům nedocházelo, pouze se neprojevily na epigrafické produkci, či alespoň ne v podobě přímé zmínky v textu či reflexe dané skutečnosti.} Při studiu nápisů je tedy nutné respektovat charakteristické rysy jednotlivých kategorií, které se ale mnohdy navzájem prolínají, ale především se neomezovat na strohou interpretaci založenou na pouhém rozboru textu. Nápisy představují komplexní souhrn kulturních prvků, které jsou všechny projevem téže kultury. Proto je vhodné nahlížet na objekt v jeho širším kontextu, aby se předešlo ztrátě podstatných informací či jejich zkreslení.

Se ztrátou informací souvisí i míra dochování nápisů, která byla ve velké části případů dílem náhody. Dochované nápisy nápisů tak zdaleka nepředstavují všechny v minulosti vytvořené nápisy, dokonce ani jejich reprezentativní vzorek, ale jedná se o náhodný soubor, který svému dochování vděčí spíše kvalitě použitého materiálu než obsahu textu. Richard Duncan-Jones (1974, 360-362) odhaduje na souboru římských nápisů ze severní Afriky, že dnes dochované texty představují zhruba 5 \letterpercent{} původně existujících nápisů.\footnote{Pokud toto překvapivě nízké číslo aplikujeme na situaci v Thrákii s cca 4600 dochovanými nápisy, dostaneme se teoreticky až na 92000 původně existujících nápisů, což se však zdá až příliš nadhodnocené, vzhledem k povaze a množství současných nálezů. I tak zde analyzovaný soubor 4600 nápisů nepředstavuje všechny vytvořené nápisy, ale pouze jejich dochovaný zlomek.}

Předpokládá se, že populace aktivně zapojená do produkce nápisů, tj. epigraficky aktivní část společnosti představovala pouze malý výsek tehdejší společnosti. Vydávat nápisy na kameni a kovu si mohli dovolit lidé střední a vyšší třídy, vzhledem k relativně vysoké ceně nápisů, zatímco materiály jako keramika či dřevo byly dostupnější, nicméně se hůře dochovávají (McLean 2002, 13-14). Z důvodů finanční náročnosti se příliš často nesetkáváme na nápisech s povoláními vykonávanými lidmi z nižší společenské a ekonomické vrstvy společnosti. Někteří autoři se snaží vypočítat na základě odhadů o velikosti populace a počtu dochovaných nápisů, jaký byl podíl epigraficky aktivních lidí v dané oblasti (Woolf 1998, 99-100; Beshevliev 1970, 64-65). Jednotlivé poměry se liší jak v závislosti na místě a čase a na stupni dochování nápisů, nicméně je možné souhrnně říci, že epigraficky aktivní obyvatelstvo tvořilo menšinu tehdejší populace a obsah a forma dochovaných nápisů poukazuje na zapojení spíše na střední až vyšší vrstvy společnosti.

Z výše řečeného plyne, že dochované nápisy představují pouze zlomek z dřívější produkce, pocházejí pravděpodobně pouze ze středních a vyšších vrstev společnosti a jsou záznamem stavu v určitém časovém a místním kontextu. I přes tato omezení nesou celou řadu informací o proměnách tehdejší společnosti, jimiž se zabývám zejména v kapitole 6 a 7.

\section[důvody-pro-publikaci-nápisů]{Důvody pro publikaci nápisů}

Zvyk zhotovovat nápisy do kamene, též tzv. {\em epigraphic habit}, nemá jednu univerzální příčinu, ale je kombinací mnoha faktorů, které měly za následek, že některé komunity začaly nápisy produkovat a jiné nikoliv (MacMullen 1982, 233; Bodel 2001, 12-13).

V prvé řadě je nápis materializovanou zprávou určenou dalším členům společnosti, v níž nápis vznikl. Psaní na nosič trvalého charakteru, jako je kámen, kov, keramika atp., je chápáno jako předem promyšlený akt, kterým člověk sleduje určitý záměr a předává konkrétní zprávu. Tím, že člověk napíše určité sdělení na trvanlivý materiál a umožní tak, aby si ho kdokoliv v~současnosti, ale i v~budoucnosti mohl přečíst, přisuzuje obsahu sdělení velkou váhu. Takto prezentované informace poukazují na důležité hodnoty a význam jaký zpráva na nápise měla pro původce zprávy, zdůrazněné úsilím a mnohdy i nemalými finanční prostředky nutnými k vytvoření nápisu (McLean 2008, 13-14).

Motivace pro zhotovení nápisu jsou různé a nejde je přisuzovat pouze jediné příčině. Nápisy bývají publikovány v důležitých životních situacích, tedy i v době kontaktu s novou kulturou a reflexe tohoto setkání, dále pak v době měnících se podmínek. V případě soukromých nápisů, tj. publikovaných jménem jednotlivce či skupiny, to může jednak být reakce na nemoc či smrt blízké osoby, reakce na nelehkou životní situaci a obrácení se o pomoc k božstvu, signalizace vlastnictví předmětu v případě, že hrozí jeho krádež či pozbytí, či v neposlední řadě v době změny společenského postavení, jako např. při udělení občanství či získání svobody (Cooley 2012, 52-54). V případě veřejných nápisů, tj. publikovaných politickou autoritou, se může jednat o reakci na hrozící či proběhlou změnu společenského uspořádání či o veřejnou signalizaci moci a autority v rámci komunity vedoucí k upevnění mocenské pozice. V rámci komunikační strategie vůči cílové skupině čtenářů mohl zhotovitel zdůraznit určité charakteristiky na úkor jiných či se záměrně stylizovat do určité zamýšlené role.\footnote{Například někteří badatelé se pokouší vysvětlit zvýšenou epigrafickou produkce na konci 2. a na začátku 3. st. n. l. jako motivovanou finančními zájmy příbuzenstva, spojených s dědictvím a zároveň jako snahu o navýšení společenské prestiže pomocí poukázání na dosažené postavení (Meyer 1990, 78, 95; Cooley 2012, 54).}

Důvody a motivace pro publikování nápisů byly na první pohled velmi odlišného charakteru, podmíněné společenskou a kulturní funkcí, kterou měl nápis plnit. Na počátku epigrafické produkce stály pravděpodobně zájmy jednotlivců, avšak za jejím rozšířením v míře, s jakou se setkáváme ve starověkém světě, zásadní roli hrál postupný vývoj společensko-politické organizace, též známý jako společenská komplexita nebo provázanost.

\section[epigrafická-produkce-a-komplexní-společnost]{Epigrafická produkce a komplexní společnost}

Produkce nápisů byla dříve považována za jedno z hlavních měřítek hellénizace, případně romanizace obyvatelstva (Woolf 1998, 78). V tomto pojetí bylo vydávání nápisů chápáno jako důsledek a přímý projev přijetí nové identity a s ní spojených kulturních zvyků. Perspektivou moderní sociologie, či archeologické teorie se spíše zdá, že k rozšíření zvyku publikovat nápisy došlo v důsledku zvýšených potřeb souvisejících s nárůstem společenské a kulturní komplexity.

Produkce nápisů ve větším měřítku, tedy větší než několik jednotlivých kusů, je proces plně provázaný s velkým počtem institucí, řemesel a dalších specializovaných povolání. K pravidelnému vytváření nápisů bylo nutné dosáhnout nejen určitého stupně kulturního vývoje, který umožnil ocenit výhody psaného slova, ale zejména vybudovat propracovanou infrastrukturu, nutnou k produkci epigrafických monumentů (McLean 2008, 4-14).\footnote{Tím se myslí nejen existence patřičných technologií, dostupnost nutného materiálu, ale zejména přítomnost lidí se specializovanými a krajně odbornými znalostmi, jako je znalost těžby a opracování kamene, doprava materiálu, znalost technologie tesání kamene, rytí, malby, výroby reliéfu atp. K vytvoření textu je dále nutná alespoň minimální gramotnost nutná objednavatele a tvůrce, a nakonec technologie nutná vytvoření samotného nápisu.} Tyto aktivity jsou poměrně finančně náročné a z hlediska strategie dlouhodobé udržitelnosti společnosti nijak nepřispívají k obživě obyvatelstva. Tvoří tedy jakousi kulturní a symbolickou nadstavbu, která není primární podmínkou existence komunit. Proto se s nápisy ve větším měřítku setkáváme pouze ve společnostech, které pro ně měly jasné využití a byly schopné ocenit výhody spojené s jejich produkcí, jako je například uchovávání informací či jejich předávání (Flannery 1972, 411).

Zajištění epigrafické produkce ve větším měřítku podmiňuje velká míra organizace pracovní síly a existence odborné specializace. To jsou typické znaky komplexní společnosti na úrovni raného centralizovaného státu tak, jak jí představuje Joseph Tainter v knize {\em The Collapse of Complex Societies} (1988). V Tainterově pojetí se lidé přirozeně sdružují do komunit, které jsou různě velké a mají různý stupeň provázanosti, tzv. komplexity. S narůstající velikostí komunity dochází i k nezbytnému nárůstu provázanosti: aby se společnost udržela v chodu dochází k postupné specializaci rolí. S tím je spojená i větší stratifikace společnosti, důsledkem čehož se centralizuje a odděluje vládnoucí vrstva od produkčních vrstev. Větší komunita má taktéž větší spotřební nároky a dochází tak nutně k zintenzivnění produkce, vytvoření specializovaných povolání, které se zaměřují pouze na jednu činnost za účelem dlouhodobého udržení společnosti. Všichni členové společnosti se navzájem doplňují, dochází k centrálně organizovaném přerozdělování materiálu tak, aby se společnost mohla dále rozvíjet, a aby jedinec byl za svou specializovanou činnost odměněn, a tudíž i nadále přispíval svou aktivitou k chodu společnosti (Tainter 1988, 22-36).

Tainter (1988, 24-31) definuje několik základních typů společností na základě jejich komplexity, které mohou mít různé varianty a stupně. Uskupení na úrovni kmenové společnosti bývají zpravidla méně početná a složení společnosti je relativně homogenní: autoritu představuje vládnoucí jedinec a skupina jeho následovníků, ale dosah jejich moci je omezený.\footnote{Tabulka 3.01 v Apendixu 1 poskytuje přehledné srovnání dvou základních stupňů vývoje společnosti na základě jejich provázanosti, jak je popisuje Tainter (1988).} Moc vládnoucí vrstvy není podmíněna silou, ale spíše vytvářenými společenskými pouty, které se pracně budují a udržují, ale mohou poměrně jednoduše zaniknout (Sahlins 1963, 295). Dále v kmenové společnosti existuje několik specializovaných povolání především ve zpracování a ve výrobě, ale převážná část obyvatelstva se zabývá zemědělskou produkcí.

Společnosti na úrovni raného státu oproti tomu vykazují větší míru společenské stratifikace a znatelný nárůst ekonomické specializace povolání: velkou roli v nich hraje narůstající byrokratický aparát, specialisté ve výrobě, ale i v jiných odvětvích, spojených s chodem společnosti (Johnson 1973, 3-4). Profesionalizovaný státní aparát má veškerou moc ve svých rukách a je schopen vyžadovat dodržování řádů a nařízení od svých členů, ať už pomocí zákonů nebo vojenské síly.\footnote{Tainter 1988, 29: „{\em The features that set states apart, abstracting from the previous discussion, are: territorial organization, differentiation by class and occupation rather than by kinship, monopoly of force, authority to mobilize resources and personnel, and legal jurisdiction.}”} Základním předpokladem dlouhodobého fungování státního uskupení bývá existence ohraničeného území, které má zásadní vliv na formování pocitu sounáležitosti se zbytkem komunity. Proto mnohdy státy vynakládají nemalé prostředky na vytyčení hranic a jejich udržení. Stejně tak vládnoucí vrstvy musí vynakládat nemalé prostředky k legitimizaci vlastního výsadního postavení a nezřídka tak vnikají centrálně organizované ideologie a náboženství. S narůstající provázaností společnosti dochází k jejímu demografickému růstu, který však nezbytně ústí v další nárůst komplexity. Tento cyklus však neomylně vede k dosažení maximálních kapacit daného politického uspořádání, které bývá zpravidla následováno kolapsem či přeměnou dané společnosti, jak dosvědčují mnohé případy vyspělých civilizací a říší (Tainter 1998, 4-18). Tehdy dochází k poměrně prudkému poklesu produkčních aktivit, decentralizaci moci a redistribuce, k úpadku byrokratického aparátu, snížení počtu specializovaných zaměstnání. V důsledku snížené role centrální autority může docházet k místním konfliktům a rozpadu na menší politické celky, které si dokáží zajistit bezpečnost a udržitelný rozvoj. Dochází též k vymizení vedlejších produktů komplexní společnosti, jako je monumentální architektura, umění a vzdělanost, ale například i nápisy (Tainter 1988, 4).

Z výše řečeného vyplývá, že pouze dobře organizovaná společnost s centrální mocí a fungujícími institucemi je schopna dlouhodobě zajistit produkci nápisů. Neznamená to nutně, že každý raný stát musel k udržení svého chodu vytvářet nápisy: v průběhu historie se setkáváme i s celou řadou anepigrafických státních zřízení, které dokázaly fungovat po dlouhou dobu i bez nápisů, jako např. říše Inků. Proč tedy mnohé komplexní společnosti produkci nápisů podporovaly, či ji dokonce organizovaly? Zde je třeba odlišit dvě skupiny nápisů, lišící se svou funkcí ve společnosti: jednak jsou to nápisy veřejné, související primárně s aktivitami státního řízení, kontroly, vedení byrokratického aparátu a legitimizace moci. Druhou skupinou jsou nápisy soukromého charakteru, které primárně nesouvisí s organizací státního zřízení a legitimizací politické moci, ale jsou tzv. vedlejším produktem komplexní společnosti.

První skupina nápisů je primárně podporována politickou autoritou, protože produkce veřejných nápisů splňuje své požadované poslání a přináší autoritě prospěch, jako např. dodržování norem, zvýšení autority státu, zvýšení finanční odpovědnosti občanů vůči státu, legitimizace moci a státní ideologie, předávání a uchovávání informací (Johnson 1973, 2-4). K pravidelnému a dlouhodobému vytváření takovýchto nápisů je potřeba existence celé řady specialistů a infrastruktur. Vytvoření a udržování této infrastruktury je však velmi nákladné a může si ji dovolit pouze stabilní společenské zřízení, které disponuje dostatkem prostředků. S objevením jednotné infrastruktury, organizované politickou autoritou, tak dochází i k jisté standardizaci procedur, a tedy i formy a obsahu nápisů, jakožto produktů či projevů těchto procedur (Johnson 1973, 3-4).

Vedlejší projevy existence této infrastruktury se po určité době projeví i soukromé sféře. Ač tyto soukromé aktivity nejsou přímo podporovány státem a nemají přímý vliv na chod státního aparátu, v komplexních společnostech dochází k poměrně rychlému nárůstu nápisů soukromé povahy. I v nich se nadále odráží společenská stratifikace a specializace povolání, která se může projevit např. popisy povolání či funkcí na funerálních a dedikačních nápisech (Tainter 1988, 111-126). Naopak s poklesem komplexity společnosti dochází i k poklesu epigrafické aktivity, a to zejména ve skupině veřejných nápisů, jejichž produkce již není nadále podporována centrální politickou autoritou. K poklesu dochází i v soukromé sféře, a to vzhledem k vymizení infrastruktury nutné k publikaci a celkovému poklesu vzdělanosti.\footnote{U soukromých nápisů však snadněji dojde k udržení jejich produkce, ač v omezené míře a v rámci menších komunit. V těchto případech je také možné sledovat přeměnu jejich funkce, jako reakci na měnící společenské uspořádání, podobně jak k tomu došlo například na konci antiky (Tainter 1988, 151-152).}

Uskupení fungující na principu rodové a kmenové pospolitosti jsou oproti společnostem na úrovni raného státu méně stabilní, a to jak na ekonomické, tak politické úrovni. Relativně nejistá pozice kmenového náčelníka a menší schopnost akumulovat majetek mají za následek, že nedochází k vytváření specializované vrstvy institucí, které mají na starosti publikaci nápisů. Pokud je primárním zájmem kmenového náčelníka takovouto infrastrukturu vytvořit, stane se tak za cenu nemalých nákladů, a její existence většinou nemá dlouhého trvání (Flannery 1972, 412). Nápisy většinou vznikají jako reakce na konkrétní události, a nedochází k jejich produkci ve větším měřítku. Nápisy zde často slouží v rámci upevnění autority a legitimizace výsadní pozice politické autority, a spíše než pro svou normativní a informativní funkci jsou nápisy oceňovány pro svou jedinečnost a symbolický význam (Earle 1989, 85). Vzhledem k nákladnosti pořízení nápisů a omezené existence vzdělané vrstvy v kmenové společnosti se zvyk vydávat nápisy nerozšiřuje do soukromé sféry nebo jen ve velmi omezené míře. Tomu odpovídá i míra produkce soukromých nápisů, která je velmi nízká a omezuje se na nápisy sloužící k prosazování ideologie vládnoucí vrstvy a jejího postavení. Tento druh specializované produkce prestižního zboží je typický právě pro kmenově uspořádané společnosti a liší se od výroby ve větším měřítku, kterou umožňují společnosti na vyšším stupni komplexity (Clark a Parry 1990, 319-323).

Míra provázanosti společenského uspořádání se projevuje i v tak na první pohled vzdálených a nečekaných součástech lidské historie, jako je epigrafická produkce. Zvyk publikovat nápisy se v antické společnosti stal jednak prostředkem udržování chodu komplexních společností, ale zároveň i vedlejším produktem těchto uskupení na úrovni raného státu. Instituce a ekonomický potenciál komplexní společnost stály za udržením tohoto zvyku po dlouhou dobu, v případě některých komunit i dlouhá staletí. Oproti tomu v případě společností fungujících na kmenovém principu zastávala epigrafická produkce pouze marginální roli a její praktické využití pro chod společnosti mělo krátké trvání. Z toho důvodu se například v Thrákii dochoval jen minimální počet nápisů, které je možné připsat thráckým kmenovým vůdcům.\footnote{Tj. pocházejících z území ovládaného thráckými kmeny a z doby, kdy kmenové uspořádání bylo jedinou formou politické organizace, jak je možné vidět v kapitole 6.} Naopak v řeckých městech na pobřeží máme doklady o dlouhotrvající státem podporované epigrafické aktivitě, svědčící o relativně stabilní politické a společenské organizaci.

\section[materializace-životního-stylu-na-nápisech]{Materializace životního stylu na nápisech}

Historikové mnohdy vnímají nápisy zejména jako prameny pro jejich obsah, v neposlední řadě se též jedná o materiální projev společnosti, vycházející ze společenských tradic a hodnot (Dietler a Herbich 1998, 244-248). Podobně i nápisy jsou produktem společensko-kulturních norem, a ve způsobu jejich provedení, v jejich obsahu, či dokonce v jejich rozmístění a společenské funkci se odrážejí existující vzorce chování dané společnosti.

Podobně jako archeologické památky, či veškerý materiální svět, který nás obklopuje je v klasickém bourdieuovském pojetí možné chápat projevy epigrafické kultury jako jednu z materializací lidského {\em habitu} v okamžiku stvoření ({\em sensu} Bourdieu 1977, 15).\footnote{Je důležité nezaměňovat {\em epigraphic habit} a {\em habitus}. {\em Habitus} je dle klasické definice Pierra Bourdieu výsledkem působení vnějších společenských norem, které jedince obklopují, a stejně tak získaných zkušeností. Tyto vnější okolnosti ovlivňují lidské chování, vkus a následně ovlivňují i materiální kulturu, kterou si každý člověk vytváří okolo sebe (Bourdieu 1984, 173-5). {\em Epigraphic habit} je materiálním a společenským projevem {\em habitu}, jinými slovy se jedná zvyk vydávat důležité zprávy na permanentním médiu a veřejně je vystavovat (MacMullen 1982; Meyer 1990; Bodel 2001, 12-13).} V textu a v materiálním provedení nápisu se odráží jednak životní postoj jednotlivce, ale také jeho vkus, a nepřímo i uspořádání okolní společnosti. Dle této teorie se člověk, který je členem epigraficky aktivní společnosti a sdílí stejné podmínky s ostatními producenty nápisů, jako např. finanční prostředky, přístup k materiálům a technologiím, podobné vzdělání, se s největší pravděpodobností také stane producentem nápisů. Dochází tak k přirozenému postupnému šíření tohoto zvyku jeho nápodobou, což vede k vytváření lokálních variant epigrafických monumentů, ovlivněných vkusem místních komunit. Změny v epigrafické kultuře a její produkci tak mohou reflektovat změny celospolečenského charakteru.

V rámci sledování změn epigrafické produkce sleduji proměňující se identifikaci a identitu jednotlivců, ale i celých skupin v závislosti na proměňujícím se politickém a kulturním kontextu. Analýzou měnících se trendů na poli formování identity se snažím zjistit, zda případná hellénizace, tedy šíření řecké kultury a stylu života, měla zásadní vliv na projevy a uspořádání společnosti, či se jednalo o kombinace faktorů nesouvisejících s řeckou kulturou.

\subsection[identita-a-identifikace-na-epigrafických-památkách]{Identita a identifikace na epigrafických památkách}

Způsob, jakým se jedinec rozhodne na nápisech vystupovat či jakým způsobem je reprezentován, vypovídá mnoho o vztahu, jaký zaujímá k svému nejbližšímu okolí a jak vnímá sám sebe v kontextu hierarchii dané společnosti. Díky tomu, že jsou nápisy „přímým pramenem”, lze se domnívat, že identifikace na epigrafických památkách patří mezi nejautentičtější vyjádření identity jednotlivců a skupin, jaké se nám z~antiky dochovala.

V pojetí identity a sebe-identifikace vycházím z díla sociologa Richarda Jenkinse v nichž je identita chápána jako základní prostředek lidí zasadit vlastní existenci do širšího rámce společnosti a je výsledkem přirozené lidské potřeby orientovat se v mezilidských vztazích (2008, 16-18). V Jenkinsově pojetí identita a reprezentace jednotlivce patří mezi nejzákladnější pojítko mezi člověkem a okolním světem. Kontakty s dalšími členy komunity lidé formují svou identitu. Identifikaci je pak možno chápat jako konkrétní projev identity, což je neustále probíhající obousměrný proces mezi jedincem a komunitou, který v sobě zahrnuje vědomou či nevědomou reflexi vzájemné pozice (Jenkins 2008, 13, 36-). Jinými slovy vědomá identifikace respektuje jedinečnost každého člověka, a zároveň ho napomáhá utvářet celospolečenské vztahy. Identita každého člověka je proměnlivá a vyvíjí se v závislosti na životní situaci. Tato interakce mezi jednotlivcem a komunitou je nikdy nekončící a stále se měnící proces, spolu s~tím, jak je jedinec konfrontován se změnami v~životním stylu a proměnami společnosti okolo něj (Jenkins 2008, 17, 36-48). Tím, že jedinec v rámci komunikace s dalšími lidmi zdůrazní určitou součást identity, dělá to proto, že je to pro něj v danou chvíli určitým způsobem výhodné, či se to ztotožňuje s jeho aktuálním světonázorem. Tím, že se jedinec situuje do určité komunity podobně se identifikujících lidí, si může zajistit bezpečí, ekonomickou soběstačnost, prestiž, nebo jen legitimizuje svou pozici ve společnosti.

Identita každého člověka se skládá z několika částí, z nichž každá složka může hrát důležitou roli v jiné životní situaci a více tak vystupuje na povrch. Každý člověk má potřebu se zároveň vůči společnosti vymezit, definovat svou vlastní identitu, a tím se zároveň zařadit do existujících komunitních struktur.\footnote{Ve své přelomové práci Fredrik Barth (1969) popisuje dynamiku fungování etnických skupin, a to jak vnitřní uspořádání, tak především interkomunitní společenské vztahy na principech stejnosti a odlišnosti. Barth tvrdí, že tyto principy fungují nejen mezi jednotlivými komunitami, ale utvářejí taktéž vnitřní dynamiku skupiny a v konečném důsledku napomáhají i identifikačním procesům jedince. Barthův model je možné použít nejen na popis fungování etnických skupin, ale jakéhokoliv kolektivu a formování identity obecně (Jenkins 2008, 119).} Jedinec se může vůči svému okolí vymezovat několika způsoby, jako například volbou použitého jazyka, dále jedinečným poznávacím znamením, jako je osobní jméno, nebo vymezením svého původu a vztahu k okolním komunitám za použití kolektivní identifikace.

\subsection[jazyková-identita]{Jazyková identita}

Nejniternější a základní součástí lidské identity je jazyk, jímž se vyjadřujeme. Volba jazyka, a s ním inherentně spojených kulturních zvyklostí vytváří most mezi interní identitou jednotlivce a strukturami společnosti okolo něj (Jenkins 2008, 143; Bourdieu 1991, 220-228). Jazyk a jeho znalost tvoří zcela jasně dané hranice komunity, která ale může být překračována v případě bilingvních i několikajazyčných mluvčí (Barth 1969, 22-25; Adams 2002, 8). Tito lidé jsou přecházením z jedné skupiny do druhé nuceni adaptovat svou identitu ve vztahu k společenským normám komunity daného jazyka. Přijetí nového jazyka a s ním často spojených společenských norem a zvyklostí, ať už v částečné či úplné podobě, představuje velký posun v rámci identity jedince. Proto se tak většinou nedělo bezdůvodně, ale v rámci adaptace na nové životní podmínky, snahy o zlepšení životní situace (Langslow 2002, 39-41) Volba epigrafického jazyka s sebou nesla jasnou zprávu o cílové skupině, které byl nápis určen, o identitě zhotovitele a často ze samotné volby jazyka vyplývá i povaha sdělení. Na příkladu Thrákie je možné vidět, že řecky psané texty určeny spíše místnímu obyvatelstvu, zatímco latinsky psané nápisy směřovaly úřednickému aparátu a státní samosprávě a členům armády středních a vyšších šarží, čemuž odpovídalo i zaměření nápisu.\footnote{Podrobněji se problematikou vztahem volby jazyka a identity v Thrákii zabývám v kapitole 5.} Jak podrobněji rozebírám v kapitole 5 v prostředí Thrákie je tedy volba řečtiny jako epigrafického jazyka spíše projevem komunikační strategie, než univerzálního vyjádření postoje mluvčích a znakem všeobecného přijetí řecké identity.

\subsection[osobní-jméno-a-identifikace-jednotlivce]{Osobní jméno a identifikace jednotlivce}

Osobní jméno je základním způsobem identifikace jednotlivce, pomáhající zasadit člověka do širšího kontextu dané komunity. Jméno může v širším kontextu sloužit jako ukazatel míry společenských tradic a konvencí a původu.\footnote{Osobní jméno se obvykle sestávalo z vlastního jména, dále mohlo obsahovat přezdívku, zpravidla vycházející z charakteristické vlastnosti nositele, která sloužila pro lepší identifikace konkrétního člověka v rámci komunity (Morgan 2003, 208; García-Ramón 2007, 47-64).} Jména si totiž většinou lidé sami nevybírají, ale je jim přiděleno nedlouho po narození na základě společenských zvyků dané komunity. Pomocí podoby osobního jména jsme schopni rozlišit pohlaví, případně i původ a společenské zařazení dané osoby (Morpurgo-Davies 2000, 20). V průběhu života se nicméně může měnit jako reakce na důležité změny v životě nositele, jako např. přijmutí náboženství, udělení občanství, či sňatek (Horsley 1987).

V archaické a klasické době se jména většinou dědila v komunitách po dlouhé generace a nedocházelo k větším proměnám fondu jmen.\footnote{Nepsané pravidlo v řecké společnosti určovalo, že prvorozený syn obvykle dostával jméno po svém dědovi z otcovy strany, což by v praxi znamenalo střídavou linii jmen, např. Athénagorás -- Hipparchos -- Athénagorás -- Hipparchos atd. (Hornblower 2000, 135; McLean 2002, 76). Druhorozený syn naopak dostával jméno po dědovi z matčiny strany.} V této době jména naznačovala na zařazení jeho nositele do širšího etno-kulturního rámce. Současní badatelé o antické prosopografii mají k dispozici obsáhlé databáze osobních jmen, které pomáhají určit, v jakých dobách a na jakém území se daná jména vyskytovala. Je tak možné alespoň přibližně určit, zda osobní jména spadají do řeckého kulturního okruhu, a tedy byla převážně používána lidmi řeckého původu, či alespoň lidmi, kteří se za Řeky označovali (jinak se též setkáváme s pojmem lingvistický původ jména, Sartre 2007, 199-201).

Od hellénismu docházelo k většímu prolínání jmen různých lingvistických původů jakožto důsledek většího pohybu řeckých, makedonských a v neposlední řadě také římských vojsk. Geografický výskyt jmen napovídá, ve kterých oblastech se uplatňoval vliv jaké skupiny, například jako důsledek kolonizace či bohatých obchodních kontaktů, které se po čase projevily i na výběru s výskytu osobních jmen (Parissaki 2007, 267-282; Habicht 2000, 119-121; McLean 2002, 87-90).

Pro římskou dobu je charakteristická větší fluktuace obyvatelstva, a to zejména vlivem pohybů římské armády, což mělo vliv i na mísení onomastických zvyků. Osobní jména již v této době není možné zcela jasně přisuzovat jednomu etno-kulturnímu rámci, vzhledem k tomu, že docházelo k jejich častému prolínání. I přesto osobní jméno samotné, případně jeho forma, napoví mnoho o jedinci samotném, o jeho kulturním pozadí, pohlaví, ale i o historii jeho rodu a zejména o jeho identitě. Za římské doby totiž lidé často přijímali nová jména jako důsledek udělení římského občanství.\footnote{Systém tří jmen se skládal z {\em praenomen}, osobního jméno, {\em nomen} gentilicium, rodového jméno, a {\em cognomen}, specifického přídavné jméno, a případně jména otce (McLean 2002, 113-148). Tzv. systém tří jmen se rozšířil i mezi místní obyvatelstvo, které si ale většinou nechávalo svá původní jména a jen k nim přidávala jména současného císaře v případě udělení občanství či svobody, jako v případě Marka Ulpia Autolyka, jehož rodina získala občanství za císaře Trajána na nápise {\em I Aeg Thrace} 68 z Abdéry z 2. st. n. l.}

V dlouhodobém horizontu jména signalizují přístup komunit k tradičním hodnotám, jejich udržování a určitou míru konzervativismu, či naopak mohou reflektovat proměny společnosti související se zvýšeným kontaktem s jinými národnostmi a kulturami. Těmito dlouhodobými trendy se zabývám podrobně v kapitole 5 a 6 a zaměřuji se převážně na proměny tradičních onomastických zvyklostí v celé společnosti.

\subsection[původ-jako-prostředek-identifikace]{Původ jako prostředek identifikace}

Označení původu je po osobním jméně druhý nejčastější způsob identifikace jedince a slouží jako bezprostřední zasazení do nejbližší komunity, či naopak jeho vymezení se vůči ní. Původ je vlastní každému člověku, bez ohledu na jeho volbu, a provází ho od narození po celý život. Původ člověka do značné míry utváří jeho pohled na svět a ovlivňuje i projevy v rámci epigrafické produkce. Nejtypičtějším vyjádřením původu je odkaz na biologický či geografický původ dané osoby.

Nejjednodušší označení biologického původu je označení rodičů, případně prarodičů. K identifikaci osoby se v řeckém světě mohlo se však užívat i jméno jiného rodinného příslušníka, např. matky, manžela, bratra (Fraser 2000, 150). Ženy většinou uváděly jméno svého otce a poté jméno manžela, případně jen jednoho z nich (McLean 2002, 94).\footnote{Areté, dcera Helléna, manželka Agathénóra, syna Artemidóra, {\em IG Bulg} 1,2 143, nedatovaný nápis z Odéssu.} Pokud to bylo důležité, mohla se přidávat ještě výjimečně informace o předchozí generaci pro zdůraznění rodové linie.\footnote{Makou, dcera Amyntóra, syna Hierónyma, {\em IG Bulg} 12 127, 2. - 3. st. n. l., z Odéssu. Dalším, poněkud méně rozšířeným, způsobem uvádění biologického původu je vypisování genealogických informací o předcházejících generacích, či o tzv. mýtických předcích a zakladatelích obcí, čím člověk legitimizoval své postavení v komunitě (Hall 2002, 25-29; Malkin 2005, 64-66).} Specifikace biologického původu zároveň ale hraje roli při legitimizaci majetkoprávních nároků při případných dědických sporech, a proto se v epigrafických pramenech velmi často setkáváme s její veřejnou prezentací na nápisech v době římské.\footnote{Podrobně se biologickým původem na nápisech zabývám v kapitola 5.}

Geografický původ je taktéž nedílnou součástí určení identity jednotlivce, která se však projevuje pouze v určitých situacích. Vyjádření geografického původu se nemusí vztahovat jen k místu, odkud člověk pochází, kde se narodil, ale může naznačovat i místo, k němuž se člověk hlásí a přijal ho za své. Člověk tak může vyzdvihovat své rodné město, ale stejně tak město, ve kterém nyní žije a je pro něj důležité zdůraznit přináležitost ke svému současnému bydlišti.\footnote{Jedinec se může vztahovat ke svému rodnému místu, současnému či minulému bydlišti, místu, odkud pochází jeho předci, jako např. Métrodotos, syn Artémóna, původem z Mandry, {\em I Aeg Thrace} 164, nedatovaný nápis z lokality Mitriko v severním Řecku.} K udávání geografického původu většinou docházelo v kontextu kontaktu s jinou kulturou či společností, kde jedinec potřeboval zdůraznit svůj původ, jako např. cizinec žijící na území jiného státu.

Vyjádření původu na nápisech je jedním z důležitých měřítek o míře konzervativismu dané společnosti. V uzavřených společnostech se setkáváme s malou mírou udávání geografického původu, či lidí cizího původu je pouze minimum, ať už z obavy možných persekucí, či z velmi malé míry fluktuace obyvatelstva. Naopak u společností na vyšším stupni společenské komplexity se setkáváme s vyšším počtem vyjádření geografického původu, což poukazuje na větší pohyb obyvatelstva a mnohonárodnostní složení těchto uskupení.

\subsection[identifikace-s-komunitou]{Identifikace s komunitou}

V průběhu života se člověk nevyhne interakcím s mnohými skupinami lidí. Kolektivní identitu, tedy jakési povědomí o sounáležitosti s kolektivem lidí, lze charakterizovat jako vědomou reflexi vztahu jedince vůči dané komunitě (Jenkins 2008, 103, Cohen 1985, 118). Na rozdíl od jazykové příslušnosti, která je většinou daná jazykovým prostředím, do nějž se narodíme a v~němž žijeme, naše afiliace se zájmovými skupinami je věcí naší vlastní volby, a proto je daleko flexibilnější. Člověk může být členem více komunit najednou, může je opouštět, nebo naopak do nových vstupovat, podle toho k jaké skupině sám cítí přináležitost.

Ke změně kolektivní identity může docházet v případě, že podmínky původní skupiny již nevyhovují nové situaci a dotyčný se již zcela neztotožňuje se zájmy skupiny, či v případě, že nová skupina nabízí všeobecně výhodnější podmínky pro rozvoj jednotlivce (Barth 1969, 21-25). Výhody nejrůznějšího druhu, plynoucí z účasti v těchto spolcích, jsou tedy hlavními důvody, proč se lidé sdružovali a stále sdružují do komunit a proč si přisuzují určitou kolektivní identitu (Morgan 2003, 10-11; Barth 1969, 133).

Komunity hrají v životě jednotlivce různě důležitou roli. Existují skupiny, které jsou jednotlivci velice blízké a se kterými přichází do kontaktu každý den, jako např. rodina, přátelé, sousedé apod. Jejich podíl na vytváření identity je zcela zásadní, a proto se objevují na epigrafických záznamech nejčastěji. Skupiny vzdálené, často abstraktního charakteru, či skupiny, s nimiž je i jejich vlastní člen konfrontován zřídka, se na epigrafických památkách objevují v menším počtu. Do této kategorie by spadala např. politická či etnická příslušnost.

\subsubsection[kolektivní-identita-a-legitimizace-společenského-postavení]{Kolektivní identita a legitimizace společenského postavení}

Hlavním rysem kolektivní identifikace je snaha lidí zařadit se do již existující skupiny lidí. Ve snaze ospravedlnit svou pozici ve skupině, člověk přistupuje na její strukturu a hierarchii, do níž se snaží zařadit. Legitimizace vlastního postavení ve společnosti je jedním z klíčových faktorů při formování identity jedince i kolektivu, a jako taková hraje i zásadní roli na epigrafických památkách.

Nápisy měly sloužit jako trvalá připomínka na vykonané skutky a dosažené postavení zainteresovaných lidí a přirozená lidská vlastnost po uznání a zisku společenského postavení tvořila důležitou součást v podstatě každé komunity. Poměrně často se na nápisech setkáváme s prezentací dosažených životních úspěchů, jako jsou zastávané funkce v rámci státní organizace, či udělené pocty patří k nejčastějším z uváděných pozic na epigrafických monumentech, jako např. {\em búleutés}, {\em proxenos}, {\em stratégos} (Van Nijf 2015, 233-243; Heller 2015, 265-266). Vzhledem k četnosti výskytů bylo důležité uvádět i dosažené úspěchy na vojenském poli či vítězství v panhellénských závodech.\footnote{Většina z těchto úspěchů zároveň poukazovala postavení v rámci vyšších vrstev společnosti, protože například profesionálně se věnovat sportu si mohli dovolit jen aristokraté a bohatí lidé, kteří se nemuseli starat o obživu (Golden 1998, 157-169).} Lidé rádi zdůrazňovali svou pozici v rámci společenské hierarchie, a z ní vyplývající výhody, a proto se na nápisech často objevují plnoprávní občané, veteráni, či propuštění otroci.\footnote{U propuštěných otroků se většinou uváděl i způsob, jakým své nově nabyté pozice dosáhl (Zelnick-Abramovitz 2005, 263-272), kdo ho propustil a z jakého důvodu.}

Skupiny, které se ve starověku podílely na tvorbě kolektivní identity, a tudíž se i objevily na nápisech, tvořily široké spektrum uskupení, které je možné rozlišit do dvou základních směrů: komunity související s chodem a řízením politických uskupení a dále kultovní a náboženské komunity.

\subsubsection[politicko-administrativní-komunity]{Politicko-administrativní komunity}

Politické komunity sdružují lidi nějakým způsobem zapojené do chodu státního uspořádání. Proměny politické identity přímo reflektují strukturální změny tehdejší společnosti, jakožto reformy institucí, hierarchického uspořádání společnosti a rozdělení moci, jako jeden z projevů nárůstu či poklesu společenské komplexity. Stejně tak se na nápisech odráží i míra ztotožnění se se společenským uspořádáním, či naopak jeho odmítnutí.\footnote{Vyjádření politické příslušnosti, či participace na politickém životě komunity, nepřímo zaznamenává vnitřní přesvědčení mluvčího. Tím, že jedinec akceptuje společenské uspořádání, či dokonce se mnohdy podílí na chodu politické jednotky a souvisejících institucí, sám se ztotožňuje s principy a hodnotami, které tato politická autorita představuje.} Nositel se sám označuje za občana dané politické jednotky a hlásí se tak ke svým právům a povinnostem vyplývajícím z členství.\footnote{Na řeckých nápisech se nejčastěji setkáváme s uváděním politické příslušnosti ve formě identifikace s členskou základnou obyvatel, nikoliv s institucemi obce. Politickou autoritu tehdy spíše představovala komunita občanů, tedy konkrétní lidé, a nikoliv abstraktní instituce, budovy, území, aparát (Whitley 2001, 165-166).}

V řeckém prostředí se setkáváme se zařazením do struktur a institucí {\em polis} či do jednotlivých kmenových uskupení ({\em ethné}; Morgan 2003, 1). {\em Polis} však nebyla jedinou politickou organizací antického světa. Na nápisech se setkáváme se s afiliací s kmenovými státy ({\em ethné}), uskupeními několika států s centrální autoritou ({\em amfiktyoniemi}), či s menšími samosprávnými jednotkami, které existovaly nezávisle na státním zřízení, pod nějž spadaly např. jednotlivé vesnice ({\em kómai}). V prostředí římské říše je to pak uznání autority Říma, což bývá vyjadřováno nápisy vztyčovanými na počest císaře, ale i vyjmenováváním zastávaných funkcí v administrativním aparátu či armádě (Van Nijf 2015, 233-243). Vojenská identita, a zejména prestiž v rámci vojenské komunity byla velice ceněna, alespoň podle četnosti nápisů patřících členům římské armády (Derks 2009, 240).

\subsubsection[kultovní-a-náboženské-komunity]{Kultovní a náboženské komunity}

Kult a víra měly vždy podstatný vliv na formování identity, a nejinak tomu bylo i v antice. Vyjádřením příslušnosti k náboženské komunitě jedince zároveň vyjadřoval i své náboženské přesvědčení a kulturní přináležitost. Nápisy s projevy náboženské příslušnosti nepřímo reflektují rozvrstvení společnosti a přináležitost obyvatelstva do kulturního okruhu. Změny ve variabilitě náboženských projevů jsou pak jedním z nepřímých projevů nárůstu či poklesu společenské, a zejména kulturní komplexity.

Druh zvolené náboženské komunity napovídá mnoho o charakteru víry a vnitřním přesvědčení jednotlivce. Kulty místních božstev vycházejí většinou z lokální tradice, reagují na aktuální potřeby komunity a pomáhají utvářet lokální identitu a pocit sounáležitost na každodenní úrovni (Parker 2011, 25, 221). Kult {\em héroů} se stal velice důležitým sjednocujícím prvkem pro celou komunitu, která se mohla scházet u jeho svatyně. Do kultů místních božstev mohli vstupovat lidé urozeného původu, ale i lidé zcela neurození a chudí, někdy dokonce i ženy a otroci. Mýty a víra obecně osvětlovaly komunitě hierarchii vztahů, propůjčovaly legitimizaci nároků na půdu a území, vysvětlovaly její původ. V antickém světě téměř každá komunita měla svého {\em héroa}, od nějž mohla odvozovat svůj původ a nárokovat si jeho autoritu (Parker 2011, 116, 119). Příkladem mohou být tzv. {\em héroové} zakladatelé ({\em oikistai}), od nichž odvozovala svůj původ celá města svůj původ, jako např. Hagnón a Brásidás v Amfipoli (Thuc. 5.11; Malkin 1987, 228-232).

Vliv na formování identity mají jistě i velká kultovní centra, kde se scházeli lidé z nejrůznějších regionů a měst. Oproti lokálním kultům {\em héroů} měla totiž tato centra multi-komunitní charakter, kam přicházeli věřící z celého Středomoří a Pontu. Tato místa se stala centrem setkávání a kulturní výměny jednotlivých komunit a vzájemného atletického soupeření (Vlassopoulos 2013, 39-40). Účast na těchto slavnostech však byla vzhledem ke své finanční náročnosti omezena spíše na bohatší vrstvy společnosti, a nikoliv na aktivity spojené s každodenními úkony víry. Stejně tak zastávání některých specializovaných funkcí v rámci kultu, jako např. kněží, bylo pouze záležitostí elit. S vykonáváním kultovních funkcí byla spojená prestiž, a někdy i finanční nákladnost, tudíž náboženské úřady často zastávali jen členové aristokracie (Parker 2011, 51). Proto se často setkáme i s nápisy věnovanými kněžími, kteří tak jasně poukazují na své vysoké společenské postavení.

\section[shrnutí]{Shrnutí}

Nápisy nabízejí jedinečný náhled do společnosti, která je vytvářela. V jejich textech, ale i v jejich provedení se zrcadlí nejen demografické složení epigraficky aktivní části populace, ale i vztah obyvatel vůči okolní komunitě, společensko-politickému uspořádání, ale i například postoj vůči cizím kulturám.

Ač byly nápisy produkovány relativně omezenou skupinou lidí, jejich charakter míra a trvání jejich produkce napovídá o struktuře společnosti více, než se na první pohled může zdát. Role, jakou nápisy hrály ve společnosti, souvisí do velké míry s vývojovým stupněm společnosti, která je vytvářela a mírou provázanosti jejích jednotlivých složek. V jednodušších a menších komunitách se nápisy objevovaly pouze zřídka a sloužily převážně legitimizaci moci a poukázání na společenskou prestiž vládnoucích vrstev. Naopak ve společnostech se složitější vnitřní strukturou mohlo docházet k organizované produkci nápisů za účelem udržení chodu celého aparátu a zajištění lepší efektivity předávání a zaznamenávání informací. Jako vedlejší produkt těchto aktivit vznikla produkce soukromého charakteru, která se začala obsahově rozvíjet nezávisle na produkci organizované politickou autoritou, ale i nadále s ní byla spojena zejména na technologické úrovni.

\chapter{Metodologie a zvolená metoda}
V disertační práci hodnotím v časovém průřezu společnost antické Thrákie na základě studia korpusu dochovaných nápisů. V rámci analýzy epigrafického materiálu sleduji nejen kvantitativní výskyt nápisů v místě a čase, ale zároveň se zaměřuji na kvalitativní analýzu obsahu nápisů a jejich reflexi uspořádání tehdejší společnosti na základě výskytu charakteristických prvků. Jako součást kvalitativní analýzy sleduji zejména diverzifikaci politického uspořádání a společenských rolí, stupeň provázanosti obyvatelstva se strukturami komplexní společnosti, míru fluktuace obyvatelstva různého kulturního pozadí, proměny publikačních a onomastických zvyklostí, a v neposlední řadě šíření symbolických hodnot, tradičně považovaných za konzervativní prvek společnosti. Na základě dostupných dat zařazuji sledované prvky do kontextu společnosti, v níž se vyskytovaly s cílem zhodnotit, zda se jedná o přímý či nepřímý důsledek kontaktů s řeckým světem a řeckou kulturou.

\section[zvolená-metoda]{Zvolená metoda}

Při analýze materiálu se držím principu falzifikovatelnosti a testovatelnosti použitých metod, ve snaze ověřit či vyvrátit z nich vycházející interpretace (Popper 2005, 57-73). Abych mohla nápisy z různých zdrojů a různé povahy vzájemně srovnávat, musela jsem přistoupit k jejich kategorizaci na základě interpretace jejich obsahu a formy. Tato míra jisté abstrakce umožňuje zjednodušený a kvantifikovatelný náhled do jednotlivých komunit s cílem je vzájemně porovnat v rámci několika časových úseků (O'Shea a Barker 1996, 13-19; Bodel 2001, 80-82). Při interpretaci nápisů se držím tradičních principů a kategorií řecké a latinské epigrafiky, stanovených již autory {\em Inscriptiones Graecae} a {\em Corpus Inscriptionum Latinarum}. Tyto principy, ač vznikly před více než 200 lety, tvoří i dnes pevné základy epigrafické disciplíny (Bodel 2001, 153-174; McLean 2002, 22, 181-182).

Pro usnadnění zpracování velkého množství nápisů jsem vytvořila elektronickou databázi, v níž jsem shromáždila přes 4600 nápisů z oblasti Thrákie. K analýze tak velkého množství nápisů jsem využívala moderní metody a nástroje, známé spíše z přírodních věd, z archeologie a z oblasti tzv. {\em digital humanities} (Bodel 2012, 275-293). Konkrétní podoba užitých postupů je však přizpůsobena specifikům epigrafického materiálu, a je reakcí na nutnost analyzovat množství materiálu s poměrně velkým počtem informací nejisté povahy, jako je například datace či umístění nápisu. Do velké míry se však jedná o inovativní přístup, který staví na několika pilotních studiích a kombinuje dohromady přístupy několika disciplín (Benefiel 2010, 45-65; Feraudi-Gruenais 2010; 14-16; Witschel 2010, 77-86; Janouchová 2014).

Ke studiu společnosti antické Thrákie na základě studia dochovaných nápisů přistupuji na třech vzájemně provázaných úrovních: a) na úrovni jednotlivých nápisů a funkce, jakou ve společnosti zastávaly; b) na úrovni společenských trendů; c) a na úrovni epigrafických produkčních center. Zajímají mě nejen geografické vzorce a relativní rozmístění nápisů vůči kulturním centrům a komunikacím, ale i všeobecné proměny společnosti v závislosti na známých společensko-politických událostech.

\subsection[funkce-nápisů]{Funkce nápisů}

Prvním bodem analýzy na úrovni jednotlivých nápisů je stanovení role, jakou nápisy v dané společnosti hrály, a jak se tato role proměňovala v závislosti na čase a místě. Funkce nápisů, kterou je možné odvodit z jejich obsahu a formy, umožňuje hodnotit povahu epigrafické produkce a roli jakou nápisná kultura hrála v životě jednotlivců, ale i celé komunity (McLean 2002, 181-182). Z povahy nápisů je možné odvodit celou řadu informací o existujícím politickém uspořádání, společenské hierarchii a infrastruktuře, která umožňuje jak fungování samotného politické uspořádání, tak stojí i za samotnou epigrafickou produkcí (Johnson 1973, 3-4; Tainter 1988, 99-106, 111-115).

\subsection[celospolečenské-trendy-a-složení-společnosti]{Celospolečenské trendy a složení společnosti}

Dále se při analýze nápisů zaměřuji na proměňující se složení populace, která se aktivně zapojovala do epigrafické produkce. V tomto ohledu mě zajímá především konkrétní způsob sebeidentifikace individuálních osob na médiu permanentního charakteru a její proměny. Zejména se zaměřuji na identifikaci s rodinou a nejbližší komunitou, dále s většími společensko-politickými celky, a případně s celky etnickými (Jenkins 2008; Derks 2009, 239-244; Schuler 2012, 63-67). V neposlední řadě mě zajímá, jaký byl vztah epigraficky aktivní populace k řeckému etniku a řeckým kulturním zvyklostech, zda docházelo k jejich přejímání či adaptaci na nové podmínky a ke vzniku zcela nového symbolického systému hodnot a dorozumívacích prostředků.

S tím velmi úzce souvisí i analýza norem chování v závislosti na společenském postavení, pohlaví a v neposlední řadě i původu. Zajímám se o konkrétní projevy a šíření inovací a kulturních prvků v pobřežních a vnitrozemských komunitách v průběhu jednotlivých století.\footnote{Podrobněji v kapitole 6 pro analýzu nápisů v jednotlivých stoletích a v kapitole 7 pro analýzu rozmístění epigrafických produkčních center.} Sleduji jednak proměny složení epigraficky aktivní populace, ale i proměny epigrafické produkce obecně. Dále se zaměřuji na výskyt konkrétních prvků společenské organizace a projevů kultury a náboženství, a zasazuji roli a význam daných prvků do kontextu sledované společnosti v daném časovém období (Dietler 2005, 66-68).

\subsection[epigrafická-produkční-centra]{Epigrafická produkční centra}

Místa se zvýšenou epigrafickou aktivitou, tedy místa v jejichž blízkosti byly nápisy nalezeny, poukazují s největší pravděpodobností i na epigrafickou aktivitu v tamní komunitě.\footnote{Rozmístění nápisů a vztah k jednotlivým produkčním centrům podrobněji rozebírám v kapitole 7.} V rámci místně zaměřené studie mě zajímá vztah rozmístění nálezových míst nápisů v krajině, a zda je z nich možné vypozorovat obecně platné trendy. Zajímá mě například, zda se nápisy objevují pouze v okolí řeckých měst, či se nalézají i v čistě thráckém kontextu a zda je například možné pozorovat rozdíly mezi umístěním nápisů v krajině v závislosti na jejich společenské funkci. Dále mě zajímá, jaké je rozmístění nápisů vůči místům s lidskou aktivitou, jakou jsou např. města či cesty, a zda je možné pozorovat vliv této infrastruktury na rozmístění nápisů, případně na jejich zvýšené koncentrace na určitých místech v závislosti na demografii regionu (Woolf 1998, 77-105; Woolf 2004, 157-164; Bodel 2001, 80-82). Navzájem srovnávám jednotlivé oblasti Thrákie s cílem postihnout opakující se vzorce chování na základě rozmístění nápisů. V neposlední řadě srovnávám epigrafická produkční centra se známými archeologickými prameny. Interpolací a statistickým zhodnocením jednotlivých epigrafických nálezů z vybraných lokalit se snažím zhodnotit míru relevance a výpovědní hodnoty epigrafických památek pro studium antické společnosti jako celku.

\section[použité-epigrafické-korpusy-a-jejich-specifika]{Použité epigrafické korpusy a jejich specifika}

Epigrafické památky nalezené na území, které je tradičně označováno jako Thrákie, tedy oblast dnešního Bulharska, severního Řecka a evropského Turecka, jsou poměrně hojné. Nalézají se zde jak nápisy psané řecky, latinsky\footnote{Celkový počet latinsky psaných nápisů nalezených na území Thrákie není přesně zmapován, nicméně se odhaduje, že jich je zhruba třikrát méně než nápisů psaných řecky. Milena Minkova (2000, 1-7) předpokládá existenci zhruba 1200-1300 latinských nápisů na území dnešního Bulharska. V porovnání se zhruba 3000 řecky psanými nápisy ze stejného území dojdeme k poměru 70:30 právě ve prospěch řeckých nápisů (Janouchová 2016, 437). Podrobněji se vyjadřuji k problematice zvoleného publikačního jazyka v kapitole 5.}, bilingvně řecko-latinsky\footnote{Diskuze k řecko-latinským nápisům je součástí kapitoly 5.}, tak existuje i několik málo nápisů klasifikovaných jako psaných thrácky\footnote{Tomaschek 1893; Detschew 1952; 1957; Dimitrov 2009. S největší pravděpodobností thráčtina představovala indoevropský jazyk, podobný řečtině, nicméně nedochoval se dostatek písemného materiálu a pramenů, aby bylo možné thráčtinu blíže charakterizovat. Do dnešní doby bylo nalezeno celkem 6 nápisů (Dimitrov 2009, 1-19), a zhruba 1400 osobních a místních jmen, které se vyskytují v řeckých a latinských nápisech (Dana 2011, 25). I když probíhají pokusy o překlad těchto nápisů, přesvědčivá a všemi přijímaná interpretace stále ještě čeká na své objevení.}. Předmětem této studie jsou zejména nápisy psané řecky, jakožto nejpočetnější dochovaný soubor nápisů, který představuje statisticky nejrelevantnější vzorek s nejvyšší reprezentativní hodnotou, ač k nápisům psaným jinými jazyky v daných situacích přihlížím též.

Epigrafické prameny jsem pro usnadnění zpracování velkého počtu dat shromáždila do databáze {\em Hellenization of Ancient Thrace}. Databáze obsahuje data z 10 ucelených epigrafických korpusů řecky psaných nápisů, které pocházejí či byly nalezeny na území Thrákie. Dále databáze obsahuje epigrafická data z menších korpusů a jednotlivých článků, které souhrnně spadají pod jedenáctou položku {\em Other}, tedy jiný zdroj.\footnote{Kompletní seznam všech nápisů, které jsou součástí databáze se nachází v Apendixu 3}

Většina korpusů je vydávána v rámci jednotlivých moderních států a nezaměřuje se na oblast Thrákie jako celku. Pro oblast moderního Bulharska používám jako hlavní zdroj informací v latině publikovaná nápisná korpora {\em Inscriptiones Graecae in Bulgaria repertae} ({\em IG Bulg}; Mihailov 1956, 1958, 1961, 1964, 1966, 1970, 1997), které zahrnují pouze území moderního Bulharska. Tato korpora představují nejucelenější sbírku řeckých nápisů z~mnou zkoumané oblasti až s~téměř 3000 exempláři. Nápisy jsou opatřeny dobrou fotografickou dokumentací a relativně dobrým popisem. Vzhledem k~faktu, že některé knihy již byly vydány před více jak 50 lety, jsou některé informace, dle moderních principů epigrafických publikací zastaralé, či nedostatečně detailní. Tento fakt jim však neubírá na výpovědní hodnotě a stále představují cenný zdroj historických informací. Mihailov editoval většinu nápisů sám, a tak není divu, že při tak velkém počtu nápisů došlo i k~chybnému vydání některých z~nich. Tyto nedostatky se však sám autor snažil napravit v~5. svazku (Mihailov 1997), a ve většině případů se mu to i zdařilo.

Nápisy publikované v Bulharsku po roce 1997 spadají do kategorie {\em Other}, která obsahuje jak nápisy publikované v rámci {\em Supplementum Epigraphicum Graecum} ({\em SEG}) od r. 1996 do r. 2010 (vydané v roce 2015), tak v rámci menších korpusů (Velkov 1991; Manov 2008), či v rámci jednotlivých článků (Velkov 2005; Gyuzelev 2002; 2005; 2013).\footnote{Nápisy bývají často publikovány v lokálních časopisech, které jsou obtížné dostupné, a proto je možné, že některé nápisy v databázi zatím zaznamenány nejsou, avšak maximálně v řádech několika desítek exemplářů. Tento fakt by však neměl nijak ubrat na hodnotě již dosud sesbíraných nápisů a jejich celkové relevanci pro vybraný druh analýzy.}

Oblast egejské části severního Řecka pokrývá nedávno v řečtině publikovaný soubor {\em Epigrafes tis Thrakis tou Aigaiou} ({\em I Aeg Thrace}; Loukoupoulou {\em et al.} 2005). Tato publikace pokrývá oblast mezi řekami Hebros a Néstos a jedná se o soubornou publikaci epigrafických památek jižní části antické Thrákie, rozkládající se na území dnešního Řecka. Soubor 500 nápisů pokrývá nápisy z~hlavních řeckých měst regionu, jako je Abdéra, Maróneia, Topeiros, Zóné, Drys a jejich bezprostřední okolí. Publikace zprostředkovává i kvalitní obrazovou dokumentaci, nutnou pro účely disertační práce. Korpus zahrnuje nalezené nápisy do r. 2005, novější nápisy je taktéž možno nalézt v {\em SEG}.

Další zdroje nápisů pokrývají oblast jižní Thrákie nacházející se na území dnešního Turecka s~celkovým počtem okolo 700 nápisů. Jedná se o jediné tři ucelené soubory nápisů z~této oblasti, které zpracovávají nápisy z řeckých měst na pobřeží a jejich bezprostředního okolí ({\em IK Sestos}, Krauss 1980 pro oblast Séstu a Thráckého Chersonésu; {\em IK Byzantion}, Lajtar 2000 pro oblast Byzantia; {\em Perinthos-Herakleia}, Sayar 1998 pro oblast Perinthu a okolí). Některé publikace jsou až 30 let staré a pro nově nalezené nápisy v~těchto oblastech je opět nutné konzultovat SEG. Problémem napříč všemi moderními zeměmi zůstávají nápisy nepublikované, které nemohu do své analýzy zahrnout. Jednak je to případ tureckých měst, kde velká část nápisů zůstává nepublikovaná v muzeích, či v sekundárním kontextu, ale jedná se například i o objekty z vykopávek, jako příklad graffit z~antické Seuthopole či Kabylé, které na svou zevrubnou publikaci stále ještě čekají mnohdy i 60 let po svém objevení.\footnote{Informace poskytnutá L. Domaradzkou v roce 2012.}

\section[organizace-a-sběr-dat]{Organizace a sběr dat}

V roce 2012 jsem dělala průzkum dostupných elektronických databází a nástrojů, které by mi pomohly zpracovat vybrané téma. Databáze dostupné na internetu byly tehdy relativně omezené a nenabízely tak široké spektrum funkcí jako v roce 2017, zejména co se týče exportu, souvisejících geografických dat. Dalším zásadním omezením bylo, že data zvolená k výzkumu pocházela z mnoha zdrojů, zpracovaných více či méně detailně, a vyskytovala se především pouze v tištěné formě. Rozhodla jsem se tedy, že data budu sbírat sama, a to metodou digitalizace tištěných korpusů a jiných zdrojů relevantních dat. Kde bylo možné využívat již digitalizovaná data, jako např. {\em Barrington Atlas of Greek and Roman World}, {\em Searchable Greek Inscriptions}, {\em Supplementum Epigraphicum Graecum} atp., postupně jsem je zapojovala do návrhu projektu, v závislosti na jejich dostupnosti.\footnote{V pilotní fází projektu jsem sbírala data do několika jednoduchých tabulek v rámci MS Excel, nicméně jsem brzy zjistila, že je tento přístup neefektivní a velice náchylný k chybovosti, zejména v rámci týmové spolupráce. V roce 2012 jsem tedy vytvořila relační databází v MS Access. V podstatě se jednalo o jednoduchý model tabulek spojených pomocí relací (klíčů), který pomohl zaznamenávat data v normované podobě (tedy neduplikující se záznamy). V roce 2013 jsem obdržela dvouletou finanční podporu od Grantové Agentury Univerzity Karlovy (GAUK 2013/546813) a mohla jsem zapojit do projektu studenty magisterského a doktorského studia na FFUK (Markéta Kobierská, Jan Ctibor, a Barbora Weissová). V té době jsem byla na studijní stáži v Austrálii na University of New South Wales, a ostatní členové týmu byli taktéž studijně v různých částech Evropy. Databáze v MS Access v roce 2013 ještě nepodporovala cloudové řešení, a nebylo v mých tehdejších technických možnostech zajišťovat její serverový hosting. Proto jsem se rozhodla pro jiné řešení, které by umožnilo přístup přes internet, a které by navíc nebylo závislé na používaném operačním systému (Windows, Linux, Mac).}

\section[technické-řešení-heurist]{Technické řešení: Heurist}

V rámci studijního pobytu v Sydney v roce 2012 jsem se seznámila s platformou {\em Heurist Scholar}, tehdy vyvíjenou na University of Sydney a rozhodla jsem se jí využít pro sběr dat v digitální podobě.\footnote{Heurist vznikl v roce 2005 pod vedením Dr. Iana Johnsona z University of Sydney. \useURL[url2][http://heuristnetwork.org/][][{\em http://heuristnetwork.org/}]\from[url2]. Od května 2013 je projekt veden jako Open Source a zájemci ho mohou využívat pro výzkumné účely bezplatně. Heurist je online databáze specificky navržená pro potřeby digitálního zaznamenávání historických dat v akademickém prostředí. Nejedná se tedy o databázi v pravém smyslu slova (například při porovnání s MySQL, PostgreSQL atp.), ale spíše o platformu umožňující sbírat, analyzovat a zveřejňovat data bez předchozích programátorských znalostí. Hlavní výhodou je přístupnost a flexibilita: kdokoliv s přístupem na internet a přístupem k databázi může přes webové rozhraní zadávat a analyzovat data. Heurist navíc v sobě kombinuje i určité funkčnosti geomapovacího softwaru, což je jedna z prerekvizit mého výzkumu. V roce 2013 epigrafické databáze ještě nezaznamenávaly geografická data a nebyly propojeny s geoinformatickými systémy (GIS) do té míry, jako je tomu v roce 2017, a tudíž tato funkčnost Heuristu měla zásadní roli na mé rozhodnutí používat právě Heurist.} Databázi jsem vytvořila v programu Heurist 3.1.0 v červnu 2013. Po prvotní fází zadávání dat jsem provedla revizi a úpravu struktury v říjnu 2013. Hlavní fáze zadávání dat probíhala od června 2013 do září 2016.\footnote{Celková doba nutná k vytvoření databáze: zhruba dva měsíce studia principů vytváření databází, jeden měsíc věnovaný testování a vylepšování databáze. Celkové zadávání nápisů do databáze bylo rozděleno mezi tři uživatele v poměru Janouchová (48 \letterpercent{}, 2362 nápisů), Kobierská (27 \letterpercent{}, 1348 nápisů), Ctibor (25 \letterpercent{}, 1270 nápisů); celkem 293 nápisů má dva autory z důvodu např. jeho několikanásobné opakované publikace, či re-editace v epigrafickém korpusu. Průměrná doba zadávání jednoho nápisu je zhruba 15 minut, s tím, že komplexnější nápisy trvaly až několik hodin, jednodušší naopak kratší dobu. Prostým vynásobením 15 minut x počet nápisů (4667) se dostaneme na šest měsíců nepřetržitého zadávání. Zadávání bylo vzhledem k studijnímu a pracovními vytížení členů týmu rozloženo do dvou let (hlavní zadávací fáze červen 2013-prosinec 2014; březen 2016-červen 2016). Po ukončení zadávání jsem jakožto administrátor kontrolovala kvalitu záznamů a jejich přesnost, a případně nápisy editovala, kde bylo nutné, což zabralo zhruba třetinu času oproti zadávání, tedy zhruba dva měsíce času. Tato fáze skončila v září 2016.} K 9. listopadu 2016 databáze obsahuje 4665 nápisů které díky své digitalizaci získaly jednotnou strukturu, ač pochází z různých epigrafických zdrojů, různé kvality a detailnosti zpracování. Data jsou volně dostupná komukoliv na internetu v několika digitálních formátech.\footnote{Heurist poskytuje možnost data především prohlížet a filtrovat, vyhledávat v nich, nicméně pro komplexnější analýzu je nutné data vyexportovat a využívat další specializované programy. Data získaná zadáváním informací do Heuristu je možné exportovat, a následně analyzovat, několika možnými způsoby. Geografická data je možné exportovat ve formátu KML ({\em keyhole markup language}) či {\em shapefile}, který je možné zobrazit v geoinformačním softwaru (ArcGIS, QGIS) či v Google Earth. Geografická data je možné exportovat ve formátu CSV ({\em comma separated value}) a transformovat do požadovaného formátu přímo v geoinformačním softwaru. Tabulární data (data textového charakteru) je možné exportovat ve formátu CSV a dále zpracovávat pomocí analytických programů jako je MS Excel, Google Spreadsheet, R. Databázi je možné jako celek exportovat jako XML formát, MySQL, PostgreSQL avšak tento způsob je vhodný spíše pro zálohování dat, než pro jejich analýzu a je vhodný pro zkušené uživatele. Data v neupraveném formátu společně s výsledky analýz a mapami jsou volně dostupná na adrese: \useURL[url3][https://github.com/petrajanouchova/hat_project][][{\em https://github.com/petrajanouchova/hat_project}]\from[url3]. V současné době HAT databáze používá verzi Heurist 4.2.8. (od 14. září 2016) a po přihlášení je dostupná na adrese: \useURL[url4][http://heurist.sydney.edu.au/h4/?db=HAT_Hellenization_of_Ancient_Thrace][][{\em http://heurist.sydney.edu.au/h4/?db=HAT_Hellenization_of_Ancient_Thrace}]\from[url4].}

\subsection[struktura-databáze-a-použitá-metodologie]{Struktura databáze a použitá metodologie}

Databáze je navržena v angličtině, pro větší přístupnost akademické veřejnosti a další použitelnost dat. Původní jazyky tištěných epigrafických korpusů jsou bulharština, řečtina, latina, němčina, francouzština a angličtina, což mnohdy znesnadňovalo práci s nimi. V současné podobě v HAT databázi jsou tak data přístupná i té části badatelů, která neovládá jeden či více z těchto jazyků. Údaje v databázi jsou jednak digitální podobou epigrafického {\em lemmatu} tak, jak ho vydali editoři korpusů, a jednak se jedná o interpretace odvozené z epigrafického {\em lemmatu}, textů nápisů a doprovodné fotografické a kresebné dokumentace.\footnote{Schéma databáze se nachází v tabulce 4.00 v Apendixu 1. Celkový přehled zpracovaných nápisů je součástí Apendixu 3.} V následující sekci se věnuji vybraným součástem struktury databáze a vysvětlení použitých termínů pro větší srozumitelnost dat obsažených v databázi. Tato struktura byla vytvořena na základě organizace tištěných korpusů a vychází především z uspořádání korpusů {\em IG Bulg.}\footnote{Podrobné vysvětlení jednotlivých položek databáze považuji za základní předpoklad práce s daty v režimu {\em Open Source}, tedy v režimu, který předpokládá šiřitelnost dat a srozumitelnost obsahu databáze pro další uživatele.}

\subsubsection[identifikační-čísla]{Identifikační čísla}

Každý nápis představuje jedinečný záznam, jemuž Heurist automaticky přidělí unikátní identifikační číslo ({\em Heurist ID}). Toto číslo není příliš intuitivní pro práci s nápisy, a proto jsem vytvořila ještě jiné identifikační číslo, které nápis rovnou přiřazuje ke korpusu, v němž byl publikován ({\em Corpus ID Numeric}). Každý korpus má daný číselný kód, a je tak možné na první pohled určit, do jakého korpusu nápis náleží. V podstatě se jedná o číslování použité v původním korpusu, ke kterému se přidá určitá hodnota dle zvoleného korpusu (10 000, 20 000,...). Korpus jiných zdrojů ({\em Other}) obsahuje nápisy z několika zdrojů s nestejným číslováním, a proto jsem zde zvolila posloupné číslování (100 000 + číslo předcházejícího nápisu + 1).\footnote{Přehled jednotlivých korpusů a přidělených řad identifikačních čísel nápisů se nachází v tabulce 4.01 v Apendixu 1, a celkový přehled všech použitých nápisů v Apendixu 3.} Každý nápis má celkem tři identifikační čísla: číslo přidělené Heuristem ({\em Heurist ID}), identifikační číslo s kódovanými hodnotami ({\em Corpus ID numeric}) a identifikační číslo uvedené v původním korpusu ({\em Corpus ID number}). Některé nápisy mají uvedené ještě identifikační číslo ze {\em Supplementum Epigraphicum Graecum} (SEG number), avšak toto číslo není uváděno systematicky u všech nápisů.\footnote{Příkladem nápisu se všemi identifikačními čísly je nápis SEG 49:992, který spadá do korpusu {\em Other}, jeho {\em Numeric ID number} je 100405, a {\em Heurist ID} je 16577. Ukázka záznamu v programu Heurist se nachází na obrázku 4.01 v Apendixu 2.}

\subsubsection[geografické-údaje]{Geografické údaje}

Většina korpusů uvádí původ nápisu pouze velmi obecně, a tudíž určit konkrétní místo nálezu může být v určitých případech obtížné. Navíc se přístup jednotlivých editorů při popisu místa nálezu liší, a tím pádem i charakter dostupných informací. Pro systematické zpracování nápisů bylo nutné geografické informace sjednotit, abych bylo možné je později souhrnně analyzovat a zobrazit formou map. K tomuto účelu jsem využívala program Heurist a jeho mapovací součást {\em Heurist Digitizer}.\footnote{Heurist využívá mapovací software {\em Google Maps}, který umožňuje manipulací s mapou či satelitním snímkem zaznamenávat zeměpisné koordináty (zeměpisnou šířku a zeměpisnou délku, v sekci Geolocation). Obrázek 4.02 v Apendixu 2 zobrazuje rozhraní programu {\em Heurist Digitizer}, v němž se zadávalo místo nálezu každého nápisu.}

Konkrétní bod míst nálezu nápisu je zvolen dle informací z informací poskytnutých editorem nápisů, uvedených v nápisném {\em lemmatu}. Editoři však udávají údaje o místě nálezu velmi obecně, a proto bylo nutné přijít s řešením zahrnujícím míru nejistoty určení místa nálezu. Na základě těchto specifik byl každému bodu přiřazen tzv. koeficient přesnosti místa nálezu ({\em Position certainty}), jehož hodnoty udávají hodnotu přesnosti místa nálezu.\footnote{Tzv. {\em buffer zone}, čili nárazníková zóna okolo daného bodu se vzdáleností do 1 km, do 5 km, do 20 km a nad 20 km, viz Tabulka 4.02 v Apendixu 1. Program Heurist udává každému místu nálezu jeden konkrétní bod na planetě, který je vyznačený zeměpisnými koordináty. Editoři nápisů naopak udávají spíše oblast, v níž byl nápis nalezen a nikoliv konkrétní bod. Zvolená metoda se snaží oba dva přístupy spojit a zároveň zaznamenat data o místu náletu v jednotné formě, díky níž je možné data vzájemně srovnávat a analyzovat. Podrobněji zvolenou metodologii rozebírám níže v této kapitole.} Nápisy, jejichž místo nálezu bylo možné určit s přesností do 20 km, využívám v rámci analýzy rozmístění nápisů v krajině v kapitole 7.

Následující položky vysvětlují způsob zaznamenávání geografických údajů do databáze:\footnote{Ukázka záznamu v programu Heurist zaznamenávající zeměpisné údaje se nachází na obrázku 4.03 v Apendixu 2.}

\startitemize
\item
  \startblockquote
  Místní údaje ({\em Location}) zaznamenávají místní jména tak, jak je obsahují epigrafické korpusy. Zaznamenává se jméno nejbližšího místa či osídlení, dále jméno antické lokality, pokud je známé, a region antického města, pod které nálezové místo spadalo ({\em Modern Location, Ancient Site, Ancient site - region}). Pokud se jedná o místní jméno, které již dnes neexistuje, pak je o tom poznámka v položce {\em Geography notes}. Do položky {\em Geography notes} se zaznamenávají veškeré údaje dostupné o místě nálezu, které není možné zadat do žádného jiného pole.
  \stopblockquote
\item
  \startblockquote
  Položka {\em Reuse} zaznamenává, zda byl nápis sekundárně použitý, např. nalezený ve zdech budovy, či přenesený do kostela. Pokud je položka prázdná, nemáme informace o sekundárním použití nápisu.\footnote{Příkladem zmínky o sekundárním použití může být {\em IG Bulg} 1,2 393: „{\em Apolloniae, olim conservatum erat in aedibus quibusdam}.”}
  \stopblockquote
\item
  \startblockquote
  Položka {\em Archaeological context} zaznamenává známý archeologický kontext nalezeného nápisu tak, jak ho udává epigrafický korpus ({\em Funerary} pro lokality s vyskytujícími se hroby, hrobkami, či archeologicky identifikovanými pohřebišti; {\em Habitation} pro lokality s archeologicky identifikovanými sídlišti, budovami a místy lidského osídlení; {\em Commercial} jako podkategorie {\em Habitation}, pro lokality s identifikovaných obchodním charakterem, jako například emporion, přístav, agora apod.; {\em Other} pro typy archeologické lokality nespadající do žádné z uvedených kategorií, s detaily uvedenými v poznámce; {\em Religious/Ritual} pro svatyně; {\em Secondary} pro nápisy nalezené v sekundárním kontextu).
  \stopblockquote
\item
  \startblockquote
  Položka {\em Mound} zaznamenává, zda editor udává, že byl nápis nalezen v pohřební mohyle, či její bezprostřední blízkosti.
  \stopblockquote
\stopitemize

\subsubsection[údaje-o-nosiči-nápisu]{Údaje o nosiči nápisu}

Tato sekce zaznamenává údaje o nosiči, na němž se nápis nachází. Informace pochází jak z epigrafického {\em lemmatu}, tak z doprovodného vizuálního materiálu, pokud byl k dispozici, jako je fotografie, kresba, fotografie oklepku. Podle charakteru epigrafického objektu je možné alespoň relativně určit, zda se jednalo o místně produkovaný předmět, či se mohlo jednat o import.\footnote{Ukázka záznamu dat o nosiči nápisu v programu Heurist se nachází na obrázku 4.04 v Apendixu 2.}

\startitemize
\item
  \startblockquote
  Položka {\em Material category} zaznamenává, z jakého materiálu byl předmět vyroben ({\em Metal} - kov; {\em Other} - jiný druh materiálu, detaily se uvádějí do poznámek {\em Decoration notes}, např. nápis vyrytý do omítky, či součástí mozaiky; {\em Perishable} - netrvanlivý materiál jako dřevo aj.; {\em Pottery} - keramika, terracotta; {\em Stone} - kámen). Pokud byl předmět vyroben z kamene, zajímá mě, z jakého druhu ({\em Stone}), případně zda je možné určit jeho původ ({\em Origin of stone}).
  \stopblockquote
\item
  \startblockquote
  Položka {\em Object category} zaznamenává, o jaký druh předmětu se jedná: {\em Architectural feature} pro architektonické prvky nesoucí nápis, {\em Mosaic} pro mozaiku obsahující nápis, {\em Other} pro předmět nespadající ani do jedné z uvedených kategorií. Případné detaily o dekoraci nosiče jsou uvedené do poznámky {\em Decoration notes}, {\em Stele} pro stélu, či tabulku nesoucí nápis; {\em Sculpture} pro sochu či její součást nesoucí nápis, {\em Vessel} pro nádobu či její součást nesoucí nápis; {\em Wall} pro stěnu stavby, či zeď nesoucí nápis.
  \stopblockquote
\stopitemize

Dále databáze zaznamenává celou řadu dat získaných jednak z textu epigrafického {\em lemmatu}, ale zejména z doprovodných vizuálních příloh. Editoři často doplnili publikaci fotografií, kresbou, fotografií oklepku, či méně často textovým popisem stavu dochování, či dekorace. Pro usnadnění zaznamenávání a následné analýzy jsem vytvořila základní typologii popisu předmětu, doplněnou o konkrétní příklady s fotografiemi.

\startitemize
\item
  \startblockquote
  Položka {\em Preservation} zaznamenává stav dochování předmětu v době vydání korpusu. Stav dochování je odhadován na základě procentuálního dochování původního tvaru a velikosti nadepsaného předmětu (100 \letterpercent{}, 75 \letterpercent{}, 50 \letterpercent{}, 25 \letterpercent{}), či zda se předmět dochoval pouze ve formě oklepku, překresby, či zda je předmět ztracený, či není možné stav jeho dochování určit.\footnote{Obrazová galerie příkladů je součástí tabulky 4.03 v Apendixu 1.}
  \stopblockquote
\item
  \startblockquote
  Položka {\em Decoration} zaznamenává typy použitých dekoračních technik - zejména mě zajímá reliéf a malovaná dekorace; další techniky se popisují v {\em Decoration Notes}. Položka {\em Relief decoration} zaznamenává konkrétní typy reliéfní dekorace. Položka {\em Architectural relief} je první typ reliéfní dekorace, zaznamenávající všechny druhy použité architektonické dekorace a její možné kombinace.\footnote{Jejich seznam a obrazová galerie je součástí tabulky 4.04 v Apendixu 1.} Položka {\em Figural relief} zaznamenává konkrétní druhy figurálního reliéfu, které se na předmětu vyskytují, a to i jejich kombinace.\footnote{Jejich seznam a obrazová galerie je součástí tabulky 4.05 v Apendixu 1.} Položka {\em Decoration notes} zaznamenává informace nespadající ani do jedné z výše uvedených kategorií či podrobnější informace uvedené editory korpusu, případně zpozorované na vizuálním materiálu.
  \stopblockquote
\item
  \startblockquote
  Položka {\em Visual record} availability zaznamenává, zda editoři korpusu přiložili též vizuální dokumentaci nápisu, jako fotografii, kresbu, fotografii oklepku.
  \stopblockquote
\stopitemize

\subsubsection[údaje-získané-z-textu]{Údaje získané z textu}

Tato sekce zaznamenává informace získané jednak z epigrafického {\em lemmatu}, jako je např. datace, ale i z formy a obsahu samotného nápisu. Přidržuji se standardní typologie řeckých nápisů, jak je známá z {\em Inscriptiones Graecae}, či jak je uvádí McLean (2002, 181-210).

\subsubsection[datace]{Datace}

Datace nápisu patří ve velké míře o interpretaci autora korpusu na základě referencí v textu, provedení nápisu a na základě analogií. Datace nápisu je zaznamenána tak, jak jí uvádějí editoři korpusu, s výjimkou, že všechna data jsou převedena do číselného intervalu pro jejich snadnější zpracování a vzájemnou konzistentnost dat. Roky před naším letopočtem se zapisují jako záporné hodnoty, a roky po našem letopočtu jako plusové hodnoty. {\em Start Year} udává maximální možné datum v minulosti ({\em terminus post quem}), do nějž může být nápis dle editora datován. {\em End Year} udává minimální možné datum v minulosti ({\em terminus ante quem}), do nějž může být nápis dle editora datován. Editoři mohou používat dataci ohraničenou konkrétními roky, případně stoletími, ale mnohdy používají popisnou dataci, zařazení do období, bez jasného vymezení jeho hranic. Pokud není stanoveno jinak, držím se v těchto případech standardního členění dle řeckých dějin (archaická doba, klasická doba, hellénismus atp.). Pokud tedy editor datuje nápis do klasické doby, pak použitá datace odpovídá maximálnímu rozsahu standardní datace řeckých dějin (479 - 338 př. n. l.).\footnote{Přehledná chronologická tabulka 4.06 s použitou metodologií převodu různých forem datace do jednotných čísel je uvedena v Apendixu 1.}

Každý editor epigrafického korpusu přistupoval k dataci jiným způsobem: někdo se snažil datovat všechny nápisy, ač někdy v rozmezí na několik století, a jiný editor dával přednost datovat jen nápisy, u nichž byl schopen přiřadit konkrétní datum, a nejisté nápisy ponechat bez datace. Tím ale vznikla poměrně velká skupina nedatovaných nápisů, kterou se mi však povedlo aplikací následujících pravidel zmenšit zhruba o třetinu, a přiřadit nápisům alespoň relativní dataci. Položka {\em Relative Date} obsahuje relativní dataci založenou na interpretaci výskytu osobních jmen, na vizuální kontrole užitého písma a výskytu osobních jmen, či jiných termínů, díky nimž je možné nápis alespoň relativně datovat jako římský, tj. spadajících do 1. - 5. st. n. l. Určujícími prvky jsou:

\startitemize
\item
  \startblockquote
  Přítomnost jmen, které je možné označit jako římská, či ovlivněná římskými onomastickými zvyky, např. jména jako Aurelios, Gaios, Ailios, Klaudios apod.
  \stopblockquote
\item
  \startblockquote
  Formy písma typicky užívané v době římské, a to zejména se zaměřením na písmena omega, lunární sigma, či na přítomnost ligatur. Proměnu stylu jsem zaznamenala u nápisů datovaných do římské doby a zpětně jsem tyto nové poznatky zpětně aplikovala na nedatované nápisy, v nichž se však vyskytovala stejná forma písma.
  \stopblockquote
\stopitemize

Položka {\em Century} je pak pouhým výčtem všech století, do nichž byl nápis datovaný. Tato položka byla vytvořena z praktického důvodu snadnějšího vyhledávaní a filtrování nápisů datovaných do hledaných století.

\subsubsection[typologie-nápisu]{Typologie nápisu}

\startitemize
\item
  \startblockquote
  Položka {\em Dialect} zaznamenává použitý řecký dialekt, alespoň nakolik ho bylo možné rozlišit z textu nápisu, či ve výjimečných případech z epigrafického {\em lemmatu}.
  \stopblockquote
\item
  \startblockquote
  Položka {\em Latin} zaznamenává, zda se v textu nápis vyskytuje latinský text či nikoliv.
  \stopblockquote
\item
  \startblockquote
  Položka {\em Language Form} zaznamenává literární formu jazyka, v němž je převážná většina textu napsána, tj. zda je psána ve verši či v próze.
  \stopblockquote
\item
  \startblockquote
  Položka {\em Script} zaznamenává druh použitého písma. Písmo je interpretováno na základě studia vizuálního materiálu. Pokud vizuální materiál chybí, analýza nebyla provedena a položka zůstala nevyplněná, či přebírá informace poskytnuté editorem korpusu.
  \stopblockquote
\item
  \startblockquote
  Položka {\em Layout} zaznamenává rozvržení nápisu a jeho pravidelnost. Zejména se jedná o zachycení datovatelných forem rozmístění textu, jako je např. {\em bústrofédon} či {\em stoichédon} (dle McLean 2002).
  \stopblockquote
\item
  \startblockquote
  Položka {\em Document Typology} zaznamenává druh dokumentu na základě studia textu a základních charakteristik předmětu nesoucí nápis. Jedná se o nápisy veřejné, soukromé, či o nápisy neurčitelné. Položka {\em Public Documents} obsahuje základní typologii veřejných nápisů (dle McLean 2002). Položka {\em Private Documents} obsahuje základní typologii soukromých nápisů (dle McLean 2002). Položka {\em Document Typology Notes} zaznamenává poznámky k typologii nápisu a obecně k intepretacím založeným na čtení textu.
  \stopblockquote
\item
  \startblockquote
  Položka {\em Extent of Lines} zaznamenává maximální dochovaný počet řádků textu tak, jak je text publikován v epigrafickém korpusu.\footnote{V případě nápisů rozdělených do několika sloupců se počet řádků sčítá.}
  \stopblockquote
\stopitemize

\subsubsection[textová-analýza-jednotlivých-termínů]{Textová analýza jednotlivých termínů}

V této sekci se zaznamenává výskyt specifických termínů ({\em keywords}) v textu nápisu. Tyto termíny postihují základní struktury společenského uspořádání typického pro komplexní společnost, jako jsou názvy institucí, názvy funkcí státního aparátu, specializovaná povolání, názvy lokalit a aktivit spojených s fungováním komplexní společnosti apod. Při analýze nápisů mě zajímá nakolik se tyto termíny vyskytovaly v konkrétních společenských kontextech, zda je možné jejich výskyt spojit s určitou společenskou skupinou. Dále mě zajímá, zda se termíny postupem času rozšiřovaly jak místně, tak i v rámci hierarchie společnosti, a zda je možné sledovat posun v jejich použití a významu. U všech hledaných termínů se určuje i jejich společensko-kulturní původ, na kolik je možné ho stanovit.\footnote{Vzhledem k faktu, že se jedná o řecky psané nápisy, které se drží řeckých vzorů i v pozdějších dobách, pak i většina termínů nutně pochází původně z řeckého prostředí.}

Hledané termíny jsou rozděleny do několika tematických skupin:

\startitemize[n][stopper=.]
\item
  \startblockquote
  Termíny týkající se fungování komplexní společnosti, administrativy a aparátu, který celou společnost udržoval v chodu ({\em administrative keywords}). Termíny se dále dělí do několika podskupin dle svého zaměření na termíny spojené s určitou lokalitou, konkrétní funkcí, institucí, s finančním, vojenským či kulturním zaměřením termínu, společenským postavením a v neposlední řadě specializací povolání.\footnote{Jejich kompletní seznam je uveden v tabulce 4.07 v Apendixu 1.}
  \stopblockquote
\item
  \startblockquote
  Opakující se formulace typické pro řeckou epigrafickou produkci, jako jsou invokační formule, terminologie spojená s publikačními aktivitami státního aparátu, spojení slov typická pro dedikační či funerální nápisy, formulace zabývající se publikováním nápisu.\footnote{Jejich kompletní seznam je uveden v tabulce 4.08 v Apendixu 1.}
  \stopblockquote
\item
  \startblockquote
  Termíny týkající se honorifikačních aktivit, jako jednoho z typických projevů epigrafické aktivity v rámci řeckého kulturního prostředí, probíhající pod patronátem státního aparátu. Termíny jsou rozděleny do několika podkategorií, jako jsou důvody pro udělení pocty, konkrétní obsah poct, opatření zabývající se publikováním poct.\footnote{Jejich kompletní seznam je uveden v tabulce 4.09 v Apendixu 1.}
  \stopblockquote
\item
  \startblockquote
  Termíny týkající se náboženství a kultovních aktivit, jednotlivých božstev, dále míst a funkcí spojených s výkonem náboženských rituálů ({\em religious keywords}).\footnote{Jejich kompletní seznam je uveden v tabulce 4.10 v Apendixu 1.}
  \stopblockquote

  \startitemize[a]
  \item
    \startblockquote
    Jako zvláštní podskupinu jsem zaznamenávala jednotlivá božská epiteta tak, jak se vyskytovala v textu nápisů. Spadají sem i pravopisné varianty jmen, u nichž rozlišuji epiteta vázající se k místním božstvům a božstvům všeobecně rozšířeným.\footnote{Jejich kompletní seznam je uveden v tabulce 4.11 v Apendixu 1.}
    \stopblockquote
  \stopitemize
\stopitemize

\subsubsection[identita-a-identifikace]{Identita a identifikace}

V této části se snažím určit společensko-kulturní kontext, v němž nápis vznikl. Na základě analýzy textu nápisu zaznamenávám, jak se lidé prezentovali na nápisech a pod jakou identitou se rozhodli vystupovat, či se k ní nějakým způsobem vyjádřit.\footnote{Podrobnější teoretický úvod k prezentaci identity na nápisech se nachází v kapitole 3.} Rozlišuji několik stupňů prezentace identity, a to jsou osobní jména a vazby na nejbližší okolí, jako např. rodinu, předky, partnery, přátele. Dále mě zajímají vazby na skupiny lidí a komunity, a společensko-politické jednotky jako jsou vesnice, města, regiony, státy apod.

\startitemize
\item
  \startblockquote
  V sekci {\em Names} zaznamenávám všechny osoby, které je možné identifikovat dle osobního jména, či jejich kombinací (osobní jméno + {\em patronymikum}, jméno matky, partnera, předků, jména potomků, sourozenců apod.). U jednotlivých jmen zaznamenávám i jejich původ, tedy kontext, v němž se jména vyskytovala původně a nejčastěji, a tradičně bývají považována za řecká, thrácká, římská či jiného původu/není možné jejich původ určit. Dále rozlišuji, zda se jedná o jméno ženské, či mužské, či zda není možné jeho příslušnost určit dle dochovaných textů.
  \stopblockquote
\item
  \startblockquote
  V rámci {\em Collective Group Names} zaznamenávám jména popisující skupinu lidí. Může se jednat jak o skupinu lidí založenou na etnickém principu, společném zájmu, či společensko-politické příslušnosti, či společném náboženském přesvědčení).\footnote{Jejich kompletní seznam je uveden v tabulce 4.12 v Apendixu 1.}
  \stopblockquote
\item
  \startblockquote
  V sekci {\em Geographic Names} zaznamenávám místní jména, tak, jak jsou uvedené v textu nápisu (osídlení, regiony, pohoří, řeky apod.).\footnote{Jejich kompletní seznam je uveden v tabulce 4.13 v Apendixu 1.}
  \stopblockquote
\stopitemize

\subsubsection[text-nápisu]{Text nápisu}

Text nápisu získávám z online databáze {\em Searchable Greek Inscriptions}\footnote{\useURL[url5][http://noapplet.epigraphy.packhum.org/][][{\em http://noapplet.epigraphy.packhum.org/}]\from[url5], 20. září 2016}, {\em Supplementum Epigraphicum Graecum}\footnote{\useURL[url6][http://referenceworks.brillonline.com/browse/supplementum-epigraphicum-graecum][][{\em http://referenceworks.brillonline.com/browse/supplementum-epigraphicum-graecum}]\from[url6], navštíveno 20. září 2016}, {\em Epigraphic Database Heidelberg}\footnote{\useURL[url7][http://edh-www.adw.uni-heidelberg.de/home][][{\em edh-www.adw.uni-heidelberg.de/home}]\from[url7], navštíveno 20. září 2016}, případně manuálním přepisem textů epigrafických korpusů. Heurist uchovává text ve formátu WYSIWYG\footnote{WYSIWYG je akronym anglické věty „{\em What you see is what you get”}, česky „co vidíš, to dostaneš”. Tato zkratka označuje způsob záznamu dokumentů v počítači, při kterém je verze zobrazená na obrazovce vzhledově totožná s výslednou verzí dokumentu, která je uložena v databázi.}, který se nejvíce podobá výsledné formě nápisu. Text užívám pro vnitřní potřebu projektu, jehož cílem není provádět nové edice nápisů, ale již vydané nápisy systematicky zpracovat a vzájemně porovnat zdroje, které nebylo pro svou nekompatibilnost nemožné konzistentně srovnávat. Texty samotné slouží jako doprovodné informace, vlastní data určená k analýze, extrahovaná z textů, jsou obsažena ve strukturované podobě databáze samotné.

\subsubsection[přílohy]{Přílohy}

V sekci {\em Attachments} se nacházejí obrazové přílohy tak, jak byly otištěny v jednotlivých korpusech. Tato sekce slouží čistě pro interní potřebu projektu a není zcela kompletní.

\section[technické-řešení-qgis-a-geografická-data]{Technické řešení: QGIS a geografická data}

Geografická data získaná exportem z Heuristu mají formát jednak KML, ale i CSV se zeměpisnými koordináty, které následně převádím na formát {\em shapefile}. Pro zpracování dat a následnou vizualizaci formou map používám {\em open source} program QGIS, verze 2.8 Wien.\footnote{Více informací o softwaru QGIS a verzi 2.8, oficiální stránka společnosti \useURL[url8][http://www.qgis.org/en/site/index.html][][{\em http://www.qgis.org/en/site/index.html}]\from[url8] (26. září 2016).} QGIS je geoinformatický software určený ke zpracování, úpravám, analýze a přehledné prezentaci zeměpisných dat. Hlavní výhodou programu QGIS je jeho volná dostupnost a relativně malá hardwarová náročnost. Výsledným produktem práce s programem QGIS jsou analýzy v kapitole 7 a mapy obsažené v této práci.

Zdigitalizovaná data o poloze a rozmístění nápisů uchovávám a analyzuji ve formátu CSV, KML a {\em shapefile}. Jednotlivé nápisy obsahují informace o zeměpisné šířce a délce nejpravděpodobnějšího místa jejich nálezu. Dále nápisy obsahují informaci o přesnosti daného místa nálezu (tedy koeficient přesnosti místa nálezu, {\em Position certainty}, viz výše), který určuje, jak velký okruh v řádech kilometrů od daného geografického bodu připadá na nejpravděpodobnější místo nálezu daného nápisu (tzv. {\em buffer zone}). Dále každý záznam obsahuje kompletní tabulární data, dle nichž je možné nápisy filtrovat, analyzovat a zobrazovat na mapě.

Další geografická data, která pro projekt používám, jsou digitalizovaná data z {\em Barrington Atlas of the Greek and Roman World} (Talbert 2000), částečně upravená data získaná z projektu {\em Pleiades}\footnote{\useURL[url9][https://pleiades.stoa.org/home][][{\em https://pleiades.stoa.org/home}]\from[url9] (26. září 2016)}, data získaná z projektu {\em Tundzha Regional Archaeological Project} (Ross {\em et al.}, v přípravě, vyjde 2017), data získaná z projektu {\em Burial Mound}s (Weissova 2013; 2016), data z projektu {\em Stroyno Archeological Project} (Tušlová {\em et al.} 2015; 2016) a satelitní snímky poskytnuté nadací {\em Digital Globe Foundation}. Jako výchozí geografická data a podkladové vrstvy používám volně dostupné mapy {\em OpenStreetMap}\footnote{\useURL[url10][https://wiki.openstreetmap.org/wiki/Cs:Main_Page][][{\em https://wiki.openstreetmap.org/wiki/Cs:Main_Page}]\from[url10] (26. září 2016)}, data získaná v průběhu mapování vodních zdrojů Bulharska společností {\em Japan International Cooperation Agency}\footnote{\useURL[url11][http://www.jica.go.jp/english/our_work/social_environmental/archive/pro_asia/bulgaria_1.html][][{\em http://www.jica.go.jp/english/our_work/social_environmental/archive/pro_asia/bulgaria_1.html}]\from[url11] a \useURL[url12][http://open_jicareport.jica.go.jp/pdf/11878667_02.pdf][][{\em http://open_jicareport.jica.go.jp/pdf/11878667_02.pdf}]\from[url12] (26. září 2016)}, data získaná digitalizací topografických map Bulharska s měřítkem 1:50 000 a 1:100 000, data ve formátu {\em shapefile} obsahující administrativní hranice moderních států a silniční sítě\footnote{\useURL[url13][http://www.vdstech.com/osm-data.aspx][][{\em http://www.vdstech.com/osm-data.aspx}]\from[url13] (26. září 2016)} a další volně dostupná vektorová data.\footnote{\useURL[url14][http://www.naturalearthdata.com/][][{\em http://www.naturalearthdata.com/}]\from[url14] (26. září 2016)}

\section[technické-řešení-r-a-tabulární-data]{Technické řešení: R a tabulární data}

Tabulární data o jednotlivých nápisech získaná ve formátu CSV zpracovávám a analyzuji v statistickém programu R verze 3.3.1 pro Windows OS a R Studio verze 0.99.903.\footnote{\useURL[url15][https://www.r-project.org/][][{\em https://www.r-project.org/}]\from[url15] a \useURL[url16][http://www.r-project.cz/about.html][][{\em http://www.r-project.cz/about.html}]\from[url16] (26. září 2016). Za tabulární data považuji strukturovaná data získaná exportem z databáze v Heuristu, ve formátu CSV. Může se jednak jak o numerická data, zeměpisné koordináty, či o text. Data pocházejí z jednotlivých epigrafických korpusů, či se jedná o interpretace odvozené z dat publikovaných v daných korpusech (relativní datace, relativní poloha místa nálezu, vizuální podoba předmětu atp.).} R je volně dostupný program, který se specializuje na statistické zpracování dat a jejich prezentaci v podobě grafů a diagramů. Výsledným produktem jsou statistické údaje grafy obsažené v této práci a skripty jednotlivých analýz, které jsou součástí digitálního Apendixu.\footnote{https://github.com/petrajanouchova/hat_project.}

\section[statistické-zpracování-epigrafických-dat]{Statistické zpracování epigrafických dat}

Statistické zpracování většího počtu nápisů v sobě nese celou řadu metodologických problémů. Epigrafická data jsou svou podstatou nedokonalá, často nekoherentní a velmi specifická. Z tohoto důvodu dochází ke statistickému zpracování nápisů relativně málo často, a to zejména v římském období (MacMullen 1982; Meyer 1990; Saller a Shaw 1984; Prag 2002; Prag 2013; Korhonen 2011). Nápisy představují nejen výborný sociologický a antropologický materiál pro studium antické společnosti, ale pokud se k nim přistupuje komplexně, mohou odhalit nové pohledy na tehdejší společnost, které by bylo jen velmi obtížné získat studiem jednotlivých nápisů. I přes tento neoddiskutovatelný potenciál statistického zpracování nápisů existuje velké množství přístupů, které nereflektují problematickou a často nekoherentní povahu informací získaných z nápisů, a může tak docházet k jejich zkreslení, či nepochopení. V následující pasáži se nejdříve věnuji míře nejistoty datace, která bývá často opomíjená v rámci epigrafických studií, a dále metodě výběru statisticky signifikantního souboru nápisů vhodných k temporální analýze celospolečenských trendů. V neposlední řadě poukazuji na problém nadhodnocování celkových čísel epigrafické produkce při statistického zpracování nápisů v jednotlivých stoletích a řešení za využití metody normalizované datace.

\subsection[pravděpodobnost-a-míra-nejistoty-datace-nápisů]{Pravděpodobnost a míra nejistoty datace nápisů}

Jednotlivé nápisy bývají datovány na konkrétní roky poměrně zřídka, ale spíše častěji bývají datovány do několika století či spadají pouze do časového období přesahující několik staletí, jako např. hellénismus, klasická doba, římská doba atp. Tento nejednotný styl datování, který se různí i dle osobního přístupu editorů epigrafických korpusů, velmi znesnadňuje jakékoliv statistické výpočty a srovnávání epigrafické produkce v rámci několika století. V rámci hledání řešení této nelehké situace jsem se inspirovala v oblasti archeologických polních sběrů, nabízejících možná metodologická řešení míry nejistoty datace.

Archeologové se potýkají s nutností analyzovat často velmi nekvalitní, a tedy i nejistě datovatelnou keramiku, která by jim pomohla datovat lokality nalezené v průběhu polních sběrů. Jednotlivé lokality bývají datovány na základě studia nalezené keramiky. Ta je zařazena do chronologických skupin dle výskytu charakteristických prvků a použitých technologií výroby a zpracování. Na základě mnohdy nejisté datace se snaží zhodnotit vzájemnou koexistenci lokalit a kontinuitu trvání v rámci chronologických období (Dewar 1991; Bevan a Conolly 2006; Crema 2011; Bevan {\em et al.} 2013; Sobotková 2017). V případě, že je keramika nekvalitní, nediagnostická, či špatně dochovaná, a tedy charakteristické prvky neumožňují jasnou dataci, archeologové přišli s řešením této nejistoty datace. Při statistickém zpracování přiřazují jednotlivým střepům procentuální vyjádření pravděpodobnosti dané datace, tedy jakýsi koeficient pravděpodobnosti přináležitosti do dané chronologické skupiny (např. Bevan {\em et al.} 2013; Crema 2011). Na základě analýzy pravděpodobnosti datace keramických střepů a zohlednění těchto výsledků pak mohou vypočítat přibližnou dobu trvání a celkový počet lokalit v daném období. Tato v mnohém inovativní metoda umožňuje zahrnout materiál, který by byl dříve z celkové analýzy vyloučen pro svou nejednoznačnost a mohlo by docházet ke zkreslení interpretací. Tím, že jsou jednotlivá chronologická období vnímána jako na sebe navazující kontinuum, a nikoliv striktně oddělené kategorie, je prvek nejednoznačnosti a nejistoty datace brán jako jedno z hodnotících kritérií. Tímto způsobem je možné daný předmět analyzovat komplexněji a sledovat provázanost mezi jednotlivými chronologickými kategoriemi.

Studie zabývající se statistickým zpracováním nápisů v dlouhém časovém intervalu jsou poměrně málo časté, vzhledem k nedostatku relevantních korpusů a problematické povaze datace nápisů. Při statistickém zpracování nápisů totiž nevyhnutelně dochází k určité redukci intervalu datace, což je nezbytné vzhledem k nutnosti zpracování velkého počtu dat a jejich prezentace v jednotném a koherentním formátu. Na rozdíl od archeologické aplikace pravděpodobnosti a temporální nejistoty, která především řeší dobu trvání jednotlivých lokalit a vztahy s ostatními lokalitami datovanými do téže doby, použití této metody v epigrafice se v několika zásadních hlediscích odlišuje. Jednak tím, že datace nápisu totiž udává nejpravděpodobnější interval, v němž nápis vznikl, a neudává dobu existence nápisu (která by ve valné většině případů trvala až do současnosti). Nápis tak má, na rozdíl od archeologických lokalit, výpovědní hodnotu vztahující se k době vzniku, nikoliv však k době své existence, a proto se částečně liší i přístup k nejistotě jeho datace. Dalším zásadním rozdílem je, že v archeologickém použití této metody jde především o odhad počtu současně existujících lokalit, avšak už se většinou dále nepracuje na úrovni jednotlivých lokalit a s nimi spojených kvalitativních prvků, jako je tomu u studia nápisů (Dewar 1991, 604). Nápisy navíc nebývají v epigrafických korpusech datovány s procentuálním vyjádřením pravděpodobnosti datace do jednotlivých století, ale interval datace má stejnou pravděpodobnost po celou dobu trvání, ať už se jedná o rok, deset let, či pět století. Povaha samotných dat tedy neumožňuje použít metodu ve formě jakou navrhuje např. Bevan či Crema, ale je nutné ji upravit, a částečně zjednodušit tak, aby odpovídala povaze epigrafických dat a metodologii datování nápisů.

Metoda definování pravděpodobnosti datace upravená pro epigrafické prostředí umožňuje nápisy rozdělit do jednotlivých časových období s přihlédnutím k možné variabilitě a poměrně širokému rozptylu jejich intervalu datace. Jinými slovy, metoda zohledňuje míru nejistoty datace a pravděpodobnost doby vzniku nápisu v rámci jednotlivých století za účelem vytvoření nejrelevantnějšího souboru nápisů pro studium konkrétních trendů v rámci daného století.\footnote{Tato metoda vznikala nezávisle na statistických studiích zabývajících se epigrafickými prameny z jiných regionů (Prag 2002; Prag 2013; Korhonen 2011), avšak základní principy a řešení nejistoty datace se do velké míry shodují.} V oblasti epigrafiky podobnou metodologii uplatňuje např. Jonathan Prag (2002; 2013) u statistických studií nápisů nalezených na Sicílii, kde rovněž postihuje období delší než deset století. Tato metoda umožňuje zařadit nápisy do kategorie jednotlivých století s přihlédnutím k celkovému intervalu doby jejich vzniku ({\em terminus post quem} a {\em terminus ante quem}). Dle dostupných údajů o dataci jsem vypočetla tzv. koeficient pravděpodobnosti datace, který zaznamenává míru pravděpodobnosti, s níž daný nápis vznikl v průběhu daného století.\footnote{Pro lepší pochopení udám konkrétní příklad: nápis je datován do 2.-3. st. n. l., což znamená, že v rámci statistik jednotlivých století by byl započítán jak v rámci druhého, tak v rámci třetího století se stejnou pravděpodobnosti 1. Nápis datovaný do 2. a 3. st. n. l. by měl stejný koeficient pravděpodobnosti pro 2. st. jako např. nápis datovaný pouze do 2. st. n. l., což jsem chtěla odlišit. U nápisů datovaných do více století jsem tedy plnou hodnotu pravděpodobnosti 1 (100 \letterpercent{}) vydělila počtem století, do nichž byly nápisy datovány a vyšel mi koeficient pravděpodobnosti datace u jednoho století. U nápisů datovaných do dvou století byl tak koeficient 0,5 pro každé z obou století, u nápisů datovaných do tří století byl koeficient 0,33 pro každé ze třech století. U nápisů datovaných do čtyř století byl koeficient 0,25, atd.} Tímto způsobem jsem rozlišila nápisy, které jsou datovány do jednoho století, a obsahují tak statisticky přesnější data o trendech jednotlivých století, od nápisů jejichž datace přesahuje do několika století, u nichž je tak těžší sledovat vývoj měnících se parametrů v čase, a jejichž výpovědní hodnota ke konkrétnímu časovému okamžiku je relativně nízká.

Průměrná délka časového intervalu datace všech datovaných nápisů je 114,6 let (aritmetický průměr); medián je 99 let, což přibližně odpovídá dělení na století, které jsem zvolila jako základní chronologické dělení pro další statistické analýzy.\footnote{Použitý R skript je součástí digitálního Apendixu, dostupného na adrese \useURL[url17][https://github.com/petrajanouchova/hat_project][][{\em https://github.com/petrajanouchova/hat_project}]\from[url17] (date_testing_hypothesis.R). Aritmetický průměr (mean) je součtem všech hodnot, vydělený celkovým počtem prvků. Medián udává střední hodnotu souboru vzestupně seřazených hodnot a dělí tak soubor na dvě stejně početné poloviny. Hodnota mediánu není zpravidla ovlivněna extrémními hodnotami, např. velmi se odlišujícími maximálními či minimálními hodnotami, na rozdíl od aritmetického průměru, a lépe tak poukazuje na střední hodnoty daného souboru. Pro srovnání udávám vždy aritmetický průměr a medián daného souboru.} Z grafu 4.05 v Apendixu 2 můžeme vidět jasné tendence přiřazování datace napříč všemi datovanými nápisy (2276 nápisů): a) zařadit nápis co nejpřesněji, v rámci intervalu menšího než 20 let, b) s přesností na jedno století, tedy intervalu menšímu či rovnajícímu se 100 letům, c) s přesností na dvě století, tedy intervalu menšímu či rovnajícímu se 200 letům. Omezený počet nápisů je datován s přesností na tři století, případně na pět, avšak tato datace je již příliš široká pro studii vývoje v rámci jednotlivých století. Procento všech datovaných nápisů, jejichž interval datace, tedy období vzniku nápisu ohraničené StartYR a EndYR v databázi, byla pro interval 1-50 let 35,15 \letterpercent{}, pro interval 51-100 let 30,13 \letterpercent{}, a pro interval 101-704 let 34,63 \letterpercent{}. Pro účely temporální studie v kapitole 6 jsem vybrala pouze nápisy, které jsou datovány v rozmezí jednoho století (jimž jsem přidělila koeficient 1), a nápisy datovány do rozmezí dvou století (koeficient 0,5).\footnote{Nápisy, které se podařilo datovat jako relativně římské, do analýzy v kapitole 6 nezařazuji, zejména vzhledem k jejich vágní dataci, a tedy malé výpovědní hodnotě v rámci temporální studie. Stejně tak nápisy, jejichž datace přesahuje dvě století.} Tyto dvě skupiny dohromady reprezentují 89,5 \letterpercent{} (2036 nápisů) všech datovaných nápisů, které však zcela neodpovídá kategoriím jednotlivých staletí, ale navzájem se částečně překrývají, a proto o nich pojednávám v kapitole 6 jako o navzájem oddělených kategoriích.

Pokud však chceme srovnat celkovou míru epigrafické produkce v jednotlivých staletích, je nutné přistoupit k metodě normalizované datace, která zobrazuje nejpravděpodobnější procentuální zastoupení epigrafické produkce v daném století. Graf 4.06 v Apendixu 2 ilustruje poměr datovaných nápisů v jednotlivých stoletích s přihlédnutím k šíři jejich datace (vyjádřenou pomocí jejich koeficientů 1 - 0,125). Ve 4. st. př. n. l. představují nápisy s koeficientem 1 až 0,5 téměř 95 \letterpercent{} všech datovaných nápisů, což značí velmi malou míru nejistoty přesné datace nápisů. Naopak ve 4. st. n. l. se jedná pouze o 70 \letterpercent{}, což značí relativně velkou míru nejistoty datace nápisů přiřazených do daného století. Důvodem může být nízký počet charakteristických datačních prvků, s nímž se potýkali autoři epigrafických korpusů v kombinaci se všeobecným poklesem publikační aktivity ve 4. st. n. l. Medián, tedy střední hodnota pro všechna sledovaná století u skupiny nápisů s koeficientem 1 až 0,5 je 85 \letterpercent{}. Celkem v osmi z jedenácti sledovaných časových období je jejich poměr vyšší než 85 \letterpercent{}. Korpus nápisů datovaných do rozmezí dvou století, tedy nápisy s koeficientem 1 a 0,5, představuje statisticky dostatečně signifikantní vzorek pro všechna měřená století a jeho analýza umožňuje sledovat měnící se celospolečenské trendy s přesností na 100, respektive 200 let. Vzhledem k tomu, že epigrafická produkce je obecně svou povahou poměrně konzervativní, společensko-kulturní změny se na formě a obsahu nápisů mohou objevit ve větším měřítku až se zpožděním několika desetiletí. Proto jsem zvolila časový interval jednoho století jako výchozí jednotku pro sledování vývoje a proměn epigrafické produkce v čase.

\subsubsection[problém-nadhodnocování-epigrafické-produkce]{Problém nadhodnocování epigrafické produkce}

V rámci analýzy celkové epigrafické produkce v jednotlivých století jsem řešila problém nadhodnocování celkového čísla datovaných nápisů do konkrétního století. Každý nápis, který byl totiž datován do více než jednoho století, může být započítán pro každé z těchto století se stejnou váhou, a může tak dojít ke zkreslení epigrafické produkce směrem nahoru. Společnost v Thrákii by se tak mohla zdát více epigraficky aktivní než jak tomu celkový počet dochovaných nápisů, protože součet počtu nápisů pro jednotlivá století byl vyšší než počet dochovaných nápisů. Jako možné řešení této situace, které by lépe reflektovalo poměrné rozložení nápisů v rámci jednotlivých století, jsem se rozhodla uplatnit koeficient pravděpodobnosti datace k získání celkového počtu nápisů datovaných do daného století. Namísto abych započítávala každý nápis s hodnotou 1 pro všechna století, kam byly nápisy datovány, jsem započítávala pouze hodnotu jejich koeficientu pravděpodobnosti, a součtem koeficientů všech nápisů jsem došla ke konečnému číslu epigrafické produkce v daném století, které lépe reflektuje poměrné rozložení nápisů a nejistotu jejich datace (normalizovaná datace).

Graf 4.07 v Apendixu 2 nabízí porovnání obou metod uplatněných na soubor datovaných nápisů z Thrákie. Výsledná linie představuje celkový počet datovaných nápisů v rámci jednotlivých století v závislosti na použité metodě (normalizovaná datace versus nenormalizovaná datace).\footnote{Výsledné číslo datace v daném století je v případě použití normalizované metody součtem koeficientů pravděpodobnosti všech nápisů. U nenormalizované datace každý nápis má hodnotu 1 pro každé století, do nějž je nápis datován (konečný součet je číslo 1 u nápisů datovaných do jednoho století, či násobky čísla jedna u nápisů datovaných do více století).} Ze srovnání obou křivek je patrné, že trendy měnícího se počtu nápisů zůstávají stejné, v závislosti na uplatnění koeficientu pravděpodobnosti datace se však mění celková čísla nápisů. Při uplatnění nenormalizované datace tak dochází k nárůstu počtu nápisů, který však neodpovídá počtu reálně existujících nápisů, viz výše. Rozdíl mezi výslednými čísly může být poměrně dramatický zejména ve stoletích, kde byly nápisy datované šířeji, například ve 2. st. n. l. tento rozdíl činí až 238 nápisů.\footnote{Obě dvě křivky zaznamenávají podobný vývoj epigrafické produkce, až na nápisy datované do 1. st. n. l., kde se trendy rozcházejí. Tento fakt může poukazovat na určitou nejistotu datace nápisů právě do zvoleného období, které je označováno jako období nástupu římské moci a transformace oblasti do římské provincie {\em Thracia}. Celkově však nenormalizovaná datace nápisů ukazuje obecně vyšší číslo nápisů, a to až o 60 \letterpercent{}, tj. o 1368 nápisů, než odpovídá skutečnosti, tj. 2276 oproti 3644 nápisům. Dochází tak ke zkreslení povahy epigrafické produkce v Thrákii, které může vést k nadhodnocování určitých trendů, zejména ve 2. a 3. st. n. l, kdy je nárůst celkového počtu nápisů markantní. Z tohoto důvodu jsem se rozhodla používat metodu normalizované datace, pokud hovořím o celkových trendech v rámci datovaných nápisů.} V kapitole věnované chronologickému přehledu nápisů dále pojednávám zvlášť o nápisech datovaných s přesností do jednoho a do dvou staletí, aby nedocházelo k nárůstu celkového počtu nápisů a zároveň i inflaci epigrafické produkce v daném časovém období.

\subsection[nejistota-rozmístění-nápisů-v-krajině]{Nejistota rozmístění nápisů v krajině}

Dalším problémem, s nímž bylo nutné se vypořádat, je nejistota spojená s určením místa nálezu nápisu. Ve velkém množství případů epigrafické korpusy udávají pouze přibližnou polohu, kde byl nápis nalezen, a není ho tak možné spojovat s konkrétní archeologickou lokalitou. Pro provedení analýzy rozmístění nápisů, která bere v potaz vzájemné prostorové vztahy a zasazení epigrafické produkce do kontextu zeměpisných podmínek, bylo nutné se vypořádat s mírou nejistoty při určování nejpravděpodobnějšího místa nálezu nápisu.

Za tímto účelem jsem vytvořila tzv. {\em position certainty index}, tedy koeficient přesnosti určení míst nálezu, který určuje velikost území, na němž byl nápis nejpravděpodobněji nalezen.\footnote{Tabulka 4.04 v Apendixu 1 přehledně shrnuje hodnoty koeficientu a celkového počtu nápisů jednotlivých skupin.} Při stanovování velikosti území jsem vycházela z informací poskytovaných editory jednotlivých autorů, zejména pak {\em IG Bulg}. Pokud bylo možné místo spojit s konkrétní archeologickou lokalitou, přesnost míry určení místa jsem stanovila ve vzdálenosti do 1 km, nesoucí koeficient 1.\footnote{Příkladem udávání místa nálezu, které bylo možno spojit s konkrétní archeologickou lokalitou je např. {\em IG Bulg} 118: „{\em Odessi reperta in via Prespa}.”; {\em I Aeg Thrace} 195: „Προέρχεται άπο τήν θέση Μάρμαρα της αρχαίας Μαρώνειας.”} Pokud autor korpusu udal místo nálezu v okolí moderního sídla s udáním konkrétních vzdáleností a směru, kde byl nápis nalezen, místo nálezu se s největší pravděpodobností nacházelo v okruhu 5 km od tohoto moderního sídla.\footnote{Příkladem relativně přesného místa nálezu je např. {\em IG Bulg} 3,2 1843: „{\em Repertus in agro quodam ad vicum nunc Dobrinovo, olim Hasbeglij dicto}.”; {\em IG Bulg} 4 2014: „{\em Reperta 1 km orientem versus a vico Gurmazovo, conservabatur penes vicanum eiusdem vici Pane Gjorev}.”} Pokud editor korpusu udal pouze všeobecnou informaci o nálezovém místě a jeho poloze vůči moderním sídlům, oblast nejpravděpodobnějšího místa nálezu byla stanovena do vzdálenosti 20 km od uvedeného moderního sídla.\footnote{Příkladem obecného udávání místa nálezu je např. {\em IG Bulg 4} 2034: „{\em Reperta ad vicum Dragoman}.”; {\em IK Byzantion} 159: „{\em Gefunden in Istanbul.}”} Skupina nápisů, jejichž místo nálezu je známé jen velmi obecně, či vůbec, ale editoři nápisu udávají, že pochází z Thrákie, nese koeficient 4, což značí velmi malou míru pravděpodobnosti konkrétního zeměpisného určení.\footnote{Příkladem nápisu s neznámou lokalitou je {\em IG Bulg} 5 5927: „{\em Thrace, région indéterminée}.”} Nápisy z této poslední skupiny vynechávám ze všech analýz rozmístění v kapitole 7, vzhledem k velmi malé výpovědní hodnotě o místě nálezu a prostorovém uspořádání těchto nápisů v krajině.

\section[shrnutí-1]{Shrnutí}

Současná epigrafika a s ní související metody výzkumu antické společnosti procházejí v posledních letech prudkým vývojem. Původně velmi konzervativní disciplína v současnosti využívá mnoho moderních přístupů ze sousedních disciplín, jako je archeologie, antropologie či sociologie a nově vzniklý obor tzv. {\em digital humanities}.

V této práci vycházím z tradičních principů epigrafické práce za využití potenciálu, jaký nám nabízí moderní technologie a digitální zpracování dat. Nesporným přínosem je možnost zpracovávat velká množství dat a navzájem je propojovat tak, že vznikají nové úhly pohledu a srovnání nejen jednotlivých regionů mezi sebou, ale i porovnání vývoje v různých časových obdobích.

Metodologie zkoumání společnosti na základě nápisů je do velké míry neprobádaným územím s velkým potenciálem do budoucnosti. Představované metodologické postupy jsou snahou vyrovnat se se specifiky epigrafického materiálu za zachování základních principů současné vědecké práce, jako je princip testovatelnosti analýz a ověřitelnosti jejich výsledků. Zcela v souladu s požadavky současného výzkumu jsou výsledná data dostupná komukoliv k ověření či vyvrácení hypotéz, případné modifikaci a pokračování výzkumu z jiné perspektivy.

\chapter{Epigrafická produkce v Thrákii}
V této kapitole charakterizuji dochovaný epigrafický materiál vzhledem k jeho rozmístění v Thrákii, fyzické podobě objektu nesoucí nápis a typovému zařazení a funkci nápisů. Dále se věnuji jednotlivým aspektům epigrafické produkce v souvislosti s prezentací identity osob, jako je třeba volba publikačního jazyka, či proměny onomastických zvyků. Výchozím souborem je 4665 nápisů, které jsem shromáždila v rámci databáze {\em Hellenization of Ancient Thrace} (HAT), zahrnující materiál celkem z 11 století, konkrétně od 6. st. př. n. l. do 5. st. n. l. (Janouchová 2014).\footnote{Digitální apendix, spolu s kompletním obsahem databáze je volně dostupný na GitHubu na adrese \useURL[url18][https://github.com/petrajanouchova/hat_project][][{\em https://github.com/petrajanouchova/hat_project}]\from[url18], stav k 9. listopadu 2016. Tabulka 5.00 v Apendixu 1 shrnuje celkový počet záznamů v databázi rozdělený dle jednotlivých druhů záznamů.}

\section[přehled-analyzovaného-materiálu]{Přehled analyzovaného materiálu}

Soubor nápisů přestavuje jedinečný komparativní soubor, reflektující společensko-kulturní změny a vývoj společnosti antické Thrákie. Databáze HAT tak krom informací o nápisech samých obsahuje i data o 5464 epigrafických osobách, které figurují v textech nápisů, jejichž jméno a identita se skládají z celkem 3956 osobních jmen a 178 termínů vyjádření kolektivní identity. Dále databáze obsahuje informace o 678 místech, kde byly nápisy nalezeny, ať už na území Thrákie či z nejbližšího okolí. V neposlední řadě se zde nacházejí data vztahující se k obsahu nápisů: celkem databáze zaznamenává 134 zeměpisných jmen, 233 epitet jednotlivých božstev, 137 termínů z náboženské oblasti, 112 termínů z oblasti organizace společnosti, 49 termínů vztahujících se k epigrafickým zvyklostem, a 37 termínů vztahující se k udílení poct v rámci epigrafické kultury.\footnote{Podrobný komentář a vysvětlení termínů se nachází v kapitole 4, věnované metodologii práce.}

\subsection[zeměpisné-uspořádání-nápisů]{Zeměpisné uspořádání nápisů}

Nápisy pocházejí z oblasti jihovýchodní části dnešního Balkánského poloostrova, a to konkrétně z území dnešních států Bulharska, Řecka a Turecka. Velká část nápisů byla nalezena, či editory zařazena do oblastí ve vzdálenosti do 20 km od mořského pobřeží, a dále z vnitrozemských městských center, která se nacházejí poblíž toků řek Hebros, Strýmón a Tonzos, případně v blízkosti hlavních cest v nížinatých oblastech. Z hornatých oblastí vnitrozemské Thrákie pochází jen menší část nápisů, a to zejména z oblasti pohoří Stara Planina, Rodopy, Pirin a Rila.\footnote{Více o zeměpisném rozložení nápisů hovořím v kapitole 7.}

Dochované nápisy jsou do značné míry výsledkem náhodných nálezů v průběhu posledních dvou století a systematického archeologického výzkumu zhruba od poloviny 20. století. Archeologické výzkumy neprobíhaly na všech místech stejnou měrou a ve stejném rozsahu, a proto i počty dochovaných nápisů mnohdy odpovídají stavu archeologického výzkumu na území Thrákie. Příkladem může být dobrý stav poznání v černomořských řeckých koloniích, které bylo možné prozkoumat velmi detailně v posledních 50 letech zejména díky prudkému nárůstu turistického ruchu a s ním spojených záchranných výzkumů (Velkov 1969; Ognenova-Marinova {\em et al.} 2005; Baralis a Panayotova 2015). Opačným příkladem je oblast turecké Thrákie, která je prozkoumána jen velmi málo. Vzhledem k složité politické a ekonomické situaci jsou data z části evropského Turecka nedostupná, až na oblast okolo Perinthu, Byzantia a řeckých měst na Thráckém Chersonésu, které se podařilo získat díky výzkumu a spolupráci německých a tureckých vědců (Krauss 1980; Lajtar 2000; Sayar 1998). Ucelená publikace nápisů z egejské Thrákie je taktéž poměrně nedávnou záležitostí (Loukopoulou {\em et a}l. 2005), avšak zde jsou nálezy nápisů doplněny dlouhodobými archeologickými výzkumy.

Mapa 5.01 v Apendixu 2 vyznačuje oranžovou barvou oblast, kterou pokrývají v databázi zpracované nápisy. Naopak místa bílá jsou zároveň místy, pro něž nejsou dostupné systematicky zpracované korpusy nápisů. Tato oblast bez nápisů se vyznačuje poměrně nepřístupným terénem a pravděpodobně byla málo zalidněná jak v antice, tak dnes. Obecně se předpokládá, že z této části Thrákie pochází jen velmi malý počet nápisů, které by tak výrazněji neměly zasáhnout do celkového obrazu epigraficky aktivní společnosti antické Thrákie, pokud dojde v budoucnu k jejich publikování.

Nápisy se podařilo lokalizovat dle dat dostupných v epigrafických korpusech, a to s následující přesností: do 1 km 1540 nápisů (33 \letterpercent{})\footnote{Následující statistiky vycházejí z celkového počtu 4665 nápisů, což představuje 100 \letterpercent{} analyzovaného souboru. V určitých případech může dojít při konečném součtu všech položek k hodnotě větší či menší než 100 \letterpercent{}. Pokud je součet větší než 100 \letterpercent{}, daná položka mohla mít více hodnot, tj. nápis mohl být určen například jako funerální a honorifikační zároveň, pokud svými charakteristickými rysy spadal do obou kategorií, V tomto případě pak konečný součet má hodnotu více jak 100 \letterpercent{}, protože položky obsahují více než jednu hodnotu. V případě statistik, kde je konečná hodnota menší než 100 \letterpercent{}, pak tento rozdíl náleží položkám, které se nepodařilo určit, jak z důvodu jejich fragmentárnosti, nedostupnosti informace apod.}, s přesností do 5 km 2323 nápisů (50 \letterpercent{}), s přesností do 20 km 590 nápisů (13 \letterpercent{}), a s přesností nad 20 km 212 nápisů (4 \letterpercent{}). Udávaná poloha většiny nápisů byla editory označena jako místo primárního nálezu (4305 nápisů, tedy 92 \letterpercent{}), tedy místo, kde byl nápis umístěn, když sloužil své primární funkci. Archeologický kontext místa, v němž byl nápis nalezen, udávají autoři korpusů zhruba u třetiny nápisů: funerální 251 nápisů, sídelní 222 nápisů, z čehož 53 bylo nalezeno v rámci obchodního kontextu daného osídlení, např. na agoře, na foru, v emporiu, v přístavu apod.; jiný kontext 16, rituální/náboženský kontext 779 nápisů. Pouze u 380 nápisů (8 \letterpercent{}) udávají místo nálezu jako sekundární kontext, což znamená, že nápis byl použit při stavbě mladší stavby, či autor korpusu uvádí, že byl nápis přesunut z původního místa, a přestal sloužit primární funkci.

\subsection[použitý-materiál-a-nosič-nápisu]{Použitý materiál a nosič nápisu}

Převážná většina nápisů byla zhotovena z kamene (4518 nápisů, což představuje 97 \letterpercent{}), a to především z mramoru (3292 nápisů), vápence (652 nápisů), syenitu (53 nápisů), pískovce (45 nápisů) a jiných, vesměs lokálních variant použitého kamene.\footnote{Agniezka Tomas (2016, 37-40) poukazuje na využití místního kamene na nápisech z okolí římského tábora v Novae. Převážně jsou využívány místní zdroje vápence a pískovce, pocházející jednak z bezprostředního okolí Novae a jednak z lokalit dostupných proti proudu řeky Jantra směrem na jih. Místní zdroj kamene byl tak využíván nejen pro epigrafické aktivity, ale zejména pro stavbu budov a jiných staveb nejen od druhé poloviny 1. st. n. l., jak naznačují dochované nápisy, ale pravděpodobněji již dříve. Tomas zaznamenává šest míst v regionu Novae, kde byly nápisy vyráběny ve větším měřítku, kde umísťuje i produkční dílny: Butovo, Novae, Dimum, Nicopolis ad Istrum, Karaisen a Pejčinovo - vše lokality zhruba v okruhu 40 km s výjimkou Nicopolis ad Istrum.} Z materiálů jiných než kámen se jednalo o 24 nápisů na kovu,\footnote{Jako je zlato, stříbro, olovo a editorem blíže neurčený kov.} dále 15 nápisů bylo na keramice, jeden nápis na dřevě, a šest na jiném druhu materiálu, jako jsou například mozaiky, či jako součást nástěnné malby. Téměř dvě třetiny všech objektů nesoucích nápis byly dekorované (2904 nápisů, 65,04 \letterpercent{}). Převládající typ dekorace byla reliéfní výzdoba (64,32 \letterpercent{} všech objektů nesoucích nápis, tedy 98,89 \letterpercent{} objektů s dekorací), nápisy s dochovanou malbou či jiným typem dekorace tvořily dohromady pouze 0,72 \letterpercent{} všech nápisů. Mezi nejčastější typy reliéfní výzdoby patřila výzdoba figurální (39,14 \letterpercent{} ze všech objektů nesoucích nápis), dále pak architektonická výzdoba či výzdoba ornamentální (23,88 \letterpercent{} ze všech objektů nesoucích nápis).\footnote{Vysvětlení jednotlivých pojmů a jejich obrazová galerie je součástí Apendixu 1. Jednotlivým druhům dekorace a jejich výskytu se věnuji podrobněji v následujících chronologicky řazených sekcích v této kapitole.} Nápisy tesané do kamene a jejich dekorace prochází v průběhu staletí vývojem směrem od prostých stél s jednoduchou florální či malovanou dekorací z klasické a hellénistické doby, až po složité reliéfní dekorace a nejrůznější motivy, nacházející se jak na stélách, tak na architektonických prvcích jako jsou sloupy, oltáře či dokonce sarkofágy v době římské.\footnote{Paradoxně k největšímu rozšíření reliéfních vyobrazení řeckých božstev a scén z řecké mytologie dochází až v době římské, nikoliv v době hellénismu, jak by se dalo očekávat.} Stupeň a způsob dochování objektů nesoucí nápis, jak je uvádějí autoři korpusu, či jak bylo patrné z přiložené vizuální dokumentace, jako jsou fotografie, kresby, oklepky apod. podrobně dokumentuje tabulka 5.01 v Apendixu 1. Je zřejmé, že výpovědní hodnota a přesnost interpretace je vyšší u nápisů, které se dochovaly v co možná nejkompletnější podobě. Bohužel u čtvrtiny nápisů nebylo možné určit stupeň dochování, a to zejména kvůli chybějící vizuální dokumentaci, a nejasnému charakteru textu. I přesto se však téměř 70 \letterpercent{} objektů nesoucí nápis dochovalo v takové podobě, že je možné z nich získat data potřebná k další analýze obsahu textu.

\section[rozsah-a-typologie-nápisů]{Rozsah a typologie nápisů}

Dochovaná délka textů nápisů se pohybuje v rozmezí 1 až 270 řádků, s průměrnou délkou nápisu 4,6 řádek (aritmetický průměr 4,6 řádek; medián 3 řádky).\footnote{Aritmetický průměr ({\em mean}) je součtem všech hodnot, vydělený celkovým počtem prvků. Medián udává střední hodnotu souboru vzestupně seřazených hodnot a dělí tak soubor na dvě stejně početné poloviny. Hodnota mediánu není zpravidla ovlivněna extrémními hodnotami, např. velmi se odlišujícími maximálními či minimálními hodnotami, na rozdíl od aritmetického průměru, a lépe tak poukazuje na střední hodnoty daného souboru. Pro srovnání udávám vždy aritmetický průměr a medián daného souboru.} U 784 nápisů se nepodařilo zjistit ani jejich přibližnou délku, a to zejména díky jejich extrémně špatnému stavu dochování. Z celkové analýzy délky dochovaného textu je zřejmé, že tři čtvrtiny nápisů mají délku do 5 řádků (3503 nápisů), z čehož jedna třetina nápisů obsahuje právě dva řádky textu (1206 nápisů). Skupina nápisů s rozsahem od 6 do 10 řádků obsahuje 777 textů, skupina nápisů s rozsahem 11-20 řádků 319 textů, skupina nápisů s rozsahem textu 21-50 řádků obsahuje 55 textů, a nad 51 řádků se celkem vyskytuje pouhých 11 nápisů. Celkově převládají texty krátkého charakteru a nápisy delší než 20 řádků se vyskytují spíše výjimečně.

Celkem 3609 nápisů bylo publikováno jednotlivými osobami či skupinami lidí pro soukromé účely, což představuje 77 \letterpercent{} všech nápisů. Nápisy veřejné povahy, vydané politickou autoritou či samosprávní jednotkou, tvoří zhruba 15 \letterpercent{} všech nápisů, zbytek nebylo možné přesně určit. Velký počet soukromých nápisů plně odpovídá převaze kratších textů, které ve velké většině mají charakter textů publikovaných pro osobní potřebu jednotlivců či skupin lidí. Jak plyne z dat uvedených v tabulce 5.02 v Apendixu 1, průměrná délka soukromého nápisu je 3,71 řádku (aritm. průměr). Oproti tomu veřejné nápisy jsou velmi často popisnějšího charakteru: jejich průměrná délka je 10,93 řádku (aritm. průměr).

Soukromé nápisy je možné podle jejich obsahu a primární funkce rozdělit na několik základních skupin. Jak je patrné z tabulky 5.03 v Apendixu 1, nejpočetnějšími dvěma skupinami soukromých nápisů jsou dedikační a funerální nápisy. Tyto dvě skupiny dohromady tvoří přes dvě třetiny všech nápisů (37 \letterpercent{} pro nápisy funerální a 38 \letterpercent{} pro nápisy dedikační) a přes 97 \letterpercent{} všech soukromých nápisů. Zbývající kategorie soukromých nápisů, jako jsou vlastnické nápisy, či soukromé nápisy nespadající do žádné z výše zmíněných kategorií, či nápisy, které nebylo možné do konkrétní kategorie přiřadit, tvoří zbývající 2,6 \letterpercent{} soukromých nápisů.

\subsection[funerální-nápisy]{Funerální nápisy}

Funerální nápisy představují spolu s dedikačními nápisy nejčastější kategorii nápisů, a to nejen z Thrákie, ale i z celého antického světa (Bodel 2001, 30-35). V databázi je zaznamenaných funerálních 1740 nápisů, což představuje druhou nejpočetnější skupinu nápisů právě po dedikačních nápisech se 1777 exempláři. Do kategorie funerálních nápisů tak spadají nápisy vydávané pozůstalými členy rodiny, či blízkými přáteli zemřelého. Jejich primární funkcí bylo označit místo pohřbu a připomínat život zemřelé osoby v nejbližší komunitě (Sourvinou-Inwood 1996, 140-142). Funerální nápisy byly velmi často umístěny přímo nad místem pohřbu, ať už se jednalo o tzv. plochý hrob, či o vyvýšenou mohylu ve formě hliněného násypu, postaveného nad místem pohřbu a podobné hroby můžeme vidět v archaické době v Athénách (Kurtz a Boardman 1971, 68-90). U nápisů z Thrákie je jejich archeologický funerální kontext znám u 251 nápisů, z toho 146 nápisů je možné identifikovat s pohřební mohylou či jejím bezprostředním okolím, nicméně celkové počty funerálních nápisů jsou mnohonásobně vyšší, avšak bez známého kontextu nálezu.

První funerální nápisy se objevují již v 6. st. př. n. l. v prostředí řeckých kolonií na pobřeží, kde se tato tradice uchovala po celou dobu antiky. V římské době se náhrobní kameny rozšířily i do vnitrozemí, kde se nacházejí zejména v okolí měst a podél cest.\footnote{Více o vývoji funerálních nápisů v jednotlivých stoletích v kapitole 6. O rozmístění funerálních nápisů v krajině více v kapitole 7.} Obecně se soudí, že funerální nápisy na náhrobcích byly v rámci řecké a římské společnosti veřejně přístupné a každý, kdo uměl číst, se mohl dozvědět více o zemřelé osobě, ale i nejbližší komunitě (Kurtz a Boardman 1971, 86; Saller 2001, 97-107). Obsahem funerálních textů jsou nejen zmínky o nebožtíkovi, jeho původu, dosažených životních úspěších a společenské prestiži, ale podobný druh informací je možné z textu vyčíst i o nejbližších, kteří nechali nápis zhotovit. V určitých případech, a to zejména v rámci thrácké komunity, skupina funerálních nápisů mohla obsahovat i nápisy umístěné uvnitř hrobu samotného, ať už jako součást pohřební výbavy, nebo hrobové architektury. V takových to případech nápisy mohou nést výpovědní hodnotu i o průběhu pohřebního ritu samotného, jakožto o zesnulém a o komunitě, z níž pocházel.\footnote{Více o konkrétním obsahu a vývoji sdělení v rámci jednotlivých komunit v kapitole 6.}

Na funerálních nápisech se dochovaly záznamy o celkem 2294 osobách, a to v podobě osobního jména či kombinace osobních jmen, případně jmen označujících rodiče či partnery. Celkem je zaznamenáno 4678 osobních jmen, což znamená, že na jednom funerálním nápise figurovala v průměru 1,31 osoby a jednalo se tedy primárně o náhrobky patřící spíše jednotlivcům než rozvětveným rodinám. Tento poměr se proměňuje v jednotlivých stoletích, ale obecně je možné sledovat odklon od individualismu klasické a hellénistické doby směrem k uvádění většího počtu osob na náhrobcích v římské době. Tento jev je pozorovatelný napříč oblastmi římské říše, kdy bylo zvykem uvádět na nápisech i členy širší rodiny, případně přátele, a to pravděpodobně z dědických důvodů (Saller a Shaw 1984, 1445).

Na funerálních nápisech převládají mužská jména řeckého původu. V 71 \letterpercent{} případů se jedná o muže, ve 21 \letterpercent{} o ženu a zhruba 5 \letterpercent{} není možné pro svou nejednoznačnost přiřadit ani do jedné skupiny. Poměr původu osobních jmen vyznívá ve prospěch jmen řeckého původu. Řeckých jmen se dochovalo 60 \letterpercent{}, thráckých jmen 10 \letterpercent{}, římských jmen 21 \letterpercent{} a jmen nejasného původu bylo 9 \letterpercent{}. Kolektivní vyjádření identit na funerálních nápisech se dochovala u 60 textů. Převážně se jedná o uvedení původu zemřelého či člena rodiny. U 48 případů je to odkaz na město se silnou řecky mluvící populací převážně v Thrákii, či Malé Asii. Odkaz na region se objevuje pouze třikrát, na příslušnost ke konkrétnímu kmeni šestkrát, dále výraz {\em barbaros} pouze dvakrát a původ označovaný místní vesnicí pouze jednou. Na 57 nápisech se dochovaly dialektální znaky, což ve většině případů byly znaky dórského dialektu a tyto nápisy pocházely z původně dórských kolonií Mesámbria a Byzantion. Celkem 628 nápisů si uchovalo tradiční epigrafické formule spojené s funerálními nápisy, jako např. {\em mnémé}, {\em theois katachthoniois, chaire parodeita} téměř po celou dobu antiky. Pro popis místa pohřbu se používaly ustálené termíny {\em tymbos}, {\em tafos}, {\em bómos}, a {\em stélé} pro označení náhrobního kamene. V římské době se objevují sarkofágy a termíny {\em latomeion}, {\em soros.} Tyto znaky poukazují na kontinuitu jazyka užívaného pro účely funerálních nápisů, ale na proměnu v rámci ritu samotného, kdy hrobky na určitou dobu doplnily i sarkofágy, případně urny, jak dokazují i archeologické nálezy (Tomas 2016, 84-87).

\subsection[dedikační-nápisy]{Dedikační nápisy}

Dedikační nápisy představují nejčastější typ dochovaných nápisů s celkem 1777 exempláři. Dedikačními nápisy jsou označovány předměty nesoucí věnování určitému božstvu a zpravidla bývají nalézány ve společnosti dalších votivních předmětů na místě bývalé svatyně daného božstva (Janouchová 2013; 2016; v tisku 2017).\footnote{V určitých případech může docházet k cyklické kategorizaci, kdy je svatyně označena svatyní pouze na základě přítomnosti dedikačních nápisů, či naopak je nápis označen jako dedikační na základě nálezu v rámci archeologického kontextu svatyně. Abych tomuto předešla, přidržuji se informací poskytovaných editory jednotlivých epigrafických korpusů, avšak primárně přihlížím k obsahu a charakteristice nápisu, a automaticky neřadím jako dedikace nápisy pocházející z lokality charakterizované jako svatyně.} Dedikační nápisy věnují jak jednotlivci, tak i skupiny lidí konkrétnímu božstvu, či božstvům, s prosbou o pomoc či jako díky za již vykonané skutky a dobrodiní. Zhotovitelé nápisu často uvádějí nejen důvod věnování nápisu, ale i svou identitu, která v kontextu dané komunity hrála relevantní roli. Na nápisy tak zaznamenávají své osobní jméno, původ, zařazení do určité rodiny a komunity, ale mnohdy zdůrazňují i své životní úspěchy, případně společenský status.\footnote{Tato sdělení a jejich vývoj v rámci komunit podrobněji rozebírám v chronologickém přehledu jednotlivých století pak v kapitole 6.}

Celkem bylo možné na dedikačních nápisech dle osobních jmen identifikovat 1649 osob, což představuje průměrně 1,077 osoby na nápis. Dedikační nápisy byly spíše individuální záležitostí a se skupinovými dedikacemi se setkáváme méně často než například s rodinnými funerálními nápisy. Dedikant se na nápisech zpravidla identifikoval jedním až dvěma osobními jmény, nejčastěji svým jménem a jménem rodiče. Celkem se dochovalo 3112 osobních jmen, z čehož jména řeckého původu představují 29,5 \letterpercent{}, jména thráckého původu 24,5 \letterpercent{}, jména římského původu 36 \letterpercent{} a jmen nejistého původu 10 \letterpercent{}. Většinu dedikantů představovali muži, a to celkem v 88 \letterpercent{} případů. Ženy dedikovaly nápisy jen v 7 \letterpercent{} případů a u 5 \letterpercent{} osobních jmen nešlo jednoznačně určit pohlaví dedikanta.

Na dedikačních nápisech zaznamenáváme větší míru zapojení thráckého obyvatelstva, než je tomu u funerálních nápisů, což úzce souvisí i s povahou dedikací a zapojení procedury věnování nápisů mezi náboženské zvyklosti thráckých kultů římské doby. Dedikace byly z převážné většiny určeny božstvům nesoucím kombinovaná řecká jména a lokální epiteta, poukazující na udržování místních tradic zároveň s adaptací nových kulturních a náboženských prvků. Nejoblíbenějšími božstvy byl Asklépios s 249 dedikacemi, Apollón se 139, Zeus se 131, Héra se 72, a dále Dionýsos a Nymfy se 38 dedikacemi, Hygieia s 31, Héraklés s 28 a Artemis s 24. Mezi božstva z východního Středomoří patří Sarápis s 12 dedikacemi, Ísis s 10, Anúbis se třemi, Harpokratión se čtyřmi, Mithra s osmi, Kybelé s jednou, syrská bohyně se dvěma, Velká Matka se sedmi, Sabazios s pěti dedikacemi.

U 586 nápisů se dochovalo božské epiteton, které u dvou třetin nápisů poukazovalo na místní původ kultu. Mezi místní {\em héroy} nesoucí lokální přízvisko patřil {\em Zylmyzdriénos} se 47 nápisy\footnote{Dochované varianty jména jsou {\em Zylmyzdriénos, Zymdrénos, Zymedrénos, Zymlydrénos, Zymydrénos, Zymyzdros, Zysdrénos}.}, {\em Salsúsénos} s 32 nápisy\footnote{Dochované varianty jména jsou {\em Saldénos, Saldobysénos, Saldokelénos, Saldoúissénos, Saldoúsenos, Saldoússénos, Saldoúusénos, Saldúisénos, Saldúsénos, Saldúusénos, Salénos, Saltobysénos, Saltobyssénos, Saltúusénos}.}, {\em Keiladeénos} s 20 nápisy\footnote{Dochované varianty jména jsou {\em Keiladeénos, Kiladeénos, Deiladebénos, Keilaiskénos, K{[}eilade{]}énos}.}, {\em Karistorénos} se 17 nápisy, {\em Karabasmos} s 11 nápisy, {\em Zbelsúrdos} s devíti nápisy.\footnote{Dochované varianty jména jsou {\em Zbelsúrdos} a {\em Zbelthiúrdos}.} U těchto nápisů je nepatrně vyšší zastoupení jmen thráckého původu s 26,5 \letterpercent{}, zatímco jména řeckého původu mírně klesla na 25,5 \letterpercent{}, jména římská zůstávají na přibližně stejné úrovni 37 \letterpercent{} a neurčená jména na 11 \letterpercent{}.\footnote{Podobné srovnání pro oblast okolo Novae ukazují, že zhruba 24 \letterpercent{} nápisů je věnováno lidmi s řeckým jménem, 14 \letterpercent{} se jménem thráckého původu a 62 \letterpercent{} se jménem latinského původu čili římského (Tomas 2016, 84).}

Dedikační nápisy se staly velmi oblíbené ve vnitrozemských komunitách, kde můžeme sledovat vznik svatyní s velkými koncentracemi nápisů zejména ve 2. a 3. st. n. l. Z těchto svatyní pochází velká část dedikací, avšak některé lokality jsou známy i díky jedinému, někdy i náhodnému nálezu votivního předmětu či nápisu.\footnote{Více o dedikačních nápisech v proměnách času v kapitole 6 a o jejich rozmístění v krajině v kapitole 7.}

\subsection[veřejné-nápisy]{Veřejné nápisy}

Celkem se dochovalo 717 veřejných nápisů, což představuje 15 \letterpercent{} všech dochovaných nápisů. Veřejné nápisy, tedy nápisy zhotovené politickou a administrativní autoritou za účelem organizace společnosti a udržení společenského řádu, je taktéž možné rozdělit podle jejich obsahu do několika částečně se prolínajících skupin. Jak plyne z dat uvedených v tabulce 5.04 v Apendixu 1, nejpočetnější skupinou jsou honorifikační nápisy, tedy nápisy vydané v rámci samosprávné jednotky k poctě jednotlivce či skupiny lidí, které představují 41,56 \letterpercent{} veřejných nápisů a 6,39 \letterpercent{} všech nápisů. Další početnou skupinou jsou státní dekrety, tedy nařízení vydaná politickou autoritou, která mohou mít normativní charakter, či se jedná o veřejně vydaná ustanovení a smlouvy (2,35 \letterpercent{} všech nápisů). Dále do kategorie veřejných nápisů spadají i seznamy nejrůznějšího charakteru (1,37 \letterpercent{} všech nápisů) a náboženské texty (1,14 \letterpercent{} všech nápisů), veřejné nápisy nespadající do předchozích kategorií (2,94 \letterpercent{} všech nápisů) a veřejné nápisy, které nebylo možné přesněji určit (1,74 \letterpercent{} všech nápisů).

Jazyk veřejných nápisů je do značné míry konzervativní: ustálené formule a termíny se opakují v nápisech stejné funkce. Celkem 532 nápisů obsahuje hledané administrativní termíny, a to celkem v 1362 výskytech, což představuje zhruba 2,6 termínu na nápis. Mezi nejčastější termíny patří {\em démos} s 217 výskyty, {\em polis} se 158, {\em búlé} se 153, {\em autokratór} se 113, {\em kaisar} s 84, {\em presbeutés} a {\em antistratégos} se 79, {\em hégémón} se 41 a {\em hypatos} se 41 výskyty. V průběhu doby dochází k nárůstu používání těchto termínů, s čímž souvisí jednak narůstají epigrafická produkce, ale i zintenzivnění společenské organizace a objevení nových administrativních institucí v době římské.

I přes jistou ustálenost formy je možné sledovat vývoj regionálních variant textu, což by poukazovalo na míru autonomie regionů, a to zejména v předřímské době. Vliv jednotné autority, který by měl za následek sjednocování formy veřejný nápisů, není v této době patrný. Pokud docházelo k ovlivňování a výpůjčkám v rámci společenské organizace, dělo se tak místně a v relativně malém měřítku. Naopak v římské době dochází k určitému sjednocení obsahu i formy veřejných nápisů na celém území Thrákie, patrně pod vlivem jednotné administrativy a rozsáhlého byrokratického aparátu.\footnote{Podrobně se vývojem veřejných nápisů v jednotlivých stoletích zabývám v kapitole 6, o jejich rozmístění v krajině pak v kapitole 7.} Příkladem mohou být například milníky, které si zachovaly téměř identickou podobu nejen v Thrákii, ale například i v sousední Bíthýnii, a i jiných místech římské říše (pro Bíthýnii French 2013){\bf .}

\section[jazyk-nápisů]{Jazyk nápisů}

Analyzovaný soubor nápisů obsahuje převážně řecky psané nápisy, které tvoří až 98 \letterpercent{} všech nápisů. Databáze také obsahuje nápisy nesoucí zároveň řecký a latinský text (1 \letterpercent{}), a v neposlední řadě nápisy nesoucí pouze latinsky psaný text (1 \letterpercent{}).\footnote{Takto nízké číslo latinských nápisů nereflektuje jejich skutečný stav dochování, ale je výsledkem omezení daného výzkumu a zaměření práce pouze na řecky psané texty. V budoucnosti je možné práci rozšířit i o latinsky psané texty, což v současné době přesahuje možnosti a omezení dané doktorským projektem. Doplnění současného projektu o latinsky psané texty by vyžadovalo vytvoření badatelského týmu a jednalo by se o projekt dlouhodobého charakteru, jehož přínos v rámci současného stavu poznání by byl zcela jistě neoddiskutovatelný.} Nápisy, které bývají někdy označovány jako thrácké, neřadím do samostatné kategorie, vzhledem k jejich problematické povaze, ale zahrnuji je do nápisů řeckých s patřičným komentářem (Dimitrov 2009, 3-19; Dana 2015, 244-245).\footnote{Badatelé se nemohou shodnout, zda se skutečně jedná o thrácké nápisy, či pouze o nesrozumitelně psané nápisy řecké (např. Dimitrov 2009; Dana 2015). Jedná se o několik (méně než 10, konkrétní číslo je záležitostí definice „thráckého” nápisu jednotlivých badatelů) velmi krátkých textů psaných alfabétou, sestávajících z jednoho, či několika málo slov. Význam těchto nápisů zůstává nejasný, a tedy i jejich interpretační hodnota je značně omezená. I přesto jsem se rozhodla je zařadit do kategorie řeckých nápisů, avšak brát v potaz jejich speciální charakter.}

Velká část nápisů z Thrákie je psána v řečtině, a to i v době římské nadvlády. Převaha epigrafických textů v řečtině zcela odpovídá vývoji i v jiných částech Balkánského poloostrova a východních částí římského impéria obecně. Řečtina se stala oficiálním publikačním jazykem poměrně záhy po objevení prvních řeckých osídlení na thráckém pobřeží. První nápisy pocházely z čistě řeckého kontextu, nicméně v průběhu staletí se i thrácké obyvatelstvo naučilo používat alfabétu a později řečtinu jako prostředek písemné komunikace. V římské době nedošlo k zásadnímu kulturnímu přelomu v užívání publikačního jazyka a řečtina zůstala preferovaným jazykem epigrafických památek, alespoň na území provincie {\em Thracia}. Na hranicích mezi provincií {\em Thracia} a {\em Moesia Inferior} se v římské době setkávaly dvě jazykové tradice, latinská a řecká, které zásadně ovlivnily jazyk publikovaných nápisů. Přes území Balkánu vedla v době římské říše tzv. Jirečkova linie, která od sebe oddělovala území, kde byla valná většina textů psána řecky, od území s převahou latinsky psaných epigrafických památek (Jireček 1911, 36-39). Jirečkova linie procházela přibližně na místě dnešního pohoří Stara Planina a pomyslně oddělovala řecky píšící jih od latinsky píšícího severu. Přesné statistiky řecky vs. latinsky psaných textů je velmi obtížné uvést, vzhledem k nedostupnosti jednotného zdroje nápisů.\footnote{V roce 2006 začal vznikat digitální korpus nápisů z Bulharska pod patronátem Univerzity {\em Sv. Kliment Ohridski} v Sofii, který měl obsahovat zhruba 3500 řecky psaných nápisů. Bohužel i v roce 2017 je tento projekt stále ve fázi vývoje a oficiální webová stránka je veřejnosti nedostupná, http://telamon.proclassics.org/index.php (navštíveno 7. března 2017). Databáze {\em Packard Humanities Institute} s názvem {\em Searchable Greek Inscriptions} (PHI) obsahuje pouze řecky psané nápisy z území Thrákie v počtu okolo 4000 exemplářů, avšak s metadaty omezenými na dataci, jméno nálezové lokality a text nápisu. K poslední úpravě databáze došlo v září 2015, což znamená, že neobsahuje nové nápisy (\useURL[url19][http://inscriptions.packhum.org/][][{\em http://inscriptions.packhum.org/}]\from[url19], navštíveno 5. března 2017). Databáze {\em Epigraphic Database Heidelberg} (EDH) obsahuje jak latinské, tak řecké nápisy, nicméně není ani zdaleka kompletní. Ač dochází k neustálému doplňování nápisů, 5. března 2017 databáze obsahovala 397 nápisů z provincie {\em Thracia} a 1938 z provincie {\em Moesia Inferior}. (\useURL[url20][http://edh-www.adw.uni-heidelberg.de/home/][][{\em http://edh-www.adw.uni-heidelberg.de/home/}]\from[url20], navštíveno 5. března 2017). Databáze latinských nápisů {\em Corpus Inscriptionum Latinarum} (CIL) obsahuje k 5. březnu 2017 79 latinsky psaných nápisů z provincie {\em Thracia} a 387 z provincie {\em Moesia Inferior} (\useURL[url21][http://cil.bbaw.de/cil_en/dateien/datenbank_eng.php][][{\em http://cil.bbaw.de/cil_en/dateien/datenbank_eng.php}]\from[url21], navštíveno 5. března 2017).} Obecně se předpokládá, že tento poměr je pro Thrákii kolem 80:20 ve prospěch řecky psaných nápisů.\footnote{Přesné číslo latinsky psaných nápisů pocházejících z území Thrákie není známé. Jednotlivé nápisy jsou publikovány v několika zdrojích, a pravděpodobně některé nápisy zůstávají nepublikované. Odhady badatelů se tak do velké míry různí: Milena Minkova (2000, 1-7) poskytuje statistiku pouze pro území moderního Bulharska, kde předpokládá existenci zhruba 1200-1300 latinsky psaných nápisů. Nicolay Sharankov (2011, 145) uvádí, že řeckých nápisů je zhruba 20krát více než latinských, avšak tento odhad je pravděpodobně příliš nízký. Pokud vezmeme 4600 řeckých nápisů z {\em Hellenization of Ancient Thrace} databáze jako výchozí číslo, dle Sharankova by latinských nápisů bylo pouze 230. Pouze z okolí samotného města Novae pochází 447 latinsky psaných nápisů (Gerov 1989, 207).} V okolí větších měst a sídel vojenských jednotek je přítomnost latinsky psaných textů vyšší vzhledem k přítomnosti byrokratického a vojenského aparátu než na thráckém venkově, kde převládají řecky psané texty. V případě Thrákie se však latinský a řecký svět do značné míry prolínaly a docházelo k vzájemnému ovlivňování a lingvistickým výpůjčkám (Dana 2015, 253). \footnote{Agniezka Tomas (2007, 31-47; 2016) zpracovala výskyt řeckých a latinských nápisů v okolí města Novae na Dunaji v provincii {\em Moesia Inferior} na pomezí tzv. Jirečkovy linie. Novae bylo sídlem římské legie již od roku 45 n. l. a vzhledem k trvalé přítomnosti římského vojska se dá předpokládat převaha latinských nápisů nad nápisy psanými řecky. To platí pro bezprostřední okolí Novae a pro sídla ve vnitrozemí, která měla na starosti zásobování města a vojenských jednotek, jako např. {\em villa} v Pavlikeni. Ve venkovských sídlech se však objevují i nečetné řecky psané nápisy, a to zejména ve svatyních poblíž vesnic Paskalevec, Butovo a Obedinenie (Tomas 2007, 44). V okolí města Nicopolis ad Istrum, které se nachází zhruba 80 km jihovýchodně, je již převaha dochovaných nápisů psaná řecky, a to jak z města, tak z venkovských oblastí.}

\subsection[identita-a-volba-jazyka]{Identita a volba jazyka}

Volba řečtiny jako publikačního jazyka nápisů byla do nedávné doby brána jako jeden ze znaků hellénizace thráckého obyvatelstva (Sharankov 2011, 139-141). Podobně pak volba latiny jako publikačního jazyka byla brána jako znak romanizace, která však na území Thrákie neproběhla v tak úspěšné míře, jako hellénizace, a to zejména vzhledem k celkovému většímu počtu dochovaných řecky psaných nápisů (Tomas 2016, 119-120). Musíme vzít v potaz, že místní thrácké obyvatelstvo bylo před příchodem řeckých osadníků na území Thrákii negramotné, a s největší pravděpodobností byla řecká alfabéta prvním písemný systémem, s nímž se setkalo a jemuž bylo vystaveno v dlouhodobém časovém horizontu. Převzetí písemného systému pak může v tomto případě představovat způsob komunikační strategie, a tedy snahu o nalezení společného dorozumívacího systému. Tzv. „thrácké” nápisy, tedy nápisy psané alfabétou, které dosud nebyly přesvědčivě interpretovány, mohou představovat mezistupeň vývoje, kdy Thrákové používali písmo ve velmi omezené podobě, avšak kontext použití byl specifický pro thrácké prostředí (Dana 2015, 244-245).

Hellénizace jakožto přijetí řečtiny jako jazyka písemné komunikace s sebou nese celou řadu problémů. Jedním z nich je fakt, že řečtinu není možné brát jako jednolitý jazyk, ale spíše jako komplex příbuzných dialektů, vycházejících, ze společného základu. To alespoň platí v archaické a klasické době, kdy se na nápisech objevují charakteristické prvky náležící např. dórskému či ionsko-attickému dialektu.\footnote{Podrobněji v kapitole 6 sekce věnované 6. až 4. st. př. n. l.} Pokud se na nápisech dochovaly konkrétní dialektální zvláštnosti, pak nápisy s největší pravděpodobností pocházely z řecké komunity, která tak chtěla poukázat na své vazby s mateřskou obcí, jak dosvědčují mnohé příklady z thráckého pobřeží. Celkem bylo v Thrákii zaznamenáno 97 nápisů nesoucích charakteristické znaky dórského dialektu, které pocházely převážně z řeckých kolonií založených dórskou Megarou jako je Mesámbria s 52 nápisy, Byzantion s 34 nápisy, a Sélymbria se 7 nápisy. Z dalších nedórských lokalit pocházejí pouze náhodné nálezy vždy po jednom nápise v Apollónii Pontské, Marcianopoli, Odéssu a lokalitě Karon Limen.\footnote{Lokalita Karon Limen, též známá jako Caron Limen či Portus Caria.} Většina (86,46 \letterpercent{}) nápisů psaných dórským dialektem pochází z 5. - 1. st. př. n. l., a největší část z nich pochází z 3. st. př. n. l. z černomořské Mesámbrie.\footnote{Z Mesámbrie 3. st. př. n. l. pochází dvojnásobně velký počet veřejných nápisů oproti ostatním obcím, a tyto texty jsou navíc psány dórským dialektem. Převážně se jedná o dekrety vydávané lidem Mesámbrie ({\em búlé} a {\em démos}), čili o nařízení regulující společenskou organizaci, dále o honorifikační dekrety a o seznamy občanů. Více v kapitole 6 v sekcích věnovaných 3. st. př. n. l.} Texty dialektálních nápisů si uchovávají velmi konzervativní charakter, a to nejen volbou dialektu mateřského města, ale i svou formou. Na těchto nápisech se objevuje velký počet standardních epigrafických formulí, stejně tak jako referencí tradičních institucí řeckých {\em poleis}, regionálních řeckých božstev. Co se týče osobních jmen a vyjádření identity, nápisy pocházejí převážně z čistě řeckého prostředí (68 nápisů, 70 \letterpercent{}), thrácká jména se vyskytují pouze na 4 nápisech (4,12 \letterpercent{}), a téměř vždy v kombinaci s řeckým jménem. Volba dórského dialektu na nápisech tak poukazuje na kontinuitu společensko-kulturních vazeb a norem i několik století po prvotním založení řeckých kolonií.

\subsection[postavení-řečtiny-v-římské-době]{Postavení řečtiny v římské době}

V případě římské doby konkrétní volba jazyka představovala volbu mezi tradičním jazykem, který se v oblasti používal jako jazyk epigrafické publikace již několik století, či mezi oficiálním jazykem římského impéria, jeho byrokratického a vojenského aparátu (Zgusta 1980, 135-137; Gerov 1980, 155-164). Volbou publikačního jazyka zhotovitel situoval sdělení daného nápisu do konkrétního kulturního prostředí, které mělo předem stanovená pravidla a očekávání. Volba jazyka v sobě nesla očekávání čtenářské obce, konkrétní komunity, jíž bylo sdělení určeno. Obsah nápisu naznačuje, že se jednalo spíše o vědomou volbu než o projev nekritického přijímání určitého jazyka, což v určitých podmínkách naznačují právě hellénizační či romanizační teorie (Sharankov 2011, 139-141; Tomas 2016, 119-120).

Práce, která by srovnávala řecky a latinsky psanou epigrafickou produkci v jejich kompletním rozsahu, by jistě byla velmi záslužná, nicméně natolik časově náročná vzhledem k současnému stavu znalostí, že se svou povahou hodí spíše pro výzkumný tým. Proto v současné práci vycházím z několika místních studií, které však vykazují podobné rysy. Srovnání řeckých a latinských nápisů ve vztahu k demografii a funkci nápisu v několika regionech Thrákie ukazují, že nápisy, v nichž se vyskytuje latina, pocházejí z prostředí vojenské a politické samosprávy, úzce související s aktivitami římského impéria. Řečtina naopak převládá ve venkovských oblastech a zejména ve funkci dedikačních nápisů, ale setkáme se s ní i v rámci veřejných nápisů (Gerov 1980, 158; Sharankov 2011, 145; Tomas 2007, 44-45; Janouchová 2017, v tisku).

Zhruba 1 \letterpercent{} (40) všech analyzovaných nápisů v databázi jsou nápisy obsahující řecký a latinský text. Ve více jak polovině textů (24) se jedná o identický text v řečtině a v latině. V těchto případech nelze říci, že by jeden z jazyků měl přednost: oba dva figurují na pozici prvního uvedeného textu rovnoměrně, pravděpodobně dle preferencí zhotovitele a očekávané komunity čtenářů. V případě nápisů s odlišnými latinskými a řeckými texty (16) vidíme jasné rozdělení na komunity, jimž byl text primárně určen, případně z jaké komunity pocházel zhotovitel. Nápisy určené latinsky mluvící komunitě jsou psány primárně latinsky se sekundárním textem řecky. Sekundární text je jednak standardní epigrafická formule, jako např. invokační formule vzývající božstvo, či se může jednat o podpis zhotovitele předmětu nesoucí nápis. V neposlední řadě do této kategorie patří i dedikace římskému císaři psané latinsky doplněné označením místní komunity psaným taktéž řecky. Tyto sekundární texty poukazují na společensko-kulturní pozadí, které bylo pro zhotovitele obvyklé a cítil potřebu se na něj odkázat i v rámci nápisu, který byl jinak určen primárně latinské komunitě.\footnote{Např. {\em IG Bulg} 2 749; {\em Perinthos-Herakleia} 292.} Zhotovitelé nápisů jsou tak důkazem, že bylo možné přestupovat mezi jednotlivými jazykovými komunitami a panovala mezi nimi vzájemná kulturní tolerance. Malý počet těchto nápisů v celkovém korpusu však poukazuje na fakt, že většina epigraficky aktivní populace si zvolila jeden či druhý epigrafický jazyk. Dle dostupných dat se na většině území Thrákie jednalo o řečtinu, a to i v době, kdy se území stalo součástí římského impéria.

\subsection[styl-jazyka]{Styl jazyka}

Kulturní a politické proměny tehdejší společnosti významnou měrou neovlivnily jazykovou formu a styl použitého jazyka. Zdá se, že styl epigrafického jazyka zachovával poměrně konzervativní podobu a vyplýval spíše ze společenské funkce nápisu, než že by podléhal aktuálním trendům (Petzl 2012, 52-58).

Celkem 121 textů nápisů, či alespoň jejich část, byla psána metricky. Jednalo se výhradně o nápisy soukromého charakteru: přes tři čtvrtiny, celkem 94 nápisů, mělo funkci funerálních nápisů. Dedikační nápisy se vyskytovaly v 17 případech, což představovalo zhruba 14 \letterpercent{}, zbytek nápisů nebylo možné určit. Poměr metricky psaných nápisů, u nichž byla známá datace, vůči celkovému počtu datovaných nápisů mírně narůstal od 1. st. př. n. l., a to zhruba o 0,5 \letterpercent{}.\footnote{Celkový počet datovaných nápisů ze 7. až 1. století př. n. l. (975,58; normalizovaná datace, více o metodě normalizované datace v kapitole 4) představuje 27,47 \letterpercent{} všech nápisů. Celkem 19 nápisů psaných metricky (normalizovaná datace) z téhož období, představuje 1,95 \letterpercent{} ze všech datovaných nápisů. U nápisů datovaných do 1. až 8. st. n. l. se jedná celkem o 2575,09 nápisů (normalizovaná datace), což představuje 72,51 \letterpercent{} všech datovaných nápisů. U nápisů psaných metricky a datovaných do 1. až 8. st. n. l. se jedná o 62 nápisů (normalizovaná datace), což představuje 2,41 \letterpercent{} všech datovaných nápisů.} Tento nárůst je statisticky nesignifikantní, a mohl být způsoben stavem a nahodilostí dochování nápisů. Celkově je tedy možné říci, že tendence publikovat metrické nápisy zůstává konstantní po celou dobu epigrafické produkce v Thrákii a nemění se v závislosti na proměnách složení populace, či v reakci na politicko-kulturní události. Metricky psané nápisy tvoří v průměru 2,2 \letterpercent{} celkové produkce, tedy její marginální část, která tvořila velmi konzervativní a neměnnou součást epigrafické produkce.

\section[onomastické-zvyklosti-a-identita-osobního-jména]{Onomastické zvyklosti a identita osobního jména}

Osobní jména, která se vyskytovala na nápisech, odhalují, že epigrafická produkce byla zaměřená především na mužskou část populace. Mužská individuální jména, tedy jména osoby, která vystupovala jako primární {\em agens} nápisu, ať už zemřelý, dedikant či honorovaný člověk, tvořila 81,04 \letterpercent{} všech jmen. V rámci onomastických zvyků se primární identita člověka skládala z několika individuálních jmen, dále ze jmen rodičů, případně prarodičů a partnerů. V rolích rodičů se z 92,32 \letterpercent{} vyskytovala mužská jména. Stejně tak v rolích prarodičů figurovala v 91,53 \letterpercent{} mužská jména. Tento fakt poukazuje na silnou tradici identifikace pomocí {\em patronymik} nebo {\em paponymik} v rámci epigraficky aktivní společnosti Thrákie. Co se týče jmen partnerů, pak se obě pohlaví vyskytují přibližně stejně často (42,98 \letterpercent{} oproti. 52,05 \letterpercent{}). Ženská jména se celkově vyskytovala na 12,26 \letterpercent{} nápisů a jako primární agens nápisu figurovala ve 13,84 \letterpercent{}, podrobněji v tabulce 5.05 v Apendixu 1.

\subsection[onomastické-tradice]{Onomastické tradice}

Podíváme-li se na tradiční kulturně-společenský původ jmen, celková čísla ukazují, že většina jmen, která se na nápisech vyskytuje, pochází z řeckého kulturního prostředí (41,36 \letterpercent{}), následují římská jména (35,79 \letterpercent{}), a thrácká jména (13,57 \letterpercent{}). Tabulka 5.06 v Apendixu 1 poskytuje souhrnné počty jmen dle jejich původu ze všech časových období.\footnote{V kapitole 6 je rámci jednotlivých století vždy jednotlivě pojednáno o měnících se poměrech původu jmen a případných demografických změnách epigraficky aktivní populace.} Pokud obě dvě statistiky srovnáme dohromady, můžeme jasně vidět rozdíly mezi onomastickými zvyky jednotlivým kulturních prostředí. Jak je patrné z tabulky 5.07 v Apendixu 1, u řeckých jmen je to silná tradice uvádění {\em patronymik} (37,78 \letterpercent{}) a {\em paponymik} (0,61 \letterpercent{}) oproti římskému kulturnímu prostředí. V římském prostředí se užívání kombinace tří individuálních jmen pro jednotlivce ({\em tria nomina}) odráží ve vysokém počtu individuálních jmen (90,07 \letterpercent{}). Thrácké prostředí sleduje spíše řeckou onomastickou tradici, a to jak v užívání {\em patronymik} a {\em paponymik}. Z analýzy thráckých onomastických zvyků však víme, že dochází i k následování římské tradice, a to zejména v užívání kombinace několika individuálních jmen i v kombinaci se jménem římským, které se taktéž podílí na vysokém počtu římských individuálních jmen. Je tedy zřejmé, že osoby nesoucí thrácká osobní jména poměrně záhy přijaly zvyk uvádění osobního jména, doplněného jménem otce či prarodiče a tento způsob identifikace přetrval i v době římské. Naopak, v době římské byl tento zvyk doplněn zvykem novým, a to přidáním původně římská jména či několika jmen k jménu původně thráckému, za nímž i nadále následovalo jméno otce či prarodiče. Setkáváme se s kombinací několika onomastických tradic a vytvářením jedinečného systému identifikace osob thráckého původu.

\subsection[zastoupení-pohlaví-v-závislosti-na-kulturním-prostředí]{Zastoupení pohlaví v závislosti na kulturním prostředí}

Pokud se podíváme na zastoupení pohlaví v jednotlivým kulturních prostředích, v řeckém prostředí je zastoupeno největší procento ženských jmen oproti mužským jménům (13,85 \letterpercent{}). Mužská jména však představují více jak 85 \letterpercent{} všech jmen jak v řeckém, římském, tak thráckém prostředí.

Jak je patrné z dat v tabulce 5.08 a 5.09 v Apendixu 1, ženská jména v řeckém prostředí figurují na místě rodičů častěji (1,67 \letterpercent{}), než v případě římského a thráckého prostředí. To poukazuje na větší důležitost mateřské linie v rámci řecké společnosti. Otcovská linie však i přesto bývá pro srovnání uváděna v rámci identifikace jednotlivce zhruba 20krát častěji, než je tomu v případě matky (36,26 \letterpercent{}). Naopak jako partnerská jména jsou v řeckém prostředí uváděna častěji mužská jména, než ženská (2,81 \letterpercent{} oproti 1,58 \letterpercent{}), což opět poukazuje na důležitější roli muže v rámci uspořádání řecké společnosti, která je nicméně minimální i v řeckém prostředí. Vysoký počet jmen rodičů poukazuje na fakt, že udávání {\em patronymika} bylo v řeckém prostředí velmi časté, vzhledem k tomu, že 78 \letterpercent{} všech individuálních jmen bylo doplněno právě jménem otce.

V římském prostředí roli při identifikaci jednotlivce hrají osobnost konkrétního člověka ({\em tria nomina}) a linie předků nemá tak význačnou pozici jako v řeckém prostředí (0,41 \letterpercent{} pro ženy a 7,26 \letterpercent{} pro muže). Role partnera při identifikace jednotlivce je v římském prostředí taktéž nižší, než v řeckém prostředí (1 \letterpercent{} pro ženy a 1,03 \letterpercent{} pro muže). Thrácké onomastické prostředí do velké míry přejímá zvyky jak řeckého, tak římského prostředí: poměry užívání individuálních jmen a jmen předků při identifikaci osoby mají v řeckém a thráckém prostředí velmi podobné hodnoty - jedinci nesoucí thrácká jména udávají svůj původ pomocí jmen otců a prarodičů. Stejně tak ale jedinci identifikují sama sebe pomocí kombinace tří jmen, z nichž jedno, případně dvě jsou římského původu. Výjimkou tvoří jména partnerů, kdy thrácká ženská jména jsou ze všech kulturních prostředí nejčastěji užívána v roli partnerek (2,38 \letterpercent{}), což může být důsledek smíšených partnerských svazků mezi Thrákyněmi a Řeky.

\subsection[smíšené-svazky-a-prolínání-tradic]{Smíšené svazky a prolínání tradic}

Údaje o smíšených svazcích mezi Řeky a Thráky se na nápisech dochovaly celkem ve 43 případech, a to v rámci identifikace jednotlivce ve formě jména primární osoby a jména partnera. Nejčastější formou jsou svazky mezi mužem z řeckého prostředí a ženou z thráckého prostředí (33 výskytů, 76,7 \letterpercent{}). Celkem je možné rozlišit šest variant v závislosti na původu jména a pohlaví osoby vystupující na nápisech:

\startitemize[A][stopper=.]
\item
  \startblockquote
  muž se jménem řeckého původu, partnerka se jménem thráckého původu (19 výskytů)
  \stopblockquote
\item
  \startblockquote
  žena se jménem řeckého původu, partner se jménem thráckého původu (4 výskyty)
  \stopblockquote
\item
  \startblockquote
  muž se jménem thráckého původu, partnerka se jménem řeckého původu (4 výskyty)
  \stopblockquote
\item
  \startblockquote
  žena se jménem thráckého původu, partner se jménem řeckého původu (13 výskytů)
  \stopblockquote
\item
  \startblockquote
  muž nesoucí jméno thráckého a římského původu, partnerka se jménem řeckého původu (2 výskyty)
  \stopblockquote
\item
  \startblockquote
  žena nesoucí jméno thráckého a římského původu, partner se jménem řeckého původu (1 výskyt).
  \stopblockquote
\stopitemize

Smíšené svazky se vyskytují zejména na nápisech pocházejících z regionu městských center a z oblastí, kde se dá předpokládat zvýšená míra kontaktů mezi thráckou a řeckou, případně římskou populací, tj. v okolí {\em emporií} či vojenských lokalit. Největší koncentrace nápisů se smíšenými svazky pochází z Odéssu s 23 nápisy, převážně z římské doby. Dále se nápisy se smíšenými svazky v době římské objevovaly v údolí středního toku řeky Strýmónu (10), a v menší míře ve městech Byzantion (3), Nicopolis ad Nestum (2), Pautália (2), Augusta Traiana (2), Filippopolis (1) a Seuthopolis (1). Nápisy jsou většinou římské, až na tři hellénistické nápisy, které pocházejí ve dvou exemplářích z Odéssu a jeden ze Seuthopole.\footnote{Právě hellénistický nápis {\em IG Bulg} 3,2 1731 ze Seuthopole dokumentuje aristokratický svazek thráckého krále Seutha III. a pravděpodobně řecké či makedonské Bereníké a jedná se o první zmínku o existenci smíšených svazků thrácké aristokracie (Tacheva 2000, 30-35; Calder 1996, 172-173). Ostatní nápisy dokumentují svazky mimo aristokratické kruhy, nakolik je možné tak soudit z mnohdy lakonického textu.} Údaje o smíšených svazcích pochází převážně z funerálních nápisů, u nichž se zejména v římské době předpokládá uvádění rodinných příslušníků a vzájemných vztahů, mimo jiné i z dědických důvodů (Saller a Shaw 1984, 145; Meyer 1990, 95).

\section[shrnutí-2]{Shrnutí}

Primárním účelem této práce je v co nejúplnější podobě zmapovat a zaznamenat řecky psané nápisy nalezené na území antické Thrákie a na jejich základě zrekonstruovat přístup tehdejší společnosti k epigrafické produkci. Soubor 4665 nápisů umožňuje sledovat složení tehdejší epigraficky aktivní populace a charakterizovat základní rysy společnosti.

Epigrafická produkce v antické Thrákii měla z velké části charakter soukromých sdělení, které z převážné většiny souvisely s pohřebním ritem a náboženskými aktivitami. Funerální nápisy pocházely převážně z řecké komunity, a osobami, které texty nápisů zmiňují, jsou převážně muži, ač ženy byly zastoupeny z jedné pětiny. Dedikační nápisy poukazují na větší zapojení thráckého obyvatelstva mužského pohlaví, a naopak úbytek ženských dedikantů. S větším zapojením místního obyvatelstva souvisí i popularita lokálních kultů, tedy kultů nesoucích nejčastěji řecké jméno božstva doplněné lokální epitetem.

Převládajícím publikačním jazykem byla v předřímské i v římské době řečtina v provincii {\em Thracia} a latina v kombinaci s řečtinou v provincii {\em Moesia Inferior}. Latina se udržovala jako jazyk vojenské komunikace administrativy spojené s chodem římské říše, avšak občas docházelo i ke kombinaci obou jazyků na jednom médiu v souvislosti s očekávanými čtenáři a jejich jazykovými znalostmi. Řečtina se prosadila spíše v kontextu náboženských textů a ve venkovských oblastech, zatímco latinské nápisy pocházejí z okolí vojenských táborů a administrativních center. Volba daného jazyka byla do velké míry záležitostí individuální volby zhotovitele nápisu, zejména vzhledem k původu zhotovitele a publiku, jemuž bylo sdělení určeno.

Osobní jména na nápisech naznačují, že epigrafická produkce byla zaměřená především na mužskou část populace, a ženy vystupovaly spíše v roli partnerek a dcer než jako primární adresát nápisu. Větší zastoupení žen se vyskytuje na funerálních nápisech z řeckých kolonií na pobřeží, kde ženy tvořily asi jednu pětinu zemřelých. Tento trend se nicméně nepřenesl do vnitrozemských oblastí a na dedikační nápisy, kde je zastoupení žen nižší. Co se týče onomastických zvyklostí, thrácká populace přijímá jak zvyky řeckého prostředí, ale i později římský způsob uvádění jména. V thráckém prostředí se typicky udávalo jméno otce, případně prarodiče, podobně jako v řeckých komunitách. Nezáleželo na tom, zda jména stejného či různého původu čili zda byla thrácká, či řecko-thrácká, a stejně tak i jména rodičů. Tento způsob identifikace Thrákové zřejmě přijali od Řeků, pokud ho však již nepoužívali v dobách před výskytem nápisů.

V případě přejímání římských onomastických zvyků se setkáváme s užíváním tří jmen ({\em tria nomina}), stejně jako bylo obvyklé v jiných částech římské říše, např. v Gallii. Thrácká jméno je obvykle kombinované s jedním či dvěma římskými jmény a poukazuje tak na vyšší společenský status nositele, případně na jeho právní postavení. I nadále je však zvyk uvádět tři osobní jména kombinován s udáváním původu v podobě jmen otce či prarodiče a setkáváme se tak s komplikovaným systémem identifikace jednotlivců za zdůraznění thráckého původu nositele.

\chapter{Epigrafická produkce napříč staletími}
V této kapitole se věnuji podrobné charakteristice epigrafické produkce v Thrákii v rámci jednotlivých staletí. Zaměřuji se především na rozšíření epigrafické produkce a její proměny v čase. Důraz kladu na projevy celospolečenských trendů a změn zvyklostí v epigraficky aktivních komunitách. Specificky mě zajímá rozšíření epigrafického zvyku v prostředí urbánních a venkovských komunit, dále těchto komunit, společenská hierarchie a šíření kulturních prvků typických pro řecky mluvící komunity v rámci osídlení thráckého pobřeží i thráckého vnitrozemí a v neposlední řadě funkce, jakou nápisy zastávaly v dané komunitě. Předmětem chronologické analýzy je celkem 2036 nápisů, u nichž je možné zařadit dobu jejich vzniku do konkrétního časového bodu či intervalu.\footnote{Celkový soubor 2036 nápisů s koeficientem 1 a 0,5 tvoří zhruba 89,5 \letterpercent{} všech 2276 datovaných nápisů (nenormalizovaná čísla), což představuje statisticky signifikantní vzorek nápisů nesoucích patřičnou míru výpovědní hodnoty pro dané století. Jako součást chronologické studie pracuji jen s nápisy datovanými s přesností do jednoho století s koeficientem 1 (1291 nápisů) a s nápisy datovaných do dvou po sobě následujících století, tedy nápisy s koeficientem 0,5 (745 nápisů). Metoda výběru nápisů je podrobně vysvětlena v kapitole 4.} Dataci nápisů přebírám tak, jak ji stanovili editoři v jednotlivých korpusech.\footnote{Seznamy nápisů přináležející do daného souboru jednoho či dvou století jsou součástí Apendixu 3.}

\section[charakteristika-epigrafické-produkce-v-6.-st.-př.-n.-l.]{Charakteristika epigrafické produkce v 6. st. př. n. l.\footnote{Dva potenciálně nejstarší nápisy z území Thrákie pocházejí z řeckého města Byzantion, které bylo založeno jako kolonie řecké Megary v roce 667 př. n. l. Dochované nápisy {\em IK Byzantion} 42 a 53a je možné datovat pouze velice široce: zhruba od poloviny 7. st. př. n. l. až do doby klasické, respektive hellénistické. Jejich datace je natolik široká, že je nelze přiřadit k jednomu či dvěma konkrétním stoletím. Protože se však může jednat o první řecky psané epigrafické památky nalezené na území Thrákie, považuji za nutné je zmínit. Typologicky se jednalo o označení váhové míry, a označení délky, v jednom případě s uvedenou příslušností ke komunitě obyvatel Byzantia. Pokud akceptujeme jejich ranou dataci, pak by se mohlo jednat o jeden z prvních projevů politické a ekonomické autority na území Thrákie, a to v rámci řecké komunity. Bohužel tyto dva nápisy jsou tak ojedinělé a nejistě datované, a tudíž jejich výpovědní hodnota je velmi nízká.}}

Nápisy ze 6. st. př. n. l. pocházejí výhradně z řeckých měst na pobřeží Černého, Marmarského a Egejského moře. Jedná se o krátké texty funerálního, a tedy soukromého charakteru. Soudě dle formy a obsahu, nápisy pocházejí čistě z řeckého kulturního prostředí.

\placetable[none]{}
\starttable[|l|]
\HL
\NC {\em Celkem:} 3

{\em Region měst na pobřeží:} Abdéra 1, Apollónia Pontská 1, Perinthos (Hérakleia) 1

{\em Region měst ve vnitrozemí:} 0

{\em Celkový počet individuálních lokalit:} 3

{\em Archeologický kontext nálezu:} neznámý 3

{\em Materiál:} kámen 3 (mramor 1, místní kámen 1 (póros\footnote{Druh porézního kamene s velkým množstvím zkamenělých ulit mořských živočichů.}), 1 neznámý);

{\em Dochování nosiče}: 100 \letterpercent{} 1, 75 \letterpercent{} 1, nemožno určit 1

Objekt: stéla 3

{\em Dekorace:} reliéfní dekorace 1 nápis (vyskytující se motiv: stojící osoba 1); architektonická dekorace 1 nápis (vyskytující se motiv: florální motiv 1, anthemion 1)\footnote{Výsledné číslo součtu nápisů nesoucích dekoraci může být vyšší než celkový počet nápisů, protože některé nápisy nesou více druhů dekorace, případně kombinace motivů.}

{\em Typologie nápisu:} soukromé nápisy 3, veřejné 0

{\em Soukromé nápisy:} funerální 3

{\em Veřejné nápisy:} 0

{\em Délka:} aritm. průměr 3, medián 3, max. délka 3, min. délka 3

{\em Obsah:} standardní epigrafické formule 1 (funerální)

{\em Identita:} pouze řecká jména, 2 ženy, 1 muž, krátké texty, celkem 3 osoby, 1 osoba na nápis

\NC\AR
\HL
\HL
\stoptable

Do 6. st. př. n. l. je možné zařadit tři nápisy s přesností datace na jedno století. Nápisy pocházejí z regionu řeckých kolonií na pobřeží Egejského, Marmarského a Černého moře; konkrétně z regionu obcí Abdéra, Apollónia Pontská a Perinthos, pozdější Hérakleia. Místa nálezu se nacházejí přímo na území daných obcí, a tedy i v bezprostřední blízkosti mořského pobřeží. Vzájemnou polohu nálezových míst ilustruje Mapa 6.01 v Apendixu 2.

Charakteristické datační prvky archaických nápisů se vyskytují na několika exemplářích.\footnote{Směr použitého písma je v některých případech bústrofédon, např. {\em IG Bulg} 1,2 404 z Apollónie Pontské, nebo je používána místní epichórická alfabéta z Thasu/Paru, např. {\em I Aeg Thrace} 30 z Abdéry (Petrova 2015, 9).} Objekty nesoucí nápis jsou vytvořeny z kamene převážně místního původu, jako je póros, a z části z mramoru, jehož původ není znám, ale dá se předpokládat jeho lokální zdroj. Vizuální podoba objektů nesoucích nápis se velmi podobá nápisům pocházejícím z ostatních řecky mluvících komunit té doby (Kurtz a Boardman 1971, 68-90, 121-127; Sourvinou-Inwood 1996, 277-297; Petrova 2015, 9-18).\footnote{Stély v Apollónii Pontské mají prostý tvar obdélníku, případně v Perinthu mají typický podlouhlý tvar s palmetovým ukončením, známým v literatuře jako {\em anthemion}.} Ze dvou třetin převládá reliéfní vyobrazení, které znázorňuje motivy typické pro náhrobní stély z řeckých kolonií té doby. Ve většině případů se jedná o vyobrazení nebožtíka s doplňkovými atributy jako je pes, psací tabulka, maková hlavice apod.\footnote{Z kontextu řeckých komunit v oblasti Thrákie se z 6. st. př. n. l. se dochovaly i další náhrobní stély podobného charakteru, které však nenesou žádný nápis, jako například stéla zobrazující stojícího muže se psem. Toto vyobrazení, které je obecně považováno za zpodobnění řeckého aristokrata, se do dnešní doby dochovalo celkem v 15 exemplářích po celém Černomoří (Petrova 2015, 11-18) a bývá datováno na přelom 6. a 5. st. př. n. l.} Není zcela jasné, zda se jedná o černomořskou produkci, či o stély importované z jiných částí řeckého světa, avšak jejich výskyt právě v Černomoří nasvědčuje na jejich původ v rámci místních řeckých komunit iónského původu.

\subsection[typologie-nápisů]{Typologie nápisů}

Dochované nápisy z 6. st. př. n. l. je možné typologicky i obsahově určit jako soukromé funerální texty, vytvořené za účelem označení místa pohřbu a jako připomínka zesnulého pro okruh nejbližších lidí. Krátký rozsah textů poukazuje na účelnost sdělení: typický text obsahuje jméno zesnulého a určení jeho biologického původu udáváním jména rodiče, případně zhotovitele nápisu. Nápisy mohou promlouvat ke čtenáři: personifikovaný náhrobní kámen oznamuje komu přesně patří a jehož život připomíná.\footnote{Např. {\em Perinthos-Herakleia} 69: Ἡγησιπόλης εἰμὶ τῆς Ἡγεκράτεος, „Náležím Hégésipole, dceři Hégékratea”.} Obsah těchto sdělení měl význam převážně v nejbližší komunitě, kde každý znal zesnulého či jeho rodinu.

Ze 6. st př. n. l. tak nemáme žádné důkazy o prolínání řeckého a thráckého obyvatelstva. Nápisy svým kontextem i obsahem pocházejí z řeckých komunit a jejich charakter poukazuje na udržování tradičních společenských norem i v rámci nově vzniklých kolonií. Osobní jména, která se na nápisech vyskytovala, byla výhradně řeckého původu, dokonce i ve sledu dvou generací. Na nápisech nenalézáme žádné další vyjádření identity, ani se zde nevyskytují hledané společensko-kulturní termíny, až na jednu výjimku výskytu formulí typických pro náhrobní nápisy v řeckém světě.\footnote{Termín {\em mnéma} pro označení hrobu či náhrobního kamene samotného.} Udržení tradiční formy i obsahu poukazuje na relativní uzavřenost tehdejších komunit a konzervativnost projevů epigraficky aktivní společnosti, tedy lidí, kteří se podíleli na publikování nápisů.

\subsection[shrnutí-3]{Shrnutí}

Dochovaný vzorek nápisů ze 6. st. př. n. l. je bohužel velmi omezený a na jeho základě není možné hodnotit případné kontakty řecké a thrácké kultury. První nápisy se objevily v řecké komunitě v prvních desetiletích po založení kolonií, a jejich charakter se velmi podobal nápisům z jiných částí řecky mluvícího světa jak formou, tak obsahem. Malý počet dochovaných nápisů může poukazovat na a) nedostatečně a nerovnoměrně archeologicky prozkoumané kulturní vrstvy 6. st. př. n. l., b) nejasný charakter nápisů, který neumožňuje nápisy datovat právě do 6. st. př. n. l., c) na fakt, že společnost řeckých měst z území Thrákie v 6. st. př. n. l. neprodukovala velké množství nápisů, protože se potýkala jednak s interními, tak s externími problémy souvisejícími s osídlováním již obydleného území.

\section[charakteristika-epigrafické-produkce-v-6.-až-5.-st.-př.-n.-l.]{Charakteristika epigrafické produkce v 6. až 5. st. př. n. l.}

Nápisy datované do 6. a 5. st. př. n. l. pocházejí rovněž z pobřežních oblastí Černého a Marmarského moře z komunit řeckého původu. Typologicky se taktéž jedná o soukromé nápisy funerálního charakteru, které si udržovaly původní charakter a nedocházelo k prolínání s thráckou kulturou.

\placetable[none]{}
\starttable[|l|]
\HL
\NC {\em Celkem:} 3

{\em Region měst na pobřeží:} Apollónia Pontská 1, Perinthos (Hérakleia) 2

{\em Region měst ve vnitrozemí:} 0

{\em Celkový počet individuálních lokalit:} 3

{\em Archeologický kontext nálezu:} funerální 1, neznámý 2

{\em Materiál:} kámen 3 (mramor 1, neznámý 2)

{\em Dochování nosiče}: 100 \letterpercent{} 1, nemožno určit 2

{\em Objekt:} stéla 3

{\em Dekorace:} reliéfní dekorace 3 nápisy; figurální dekorace 3 nápisy (vyskytující se motiv: stojící osoba 2, sedící osoba 1, zvíře 1); architektonická dekorace 1 nápis (vyskytující se motiv: naiskos 1)

{\em Typologie nápisu:} soukromé nápisy 3, veřejné 0

{\em Soukromé nápisy}: funerální 3

{\em Veřejné nápisy:} 0

{\em Délka:} aritm. průměr 3,3 řádky, medián 3, max. délka 4, min. délka 3

{\em Obsah:} bez hledaných termínů, náhrobní kameny vzpomínající na zemřelého

{\em Identita:} pouze řecká jména, krátké texty, 1 osoba na nápise

\NC\AR
\HL
\HL
\stoptable

Do 6. a 5. st. př. n. l. jsou datovány celkem tři nápisy, které pocházejí z regionu řeckých kolonií na pobřeží Marmarského a Černého moře, tedy do stejných oblastí jako skupina nápisů datovaných do 6. st. př. n. l., jak ilustruje Mapa 6.01 v Apendixu 2.

\subsection[typologie-nápisů-1]{Typologie nápisů}

Dochované nápisy z 6. až 5. st. př. n. l. jsou typologicky totožné se skupinou nápisů ze 6. st. př. n. l. Jedná se taktéž o funerální texty, vytvořené za účelem označení místa pohřbu a na památku zesnulého v nejbližší komunitě. Nosiče nápisu jsou zhotovené z místního zdroje kamene a tvarově zachovávají podobu stél s jednoduchou dekorací. Jazyk nápisu je řecký, stejně tak jako původ dochovaných jmen. Text nápisů je ve všech třech případech psán v první osobě a obrací se na čtenáře, kterého informuje o identitě zemřelého a jeho původu, případně vyzdvihuje jeho kladné vlastnosti. Nakolik můžeme z takto malého vzorku soudit, zvyk zhotovovat náhrobní kameny s nápisy se i nadále udržel pouze v rámci řecké komunity.

\subsection[shrnutí-4]{Shrnutí}

Dochovaný vzorek nápisů datovaných do 6. až 5. st. př. n. l. je taktéž velmi omezený a na jeho základě nelze hodnotit případné kontakty řecké a thrácké kultury. Počet dochovaných nápisů v žádném případě nevypovídá o skutečném stavu tehdy existujících osídlení a jednotlivých vztahů mezi nimi, ale spíše reflektuje nedostatečný stav našich znalostí a probádanost archeologických vrstev z daného období.

\section[charakteristika-epigrafické-produkce-v-5.-st.-př.-n.-l.]{Charakteristika epigrafické produkce v 5. st. př. n. l.}

V 5. st. př. n. l. dochází ke znatelnému nárůstu epigrafické produkce a rozšíření nápisů do většího počtu řeckých komunit na pobřeží, kde i nadále převládají funerální nápisy. V thráckém vnitrozemí se poprvé objevují nápisy použité v rámci thráckého funerálního kontextu, patrně patřící thrácké aristokracii a související s prominentní pozicí, kterou aristokraté ve společnosti zastávali.

\placetable[none]{}
\starttable[|l|]
\HL
\NC {\em Celkem:} 60 nápisů

{\em Region měst na pobřeží:} Abdéra 10, Apollónia Pontská 9, Mesámbria 4, Perinthos (Hérakleia) 2, Strýmé 20, Zóné 9 (celkem 54 nápisů)

{\em Region měst ve vnitrozemí:} Pulpudeva (pozdější Filippopolis) 5\footnote{Jeden nápis byl nalezen mimo region známých měst, a není ho možné zařadit do regionu měst na pobřeží, ani ve vnitrozemí.}

{\em Celkový počet individuálních lokalit:} 13

{\em Archeologický kontext nálezu:} funerální 12, sídelní 1, nábož. 1, sekundární 11, neznámý 35

{\em Materiál:} kámen 54 (mramor 26, z toho mramor z Thasu 1, vápenec 6, jiný 20, z čehož je pískovec 1, póros 5; neznámý 2), kov 3 (stříbro, zlato), keramika 3

{\em Dochování nosiče}: 100 \letterpercent{} 10, 75 \letterpercent{} 10, 50 \letterpercent{} 23, 25 \letterpercent{} 7, kresba 2, nemožno určit 8

{\em Objekt:} stéla 49, architektonický prvek 7, nádoba 3, jiné 1

{\em Dekorace:} reliéf 12, malovaná dekorace 0, bez dekorace 48; reliéfní dekorace figurální celkem 2 nápisy (vyskytující se motiv: stojící osoba 2, sedící osoba 1, obětní scéna 1), architektonické prvky 14 (vyskytující se motiv: naiskos 3, sloup 2, báze sloupu či oltář 5, florální motiv 1, architektonický tvar 5)

{\em Typologie nápisu:} soukromé 55, veřejné 3, neurčitelné 2

{\em Soukromé nápisy:} funerální 43, dedikační 6, vlastnictví 4, jiný 1, neznámý 1

{\em Veřejné nápisy:} nařízení 1, jiný 2

{\em Délka:} aritm. průměr 2,68 řádku, medián 2, max. délka 16, min. délka 1

{\em Obsah:} dórský dialekt 2, iónsko-attický 4; bústrofédon 1, stoichédon 1; hledané termíny (administrativní 4 - celkem 4 výskyty, epigrafické formule 3 - celkem 7 výskytů, honorifikační 0, náboženské 6 - celkem 8 výskytů, epiteton 6 - celkem 6 výskytů)

{\em Identita:} řecká božstva, regionální epiteta, kolektivní identita 4 - pouze obyvatelé řeckých obcí mimo oblast Thrákie, celkem 56 osob na nápisech, 46 nápisů s jednou osobou; max. 2 osoby na nápisech, aritm. průměr 0,93 osoby na nápis, medián 1; komunita převládajícího řeckého charakteru, jména pouze řecká (65 \letterpercent{}), thrácká (1,67 \letterpercent{}), kombinace řeckého a thráckého (1,67 \letterpercent{}), jména nejistého původu (10 \letterpercent{})

\NC\AR
\HL
\HL
\stoptable

Do 5. století celkem spadá 60 nápisů a k celkovému rozšíření nálezových lokalit na 13 oproti třem lokalitám z 6. st. př. n. l., a s tím i spojeným nárůstem epigrafické produkce.\footnote{Nálezové lokality nesou jméno dle nejbližšího moderního osídlení, pokud není známo jejich antické jméno. V databázi jsou vedeny jako „{\em Modern Location}”. V regionu antického města je zpravidla více nálezových lokalit, které mohou, ale nutně nemusí korespondovat s archeologickou lokalitou.} Mapa 6.02 v Apendixu 2 ilustruje rozložení nápisů na území Thrákie v 5. st. př. n. l. Většina nálezových lokalit pochází z území řeckých kolonií na pobřeží Egejského, Černého a Marmarského moře, v maximální vzdálenosti do 20 km od pobřeží. Největším producentem s 20 nápisy je řecké osídlení Strýmé poblíž moderního Molyvoti na egejském pobřeží.\footnote{Z dalších měst, která v té době existovala, se nápisy do dnešní doby nedochovaly, či ještě nebyly objeveny. Možné je však také, že tato města v 5. st. př. n. l. nápisy neprodukovala, ať už z důvodů nestabilních podmínek, či odlišného přístupu tamních obyvatel k publikaci nápisů.} Z thráckého vnitrozemí se dochovaly nápisy v řádu pěti kusů z okolí moderní vesnice Duvanlij. Nápisy byly objeveny jako součást bohaté pohřební výbavy thráckých aristokratů v monumentálních hrobkách severně od řeky Hebros v regionu thráckého osídlení {\em Pulpudeva}, pozdější Filippopole. Tyto vnitrozemské nápisy se neodlišují pouze svou polohou, ale i použitým materiálem a funkcí, kterou objekt zastával v rámci společnosti.

Podobně jako v 6. st. př. n. l. je převážná většina 90 \letterpercent{} nosičů nápisu zhotovena z kamene, nicméně zbývajících 10 \letterpercent{} nápisů se nachází na kovových předmětech a na keramických nádobách.\footnote{Z nápisů tesaných do kamene má výrazná většina 82 \letterpercent{} charakter soukromého nápisu, tedy nápisu sloužícího pro soukromé účely jedince či skupiny lidí. Téměř ze tří čtvrtin převládají funerální nápisy se 43 exempláři, dedikační nápisy představují zhruba 10 \letterpercent{} a zbývající nápisy není možné přesněji určit.} Hlavní role nápisů psaných na kameni byla předat a sdělit specifickou zprávu v rámci lokální komunity, proto byly z převážné většiny určené k veřejnému vystavení, přístupné všem. Na lokální původ nápisů tesaných do kamene poukazuje i použitý materiál jako je vápenec, pískovec či porézní kámen, tzv. póros.\footnote{Tento fakt poukazuje na regionální charakter místních komunit, které primárně využívaly místně dostupný materiál a dále podporuje teorii, že nápisy byly produkovány místně a nestávaly se předmětem dálkového obchodu, alespoň v 5. st. př. n. l.} Oproti tomu primární funkce nápisů na kovových předmětech a keramických nádobách bylo v pěti z šesti nápisů označení vlastnictví či autorství a v jednom případě dedikace božstvu. Tato skupina nápisů sloužila primárně pro interní potřebu majitele a zpravidla se nejednalo o nápisy veřejně přístupné komukoliv, ale pouze vybraným členům z okolí majitele předmětu. Použitý materiál byl často drahý kov, jako je zlato či stříbro, což naznačuje na vysoké společenské postavení majitele. Mnohdy měly tyto předměty po smrti majitele využití jako sekundární funerální nápisy, a to zejména ve společnostech s kmenovým uspořádáním založených na charismatu a společenské prestiži jedince (Whitley 1991, 354-361; Bliege Bird a Smith 2005, 221-222, 233-234).

\subsection[funerální-nápisy-1]{Funerální nápisy}

Nápisy označené jako funerální je možné rozdělit do dvou kategorií dle jejich původní funkce a vztahu ke známému archeologickému kontextu: primární funerální nápisy, které byly vytvořeny pro účel pohřebního ritu, tedy např. vnitřní vybavení hrobky, architektonické součásti hrobky, a dále sekundární funerální nápisy, jejichž funkce byla původně jiná, ale pro svou sentimentální a společenskou hodnotu předměty nesoucí nápis tvořily součást pohřební výbavy (Janouchová a Weissová 2015).\footnote{Dosud nepublikovaný příspěvek na konferenci Symposium on Mediterranean Archaeology, 12. - 14. listopadu 2015, Kemer-Antalya, Turecko.}

\subsubsection[primární-funerální-nápisy]{Primární funerální nápisy}

Primární funerální nápisy jsou typicky kamenné funerální stély či jiné předměty, které sloužily k označení místa pohřbu a připomínání zemřelého. Dochovalo se jich celkem 37 a pocházejí výhradně z území řeckých měst v pobřežních oblastech.

Přes 91 \letterpercent{} primárních funerálních nápisů nese pouze řecká jména, s výjimkou jednoho nápisu z Apollónie, kde spolu figurují řecká a thrácká jména. Na černomořském pobřeží bylo možno u dvou nápisů určit použití dórského dialektu, a to u nápisů z Mesámbrie, která byla založená jako dórská kolonie, a dále i použití dialektu iónsko-attického u dvou nápisů z původně iónských kolonií Perinthu a Apollónie. Tento fakt souvisí s dialektem užívaným v rámci řeckých obcí, které oblasti osídlily a s nimiž je pojilo silné kulturně-historické pouto. Vyjádření identity na funerálních nápisech se objevuje celkem na čtyřech nápisech: vždy se jedná o vyjádření příslušnosti k řeckému městskému státu a většinou se nachází v kombinaci s řeckým osobním jménem.\footnote{Vyskytující se termíny: Aigínétés, Athénaios, Kyzikénos a Paroités.} Geografické jméno se na funerálních nápisech vyskytuje pouze jednou a jedná se o město Perinthos, tedy o řecké osídlení z území Thrákie samotné. Co se týče hledaných termínů a vyjádření identity na funerálních nápisech se vyskytuje pouze jediný administrativní termín, popisující identitu ženy nesoucí jméno řeckého původu jako propuštěnou otrokyni.\footnote{Nápis {\em IG Bulg} 1,2 334octies z~Mesámbrie.}

Z dochovaných funerálních nápisů je patrné, že místní řecké komunity byly i v 5. st. př. n. l. poměrně uzavřené a ke kontaktu mezi thráckým obyvatelstvem docházelo v minimální míře v okolí Apollónie Pontské. Naopak ke kontaktu s dalšími částmi řecky mluvícího světa docházelo na pobřeží Egejského moře, které bylo v této době místem zvýšeného zájmu řeckých obcí. Nedostatek interakcí na epigrafickém materiále však nutně nemusí znamenat neexistenci kontaktů, ale spíše poukazuje na charakter nápisů jako na velmi selektivní médium, zachycující pouze malou část tehdejší společnosti. Archeologické výzkumy naopak dokazují, že mezi řeckými a thráckými komunitami docházelo v 5. st. př. n. l. k vzájemné interakci na každodenní bázi, která se ale bohužel neprojevila na nápisech (Kostoglou 2010, 180-185; Ilieva 2007, 212-221).

\subsubsection[sekundární-funerální-nápisy]{Sekundární funerální nápisy}

Sekundární funerální nápisy se nacházejí na kovových či keramických předmětech, které primárně nebyly vyrobeny pro pohřební ritus, ale do dnešní doby se dochovaly právě jako součást pohřební výbavy. Celkem se jedná o tři nápisy na kovových nádobách a tři na keramice, které pocházejí převážně z thráckého vnitrozemí z kontextu aristokratických pohřbů.

Nápisy na předmětech z drahých kovů, případně na importované keramice kvalitního provedení\footnote{Tři nápisy na keramice byly taktéž nalezeny na thráckém území v monumentálních hrobkách: dva z nápisů pocházejí taktéž z nekropole u vesnice Duvanlij ({\em SEG} 47:1061,4 a {\em SEG} 47:1061,5) a jeden nápis pochází z nekropole města Apollónia Pontská na černomořském pobřeží ({\em SEG} 54:630). Ve všech případech se jednalo o keramické nádoby řecké provenience, jako je attická hydria, dále střep keramického talíře nesoucí mužské jméno řeckého původu, a nakonec skyfos nesoucí řecké jméno a věnování Afrodíté. Předměty mohly původně sloužit jako obchodní artikl, či dar, a text nápisu nemusí mít žádnou souvislost se sekundární depozicí v hrobce, ani s majitelem hrobky. I přesto, že tyto nápisy nemají přímou výpovědní hodnotu k průběhu funerálního ritu samotného, jedná se o důkaz cirkulace předmětů mezi thráckým vnitrozemím a černomořským pobřežím.}, a tedy i vysoké hodnoty, byly nalezeny ve funerálním kontextu v thráckém vnitrozemí: předměty pocházejí z monumentálních hrobek, které se v 5. st. př. n. l. nacházely na území kmene Odrysů, nalezených v blízkosti moderní vesnice Duvanlij (Filov {\em et al.} 1934). Monumentalita hrobek, kterou je možné spatřit ještě dnes, a nákladnost nalezené pohřební výbavy dává soudit, že se jednalo o významné jedince, pravděpodobně elitní členy kmene Odrysů (Archibald 1998, 154-171). Krátké nápisy na kovových nádobách jsou psány řeckou alfabétou a mají charakter soukromého nápisu. Protože předměty byly zhotoveny ze stříbra a ze zlata, pravděpodobně si je mohli dovolit jen nejbohatší členové tehdejší společnosti. Dochovaná jména jsou thráckého původu a bývají interpretována jako jména majitele hrobky a pohřební výbavy ({\em SEG} 46:871; Filov {\em et al.} 1934).\footnote{Konkrétně se jedná se o dva zlaté pečetní prsteny, stříbrné nádoby, součásti luxusního picího servisu.} Charakter dochovaných nápisů na kovových předmětech napovídá, že se jednalo předměty sloužící již za života majitele, které byly po jeho smrti přeneseny do hrobky jako hodnotný předmět, ukazatel společenského postavení majitele a důkaz prestiže v rámci komunity (Sahlins 1963; Whitley 1991, 354-361; Bliege Bird and Smith 2005, 221-222, 233-234).

Využití písma ve funerálním ritu se thráckém kulturním prostředí v 5. st. př. n. l. se poměrně zásadně odlišovalo od funkcí, které písmo zastávalo v řeckých komunitách na thráckém pobřeží. Docházelo-li ke kontaktu za účelem obchodní výměny zboží, nedocházelo ještě v této době k prolínání kulturních zvyklostí a společenského uspořádání, jak by se dalo očekávat. Pokud k nim přesto docházelo, výrazně se ale neprojevily na výsledné podobě a využití funerálních nápisů.

\subsection[dedikační-nápisy-1]{Dedikační nápisy}

Dedikačních nápisů je celkem pět a pocházejí z pobřeží Egejského moře: tři z Abdéry a dva ze Zóné. Nápisy jsou datovány do druhé poloviny 5. st. př. n. l. a jedná se výhradně o věnování řeckým božstvům, jako Afrodíté {\em Syria}, Hermés {\em Agoráios}, Pýthia a Hestiá. Věnování obsahují tradiční dedikační formule užívané v řecké epigrafické tradici, avšak pouze jeden nápis obsahuje výlučně řecká jména. Další nápisy obsahují jména velmi špatně dochovaná, u nichž bohužel není možné zjistit jejich původ. V jednom případě je nápis psán iónsko-attickým dialektem a v jednom případě je použita epichórická alfabéta z Thasu/Paru. Z těchto nepřímých důkazů je tak možné usuzovat, že dedikace pocházely převážně z řecky mluvících komunit v egejské oblasti. Dochované dedikační nápisy nasvědčují, že v 5. st. př. n. l. nedocházelo k prolínání řeckých a thráckých náboženských představ, ale tehdejší náboženství a jeho projevy na nápisech si udržovaly poměrně konzervativní přístup.

\subsection[veřejné-nápisy-1]{Veřejné nápisy}

Dochované tři veřejné nápisy reprezentují pouze 5 \letterpercent{} nápisů z daného souboru. Ve dvou případech se jedná o nápis vymezující hranice náboženského okrsku řeckých božstev Dia, Athény a {\em héróů} se jmény {\em Podalirios}, {\em Machaón} a {\em Periéstos}. Tyto nápisy pocházejí regionu řeckého města Strýmé na pobřeží Egejského moře. Nápis z Abdéry představuje velmi fragmentárně dochované nařízení vydané blíže neznámou politickou autoritou mezi lety 485 a 475 př. n. l. Na žádném z nápisů nebylo možné určit kontext komunity, z které nápis pocházel, vzhledem k chybějícím osobním jménům a vyjádřením identity. Místa nálezů pocházejí z regionu řeckých měst Abdéra, Strýmé, a tudíž se dá předpokládat, že byly vytvořeny zde žijícím řeckým obyvatelstvem pro interní potřeby řecké komunity. Použitý lokální materiál, jako je vápenec a mramor, více než nasvědčují omezení produkce veřejných nápisů pouze na řeckou komunitu v pobřežních oblastech.

\subsection[shrnutí-5]{Shrnutí}

Epigrafické památky z 5. st. př. n. l. pocházející z řeckých měst na egejském a černomořském pobřeží ukazují určitou míru konzervativismu vůči thráckým obyvatelům, alespoň dle obsahu dochovaných nápisů. Zcela odlišné pojetí funkce písma mezi řeckou a thráckou komunitou však nasvědčuje, že kontakty v 5. st. př. n. l. zůstávaly spíše na obchodní úrovni, a přenos kulturních zvyklostí byl i nadále minimální a omezoval se především na oblasti bezprostředně sousedící s řeckými městy na pobřeží. Nápisy v thráckém kontextu figurovaly pouze v elitních kruzích a jejich užití nasvědčuje o využití ryze pro soukromé účely prominentních jedinců. Oproti tomu v řeckém kontextu je epigrafická produkce rozšířena do větší části populace, nápisy jsou součástí života komunity a stejně tak tomu odpovídá i jejich obsah.

K vzájemným kontaktům Řeků a Thráků docházelo dle epigrafického materiálu ve velmi omezené míře v okolí Apollónie, která se nacházela v sousedství území thráckých kmenů a pravděpodobně sloužila jako středisko obchodní výměny mezi řeckým světem a thráckým obyvatelstvem. Není tedy vůbec překvapivé, že se thrácké obyvatelstvo objevuje i v onomastických záznamech pocházejících z Apollónie, což může dokazovat nově vznikající příbuzenské vztahy, či mísení onomastických tradicí obou komunit. Jedná se nicméně o první epigraficky postihnuté kontakty Thráků a Řeků na nearistokratické úrovni.

\section[charakteristika-epigrafické-produkce-v-5.-až-4.-st.-př.-n.-l.]{Charakteristika epigrafické produkce v 5. až 4. st. př. n. l.}

Převážná většina produkce nápisů datovaných do 5. a 4. st. př. n. l. pocházela i nadále z pobřežích oblastí, pouze s jedním nápisem pocházejícím z thráckého vnitrozemí. Nápisy i nadále sloužily převážně funerální funkci. Vzhled i obsah nápisů poukazoval na řecký původ nápisů. Stejně tak i dochovaná osobní jména potvrzují, že epigrafická produkce se i nadále soustředila v rámci řeckých komunit.

\placetable[none]{}
\starttable[|l|]
\HL
\NC {\em Celkem:} 124 nápisů

{\em Region měst na pobřeží:} Abdéra 2, Apollónia Pontská 96, Chersonésos Molyvótés 1, Mesámbria 2, Odéssos 8, Strýmé 9, Zóné 1; (celkem 119 nápisů)

{\em Region měst ve vnitrozemí:} Beroé (pozdější Augusta Traiana) 1\footnote{Celkem čtyři nápisy pocházely z blíže neznámého místa v Thrákii: dva z nich pocházely z pobřeží Egejského moře a dva z území Bulharska.}

{\em Celkový počet individuálních lokalit:} 15 (76 \letterpercent{} všech nápisů z Apollónie Pontské)

{\em Archeologický kontext nálezu:} funerální 70, sekundární 7, neznámý 47

{\em Materiál:} kámen 123 (mramor 29, vápenec, 51, jiný 37, z čehož je pískovec 31, a dále místní kámen; neznámý 6), kov 1 (stříbro)

{\em Dochování nosiče}: 100 \letterpercent{} 62, 75 \letterpercent{} 13, 50 \letterpercent{} 16, 25 \letterpercent{} 5, oklepek 1, kresba 10, ztracený 3, nemožno určit 14

{\em Objekt:} stéla 119, architektonický prvek 4, jiný 1

{\em Dekorace:} reliéf 27, malovaná dekorace 11 (písmena červenou barvou), bez dekorace 87; reliéfní dekorace figurální 2 nápisy (vyskytující se motiv: jezdec 1, funerální scéna 1, stojící osoba 1, funerální portrét 1), architektonické prvky 17 nápisů (vyskytující se motiv: naiskos 22, sloup 3, báze sloupu či oltář 14, florální motiv 6, architektonický tvar 29, jiný 1)

{\em Typologie nápisu:} soukromé 120, veřejné 1, neurčitelné 3

{\em Soukromé nápisy:} funerální 119, vlastnictví 1

{\em Veřejné nápisy:} náboženské 1

{\em Délka:} aritm. průměr 2,25 řádku, medián 2, max. délka 7, min. délka 1

{\em Obsah:} použitý dialekt atticko-iónský 1, stoichédon 1; řecká božstva 1, epiteton regionální 1, epigrafické formule funerální 2 - celkem 2 výskyty, ostatní hledané termíny 0

{\em Identita:} místa zmíněná v nápisech z Thrákie (Perinthos 1), celkem 123 osob na nápisech, 104 nápisů s jednou osobou, max. 2 osoby na nápisech, aritm. průměr 0,99 osoby na nápis, medián 1; komunita převládajícího řeckého charakteru, osobní jména řecká (85 \letterpercent{}), thrácká (0,8 \letterpercent{}), kombinace řeckého a thráckého (1,61 \letterpercent{}), kombinace řeckého a jména nejistého původu (5,64 \letterpercent{}) jména nejistého původu (1,61 \letterpercent{}), bez jména (5,64 \letterpercent{}); vyjádření kolektivní identity 1 (občan řeckého města Hérakleia)

\NC\AR
\HL
\HL
\stoptable

Celkem se dochovalo 124 a podobně jako v předcházejících obdobích pocházejí z okolí řeckých měst na mořském pobřeží, jak je patrné z Mapy 6.02 v Apendixu 2. Největším producentem nápisů je Apollónia Pontská, odkud pochází 96 nápisů, což představuje 77 \letterpercent{} všech nápisů z daného období.\footnote{Za vysoký počet nápisů pocházejících právě z Apollónie může pravděpodobně stav archeologického výzkumu v oblasti a nedávné objevení několika nekropolí přímo na území antického města (Velkov 2005; Gyuzelev 2002, 2005, 2013; Baralis a Panayotova 2013, 2015; Hermary {\em et al}. 2010). V této době byla Apollónia ekonomické centrum a měla četné kontakty i s vnitrozemím, jak dokazují i např. mincovní nálezy (Paunov 2015, 268-269). Apollónia Pontská, kolonie Mílétu, byla založená koncem 7. st. př. n. l. Poměrně záhy se stala obchodním centrem regionu a svou výsadní pozici si udržela i v průběhu 5. st. př. n. l., kdy dokonce začala razit vlastní mince, které se nacházejí nejen v Thrákii, ale i ve Středomoří (Isaac 1986, 242-246). Apollónia tak pravděpodobně sloužila jako středisko obchodní výměny a nevyhnutelně zde docházelo i ke kulturní výměně mezi řeckým světem a thráckým obyvatelstvem.} Pouze jeden nápis pochází z vnitrozemí z okolí moderního města Kazanlak, spadajícího do regionu antického města {\em Beroé}, později známého pod římským jménem {\em Augusta Traiana}. Lešnikova Mogila, v níž byl nápis nalezen jako součást pohřební výbavy, bývá interpretována jako místo pohřbu bohatého aristokrata thráckého původu, viz níže (Kitov 1995, 19-21).

Nápisy datované do 5. až 4. st. př. n. l. mají velmi podobný charakter jako nápisy datované do 5. st. př. n. l. Většina nápisů je zhotovena z místního kamene, a má podobu jednoduché stély s minimem dekorací, jako jsou florální motivy či červeně malovaná písmena. Hlavní funkcí nápisů je stále označení místa pohřbu v řeckých komunitách a v thráckém vnitrozemí slouží nápisy i nadále úzkému okruhu aristokratů.

\subsection[funerální-nápisy-2]{Funerální nápisy}

Skupina nápisů datovaných do 5. a 4. st. př. n. l. obsahuje 119 primárních funerálních nápisů tesaných do kamene. Až 95 z nich pochází z Apollónie Pontské, což je pravděpodobně důsledek nedávného archeologického výzkumu ve městě, při němž byly objeveny rozsáhlé nekropole Kalfata a Budžaka (Gyuzelev 2002, 2005, 2013; Velkov 2005).

Oproti předcházejícímu období je možné pozorovat jen velmi pozvolný nárůst výskytu thráckých jmen na funerálních nápisech. Řecká a thrácká jména se společně vyskytují na třech funerálních nápisech, což představuje pouhých 2,4 \letterpercent{} funerálních nápisů z daného období. Vyskytující se thrácká jména patří výhradně ženám a všechny tři nápisy pocházejí z nekropole Kalfata ve městě Apollónia Pontská, kde pravděpodobně docházelo ke kontaktu mezi Thráky a Řeky a k navazování partnerských vztahů mezi jedinci s odlišným původem.\footnote{Žena thráckého původu partnerka či dcera muže nesoucí jméno řeckého původu se jménem Paibiné Augé, partnerka/dcera Hermaia z nápisu {\em IG Bulg} 1,2 430; tak jako žena se jménem řeckého původu partnerka či dcera muže se jménem thráckého původu: Faniché, partnerka/dcera Kerzea (Gyuzelev 2002, 20), a dále Dioskoridé, partnerka/dcera Basstakilea z nápisu {\em IG Bulg} 1,2 440. Fakt, že dochované nápisy dokumentují vytváření příbuzenských kontaktů mezi Thráky a Řeky, znamená, že obě mezi oběma komunitami docházelo ke kontaktu již delší dobu a vznikly zde vztahy, které epigrafické prameny nepostihují vůbec, či nepřímo a se zpožděním i desítek let.}

Tzv. sekundárně funerální se dochoval pouze jeden nápis zhotovený na stříbrné nádobě. Tento nápis taktéž pochází z kontextu bohaté pohřební výbavy hrobky thráckého aristokrata, která je známá pod názvem Lešnikova Mogila a nachází se u moderního města Kazanlak ({\em SEG} 55:742; Kitov 1995, 19-21). Podobně jako u stejné skupiny nápisů datovaných do 5. st. př. n. l. se jedná o stříbrnou nádobu sloužící za života majitele, jehož jméno pravděpodobně nápis nese.\footnote{O přesné podobě jména a znění nápisu se badatelé nemohou shodnout, vzhledem k tomu, že se jedná o jediný výskyt tohoto jména. Dimitrov (2009, 31-32) navrhuje interpretaci „(nádoba) Dynta, syna Zeila”. Theodossiev (1997, 174) navrhuje znění „(fiálé) Dynta, syna Zemya” a Dana (2015, 247) navrhuje znění „(majetek) Dyntozelmia”.} Nápis sloužil v okruhu thráckých aristokratů jakožto předmět poukazující na společenskou prestiž majitele, podobně jako u nápisů na kovových nádobách datovaných do 5. st. př. n. l.

\subsection[dedikační-nápisy-2]{Dedikační nápisy}

Z této skupiny nápisů se nedochoval žádný dedikační nápis.

\subsection[veřejné-nápisy-2]{Veřejné nápisy}

Z této skupiny nápisů se dochoval pouze jeden nápis {\em IG Bulg} 1,2 398, který Georgi Mihailov označuje jako {\em res sacrae}, ale svou povahou se nápis nachází na pomezí veřejného, dedikačního a stavebního textu (Mihailov 1970, 365-366). Tento nápis sloužil k označení místa v regionu Apollónie Pontské, kde stál chrám, {\em megaron}, řecké bohyně Gé {\em Chthonios}. Nejedná se tedy v pravém slova smyslu o nápis vytvořený politickou autoritou, který by souvisel s chodem státu či jiné politické organizace, ale spíše o nápis dokumentující rozčlenění půdy a existenci chrámu řecké bohyně na území řecké obce.

\subsection[shrnutí-6]{Shrnutí}

Nápisy datované do 5. až 4. st. př. n. l. vykazují velkou míru podobnosti se skupinou nápisů datovaných do 5. st. př. n. l. Převahu tvoří nápisy funerální nápisy, které pocházejí pouze z prostředí řeckých měst na pobřeží, a které si udržují tradiční formu i obsah. Z vnitrozemí pocházejí ojedinělé nálezy nápisů na kovových nádobách, které sloužily thrácké aristokracii jako luxusní předmět a bylo tak s nimi zacházeno i po smrti jejich majitele. Zvyk vytvářet nápisy tak i nadále zůstává znakem řecké kultury, která se téměř neprojevuje na zvyklostech thráckých obyvatel.

\section[charakteristika-epigrafické-produkce-ve-4.-st.-př.-n.-l.]{Charakteristika epigrafické produkce ve 4. st. př. n. l.}

Většina epigrafické produkce ve 4. st. př. n. l. pochází z řeckých komunit podél mořského pobřeží. I nadále převládají nápisy funerální, ale pozvolna narůstá i počet veřejných nápisů, sloužících převážně k organizaci a uplatnění politické svrchovanosti řeckých komunit, ale i jako prostředek vyjádření diplomatických vztahů mezi řeckými obcemi a thráckými aristokraty. Vyskytující se jména na nápisech nasvědčují na omezené mísení řecké a thrácké onomastické tradice v bezprostředním okolí řeckých měst. Nápisy se objevují ve velmi omezené míře i v thráckém vnitrozemí, kde slouží především potřebám thrácké aristokracie, stejně jako v 5. st. př. n. l.

\placetable[none]{}
\starttable[|l|]
\HL
\NC {\em Celkem:} 168 nápisů

{\em Region měst na pobřeží:} Abdéra 16, Apollónia Pontská 55, Byzantion 7, Chersonésos Molyvótés 1, Dionýsopolis 1, Maróneia 9, Mesámbria 11, Odéssos 5, Perinthos (Hérakleia) 2, Strýmé 29, Zóné 18 (celkem 154 nápisů)

{\em Region měst ve vnitrozemí:} Beroé (Augusta Traiana) 2, Pistiros 1\footnote{Celkem 11 nápisů nebylo nalezeno v rámci regionu známých měst, editoři korpusů udávají jejich polohu vzhledem k nejbližšímu modernímu sídlišti (čtyři lokality s celkem pěti nápisy), či uvádějí jejich původ jako blíže neznámé místo v Thrákii (šest nápisů).}

{\em Celkový počet individuálních lokalit}: 26

{\em Archeologický kontext nálezu:} funerální 58, sídelní 3, nábož. 2, sekundární 21, neznámý 84

{\em Materiál:} kámen 162 (mramor 92, z toho mramor z Thasu 3, vápenec 44, jiné 20, z čehož je pískovec 4, póros 2, vulkanický kámen 3; neznámý 6), kov 4 (stříbro 2, zlato 1, neznámý kov 1), jiný materiál 1, neznámý 1

{\em Dochování nosiče}: 100 \letterpercent{} 54, 75 \letterpercent{} 33, 50 \letterpercent{} 34, 25 \letterpercent{} 16, kresba 6, ztracený 1, nemožno určit 24

{\em Objekt:} stéla 142, architektonický prvek 18, nádoba 2, socha 1, nástěnná malba 1, jiný 2, neznámý 2

{\em Dekorace:} reliéf 67, malovaná dekorace 6, jiná dekorace 1, bez dekorace 94; reliéfní dekorace figurální 6 nápisů (vyskytující se motiv: jezdec 1, stojící osoba 2, sedící osoba 2, funerální scéna 1), architektonické prvky 61 nápisů (vyskytující se motiv: naiskos 24, sloup 4, báze sloupu či oltář 14, florální motiv 5, architektonický tvar/forma 19, jiný 1)

{\em Typologie nápisu:} soukromé 153, veřejné 9, neurčitelné 6

{\em Soukromé nápisy:} funerální 142, dedikační 5, vlastnictví 4, jiný 1, neznámý 1

{\em Veřejné nápisy:} nařízení 2, náboženské 1, seznamy 1, honorifikační dekrety 1, státní dekrety 2, jiný 1, neznámý 1\footnote{Součet nápisů jednotlivých typů je vyšší než počet veřejných nápisů vzhledem k možným kombinacím jednotlivých typů v rámci jednoho nápisu.}

{\em Délka:} průměr 2,96 řádku, medián 2, max. 46, min. 1

{\em Obsah:} dórský dialekt 12, iónsko-attický dialekt 3; stará attická alfabéta 2, stoichédon 3, graffiti 1; hledané termíny (administrativní 11 - celkem 16 výskytů, epigrafické formule 3 - celkem 3 výskyty, honorifikační 3 - celkem 4 výskyty, náboženské 8 - celkem 10 výskytů, epiteton 2)

{\em Identita:} řecká božstva 6, subregionální hérós 2, kolektivní identita 10 - obyvatelé řeckých obcí, Thrákové jako kolektivní pojmenování, celkem 174 osob na nápisech, 121 nápisů s jednou osobou; max. 7 osob na nápis, aritm. průměr 1,03 osoby na nápis, medián 1; komunita převládajícího řeckého charakteru, jména pouze řecká (72 \letterpercent{}), thrácká (2,97 \letterpercent{}), kombinace řeckého a thráckého (0,59 \letterpercent{}), jména nejistého původu (14,28 \letterpercent{}); geografická jména z oblasti Thrákie 4, geografická jména mimo Thrákii 1;

\NC\AR
\HL
\HL
\stoptable

Do 4. st. př. n. l. bylo datováno 168 nápisů, což znamená nárůst o 180 \letterpercent{} oproti skupině nápisů datovaných do 5. st. př. n. l. Ve 4. st. př. n. l. i nadále narůstá počet individuálních lokalit, v nichž byly nápisy nalezeny. Jak dokazuje mapa 6.03 v Apendixu 2, většina nápisů stále pochází z řeckých měst na pobřeží Černého, Marmarského a Egejského moře. Téměř jedna třetina nápisů pochází z řeckého města Apollónia Pontská, která v této době patří k hlavním kulturním a ekonomickým centrům regionu, ale zároveň také k nejlépe archeologicky prozkoumaným městům.\footnote{V posledních několika desetiletích zde byly nalezené nové nekropole s velkým počtem funerálních nápisů (Isaac 1986, 246; Avram, Hind a Tsetkhladze 2004, 931-932; Velkov 2005; Gyuzelev 2002, 2005, 2013). Oproti nápisům z 5. až 4. st. př. n. l. jsou nicméně celková čísla dochovaných nápisů z Apollónie nicméně zhruba o 40 \letterpercent{} nižší, což může být jednak důsledek lehkého útlumu postavení Apollónie, či výsledek do značné míry náhodných archeologických nálezů a široké datace nápisů.} Mezi producenty střední velikosti patří řecká města z pobřežních oblastí jako je Abdéra, Byzantion, Maróneia, Mesámbria, Odéssos, Strýmé a Zóné. Oproti 5. st. př. n. l. je možné zaznamenat objevení nových lokalit ve vnitrozemské Thrákii, a to zejména v okolí dvou hlavních toků, Hebros a Tonzos. Lokality sídelního a funerálního charakteru ve vnitrozemí jsou, dle obecně přijímaného konsenzu, jak řeckého, tak thráckého původu.\footnote{Lokalita Pistiros je archeology považována za řeckou obchodní stanici a říční přístav (Bouzek {\em et al.} 1996, 9). Lokalita Seuthopolis je interpretována jako sídlo thráckého panovníka Seutha III. (Dimitrov, Chichikova a Alexieva 1978, 3-5). Lokality Sborjanovo, Smjadovo, Alexandrovo, Kupino, Naip jsou považovány za hrobky bohatých thráckých aristokratů z kmene Odrysů (Stoyanov 2001, 207-218; Atanasov 2006, 6; Kitov 2001, 15-29; Gergova 1995, 385-392; Delemen 2006, 261-263). Archeologický kontext míst nálezu nápisů je bohužel z poloviny neznámý, dále z jedné třetiny funerální, což znamená, že nápisy byly nalezeny buď v konkrétní hrobce, její blízkosti, či v blízkosti pohřebiště. Ve třech případech pochází nápisy ze sídelního archeologického kontextu, a ve dvou případech bylo místo jejich nálezu určeno jako náboženského původu, např. u nápisů pocházejících ze svatyně. Sekundární archeologický kontext je znám celkem u 21 nápisů, což je zhruba osmina souboru.}

Podobně jako v 5. st. př. n. l. je možné sledovat odlišné charakteristiky nápisů zhotovených na kameni a na kovu či keramice. Jednak se jedná o jejich odlišnou kulturně-společenskou funkci, ale také i jejich výskyt v rámci prostorově odlišných komunit. nápisy na kameni (96 \letterpercent{}) se vyskytovaly převážně na pobřeží, nápisy na jiném médiu pocházejí převážně z vnitrozemí (3 \letterpercent{}).\footnote{U jednoho nápisu editor korpusu neudává materiál nosiče a je tedy neznámý.} Nápisy na kameni jsou datovány do 4. st. př. n. l. v počtu 162 nápisů, z nichž 59 \letterpercent{} nápisů je zhotoveno z mramoru, 28 \letterpercent{} z vápence a zbývající část z místně dostupného kamene.\footnote{V případě pobřeží Egejského moře zdroj mramoru pochází např. z lomů na Thasu.} Podobně jako v 5. st. př. n. l. dochází k využívání místních zdrojů a nemáme důkazy o tom, že by se nápisy staly předmětem dálkového obchodu. \footnote{Stejně jako v 5. st. př. n. l. má převážná část nápisů na kameni charakter soukromého nápisu, s podílem 91 \letterpercent{} nápisů z daného období. Dále jsou to nápisy veřejné se zastoupením 5 \letterpercent{} a nápisy, které nebylo možné určit, které představují zhruba 4 \letterpercent{}.} Oproti tomu nápisy z kovu a nápisy vyryté do stěny hrobky pocházejí z thráckého vnitrozemí či z oblastí na pobřeží Egejského moře tradičně ovládanými thráckým kmenem Odrysů v okolí hory Ganos, mezi řeckými městy Ainos a Perinthos (Archibald 1998, 109-111).\footnote{Stejně jako v případě nápisů na jiném materiálu než kameni, datovaných do 5. st př. n. l., i tato skupina čtyř nápisů pochází z kontextu bohatých pohřbů, patřícím pravděpodobně thrácké aristokracii, jako je Dalakova Mogila u Kazanlaku, lokalita Naip u hory Ganos na pobřeží Marmarského moře, neznámá mohyla u Kazanlaku (Delemen 2006, 251-273; Kitov 1995, 17; Kitov a Dimitrov 2008, 25-32). Jedná se jak o funerální nápisy primární, tak sekundárně umístěné.}

\subsection[funerální-nápisy-3]{Funerální nápisy}

Primární funerální nápisy si i ve 4. st. př. n. l. uchovávají stejný charakter jako v 5. st. př. n. l., doplněné o nápisy na vnitřní architektuře hrobek. Primární funerální nápisy i nadále pocházejí ze stejných či charakteristikou velmi podobných komunit na pobřeží, stejně tak jako sekundární funerální nápisy převážně pocházejí z kontextu thrácké aristokracie ve vnitrozemí.

\subsubsection[primární-funerální-nápisy-1]{Primární funerální nápisy}

Celkem se jedná o 142 nápisů tesaných do kamene, zhotovených nejčastěji ve tvaru stély.\footnote{Typický text funerálního nápisu představuje jméno zemřelého se jménem otce či partnera v genitivu. Pouze v jednom případě tento jednoduchý text doplňují okolnosti smrti a detaily z života zemřelého (Gyuzelev 2013, 20). Průměrná délka funerálního nápisu je 2,3 řádku, nicméně nejdelší text má až šest řádků.} Funerální nápisy pocházejí opět z řeckých komunit podél mořského pobřeží: největší skupina 55 funerálních nápisů pochází z Apollónie Pontské na pobřeží Černého moře. Další významnou skupinu tvoří 29 nápisů ze Strýmé na egejském pobřeží.\footnote{V sedmi případech je dokonce dochováno použití dórského dialektu v původně dórských koloniích Mesámbria s pěti nápisy, Byzantion s jedním; jeden nápis pochází z původně nedórského Odéssu.}

Osobní jména na funerálních nápisech jsou z 85 \letterpercent{} řeckého původu. V jednom případě nápisu {\em I Aeg Thrace} 153 ze Strýmé na egejském pobřeží je zaznamenáno mísení onomastických tradic, kdy jedna osoba nesoucí řecké jméno může mít thráckého otce.\footnote{Jméno otce Dadás není výhradně thrácké, tudíž není možné na jeho thrácký původ poukázat se 100 \letterpercent{} jistotou.} Vyjádření identity na funerálních nápisech na kameni poukazuje na čtyři případy příslušnosti k řecky mluvícím komunitám, naznačené užitím termínů Kyrenáios, Abdérítés, Amfipolítés a Kyzikénos, což poukazuje na stále relativně konzervativní společnost, kde imigrace nebyla častá, či nebylo nutné uvádět svůj geografický původ v rámci funerálních nápisů.\footnote{Geografická jména zmiňují pouze v jednom případě Olynthos, avšak text nápisu {\em I Aeg Thrace} 214 je příliš poškozen, abychom z něj mohli usuzovat něco dalšího.} Hledané termíny poukazují opět na převážně řeckou komunitu, která udržovala tradiční zvyky, nicméně tradiční formule náležící funerálním nápisům se vyskytla pouze jednou. Nesetkáváme se ani s individualizovanou prezentací jednotlivých zemřelých a poukazování na jejich společenský status, výjimkou je jeden nápis propuštěné otrokyně na nápise {\em IG Bulg} 1,2 334novies a.

Dva nápisy zhotovené na stěně vnitřní komory mohylové hrobky pocházejí z okolí hory Ganos\footnote{Text nápisu SEG 56:827bis: ΚΑΘΑΘΑ. Z téže hrobky dále pochází stříbrná nádoba {\em SEG} 56:828, označující majitele nádoby, thráckého aristokrata Térea (Delemen 2006, 261-262). Z historických pramenů víme, že jméno Térés tradičně patřilo panovníkům z rodu thráckých Odrysů, avšak nedokážeme s přesností posoudit, zda se jednalo o téhož Térea, majitele této hrobky (Hdt. 4.80, 7.137; Thuc. 2.29, 2.67, 2.95; Dem. 12.8).} na pobřeží Marmarského moře a z lokality u vesnice Smjadovo v severní části Thrákie.\footnote{Nápis {\em SEG} 52:712 ze Smjadova byl nalezen na překladovém kameni, který byl součástí vnitřní architektury hrobky, patřící pravděpodobně členovi či člence thrácké aristokracie. Vzhledem k tomu, že se jedná o nápis pocházející z vnitřního prostoru hrobky, jednalo se nejspíše o jméno majitelky hrobky: Gonimase(ze), ženy Seutha. Přesné znění nápisu {\em SEG} 52:712 není zcela jasné. Dimitrov (2009, 17-18) navrhuje překlad „Gonimaseze, Seuthova žena”, zatímco Dana (2015, 246-247) text čte následovně „Gonimase, Seuthova žena, (i nadále) žije!”.} Z uvedeného charakteru nápisů plyne, že přístup do samotných hrobek či k předmětům na kovových předmětech, měla velmi omezené skupina lidí z nejbližšího okolí majitele. Dle původu dochovaných jmen a charakteru pohřební výbavy je možné soudit, že se jednalo o členy thrácké aristokracie. Užití písma v tomto případě bylo omezeno převážně na utilitární funkci označení majitele či zhotovitele a obsah textu se nevázal k funerálnímu ritu samotnému. Svým charakterem se tak nápisy z vnitřní architektury hrobek řadí spíše k nápisům na kovových nádobách a jiných materiálech, nalezených v thráckém vnitrozemí v kontextu aristokratických hrobek v 5. až 3. st. př. n. l.

\subsubsection[sekundární-funerální-nápisy-1]{Sekundární funerální nápisy}

Do skupiny sekundárních funerálních nápisů patří celkem čtyři nápisy na kovových předmětech, zhotovené z drahých kovů: jeden zlatý pečetní prsten {\em SEG} 58:699 nese jméno a podobiznu pravděpodobně svého majitele Seutha, syna Térea. Dále do této skupiny patří stříbrná nádoba nesoucí nápis {\em SEG} 56:828 se jménem Térea, označující pravděpodobně majitele nádoby, zmíněná výše. Další nápis {\em SEG} 53:706 na stříbrné nádobě s textem „{\em Kotys, z Ergiské}”, který taktéž označuje majitele a geografický termín Ergiské označuje buď původ majitele či původ nádoby samotné. Bohužel u této nádoby není zcela možné určit její přibližné místo nálezu, protože pochází z aukce, avšak podobá se nádobám pocházejícím z území Thrákie (Loukopoulou 2008, 158-159). Poslední nápis {\em SEG} 46:851 na kovovém kratéru nese nápis označující pravděpodobně její obsah.\footnote{Nápis naznačuje, že je obsah jsou čtyři kyliky, tedy číše na víno.}

Ač byly všechny předměty nalezeny ve funerálním kontextu, obsah textu se nevtahuje specificky k pohřebnímu ritu. Nápisy představují součást pohřební výbavy uložené do hrobu společně se zemřelým a mají čistě utilitární funkci. Původ majitele, alespoň na kolik je možné soudit dle osobních jmen, byl thrácký.\footnote{Seuthés, Térés, Kotys jsou tradiční jména thrácké aristokracie, vyskytující se i v literárních pramenech (Thuc. 2. 29, 2.67, 2.95, 2. 97; Strabo 12.3.29; Tac. {\em Ann}. 2.64-65).} Naopak obsah nádoby je označen typicky řeckým způsobem, což může poukazovat na její řecký původ.\footnote{Písmeno delta označuje číslo 4, a slovo kylix pochází z řečtiny a označuje číši na pití vína.} Nicméně tento nápisy mohl vzniknout daleko dříve, než se nádoba dostala do thráckého prostředí, a její přítomnost dokládá pouze obchodní či diplomatické kontakty a výměnu zboží mezi thráckým vnitrozemím a řeckým pobřežím.

\subsection[dedikační-nápisy-3]{Dedikační nápisy}

Celkový počet dedikačních nápisů se oproti 5. st. př. n. l. výrazně nezměnil. Dochovalo se celkem pět nápisů z oblastí na pobřeží Egejského a Černého moře. Nápisy obsahují tradiční formule dedikační nápisů ({\em epoiésen}, {\em euchén}) a byly věnovány převážně řeckým božstvům Démétér, Kybelé a {\em héróům}, s přízviskem {\em Epénór} a {\em Mesopolités}. Nic nenasvědčuje, že by se Thrákové aktivně podíleli na vydávání dedikačních nápisů či že by nápisy pocházely z čistě thráckého prostředí. Texty jsou krátké a psány řecky, gramaticky a ortograficky nevykazují žádné zvláštnosti oproti standardnímu užití řečtiny a obsahují pouze řecká jména. Osobní jména, jména božstev, forma a obsah nápisů navíc poukazují na řecký původ dedikantů a převážně řecký kontext komunit, z nichž nápisy pocházely.

\subsection[veřejné-nápisy-3]{Veřejné nápisy}

Veřejných nápisů se dochovalo celkem devět, což představuje oproti 5. st. př. n. l. mírný nárůst. Narůstající číslo odpovídá i upevňování pozice politických autorit v regionu v důsledku stabilizace podmínek a nárůstu počtu nových osídlení.\footnote{Jedním z efektů dlouhodobé stabilizace politické situace může být i nárůst společenské komplexity. Tento jev se v rámci epigrafiky může projevit nejen nárůstem celkové epigrafické produkce, ale i vznikem nových funkcí a povolání, a tedy i jejich následnému objevení v textu nápisů.} Typologicky se jedná o šest dekretů, z čehož tři jsou honorifikační udílející pocty významným jedincům. Dále sem patří dvě nařízení, regulující obchodní výměnu, jeden seznam obsahující záznam pravděpodobně dlužných částek či vynaložených nákladů, jeden text náboženského charakteru na pomezí soukromého textu a dva nápis jiného či blíže neznámého charakteru.

Ve 4. st př. n. l. vidíme první náznaky využití nápisů v rámci regulace veřejného politického a ekonomického života: objevují se termíny jako je {\em polis}, {\em démos}, {\em búlé} apod. které poukazují na existenci samosprávných institucí. Tyto instituce dosud nebyly epigraficky potvrzeny z území Thrákie, nicméně pravděpodobně existovaly v daných městech již dříve. Termíny se objevují na nápisech pocházejících z řeckých komunit v Mesámbrii, Dionýsopoli, Zóné, Perinthu a Byzantiu, což svědčí o tradičním uspořádání těchto řeckých měst, o jejich politické autonomii a o fungujícím státním aparátu, schopném vydávat veřejná nařízení.\footnote{Příkladem může být nařízení {\em I Aeg Thrace} 3 z Abdéry datované do poloviny století regulující obchod s otroky a hospodářskými zvířaty.} O existenci diplomatických a ekonomických vztahů řeckých měst a thráckých aristokratů svědčí nápis {\em SEG} 49:911.\footnote{Tento nápis pochází pravděpodobně z prostředí řeckého {\em emporia} Pistiros v thráckého vnitrozemí a jedná se dohodu, v jejímž rámci se reguluje obchod mezi odryským panovníkem Kotyem a řeckými městy na pobřeží Egejského moře, a to konkrétně Maróneiou, Apollónií a Thasem. Dana (2015, 247-248) poukazuje na několik desítek {\em graffit} nalezených v okolí Pistiru, často nesoucí řecká jména či věnování řeckým božstvům, což může naznačovat trvalý pobyt osob řeckého původu (Domaradzka 1996, 2002, 2005, 2013). V okolí Pistiru se našlo několik dalších nápisů psaných osobami s řeckými jmény, mimo jiné i slavný dekret z Batkunu {\em IG Bulg} 3,1 1114, o němž hovořím níže. Vše tedy nasvědčuje, že {\em emporion} Pistiros bylo skutečně řeckou obchodní stanicí, jejíž část, nebo možná celá populace, byla řeckého původu.} Jedná se vůbec o první smlouvu ekonomického charakteru pocházející z území Thrákie, kde thrácký panovník vystupuje jako svrchovaná politická autorita, rovná autoritě řeckých měst na pobřeží. Dle současných interpretací šlo o regulaci již probíhajících ekonomických kontaktů, reagující na aktuální či minulé problémy ve snaze o nápravu a kodifikaci vzájemných vztahů (Velkov a Domaradzka 1996, 209-215; Bravo a Chankovski 1999, 279-290; Graninger 2013, 108-109). O vnitřním uspořádání území ovládaného Odrysy však z nápisu není bohužel možné vyčíst nic dalšího, ale vzhledem o ojedinělému výskytu podobného textu je možné říci, že thrácká aristokracie ve 4. st. n. l. nevyužívala nápisy k uplatnění politické moci a veřejné prezentaci svrchované autority stejným způsobem jako bylo obvyklé v řeckých městech na pobřeží. Naopak řecký charakter osídlení {\em emporia} Pistiros nasvědčuje, že thrácký panovník přistoupil na formu komunikace obvyklou pro řeckou komunitu, ať už pro usnadnění srozumitelnosti sdělení, či jako diplomatické gesto. Nápis je psán řecky, upravuje podmínky vzájemné interakce a zajišťuje ochranu řeckých obchodníků na území Thrákie, která je však garantována autoritou odryského panovníka. V případě tzv. pistirského nápisu se tedy jedná o vzájemnou dohodu řecké a thrácké politické autority, k jejíž prezentaci slouží výrazové a kulturně-společenské prvky pouze jedné ze zmíněných stran, a to strany řecké.

\subsection[shrnutí-7]{Shrnutí}

Z uvedeného je patrné, že nápisy soukromého charakteru ve 4. st. př. n. l. sledovaly podobné trendy jako nápisy v 5. st. př. n. l., a není zde patrný žádný společensko-kulturní předěl. Většina epigrafické produkce se koncentrovala na území řeckých komunit podél mořského pobřeží, nicméně se objevují nápisy i ve vnitrozemí v prostředí thrácké aristokracie.

Thrácká komunita využívala nápisy převážně utilitárně pro označení vlastníka, autora či v souvislosti s funkcí předmětu. Tyto nápisy byly zhotoveny pro velmi úzkou skupinu a jejich vlastnictví či vlastnictví předmětu nesoucí nápis poukazovalo na vysoký společenský status majitele. Jejich výsadní postavení v rámci společnosti nasvědčovalo i jejich uložení v hrobce jako součást pohřební výbavy. Oproti tomu nápisy pocházející z řeckých komunit poukazují na uchovávání kulturních tradic a zvyklostí a pouze omezený kontakt s thráckým obyvatelstvem. Převažují veřejně vystavené funerální nápisy, napomáhající vytváření povědomí o jednotné komunitě a kolektivní paměti.

V případě veřejných nápisů se setkáváme s mírným nárůstem jejich celkového počtu. Jednou z příčin rozšíření epigrafické produkce i mimo soukromou sféru může být jak nárůst společenské komplexity, jak v prostředí řeckých komunit, tak i nárůst politické moci kmene Odrysů. Veřejné nápisy v thráckém prostředí slouží jako prostředek komunikační strategie vůči Řekům, a pravděpodobně neslouží pro komunikaci uvnitř thrácké komunity.

\section[charakteristika-epigrafické-produkce-ve-4.-až-3.-st.-př.-n.-l.]{Charakteristika epigrafické produkce ve 4. až 3. st. př. n. l.}

Nápisy datované od 4. do 3. st. př. n. l. pocházejí převážně z řeckého prostředí na pobřeží. I nadále má největší počet nápisů funerální funkci, avšak začínají se objevovat i dedikace věnované božstvům řeckého původu. Nečetné nápisy z vnitrozemí i nadále slouží potřebám thrácké aristokracie a jejich společenská funkce se odlišuje od využití nápisů v řeckých městech. Komunity si i nadále udržují konzervativní charakter a k jejich prolínání dochází ve velmi omezené míře.

\placetable[none]{}
\starttable[|l|]
\HL
\NC {\em Celkem:} 25 nápisů

{\em Region měst na pobřeží:} Abdéra 2, Apollónia Pontská 4, Byzantion 3, Maróneia 3, Mesámbria 3, Perinthos (Hérakleia) 1, Zóné 1 (celkem 17 nápisů)

{\em Region měst ve vnitrozemí:} Beroé (Augusta Traiana) 1, Filippopolis 3, (Hadriánúpolis) 1\footnote{Celkem tři nápisy nebyly nalezeny v rámci regionu známých měst, editoři korpusů udávají jejich polohu vzhledem k nejbližšímu modernímu sídlišti (tři lokality ve vnitrozemí).}

{\em Celkový počet individuálních lokalit}: 14

{\em Archeologický kontext nálezu:} funerální 5, sídelní 3, sekundární 3, neznámý 14

{\em Materiál:} kámen 23 (mramor 16, vápenec 3, jiný 2, neznámý 4), jiný 1, neznámý 1

{\em Dochování nosiče}: 100 \letterpercent{} 3, 75 \letterpercent{} 1, 50 \letterpercent{} 6, 25 \letterpercent{} 2, kresba 4, nemožno určit 9

{\em Objekt:} stéla 23, nástěnná malba 1

{\em Dekorace:} reliéf 8, bez dekorace 17; figurální dekorace 0, architektonické prvky 10 nápisů (vyskytující se motiv: naiskos 7, florální motiv 2, neznámý 1)

{\em Typologie nápisu:} soukromé 20, veřejné 3, neurčitelné 2

{\em Soukromé nápisy:} funerální 14, dedikační 4, vlastnictví 1, jiný 1

{\em Veřejné nápisy:} honorifikační dekrety 1, státní dekrety 1, neznámý 1

{\em Délka:} aritm. průměr 4,24 řádku, medián 2, max. délka 20, min. délka 1

{\em Obsah:} dórský dialekt 1; stoichédon 1; hledané termíny (administrativní 4 - celkem 4 výskyty, epigrafické formule 6 - celkem 6 výskytů, honorifikační 7 - celkem 7 výskytů, náboženské 3 - celkem 4 výskyty, epiteton 0)

{\em Identita:} řecká božstva 2, kolektivní identita 2 - obyvatelé řeckých obcí z oblasti Thrákie, celkem 23 osob na nápisech, 17 nápisů s jednou osobou; max. 2 osob na nápis, aritm. průměr 1,15 osoby na nápis, medián 1; komunita převládajícího řeckého charakteru, jména pouze řecká (60 \letterpercent{} - celkem 15 nápisů), thrácká (8 \letterpercent{} - celkem 2 nápisy), kombinace řeckého a thráckého (0 \letterpercent{}), jména nejistého původu (12 \letterpercent{}); geografická jména 0;

\NC\AR
\HL
\HL
\stoptable

Nápisů datovaných do 4. až 3. st. př. n. l. se dochovalo 25, což představuje úbytek o 80 \letterpercent{} oproti podobné skupině nápisů datovaných do 5. až 4. st. př. n. l. Skupina nápisů ze 4. až 3. st. př. n. l. vykazuje mírný nárůst lokalit v thráckém vnitrozemí v okolí řeckých a makedonských sídel v Pistiru a Seuthopoli a podél hlavních toků, nicméně i přesto se většina epigrafické produkce nachází na pobřeží Černého, Marmarského a Egejského moře, jak dokazuje následující mapa 6.03 v Apendixu 2.

Nápisy datované do 4. až 3. st. př. n. l. pokračují v tradicích ustanovených v předcházejících obdobích a nedochází k zásadní kulturní změně zaznamenané na epigrafické produkci.\footnote{V 58 \letterpercent{} se jedná o funerální nápisy z okolí řeckých měst, ve 32 \letterpercent{} o nápisy dedikační a ve 24 \letterpercent{} o nápisy veřejné. Nosiče nápisů jsou převážně zhotoveny z kamene, sloužící soukromým účelům a veřejnému vystavení, nicméně dva nápisy pocházejí z interiéru hrobky thráckého aristokrata a typologicky odpovídají podobným utilitárním nápisům na cenných předmětech z 5. a 4. st. př. n. l. Nápis {\em SEG} 39:653 nespadá ani do kategorie funerálních či dedikačních nápisů, nicméně je důležitý vzhledem k tomu, že pochází z vnitrozemí, z regionu pozdější Hadriánopole. Nápis byl publikován pouze částečně, a je znám v podstatě pouze jeho text, který zní „{\em Hebryzelmis, syn Seutha, Prianeus}”. Hebryzelmis i Seuthés jsou thrácká aristokratická jména, nicméně badatelé si nejsou jisti původem etnického jména Prianeus, ale obecně ho považují za jméno thráckého kmene z egejské oblasti (Veligianni 1995, 158). Bohužel více informací o nápisu není dostupných, což znemožňuje jakékoliv další závěry.}

\subsection[funerální-nápisy-4]{Funerální nápisy}

Celkem se z této doby dochovalo 13 funerálních nápisů patřící do kategorie primárních funerálních nápisů na stélách. Těchto 13 nápisů pochází výhradně z řeckých měst na mořském pobřeží a vyskytují se na nich pouze řecká jména, podobně jako v předcházejících staletích.

Dva nápisy pocházejí z vnitřní architektury mohylových hrobek, které obě pravděpodobně patřily thráckým aristokratům, soudě dle bohaté pohřební výbavy a jejich umístění na území ovládaném kmenem Odrysů. Jeden nápis pochází z hrobky u moderní vesnice Alexandrovo na dolním toku řeky Hebros a druhý z hrobky u moderní vesnice Kupinovo u Veliko Turnova. Texty jsou vytesány či vyškrábány do stěn vnitřní komory hrobek. V případě hrobky z Alexandrova poukazují na majitele hrobky či zhotovitele hrobky či její výzdoby - nástěnných maleb zobrazující lov divokého kance, do nichž je právě vyryto zmíněné {\em graffito} {\em SEG} 54:628 (Kitov 2004, 45-46). Tento malíř je znám ještě z nápisu z hrobky v Kazanlaku, která bývá datována do 3. st. n. l. (Sharankov 2005, 29-35). Druhý nápis 46:852 se nachází na říčním kameni vestavěném do vnitřní konstrukce hrobky a nese nápis σκιάς nejasného významu. Z výše uvedeného plyne, že použití písma v kontextu thrácké aristokracie je identické s nápisy z 5. a 4. st. př. n. l. a nedochází k zásadním proměnám kulturních zvyklostí.

\subsection[dedikační-nápisy-4]{Dedikační nápisy}

V tomto období se poprvé objevily i tři dedikační nápisy ve vnitrozemí, z celkově čtyř dochovaných nápisů. Tyto nápisy nalezené ve vnitrozemí pocházejí komunit tradičně označovaných jako řecké, jako je tomu v případě nápisu z Pistiru či řecko-thrácké se silnými řeckými konotacemi a aristokratickými vazbami na řeckou kulturu, v případe nápisu ze Sborjanova či z okolí Seuthopole.\footnote{Seuthopolis byla sídlem odryského panovníka Seutha III., kterou si nechal postavit na březích řeky Tonzos v dnešním Kazanlackém údolí ve střední Thrákii. Jednalo se o opevněné sídlo o velikosti 4 ha, postavené dle vzorů rezidencí hellénistických vladařů. Byly zde nalezeny domy řecké typu, pravoúhlé uspořádání domů a ulic, množství řeckých importů, graffit a mimo jiné i řecky psaný nápis, tzv. seuthopolský nápis {\em IG Bulg} 3, 2 1731, o němž podrobněji hovořím níže (Dimitrov a Chichikova 1978; Čičikova 1984; Domaradzka 2005, 299). Nedaleko vesnice Sborjanovo v severovýchodním Bulharsku se našlo sídlo gétských panovníků, často označované jako hlavní město kmene Getů a ztotožňované se sídlem panovníka Dromichaita, Chelis (Stoyanov 1997; 2001, 207-219). I odsud pocházejí graffiti s thráckými jmény (Domaradzka 2005, 298) a dedikační nápis {\em SEG} 55:739.} Nápis {\em SEG} 55:739 z lokality poblíž moderního Sborjanova obsahoval tradiční formuli dedikační nápisů ({\em euchén}) a byl věnován bohyni Fosforos, která je nejčastěji spojována s Artemidou.\footnote{V dalším thráckém městě Kabylé se bohyně Fosforos stala dokonce patronkou města a hlavním vyobrazením na mincích ražených v hellénismu v Kabylé (Janouchová 2013, 103-104). Kult Artemis {\em Fosforos} je znám i z Byzantia, kde se konaly slavnosti Bosporií, doprovázené průvodem s pochodněmi, podobně jako ve 4. st. př. n. l. v Athénách (Janouchová 2013, 97; Lajtar 2000, 39-41: {\em IK Byzantion} 11, nedatovaný nápis z Byzantia).} Zbylé texty neobsahují věnování božstvu, ale pouze krátkou identifikaci dedikanta.\footnote{Dochovaná jména jsou pouze řeckého původu, a to jak v případě jmen dedikantů, tak i jejich rodičů. Krátké texty jsou psány řecky a nevykazují žádné odchylky a nepravidelnosti v užití řečtiny.} Ač byly tyto dedikační nápisy nalezeny v thráckém vnitrozemí, vše nasvědčuje na udržení kontinuity řeckých náboženských tradic a nedokazuje ovlivnění místními thráckými náboženskými představami.

\subsection[veřejné-nápisy-4]{Veřejné nápisy}

Do této skupiny patří celkem tři nápisy, z nichž dva jsou honorifikační dekrety vydané autoritou řecké {\em polis}\footnote{Nápis {\em I Aeg Thrace} 400 pochází z antického města Drys na egejském pobřeží a jedná se o honorifikační dekret pro Polyárata, syna Histiáiova, kterému byla udělena privilegium proxénie, společně s udělením občanství, odpuštěním daní a právo volného vstupu do města. Obyvatele města Drys zastupují v tomto případě archónti, kteří reprezentují politickou autoritu daného města.} a jeden z nich pochází z thráckého vnitrozemí.\footnote{Honorifikační dekret {\em IG Bulg} 3,1 1114 pochází přímo z vnitrozemí z okolí moderní vesnice Batkun v regionu Filippopole, nedaleko řeckého {\em emporia} Pistiros. Text nápisu je znám bohužel jen částečně, nicméně z dostupného textu plyne, že občané neznámého města nechali neznámým bratrům postavit sochu a umístit jí ve svatyni Apollóna, a navíc se zavázali, že bratři budou vyznamenáni věncem při každé slavnosti konané pravděpodobně v dané svatyni. Nápis byl pravděpodobně nalezen u vesnice Batkun v kontextu venkovské svatyně věnované Asklépiovi {\em Zymdrénovi}, která se nachází na svazích pohoří Rodopy nedaleko Filippopole (Tsonchev 1941). Bravo a Chankowski (1990, 296-299) zpochybňují svatyni v Batkunu jako původní místo, kde nápis stál a jako alternativu navrhují kontext řecké komunity v nedalekém řecké {\em emporiu} Pistiros.} Není zcela jasné, kdo nápis vydal, ale usuzuje se, že se mohlo jednat o instituci reprezentující lid nedaleké Filippopole či Seuthopole (Archibald 2004, 886-889). Jedná se tak o první dochovaný dekret z vnitrozemí vydaný jinou politickou autoritou, než byl thrácký panovník, což může poukazovat na pomalu se měnící politické uspořádání v Thrákii.

\subsection[shrnutí-8]{Shrnutí}

Nápisy datované do 4. až 3. st. př. n. l. poukazují na narůstající pronikání epigrafické produkce do thráckého vnitrozemí, které je patrné již u skupiny nápisů datovaných do 4. st. př. n. l. Ve vnitrozemí se nápisy objevují zejména v okolí řeckých a makedonských osídlení, v nichž mohli žít i místní obyvatelé či minimálně se s nimi museli stýkat na každodenní bázi. Nicméně i v okolí těchto sídel si epigrafická produkce udržuje tradiční řecký charakter a dle osobních jmen je do publikační činnosti zapojena pouze populace nesoucí řecká jména a dodržující řecké zvyklosti.

\section[charakteristika-epigrafické-produkce-ve-3.-st.-př.-n.-l.]{Charakteristika epigrafické produkce ve 3. st. př. n. l.}

Nápisy v této době pocházejí převážně z pobřežních oblastí, nicméně nápisy se objevují i v kontextu thrácké aristokracie ve vnitrozemí stejně jako v 5. a 4. st. př. n. l. V řeckých komunitách na pobřeží dochází ke kontaktu s ostatními řecky mluvícími komunitami mimo Thrákii a v malé míře i s thráckým kulturním prostředím. Nárůst počtu veřejných nápisů a výskyt hledaných termínů nasvědčují nárůstu společenské komplexity, a to zejména v řeckém prostředí.

\placetable[none]{}
\starttable[|l|]
\HL
\NC {\em Celkem}: 117 nápisů

{\em Region měst na pobřeží}: Abdéra 5, Agathopolis 1, Anchialos 2, Apollónia Pontská 5, Byzantion 16, Dionýsopolis 6, Doriskos 1, Maróneia 21, Mesámbria 42, Naulochos 1, Odéssos 5, Perinthos (Hérakleia) 1 (celkem 106 nápisů)

{\em Region měst ve vnitrozemí}: Beroé (Augusta Traiana) 4, (Marcianopolis) 1, (Plótinúpolis) 1\footnote{Celkem pět nápisů nebylo nalezeno v rámci regionu známých měst, editoři korpusů udávají jejich polohu vzhledem k nejbližšímu modernímu sídlišti (jedna lokalita s jedním nápisem), či uvádějí jejich původ jako blíže neznámé místo v Thrákii (dva nápisy), či jako nápisy pocházející z území mimo Thrákii (dva nápisy).}

Celkový počet individuálních lokalit: 18

{\em Archeologický kontext nálezu}: funerální 10, sídelní 3, náboženský 2, sekundární 10, neznámý 92

{\em Materiál}: kámen 116 (mramor 103, z toho mramor z Thasu 1, vápenec 3, jiné 1; z čehož je syenit 1; neznámý 9), jiný materiál 1

{\em Dochování nosiče}: 100 \letterpercent{} 19, 75 \letterpercent{} 6, 50 \letterpercent{} 35, 25 \letterpercent{} 28, oklepek 1, kresba 2, nemožno určit 26

{\em Objekt}: stéla 110, architektonický prvek 5, socha 1, nástěnná malba 1

{\em Dekorace}: reliéf 61, malovaná dekorace 2, bez dekorace 54; reliéfní dekorace figurální 13 nápisů (vyskytující se motiv: jezdec 1, stojící osoba 3, sedící osoba 8, skupina lidí 4), architektonické prvky 44 nápisů (vyskytující se motiv: naiskos 22, sloup 5, báze sloupu či oltář 1, florální motiv 10, geometrický motiv 1, architektonický tvar/forma 5)

{\em Typologie nápisu}: soukromé 79, veřejné 36, neurčitelné 2

{\em Soukromé nápisy}: funerální 71, dedikační 5, jiný (jméno autora) 1, neznámý 2

{\em Veřejné nápisy}: nařízení 1, náboženské 2, seznamy 2, honor. dekrety 9, státní dekrety 23\footnote{V určitých případech může docházet ke kumulaci jednotlivých typů textů v rámci jednoho nápisu, či jejich nejednoznačnost neumožňuje rozlišit mezi několika typy a těmto nápisů jsou přiřazeny několikanásobné hodnoty (např. nápis náboženský a zároveň seznam). V těchto případech pak součet všech typů nápisů může přesahovat celkové číslo nápisů.}

{\em Délka}: aritm. průměr 5,56 řádku, medián 3, max. délka 37, min. délka 1

{\em Obsah}: dórský dialekt 26, graffiti 1; hledané termíny (administrativní termíny 25 - celkem 77 výskytů, epigrafické formule 11 - 46 výskytů, honorifikační 19 - 65 výskytů, náboženské 18 - 42 výskytů, epiteton 1 - počet výskytů 1)

{\em Identita}: řecká božstva, pojmenování míst a funkcí typických pro řecké náboženské prostředí, regionální epiteton 1, kolektivní identita 15 termínů, celkem 18 výskytů - obyvatelé řeckých obcí z oblasti Thrákie, ale i mimo ni, kolektivní pojmenování kmenové příslušnosti (Thessalos, Aitólos, Krés), celkem 166 osob na nápisech, 70 nápisů s jednou osobou; max. 12 osob na nápis, aritm. průměr 1,41 osoby na nápis, medián 2; komunita převládajícího řeckého charakteru, jména pouze řecká (75 \letterpercent{}), thrácká (3,41 \letterpercent{}), kombinace řeckého a thráckého (2,56 \letterpercent{}), jména nejistého původu (14,52 \letterpercent{}); geografická jména z oblasti Thrákie 5, geografická jména mimo Thrákii 3;

\NC\AR
\HL
\HL
\stoptable

Celkem se dochovalo 117 nápisů, zejména z řeckých měst či jejich bezprostředního okolí. Z vnitrozemí pocházejí nápisy z lokalit na největších řekách spojujících vnitrozemskou Thrákii s Egejskou oblastí, konkrétně z povodí řek Tonzos, Hebros a Strýmón. Mapa 6.04 v Apendixu 2 zachycuje konkrétní rozložení lokalit v nichž byly nalezeny nápisy. Apollónia Pontská přestává být hlavním producentem nápisů a na její místo nastupuje její jižní soused Mesámbria, odkud pochází 36 \letterpercent{} nápisů z daného období. Dalšími významnými producenty nápisů jsou Maróneia, Byzantion a Odéssos. Archeologický kontext nálezu nápisů většinou neznámý, nicméně zhruba u 14 nápisů byl tento kontext určen jako funerální, tj. pochází z pohřebiště či mohyly. Archeologické lokality dosvědčují náhlé změny poměrů, pravděpodobně související s příchodem keltských kmenů do Thrákie, které měly za následek destrukci některých lokalit, případně jejich úplný zánik, jako v případě {\em emporia} Pistiros (Bouzek {\em et al.} 2016).

Převládajícím materiálem je z 99 \letterpercent{} i nadále kámen: z téměř 91 \letterpercent{} mramor a dále kámen místního původu, jako je vápenec, syenit či varovik. Nápisy tesané do kamene pocházely z velké části z řeckých měst či jejich bezprostředního okolí. Výjimku tvoří několik nápisů pocházejících z thráckého vnitrozemí, a to údolí středního Strýmónu, dále oblast střední Thrákie s lokalitami Seuthopolis a Kabylé. V těchto místech se předpokládá buď přítomnost řeckého či makedonského obyvatelstva nebo při nejmenším velmi intenzivní kontakty thrácky a řecky mluvících komunit. Dochází zde k náznakům podobného využití písma pro regulaci společenských vztahů ve formě, na jakou jsme zvyklí z řeckých obcí na pobřeží.\footnote{V jednom případě se dochoval nápis na jiném materiále než na kameni, a to na fresce uvnitř hrobky z thráckého prostředí.} Podobně jako v předcházejících dvou stoletích se setkáváme s odlišným pojetím písma a jeho kulturně-společenské funkce v rámci řecké a thrácké komunity v malé míře i ve 3. st. př. n. l.

\subsection[funerální-nápisy-5]{Funerální nápisy}

Funerální nápisy tvoří nejpočetnější skupinu soukromých nápisů, podobně jako v předcházejících stoletích. Celkem se dochovalo 71 primárních funerálních nápisů datovaných do 3. st. př. n. l. Do skupiny sekundárních funerálních nápisů patří jeden nápis vyrytý do fresky uvnitř mohylové hrobky.

\subsubsection[primární-funerální-nápisy-2]{Primární funerální nápisy}

Do této skupiny již tradičně patří funerální stély či jiné předměty, jejichž hlavní funkcí bylo označovat místo pohřbu a připomínat zemřelého. Celkem se jich dochovalo 71 a pocházely především z řeckých obcí na mořském pobřeží. Největší koncentrace nápisů pochází z černomořské Mesámbrie s 23 nápisy, z Byzantia a z egejské Maróneie se 17 nápisy. Mimo pobřežní oblasti byl nalezen jeden náhrobních nápis z okolí moderní vesnice Červinite Skali v oblasti středního toku Strýmónu.

Skupina nápisů pocházejících z pobřežních oblastí naznačuje, že nápisy z řeckých měst a jejich bezprostředního okolí nadále udržují funerální ritus v podobě, v jaké se projevoval v předcházejících staletích.\footnote{Texty funerálních nápisů byly převážně jednoduché, v rozsahu dvou až tří řádků. Nejdelší text měl 11 řádků, ale to se jednalo spíše o výjimku. Typický nápis obsahoval jméno zemřelého a jméno jeho rodiče či partnera. Pouze výjimečně nápis obsahoval informace z nebožtíkova života či ve čtyřech případech vyjádření zármutku pozůstalých zhotovené metricky. Texty tedy spíše následovaly tradici jednoduchých textů poukazujících pouze na osobu zemřelého, jak bylo obvyklé v předcházejících stoletích.} Analýza osobních jmen potvrdila, že 84 \letterpercent{} osobních jmen je řeckého původu, 13 \letterpercent{} není možné určit a pouze 3 \letterpercent{} jmen jsou pravděpodobně thráckého původu. Thrácká jména se vyskytovala celkem na třech nápisech, z nichž dva pravděpodobně pocházely z Odéssu a jeden z Byzantia. V jednom případě se jednalo o ženu, v jednom případě o manželský pár-sourozence a v jednom případě o muže. Z toho plyne, že funerální stély tedy byly využívány výhradně obyvateli se jmény řeckého původu. Další vyjádření identity jako kolektivní pojmenování odkazují na řecký původ zemřelých či jejich rodinných příslušníků s kořeny v okolí thráckého regionu.\footnote{Patří sem například termíny jako Filippeus, Lýsimacheus, Hérakleótés, a pak dále Krés a Rhodios. Geografické pojmy se vyskytují jen v jednom případě a poukazují na původ jistého Apollónia z Babylónu. Celkem šest lidí udávalo svůj geografický původ, který byl z více než poloviny mimo oblast Thrákie, nicméně jejich jména byla řeckého původu.} Ač nečetné, jedná se o zdokumentované případy migrace části obyvatel. Nelze však tvrdit, že by ve 3. st. př. n. l. migrace obyvatelstva byla častější než ve 4. st. př. n. l. pouze na základě většího výskytu geografických termínů. Kladení důrazu na geografický původ může souviset s narůstající propojeností hellénistického světa, která je jedním z průvodních znaků hellénismu a se snahou udržet si svou identitu i v rámci nové komunity, a proto se tato vyjádření začala objevovat častěji.

Nejen osobní jména a vyjádření identity, ale i hledané termíny poukazují na udržování tradic řeckého kulturního prostředí i v rámci funerálního ritu.\footnote{V osmi případech vyskytují se zde typické invokační formule ({\em chaire}). Pro popis hrobky se ve dvou případech používá taktéž zcela typické vyjádření ({\em tymbos}), z čehož v jednom případě na nápise {\em IK Byzantion} 305 byl termín upřesněn jako navršený hrob čili mohyla ({\em chóstos tymbos}). Dále se zde vyskytuje jednou termín pro hrob samotný ({\em tafos}).} Všechny dochované termíny jsou použity v jejich původním významu a v kontextu v jakém bychom je mohli najít i ve zbytku řecky mluvícího světa té doby. Veškeré důkazy tedy poukazují na přetrvávání tradičních funerálních zvyklostí a jejich projevů na nápisech bylo ve 3. st. př. n. l. omezené pouze na řecké komunity.

\subsubsection[sekundární-funerální-nápisy-2]{Sekundární funerální nápisy}

Nápis pochází z thrácké aristokratické hrobky z Kazanlackého údolí na řece Tonzos, na území tradičně ovládaném kmenem Odrysů. Dle bohaté pohřební výbavy se usuzuje, že se mohlo jednat o členy thrácké aristokracie, podobně jako v podobných případech v předcházejících stoletích (Sharankov 2005, 29-35). {\em Dipinti} {\em SEG} 58:703 bylo nalezeno na fresce v hrobové komoře a má dvě části. První části nese jméno Seutha, syna Rhoigova a souvisí s ním kresba mladého muže, která je umístěna v bezprostřední blízkosti. Druhá část je tzv. autorský podpis malíře jménem Kozimazés, který je znám jako zhotovitel další fresky {\em SEG} 58:674 v hrobce v Alexandrovu, která je datovaná do 4. až 3. st. př. n. l. a o níž jsem hovořila dříve. Charakterem tento nápis odpovídá podobným skupinám funerálních nápisů z 5. a 4. st. př. n. l., kdy thrácká aristokracie využívala písmo zejména utilitárně pro svou soukromou potřebu vně komunity, tedy odlišným způsobem, než můžeme vidět v řeckých komunitách.

\subsection[dedikační-nápisy-5]{Dedikační nápisy}

Dedikačních nápisů se dochovalo celkem pět a, až na jednu výjimku ze Seuthopole, pocházejí všechny z regionu řeckých měst na pobřeží. Věnování byla převážně řeckým božstvům jako je Zeus, Dionýsos a Démétér, případně tehdejším významným panovníkům.\footnote{V jednom případě byl nápis věnován Diovi a králi Filippovi, s titulem zachránce ({\em sótér}). Tento nápis {\em I Aeg Thrace} 186 pochází z Maróneie a králem může být myšlen jak makedonský Filip II., tak Filip V. (Loukopoulou {\em et al.} 2005, 372). Další podobná dedikace {\em IK Sestos} 39 určená králi Filippovi pochází z Lýsimachei na Thráckém Chersonésu, nicméně v tomto případě se jedná spíše o makedonského krále Filippa V. (Krauss 1980, 92).}

Výskyt osobních jmen poukazuje na převahu řeckého prvku: řeckých osobních jmen se vyskytlo celkem šest, zatímco thrácká jména se vyskytla pouze jedno, a to na nápise {\em IG Bulg} 3,2 1732 pocházejícím z vnitrozemské Seuthopole, které věnoval Amaistas, syn Medista, kněz Dionýsova kultu.\footnote{Nápisy {\em IG Bulg} 3,2 1731 a 1732 dosvědčují existenci řeckých kultů na území Seuthopole, nebo kultů nesoucí řecká jména. Epigraficky je zde doložen jak kult Dionýsa, tak svatyně Velkých božstev ze Samothráké, nicméně archeologicky se existenci těchto kultů nepodařilo zcela prokázat, ač zde byla nalezena terakotová soška Kybelé a sošky inspirované uměním východního Středomoří (Nankov 2007, 63; Barrett a Nankov 2010, 17).} Tento fakt svědčí o existenci relativně zavedených náboženských praktik, a to zejména na území řeckých měst, ale v případě Seuthopole i mimo ně. Dedikační nápisy se vyskytují převážně v kontextu řeckých komunit. Jedinou výjimkou je nápis pocházející z vnitrozemské Seuthopole, která je považována za původně thrácké osídlení se silnou přítomností řeckého či makedonského prvku (Nankov 2011, 120). Není tedy překvapivé, že tento v thráckém prostředí relativně ojedinělý dedikační nápis pochází z komunity, která nebyla čistě thrácká, ale docházelo zde ke kulturnímu a náboženskému synkretismu.

\subsection[veřejné-nápisy-5]{Veřejné nápisy}

Ve 3. st. př. n. l. je možné sledovat narůstající využití psaného slova za účelem uplatňování autority existujících autonomních politických jednotek na území Thrákie. Celkem se dochovalo 36 veřejných nápisů, reprezentujících až 31 \letterpercent{} všech nápisů z daného období.\footnote{Oproti 4. st. př. n. l. je možné sledovat nárůst až na čtyřnásobek celkového počtu veřejných nápisů.} Veřejné nápisy až na tři výjimky pocházejí z řeckých měst na pobřeží, a z Mesámbrie na černomořském pobřeží dokonce až 18 exemplářů.\footnote{Tento vysoký poměr veřejných ale i soukromých nápisů z Mesámbrie z 3. st. př. n. l. může být dán jak stavem archeologických výzkumů, které se odehrávaly zejména v 60. až 80. létech 20. století, ale částečně i ekonomickým a politickým významem, který Mesámbria v průběhu 3. st. př. n. l. měla (Velkov 1969; 2005; Venedikov 1969; Ognenova-Marinova {\em et al.} 2005).} Dva nápisy z vnitrozemí pocházejí z původně thráckých osídlení, které hrály důležitou roli i za makedonských místodržících, konkrétně ze Seuthopole a Kabylé. Typologicky se jedná 31 dekretů vydaných politickou autoritou ({\em pséfisma}), z čehož devět bylo honorifikačních nápisů.

Dekrety jsou nejčastějších dochovaným typem veřejného nápisu, protože sloužily jako jeden z projevů suverenity politické autority, která je vydala.\footnote{Veřejné nápisy byly tesány do kamene a byly určeny pro veřejné vystavení. Jejich primárním účelem bylo jednak informovat o daném nařízení či usnesení politické autority, kterou v případě řeckých {\em poleis} byl lid, zastoupený termíny {\em démos} a {\em búlé,} a v případě thrácké aristokracie kmenový vůdce, označovaný termínem {\em basileus}. Předpokládá se, že každý si mohl nařízení přečíst, pokud toho byl schopen, a případně se na něj odvolat, pokud by docházelo k jeho porušování.} Zajišťovaly tak osobám, které spadaly pod vliv dané autority ochranu, základní práva a určitou prestiž v případě honorifikačních nápisů, výměnou za jejich loajalitu. Dále se částečně dochovalo jedno nařízení z Abdéry o platbách za dodávání pravdivých informací státu ({\em I Aeg Thrace} 2), jeden veřejný nápisy s náboženskou tematikou, seznam osob nejistého významu a jeden blíže neurčený nápis veřejného charakteru.

Ve zvýšené míře se na nápisech nachází administrativní termíny, a to celkem 25 termínů v 77 výskytech.\footnote{Předně se jedná o instituce a funkce zajišťující chod státního aparátu či symbolizující politickou autoritu samotnou jako je {\em démos}, {\em búlé}, {\em archón}, {\em stratégos}, {\em basileus} atd. Nejčastěji se opakujícím termínem byl {\em démos}, který se objevil 17krát, dále {\em polis} s 11 výskyty a {\em búlé} s šesti výskyty.} Častý výskyt tradičních termínů vychází z ustálené formy dekretů a formulí, které se používaly v celém řeckém světě ve velmi podobné formě. Texty nápisů jsou často velmi popisné a detailně zmiňují veškeré situace, v nichž je daný text nápisů platný a k čemu konkrétně dává pravomoci. Dále se zde vyskytuje celá řada specializovaných povolání, funkcí a referencí na existující společenskou hierarchii, což svědčí o narůstající komplexitě komunit, které nápisy vydávaly (Tainter 1988, 106-108).\footnote{Mezi objevující se nová povolání a funkce patří zejména funkce spojené s výkonem chodu státu a zajištění dodržování publikovaných nařízení, jako například {\em tamiás}, {\em argyrotamiás}, {\em gymnasiarchés}, {\em polítés}, {\em oikonomos}, {\em archón}, {\em stratégos}, {\em basileus}, {\em kéryx}, {\em theóros} či {\em presbys}.} Celá řada specializovaných termínů a formulí odkazuje na komplexní procedury spojené s pořizováním nápisů a jejich veřejným vystavováním. Jednalo se tedy pravděpodobně o ustanovenou proceduru, organizovanou obcí, podobně jak ji známe i z jiných řeckých komunit té doby. Zejména u honorifikačních nápisů z řeckých obcí je možné pozorovat výskyt ustálených formulí, které reflektují ustálené administrativní procedury a existující instituce pověřené epigrafickou produkcí. Ač je obsah formulí stejný, či velmi podobný, jejich konkrétní forma se liší město od města, často i v témže městě.\footnote{Text dekretů se typicky sestává z uvedení instituce udělující privilegia ({\em búlé}, {\em démos}, {\em polis} či kolektivní pojmenování občanů města), komu jsou udělena a proč, dále následuje výčet a specifika udělených privilegií a praktické informace o financování a veřejném vystavení nápisu. V textu je mnoho variant počínaje pořadím privilegií, udělenými poctami, způsobem veřejného vystavení, institucemi vydávajícím dekret, a dále i místní varianty použití určitých slovesných tvarů či infinitní větné konstrukce ({\em edoxe} vs. {\em dedochthai}, {\em edókan} vs. {\em dedosthai}). Příkladem může být udělení hostinného přátelství ({\em proxeniá}), které se objevilo celkem osmkrát.} Nejjednotnější formu mají nápisy z Odéssu, kde většina nápisů vycházela ze stejného vzoru, což poukazuje na zavedený systém byrokratických procedur v Odéssu. Nápisy z ostatních řeckých měst obsahují lehce pozměněné formulace, a lokální varianty, což nasvědčuje o velmi podobném politickém uspořádání a totožných procedurách v řeckých městech, avšak tento fakt poukazuje na absenci jednotné autority a normy, která by sjednocovala formální stránku nápisů. Řecká města tedy i ve 3. st. př. n. l. vystupovala a jednala jako autonomní politické jednotky, jejichž procedurální postupy byly do značné míry konzervativní.

Veřejné nápisy pocházející z thráckého kontextu přejímají jazyk a formu řeckých usnesení, ač jejich obsah nemá příliš paralel v celém antickém světě. Konkrétně se jedná o nápisy {\em IG Bulg} 3,2 1731 ze Seuthopole a {\em Kabyle} 2 z Kabylé, které byly v této době původně thráckými komunitami se silnou makedonskou přítomností, a to se odráží i na projevu politické autority na nápisech.\footnote{Nápis {\em IG Bulg} 3,2 1731 ze Seuthopole vydala vdova po thráckém panovníkovi Seuthovi III. Bereníké, která tak formou přísahy řeší nastalou situaci s paradynastou Spartokem z Kabylé, které se nachází zhruba 90 km po toku řeky Tonzos. Bereníké uzavírá se Spartokem dohodu, pravděpodobně za účelem uchování postavení a moci pro své syny Hebryzelma, Térea, Satoka a Sadala (Ognenova-Marinova 1980, 47-49; Velkov 1991, 7-11; Calder 1996, 172-173; Tacheva 2000; Archibald 2004. 886). Částečně dochovaný nápis {\em Kabyle} 2 (Velkov 1991, 11-12) byl vydán pravděpodobně v polovině 3. st. př. n. l. politickou autoritou řecké obce. Velkov navrhuje jako nejpravděpodobnější místo vzniku Mesámbrii a jednalo by i se tak o vzájemnou smlouvu mezi Kabylé a blízkou Mesámbrií na černomořském pobřeží. Po polovině století Kabylé i Mesámbria pravděpodobně spadaly pod vliv keltského panovníka Kavara, který v obou městech nechal razit mince pro svou potřebu (Draganov 1993, 75-86, 107).} Podobně jako v případě nápisu z Pistiru ze 4. st. př. n. l. se mohlo jednat o komunikační strategii thrácké aristokracie vůči řecky mluvícímu obyvatelstvu, či o snahu o zavedení zvyklostí typických pro řeckou komunitu, které však neměly dlouhého trvání. Vzhledem k ojedinělému užití nápisů ve veřejné funkci nejde příliš mluvit o dlouhodobém přejímání zvyků či organizace, ale spíše o dokumenty vzniklé jako reakce na aktuální politickou situaci a snahu se s ní vyrovnat způsobem obvyklým pro jednu zúčastněnou stranu. Absence institucí a úřadů typických pro řecké {\em poleis} poukazuje na specificitu thrácké politické autority a společenské organizace, kde hlavní roli hrál panovník a nejbližší okruh aristokratů, a nikoliv byrokratický aparát a státní instituce.

\subsection[shrnutí-9]{Shrnutí}

Ve 3. st. př. n. l. i nadále většina epigrafické produkce pochází z řeckých měst na pobřeží a vykazuje všechny charakteristické rysy typické pro řecký svět té doby. Zvýšení celkového počtu veřejných nápisů z těchto komunit může nasvědčovat na zavádění nových procedur a institucí do měst, či alespoň větší míru využívání nápisů pro účely politické organizace a vedení administrativy. Naopak snížení počtu soukromých nápisů může poukazovat na jistou míru nejistoty, které museli obyvatelé čelit, což mohlo souviset s invazí keltských kmenů na počátku 3. st. př. n. l. Na poměrně bouřlivý charakter této doby poukazují i destrukční vrstvy z četných archeologických nalezišť, mimo jiné i z Pistiru.

Ve vnitrozemí se nápisy objevují pouze v kontextu aristokratických kruhů a slouží zejména k upevnění společenské pozice v rámci komunity, ale v ojedinělých případech slouží i jako prostředek komunikace s řeckými či makedonskými partnery. Výjimečnou roli zaujímají multikulturní komunity v Seuthopoli a Kabylé, které vycházejí z thráckých kořenů, avšak v určité míře přijímají i prvky tradičně označované jako řecké či makedonské. Dalším potenciálním místem kontaktu kultur je Hérakleia Sintská, původně makedonská vojensko-obchodní stanice u středního toku řeky Strýmónu. Epigrafické důkazy však i nadále poukazují spíše na uzavřený charakter komunit a přetrvávání tradičních kulturních hodnot v jejich rámci.

\section[charakteristika-epigrafické-produkce-ve-3.-až-2.-st.-př.-n.-l.]{Charakteristika epigrafické produkce ve 3. až 2. st. př. n. l.}

Nápisy datované do 3. až 2. st. př. n. l. pocházejí převážně z řeckého kulturního prostředí. Dochází k nárůstu produkce veřejných nápisů, stejně tak k výskytu hledaných termínů. Prolínání řeckých a thráckých onomastických tradic je možné sledovat v omezené míře, dále i rozšíření lokálních, řeckých a egyptských náboženských tradic.

\placetable[none]{}
\starttable[|l|]
\HL
\NC {\em Celkem:} 59 nápisů

{\em Region měst na pobřeží:} Abdéra 5, Apollónia Pontská 4, Byzantion 12, Dionýsopolis 1, Lýsimacheia 1, Maróneia 10, Mesámbria 9, Odéssos 7, Perinthos (Hérakleia) 3 (celkem 52 nápisů)

{\em Region měst ve vnitrozemí:} Beroé (Augusta Traiana) 1, Filippopolis 1, údolí středního toku řeky Strýmónu 3\footnote{Celkem dva nápisy nebylo nalezeny v rámci regionu známých měst, editoři korpusů udávají jejich polohu vzhledem k nejbližšímu modernímu sídlišti (jedna lokalita s jedním nápisem), či uvádějí jejich původ jako blíže neznámé místo v Thrákii (jeden nápis).}

{\em Celkový počet individuálních lokalit}: 14

{\em Archeologický kontext nálezu:} funerální 4, sídelní 1, náboženský 1, sekundární 4, neznámý 48

{\em Materiál:} kámen 56 (mramor 48, jiný 2, neznámý 6), keramika 1, kov 1, neznámý 1

{\em Dochování nosiče}: 100 \letterpercent{} 3, 75 \letterpercent{} 10, 50 \letterpercent{} 8, 25 \letterpercent{} 14, kresba 1, ztracený 2, nemožno určit 21

{\em Objekt:} stéla 50, socha 1, nádoba 1, architektonický prvek 4, jiné 2, neznámý 1

{\em Dekorace:} reliéf 29, malovaná dekorace 1, bez dekorace 29; figurální 8 nápisů (vyskytující se motiv: skupina lidí 1, sedící postava 2, stojící postava 1, funerální scéna 2, jezdec 1), architektonické prvky 21 nápisů (vyskytující se motiv: naiskos 9, florální motiv 7, sloup 1, báze sloupu či oltář 3, věnec 1, architektonický tvar/forma 3)

{\em Typologie nápisu:} soukromé 37, veřejné 20, neurčitelné 2

{\em Soukromé nápisy:} funerální 30, dedikační 6, vlastnictví 1, jiné (jméno autora) 1\footnote{Vzhledem ke kumulaci typů je celkový součet vyšší než počet soukromých nápisů.}

{\em Veřejné nápisy:} náboženské 1, seznamy 1, honorifikační dekrety 5, státní dekrety 11, funerální na náklady obce 1, neznámý 1

{\em Délka:} aritm. průměr 5,08 řádku, medián 3, max. délka 25, min. délka 1

{\em Obsah:} dórský dialekt 7, graffiti 1; hledané termíny (administrativní termíny 11 - celkem 39 výskytů, epigrafické formule 7--20 výskytů, honorifikační 11--22 výskytů, náboženské 14--20 výskytů, epiteton 1 - počet výskytů 1)

{\em Identita:} řecká božstva, místní božstva, egyptská božstva, kolektivní identita 3 - obyvatelé řeckých obcí, z toho z oblasti Thrákie 1, mimo Thrákii 2, celkem 71 osob na nápisech, 34 nápisů s jednou osobou; max. 6 osob na nápis, aritm. průměr 1,2 osoby na nápis, medián 1; komunita převládajícího řeckého charakteru, jména pouze řecká (52 \letterpercent{} - celkem 31 nápisů), thrácká (3,38 \letterpercent{} - celkem 2 nápisy), kombinace řeckého a thráckého (5,08 \letterpercent{} - 3 nápisy), pouze římská jména (1,69 \letterpercent{} - 1 nápis), jména nejistého původu (16,94 \letterpercent{}); geografická jména míst v Thrákii 1;

\NC\AR
\HL
\HL
\stoptable

Celkem se dochovalo 59 nápisů, které i nadále pocházejí převážně z řeckých komunit na pobřeží a jen minimum nápisů je nalezeno v thráckém vnitrozemí, jak dokazuje mapa 6.04 v Apendixu 2. Nápisy z vnitrozemí pocházejí z okolí Kazanlackého údolí a středního toku Strýmónu, tedy oblastí spojovaných se zvýšenou přítomností makedonských či thráckých vojáků v řeckých službách (Nankov 2012; 2015).

Převládajícím materiálem je již tradičně kámen, nápisy na kovu a na keramice se dochovaly vždy po jednom exempláři. V obou případech nápisů na jiném nosiči, než na kameni se jedná o identifikaci autora předmětu či vnitřní dekorace hrobky v kombinaci se jménem majitele předmětu, tedy podobné použití jako v 5. až 3. st. př. n. l. Přes polovinu nápisů představují funerální nápisy, ale setkáváme se i s označením vlastnictví z prostředí thrácké aristokracie.\footnote{Do této kategorie spadá nápis {\em IG Bulg} 5 5638bis nalezený na amfoře v antickém městě Kabylé na středním toku řeky Tonzos. Kabylé bylo známo jako makedonské vojensko-obchodní osídlení založené na místě dřívějšího thráckého sídliště (Handzhijska a Lozanov 2010, 260-263). Nápis představuje pravděpodobně jméno majitele či objednatele amfory, Sadalás, syn Téreův. Obě jména jsou typicky thrácká, většinou patřící thrácké aristokracii (Dana 2014, 298-301; 355-361).}

\subsection[funerální-nápisy-6]{Funerální nápisy}

Celkem se dochovalo 30 funerálních nápisů datovaných do 3. až 2. st. př. n. l., které pocházejí převážně z řecké komunity na pobřeží a z řecké či makedonské komunity v okolí Hérakleie Sintské, jak dokazuje přítomnost zejména řeckých jmen zemřelých osob.\footnote{Jediné jméno, které je možné průkazně spojovat s thráckým původem, je jméno Amatokos z {\em IK Byzantion} 325, které je použito jako jméno rodiče Hermia.} Charakteristika nápisů je totožná s funerálními nápisy pocházející z řeckých komunit 5. až 3. st. př. n. l. Za zvláštní pozornost nicméně stojí skupina tří funerálních nápisů z Hérakleie Sintské, na nichž se dochovala jména celkem pěti osob, z čehož byly čtyři ženy.\footnote{Jednotlivé osoby byly identifikovány nejen pomocí osobního jména, ale i pomocí údajů o rodičích a partnerech a veškerá dochovaná jména jsou jména řeckého či makedonského původu. Jiná vyjádření identity, či podoby jazyka, která by pomohla komunitu lépe zařadit, se nedochovala.} Nosiče nápisů byly ve dvou případech vyrobeny z místně dostupného materiálu jako je tuf, vápenec a varovik, ale uchovávaly si tradiční vzhled jednoduchých funerálních stél s akroteriem, figurální funerální scénou a v jednom případě písmeny malovanými červenou barvou. Zdroj materiálu byl sice místní, nicméně provedení odkazuje na techniky a motivy tradičně používané v rámci řecké či makedonské komunity. Z archeologických zdrojů však víme, že v době hellénismu byla v Hérakleii Sintské založena pravděpodobně makedonská vojensko-obchodní stanice, která se později rozrostla na město (Nankov 2015, 7-10, 22-27). Není zcela jasné, zda se jednalo o osídlení čistě makedonské, či bylo obývané jak Makedonci, tak Thráky, jak bývalo obvyklé u měst zakládaných Filippem II. (Adams 2007, 9-11). V současné době zde neustále probíhají archeologické výzkumy, a tak je možné, že se do budoucna objeví ještě více důkazů. Zatím je však zřejmé, že mimo pobřežní oblasti byl zvyk stavění náhrobních kamenů ve 3. až 2. st. př. n. l. rozšířen pouze v oblasti obývané řeckými či makedonskými osadníky, a nevyskytoval se v čistě thrácké komunitě.

Podobně jako v 5. a 4. st. př. n. l. je písmo v kontextu thrácké aristokracie využíváno pro velmi specifický účel a v okruhu velmi omezeného počtu lidí. Hlavním účelem je ztotožnit majitele, který patřil do okruhu thrácké aristokracie, či zhotovitele, který mohl být jak thráckého, případně řeckého původu. V případě nápisu na kovovém předmětu {\em SEG} 59:759 se jedná o jméno tvůrce na zlatém diadému, který se nalezl uvnitř hrobky patřící pravděpodobně ženě. Jméno zhotovitele předmětu je řeckého původu a používá typicky řeckou formuli {\em epoi{[}é{]}sen}, tedy zhotovil Démétrios (Manov 2009, 27-30).\footnote{Na témže diadému se nachází ještě pravděpodobně thrácké jméno Kortozous v genitivu singuláru, u nějž není jisté, zda patřilo muži či ženě. Manov usuzuje, že je to jméno majitele diadému, a pokud jím byla žena pohřbená v hrobce, kde byl předmět nalezen, pak diadém mohl patřit právě jí. Nejedná se tedy o primárně funerální nápis, ale o předmět osobní potřeby, který byl uložen do hrobu po smrti majitele.} Pořizování předmětů s nápisy nicméně stále nepatřilo k běžnému standardu ani mezi thráckými aristokraty, natož mezi běžnou thráckou populací a přístup k písmu byl odlišný v rámci řecké a thrácké komunity, což dokazuje i nadále pokračující absence pohřební stél z thráckého kontextu.

\subsection[dedikační-nápisy-6]{Dedikační nápisy}

Zvyk věnovat stély s nápisy se i nadále na přelomu 3. a 2. st. př. n. l. vyskytoval v řeckých komunitách na pobřeží. Celkem se dochovalo šest dedikačních nápisů, které pocházely z území řeckých měst a věnování provedly osoby nesoucí téměř výhradně řecká jména. Věnování byla určena božstvům řeckého a egyptského původu, jako je Ísis, Sarápis a Anúbis.\footnote{Věnování Ísidě, Sarápidovi a Anúbidovi na nápise {\em IG Bulg} 1,2 322ter z černomořské Mesámbrie, a Ísidě a Afrodíté na nápise {\em Perinthos-Herakleia} 42 z Perinthu.} Rozšíření kultu egyptských božstev v Thrákii bývá vysvětlováno zvýšenou přítomností hellénistických vojsk a politického vlivu Ptolemaiovců v oblasti Perinthu (Tacheva-Hitova 1983, 54-58; Barrett a Nankov 2010, 17). I nadále nemáme důkazy o zapojení thráckého obyvatelstva a thráckého náboženství do procesu epigrafické produkce.

\subsection[veřejné-nápisy-6]{Veřejné nápisy}

Celkem se dochovalo 20 nápisů spadající do kategorie veřejných nápisů: převážná většina z nich byly dekrety vydávané v rámci řeckých městských států na pobřeží, přičemž šest dekretů pochází z Odéssu a pět z Maróneie. Politickou autoritu jednotlivých měst zastupovaly orgány jako {\em démos} a {\em búlé} a nejčastější druhem dokumentu jsou honorifikační dekrety vystavené pro význačné osoby či osoby, které se výjimečným způsobem zasloužili o udělení poct. Dle dochovaných osobních jmen se většinou jednalo o muže řeckého původu. V několika případech známe jejich mateřskou obci: Athény, Kallatis, Chersonésos a Antiocheia.\footnote{{\em I Aeg Thrace} 172, {\em IG Bulg} 1,2 13ter, {\em IG Bulg} 1,2 39, {\em IG Bulg} 1,2 41. V případě seznamu osob {\em Perinthos-Herakleia} 62 z Perinthu se dochovala jména šesti mužů řeckého původu. Svou identitu udávají pomocí osobního jména a jména otce, a dále specifikují svůj původ: tři muži pocházejí z fýly {\em Théseis}, dva z fýly {\em Basileis} a jeden z města Byzantion.}

Texty honorifikačních nápisů vycházely ze stejného základu, nicméně každý nápis byl přizpůsoben konkrétní situaci. Konkrétní znění dekretů se lišilo město od města, což poukazuje na jejich politickou samostatnost a nezávislý vývoj epigrafických formulí, které nicméně vycházejí ze společného základu. Součástí textu byly občas i podmínky zveřejnění nápisu, což pro řecké komunity zpravidla bývalo umístění veřejného nápisu do svatyně patrona daného města, jak je typické i pro jiné řecké komunity.\footnote{Text nápisů {\em IG Bulg} 1,2 41 a 42 z Odéssu nařizuje umístit honorifikační nápis v podobě sochy do svatyně samothráckých božstev a do svatyně anonymního božstva.}

\subsection[shrnutí-10]{Shrnutí}

Nápisy datované do 3. a 2. st. př. n. l. nezaznamenávají změnu trendu nastaveného v předchozích stoletích. Řecká a thrácká komunita interagují pouze v omezené míře a udržují si své tradiční zvyklosti, alespoň soudě dle dochovaných nápisů. Poprvé se v této době objevují projevy rozšíření egyptského náboženství do Thrákie, což můžeme vidět zejména na Řeky osídleném pobřeží a v jihovýchodní části Thrákie, kde získali určitý politický vliv Ptolemaiovci.

\section[charakteristika-epigrafické-produkce-ve-2.-st.-př.-n.-l.]{Charakteristika epigrafické produkce ve 2. st. př. n. l.}

Nápisy datované do 2. st. př. n. l. pocházejí výhradně z měst na pobřeží. Mezi oblasti, kde se nápisy začínají objevovat nově, patří Thrácký Chersonésos a lokality na pobřeží Marmarského moře. Téměř pětinu nápisů představují nápisy veřejné, a to zejména dekrety vydávané politickou autoritou. Dochází také k nárůstu výskytů hledaných slov, nápisy se stávají delší a obsahově komplexnější. Celkově dochází k většímu otevírání původně řeckých komunit na pobřeží a k výskytu nových prvků. Zároveň s tím ale dochází k útlumu epigrafických aktivit thrácké aristokracie ve vnitrozemí.

\placetable[none]{}
\starttable[|l|]
\HL
\NC {\em Celkem:} 115 nápisů

{\em Region měst na pobřeží:} Abdéra 7, Ainos 1, Anchialos 1, Apollónia Pontská 2, Bisanthé 1, Bizóné 3, Byzantion 69, Dionýsopolis 1, Lýsimacheia 1, Maróneia 15, Mesámbria 5, Odéssos 2, Perinthos (Hérakleia) 1, Sélymbria 1, Séstos 1, Topeiros 1 (celkem 112)

{\em Region měst ve vnitrozemí:} 0\footnote{Celkem tři nápisy byly nalezeny mimo území Thrákie, avšak editoři korpusů je vzhledem k jejich obsahu zařadili mezi nápisy pocházející z Thrákie.}

{\em Celkový počet individuálních lokalit}: 23

{\em Archeologický kontext nálezu:} sídelní 3, náboženský 4, sekundární 12, neznámý 96

{\em Materiál:} kámen 113 (mramor 108, z toho mramor z Prokonnésu 2, vápenec 1, jiné 1), keramika 1, neznámý 1

{\em Dochování nosiče}: 100 \letterpercent{} 6, 75 \letterpercent{} 8, 50 \letterpercent{} 12, 25 \letterpercent{} 11, oklepek 1, kresba 1, nemožno určit 76

{\em Objekt:} stéla 106, architektonický prvek 5, socha 1, nádoba 1;

{\em Dekorace:} reliéf 78, bez dekorace 37; reliéfní dekorace figurální 55 nápisů (vyskytující se motiv: jezdec 1, stojící osoba 1, sedící osoba 1, skupina lidí 1, funerální scéna/symposion 8, jiné 1), architektonické prvky 29 nápisů (vyskytující se motiv: naiskos 3, sloup 1, báze sloupu či oltář 4, věnec 1, florální motiv 11, geometrický motiv 0, architektonický tvar/forma 4, jiné 1)

{\em Typologie nápisu:} soukromé 86, veřejné 25, neurčitelné 4

{\em Soukromé nápisy:} funerální 80, dedikační 8, jiné (jméno autora) 1\footnote{V určitých případech může docházet ke kumulaci jednotlivých typů textů v rámci jednoho nápisu, či jejich nejednoznačnost neumožňuje rozlišit mezi několika typy. V těchto případech pak součet všech typů nápisů může přesahovat celkové číslo nápisů.}

{\em Veřejné nápisy:} náboženské 1, seznamy 2, honorifikační dekrety 7, státní dekrety 15, funerální 2\footnote{Možnost kombinace kategorií, součet všech typů může přesahovat celkové číslo pro danou kategorii.}

{\em Délka:} aritm. průměr 7,93 řádku, medián 2, max. délka 107, min. délka 1

{\em Obsah:} dórský dialekt 10; hledané termíny (administrativní termíny 38 - celkem 125 výskytů, epigrafické formule 16 - 44 výskytů, honorifikační 27 - 91 výskytů, náboženské 26 - 54 výskytů, epiteton 7 - počet výskytů 7)

{\em Identita:} řecká božstva 12 pojmenování, egyptská božstva 2 pojmenování, římská božstva 2 pojmenování, dále názvy míst a funkcí typických pro řecké náboženské prostředí, regionální epiteton 7, kolektivní identita 15 termínů, celkem 25 výskytů - obyvatelé řeckých obcí z oblasti Thrákie, ale i mimo ni, kolektivní pojmenování kmenové příslušnosti (Bíthýnos, Thráx), kolektivní pojmenování Římanů (Rómaios), celkem 186 osob na nápisech, 78 nápisů s jednou osobou; max. 25 osob na nápis, aritm. průměr 1,61 osoby na nápis, medián 1; komunita převládajícího řeckého charakteru, jména pouze řecká (60 \letterpercent{}), pouze thrácká (1,73 \letterpercent{}), pouze římská (4,34 \letterpercent{}), kombinace řeckého a thráckého (4,34 \letterpercent{}), kombinace řeckého a římského (2,6 \letterpercent{}), jména nejistého původu (19,97 \letterpercent{}), beze jména (6,95 \letterpercent{}); geografická jména z oblasti Thrákie 7, geografická jména mimo Thrákii 6;

\NC\AR
\HL
\HL
\stoptable

Celkový počet nápisů datovaných do 2. st. zůstává přibližně na stejné úrovni jako ve 3. st. př. n. l. Naprostá většina nápisů pochází z pobřežních oblastí, z bezprostřední blízkosti řeckých měst, jak dokazuje mapa 6.05 v Apendixu 2. Hlavní produkční centrum pro celý region se přesunulo z Mesámbrie a Maróneie do Byzantia, což je pravděpodobně odrazem nárůstu politického postavení Byzantia, co by spojence Říma (Jones 1973, 7).

Materiálem nesoucím nápisy je téměř výhradně kámen, s jednou výjimkou nápisu {\em SEG} 54:633 na střepu keramické nádoby.\footnote{Na pomezí soukromého a veřejného nápisu je dochované {\em ostrakon} {\em SEG} 54:633 na střepu keramické nádoby, která pochází z Apollónie Pontské a nese řecké jméno Aris{[}s{]}teidés. Pokud by se skutečně jednalo o {\em ostrakon} v pravém slova smyslu, byl by to první důkaz o použití ostrakismu ve 2. st. př. n. l. na území Thrákie. Vzhledem k ojedinělosti nálezu je však možné, že se jedná o označení vlastníka nádoby, či zkušební materiál, na němž se mohl dotyčný cvičit v psaní vlastního jména. Tato poslední možnost je však relativně málo pravděpodobná, vzhledem k tomu, že v té době jistě existovala celá řada materiálů, na něž bylo jednodušší psát než na poměrně tvrdou keramiku. Do doby než, však bude nalezeno více podobných nápisů, které by dosvědčovaly existenci ostrakismu v Thrákii, je vhodné tento konkrétní nápis považovat spíše za označení vlastnictví, než za projev poměrně sofistikované metody vyjádření politického názoru a uplatnění politické moci v Apollónii Pontské.} Nosiče nápisů jsou převážně zhotovovány z místního zdroje kamene a dá se tedy i nadále předpokládat, že materiál pro zhotovování nápisů byl získáván v nejbližším okolí produkčních center a nebyl předmětem dálkového obchodu.

\subsection[funerální-nápisy-7]{Funerální nápisy}

Ze 2. st. př. n. l. pochází 80 funerálních nápisů, jejichž primární funkcí bylo sloužit jako náhrobní kámen a označovat hrob. Oproti předcházejícím obdobím se nedochovaly nápisy na předmětech osobní potřeby, které se staly součástí pohřební výbavy. Tento fakt může poukazovat na stav prozkoumání kulturních vrstev z této doby, či na upadající vliv thráckých aristokratů, a tudíž i na pokles epigrafických aktivit spojených a jejich aktivitami.

Celkově dochází k poklesu počtů funerálních nápisů napříč celou Thrákií, s výjimkou řeckého Byzantia, kde dochází k poměrně markantnímu nárůstu. Nápisy z pobřežních oblastí nesou ze 73 \letterpercent{} jména řeckého původu, nicméně zejména v oblasti Byzantia se setkáváme i s jmény thráckými, zastoupenými zhruba v 5 \letterpercent{}, a římskými, zastoupenými zhruba 7 \letterpercent{}, zbylých 15 \letterpercent{} jmen je cizího či nejistého původu.\footnote{Nápisů s pouze řeckými jmény je celkem 53, nápisy s řeckým a římským jménem jsou dva, a nápisy s řeckým a thráckým jménem jsou čtyři, všechny z Byzantia. Nápis {\em IK Byzantion} 214 jako jediný prokazatelně patří muži nesoucí thrácké jméno, jehož otec nese jméno řeckého původu: Mokaporis, syn Moscha. Thrácká jména se vyskytují celkem na šesti nápisech, z nichž pouze na jednom se setkáváme s kombinací čistě thráckých jmen. Nápis {\em IK Byzantion} 340 patřil Mokazoiré, dceři Dinea.} Necelých 30 \letterpercent{} nápisů je určeno ženám, které jsou dále identifikovány jako dcery či partnerky. Kombinace osobních jmen je možné interpretovat jako důsledek smíšených sňatků, či přejímání onomastických zvyků dané kultury. Tento jev je omezený na oblast Byzantia a nelze tedy hovořit o celospolečenském fenoménu.

Geografický původ je uváděn vždy po jednom případě zemřelých pocházejících z Galatie a Bíthýnie, Apameie, tedy z Malé Asie z oblastí sousedících s Thrákií, což poukazuje na přesuny obyvatel mezi Evropou a Asií, ač na omezené úrovni.\footnote{Nápis {\em IK Byzantion} 120 patřil Theodórovi, jehož otec pocházel z města Bíthýnion v Bíthýnii, ale sám Theodóros se cítil být občanem Byzantia, kde také byl pochován.} Nápisy jsou většinou krátké v rozsahu jednoho až tří řádků. Výjimku tvoří několik veršovaných nápisů z Maróneie a Byzantia o rozsahu až 13 řádek, které podrobně líčí životní osudy a vykonané skutky zemřelého tak, jak bývá obvyklé zejména v pozdějších dobách,\footnote{Ač jsou všechna jména na těchto delších nápisech až na jednu výjimku řeckého původu, jedná se spíše o zvyk, který je pozorovatelný v římské době.} nicméně vyskytující se epigrafické formule i nadále odpovídají funerálním nápisům tak, jak jsme je mohli vidět v řeckých komunitách v 5. - 3. st. př. n. l.\footnote{Celkem 11krát nápisy zdraví čtenáře. Místo pohřbu je v jednom případě označeno termínem {\em tafos}, jednou je použit termín {\em mnéma}, které označuje jak hrob, tak zároveň i náhrobní kámen.}

\subsection[dedikační-nápisy-7]{Dedikační nápisy}

Celkem osm dedikačních nápisů pochází výhradně z okolí řeckých kolonií na pobřeží. Věnování provedli muži nesoucí jména řeckého původu a nápisy byly dedikovány božstvům řeckého a egyptského původu, jako např. Dioskúrům, Afrodíté s přízviskem {\em Ainijská} či Sarápidovi a Ísidě. Poprvé se objevuje věnování Diovi a Rómé, personifikovanému božstvu představujícímu autoritu města Říma.\footnote{{\em I Aeg Thrace} 187 pochází z egejské Maróneie.}

\subsection[veřejné-nápisy-7]{Veřejné nápisy}

Celkem 25 dochovaných veřejných nápisů, což oproti minulému století představuje mírný propad. Nápisy pocházejí výhradně z řeckých měst na pobřeží, případně byly nalezeny mimo oblast Thrákie, ale jejich obsah je zcela jasně s Thrákií spojuje. Nejvíce textů pochází z Maróneie s devíti exempláři. Celkem 80 \letterpercent{} veřejných nápisů představují dekrety vydané institucemi řecké {\em polis}, jako je {\em búlé} a {\em démos}.

Honorifikační nápisy jsou určeny význačným mužům řeckého, tak římského původu. Řím se objevuje v nápisech jako silný hráč na poli mezinárodní politiky, v mnoha případech vystupuje i jako spojenec řeckých států na egejském pobřeží, např. na nápise {\em I Aeg Thrace} 168. Pokud jde o zmínky o Thrácích, velmi záleží na konkrétním případě a záměru daného nápisu. Konkrétně na nápise {\em IK Sestos} 1 jsou Thrákové zmiňováni jako sousedé, kteří mohou ohrožovat bezpečí obyvatel Séstu. V dalším případě je thrácký panovník Kotys vnímán jako suverénní politická autorita {\em I Aeg Thrace} 5, stojící na podobné úrovni jako Řím a abdérský lid. V kontextu veřejných nápisů tedy nefigurují Thrákové jako barbaři či nepřátelé, ale spíše jako mocní spojenci a nevyzpytatelní sousedé autonomních řeckých měst.

Dochované administrativní termíny poukazují na nárůst vyskytujících se termínů na 35, které se dohromady objevují 117krát.\footnote{Nejpoužívanějšími termíny byly pojmy bezprostředně reprezentující politickou autoritu a vztahující k proceduře vydávání nařízení jako {\em démos}, {\em búlé}, {\em polis} a {\em pséfisma}. Objevují se nové funkce a instituce, které se nápisech dříve nevyskytovaly, jako například {\em synedrion}, {\em symmachiá}, {\em gymnasion}, {\em grammateus}, {\em chorériá}, {\em efébos} a {\em basilissa.} Existence {\em gymnasia} na území Thrákie je poprvé potvrzena epigraficky, a to v Apollónii a v Séstu, instituce {\em efébie} je potvrzena také v Séstu. Dále se zde vyskytují termíny s finanční a obchodní tematikou jako je {\em emporion}, {\em chrémata}, {\em analóma}, {\em trápeza}, {\em chóra}, {\em oikos} a {\em katoikia}. Zcela poprvé se objevuje termín {\em autokratór}, který je v následujících stoletích používán výhradně pro římského císaře, ale v tomto případě na nápise {\em IG Bulg} 1,2 388bis označuje velitele námořnictva z Istru, který pomohl Apollónii v době války mezi Apollónií a Mesámbrií.} Nárůst celkového počtu termínů a objevení nových institucí a funkcí související s organizací politické autority a jejím výkonem naznačuje, že docházelo k nárůstu společenské komplexity a organizace v rámci řeckých komunit na pobřeží. Nárůst dochovaných ujednání mezi politickými autoritami regionu naznačuje i nárůst diplomatických kontaktů, nebo alespoň jejich kodifikace a častější zaznamenávání na trvalé médium nápisu.

\subsection[shrnutí-11]{Shrnutí}

V průběhu 2. st. př. n. l. dochází k prvním projevům mísení řeckého a římského kulturního prostředí, avšak se znatelnou převahou řeckého elementu. Thrákové se do produkce soukromých nápisů zapojují minimálně, a to pouze v regionu Byzantia a na úrovni běžného obyvatelstva, nikoliv thrácké aristokracie, jak bylo zvykem v předcházejících stoletích. Obsah a forma veřejných nápisů naznačují objevení nových institucí, intenzifikaci diplomatických kontaktů mezi jednotlivými politickými autoritami a jejich kodifikaci v epigrafické produkci. Thráčtí králové jsou na veřejných nápisech pocházejících z řeckého kontextu vnímáni jako rovnocenní spojenci, nikoliv jako primitivní barbaři.

\section[charakteristika-epigrafické-produkce-ve-2.-až-1.-st.-př.-n.-l.]{Charakteristika epigrafické produkce ve 2. až 1. st. př. n. l.}

Nápisy datované do 2. a 1. st. př. n. l. pocházejí i nadále převážně z řeckého prostředí, nicméně se začíná projevovat vliv římské přítomnosti v regionu. To s sebou nese zvýšenou přítomnost římských jmen a stále pokračující nárůst počtu veřejných nápisů, v nichž se Řím objevuje jako mocný spojenec. Nápisy pocházejí převážně z pobřežních oblastí, avšak je možné pozorovat nárůst epigrafické aktivity v oblasti středního toku Strýmónu. Zapojení thrácké aristokracie na produkci epigrafických pramenů je pozorovatelné v malé míře.

\placetable[none]{}
\starttable[|l|]
\HL
\NC {\em Celkem:} 125 nápisů

{\em Region měst na pobřeží:} Abdéra 3, Apollónia Pontská 1, Byzantion 79, Maróneia 10, Mesámbria 7, Odéssos 14, Perinthos (Hérakleia) 1, Sélymbria 3 (celkem 118 nápisů)

{\em Region měst ve vnitrozemí:} Beroé (Augusta Traiana) 2, (Marcianopolis) 1, údolí střední toku řeky Strýmónu 3, Hérakleia Sintská 1

{\em Celkový počet individuálních lokalit}: 21

{\em Archeologický kontext nálezu:} funerální 3, sídelní 1, sekundární 9, neznámý 112

{\em Materiál:} kámen 125 (mramor 118, z toho mramor z Prokonnésu 1, místní mramor 1, jiné 1; z čehož je varovik 1)

{\em Dochování nosiče}: 100 \letterpercent{} 8, 75 \letterpercent{} 14, 50 \letterpercent{} 10, 25 \letterpercent{} 8, kresba 1, nemožno určit 84

{\em Objekt:} stéla 119, architektonický prvek 4, jiné 1

{\em Dekorace:} reliéf 103, malovaná dekorace 1, bez dekorace 22; reliéfní dekorace figurální 81 nápisů (vyskytující se motiv: jezdec 2, stojící osoba 5, sedící osoba 6, skupina lidí 2, zvíře 5, funerální scéna/symposion 15, scéna oběti 4, jiný 2), architektonické prvky 24 nápisů (vyskytující se motiv: naiskos 6, sloup 1, báze sloupu či oltář 1, věnec 1, florální motiv 11, architektonický tvar/forma 2, jiný 2)

{\em Typologie nápisu:} soukromé 107, veřejné 14, neurčitelné 4

{\em Soukromé nápisy:} funerální 98, dedikační 11, jiné 1\footnote{Součet nápisů jednotlivých typů je vyšší než počet veřejných nápisů vzhledem k možným kombinacím jednotlivých typů v rámci jednoho nápisu.}

{\em Veřejné nápisy:} náboženské 1, seznamy 1, honorifikační dekrety 1, státní dekrety 6, neznámý 2

{\em Délka:} aritm. průměr 3,92 řádku, medián 3, max. délka 60, min. délka 1

{\em Obsah:} dórský dialekt 11; hledané termíny (administrativní termíny 16 - celkem 44 výskytů, epigrafické formule 13 - 46 výskytů, honorifikační 14 - 19 výskytů, náboženské 24 - 35 výskytů, epiteton 2 - počet výskytů 2)

{\em Identita:} řecká božstva, pojmenování míst a funkcí typických pro řecké náboženské prostředí, místní thrácká božstva, regionální epiteton 2, kolektivní identita 3 termíny, celkem 3 výskyty - obyvatelé řeckých obcí z oblasti Thrákie 1, ale i mimo ni 1, kolektivní pojmenování barbaroi 1; celkem 217 osob na nápisech, 86 nápisů s jednou osobou; max. 63 osob na nápis, aritm. průměr 1,73 osoby na nápis, medián 1; komunita řeckého charakteru se zastoupením římského a thráckého prvku, jména pouze řecká (56 \letterpercent{}), pouze thrácká (2,4 \letterpercent{}), pouze římská (1,6 \letterpercent{}), kombinace řeckého a thráckého (3,2 \letterpercent{}), kombinace řeckého a římského (8 \letterpercent{}), kombinace thráckého a římského (0,8 \letterpercent{}), jména nejistého původu (15,2 \letterpercent{}), beze jména (11,2 \letterpercent{}){\bf ;} geografická jména z oblasti Thrákie 0, geografická jména mimo Thrákii 7;

\NC\AR
\HL
\HL
\stoptable

Oproti nápisům datovaným do 3. až 2. st. př. n. l. je u skupiny nápisů datovaných do 2. až 1. st. př. n. l. pozorovatelný téměř 200 \letterpercent{} nárůst celkového počtu nápisů. Většina produkčních center se nachází na pobřeží, nicméně individuální nápisy byly nalezeny i v thráckém vnitrozemí, zejména v okolí řeky Strýmón, jak je možné vidět na mapě 6.05 v Apendixu 2. Hlavním produkčním centrem je Byzantion, avšak pozici menších produkčních center si i nadále udržuje Odéssos a Maróneia.

Materiálem, z nějž jsou nápisy zhotovovány, je výhradně kámen a většina nápisů má tvar stély. Převládající funkce nápisů je funerální a v malé míře i dedikační. Objevují se i nápisy veřejné, ač v menším počtu než v předcházejících obdobích.

\subsection[funerální-nápisy-8]{Funerální nápisy}

Celkem 98 funerálních nápisů pochází z 2. až 1. st. př. n. l. Nejvíce nápisů pochází z Byzantia, celkem 78, což představuje téměř 80 \letterpercent{} všech funerálních nápisů z daného období. Z vnitrozemí pocházejí pouze tři nápisy, a to z údolí středního toku Strýmónu z regionu Hérakleie Sintské, podobně jako v předcházejícím období. Celkově dochází k poklesu počtů funerálních nápisů napříč celou Thrákií, s výjimkou řeckého Byzantia, kde dochází k poměrně markantnímu nárůstu. Podobně jako ve 2. st. př. n. l. zcela chybí sekundární funerální nápisy, které by bylo možné spojovat s thráckou aristokracií, což může svědčit o oslabení politické a ekonomické moci thráckých aristokratů či o proměnách přístupu Thráků k užití písma a k funerálním ritu obecně.

Skupina tří nápisů z thráckého vnitrozemí z regionu Hérakleie Sintské zaznamenává jména řeckého původu. Jedná se o funerální stély tří žen, jejichž otcové nesou taktéž jména řeckého (či makedonského) původu. Jak je již patrné v předcházejícím období, tato komunita si uchovává tradiční hodnoty, alespoň co se týče epigrafického projevu.

\subsection[dedikační-nápisy-8]{Dedikační nápisy}

Dedikačních nápisů se dochovalo celkem 11 a většina z nich pochází z pobřežních oblastí, až na jeden nápis z lokality Madara, která leží ve vnitrozemí zhruba ve vzdálenosti 70 km východně od Odéssu. Jména dedikantů jsou výhradně řeckého původu, až na jednu výjimku z regionu Topeiru, odkud pochází nápis {\em I Aeg Thrace} 105 se jménem pravděpodobně thráckého původu.

Vyskytující se epigrafické formule dosvědčují udržení řeckých tradic typických pro dedikační nápisy, avšak v menší míře než v předcházejícím období. Typické věnovací formule jako {\em charistérion} se objevuje pouze jednou, v případě formule {\em euchén} pouze dvakrát. V jednom případě se dochovalo {\em enkómion} {\em I Aeg Thrace 205} určené egyptské bohyně Ísidě s délkou přes 44 řádků a původem z Maróneie. Jedná se na svou dobu o neobvyklý nápis jak formou, tak obsahem a poukazuje na trvající vliv řeckého náboženství v jižních oblastech Thrákie (Loukopoulou {\em et al.} 2005, 385).

Poprvé se v nápisech objevují i místní thrácké kulty, a i nadále se rozšiřující vliv náboženství egyptského původu. Výskyt nápisů věnovaných místním božstvům je omezen na jeden region, či dokonce jednu svatyni v Thrákii, jako např. {\em hérós} {\em Karabasmos} či {\em Perkón} z Odéssu. Dále se zde objevují božstva řecká, jako např. Zeus {\em Hypsistos} či samothrácká božstva. Objevuje se opět i bohyně {\em Fosforos}, nejčastěji ztotožňovaná s Artemidou, Hekaté či Bendidou (Janouchová 2013, 103-104). Podobně jako v předcházejícím století se vyskytují i dedikace původně egyptským božstvům Sarápidovi, Ísidě, Anúbidovi a Harpokratiónovi, a dále zbožštělým egyptským vládcům Ptolemaiovi a Kleopatře. Obecně dochází k většímu prolínání náboženských systémů a představ a jejich zaznamenávání na permanentní médium nápisu, což může nasvědčovat na větší otevřenost společnosti, zvýšenou míru kulturního kontaktu a změny ve společnosti, které vyústily v proměnu náboženského systému, tedy obecně jedné z nejkonzervativnějších částí kultury.

\subsection[veřejné-nápisy-8]{Veřejné nápisy}

Celkem se dochovalo 14 veřejných nápisů, což značí propad oproti předcházejícím obdobím. Krom jednoho nápisu z thráckého Kabylé všechny nápisy pocházejí z okolí řeckých měst na pobřeží. Klesající počet nápisů může naznačovat na pokles moci publikujících politických autorit, či poukazuje na nedostatečnou míru prozkoumání kulturních vrstev 2. a 1. st. př. n. l. Nejčastější termíny jsou stále {\em démos}, {\em búlé}, {\em politai} a {\em pséfisma}, a i nadále fungují dříve ustanovené procedury spojené se zhotovováním a vystavováním nápisů, nicméně klesající počet termínů označujících instituce může naznačovat jejich postupný úpadek či pozbytí významu v rámci fungování obce. Forma honorifikačních dekretů z Odéssu si udržela stejnou formu jako u nápisů datovaných do 2. st. př. n. l. a pravděpodobně vycházela ze stejných předpisů a pravidel. Honorifikační nápisy z Mesámbrie a Apollónie mají zcela jinou formu a používají jiné formule, což nasvědčuje o stále trvající regionální autonomii řeckých měst.

Nápis {\em I Aeg Thrace} 212 představuje seznam věřících kultu Sarápida a Ísidy z Maróneie a objevuje se na něm až 75 jmen převážně řeckého původu, avšak i se sedmi jmény římskými a dvěma thráckými. Z přítomnosti osobních jmen je patrné, že kult byl oblíben převážně u mužů nesoucí řecká jména, nicméně byl přístupný i Thrákům a Římanům. Funkce kněžích zastávali muži nesoucí řecká jména a z velké části nedocházelo k mísení onomastických tradic, tj. římská jména tvořila osobní jméno výhradně s dalším jedním či dvěma římskými jmény a nedocházelo k jejich prolínání s řeckými či thráckými jmény. Relativní izolovanost římských jmen svědčí o tom, že zvyk přijímat římská jména se v tomto období ještě neprosadil v podobě, jaká bude běžná v následujících stoletích.

\subsection[shrnutí-12]{Shrnutí}

Nápisy 2. až 1. st. př. n. l. dokumentují omezení publikačních aktivit thrácké aristokracie pro potřeby udržení společenského postavení v rámci komunity. Do regionu poprvé vstupuje se vší silou Řím, což se projevuje jak na obsahu a formě veřejných nápisů, narůstající variabilitě náboženských systémů, ale i na proměňující se skladbě osobních jmen. Římané jsou členy kultů a jsou na území Thrákie pohřbíváni, zejména pak v okolí Byzantia. Nedochází však k mísení onomastických tradic, ale Římané a Řekové si i nadále udržují kulturní odstup.

\section[charakteristika-epigrafické-produkce-v-1.-st.-př.-n.-l.]{Charakteristika epigrafické produkce v 1. st. př. n. l.}

Celková epigrafická produkce v 1. st. př. n. l. poměrně výrazně klesá. Epigraficky aktivní komunity jsou i nadále z poloviny řecké, avšak i nadále dochází k jejich postupnému otevírání a mísení onomastických a náboženských tradic. Veřejné nápisy představují až třetinu celkové produkce a poprvé se objevují nápisy obsahující latinský text. V malé míře narůstají i počty dochovaných římských jmen.

\placetable[none]{}
\starttable[|l|]
\HL
\NC {\em Celkem:} 69 nápisů

{\em Region měst na pobřeží:} Abdéra 2, Apollónia Pontská 1, Byzantion 36, Dionýsopolis 1, Maróneia 10, Mesámbria 11, Odéssos 4, Perinthos (Hérakleia) 1, Sélymbria 1, Séstos 1 (celkem 68 nápisů)

{\em Region měst ve vnitrozemí:} Didymoteichon (Plótinúpolis) 1

{\em Celkový počet individuálních lokalit}: 13

{\em Archeologický kontext nálezu:} funerální 2, sídelní 1, náboženský 1, sekundární 6, neznámý 59

{\em Materiál:} kámen 67 (mramor 67, z toho mramor z Thasu 1), neznámý 2

{\em Dochování nosiče}: 100 \letterpercent{} 4, 75 \letterpercent{} 5, 50 \letterpercent{} 5, 25 \letterpercent{} 12, kresba 2, nemožno určit 41

{\em Objekt:} stéla 64, architektonický prvek 3, neznámý 2

{\em Dekorace:} reliéf 48, bez dekorace 21; reliéfní dekorace figurální 34 nápisů (vyskytující se motiv: jezdec 1, stojící osoba 2, sedící osoba 1, funerální scéna/symposion 3), architektonické prvky 16 nápisů (vyskytující se motiv: naiskos 5, báze sloupu či oltář 3, věnec 1, florální motiv 3, architektonický tvar/forma 5)

{\em Typologie nápisu:} soukromé 50, veřejné 17, neurčitelné 2

{\em Soukromé nápisy:} funerální 42, dedikační 11, jiný 1\footnote{Jeden nápis měl vzhledem ke své nejednoznačnosti kombinovanou funkci funerálního a zároveň dedikačního nápisu, proto je součet nápisů obou typů vyšší než celkový počet soukromých nápisů.}

{\em Veřejné nápisy:} náboženské 2, seznamy 3, honorifikační dekrety 2, státní dekrety 8, jiné 1, neznámý 1

{\em Délka:} aritm. průměr 5,4 řádku, medián 2, max. délka 49, min. délka 1

{\em Obsah:} dórský dialekt 9, latinský text 1 nápis; hledané termíny (administrativní termíny 18 - celkem 44 výskytů, epigrafické formule 9 - 28 výskytů, honorifikační 13 - 17 výskytů, náboženské 23 - 35 výskytů, epiteton 4 - počet výskytů 5)

{\em Identita:} řecká božstva 10, egyptská božstva 3, pojmenování míst a funkcí typických pro řecké náboženské prostředí, regionální epiteton 4, kolektivní identita 4 termíny, celkem 6 výskytů - obyvatelé řeckých obcí z oblasti Thrákie 1, ale i mimo ni 1, kolektivní pojmenování Thráx 3, Rómaios 1; celkem 141 osob na nápisech, 41 nápisů s jednou osobou; max. 20 osob na nápis, aritm. průměr 2,04 osoby na nápis, medián 1; komunita řeckého charakteru s částečným zastoupením římského a thráckého prvku, jména pouze řecká (46,47 \letterpercent{}), pouze thrácká (1,44 \letterpercent{}), pouze římská (4,34 \letterpercent{}), kombinace řeckého a thráckého (4,34 \letterpercent{}), kombinace řeckého a římského (5,79 \letterpercent{}), kombinace thráckého a římského (1,44 \letterpercent{}), kombinovaná řecká, thrácká a římská jména (5,79 \letterpercent{}), jména nejistého původu (21,71 \letterpercent{}), beze jména (8,69 \letterpercent{}); geografická jména z oblasti Thrákie 3, geografická jména mimo Thrákii 2;

\NC\AR
\HL
\HL
\stoptable

Do 1. st. př. n. l. bylo datováno 69 nápisů, což znamená pokles epigrafické produkce o 40 \letterpercent{} oproti předcházejícímu období. Jak je patrné na mapě 6.06 v Apendixu 2, nápisy pocházejí téměř výhradně z pobřežních oblastí s výjimkou lokality Didymoteichon, která se nachází v regionu pozdějšího města Plótinúpolis na řece Tonzos. Hlavní produkční centrum je i nadále v Byzantiu, odkud pochází 43 \letterpercent{} nápisů. Další středně velká produkční centra se nachází v Maróneii, v Mesámbrii, a částečně i v Odéssu, podobně jako v předcházejícím období.

Archeologický kontext míst nálezu je bohužel opět z velké části neznámý, případně sekundární a materiál použitý na výroby nosičů nápisů je opět výhradně kámen, zejména mramor. Nejrozšířenějším nosičem je kamenná stéla a opět zcela chybí nápisy na keramice či na kovových předmětech, které se v Thrákii vyskytovaly mezi 5. a 3. st. př. n. l.

\subsection[funerální-nápisy-9]{Funerální nápisy}

Celkem se dochovalo 42 funerálních nápisů, z nichž všechny sloužily jako primární funerální nápisy, tedy k označení hrobu a předání informace o zemřelém jeho nejbližšímu okolí. Všechny nápisy pocházely z pobřežních oblastí, s největší koncentrací 33 funerálních nápisů pocházejících z Byzantia. Pravděpodobnosti v souvislosti s politickým a ekonomickým rozvojem se Byzantion stal již koncem 2. st. př. n. l. epigrafickým centrem Thrákie a tento trend pokračuje i v průběhu 1. st. př. n. l. a na přelomu letopočtu.

Text funerálních nápisů si udržuje i nadále tradiční podobu\footnote{Invokační formule {\em chaire} se objevuje 12krát, v jednom případě se text nápisu obrací na podsvětní božstva, Háda a Acherón. Rozsah většinou nepřesahuje tři řádky, pouze v jednom případě dosahuje maximální délky 10 řádků. Tento obsáhlejší nápis {\em IG Bulg} 1,2 344 z Mesámbrie mluví o dosažených úspěších jistého Aristóna, který udatně bojoval s nepřátelským kmenem Bessů. Bessové představovali jeden z thráckých kmenů, který se dostal do popředí zájmu pramenů zejména v 1. st. př. n. l. v souvislosti s politickým vývojem v regionu a nástupem aristokratů právě z kmene Bessů k moci.} a identita zemřelých a jejich nejbližších ukazuje na převážně řecký původ nápisů. Jedná se o společnost, v níž se potkává několik kulturních tradic, což se projevuje zejména na proměňujících se onomastických zvyklostech. Zhruba dvě třetiny jmen jsou původem řecká, nicméně 13 \letterpercent{} jmen je možné zařadit jako jména římská a 8 \letterpercent{} jako jména thrácká, zbývajících 16 \letterpercent{} představuje jména cizího či neznámého původu. V omezené míře dochází k mísení řeckých jmen se jmény římskými, což je pravděpodobně důsledek smíšených sňatků a zvýšené přítomnosti římských občanů na území Byzantia.\footnote{Příkladem může být nápis {\em IK Byzantion} 269, který představuje náhrobní kámen členů jedné rodiny, kde dochází ke smíšenému sňatku mezi mužem s řeckými jmény a ženou se jmény římského původu: Hadys, syn Poseidónia a jeho manželka Veigellia Katyla a Dionysis, syn Hadyův.} Poprvé se také začínají ve velmi malé míře objevovat řecká či thrácká jména v kombinaci s římskou nomenklaturou.\footnote{Např. Markos Antonios Dadas z nápisu {\em IK Byzantion} 198.}

\subsection[dedikační-nápisy-9]{Dedikační nápisy}

Celkem se dochovalo 11 dedikačních nápisů, které pocházejí převážně z Mesámbrie, Maróneie a Byzantia, s výjimkou jednoho nápisu z lokality Didymoteichon na dolním toku řeky Tonzos. Čtyři nápisy byly věnované egyptským božstvům Sarápidovi, Ísidě, Anúbidovi a Harpokratiónovi. Egyptské kulty se objevují v Byzantiu, Maróneii a Mesámbrii, tedy ve všech hlavních produkčních centrech té doby, což poukazuje důležitost nejen pro jako místa setkávání kultur, ale i regionální centra té doby. Dále se objevila věnování vždy po jednom nápisu Diovi {\em Aithriovi}, Athéně a Neikonemesis {\em Sóteiře} a Hérakleovi {\em Sótérovi}.

Osobní jména dedikantů poukazují na jejich řecký původ, nicméně římský prvek začíná hrát důležitou roli i na dedikačních nápisech.\footnote{Celkem 26 jmen mělo řecký původ, 11 římský, dva thrácký a sedm nebylo možné s jistotou určit. Nejvíce jmen se vyskytlo na nápise {\em IK Byzantion} 19 z Byzantia, kde je možné napočítat až 26 jmen různého původu: jedná se o věnování Diovi {\em Aithriovi} obyvateli neznámé vesnice, kde funkce kněžích zastávají muži se třemi římskými jmény ({\em tria nomina}). Zhruba polovina obyvatel na této dedikaci má taktéž římské jméno, které je v několika případech kombinované se jménem řeckým (Lajtar 2000,50-51).} Na míru zapojení thráckých elit do praxe věnování nápisů božstvům, tedy zvyklostí do této doby omezené převážně na řeckou komunitu. Nápis {\em I Aeg Thrace} 458 věnovaný původně řeckému božstvu Hérakleovi Sótérovi z lokality Didymoteichon věnoval thrácký král Kotys, syn Rháskúporida mezi lety 42 a 31 př. n. l. \footnote{Tento nápis je na pomezí soukromého a veřejného nápisu, vzhledem k tomu, že byl věnován jménem krále Kotya, tedy svrchované politické autority.}

\subsection[veřejné-nápisy-9]{Veřejné nápisy}

Veřejných nápisů se dochovalo celkem 17, což představuje mírný pokles oproti 2. st. př. n. l. Nápisy pocházejí výhradně z pobřežních komunit, nejvíce jich bylo nalezeno na černomořském a egejském pobřeží: čtyři v Odéssu a Maróneii, tři v Mesámbrii a dále vždy po jednom nápise. Místa s největším počtem veřejných nápisů jsou shodná s největšími producenty soukromých nápisů.

Nápisy mohou být nepřímým důkazem o politických událostech a o zvyšujícím se vlivu Říma v regionu, avšak stále za udržení tradičních zvyklostí a procedur spojených s vydáváním veřejných nápisů. Nápisy obsahují 14 hledaných termínů v 35 výskytech, což je představuje výrazný pokles oproti předcházejícímu období. Nejvíce se vyskytuje termín {\em démos} v 10 případech, dále {\em búlé} a {\em polis} se čtyřmi výskyty a {\em basileus} a {\em pséfisma} se třemi výskyty. Dochází i k poklesu celkového počtu termínů označujícího společenské funkce a instituce. Většinu veřejných nápisů představují dekrety vydávané politickou autoritou, což v tomto případě byly instituce řecké {\em polis}, nicméně nápisy zmiňují představitele římské říše a thrácké krále jako rovnocenné partnery a uznávají jejich autoritu.\footnote{Příkladem mohou být nápisy {\em IG Bulg} 1,2 13, {\em IG Bulg} 1,2 43, {\em IG Bulg} 1,2 314a. Jedním takovýmto příkladem je dekret {\em IG Bulg} 1,2 314 z Mesámbrie, díky němuž se dozvídáme, že makedonský místodržící M. Terentius Lucullus v roce 72/71 př. n. l. ustanovil vojenskou posádku na území Thrákie a upevnil tak římský vliv v oblasti (Lozanov 2015, 77). O délce trvání římské přítomnosti a o míře vlivu však nemáme žádné další informace, nicméně je to jeden z prvních náznaků římského zájmu v oblasti, který se projevil i v epigrafice.}

Thrácká aristokracie i nadále hraje důležitou politickou roli v regionu: na nápisech se vyskytují celkem čtyři muži, nesoucí označení král Thráků: Kotys, syn Rháskúporida, Rhoimetalkás, Sadalás a Burebista. Z historických zdrojů víme, že v 1. st. př. n. l. došlo ke sjednocení Thráků pod kmenem Sapaiů, a tedy i k upevnění pozice thráckého panovníka se sídlem v Bizyi v jihovýchodní Thrákii (Lozanov 2015, 78). Rozmístění nápisů odpovídá i přibližné rozmístění sféry vlivu jednotlivých dynastií: Sapaiové na jihovýchodě, Odrysové ve středu a Getové na severu Thrákie.\footnote{Do dynastie Sapaiů spadají Rhaskúporis a Kotys s nápisy pocházejícími z Bizóné, Maróneie a lokality Dydimoteichon. Sadalás pravděpodobně patří do dynastie Odrysů a vyskytuje se na nápise {\em IG Bulg} 1,2 43 z Odéssu a Burebista, zmíněný na nápise {\em IG Bulg} 1,2 13 z Dionýsopole pochází z kmene Getů.}

\subsection[shrnutí-13]{Shrnutí}

Počty dochovaných nápisů z 1. st. př. n. l. poukazují na celkové snížení produkce nápisů a jejich vymizení z vnitrozemí, pravděpodobně související s celkovým úpadkem ekonomické prosperity thráckého vnitrozemí (Lozanov 2015, 84). Hlavním ekonomickým a produkčním centrem je i nadále Byzantion, ale nápisy se v menší míře vydávají i v dalších městech na pobřeží. Řecký prvek hraje i nadále důležitou roli, ale je doplněn jak thráckým, tak římským i dalšími prvky. Vzhledem k narůstající moci Říma se začíná proměňovat i složení epigraficky aktivní populace a její zvyky, ať už se jedná o projevy náboženství či nové onomastické trendy, případně o detailnější obsah funerálních nápisů.

Dochází k proměně na politické scéně, kdy se v 1. st. př. n. l. se opět objevuje thrácká aristokracie, tentokrát v podobě dynastie Sapaiů, Odrysů a Getů. Její přítomnost a projevy v epigrafice se však nepodobají nápisům 5. až 3. st. př. n. l., nicméně mají formu obvyklou spíše u politické autority typu řecké {\em polis}. Je tedy možné říci, že tato nově reformovaná thrácká aristokracie přistoupila na společenské normy a zvyklosti svých nejbližších sousedů a partnerů a jako komunikační strategii zvolila formu veřejných dekretů a usnesení.

\section[charakteristika-epigrafické-produkce-1.-st.-př.-n.-l.-až-1.-st.-n.-l.]{Charakteristika epigrafické produkce 1. st. př. n. l. až 1. st. n. l.}

Nápisy datované do 1. st. př. n. l. až 1. st. n. l. se vyznačují narůstající otevřeností a multikulturalitou epigraficky aktivních komunit. Tomu nasvědčuje i prolínání onomastických tradic, rozvoj místního náboženství, ale i kultů nethráckého původu. Epigrafická aktivita je celkově nižší oproti předcházejícím stoletím, avšak narůstá celkový počet osob vystupujících na nápisech. Thrácká aristokracie se nicméně z epigrafických dokumentů vytrácí a stejně tak i ustávají epigrafické aktivity v thráckém vnitrozemí.

\placetable[none]{}
\starttable[|l|]
\HL
\NC {\em Celkem:} 56 nápisů

{\em Region měst na pobřeží:} Abdéra 3, Anchialos 1, Bizóné 1, Byzantion 30, Dionýsopolis 1, Maróneia 8, Mesámbria 1, Odéssos 3, Perinthos (Hérakleia) 2, Sélymbria 2, Séstos 1, Topeiros 1 (celkem 54 nápisů)

{\em Region měst ve vnitrozemí:} 0\footnote{Celkem dva nápisy nebyly nalezeny v rámci regionu známých měst, editoři korpusů udávají jejich polohu vzhledem k nejbližšímu modernímu sídlišti či muzeu, kde se v současnosti nacházejí.}

{\em Celkový počet individuálních lokalit}: 19

{\em Archeologický kontext nálezu:} sídelní 1, náboženský 1, sekundární 5, neznámý 49

{\em Materiál:} kámen 54 (mramor 52, neznámý 2), neznámý 2

{\em Dochování nosiče}: 100 \letterpercent{} 4, 75 \letterpercent{} 2, 50 \letterpercent{} 5, 25 \letterpercent{} 10, nemožno určit 35

{\em Objekt:} stéla 48, architektonický prvek 4, socha 2, neznámý 2

{\em Dekorace:} reliéf 39, bez dekorace 17; reliéfní dekorace figurální 32 nápisů (vyskytující se motiv: jezdec 3, sedící osoba 1, skupina lidí 1, zvíře 1, funerální scéna/symposion 7), architektonické prvky 8 nápisů (vyskytující se motiv: naiskos 3, sloup 2, báze sloupu či oltář 2, architektonický tvar/forma 1)

{\em Typologie nápisu:} soukromé 45, veřejné 9, neurčitelné 2

{\em Soukromé nápisy:} funerální 35, dedikační 9, neznámý 1

{\em Veřejné nápisy:} seznamy 1, honorifikační dekrety 6, státní dekrety 1, neznámý 1

{\em Délka:} aritm. průměr 4,14 řádku, medián 2, max. délka 48, min. délka 1

{\em Obsah:} dórský dialekt 2, latinský text 1 nápis, písmo římského typu 1; hledané termíny (administrativní termíny 10 - celkem 16 výskytů, epigrafické formule 5 - 11 výskytů, honorifikační 6 - 8 výskytů, náboženské 11 - 18 výskytů, epiteton 4 - počet výskytů 5)

{\em Identita:} řecká božstva 2, egyptská božstva 3, pojmenování míst a funkcí typických pro řecké náboženské prostředí, místní thrácká božstva, regionální epiteton 1, subregionální epiteton 3, kolektivní identita 14 termínů, celkem 14 výskytů - obyvatelé řeckých obcí z oblasti Thrákie 10, ale i mimo ni 2, kolektivní pojmenování Thráx 1, Rómaios 1; celkem 122 osob na nápisech, 31 nápisů s jednou osobou; max. 46 osob na nápis, aritm. průměr 2,18 osoby na nápis, medián 1; komunita multikulturního charakteru se zastoupením řeckého, římského a thráckého prvku, jména pouze řecká (33,9 \letterpercent{}), pouze thrácká (3,57 \letterpercent{}), pouze římská (8,92 \letterpercent{}), kombinace řeckého a thráckého (10,71 \letterpercent{}), kombinace řeckého a římského (7,14 \letterpercent{}), kombinace thráckého a římského (3,57 \letterpercent{}), kombinovaná řecká, thrácká a římská jména (5,35 \letterpercent{}), jména nejistého původu (12,49 \letterpercent{}), beze jména (14,28 \letterpercent{}); geografická jména z oblasti Thrákie 9, geografická jména mimo Thrákii 1;

\NC\AR
\HL
\HL
\stoptable

Oproti nápisům datovaným do 2. až 1. st. př. n. l. je u skupiny nápisů datovaných do 1 st. př. n. l. až 1. st. n. l. pozorovatelný pokles o 45 \letterpercent{} celkového počtu nápisů. Produkční centra se nachází výhradně na pobřeží, jak je možné vidět na mapě 6.06 v Apendixu 2. Hlavním produkčním centrem je i nadále Byzantion, odkud pochází přes polovinu všech nápisů. Pozici menšího produkčního centra si i nadále udržuje Maróneia s osmi nápisy, což představuje 14 \letterpercent{} celkového počtu nápisů. Materiálem, z nějž jsou nápisy zhotovovány, je výhradně kámen, většina nápisů má tvar stély a slouží jako funerální nápis. Zhruba 16 \letterpercent{} nápisů představují nápisy veřejné, což je nepatrně větší zastoupení než v předcházejícím období.

\subsection[funerální-nápisy-10]{Funerální nápisy}

Dochovaných 35 funerálních nápisů pochází výhradně z kontextu řeckých komunit na pobřeží, nicméně onomastické záznamy nasvědčují na proměňující se zvyklosti a pravděpodobně i složení tamější populace. Tři čtvrtiny nápisů pocházejí z Byzantia, kde lze pozorovat narůstající politickou moc Říma, která se projevila i na charakteru nápisů.

O narůstající roli římského kulturního vlivu svědčí i nápis {\em I Aeg Thrace} 72 je psán výhradně latinsky. Navíc se v této době u funerálních nápisů začíná zvyk uvádět roky, jichž se zemřelý dožil, většinou zaokrouhlené na pět let, což je zvyk typický pro římské nápisy (MacMullen 1982, 238). Mezi jmény vyskytujících se na nápisech nicméně i nadále převládají řecká jména, kterých je přibližně čtyřikrát více než jmen římských a pětkrát více než jmen thráckých. Jako měst svého původu označuje Byzantion na nápise {\em IK Byzantion} 352 muž nesoucí čistě římská jména Gaios Ioulios, což svědčí o jeho dvojí loajalitě směrem k římské tradici, ale i politické příslušnosti k Byzantiu, tedy původně řecké kolonii. Z toho lze soudit, že jména v této době přestávají být jednoznačným ukazatelem původu, ale spíše se jedná o uvědomělou volbu identity, vědomým prohlášením přináležitosti k určité komunitě.

\subsection[dedikační-nápisy-10]{Dedikační nápisy}

Dedikační nápisy se pomalu začínají prosazovat i v rámci místních thráckých kultů, avšak řecká božstva mají stále převahu. Dedikačních nápisů se dochovalo celkem devět, z nichž tři jsou věnovány Apollónovi, který v jednom případě nesl lokální přízvisko {\em Eptaikenthos} a v jednom {\em Toronténos}, dále tři nápisy věnované {\em héróovi}, který dvakrát nesl přízvisko {\em Stomiános}, jednou {\em Perkón}. V jednom případě je nápis věnován neznámým božstvům ze Sélymbrie. Nápisy pocházejí převážně od osob nesoucí řecká jména, nicméně v případě nápisu {\em Perinthos-Herakleia} 51 věnovaného Apollónovi {\em Toronténovi} se setkáváme s dedikanty nesoucími thrácká jména. Římská jména se objevují pouze v jednom případě na nápise {\em I Aeg Thrace} 202 z Maróneie.

\subsection[veřejné-nápisy-10]{Veřejné nápisy}

Veřejných nápisů se dochovalo celkem devět, z nichž šest představují honorifikační dekrety udělené institucemi řeckých měst významným jedincům.\footnote{Jako např. thrácký král Kotys na nápise {\em I Aeg Thrace} 207 z Maróneie,} Politickou autoritu představují jednotlivé instituce řeckých {\em poleis}, osoba thráckého krále a jednotliví stratégové, případně sama autorita Říma.\footnote{Příkladem je nápis {\em IG Bulg} 5 5011 z Dionýsopole; {\em IK Sestos} 1 ze Séstu.}

Osobní jména na veřejných nápisech dokazují stále ještě řecký charakter epigraficky aktivní populace: dochovalo se celkem 93 řeckých jmen, 11 thráckých a devět římských.\footnote{Nápis {\em IG Bulg} 1,2 46 představuje seznam kněžích nejmenovaného kultu z Odéssu a dochovalo se na něm celkem 50 jmen, z čehož 46 bylo řeckého původu a dvě byla identifikována jako jména thrácká a dvě římská. Celkem 46 kněžích byli výhradně muži a původ jejich jmen byl takřka výhradně řecký, což může nasvědčovat i jisté konzervativnosti nejmenovaného kultu.} Formule použité v honorifikačních nápisech dokumentují jisté přetrvání zvyklostí a procedur spojených s vystavením nápisů, podobně jako v předcházejících stoletích, avšak pokles výskytu tradičních formulí spojených s udílením poct značí proměnu vnitřního uspořádání politických autorit, která se odráží i v použitém jazyce veřejných nápisů.\footnote{Nápis {\em IG Bulg} 1,2 320 z Mesámbrie popisuje proceduru korunovace, což byla jedna z poct udílených v rámci řecké {\em polis}.}

\subsection[shrnutí-14]{Shrnutí}

Na přelomu 1. st. př. n. l. a 1. st. n. l. se setkáváme s projevy celospolečenských změn, které se odráží i na charakteru epigrafické produkce z Thrákie. Dochází k poklesu celkové produkce, jejímu přesunutí do městských center na pobřeží, v čele s Byzantiem. Řím začíná hrát důležitou roli jak v měnícím se charakteru a uspořádání společnosti, ale i pozvolna se vyvíjejících zvyklostech, spojených např. s proměnou tradičního epigrafického jazyka či přijímáním nových osobních jmen.

\section[charakteristika-epigrafické-produkce-v-1.-st.-n.-l.]{Charakteristika epigrafické produkce v 1. st. n. l.}

Většina nápisů z 1. st. n. l. i nadále pochází z pobřežních oblastí, kde však dochází k proměně komunit směrem k větší otevřenosti a multikulturalitě a narůstající roli institucionální epigrafické produkce. Celková délka nápisů se prodlužuje, stejně tak se zvyšuje i počet osob vystupujících na nápisech. Římský kulturní prvek se začíná prosazovat jak na poli proměňujících se osobních jmen, tak i přítomností latinsky psaných nápisů či jejich částí.

\placetable[none]{}
\starttable[|l|]
\HL
\NC {\em Celkem:} 68 nápisů

{\em Region měst na pobřeží:} Abdéra 2, Abritus 1, Anchialos 1, Apollónia Pontská 1, Byzantion 15, Ferai 1, Koila 1, Madytos 1, Maróneia 13, Mesámbria 1, Odéssos 2, Perinthos (Hérakleia) 13, Topeiros 3 (celkem 55 nápisů)

{\em Region měst ve vnitrozemí:} Filippopolis 4, Neiné 1, Nicopolis ad Nestum 1, Serdica 1, údolí středního toku řeky Strýmón 1 (celkem 8 nápisů)\footnote{Celkem čtyři nápisy nebyly nalezeny v rámci regionu známých měst, editoři korpusů udávají jejich polohu vzhledem k nejbližšímu modernímu sídlišti či muzeu, kde se v současnosti nacházejí. Jeden nápis byl nalezen mimo území Thrákie, avšak editoři uvádějí jeho původ jako thrácký na základě širšího kontextu.}

{\em Celkový počet individuálních lokalit}: 27

{\em Archeologický kontext nálezu:} sídelní 5, náboženský 3, sekundární 8, neznámý 52

{\em Materiál:} kámen 64 (mramor 55, z toho z Thasu 1, z okolí Maróneii 2; vápenec 1; jiný 4, z toho póros 1; neznámý 4), kov kombinovaný s kamenem 1, neznámý 3

{\em Dochování nosiče}: 100 \letterpercent{} 6, 75 \letterpercent{} 7, 50 \letterpercent{} 7, 25 \letterpercent{} 10, oklepek 2, kresba 2, ztracený 1, nemožno určit 33

{\em Objekt:} stéla 46, architektonický prvek 13, socha 1, jiný 2, neznámý 6, (celkem 3 sarkofágy)

{\em Dekorace:} reliéf 25, malovaná 1, bez dekorace 42; reliéfní dekorace figurální 13 nápisů (vyskytující se motiv: jezdec 1, stojící osoba 2, lovecká scéna 1, funerální scéna/symposion 2, obětní scéna 2, jiný 1), architektonické prvky 11 nápisů (vyskytující se motiv: naiskos 3, sloup 2, báze sloupu či oltář 4, architektonický tvar/forma 4, florální motiv 2, jiný 2)

{\em Typologie nápisu:} soukromé 37, veřejné 28, neurčitelné 3

{\em Soukromé nápisy:} funerální 24, dedikační 10, jiný 1, neznámý 1

{\em Veřejné nápisy:} seznamy 3, honorifikační dekrety 10, státní dekrety 2, náboženské 5, nařízení 1, funerální na náklady obce 1, jiný 4, neznámý 2

{\em Délka:} aritm. průměr 7,53 řádku, medián 5, max. délka 94, min. délka 1

{\em Obsah:} latinský text 12 nápisů, písmo římského typu 1; hledané termíny (administrativní termíny 29 nápisů - celkem 52 výskytů, epigrafické formule 13 - 24 výskytů, honorifikační 5 - 7 výskytů, náboženské 25 - 45 výskytů, epiteton 5 - počet výskytů 9)

{\em Identita:} řecká božstva 11, egyptská božstva 2, pojmenování míst a funkcí typických pro řecké náboženské prostředí, objevující se místní thrácká božstva, regionální epiteton 1, subregionální epiteton 4, kolektivní identita 6 termínů, celkem 10 výskytů - obyvatelé řeckých obcí z oblasti Thrákie 3, mimo ni 0, kolektivní pojmenování Thráx 4, Rómaios 2, Kimbros 1; celkem 152 osob na nápisech, 27 nápisů s jednou osobou; max. 34 osob na nápis, aritm. průměr 2,35 osoby na nápis, medián 1; komunita multikulturního charakteru se zastoupením řeckého, thráckého prvku a narůstající pozicí římského prvku, jména pouze řecká (13,23 \letterpercent{}), pouze thrácká (4,4 \letterpercent{}), pouze římská (25 \letterpercent{}), kombinace řeckého a thráckého (8,82 \letterpercent{}), kombinace řeckého a římského (16,17 \letterpercent{}), kombinace thráckého a římského (4,4 \letterpercent{}), kombinovaná řecká, thrácká a římská jména (5,88 \letterpercent{}), jména nejistého původu (7,31 \letterpercent{}), beze jména (14,7 \letterpercent{}); geografická jména z oblasti Thrákie 18, z toho pojmenování stratégií v Thrákii 11, geogr. jména mimo Thrákii 0;

\NC\AR
\HL
\HL
\stoptable

Do 1. st. n. l. bylo celkem datováno 68 nápisů, což je zhruba stejně jako v 1. st. př. n. l. Jak je dobře vidět na mapě 6.07 v Apendixu 2, nápisy pocházejí převážně z pobřežních oblastí, nicméně osm nápisů bylo nalezeno i ve vnitrozemí. Zatímco velké produkční centrum v této době chybí, produkční centra střední velikosti se nacházejí i nadále v Byzantiu, dále v Perinthu a Maróneii.

Téměř výhradně je použitým materiálem kámen, nejčastější formou nosiče jsou i nadále stély, avšak v malé míře se začínají objevovat i sarkofágy. Přes polovinu nápisů představují funerální texty, nicméně počet veřejných nápisů narostl na více než 40 \letterpercent{}, což naznačuje větší míru institucionálního zapojení na epigrafické produkci. Dělo se tak zejména v okolí Perinthu, který se stal v druhé polovině století hlavním městem nově vzniklé provincie {\em Thracia} (Sharankov 2005, 521; Sayar 1998, 74).

\subsection[funerální-nápisy-11]{Funerální nápisy}

Celkem se dochovalo 25 funerálních nápisů, z nichž většina pochází z Byzantia, Maróneie a Perinthu. V této době se poprvé objevily funerální sarkofágy, které nesly nápis, a zároveň sloužily jako úložiště tělesných pozůstatků zemřelého. Sarkofágy pocházely z Istanbulu, Perinthu a Maróneie, tedy měst, v němž měl Řím silné politické postavení.

Převaha osobních jmen řeckého původu poukazuje na přetrvávající charakter produkčních center, avšak se silným římským vlivem na podobu osobních jmen. Celkem 27 osob neslo řecká jména, 18 jména římská, šest jmen thráckých a tři jména neznámého či cizího původu. K mísení řeckých a římských, případně thráckých jmen dochází zcela výjimečně, přijímání systému tří římských jmen není ještě rozšířené. Thrácká jména se vyskytují pouze na třech nápisech a jejich nositelé pocházejí z vyšších společenských vrstev.\footnote{Jako např. stratégos Rhoimetalkás, syn Diasenea se svou ženou Besoulou, dcerou Moukaporida, a dětmi Kaproubebou a Daroutourme na nápise {\em I Aeg Thrace} 387.} Poprvé se objevují osobní jména (Titos) Flavios/Flavia a (Titos) Klaudios.\footnote{{\em I Aeg Thrace} 317, {\em Perinthos-Herakleia} 72, {\em Perinthos-Herakleia} 128.} Přijímání jmen po rodových jménech římských císařů byla typická praxe při udílení římského občanství, kdy se přidávaly k původnímu jménu nositele jako ocenění za služby ve správě provincie či v armádě (Topalilov 2012, 13). V těchto konkrétních případech je jméno tvořeno i dalšími osobními jmény římského původu a nasvědčuje to spíše faktu, že se nejednalo o případy nově uděleného občanství, ale o osoby, které jména získali po svých předcích, a šlo tedy o Římany, nikoliv o Řeky či Thráky, kteří přijali nová jména.\footnote{Vyjádření kolektivní identity se vyskytují pouze jednou na podstavci, na němž byla pravděpodobně umístěna socha thráckého gladiátora. Nápis {\em I Aeg Thrace} 484 označuje etnický původ (Thráx) a označení druhu gladiátora ({\em mormillón}, lat. {\em mirmillo}).}

Jazyk funerálních nápisů si udržuje do jisté míry tradiční charakter, jak naznačují vyskytující se formule, ale dochází i k určitým inovacím ve funerálním ritu a jeho projevech v epigrafice.\footnote{Invokační formule {\em chaire} je použita třikrát, nebožtík je titulován jako {\em hérós} čtyřikrát, okolo jdoucí je oslovován jednou ({\em parodeita}).} Poprvé se setkáváme s formulí, která poskytuje právní ochranu hrobu, či spíše sarkofágu, před jeho dalším použitím.\footnote{{\em Perinthos-Herakleia} 88. Pokud by došlo k tomu, že budou do hrobu „{\em vloženy ostatky někoho jiného, zaplatí viník částku {[}x{]} městu/pokladníkovi}”, do jehož regionu nápis spadá. Výše částky se lišila v dalších stoletích město od města, stejně tak i přesné znění formule.} Kdo by toto nařízení překročil, hrozí mu finanční postih, který je vymahatelný autoritami města. Tento text je typicky na konci funerální nápisu, a vyskytuje se výhradně na nápisech z Perinthu, tedy místě se silnou římskou přítomností. V dalších stoletích dojde k rozšíření této formule i do jiných míst, nicméně její výskyt pravděpodobně souvisí s administrativní změnou a regulací funerálního ritu v oblasti jihovýchodní Thrákie.

\subsection[dedikační-nápisy-11]{Dedikační nápisy}

Římská politická přítomnost v regionu se začíná projevovat i na měnícím se složení epigraficky aktivní části populace a dochází k opětovnému zapojení místních elit. Zvyk věnovat nápisy božstvům thráckého i řeckého původu se v této době rozšířil zejména mezi thráckou aristokracii, zastávající vysoké úřednické funkce v provinciální správě, a částečně i mezi veterány římské armády. Dedikačních nápisů se dochovalo celkem 10, z nichž tři pocházejí z vnitrozemí a sedm z pobřežních oblastí. Vnitrozemské dedikace jsou ve dvou případech věnovány stratégy nesoucí římská a thrácká jména, což poukazuje na jejich thrácký aristokratický původ\footnote{{\em SEG} 54:639, {\em IG Bulg} 4 2338.} a v jednom případě veteránem římské armády, který nese výhradně římská jména.\footnote{Nápis {\em IG Bulg} 3,1 1410 je psaný z větší čísti latinsky, nicméně jméno božstva a věnování je psáno řecky. Jméno veterána a jeho vojenské zařazení je psáno latinsky, stejně tak jako časové určení, které udává dobu vlády císaře Vespasiána a jeho sedmý konzulát, tedy rok 76 n. l.} Věnování těchto vnitrozemských dedikací patří Héře {\em Sonkéténé}, božstvu {\em Médyzeovi} a Artemidě Kyperské. Dedikace z pobřeží pak náleží císaři Vespasiánovi, Apollónovi {\em Karsénovi}, Diovi {\em Patróovi}, Diovi {\em Zbelsúrdovi}, Hygiei, Nymfám a Néreovi. Tato věnování pocházejí od stratégů nesoucího římská, řecká i thrácká jména, propuštěnce nesoucího řecká jméno a od nejvyššího kněze Dionýsova kultu nesoucího řecká jména.

Nápisy zmiňující stratégy pocházejí převážně z jihovýchodní části Thrákie, z okolí Perinthu a Anchialu, kde pravděpodobně docházelo ke kulturnímu transferu ve zvýšené míře, alespoň na úrovni nejvyšších představitelů politické organizace. Jména římsko-thrácká se vyskytují pouze na nápise {\em IG Bulg} 4 2338, kde figuruje Flavios Dizalas, syn Esbenida a jeho partnerka Reptaterkos, dcera Hérakleidova.\footnote{Nápis pochází z regionu města Nicopolis ad Nestum, datován do poslední čtvrtiny 1. st. n. l.} Jak prozrazuje text nápisu, thrácký Dizalás byl stratégem v Thrákii a občanství a s ním i právo nosit jméno Flavios si získal ve službách římské říši. Jedná se tak o jeden z prvních potvrzených případů, kdy Thrák získal římské jméno za své zásluhy, a nikoliv jako výsledek smíšených manželství.\footnote{Vzhledem k tomu, že Dizalás zastával úřad stratéga, jednalo se pravděpodobně o potomka thráckých aristokratů, který byl za svou loajalitu odměněn římským občanstvím, a tedy i právem nosit jméno Flavios.} Zvyklosti dedikovat nápisy se uplatňují v thrácké komunitě pouze v nejvyšších vrstvách aristokratů, případně veteránů, a stále nepronikly mezi běžné obyvatelstvo.

\subsection[veřejné-nápisy-11]{Veřejné nápisy}

Veřejných nápisů se dochovalo celkem 28, z čehož šest pochází z Perinthu a Byzantia a tři z Filippopole. Perinthos se po polovině 1. st. n. l. stal sídlem místodržícího provincie {\em Thracia}, a tak není výskyt celkem 10 nápisů překvapivý (Lozanov 2015, 76).\footnote{Hlavním městem provincie {\em Thracia} se stává Perinthos, kde vznikla i většina institucí a město se stalo jak sídlem místodržícího, tak zde sídlila vojenská posádka, a město se tak stalo přirozeným produkčním centrem veřejných nápisů (Sharankov 2005, 521; Sayar 1998, 74).} V 10 případech se jedná o honorifikační dekrety, v pěti případech o texty náboženského charakteru a další kategorie jsou zastoupeny v řádech kusů nápisů. Politickou autoritu představuje římský císař, případě vysocí provinciální úředníci, kteří ho zastupují. Instituce jednotlivých měst mohou vydávat nařízení pod patronátem římské říše či thráckého krále, avšak převážně se veřejné nápisy vydané městskými institucemi omezují na honorifikační dekrety.\footnote{Např. nápis {\em SEG} 55:752 z Filippopole či {\em I Aeg Thrace} 83 z Abdéry. Politická autorita se proměňuje v závislosti na změně politické situace. Do poloviny 1. st. n. l. jako autority vystupuje jednak lid, instituce řeckých měst a thráčtí panovníci, kteří však byli smluvně zavázání Římu. Tito panovníci z kmene Odrysů, Astů a Sapaiů jsou známí jako vazalští králové Říma, kteří si částečně udržují autonomii, ale ve velké míře podléhají vlivu a politickým rozhodnutím Říma (Lozanov 2015, 78-80). Pravděpodobně na začátku 1. st. n. l. dochází k rozdělení Thrákie na tzv. stratégie, tedy administrativní a vojenské regiony, které byly oficiálně řízeny thráckými stratégy z řad thráckých aristokratů. Gabriella Parissaki navrhuje, že se jednalo o mezistupeň mezi tradičním kmenovým uspořádáním a centralizovanou samosprávou římské provincie, kde thráčtí aristokraté sehráli významnou roli (Parissaki 2009, 320-328).} V této době se taktéž objevuje první milník, informující o době vzniku daného úseku cesty, a první nápisy informující o císařském provinciálním stavebním programu.\footnote{{\em I Aeg Thrace} 453 z lokality Ferai, datovaný do doby vlády císaře Nerona a {\em IG Bulg} 5 5691 z města Serdica, což představuje první důkaz o císařem organizované stavbě komunikací v Thrákii na pobřeží Egejského moře i ve vnitrozemí. Budování cest pravděpodobně probíhalo již v dřívějších stoletích, ale z té doby se nám nedochovaly milníky či jiné epigrafické záznamy (Lozanov 2015, 76; Madzhahov 2009, 63-65).; {\em IG Bulg} 1,2 57 z Odéssu a {\em IK Sestos} 29 ze Séstu slouží jako doložení stavebních aktivit.} Zmínky o thrácké aristokracii jako takové a thráckých králích z epigrafických záznamů zcela mizí po polovině 1. st. n. l., ač systém stratégií se ještě udržuje několik desetiletí (Lozanov 2015, 78-81). Thrácká aristokracie se tak pravděpodobně adaptovala na nové podmínky a zaujala roli v provinciální samosprávě, právě v roli stratégů.

Dochází také k pozvolné proměně použitého jazyka veřejných nápisů a instituce spojené s chodem římské provincie se začínají objevovat ve zvýšené míře.\footnote{Celkem 24 administrativních termínů se vyskytuje ve 37 případech. Většina termínů se již objevila v minulých stoletích, mezi nové termíny však patří {\em hegémón}, {\em kaisar} jako termíny označující římského císaře, dále {\em búleutés} jako člen {\em búlé}, dále {\em agoranomos}, {\em nauarchos}, a nakonec termín pro hlavní město {\em métropolis}. Mezi nejčastěji se opakující termín patří i nadále {\em démos} s 6 výskyty, a {\em gymnasiarchés} se čtyřmi výskyty.} Osobní jména poukazují na proměňující se trendy při výběru jmen a přijímání římských onomastických tradic zejména mezi Thráky sloužícími v římské armádě a vykonávajícími vysoké úřednické funkce. Lidé zastávající funkci stratéga běžně nesou tři jména, kombinující římská jména, jako Titos, Flavios, Tiberios, Gaios, Ioulios nebo Klaudios, spolu s osobními jmény thráckého původu.\footnote{Např. na nápise {\em I Aeg Thrace} 84 z Topeiru.}

\subsection[shrnutí-15]{Shrnutí}

Z dostupných nápisů je patrné, že připojení Thrákie k Římu s sebou neslo celou řadu změn. V reakci na zapojení thráckého obyvatelstva do římské armády se proměnily onomastické zvyky vojáků a veteránů. Epigrafická produkce se opět navrací do vnitrozemí, nicméně s příchodem autority římského císaře se proměňují projevy thrácké aristokracie. Projevy zvyšujícího vlivu římské administrativy jsou patrné zejména v druhé polovině 1. st. n. l., kdy dochází k budování provinciální infrastruktury a spolu s tím i k nárůstu počtu veřejných nápisů. Celkové počty dochovaných nápisů jsou však poměrně nízké, z čehož se dá usuzovat, že nárůst vlivu Říma byl postupný a reformy byly spíše dlouhodobého charakteru.

\section[charakteristika-epigrafické-produkce-v-1.-st.-n.-l.-až-2.-st.-n.-l.]{Charakteristika epigrafické produkce v 1. st. n. l. až 2. st. n. l.}

Nápisy datované do 1. a 2. st. n. l. pocházejí i nadále z pobřežních oblastí, s hlavním produkčním centrem v Perinthu. Soukromé nápisy funerální typu i nadále převažují, prodlužuje se průměrná délka nápisu a stejně tak i celkový poměr přítomnosti římských jmen v epigraficky aktivních komunitách. Osoby nesoucí thrácká jména se dostávají do izolace, nedochází k jejich kombinaci s řeckými či římskými jmény v rámci jednoho nápisu.

\placetable[none]{}
\starttable[|l|]
\HL
\NC {\em Celkem:} 94 nápisů

{\em Region měst na pobřeží:} Abdéra 2, Anchialos 1, Apollónia Pontská 1, Bizóné 1, Byzantion 6, Lýsimacheia 1, Maróneia 12, Mesámbria 3, Odéssos 1, Perinthos (Hérakleia) 53, Sélymbria 1, Topeiros 1, Zóné 1 (celkem 84 nápisů)

{\em Region měst ve vnitrozemí:} Augusta Traiana 2, Hérakleia Sintská 1, Filippopolis 2, údolí středního toku řeky Strýmón 3 (celkem 8 nápisů)\footnote{Celkem dva nápisy nebyly nalezeny v rámci regionu známých měst, editoři korpusů udávají jejich polohu vzhledem k nejbližšímu modernímu sídlišti.}

{\em Celkový počet individuálních lokalit}: 26

{\em Archeologický kontext nálezu:} funerální 1, sídelní 2, náboženský 2, sekundární 6, neznámý 83

{\em Materiál:} kámen 89 (mramor 81, jiný 1, neznámý 7), kov 2, neznámý 3

{\em Dochování nosiče}: 100 \letterpercent{} 16, 75 \letterpercent{} 3, 50 \letterpercent{} 13, 25 \letterpercent{} 13, kresba 5, nemožno určit 43

{\em Objekt:} stéla 51, architektonický prvek 17, socha 1, nádoba 1, jiný 19 (z toho sarkofág 16), neznámý 2

{\em Dekorace:} reliéf 43, bez dekorace 51; reliéfní dekorace figurální 20 nápisů (vyskytující se motiv: jezdec 3, stojící osoba 4, skupina lidí 1, zvíře 3, obětní scéna 1, Héraklés 1, Artemis 1, tři Nymfy 2, funerální scéna/symposion 1, funerální portrét 1, jiný 2), architektonické prvky 31 nápisů (vyskytující se motiv: naiskos 11, sloup 1, báze sloupu či oltář 7, architektonický tvar/forma 7, geometrický motiv 2, florální motiv 12, věnec 1, jiný 2)

{\em Typologie nápisu:} soukromé 73, veřejné 15, neurčitelné 6

{\em Soukromé nápisy:} funerální 57, dedikační 15, jiný 1

{\em Veřejné nápisy:} seznamy 1, honorifikační dekrety 4, státní dekrety 1, funerální na náklady obce 2, náboženský 3, jiný 4 (z toho veřejné dedikace 3)

{\em Délka:} aritm. průměr 6,05 řádku, medián 5, max. délka 18, min. délka 1

{\em Obsah:} dórský dialekt 1, latinský text 6 nápisů, písmo římského typu 14; hledané termíny (administrativní termíny 17 - celkem 44 výskytů, epigrafické formule 18 - 109 výskytů, honorifikační 2 - 2 výskytů, náboženské 20 - 35 výskytů, epiteton 5 - počet výskytů 5)

{\em Identita:} řecká božstva 10, egyptská božstva 1, pojmenování míst a funkcí typických pro řecké náboženské prostředí, místní thrácká božstva, regionální epiteton 3, subregionální epiteton 2, kolektivní identita 7 termínů, celkem 7 výskytů - obyvatelé řeckých obcí z oblasti Thrákie 7, mimo ni 0; celkem 132 osob na nápisech, 40 nápisů s jednou osobou; max. 8 osob na nápis, aritm. průměr 1,4 osoby na nápis, medián 1; komunita multikulturního charakteru se zastoupením řeckého, římského a thráckého prvku, jména pouze řecká (23,4 \letterpercent{}), pouze thrácká (2,12 \letterpercent{}), pouze římská (18,08 \letterpercent{}), kombinace řeckého a thráckého (0 \letterpercent{}), kombinace řeckého a římského (22,34 \letterpercent{}), kombinace thráckého a římského (0 \letterpercent{}), kombinovaná řecká, thrácká a římská jména (5,31 \letterpercent{}), jména nejistého původu (9,55 \letterpercent{}), beze jména (19,14 \letterpercent{}); geografická jména z oblasti Thrákie 2, geografická jména mimo Thrákii 4;

\NC\AR
\HL
\HL
\stoptable

Nápisů pocházejících z období mezi 1. a 2. st. n. l. je přibližně o 67 \letterpercent{} víc než v předcházejícím období. Většina nápisů pochází z oblasti na pobřeží, z okolí původně řeckých měst. Nápisy ve vnitrozemí se objevují podél cest, které jsou pravidelně udržovány a spravovány od druhé poloviny 1. st. n. l., a zejména v okolí rozvíjejících se regionálních urbánních center, jako je Filippopolis či Kabylé, a dále v oblasti středního toku řeky Strýmón, jak ilustruje mapa 6.07 v Apendixu 2. Největším epigrafickým producentem je od druhé poloviny 1. st. n. l. Perinthos (Hérakleia) odkud pochází 56 \letterpercent{} celkové produkce.\footnote{Nárůst epigrafické produkce Perinthu souvisí s faktem, že se v polovině 1. st. n. l. stalo hlavním městem nově vytvořené provincie {\em Thracia} a je logické, že se zde soustředila většina produkce a byla zde nalezena většina nápisů.} Pozici středního producenta si i nadále udržuje Maróneia, zatímco Byzantion, které hrálo roli největšího epigrafického producenta v předchozích dvou stoletích, nyní produkuje zhruba 6 \letterpercent{} celkové produkce.

Materiálem nosiče nápisů je z 94 \letterpercent{} kámen, z převážné části je to mramor, dále vápenec a místní zdroj kamene. Nápisy na kovových předmětech se dochovaly pouze dva.\footnote{Do skupiny nápisů datovaných do 1. až 2. st. n. l. patří vojenský diplom AE 2007, 1260 psaný na bronzové destičce, který uděluje římské občanství, právo nosit římské jméno, pořídit si manželku a právo na vlastní pozemek. Text je psaný latinsky a pochází z okolí vesnice Trapoklovo v jihovýchodním Bulharsku. Bohužel text je velmi poškozený, a tudíž není možné dělat žádné další závěry. Nicméně se jedná o jeden z prvních projevů udělení římského občanství vojákovi za jeho dlouholeté služby, který byl nalezen přímo na území Thrákie. Zároveň se jedná o další důkaz narůstajícího vlivu římské armády nejen na složení obyvatelstva, a jejich onomastické zvyky, ale ovlivňující i přítomnost veteránů ve vnitrozemské Thrákii (Dana 2013, 239-264).} Nejoblíbenější formou kamenného nápisu je i nadále stéla, nicméně v této době se stávají populární i funerální nápisy na sarkofázích, pocházejících v 15 případech z Perinthu (Hérakleii) a v jednom z Byzantia.

\subsection[funerální-nápisy-12]{Funerální nápisy}

Celkem 59 nápisů je interpretováno jako funerální\footnote{57 nápisů je soukromého charakteru a dva jsou veřejného charakteru.}, což představuje zhruba dvě třetiny celého souboru, přibližně stejně jako v 1. st. př. n. l. Většina nápisů je zhotovena na stéle, ale spadá sem i skupina 16 sarkofágů, které nesly nápis, a zároveň sloužily jako úložiště tělesných pozůstatků zemřelého. Hlavní produkční centra funerálních nápisů jsou Perinthos s 42 nápisy, Byzantion a Maróneia s pěti nápisy.

Sarkofágy z Perinthu užívají téměř totožné formule a termíny, a na většině z nich se vyskytuje formule o ochraně sarkofágu a pozůstatku zemřelého politickou autoritou města, jako ve skupině nápisů z 1. st. př. n. l. Osobní jména na sarkofázích nesou asi v polovině případů římská jména, a ve zbývající polovině jména řecká. Thrácká jména se na sarkofázích vůbec nevyskytují, což naznačuje římský původ této zvyklosti.

Převaha nápisů stále pochází z řeckých komunit, které si i nadále udržují relativně konzervativní charakter. Nejčastěji se na nápisech objevují řecká jména, zhruba v polovině příkladů. Dále ve 40 \letterpercent{} jména římská a zbylých 10 \letterpercent{} připadá jménům thráckým a jménům nejistého původu. Thrácká jména se objevují pouze na dvou nápisech, a to jak samotná, tak v kombinaci s řeckým jménem, výhradně v souvislosti s funerálním nápisem zemřelého zastávající funkci stratéga.\footnote{Nápis Manov 2008 199 a nápis {\em I Aeg Thrace} 87.} Geografický původ uvádí pouze jeden nápis, odkazující na bíthýnskou Nikomédii.\footnote{{\em Perinthos-Herakleia} 144.}

Obsah nápisů si udržoval tradiční formu, nicméně se vyskytují termíny popisují nově zavedené součásti funerálního ritu, jako sarkofág, ale objevují se i termíny pro nově vzniklá povolání a funkce.\footnote{Invokační formule oslovující okolo jdoucího čtenáře {\em (chaire}) se objevily celkem 26krát, ({\em parodeita}) 21krát. Funerální nápisy sloužily ve většině případů pro rodinné pohřby: celkem 24krát se objevuje text zhotovený jedním z partnerů pro sebe a pro manžela či manželku, či jiného člena nejbližší rodiny. Pro popis hrobu samotného sloužily termíny {\em mnémeion} jednou, {\em latomeion} v devíti případech a {\em soros} v 11 případech, popisující sarkofág, dále {\em stéla} se sedmi výskyty jako označení hrobu a {\em bómos} se dvěma výskyty a po jednom výskytu {\em tymbos} a séma označují hrob samotný.} Text nápisů nesl více detailů se života zemřelého: setkáváme se s šesti vojáky různých hodností, jako například legionář, jezdec, {\em stratégos} či {\em pragmatikos}. Dále se setkáváme se stavitelem domů ({\em domotektón}), ale i notářem ({\em notários}). V 17 případech se dozvídáme věk, kterého se zemřelý dožil, typicky zaokrouhlený na pět let. Jedná se o typický římský kulturní zvyk, který se na území Thrákie vyskytl již v 1. st. n. l. ve velmi omezeném počtu, nyní však u celé třetiny nápisů.

\subsection[dedikační-nápisy-12]{Dedikační nápisy}

Dedikačních nápisů se dochovalo celkem 19, z čehož 16 je soukromé povahy a tři jsou na pomezí soukromého a veřejného nápisu, což oproti minulému období představuje dvojnásobný nárůst. Polovina nápisů pochází z Perinthu a na dalších místech na pobřeží a ve vnitrozemí se vyskytují nápisy od jednoho do tří exemplářů. Nápisy jsou zhotoveny převážně z kamene, jen v jednom případě se jedná o zlatý amulet z Perinthu.

Věnování jsou určena jak řeckým, tak lokálním božstvům, která částečně využívají řecká jména pro božstva v kombinaci s místním epitetem. Dedikace jsou určeny Apollónovi {\em Iatrovi}, Asklépiovi {\em Zylmyzdriénovi}, Diovi s přízviskem {\em Lofeités}, Sarápidovi, Héře a anonymním božstvům. Osobní jména dedikantů poukazují na smíšené složení věřících. U nápisů věnovaných božstvům s místním (subregionálním) epitetem se vyskytují jak thrácká jména ve vyšším poměru než u ostatních dedikací, což naznačuje, že lokální kulty byly oblíbené především u místního thráckého obyvatelstva. Přítomnost jmen smíšených řecko-thráckých a římsko-řeckých však ukazuje, že lokální kulty byly přístupné i lidem mimo thráckou komunitu.

\subsection[veřejné-nápisy-12]{Veřejné nápisy}

Celkem se dochovalo 15 veřejných nápisů: čtyři z Maróneie a Perinthu a po jednom nápise z ostatních měst na pobřeží. Politickou autoritu představují instituce řeckých {\em poleis}, jako {\em búlé} a {\em démos} v případě Maróneie, a {\em démos} v případě Odéssu a Perinthu, a dále římský císař a vysoce postavení úředníci římské říše. Thráčtí králové se na veřejných nápisech nevyskytují, stejně tak jako stratégové. Nápisy mají nejčastěji povahu honorifikačních dekretů či seznamů věřících a textů náboženské povahy. Tradiční forma honorifikačních dekretů tak, jak jí známe např. z 3. st. př. n. l. se nedochovala, z čehož se dá usuzovat, že došlo k proměně celé procedury udílení poct, a tedy i k proměně podoby honorifikačních textů, podobně jako např. v Malé Asii (Van Nijf 2015, 240).\footnote{Bohužel dochovaný soubor nápisů je natolik fragmentární, že není možné blíže popsat jednotlivé změny.}

\subsection[shrnutí-16]{Shrnutí}

S větší participací thráckého prvku na epigrafické produkci může souviset zapojení Thráků v provinciální administrativě a v římské armádě, což s sebou neslo jednak přijetí nových zvyků a jednak i nárůst gramotnosti, a tedy i změny přístupu k publikaci nápisů. Postupné proměny onomastických záznamů nasvědčují na aktivní zapojení thráckých mužů v římské armádě a následné proměny onomastických zvyků u této části populace. S tím může nepřímo i souviset i mírný nárůst počtu místních kultů, které však zůstávají otevřené všem. Zvyky civilní části obyvatelstva zůstávají podobné, jako v období před vznikem provincie {\em Thracia}, což nasvědčuje o plynulém přechodu bez znatelného kulturního zlomu.

\section[charakteristika-epigrafické-produkce-ve-2.-st.-n.-l.]{Charakteristika epigrafické produkce ve 2. st. n. l.}

Ve 2. st. n. l. dochází k prudkému nárůstu epigrafické aktivity, a to zejména ve vnitrozemí, odkud pochází více nápisů než z pobřeží. Institucionální uspořádání a politický vliv Říma mají zásadní vliv na rozšíření epigrafické produkce. Veřejné nápisy představují téměř polovinu celého souboru, častější je i výskyt hledaných termínů a latinského textu. Římská jména se vyskytují na polovině všech nápisů, ať už samostatně, či v kombinaci. Dochází k prolínání kulturních a náboženských tradic, objevují se ale i nové prvky a jména kultů. Politická příslušnost a status nabírají na důležitosti, a proto narůstá i počet honorifikací.

\placetable[none]{}
\starttable[|l|]
\HL
\NC {\em Celkem:} 254 nápisů

{\em Region měst na pobřeží:} Abdéra 5, Anchialos 2, Apollónia Pontská 1, Byzantion 22, Dionýsopolis 1, Kallipolis 2, Madytos 1, Maróneia 19, Mesámbria 1, Odéssos 17, Perinthos (Hérakleia) 21, Sélymbria 1, Topeiros 4, Zóné 1 (celkem 98 nápisů)

{\em Region měst ve vnitrozemí:} Augusta Traiana 23, Carasura 1, Discoduraterae 3, Filippopolis 29, Hadriánopolis 1, Hérakleia Sintská 2, Marcianopolis 4, Neiné 1, Nicopolis ad Istrum 28, Pautália 5, Plótinúpolis 2, Serdica 18, Traianúpolis 2, údolí středního toku řeky Strýmón 30 (celkem 149 nápisů)\footnote{Celkem sedm nápisů nebylo nalezeno v rámci regionu známých měst, editoři korpusů udávají jejich polohu vzhledem k nejbližšímu modernímu sídlišti, či uvádí muzeum, v němž se nachází.}

{\em Celkový počet individuálních lokalit}: 69

{\em Archeologický kontext nálezu:} funerální 5, sídelní 43 (z toho obchodní 10), náboženský 5, sekundární 25, jiný 3, neznámý 173

{\em Materiál:} kámen 246 (mramor 169, z Chalkedónu 1, z Prokonnésu 1; vápenec 49, jiný 10, z toho syenit 5, póros 1; neznámý 18), kov 2, neznámý 6

{\em Dochování nosiče}: 100 \letterpercent{} 36, 75 \letterpercent{} 35, 50 \letterpercent{} 44, 25 \letterpercent{} 43, oklepek 2, kresba 8, ztracený 8, nemožno určit 78

{\em Objekt:} stéla 148, architektonický prvek 89, socha 2, jiný 8 (z toho {\em instrumentum domesticum} 1, vojenský diplom 1) neznámý 7

{\em Dekorace:} reliéf 144, bez dekorace 110; reliéfní dekorace figurální 44 nápisů (vyskytující se motiv: jezdec 8, stojící osoba 6, skupina lidí 1, zvíře 1, Artemis 1, scéna lovu 1, funerální scéna/symposion 10, funerální portrét 9, jiný 3), architektonické prvky 99 nápisů (vyskytující se motiv: naiskos 9, sloup 9, báze sloupu či oltář 45, architektonický tvar/forma 18, geometrický motiv 4, florální motiv 18, věnec 2, jiný 7)

{\em Typologie nápisu:} soukromé 130, veřejné 117, neurčitelné 7

{\em Soukromé nápisy:} funerální 84, dedikační 49, vlastnictví 1, jiný 2\footnote{Několik nápisů mělo vzhledem ke své nejednoznačnosti kombinovanou funkci, proto je součet nápisů obou typů vyšší než celkový počet soukromých nápisů.}

Veřejné nápisy: seznamy 2, honorifikační dekrety 63, státní dekrety 6, nařízení 5, náboženský 13, jiný 24, neznámý 4

Délka: aritm. průměr 6,52 řádku, medián 5, max. délka 62, min. délka 1

{\em Obsah:} dórský dialekt 2, latinský text 22 nápisů{\bf ,} písmo římského typu 85; hledané termíny (administrativní termíny 40 - celkem 284 výskytů, epigrafické formule 29 - 199 výskytů, honorifikační 10 - 13 výskytů, náboženské 46 - 156 výskytů, epiteton 26 - počet výskytů 33)

{\em Identita:} řecká božstva 19, egyptská božstva 2, římská božstva 2, thrácká božstva 2, pojmenování míst a funkcí typických pro řecké náboženské prostředí{\bf ,} regionální epiteton 15, subregionální epiteton 11, kolektivní identita 18 termínů, celkem 55 výskytů - obyvatelé řeckých obcí z oblasti Thrákie 12, mimo ni 0; kolektivní pojmenování etnik či kmenů (Thráx 13, Rómaios 9, barbaros 1, Asianos 1, Kappadox 1), člen fýly 1; celkem 511 osob na nápisech, 94 nápisů s jednou osobou; max. 29 osob na nápis, aritm. průměr 2,02 osoby na nápis, medián 1{\bf ;} komunita multikulturního charakteru se zastoupením řeckého, římského a thráckého prvku, se silnou přítomností římského prvku, jména pouze řecká (11,06 \letterpercent{}), pouze thrácká (1,18 \letterpercent{}), pouze římská (31,25 \letterpercent{}), kombinace řeckého a thráckého (3,95 \letterpercent{}), kombinace řeckého a římského (15,81 \letterpercent{}), kombinace thráckého a římského (3,95 \letterpercent{}), kombinovaná řecká, thrácká a římská jména (6,32 \letterpercent{}), jména nejistého původu (12,23 \letterpercent{}), beze jména (14,22 \letterpercent{}); geografická jména z oblasti Thrákie 15, geografická jména mimo Thrákii 4;

\NC\AR
\HL
\HL
\stoptable

Do 2. st. n. l. bylo datováno celkem 254 nápisů, což představuje nárůst o 272 \letterpercent{} oproti předcházejícímu období. Jak je patrné z mapy 6.08 v Apendixu 2, poprvé v tomto období převládá epigrafická produkce ve vnitrozemí, a nikoliv na pobřeží. Většina lokalit s nápisy ve vnitrozemí se nacházela v povodí velkých řek či na trase {\em Via Diagonalis}, římské cesty spojující severozápad s jihovýchodem, procházející skrz města Serdicu, Filippopolis, Perinthos a Byzantion (Jireček 1877; Madzharov 2009, 70-31). Nápisy pocházely převážně z níže položených sídel, v horských oblastech se našly jednotlivé nápisy pouze v oblasti cest a průsmyků či v okolí vojenských posádek. Hlavními produkčními centry byla oblast údolí středního toku Strýmónu s městy Hérakleia Sintská a Neiné, Filippopolis, Nicopolis ad Istrum, Augusta Traiana, Byzantion a Perinthos. Oproti předcházejícím obdobím dochází k přesunu epigrafické produkce do většího počtu měst do jednoho dominantního centra. Oproti 1. st. n. l. upadá celková produkce Byzantia a Perinthu, což souvisí s přesunem hlavního města provincie do Filippopole, kde je naopak možné pozorovat markantní nárůst nápisů (Topalilov 2012, 13).

Použitým materiálem je výhradně kámen, a to zejména mramor ze 70 \letterpercent{}, vápenec z 18 \letterpercent{} a syenit z 2 \letterpercent{}. Nosiče nápisů mají ve dvou třetinách tvar stély, v necelé třetině tvar architektonického prvku, dále čtyři nápisy se nacházejí na sochách, jeden na mozaice a 19 na sarkofázích či jejich fragmentech.

\subsection[funerální-nápisy-13]{Funerální nápisy}

Funerálních nápisů se dochovalo celkem 84, z čehož 80 má povahu soukromého nápisu a čtyři byly zhotoveny na náklady města, konkrétně Maróneie. Oproti předcházejícímu období počet funerálních nápisů narostl více než třikrát, což je možné spojovat s nárůstem celkového počtu obyvatel, ale i proměnami přístupu obyvatelstva k zřizování funerálních nápisů. Nápisy pocházejí z celého území Thrákie, tedy jak pobřeží, tak vnitrozemí. Hlavními produkčními centry je údolí středního toku Strýmónu s městem Hérakleia Sintská, Neiné a Parthicopolis, odkud pochází 25 nápisů, dále Byzantion s 18 nápisy, Perinthos a Maróneia s 11 nápisy. Obecně funerální nápisy pocházejí z okolí velkých městských center, kde žilo nejvíce lidí.

Text funerálních nápisů poukazují na proměňující se složení společnosti a nárůst důležitosti římského vojska. Texty nesou standardní formule, podobně jako v předcházejících stoletích, \footnote{Invokační formule {\em chaire} se objevuje 16krát, oslovení okolojdoucího ({\em parodeita}) 11krát. termín označující hrob ({\em tymbos}) se objevuje jednou, termín {\em stélé} třikrát, termín {\em mnémé} 15krát, {\em mnémeion} čtyřikrát, termín pro sarkofág celkem třikrát ({\em soros}). Také se objevuje termín {\em chamosorion}, popisující pravděpodobně sarkofág umístěný na podstavci či přímo na zemi. Poprvé se čtyřikrát objevuje invokační formule vzývající podsvětní bohy v latině ({\em Dis Manibus}) a dvakrát v řečtině ({\em Theoi Katachthonioi}).} a jsou zhotovovány členy nejbližší rodiny a hrobky slouží k pohřbům více členů rodiny.\footnote{Celkem 22 nápisů vyjmenovává několik členů rodiny, po čtyřech nápisy zmiňují potomky a sourozence, v jednom případě rodiče a ve dvou přítele. Ve dvou případech se setkáváme s nápisem zhotoveným propuštěným otrokem pro svého bývalého pána.} Celkem 20 nápisů, pocházejících zejména z Byzantia a Perinthu, uvádí věk zemřelého, což je zvyk typický pro římské nápisy. Latinský text se objevil na šesti nápisech, většinou u nápisů patřícím vojákům či veteránům, kteří nesli římská jména. Vojáků se na nápisech objevuje celkem pět, od běžných vojáků až po legionáře a jezdce. Jiná, než vojenská povolání zmiňují funkce vždy po jednom výskytu kněžího ({\em hiereus}), člena městské rady ({\em búleutés}), člena {\em gerúsie} ({\em gerúsiastés}). Na nápisech figurují taktéž čtyři otroci, z čehož dva jsou propuštěnci. Sarkofágy, podobně jako v 1. st. n. l., pocházejí převážně z Perinthu (Hérakleii) a Byzantia a dalších pobřežních lokalit, ale dva pocházejí i z thráckého vnitrozemí. Z osmi sarkofágů z Perinthu (Hérakleie) a Byzantia pouze dva nesou ochrannou formuli, která zakazuje nové použití sarkofágu pod peněžní pokutou\footnote{Tato formule se objevila již na sarkofázích z Perinthu (Hérakleii) v 1. st. n. l. a tento zvyk se rozšířil do nedalekého Byzantia.}, ale téměř na všech se udává věk zemřelého, případně jeho původ a kdo nechal sarkofág zhotovit.

Geografický původ zemřelého poukazuje na zvýšenou migraci z oblastí Malé Asie, což je patrné zejména u nápisů pocházejících z Filippopole (Topalilov 2012, 13; Sharankov 2011, 143).\footnote{Celkem sedm nápisů udává geografický původ zemřelého, a to jako pocházejícího z Byzantia, Abdéry (dvakrát), Hérakleie, Filippopole, Kappadokie, Níkaie a Apameie v Bíthýnii.} Vyskytující se osobní jména i jsou nadále převážně řecká, nicméně je možné pozorovat nárůst jak jmen římských, tak především i jmen thráckých. Řecká jména představují 42 \letterpercent{}, thrácká jména představují 18 \letterpercent{}, římská jména představují necelou třetinu všech jmen. Thrácká jména se vyskytují jak samostatně, tak v malé míře i v kombinaci s řeckými i římskými jmény. Celé dvě třetiny nápisů s thráckým jménem pocházejí z údolí středního toku Strýmónu, kde se thrácká populace zapojovala do zvyku zhotovovat veřejně vystavené funerální nápisy zejména v první polovině 2. st. n. l.\footnote{Z celkem 22 nápisů v nichž se vyskytuje alespoň jedno thrácké jméno jich 14 pochází z okolí řeky Strýmón. Celkem se zde vyskytuje 40 osob, z nichž osm jsou ženy} Římská jména se vyskytují převážně samostatně, v menší míře v kombinaci s řeckými, a téměř minimálně v kombinaci s thráckými jmény. V kombinaci s thráckými jmény se vyskytují většinou u jedinců, kteří sloužili v římské armádě a přijali nové jméno, nicméně se i nadále odkazují na svůj thrácký původ.\footnote{Např. na nápise {\em IG Bulg} 5 5462 z Filippopole Ailios Polemón, {\em benefikarios}, věnoval svým předkům Beithytraleiovi, synovi Taséa(?) a Kouété, dceři Dydéa(?). Nápis Manov 2008 82 dokumentuje případ veterána se jmény Gaios Valerios Poudens, jehož otec nesl jméno Gaios Ourginios Poudens a matka jméno Moukasoké. Veteránova žena nesla jméno Severa, syn jméno Ioulios Maximos, a sestra Zaikaidenthé.} Kombinovaná jména, která by odkazovala na nedávné přijetí římského jména a s ním i spojených privilegií, se na rozdíl od 1. st. n. l. dochovala pouze v minimální míře. To může značit, že Thrákové nedostávali za svou službu právo nosit římská jména, či se plně adaptovali na systém tří jmen, zcela opustili od užívání původních thráckých jmen a stali se tak nerozlišitelnými od Římanů. Nicméně vzhledem k udržení thráckých jmen na nápisech ze 3. st. n. l. je toto vysvětlení nedostačující.

\subsection[dedikační-nápisy-13]{Dedikační nápisy}

Dedikačních nápisů se dochovalo celkem 49, z čehož 44 bylo soukromé povahy a pět bylo zhotoveno jako veřejný text, tj. dedikace císaři či text zhotovený na náklady obce. Nejvíce nápisů pochází z regionu Augusty Traiany, celkem s osmi nápisy, dále z údolí středního toku Strýmónu sedm nápisů, poté je Serdica s pěti nápisy.

Dedikace jsou věnovány jak božstvům nesoucím řecká jména, tak božstvům místním, případně božstvům, jež vznikla smíšením řeckých a místních tradic.\footnote{Objevuje se osm Asklépiovi, sedm Diovi, šest Dionýsovi a Artemidě, pět Apollónovi, jedna Hygiei, Sabaziovi, Sarápidovi, Héře, Athéně a Semelé.} Místní epiteta vznikla pravděpodobně ze jména místa, kde se vyskytovala svatyně.\footnote{Z místních božstev je to především hérós/theos {\em Karabasmos}, {\em Manimazos}, {\em Marón}, {\em Pyrmeroulas}, {\em Zbelsúrdos} a dále místní přízviska spojená s Apollónem a Asklépiem a Artemidou, jako např. {\em Aulariokos}, {\em Epékoos}, {\em Tadénos}, {\em Beeuchios}, {\em Patróos}, {\em Suidénos} (Janouchová 2017 k rozšíření přízviska {\em Patróos}).} Lokální thrácké kulty se vyskytovaly na více než jednom místě pouze v ojedinělých případech, většinou se jednalo kult pevně svázaný s daným místem.

Přes polovinu osobních jmen na dedikačních nápisech tvořila římská jména, třetinu jména řecká a zhruba 10 \letterpercent{} jména thrácká. Nápisy s thráckými jmény pocházely výhradně z vnitrozemských oblastí z okolí Augusty Traiany a středního toku Strýmónu a byly věnovány jak řeckým božstvům, tak řeckým božstvům nesoucím místní přízviska.\footnote{Příkladem nápisů věnovaných řeckému božstvu s místním přízviskem jsou dva nápisy z Kabylé, kde byl ve 2. st. n. l. umístěný vojenský tábor pomocných jednotek a pravděpodobně zde žili Thrákové společně s Řeky. Nápisy {\em Kabyle} 18 a 19 nesou věnování Apollónovi {\em Tadénovi}, provedené členy {\em cohors II Lucensium}, kteří sami nesli thrácká osobní jména (Velkov 1991, 23-24).} Zhruba polovina těchto dedikací byla věnována Thráky sloužícími v římské armádě, kteří zároveň nesli římská a thrácká jména.

\subsection[veřejné-nápisy-13]{Veřejné nápisy}

Poprvé větší část veřejných nápisů pochází z vnitrozemí, a to zejména z velkých městských center a jejich nejbližšího okolí. Tento fakt jistě souvisí s urbanizačními aktivitami a posílením politické moci a významu městské samosprávy za vlády Trajána a Hadriána (Topalilov 2012, 14-15). Celkem se dochovalo 117 veřejných nápisů, což představuje oproti 1. st. n. l. zhruba čtyřnásobný nárůst celkového počtu. Tento trend je pozorovatelný zejména ve oblasti vnitrozemské Thrákie v okolí měst Nicopolis ad Istrum a Filippopolis, odkud pochází 21 nápisů a Augusta Traiana se 14 nápisy. Nejvíce se dochovalo honorifikačních nápisů, ale objevují se ve větší míře i milníky a nápisy dokumentující stavební aktivity.

Honorifikační nápisy představují nejvýznamnější skupinu veřejných nápisů. Oproti předcházejícímu období se jedná o téměř osminásobný nárůst z osmi na 63 honorifikačních nápisů, což dokazuje oblíbenost tohoto druhu nápisů. Honorifikační nápisy pocházejí z měst jak na pobřeží, tak zejména ve vnitrozemí z regionu města Nicopolis ad Istrum, odkud pochází 18 textů, a z regionu Filippopole, kde bylo nalezeno 14 textů. Honorifikační dekrety jsou i nadále vydávány politickou autoritou města, a to pod patronátem římského císaře.\footnote{Nejčastějšími termíny reprezentujícími politickou autoritu měst jsou {\em búlé} s 24 výskyty, {\em démos} s 25, {\em polis} se 7 výskyty.} Tradiční forma honorifikačních nápisů, kdy se dobrodinci ({\em euergetés}) udělují konkrétní privilegia za vykonané dobrodiní, je nicméně vystřídána novou formou, kdy namísto významného jedince je hlavní pozornost věnována císaři, na jehož počet většina nápisů a s nimi spojených monumentů a budov vzniká.\footnote{Texty obsahují tradiční invokační formuli ve 27 případech ({\em agathé týché}) a jsou převážně psány řecky. Dva texty jsou psány latinsko-řecky, kde latinský text je uváděn první a řecký text je překladem latinského textu a hlavní autoritou těchto nápisů je vždy římský císař. Další dva nápisy jsou čistě latinské a jsou určeny přímo římskému císaři.} Tato proměna honorifikačních nápisů je patrná již v průběhu 1. st. n. l. i v jiných částech římské říše (Van Nijf 2015, 240). Důraz je kladen převážně na osobní kvality a zásluhy jedinců, jejich pověst a společenskou prestiž, což může odrážet i větší míru stratifikace tehdejší společnosti.

Milníky a nápisy dokumentující stavební aktivity organizované provinciální samosprávou poukazují na nárůst stavebních aktivit v 2. st. n. l., zejména po roce 116 n. l. Celkem devět milníků nápisů pochází jak z okolí významných měst ve vnitrozemí, tak na pobřeží. Silnice {\em Via Diagonalis} byla v této době doplněna dalšími menšími silnicemi spojujícími významná města právě s touto nejdůležitější dopravní tepnou celého Balkánu. Dochované milníky poukazují na existenci cest v druhé polovině 2. st. n. l. v okolí Marcianopole a Odéssu, dále i na opravy již zmíněné {\em Via Diagonalis} spojující Serdicu a Byzantion. Silnice využívala primárně římská armáda, nicméně z této infrastruktury těžilo i místní obyvatelstvo a silnice umožnily lepší propojení regionu a pohyb obyvatel. \footnote{Císař mohl udělit speciální práva a ochranu konkrétní skupině obyvatel, jak je tomu na nápise {\em I Aeg Thrace} 185 z Maróneie. Císař Hadrián tímto dekretem určeným obyvatelům Maróneie z roku 131 n. l. mimo jiné udělil právo k užívání a ochranu po dobu putování po cestě z Maróneii do Filipp, čímž se pravděpodobně myslela {\em Via Egnatia}, jedna z nejvýznamnějších římských cest. Na civilní využití menších cest a mostů v okolí Pautálie poukazuje nápis {\em SEG} 54:648 z poloviny 2. st. n. l.}

Další nápisy zaznamenávají stavbu akvaduktů v Odéssu, městského opevnění ve Filippopoli a Serdice, a lázní v Pautálii a Augustě Traianě či rekonstrukci divadla ve Filippopoli. Dva dochované nápisy označují hranice území Abdéry jako samosprávné jednotky pod autoritou císaře.\footnote{{\em I Aeg Thrace} 78 a 79. Předpokládá se, že hraniční kameny se vyskytovaly na pomezí území měst vcelku často, do dnešních dnů se jich dochovalo bohužel pouze několik exemplářů.} Stavební aktivity většinou financovali místodržící či vysocí úředníci na počest císaře a soudě dle osobní jmen, byli tito vysoce postavení muži zejména římského původu, ale vyskytovali se i jedinci nesoucí kombinaci thráckých a římských jmen.\footnote{Jako např. Titus Vitrasius Pollio z nápisu {\em IG Bulg} 1,2 59 a Poplios Ailios Aulouporis z nápisu {\em IG Bulg} 5 5334.}

Veřejným textům náboženského charakteru zcela dominuje císařský kult, nicméně v seznamech věřících převládají řecká a thrácká jména.\footnote{Na nápise {\em IG Bulg} 4 1925 se setkáváme se seznamem příslušnic mystérií Velké Matky a Attida, kde se vyskytují jak řecká, tak římská ženská jména, avšak text je poměrně špatně dochovaný.} Osoba císaře hraje zásadní roli i v rozložení politické moci.\footnote{Termíny označující autoritu císaře jsou s 30 výskyty {\em autokratór}, dále s 26 výskyty {\em kaisar}, s 12 {\em hypatos} a se 17 {\em hegémón}. Císař bývá někdy titulován i jako otec vlasti, {\em patér patridos}, a {\em archiereus}, což je řecká verze titulu {\em pontifex maximus} (Mason 1974, 115-116).} Autoritu císaře přímo v provincii reprezentuje místodržící a jeho zástupce, na nápisech často titulovaný jako {\em presbeutés} a {\em antistratégos}, latinsky {\em legatus Augusti pro praetore} (Mason 1974, 153-155), a to konkrétně ve 24 případech. Jako zástupce císařské moci vystupuje v několika případech i {\em thrakarchés}, tedy vysoký úředník s náboženským zaměřením, podobným nejvyššímu knězi celé provincie (Lozanov 2015, 82-83). Úřad {\em thrakarcha} obvykle zastupoval muže nesoucí kombinovaná thrácká a římská jména, což dokazuje jeho aristokratický thrácký původ, který nebyl překážkou kariéry v rámci provinciálních institucí (Sharankov 2005, 532).

S dobou vlády Trajána se taktéž spojují administrativní reformy, díky nimž narostla autorita měst ve vnitrozemí na úkor dřívějšího uspořádání založeného na kmenovém principu, což se projevilo i na výskytu termínů užívaných na nápisech.\footnote{Parissaki (2013, 83-84) se domnívá, že dříve fungující systém stratégií byl za doby vlády Trajána nahrazen novým administrativním systémem, v němž hrála zásadní roli městská samospráva.} Z jednotlivých úřadů či osob zastávajících funkce v rámci samosprávy se objevuje {\em búlé} ve 34 případech {\em démos} ve 42, {\em polis} ve 24, dále {\em archón} ve čtyřech nápisech, po dvou nápisech {\em fýlé} a {\em pontarchés}\footnote{Úřad {\em pontarcha} je znám z císařské doby a jedná se o nejvyššího úředníka uskupení šesti měst na pobřeží Černého moře, který zastupoval císařskou autoritu a zároveň vykonával úřad nejvyššího kněze (Cook 1987, 30). Toto uskupení černomořské Hexapole zahrnovalo pobřežní města převážně z provincie {\em Moesia Inferior} jako Histria, Tomis, Kallatis, Dionýsopolis, Odéssos a Mesámbria a usuzuje se, že vzniklo v době vlády císaře Trajána či Hadriána, tedy před polovinou 2. st. n. l. Vytvořením této organizace původně řecká města fakticky ztratila svou autonomii, či její pozůstatky, a spadala přímo pod autoritu císaře (Lozanov 2015, 83).} a po jednom nápise {\em agaranomos} a {\em gerúsia}. V rámci posílení pozice se města z Thrákie společně sdružovala v tzv. {\em koinon tón Thrákón}, s Filippopolí jako hlavním městem a sídlem spolkového sněmu ({\em métropolis}; Sharankov 2005, 518-531).\footnote{Přesná funkce tohoto uskupení měst není jasná, nicméně jeho vznik byl pravděpodobně inspirován vzorem ze sousední Bíthýnie pro usnadnění provinciální administrativy (Lozanov 2015, 82).} Politické centrum tehdejší provincie se tak přesunulo z Perinthu do vnitrozemské Filippopole, čemuž odpovídá i přesun jednoho z největších producentů veřejných, tak i soukromých nápisů.

\subsection[shrnutí-17]{Shrnutí}

Na nápisech z 2. st. n. l. lze velmi dobře pozorovat rostoucí vliv Říma nejen na celkovou epigrafickou produkci, ale i na proměňující se strukturu společnosti. Zvyšující se epigrafická produkce souvisí s narůstající společenskou komplexitou, která se mimo jiné projevuje narůstající administrativou a institucionální zátěží spojenou s vedením velké říše. To se projevuje jak větším počtem veřejných nápisů regulujících společenské uspořádání římské provincie, ale i výsadní pozicí císaře, coby svrchované politické autority. Rostoucí role vojska je patrná jak na obsahu nápisů, často publikovaných vojáky či veterány, proměňujícími se zvyklostmi a přejímání nových vzorců chování, ale i samotným rozmístěním epigraficky aktivních komunit. Patrná je koncentrace epigrafické produkce v okolí městských center, podél vojenských cest a v okolí vojenských a polovojenských osídlení.

\section[charakteristika-epigrafické-produkce-ve-2.-st.-n.-l.-až-3.-st.-n.-l.]{Charakteristika epigrafické produkce ve 2. st. n. l. až 3. st. n. l.}

Nápisy datované do 2. a 3. st. př. n. l. pocházejí ze dvou třetin z vnitrozemí s největší produkcí v okolí Augusty Traiany, středního toku řeky Strýmón a Filippopole. Poprvé dedikační nápisy převažují nad funerálními nápisy, což pravděpodobně souvisí i se zvyšujícím se počtem místních kultů a variabilitou vyobrazení božstev na nápisech. Složení společnosti je i nadále multikulturní, s dominantním římským prvkem, nicméně celkově dochází k většímu zapojení thrácké populace do epigrafické produkce.

\placetable[none]{}
\starttable[|l|]
\HL
\NC {\em Celkem:} 182 nápisů

{\em Region měst na pobřeží:} Anchialos 2, Byzantion 7, Dionýsopolis 2, Madytos 1, Maróneia 8, Mesámbria 3, Odéssos 9, Perinthos (Hérakleia) 18, Sélymbria 3, Zóné 1 (celkem 54 nápisů)

{\em Region měst ve vnitrozemí:} Augusta Traiana 49, Filippopolis 16, Hadriánopolis 2, Marcianopolis 5{\bf ,} Nicopolis ad Nestum 2, Nicopolis ad Istrum 11, Pautália 3, Plótinúpolis 4, Serdica 8, Traianúpolis 3, údolí středního toku řeky Strýmón 21 (celkem 124 nápisů)\footnote{Celkem dva nápisy nebyly nalezeny v rámci regionu známých měst, editoři korpusů udávají jejich polohu vzhledem k nejbližšímu modernímu sídlišti, či uvádí muzeum, v němž se nachází. Další dva nápisy pocházejí z neznámého místa v Bulharsku.}

{\em Celkový počet individuálních lokalit}: 61

{\em Archeologický kontext nálezu:} funerální 1, sídelní 14 (z toho obchodní 7), náboženský 36, sekundární 22, neznámý 109

{\em Materiál:} kámen 181 (mramor 138; vápenec 28, jiný 3, z toho syenit 2, varovik 1; neznámý 12), neznámý 1

{\em Dochování nosiče}: 100 \letterpercent{} 21, 75 \letterpercent{} 27, 50 \letterpercent{} 27, 25 \letterpercent{} 44, kresba 13, ztracený 2, nemožno určit 48

{\em Objekt:} stéla 136, architektonický prvek 36, socha 2, mozaika 1, jiný 5, neznámý 2

{\em Dekorace:} reliéf 120, malovaná 2, bez dekorace 58; reliéfní dekorace figurální 73 nápisů (vyskytující se motiv: jezdec 33, sedící osoba 1, stojící osoba 5, skupina lidí 1, zvíře 1, Artemis 3, Asklépios 2, Héraklés 2, Zeus a Héra 1, scéna lovu 2, funerální scéna/symposion 7, funerální portrét 8, jiný 8), architektonické prvky 48 nápisů (vyskytující se motiv: naiskos 8, sloup 5, báze sloupu či oltář 21, architektonický tvar/forma 13, florální motiv 6, věnec 4, jiný 7)

{\em Typologie nápisu:} soukromé 126, veřejné 46, neurčitelné 10

{\em Soukromé nápisy:} funerální 54, dedikační 76, jiný 1\footnote{Několik nápisů mělo vzhledem ke své nejednoznačnosti kombinovanou funkci, proto je součet nápisů obou typů vyšší než celkový počet soukromých nápisů.}

{\em Veřejné nápisy:} seznamy 3, honorifikační dekrety 30, státní dekrety 3, náboženský 4, jiný 3, neznámý 3

{\em Délka:} aritm. průměr 5,0 řádku, medián 4, max. délka 48, min. délka 1

{\em Obsah:} dórský dialekt 0, latinský text 4 nápisy, písmo římského typu 74; hledané termíny (administrativní termíny 31 - celkem 139 výskytů, epigrafické formule 24 - 128 výskytů, honorifikační 3 - 3 výskyty, náboženské 42 - 112 výskytů, epiteton 26 - počet výskytů 40)

{\em Identita:} řecká božstva 19, egyptská božstva 4, římská božstva 2, pojmenování míst a funkcí typických pro řecké náboženské prostředí, nárůst počtu lokálních kultů, regionální epiteton 12, subregionální epiteton 14, kolektivní identita 13 termínů, celkem 24 výskytů - obyvatelé řeckých obcí z oblasti Thrákie 7, mimo ni 2; kolektivní pojmenování etnik či kmenů (Thráx 2, Rómaios 3, Bíthýnos 1, Acháios 1); celkem 256 osob na nápisech, 85 nápisů s jednou osobou; max. 24 osob na nápis, aritm. průměr 1,41 osoby na nápis, medián 1; komunita multikulturního charakteru se zastoupením řeckého, římského a thráckého prvku, se silnou přítomností římského prvku, jména pouze řecká (11,53 \letterpercent{}), pouze thrácká (6,59 \letterpercent{}), pouze římská (24,72 \letterpercent{}), kombinace řeckého a thráckého (4,39 \letterpercent{}), kombinace řeckého a římského (12,08 \letterpercent{}), kombinace thráckého a římského (2,74 \letterpercent{}), kombinovaná řecká, thrácká a římská jména (4,94 \letterpercent{}), jména nejistého původu (6,02 \letterpercent{}), beze jména (26,92 \letterpercent{}); geografická jména z oblasti Thrákie 12, geografická jména mimo Thrákii 4;

\NC\AR
\HL
\HL
\stoptable

V případě skupiny 182 nápisů datovaných do 2. a 3. st. n. l. se jedná o nárůst o 93 \letterpercent{} oproti skupině nápisů z 1. až. 2. st. n. l. Nápisy pocházejí z většiny z thráckého vnitrozemí z okolí hlavních městských center a okolí hlavních komunikačních tepen, jak je patrné z mapy 6.08 v Apendixu 2. Největším producentem se stává Augusta Traiana, jakožto hlavní město provincie {\em Thracia}. Producenty střední velikosti jsou další velká centra městského typu, jako Perinthos, Nicopolis ad Istrum a oblast středního toku Strýmónu

Převážná část nosičů nápisů je z kamene, nicméně dva nápisy jsou zhotovené z kovu, a jeden na mozaice.\footnote{Nápisy na kovu představují vojenský diplom ({\em AE} 2007, 1259) a {\em instrumentum domesticum} čili předmět běžné denní potřeby ({\em AE} 2004, 1302, {\em instrumentum domesticum}, Chaniotis 2005, 92). Nápis {\em SEG} 54:652 se nachází na mozaice. Užitá osobní jména poukazují jak na řecký původ, tak na římské onomastické tradice u osob, které na nápisech figurují, ať už jako majitelé, či zhotovitelé. Nápisy pocházejí z okolí vojenského tábora v Kabylé a z vojenského tábora ve městě Nicopolis ad Istrum na Dunaji. Nápisy pocházejí nepřímo z kontextu spojeného s přítomností a projevy armádních struktur na území Thrákie. Vzhledem k jejich malému počtu a krátkému rozsahu však není možné vyvozovat nic dalšího.}

\subsection[funerální-nápisy-14]{Funerální nápisy}

Funerálních nápisů se dochovalo celkem 54, z čehož 52 má povahu soukromého nápisu a dva byly zhotoveny na náklady města.\footnote{{\em SEG} 48:894 a {\em I Aeg Thrace} 217.} Nápisy pocházejí z celého území Thrákie, tedy jak z pobřeží, tak z vnitrozemí. Hlavními produkčními centry je údolí středního toku Strýmónu, odkud pochází 14 nápisů, dále Perinthos s devíti nápisy a Maróneia se šesti nápisy.

Text nápisů pozvolna opouští od tradičně používaných formulí a stejně jako u skupiny nápisů z 2. st. n. l. se objevují nové termíny, které popisují nově vzniklé skutečnosti a předměty, jako termíny pro sarkofág a formule zajišťující právní ochranu jejich obsahu.\footnote{Invokační formule {\em chaire} se objevuje pouze čtyřikrát, oslovení okolojdoucího ({\em parodeita}) pouze čtyřikrát. termín označující hrob ({\em tymbos}) se objevuje jednou, termín {\em stélé} čtyřikrát, {\em mnémeion} jednou, termín pro sarkofág celkem pětkrát ({\em soros, latomeion, diathéké}). Třikrát objevuje invokační formule vzývající podsvětní bohy v latině ({\em Dis Manibus}) a ani jednou v řečtině ({\em Theoi Katachthonioi}). Celkem čtyři nápisy uvádí věk zemřelého, což je typický zvyk pro římské nápisy. Podobně jako u nápisů datovaných do 2. st. n. l. pochází skupina devíti sarkofágů zejména z lokalit v okolí Propontidy (Perinthos, Byzantion), z nichž čtyři nesou ochrannou formuli, která zakazuje nové použití sarkofágu pod peněžní pokutou. Osobní jména na těchto sarkofázích jsou převážně řeckého původu.} Obsah textů poukazuje na nadále se proměňující složení společnosti, či alespoň narůstající manifestaci nejrůznějších životních drah a profesí na nápisech. Stejně tak se i nadále objevují zvyklosti typické pro nápisy z římské doby, stejně jako u nápisů z 1. a zejména z 2. st. n. l.\footnote{Texty jsou zhotovovány členy nejbližší rodiny a hrobky slouží k pohřbům více členů rodiny. Celkem 26 nápisů zmiňuje společný hrob s partnerem zemřelého či zmiňuje členy rodiny. Vojáci se na nápisech objevují celkem třikrát, konkrétně jde o legionáře, jezdce a {\em carceraria}, hlídače vojenského vězení a jednou jako veterán. Jiná, než vojenská povolání zmiňují funkce vždy po jednom výskytu kněžího ({\em archiereus}, {\em hiereus}), člena městské rady ({\em búleutés}), námořního obchodníka ({\em naukléros}), či patrona. Geografický původ zemřelého nápisy neudávají.}

Vyskytující se osobní jména i jsou nadále převážně řecká, nicméně je možné pozorovat nárůst jak jmen římských, tak především i jmen thráckých. Řecká jména představují 38 \letterpercent{}, což představuje zhruba 12 \letterpercent{} pokles oproti nápisům z 1. až 2. st. n. l. Římská jména představují 30 \letterpercent{}, což značí 10 \letterpercent{} pokles. Thrácká jména naopak zaznamenala nárůst ze zhruba 3 \letterpercent{} až na 20 \letterpercent{}. Největší koncentrace devíti nápisů nesoucích thrácká jména pochází v údolí středního toku Strýmónu a dále z okolí Augusty Traiany a Filippopole po dvou nápisech. Osoby nesoucí thrácká jména většinou odkazují na svůj původ i pomocí thráckého jména rodiče či dalších členů rodiny, případně i prohlášením o původu.\footnote{Příkladem je nápis {\em IG Bulg} 3,2 1794, který patří Apollónidovi, synovi Aulozénida ze Sapaiké, který je však pochován v thráckém vnitrozemí v lokalitě známé jako Dodoparon nedaleko středního toku řeky Tonzos (Janouchová 2017).} K mísení thráckých a římských onomastických tradic dochází na funerálních nápisech zcela minimálně a důležitou roli začíná v tomto období hrát spíše thrácká identita a příslušnost k místní komunitě.

\subsection[dedikační-nápisy-14]{Dedikační nápisy}

Dedikačních nápisů se dochovalo celkem 76, což představuje čtyřnásobný nárůst oproti nápisům z 1. až 2. st. n. l. Nejvíce nápisů pochází z regionu Augusty Traiany celkem se 33 nápisy, dále sem patří Serdica s osmi nápisy, obce z údolí středního toku Strýmónu se sedmi nápisy a Filippopolis s šesti nápisy.

Dedikace jsou věnovány jak božstvům nesoucím řecká jména, tak božstvům místním či smíšeným.\footnote{Celkem se dochovalo 24 dedikací Apollónovi, šest Asklépiovi, tři Diovi a Artemidě, dvě Hygiei a Sabaziovi, po jedné Héře a Sarápidovi, Ísidě, Anúbidovi, Harpokratiónovi, Músám, Horám, Athéně, Héfaistovi, Matce bohů, Hérakleovi, Déméter a Télesforovi. Z místních božstev je to především {\em hérós}/{\em theos} nesoucí lokální přízviska či božstvo s lokálním přízviskem, jako Maró{\em n, Poénos, Kersénos, Dabatopiénos, Zerdénos, Estrakénos, Raniskelénos, Zbelsúrdos, Aularkénos, Téradéenos} či {\em Kendreisos}.} Podle počtu výskytů na nápisech představuje Apollón nejpopulárnější božstvo, které bylo dle množství místních epitet rozšířeno do celé řady míst a do velké míry splynulo s místními kulty. Apollón se stal nejoblíbenějším božstvem a je mu určeno celkem 24 dedikací pocházejících z oblastí jižně od pohoří Haimos, zejména v okolí řeky Tonzos a Hebros, a dále v Sélymbrii a Serdice. Mezi jeho nejčastěji užívaná místní epiteta patří: {\em Foibos}, {\em Poénos}, {\em Kersénos}, {\em Zerdénos}, {\em Epékoos}, {\em Genikos}, {\em Geniakos}, {\em Patróos}, {\em Raniskelénos}, {\em Aularkénos}, {\em Lykios}, {\em Kendreisos} a {\em Téradéenos}, ale bývá oslovován též jako {\em hérós}, {\em theos} či {\em kyrios} (Goceva 1992; Bujukliev 1997). Největší počet dedikací pochází ze dvou velkých svatyní Apollóna s místním přízviskem {\em Téradéenos} a {\em Zerdénos}, nalézajících se u vesnice Kran v Kazanlackém údolí, nedaleko místa hellénistické Seuthopole (Tabakova 1959; Tabakova-Tsanova 1980).\footnote{Ač Seuthopole zanikla již na začátku 3. st. př. n. l., je možné, že určité povědomí o řeckém náboženství zůstalo zakořeněno v místní populaci a postupem času se transformovalo do podoby místních kultů, které se v římské době začínají objevovat i na nápisech.} Jak dokazují dochovaná osobní jména, kult Apollóna byl přístupný všem a relativně vysoký počet thráckých jmen může naopak naznačovat oblíbenost kultu Apollóna mezi místní populací.\footnote{Celkem se dochovala jména 12 mužů, z nichž čtyři nesli thrácká jména, čtyři pouze římská jména a tři řecká jména, jedno jméno nebylo možné určit vzhledem ke špatnému stavu dochování. Dedikanti byli ve dvou případech vojáci, v jednom případě členem {\em búlé} a v jednom případě se jednalo o dohlížitele nad agorou z Plotínopole.} Teorii o splynutí kultu Apollóna s lokálními kulty taktéž nahrává forma nosiče nápisů a jeho dekorace: 16 nápisů neslo reliéfní dekoraci zobrazující jezdce na koni, taktéž známého jako fenomén tzv. thráckého jezdce, který byl oblíben i mezi thráckými vojáky a veterány (Kazarow 1938; Dimitrova 2002; Boteva 2002; 2005; 2007; Oppermann 2006). Totéž platí i o dalších oblíbených kultech, jako je kult Asklépia.\footnote{Asklépiovi je věnováno celkem šest dedikací a božstvo je s oslovováno jako {\em kyrios}, {\em theos} či přízviskem {\em Liménios}. Asklépios je uctíván dohromady s Hygieí a Télesforem, a to lidmi se jmény řeckého i římského původu, nikoliv však thráckého. Nápisy pocházejí především ze svatyně u vesnice Slivnica v regionu města Serdica (Boteva 1985).}

Obecně je možné sledovat větší zapojení Thráků do publikace nápisů. Pokud srovnáme poměr osobních jmen na dedikačních nápisech vůči poměru nápisů na všech nápisech z daného období, je možné sledovat 16,5\letterpercent{} nárůst z 12,5\letterpercent{} zastoupení thráckých jmen na všech nápisech z 2. st. n. l. na 29 \letterpercent{} zastoupení na dedikačních nápisech.\footnote{U jmen řeckých je to pokles o 10 \letterpercent{} a u jmen římských pokles o 1,5 \letterpercent{}, nicméně i přesto římská jména představují přes 40 \letterpercent{} všech jmen na dedikačních nápisech a řecká jména jednu čtvrtinu.} Oproti jiným druhům nápisů byli Thrákové byli více epigraficky aktivní, pokud šlo o náboženské aktivity, čemuž by odpovídalo i narůstající množství lokálních kultů, které vznikly spojením místní tradice a řeckého náboženského systému.

\subsection[veřejné-nápisy-14]{Veřejné nápisy}

Celkem se dochovalo 46 veřejných nápisů datovaných do 2. až 3. st. n. l., což představuje trojnásobný nárůst oproti skupině nápisů datovaných do 1. až 2. st. n. l. Nejvíce nápisů pochází z Augusty Traiany s 10 nápisy, z Perinthu a Nicopolis ad Istrum se sedmi nápisy a z Filippopole s pěti nápisy. Nejčastějším typem s 29 výskyty jsou i nadále honorifikační nápisy vydávané městskými institucemi. Dochází k pokračujícím změnám epigrafického jazyka a standardizaci veřejných nápisů, stejně tak k proměně formulí typických pro veřejné nápisy. I nadále dekrety vydává {\em búlé} a {\em démos} pod patronátem císaře\footnote{Císař je označován termíny {\em kaisar} sedmkrát a {\em autokratór} 17krát.}, nicméně nyní se setkáváme i s přídomky {\em kratistos, hieros, hierotatos} či {\em lamprotatos}, které jsou spojovány s jednotlivými městskými institucemi. Podobné označení se městským institucím odstávalo i v římských provinciích Malé Asie, a to přibližně ve stejné době, což je příznakem jisté standardizace epigrafického jazyka napříč římskými provinciemi (Heller 2015, 250-253). K určitému sjednocení formy nápisů napříč městy dochází i pokud jde o uvádění císařských titulů, císařovi rodiny a zařazení nápisů do daného roku císařovi vlády.\footnote{Např. na nápise {\em IG Bulg} 2 617 a 624 z Nicopolis ad Istrum.} Na nápisech jsou často uváděni i vysocí provinciální úředníci, za jejichž služby došlo k vydání nápisu, či došlo např. k stavebním aktivitám.\footnote{Termín {\em presbeutés} a {\em antistratégos} se vyskytuje sedmkrát, {\em thrakarchés} jednou, {\em búleutés} dvakrát, {\em efébarchos} jednou, {\em synedrion} jednou, {\em métropolis} jednou.}

Z dalších druhů nápisů se dochoval pouze jeden milník, stejně tak jako jeden nápis dokumentující stavební aktivity veřejného charakteru, či označení hranic regionu města.\footnote{{\em Perinthos-Herakleia} 38, 40 a {\em IG Bulg} 5 5540.} Z dalších dokumentů se dochovaly dva seznamy osob, které dobře zachycují onomastické zvyky ve vztahu ke kolektivní identitě. Na nápisech {\em IG Bulg} 1,2 50 a 51 představující seznam věřících boha Dionýsa a seznam efébů z Odéssu zcela převládají řecká a thrácká jména. To může být důsledek kulturně-společenských norem, které měly za následek, že osoby nesoucí římská jména do těchto skupin nevstupovali, případně si v daném společensko-kulturním kontextu vystupovali pod jinou součástí své identity, jejíž součástí nebylo římské jméno.

\subsection[shrnutí-18]{Shrnutí}

Ač epigraficky aktivní populace zůstává převážně řecká, dochází k nárůstu římského elementu, ale zejména i k sebeuvědomění a zapojení thráckého obyvatelstva. Thrákové se objevují zejména v souvislosti s vojenskou službou a nápisy využívají jako formu prezentace společenského statutu. Thrácký prvek se objevuje zejména v místních svatyních, které jsou ale přístupné všem obyvatelům. Nadále dochází i k upevňování zvyklostí a jejich epigrafických projevů, které se poprvé objevily s římskou přítomností, jako např. rozšíření funerálních sarkofágů či uvádění věku zemřelého. Veřejné nápisy pak poukazují na míru regionalismu a autonomie jednotlivých městských samospráv, avšak zaštítěného všudypřítomnou autoritou římského císaře a provinciálních institucí.

\section[charakteristika-epigrafické-produkce-ve-3.-st.-n.-l.]{Charakteristika epigrafické produkce ve 3. st. n. l.}

Nápisy datované do 3. st. n. l. pocházejí ze tří čtvrtin z vnitrozemských městských center, jako je Augusta Traiana, Filippopolis a Serdica. Zvýšený výskyt geografických termínů a kolektivních pojmenování z oblastí mimo Thrákii naznačuje otevírání společnosti a migraci lidí z dalších částí římské říše. I nadále převládají dedikační nápisy, často věnované božstvům lokálního charakteru. Veřejné nápisy představují téměř polovinu všech nápisů, poprvé se v nich objevuje i místní samospráva na úrovni vesnic, nikoliv pouze na úrovni měst.

\placetable[none]{}
\starttable[|l|]
\HL
\NC {\em Celkem:} 390 nápisů

{\em Region měst na pobřeží:} Abdéra 5, Anchialos 3, Bizóné 1, Byzantion 20, Caron Limen 2, Dionýsopolis 7, Ferai 1, Maróneia 21, Maximiánúpolis 1, Mesámbria 1, Odéssos 7, Perinthos (Hérakleia) 16, Sélymbria 2, Strýmé 1, Topeiros 9 (celkem 97 nápisů)

{\em Region měst ve vnitrozemí:} Augusta Traiana 58, Discoduraterae 7, Filippopolis 49, Hadriánopolis 1, Hérakleia Sintská 2, Marcianopolis 4, Neiné 3, Nicopolis ad Istrum 25, Pautália 7, Plótinúpolis 4, Serdica 91, Traianúpolis 1, údolí středního toku řeky Strýmón 29 (celkem 281 nápisů)\footnote{Celkem 12 nápisů nebylo nalezeno v rámci regionu známých měst, editoři korpusů udávají jejich polohu vzhledem k nejbližšímu modernímu sídlišti. Devět nápisů pochází z pobřeží Egejského moře a tři z bulharského vnitrozemí.}

{\em Celkový počet individuálních lokalit}: 115

{\em Archeologický kontext nálezu:} funerální 9, sídelní 47 (z toho obchodní 20), náboženský 79, sekundární 44, jiný 3, neznámý 208

{\em Materiál:} kámen 385 (mramor 237, z Prokonnésu 1; vápenec 104, jiný 17, z toho syenit 6, varovik 1, granit 1, tuf 1, břidlice 1; neznámý 27), keramika 2, neznámý 3

{\em Dochování nosiče}: 100 \letterpercent{} 45, 75 \letterpercent{} 47, 50 \letterpercent{} 74, 25 \letterpercent{} 60, oklepek 6, kresba 25, ztracený 8, nemožno určit 125

{\em Objekt:} stéla 215, architektonický prvek 145, socha 18, jiný 1, neznámý 11

{\em Dekorace:} reliéf 259, malovaná 1, bez dekorace 130; reliéfní dekorace figurální 115 nápisů (vyskytující se motiv: jezdec 59, sedící osoba 2, stojící osoba 7, skupina lidí 3, zvíře 1, Artemis 1, Asklépios 3, Héraklés 1, Zeus a Héra 1, Dionýsos 2, scéna lovu 2, funerální scéna/symposion 3, funerální portrét 13, socha 1, jiný 8), architektonické prvky 136 nápisů (vyskytující se motiv: naiskos 12, sloup 49, báze sloupu či oltář 70, architektonický tvar/forma 17, geometrický motiv 2, florální motiv 15, věnec 2, jiný 13)

{\em Typologie nápisu:} soukromé 203, veřejné 176, neurčitelné 11

{\em Soukromé nápisy:} funerální 76, dedikační 128, vlastnictví 2, jiný 5\footnote{Několik nápisů mělo vzhledem ke své nejednoznačnosti kombinovanou funkci, proto je součet nápisů obou typů vyšší než celkový počet soukromých nápisů.}

{\em Veřejné nápisy:} seznamy 7, honorifikační dekrety 97, státní dekrety 11, náboženský 1, jiný 52, neznámý 9

{\em Délka:} aritm. průměr 8,62 řádku, medián 7, max. délka 270, min. délka 1

{\em Obsah:} dórský dialekt 1, iónsko-attický 1, latinský text 12 nápisů, písmo římského typu 168; hledané termíny (administrativní termíny 53 - celkem 502 výskytů, epigrafické formule 26 - 315 výskytů, honorifikační 9 - 22 výskytů, náboženské 30 - 177 výskytů, epiteton 24 - počet výskytů 55)

{\em Identita:} řecká božstva 13, egyptská božstva 0, římská božstva 1, pojmenování míst a funkcí typických pro řecké náboženské prostředí, nárůst počtu lokálních kultů, regionální epiteton 11, subregionální epiteton 13, kolektivní identita 38 termínů, celkem 162 výskytů - obyvatelé řeckých obcí z oblasti Thrákie 20, mimo ni 3; kolektivní pojmenování etnik či kmenů (barbaros 3, Thráx 64, Rómaios 11, Kampános 1, Makedón 1), obyvatelé thráckých vesnic 10; celkem 1129 osob na nápisech, 134 nápisů s jednou osobou; max. 154 osob na nápis, aritm. průměr 2,89 osoby na nápis, medián 1; komunita multikulturního charakteru se zastoupením řeckého, římského a thráckého prvku, se nadpoloviční přítomností římského prvku, jména pouze řecká (6,66 \letterpercent{}), pouze thrácká (3,58 \letterpercent{}), pouze římská (27,43 \letterpercent{}), kombinace řeckého a thráckého (2,56 \letterpercent{}), kombinace řeckého a římského (19,23 \letterpercent{}), kombinace thráckého a římského (5,12 \letterpercent{}), kombinovaná řecká, thrácká a římská jména (10,25 \letterpercent{}), jména nejistého původu (8,18 \letterpercent{}), beze jména (16,92 \letterpercent{}){\bf ;} geografická jména z oblasti Thrákie 24, geografická jména mimo Thrákii 19;

\NC\AR
\HL
\HL
\stoptable

Ve 3. st. n. l. je úroveň epigrafické produkce přibližně o polovinu vyšší než v předcházejícím století. I nadále pocházejí tři čtvrtiny nápisů z vnitrozemských oblastí, a to zejména z lokalit nalézajících se podél římské cesty {\em Via Diagonalis,} jako je Augusta Traiana a Filippopolis. Tato cesta procházející ze severovýchodu na jihovýchod Thrákie spojovala Evropu s Malou Asií a sloužila k poměrně častému přesunu vojsk oběma směry (Madzharov 2009, 70-131). Zbývající čtvrtina epigrafické produkce i nadále pochází z tří regionů řeckých měst na pobřeží: z okolí Perinthu a Byzantia na pobřeží Marmarského moře, z okolí Abdéry a Maróneie na pobřeží Egejského moře, a dále z Odéssu a Dionýsopole na pobřeží Černého moře. Konkrétní polohu míst nálezů nápisů ilustruje mapa 6.09 v Apendixu 2.\footnote{Archeologický kontext je podobně jako u předcházejícího období z velké části neznámý, avšak zhruba u 18 \letterpercent{} lokalit je archeologický kontext náboženský a zhruba u 11 \letterpercent{} lokalit sídelní. Oproti předcházejícímu období dochází k mírnému nárůstu u obou kategorií, a téměř každý třetí nápis pochází buď z osídlení, či svatyně. Předpokládá se, že materiál na výrobu nápisů pocházel z místních zdrojů, jako např. z ostrova Prokonnésos v Marmarském moři. Zcela v tomto období chybí nápisy na kovových předmětech.}

Soukromé nápisy představují přes polovinu celého souboru. Podobně jako u nápisů datovaných do 2. až 3. st. n. l. převládají dedikační nápisy nad nápisy funerálními. Celý souboru doplňuje nebývale vysoký počet veřejných nápisů, které představují 45 \letterpercent{} všech nápisů z daného období, a dále 2,5 \letterpercent{} nápisů, jejichž typ nebylo možné přesněji určit.

\subsection[funerální-nápisy-15]{Funerální nápisy}

Funerálních nápisů ze 3. st. n. l. se dochovalo celkem 76, což představuje zhruba nárůst o 40 \letterpercent{} oproti předcházejícímu století. Podobně jako v předcházejícím období si funerální nápisy uchovávají standardní formule, typické pro tento druh nápisů, nicméně jejich celkový počet se snižuje, pravděpodobně jako reakce na proměňující se funerální ritus.\footnote{Např. typická formule na památku ({\em mnémé charin} či {\em mnéiás charin)} ve dochovala ve 21 případech. Místo pohřbu je nejčastěji nazýváno {\em mnémeion} ve třech, {\em larnax} v jednom případě pro sarkofág, {\em latomeion} ve třech případech taktéž pro sarkofág, {\em soros} v sedmi případech pro urnu, {\em chamosorion} pro plochou hrobku umístěnou na zemi. Celkem tři nápisy zmiňují vztyčení stély a ve dvou případech i zhotovení textu nápisu. Celkem čtyři nápisy oslovují okolo jdoucího poutníka ({\em chaire/chairete parodeita}), což je znatelně méně než v předcházejícím období.}

Typický text nápisu nese jméno nebožtíka, jeho zařazení v rámci komunity, jeho původ a jeho dosažené postavení či vykonané skutky. Dále je zde uveden zhotovitel nápisu a jeho vztah k zemřelému. Většinou se jedná o člena rodiny, přítele či kolegu z armády. Důvody pro uvádění pozůstalých jsou pravděpodobně spojeny s dědickými nároky a povinnostmi (MacMullen 1982; Meyer 1990). Geografická jména poukazují na původ nebožtíků či jejich blízkých, jednak z Filippopole a oblasti Haimu, ale i z měst Malé Asie jako je Níkaia či Smyrna, avšak výskyt na pouhých čtyřech nápisech svědčí o tom, že geografický původ nebyl nejdůležitějším faktorem identifikace jednotlivce.\footnote{V jednom případě se setkáváme s označením barbar, které popisuje skupinu lupičů neznámého původu, jimž se podařilo uniknout knězi Aureliovi Flaviovi Markovi na nápise {\em IG Bulg} 1,2 1 z lokality Tvardica v regionu Caron Limen. Termín {\em barbaros} tak nelze spojovat výlučně s thráckým etnikem, a pokud ano, jedná se o etický emotivně zabarvený popis nespecifické etnické skupiny. Etnická příslušnost tak ve 3. st. n. l. nehraje téměř žádnou roli, a pokud ano, pak pouze v samosprávě a rozdělení obyvatel v rámci provincie.} Povolání zemřelého se dochovala na pětině nápisů, což poukazuje na narůstající váhu prezentace povolání a dosaženého společenského postavení.\footnote{Setkáváme se s šesti vojáky různých hodností, dále s čtyřmi gladiátory nejrůznějších specializací, jedním strážným, jedním knězem, jedním propuštěncem, jedním členem {\em búlé} a jedním členem {\em gerúsie}.}

Dle výskytu osobních jmen v průběhu ve 3. st. n. l. dochází k většímu zapojení osob nesoucích thrácká jména do praxe vztyčování funerálních nápisů a jejich podíl je nyní již takřka pětinový.\footnote{Celkem se na nápisech dochovalo 182 jmen, z nichž 43 \letterpercent{} je řeckých, 32 \letterpercent{} římských, 18,5 \letterpercent{} thráckých a 6,5 \letterpercent{} je nejistého původu. Řecká jména, ať už samotná, či v kombinaci, se vyskytovala zejména v okolí řeckých měst na pobřeží Egejského moře, Perinthu a Byzantia. Ve vnitrozemské Thrákii se řecká jména vyskytovala zejména v údolí střední toku Strýmónu. Až polovina osob nesoucí řecké jméno nesla i jméno římské, což značí přijetí římského onomastického systému. Římská jména bez kombinace s thráckým, řeckým či jiným jménem se dochovala pouze na deseti nápisech rovnoměrně rozmístěných na území Thrákie.} Thrácká jména pocházela především z nápisů nalezených v údolí středního toku Strýmónu a částečně v okolí Maróneie a Topeiru. Polovina thráckých jmen se vyskytovala v kombinaci se jménem římským, což taktéž naznačovalo proměnu onomastické tradice směrem ke způsobům praktikovaným v rámci římské říše. Přijetí římského jména a jeho uvedení na nápisech mohlo signalizovat jak společenské postavení dané osoby, tak i jeho nejbližší rodiny, vzhledem k tomu, že až do roku 212 n. l. se udílelo především za individuální zásluhy. V roce 212 n. l. však bylo uděleno římské občanství a právo nosit jméno římského původu všem obyvatelům římské říše, a právo nosit římské jméno tak postupně ztrácelo váhu a společenskou prestiž, kterou mělo před rokem 212 n. l. (Blanco-Perez 2016, 271).

V této době je možné pozorovat větší nárůst geografických jmen, často označujících geografický původ osoby, odkazující zejména na místa v Malé Asii, z toho čtyři v sousední Bíthýnii. Zvýšená přítomnost osob z Malé Asie nasvědčuje migračním proudům z této oblasti, které začaly již koncem 2. st. n. l., ale na nápisech se projevily zejména až ve 3. st. n. l. (Delchev 2013, 18-19; Sharankov 2011, 141, 143). Na nápisech je možné pozorovat i zvýšenou přítomnost osob ze západních římských provincií, která ale nedosahuje úrovně přítomnosti osob z maloasijských regionů.

Převaha textů byla i nadále v řecké jazyce a latinské nápisy se objevovaly především v souvislosti s vojáky či veterány a pocházely z míst s permanentně umístěnou vojenskou posádkou.\footnote{Latinský text se objevil u funerálních nápisů pouze pětkrát, a to u zemřelých vojáků sloužících v římské armádě ve čtyřech případech a u propuštěnce v jednom případě. Čistě latinský text nápisu se objevil třikrát v Byzantiu, a to zejména u vojáků nesoucích čistě jména římského či nethráckého původu. Jeden bilingvní překlad identického řeckého textu se objevil ve Filippopoli, a to u vojáka nesoucí římské a thrácké jméno, který byl thráckého původu, ale kariéru si vybudoval v rámci římské armády. Latinský text byl uvedený na prvním místě, řecký text až jako druhý. Text věnovaný propuštěncem svému pánovi, římskému centurionovi thráckého původu, obsahuje invokační formuli v řečtině a text nápisu je v latině a pochází z Perinthu.} Latina sloužila jako oficiální jazyk římské armády a do jisté míry ovlivnila i podobu funerálních nápisů zemřelých vojáků. Volba jazyka byla dána spíše funkcí nápisu a publikem, jemuž byla určena.\footnote{Pokud byl text určen pouze vojákům nethráckého původu, případně pokud se mělo poukazovat na postavení zemřelého v rámci armády, byla volena spíše latina jako je tomu u nápisů z Byzantia. Pokud však měl být nápis určen jak pro vojenskou, tak nevojenskou komunitu, tak docházelo k prolínání latiny a řečtiny či ke kompletním překladům textů, jako je tomu i nápisů z Perinthu a Filippopole.}

Ve 3. st. n. l. dále narůstá míra standardizace funerálních nápisů a souvisejících procedur. Ve 18 případech se setkáváme s frází na konci nápisu postihujícím případné nové použití hrobky, ale zejména sarkofágu a urny, stejně jako v 1. a 2. st. n. l.\footnote{Nejvíce takovýchto nápisů bylo nalezeno v Byzantiu, celkem šest, dále pak v Maróneii čtyři nápisy, v Perinthu tři nápisy, dva v Topeiru a jeden v Abdéře, Sélymbrii a Filippopoli. Převážně se jednalo o osoby nesoucí původní řecká jména a nově přijaté jméno Aurelios či Aurelia. V jednom případě se jedná o člena {\em gerúsie} z Filippopole, člena {\em búlé} z Perinthu a dceru veterána z Maróneie. Sarkofágy většinou sloužily pro dva a více členů rodiny a byly věnovány partnery či potomky zemřelých.} Z poměrně vysoké frekvence a rozšíření tohoto zvyku můžeme soudit, že se jednalo o poměrně častou praxi, kdy byla místa pohřbu využívána sekundárně a bylo nutné se chránit. Politická autorita města, pod jehož ochranu nápis spadal, tak určitou ochranu pravděpodobně zajišťovala, jinak by nedošlo k tak hojnému rozšíření tohoto zvyku, a tím pádem i rozšíření epigrafické formule do několika produkčních center.

\subsection[dedikační-nápisy-15]{Dedikační nápisy}

Dedikačních nápisů se ze 3. st. n. l. dochovalo 128, což představuje zhruba trojnásobný nárůst oproti 2. st. n. l. Dedikační nápisy i nadále převládají nad nápisy funerálními, podobně jako u skupiny nápisů datovaných do 2. až 3. st. n. l.\footnote{Na konci 3. st. n. l. opět však dochází ke změně a celkový počet funerálních nápisů opět převládá nad dedikacemi.} Pokračujícím fenoménem ve 3. st. n. l. jsou velké svatyně v blízkosti velký měst či v podhorských oblastech v dostupnosti cest, které se staly populární jak mezi Thráky, tak i lidmi nesoucí jiná než thrácká jména.\footnote{Takřka polovina nápisů pochází z jedné svatyně v regionu města Serdica. Dalších 18 nápisů pochází z několika míst z regionu Augusty Traiany, a dalších devět nápisů z údolí středního toku Strýmónu.} Zvyk věnovat nápisy nebyl ve 3. st. n. l. omezen pouze na jednu část společnosti, ale podíleli se na něm lidé jak thráckého, tak jiného původu a různého společenského postavení.

Věnování jsou určena zejména Asklépiovi ve 27 případech, Diovi ve 12 případech, Apollónovi v pěti případech, Áreovi ve dvou případech. Tato božstva nesla většinou místní epiteton, což naznačuje pokračující propojení místních kultů s původně řeckými kulty, stejně jako v 2. st. př. n. l.\footnote{Mezi tato božstva patří např. Zeus {\em Zbelthiúrdos}, Zeus {\em Paisoulénos}, Apollón {\em Dortazénos}, Asklépios {\em Kúlkússénos}, Árés {\em Saprénos}, dále {\em theos} {\em Salénos}, {\em Aularchénos}, {\em hérós} {\em Tisasénos} a {\em Marón}, {\em theos} {\em Asdúlos}.} Ve většině případů se jedná o prostou dedikaci bez většího množství detailů, ale zhruba čtvrtina nápisů poskytuje detailnější informace o dedikantech. Dedikace ve 14 případech zhotovili vojáci, ve třech případech členové {\em búlé}, ve dvou {\em thrakarchové} a v jednom případě {\em gymnasiarchés} a {\em archón} v jedné osobě. V jednom případě se jedná o hromadnou dedikaci {\em saltariů}, tj. římské verze polesných a lidí starajících se o lesní porost (Mihailov 1966, 275). Osobní jména poukazují na převahu původně římských jmen.\footnote{Celkem se dochovalo 185 jmen, z čehož zhruba polovina je římského původu, 23 \letterpercent{} řeckého původu, 18 \letterpercent{} thráckého původu a 10 \letterpercent{} jmen nebylo možné přesněji určit.} Dedikanti thráckého původu se vyskytovali téměř na pětině nápisů, což je zhruba o 8 \letterpercent{} více než ve století předcházejícím.\footnote{V kombinaci s římským jménem se thrácké jméno objevilo na 13 nápisech, v kombinaci s řeckým na devíti nápisech, a samostatně stojících na devíti nápisech.} Nápisy s thráckými jmény pocházely výhradně z vnitrozemí z okolí městských center: 11 z regionu Augusty Traiany, čtyři z regionu města Serdica, čtyři z regionu Filippopole. Celkem v šesti případech se jednalo o věnování vojáka či zastupitele zastávajícího vysokou pozici v administrativním aparátu provincie. Deset dedikačních nápisů s thráckým jménem neslo dekoraci v podobě jezdce na koni, prvkem tradičně spojovaným právě s thráckou populací.

Již na přelomu 2. a 3. st. se objevují nové motivy reliéfní dekorace, jako jsou například výjevy znázorňující Asklépia, Héraklea, Dia a Héru, Dionýsa. Nejčastěji se opakujícím motivem je však jezdec na koni, který v mnoha případech může být spojován s fenoménem tzv. thráckého jezdce (Kazarow 1938; Dimitrova 2002; Oppermann 2006).\footnote{Jezdec na koni se objevil celkem 55krát, z čehož 44 nápisů pochází ze svatyně Asklépia {\em Liménia} ze Slivnice v okolí města Serdica, dalších šest ze svatyně u vesnice Viden v regionu města Augusta Traiana (Boteva 1985; Tabakova-Tsanova 1961).} Celkem 30 nápisů s dekorací jezdce na koni nese osobní jména: deset z nich obsahuje thrácká jména samostatně stojící či v kombinaci, a 20 obsahuje jiná než thrácká jména. Nelze tedy tvrdit, že dedikace s vyobrazením jezdce na koni byla výsadně záležitostí thrácké populace. Naopak, tzv. thrácký jezdec se stal ve 3. st. n. l. populární i mezi lidmi nesoucí řecká a římská jména. Celkem u osmi nápisů o sobě dedikant přímo uvádí, že se jedná o vojáka, což představuje zhruba 15 \letterpercent{} všech dedikací nesoucích motiv jezdce na koni z daného období a nelze tedy s jistotou tvrdit, že fenomén thráckého jezdce byl rozšířen především v komunitě vojáků thráckého původu, nicméně byl mezi vojáky oblíben (Boteva 2005, 204).

Jako ilustrativní příklad složení epigraficky aktivní populace a náboženských zvyklostí 3. st. n. l. mohou sloužit nálezy ze svatyně Asklépia {\em Liménia} ze Slivnice. Tato svatyně patří k nejvýznamnějším svatyním, alespoň co do počtu nalezených nápisů. Bylo zde nalezeno celkem 69 dedikací nesoucích nápis a 286 anepigrafických votivních předmětů (Boteva 1985, 31; Mihailov 1997, 318).\footnote{Celkem 62 z těchto dedikací splňovalo chronologická kritéria a bylo datováno s přesností do jednoho až dvou století.} Celkem 24 nápisů bylo určeno Asklépiovi, z toho 14 s přízviskem {\em Liménios}, ve čtyřech případech označený jako {\em theos}, v sedmi jako {\em kyrios} a ve jednom jako {\em sótér}, zachránce. Velká část dedikací, konkrétně 44, nese vyobrazení jezdce na koni, pouhé tři nápisy nesou reliéfní zobrazení Asklépia, Hygiei a Télesfora. Pokud jde o identitu dedikantů, zdá se, že tato lokalita byla navštěvována obyvatelstvem nesoucím řecká i thrácká jména, ale i římskými občany a vojáky.\footnote{Celkem 29 nápisů nese osobní jména: osm řeckých, čtyři thrácká, 21 římských a 11 bez přesného určení. Římská jména se vyskytovala samostatně na osmi nápisech a v kombinaci na 13 nápisech. V 11 případech se opakuje římské jméno Aurelios v kombinaci s řeckým či thráckým jménem, což značí že dedikanti přijali po roce 212 n. l. toto jméno, co by znak římského občanství. Nápisy věnovali vojáci v šesti případech, což představuje pouze 9,5 \letterpercent{} dedikací nesoucích nápis z této lokality.} Předměty nesoucí nápisy představují pouze pětinu dochovaných votivních předmětů, což značí že zvyk věnovat nápisy nebyl zcela běžnou součástí rituálu, ale spíše ojedinělou záležitostí. Větší zapojení thrácké populace je pravděpodobně důsledkem nárůstu gramotnosti mezi thráckou populací a proměny funerálních zvyklostí v souvislosti s vojenskou či civilní službou Thráků.

\subsection[veřejné-nápisy-15]{Veřejné nápisy}

Oproti 2. st. n. l. je pozorovatelný další nárůst počtu veřejných nápisů zhruba o 50 \letterpercent{} na 176 exemplářů. Veřejné nápisy pocházejí většinou z thráckého vnitrozemí z bezprostředního okolí {\em Via Diagonalis} a městských center ležících na této významné spojnici. Mezi největší producenty veřejných nápisů patří Augusta Traiana s 38 nápisy, Filippopolis s 37 nápisy, Serdica s 27 nápisy, Nicopolis ad Istrum s 22 nápisy, Perinthos (Hérakleia) s 9 nápisy. Naopak veřejné nápisy téměř vymizely z Byzantia, odkud pocházejí pouze dva texty, či z Maróneie, kde byl nalezen jeden veřejný nápis.\footnote{Dekrety jsou stále nejčastějším typem dokumentu s 105 nápisy, z nichž 95 představuje dekrety honorifikační. Dále sem patří sedm seznamů a 51 nápisů jiného typu, z čehož 42 jsou milníky, tři hraniční kameny a jeden nápis dokumentují stavební aktivity.}

Honorifikační nápisy jsou i nadále nejpočetnější skupinou veřejných nápisů a pocházejí především z velkých městských center té doby s existujícími samosprávnými institucemi a velkým počtem obyvatelstva. Většina textů pochází z vnitrozemí: 23 nápisů pochází z Augusty Traiany, 21 nápisů z Nicopolis ad Istrum, 20 z Filippopole, 13 z Perinthu a pět z města Serdica. Nápisy jsou vydávány místní samosprávou, reprezentovanou {\em búlé} a {\em démem}, případně {\em gerúsií}, avšak zcela pod patronátem římského císaře, kterého zastupují místodržící a vysoce postavení úředníci daného města, např. {\em epimelúmenos} a {\em logistés}, lat. {\em curator}, {\em argyrotamiás}, lat. {\em questor}, či {\em thrakarchés} (Mason 1974, 25; 46). Mezi honorovanými jedinci se objevují členové {\em búlé}, {\em gymnasiarchové}, vojáci, kněží, nižší úředníci jako např. {\em grammateus}, či sportovci ({\em agonothétés}). Jak dokazují opakující se formule, na počest významných jedinců vztyčovány stély či jejich sochy byly umístěny na veřejných místech, což poukazuje na důležitou roli, jakou v tehdejší společnosti hrál společenský status (Van Nijf 2015, 233-243). Většina osobních jmen byla římského původu, což může naznačovat, že vysoké funkce zastávali římští občané italického původu, či místní Thrákové zcela upustili od tradice zachovávání thráckých jmen a přijali pouze jména římského původu.\footnote{Pokud by tomu tak bylo, ve většině případů je nedokážeme navzájem odlišit, pokud specificky neudávají geografický svůj původ, což se neděje.}

Významem zůstává obsah honorifikačních nápisů podobný, nicméně každé větší město používá specifické formule a slovní obraty, které se na jiných místech nevyskytují, či se vyskytují v jiné formě.\footnote{Příkladem může být Filippopolis, které je vždy označována jako {\em lamprotaté polis}, tedy nejslavnější z měst, a Perinthos, který v textech nápisů figuruje jako {\em lampra polis} čili jako slavné město.} Z variability formulací je možné soudit, že podoba honorifikačních nápisů nebyla striktně regulována vyšší autoritou, ale že se jednalo vždy o místní interpretaci sdělení. Je pravděpodobné, že se jednotlivé městské samosprávy navzájem inspirovaly, nicméně finální podoba nařízení byla ponechána zcela místním institucím. Jinak tomu však bylo v případě milníků, které vykazují shodný charakter včetně formy a obsahu napříč různými částmi římského impéria. To může poukazovat jednak snahu o zvýšení srozumitelnosti sdělení při přesunech mnohonárodnostního římského vojska, ale i jistý stupeň centralizovanosti byrokratického aparátu, který měl zřizování cest, a tedy i milníků, na starosti.

Dochované milníky dokumentují existenci fungující infrastruktury a stavebních aktivit v průběhu 3. st. n. l. Především koncem 2. st. a začátkem 3. st. n. l. dochází k zintenzivnění aktivit spojených s výstavbou a zejména opravou již existujících cest, které sloužily primárně pro přesuny římské armády, ale i k obchodu a zásobování (Madzharov 2009, 64). Milníky udávaly vzdálenost do nejbližšího města, které zároveň bylo i samosprávné jednotkou, na jehož území se milníky nacházely a které mělo za povinnost se starat o údržbu daného úseku cesty.\footnote{V deseti případech to byla Filippopolis, osmi případech Serdica, v sedmi Augusta Traiana, v pěti Pautália, ve dvou Perinthos (Hérakleia) a Hadrianúpolis, a v jednom Traianúpolis. Udávané vzdálenosti se pohybovaly v rozmezí dvou až 37 mil, tj. zhruba tří až 55 kilometrů.} Většina milníků pochází z trasy {\em Via Diagonalis}, která spolu s {\em Via Egnatia} spojovala evropské provincie s maloasijskými. Tato aktivita započatá za Severovců pokračuje i v průběhu celého 3. st. n. l., jak dokazuje počet dochovaných milníků.\footnote{Z doby severovské dynastie se dochovalo celkem 14 milníků, z období po roce 235 n. l. do konce století dalších 24, a do doby na přelomu 3. a 4. st. n. l. dalších pět.}

O rozsahu centrálně řízených stavebních aktivit a existující infrastruktury vypovídají dochované veřejné nápisy.\footnote{Zmínky na nápisech dokumentují stavbu agory v blíže neznámém městě v údolí řeky Strýmónu, jako je tomu v případě nápisu {\em IG Bulg} 4 2264 z moderního města Sandanski. Další nápis {\em I Aeg Thrace} 433 z Traiánúpole poukazuje na existující praxi vyměřování o vnitřní rozdělení a uspořádání území Traiánúpole na egejském pobřeží v roce 202 n. l. Velmi podobný nápis {\em I Aeg Thrace} 447, avšak hůře dochovaný, pochází i z nedaleké Alexandrúpole. Nápis datovaný taktéž do r. 202 n. l. nasvědčuje, že za Severovců došlo k novému vyměření území a jejich vnitřní struktury u nejméně dvou měst na egejském pobřeží.} O vnějším uspořádání měst a jejich území svědčí hraniční kameny ({\em horoi}), kterých se dochovalo celkem osm, což je největší počet ze všech předcházejících období. Města těmito kameny vymezovala rozsah svého území, a to jednak pro samosprávní účely, výběr daní, ale například i pro jasnější financování infrastrukturních projektů, jako byla např. údržba cest.\footnote{Hraniční kameny se našly na hranicích území dvou měst, jako je to v případě jednoho nápisu z Odéssu a dvou nápisů z Marcianopole, a dále na hranicích samosprávné jednotky na úrovni vesnice ({\em kómé, chórion}) či jiného osídlení neměstského typu jako v případě tří nápisů z Bendipary v blízkosti Filippopole, a jednoho nápisu patřící Eresénským, tedy obyvatelům Eresy(?) v blízkosti Maróneie. Hraniční kameny byly vydány pod hlavičkou tehdejšího císaře či místodržícího úředníka.}

Stavební aktivita se na konci 2. st. n. l. a na počátku 3. st. n. l. nevztahovala pouze na cesty, ale v epigrafických záznamech objevují nově vzniklá {\em emporia}: Discoruraterae, Pizos, dále sem patří i Pirentensium a Pautália (Lozanov 2015, 84-85). Tato {\em emporia} spadala pod samosprávu nejbližšího města, což byla Augusta Traiana, Nicopolis ad Istrum. Tato lokální centra obchodu zajišťovala městským celkům a vojenským jednotkám jednak zemědělské produkty, řemeslné výrobky, ale zároveň byla umístěna na důležitých vojenských, ale i obchodních cestách a umožňovala výměnu zboží mezi městem a venkovem. Nápis {\em IG Bulg} 3,1 1690 o délce 270 řádek zmiňuje {\em emporion} Pizos a podává výčet obyvatel vesnic z okolí, kteří se podíleli na zakládání {\em emporia}.\footnote{Pizos ležel na {\em Via Diagonalis} mezi přepřahacími stanicemi ({\em mutatio}) Ranilum, Arzus a Cillae a v blízkosti opevněného osídlení vojenského charakteru v Carasuře. Odbytiště pro produkty a výrobky z {\em emporia} zajišťovala z velké části římská armáda, a částečně i sama Augusta Traiana. Jedno z privilegií nového {\em emporia} bylo oproštění od dovozních cel do Augusty Traiany, což oproti místním vesnicím poskytovalo {\em emporiu} nespornou ekonomickou výhodu. {\em Emporion} ve 3. st. n. l. tak zároveň usnadňovalo výměnu zboží a zásobování mezi městskými centry a venkovem. Podobně tomu tak bylo i v případě dalších {\em emporií}, z nichž se nám však na nápisech dochovalo výrazně méně detailních informací o jejich fungování.} To vzniklo sestěhováním obyvatel nejméně devíti thráckých vesnic v době vlády Septimia Severa v roce 202 n. l.\footnote{Jména sestěhovaných vesnic jsou typicky thrácká: {\em Skedabria, Stratopara, Krasalopara, Skepte, Gelúpara, Kúrpisos, Bazopara, Strúneilos a Búsipara}. Zajímavým faktem je, že jedna skupina obyvatel měla sestěhování nařízeno od místodržícího provincie a další skupina se sestěhování účastnila dobrovolně (Boyanov 2014, 185).} V textu se vyskytují pouze mužská jména a mělo se jednat o první osadníky nově vzniklého {\em emporia}, které se nacházelo v regionu Augusty Traiany. Jména 154 sestěhovaných obyvatel jsou z více než dvou třetin thrácká, poukazující na thrácký původ jak nositelů samotných, tak jejich rodičů či sourozenců. Na jejich vyšší společenský status v rámci rurální společnosti poukazují použité funkce jako {\em toparchés} či {\em búleutés}.\footnote{Heller (2015, 266) na příkladech honorifikačních nápisů z Malé Asie dokazuje, že {\em búleutés} zaujímal spíše střední až nižší postavení v rámci hierarchie provinciálních úředníků, nicméně toto postavení bylo stále vyšší než u většiny běžné populace, a proto bylo na nápisech vyzdvihováno.}

Proměnu onomastických zvyklostí ve 3. st. n. l. v reakci nové politické uspořádání nejlépe ilustrují poměrně obsáhlé seznamy osob, kde lze velmi dobře sledovat proměňující se přístup k přijímání římských jmen v rámci vývoje širší společensko-politické situace v římské říši. Před rokem 212 n. l. bylo užití jmen jako Flavios, Iulios, Ulpios, Klaudios pouze záležitostí velmi úzké skupiny lidí, a to jak na pobřeží, tak i ve vnitrozemí, což zvyšovalo jejich prestiž (Parissaki 2007, 286). V dřívějších dobách bylo právo nosit římské jméno výsadou vysloužilých vojáků či vysokých úředníků jako odměna za jejich služby, avšak po roce 212 n. l. mohl toto jméno nosit každý svobodný obyvatel římské říše (Beshevliev 1970, 28-32). V roce 212 n. l. císař Caracalla ediktem známým jako {\em Constitutio Antoniniana} udělil římské občanství všem obyvatelům římské říše, a spolu s ním i právo nosit rodové jméno římského císaře (Beshevliev 1970, 31-32). Z této doby se dochovalo sedm nápisů obsahujících 437 osob a 658 osobních jmen\footnote{Nejvíce osob se nachází na již diskutovaném nápise {\em IG Bulg} 3,1 1690 o založení emporia Pizos, a to 154 osob. Dále sem patří tři nápisy se seznamy efébů z Odéssu ({\em IG Bulg} 1,2 47, 47bis, 48) a jeden z Dionýsopole ({\em IG Bulg} 1,2 14), jeden seznam věřících Dionýsova kultu z Cillae ({\em IG Bulg} 3,1 1517) a jeden seznam věřících blíže neznámého kultu z Augusty Traiany ({\em SEG} 58:679).} a seznamy vydané bezprostředně po roce 212 n. l. vykazují výrazně vyšší poměr přijatých římských jmen, především jména Aurelios, než nápisy např. z poloviny století.\footnote{Příkladem z Thrákie jsou nápisy se seznamy efébů {\em IG Bulg} 1,2 14 a 47 datovaných do doby krátce po r. 212 n. l., kde přes 95 \letterpercent{} osob přijalo ke svému jménu císařské jméno Aurelios. Na dalším seznamu efébů {\em IG Bulg} 1,2 47bis z roku 221 n. l. tento poměr klesl na 85 \letterpercent{}, a u nápisu {\em IG Bulg} 3,1 1517 datovaného do let 241-244 n. l. klesl na 80 \letterpercent{}. Na seznamech efébů se většinou vyskytují muži s kombinovanými řeckými a římskými jmény a zcela výjimečně jména thrácká v necelých 2,4 \letterpercent{}.} Tento trend se ve velké míře objevuje i v dalších částech římské říše, jak dokazuje nedávná studie z Malé Asie (Blanco-Perez 2016, 279). Jedno z možných vysvětlení může naznačovat, že s přibývajícím časovým odstupem od hromadného udělení římského občanství společenská hodnota a prestiž v rámci komunity klesala, a tudíž se jméno Aurelios, a vše co toto jméno představovalo, na nápisech objevovalo méně často.

Spolu s měnícími se onomastickými zvyklostmi je ve 3. st. n. l. taktéž pozorovatelná větší provázanost identity jednotlivce s institucemi a společenskou organizací římské říše, a to jak ve veřejných, tak na soukromých nápisech. Politická identita, respektive přináležitost do samosprávní jednotky na úrovni města, případně lokální samosprávy na úrovni vesnice, se stává důležitou součástí veřejných nápisů.\footnote{Výskyt kolektivního vyjádření identity s příslušností k městu se vyskytuje na soukromých i veřejných nápisech ze 3. st. n. l. celkem 55krát, s příslušností k vesnici sedmkrát.} Tato politická identita zcela nahrazuje etnickou příslušnost a její vyjádření na nápisech.\footnote{Kolektivní termíny Thrákové a Thrákie jsou zmiňovány na 66 soukromých a veřejných nápisech pouze v souvislostí s obyvateli provincie {\em Thracia}, či jako obecné pojmenování typu gladiátora. Zmínka o konkrétních thráckých kmenech je omezena pouze na Serdicu, o níž se na 17 nápisech hovoří jako o městu kmene Serdů ({\em hé Serdón polis}).}

\subsection[shrnutí-19]{Shrnutí}

Dochované materiály ze 3. st. n. l. poukazují na výsadní roli římské administrativy a vojenské organizace, což se projevovalo i v rámci epigrafické produkce. Epigrafické záznamy dokládají nárůst stavebních aktivit a aktivit spojených s udržováním již existující infrastruktury, jako vojenských cest či táborů, ale i intenzivní rozvoj městských center, zejména ze začátku 3. st. n. l. Narůstají komplexita společnosti se projevuje i ve zvyšujícím se množství specializovaných funkcí, které se na nápisech objevují. Většina z nich je do větší či menší míry spojena s administrativním řízením provincie či s vojenskou službou.

Složení epigraficky aktivní společnosti je podobné jako ve 2. st. př. n. l., ale dochází k nárůstu výskytu osob thráckého původu na nápisech, zejména na nápisech soukromé povahy. Thrákové již zcela běžně slouží v římské armádě, kde zastávají nižší a střední posty. V rámci civilní samosprávy se podílejí na vedení provincie, a to dokonce i v roli vyšších úředníků. Přítomnost Thráků je ve zvýšené míře zaznamenána u nápisů pocházejících z lokálních svatyní ve vnitrozemí, nicméně i v těchto kultech nepřesahuje zastoupení osob s thráckými jmény jednu pětinu dochovaných jmen. Z toho plyne, že i místní kulty byly otevřeny i osobám jiného původu a nejednalo se o kulty přístupné výlučně Thrákům.

Ve 3. st. n. l. podobně jako v předcházejícím století spíše, než původ hraje roli status a postavení v rámci komunity: velký důraz je kladen na dosažené postavení, zastávané funkce, získané pozice v armádě, tak i na afiliaci s římskou říší v podobě proměněných onomastických zvyků, které poukazují na římské občanství a dosažený status nositele.

\section[charakteristika-epigrafické-produkce-ve-3.-st.-n.-l.-až-4.-st.-n.-l.]{Charakteristika epigrafické produkce ve 3. st. n. l. až 4. st. n. l.}

V epigrafické produkci dochází na přelomu 3. a 4. st. n. l. k velkému propadu. Zcela dochází k vymizení thráckých osobních jmen a místních kultů z dochovaných nápisů. Komunity se opět uzavírají a pokles epigrafické produkce je velmi patrný na všech místech. Veřejné nápisy v této době poprvé převažují nad nápisy soukromými, z nichž polovinu tvoří milníky, které se nacházely podél významných silnic.

\placetable[none]{}
\starttable[|l|]
\HL
\NC {\em Celkem:} 24 nápisů

{\em Region měst na pobřeží:} Anchialos 1, Maróneia 7, Perinthos (Hérakleia) 8 (celkem 16 nápisů)

{\em Region měst ve vnitrozemí:} Augusta Traiana 1, Filippopolis 1, Serdica 2, Traianúpolis 1, údolí středního toku řeky Strýmón 1 (celkem 6 nápisů)\footnote{Celkem dva nápisy nebyly nalezeny v rámci regionu známých měst, editoři korpusů udávají jejich polohu vzhledem k nejbližšímu modernímu sídlišti.}

{\em Celkový počet individuálních lokalit}: 12

{\em Archeologický kontext nálezu:} sídelní 3 (z toho obchodní 1), sekundární 3, neznámý 19

{\em Materiál:} kámen 24 (mramor 14; vápenec 3, jiný 2; neznámý 5)

{\em Dochování nosiče}: 100 \letterpercent{} 3, 75 \letterpercent{} 4, 50 \letterpercent{} 2, 25 \letterpercent{} 4, kresba 1, ztracený 1, nemožno určit 9

{\em Objekt:} stéla 11, architektonický prvek 11, jiný 1, neznámý 1

{\em Dekorace:} reliéf 7, bez dekorace 17; reliéfní dekorace figurální 0 nápisů, architektonické prvky 7 nápisů (vyskytující se motiv: sloup 6, jiný 1)

{\em Typologie nápisu:} soukromé 11, veřejné 12, neurčitelné 1

{\em Soukromé nápisy:} funerální 9, dedikační 2, vlastnictví 1, jiný 1\footnote{Několik nápisů mělo vzhledem ke své nejednoznačnosti kombinovanou funkci, proto je součet nápisů obou typů vyšší než celkový počet soukromých nápisů.}

{\em Veřejné nápisy:} honorifikační dekrety 4, jiný 7 (z toho milník 6)

{\em Délka:} aritm. průměr 9,2 řádku, medián 10, max. délka 21, min. délka 1

{\em Obsah:} latinský text 5 nápisů, písmo římského typu 3; hledané termíny (administrativní termíny 5 - celkem 16 výskytů, epigrafické formule 9 - 21 výskytů, honorifikační 1 - 2 výskyty, náboženské 2 - 2 výskyty, epiteton 24 - počet výskytů 55)

{\em Identita:} řecká božstva 1, egyptská božstva 0, římská božstva 0, křesťanství 0, prudký pokles náboženské terminologie, včetně vymizení lokálních kultů z nápisů, regionální epiteton 0, subregionální epiteton 0{\bf ,} kolektivní identita 3 termíny, celkem 9 výskytů {\bf -} obyvatelé řeckých obcí z oblasti Thrákie 1, mimo ni 0; kolektivní pojmenování etnik či kmenů (Thráx 2), obyvatelé thráckých vesnic 1; celkem 50 osob na nápisech, 5 nápisů s jednou osobou; max. 6 osob na nápis, aritm. průměr 2,08 osoby na nápis, medián 2; komunita se zastoupením řeckého a římského, thrácký prvek zcela chybí, jména pouze řecká (4 \letterpercent{}), pouze thrácká (0 \letterpercent{}), pouze římská (52 \letterpercent{}), kombinace řeckého a thráckého (0 \letterpercent{}), kombinace řeckého a římského (12 \letterpercent{}), kombinace thráckého a římského (0 \letterpercent{}), kombinovaná řecká, thrácká a římská jména (4 \letterpercent{}), jména nejistého původu (4 \letterpercent{}), beze jména (24 \letterpercent{}); geografická jména nejistého původu 1;

\NC\AR
\HL
\HL
\stoptable

U skupiny nápisů datovaných do 3. a 4. st. n. l. je pozorovatelný prudký pokles epigrafické produkce až o 86 \letterpercent{} oproti skupině nápisů datovaných do 2. až 3. st. n. l. Tento jev odpovídá dění i jiných částech římské říše, kdy po prudkém nárůstu koncem 2. st. a na začátku 3. st. dochází ve druhé polovině 3. a na začátku 4. st. n. l. k výraznému snížení počtu dochovaných nápisů, a tedy i k úpadku epigrafické produkce (MacMullen 1982, 245; Meyer 1990, 82-94). Produkční centra se v malé míře uchovávají v pobřežních oblastech, zejména v Maróneii a Perinthu, jak je patrné na mapě 6.09 v Apendixu 2.

\subsection[funerální-nápisy-16]{Funerální nápisy}

Funerálních nápisů se dochovalo celkem devět, což představuje takřka šestinásobný propad v produkci oproti nápisům datovaným do 2. až 3. st. n. l. Celkem čtyři nápisy pocházejí z Maróneie, dva z Perinthu. V jednom případě se jedná o sarkofág, jehož neoprávněné použití bylo chráněno institucemi v Maróneii, podobně jako u stejných nápisů z 1. až 3. st. n. l.\footnote{Nápis {\em I Aeg Thrace} 312.} Většina dochovaných osobních jmen je řeckého a římského původu, pouze v údolí středního toku Strýmónu se dochovalo šest osob se jmény pravděpodobně thráckého původu.\footnote{Nápis Manov 2008 168.} Úbytek počtu nápisů je jedinou zásadní změnu, celkový charakter funerálních nápisů zůstává stejný jako v předcházejícím období, nakolik je možné usuzovat z omezeného vzorku nápisů.

\subsection[dedikační-nápisy-16]{Dedikační nápisy}

Dedikační nápis se dochoval pouze jeden, což představuje velký propad oproti předcházejícímu období. {\em IG Bulg} 3,2 1835 je psán řecky a latinsky a je věnován císaři Diokletiánovi, Maximiánovi, Konstantinu Chloru a Galeriovi vojákem nesoucí jména Aurelios Ioulianos.

\subsection[veřejné-nápisy-16]{Veřejné nápisy}

Nápisů datovaných do 3. až 4. st. n. l. se dochovalo dohromady 12, což představuje téměř 50 \letterpercent{} všech nápisů z dané skupiny. Celkem se jedná o čtyři honorifikační dekrety, čtyři milníky, tři hraniční kameny a jeden nápis s věnováním císařům. Všechny nápisy pochází z období tetrarchie a jsou úzce spojeny s osobnostmi císařů.

Všechny honorifikační dekrety pocházejí z Hérakleie (původního Perinthu), která se v této době hrála důležitou roli regionálního centrum, než se jím v roce 330 n. l. stala Konstantinopol, bývalé Byzantion. Všechny čtyři nápisy jsou věnované obyvateli Hérakleie tehdejším císařům Diokletiánovi, Maximiánovi, Konstantinu Chloru a Galeriovi. Stejně tak i dochované tři hraniční kameny jsou věnovány tetrarchům, kteří tak vyměření hranic území daného sídla propůjčují patřičnou legitimitu. \footnote{Ve všech případech se jedná o hraniční kámen místní thrácké vesnice, neznámé z dalších historických zdrojů. {\em I Aeg Thrace} 398 jedná se o vyznačení hranic osídlení ({\em chórióma}) Barilos (?). {\em I Aeg Thrace} 382 vyznačení hranic vesnice Eresén{[}-{]}. {\em I Aeg Thrace} 383 vyznačení hranic neznámé vesnice.} Dochované čtyři milníky pocházejí z okolí {\em Via Diagonalis} (Serdica, Augusta Traiana a Perinthos, tehdy již jako Hérakleia) a jsou datované na přelom 2. a 3. st. n. l., kdy docházelo k stavebním aktivitám a úpravám této významné silnice organizovaným tehdejšími císaři na začátku 4. st. n. l.

\subsection[shrnutí-20]{Shrnutí}

Na přelomu 3. a 4. st. n. l. dochází k útlumu epigrafických aktivit a výraznému poklesu dochovaných nápisů. To může souviset s celospolečenskou krizí římské říše a ekonomickým úpadkem, způsobeným dlouholetými vojenskými spory pocházejícími jak z nitra říše samotné, tak i nájezdy nepřátel na severních hranicích. S úpadkem institucí tak, jak je známe dosud a omezením jejich činnosti dochází i k vymizení veřejných nápisů a přeměně soukromých nápisů, jak je patrné dále ve 4. st. n. l.

\section[charakteristika-epigrafické-produkce-ve-4.-st.-n.-l.]{Charakteristika epigrafické produkce ve 4. st. n. l.}

Nápisy 4. st. n. l. následují trend prudkého poklesu produkce, který začal již na přelomu 3. a 4. st. n. l. Dochází k uzavírání komunit, snižuje se i variabilita forem i obsahu nápisů. Spolu s tím se objevují i jasné projevy křesťanské tematiky, a to zejména v oblasti Perinthu, Abdéry a Byzantia. Opět převládá funerální funkce nápisů, jejich obsah je však podstatně odlišný.

\placetable[none]{}
\starttable[|l|]
\HL
\NC {\em Celkem:} 23 nápisů

{\em Region měst na pobřeží:} Abdéra 2, Bizóné 1, Byzantion 5, Kallipolis 1, Mesámbria 2, Perinthos (Hérakleia) 7 (celkem 18 nápisů)

{\em Region měst ve vnitrozemí:} Augusta Traiana 3, Nicopolis ad Istrum 1, Traianúpolis 1, (celkem 5 nápisů)

{\em Celkový počet individuálních lokalit}: 11

{\em Archeologický kontext nálezu:} funerální 2, sídelní 4, náboženský 1, sekundární 2, neznámý 14

{\em Materiál:} kámen 22 (mramor 12; vápenec 3, neznámý 7), jiný 1

{\em Dochování nosiče}: 100 \letterpercent{} 3, 50 \letterpercent{} 3, 25 \letterpercent{} 3, nemožno určit 15

{\em Objekt:} stéla 17, architektonický prvek 4, nástěnná malba 1, jiný 1

{\em Dekorace:} reliéf 6, malovaná 1, bez dekorace 16; reliéfní dekorace figurální 1 nápis (vyskytující se motiv: zvíře 1), architektonické prvky 7 nápisů (vyskytující se motiv: naiskos 2, sloup 1, báze sloupu či oltář 1, geometrický motiv 1, florální motiv 1, jiný 2)

{\em Typologie nápisu:} soukromé 14, veřejné 6, neurčitelné 3

{\em Soukromé nápisy:} funerální 10, dedikační 1, jiný 3 (z toho popis sochy 2)

{\em Veřejné nápisy:} honorifikační dekrety 2, jiný 4 (z toho milník 2)

{\em Délka:} aritm. průměr 8,34 řádku, medián 6, max. délka 33, min. délka 1

{\em Obsah:} latinský text 2 nápisy, písmo římského typu 4; hledané termíny (administrativní termíny 13 - celkem 18 výskytů, epigrafické formule 11 - 22 výskytů, honorifikační 0 - 0 výskytů, náboženské 5 - 5 výskytů, epiteton 1 - počet výskytů 1)

{\em Identita:} řecká božstva 1, egyptská božstva 0, římská božstva 0, křesťanská tematika 9, pokles náboženské terminologie, včetně vymizení lokálních kultů z nápisů, regionální epiteton 1, subregionální epiteton 0, kolektivní identita 3 termíny, celkem 3 výskyty - obyvatelé řeckých obcí z oblasti Thrákie 3, mimo ni 0; celkem 34 osob na nápisech, 11 nápisů s jednou osobou; max. 8 osob na nápis, aritm. průměr 1,48 osoby na nápis, medián 1; komunita řeckého a římského charakteru, thrácký prvek zcela chybí, jména pouze řecká (26,08 \letterpercent{}), pouze thrácká (0 \letterpercent{}), pouze římská (4,34 \letterpercent{}), kombinace řeckého a thráckého (0 \letterpercent{}), kombinace řeckého a římského (39,13 \letterpercent{}), kombinace thráckého a římského (0 \letterpercent{}), kombinovaná řecká, thrácká a římská jména (0 \letterpercent{}), jména nejistého původu (4,34 \letterpercent{}), beze jména (26,08 \letterpercent{}); geografická jména z oblasti Thrákie 3, mimo Thrákii 0;

\NC\AR
\HL
\HL
\stoptable

Epigrafická produkce ve 4. st. n. l. podstupuje velké změny. V první řadě se jedná o celkový prudký úpadek produkce o 94 \letterpercent{} ze 390 na 23 nápisů. Příčiny tohoto procesu je možné sledovat již v druhé polovině 3. st. n. l. Tehdejší společensko-politická krize vyústila v nestabilitu říše, způsobenou jak vnitřními rozbroji, ekonomickou krizí, ale i hrozícím nebezpečím za hranicemi římského impéria (Lozanov 2015, 87-88). Tyto jevy se samozřejmě podepsaly i na úpadku epigrafické produkce, což vyústilo v poměrně radikální proměnu jejího charakteru. Většina nápisů pochází z pobřežních oblastí, s největší produkcí v Hérakleii, býv. Perinthu, jak je patrné na mapě 6.10 v Apendixu 2.

Nejčastější formou objektu nesoucí nápis je již tradičně mramorová stéla, na níž začínají převládat křesťanské motivy, jako je kříž či christogram.

\subsection[funerální-nápisy-17]{Funerální nápisy}

Zvyk zhotovovat funerální nápisy se tak ve 4. st. n. l. uchoval v Thrákii pouze v rámci křesťanské komunity, převážně z okolo Hérakleie. Funerálních nápisů se celkem dochovalo 10 a přestavují tak nejčetnější skupinu nápisů ze 4. st. n. l. Devět nápisů nese jasné znaky křesťanské víry nebožtíků či pozůstalých, ať už je to slovní vyjádření, zobrazení christogramu či vyobrazení křížů.\footnote{Např. na nápise {\em Perinthos-Herakleia} 219 se setkáváme s frází „Χρειστιανοὶ δὲ πάντες ἔνεσμεν”, tedy vědomé přihlášení se všech zmíněných osob ke křesťanské víře.} Fakt, že většina nápisů nese textové či vizuální konotace na křesťanskou víru, není nijak překvapivý, vzhledem k tomu, že hlavních produkční centrum Hérakleia, odkud pochází šest nápisů, byla zároveň jedním ze hlavních sídel křesťanské komunity pro region jihovýchodní Thrákie a křesťanství se v průběhu 4. st. n. l. stalo uznávaným náboženstvím a dokonce oficiální vírou římské říše (Dumanov 2015, 92-96). Osoby na nápisech i nadále nesou jména vzniklá kombinací římského a řeckého jména, a i nadále se v mírně pozměněné formě udržují kulturní zvyklosti předcházejících období, jako je např. fráze o ochraně hrobu či texty nápisů promlouvající ke kolemjdoucím.\footnote{Na nápise {\em Perinthos-Herakleia} 177 vystupuje Aurelios Afrodeiseios spolu s manželkou Aurelií Deionoisií. V nápisu je také uvedeno, že kdokoliv se odváží či pokusí neoprávněně použít hrobku, musí zaplatit pokutu právoplatným dědicům. Celkem se tato fráze objevuje na čtyřech sarkofázích z Hérakleie. Na závěr nápisu je taktéž ve čtyřech případech uvedena tradiční formule {\em chaire parodeita}, avšak s mírně pozměněnou ortografií, která pravděpodobně reflektovala tehdejší výslovnost.}

\subsection[dedikační-nápisy-17]{Dedikační nápisy}

Dedikační nápisy ve 4. st. n. l. téměř vymizely, což může souviset i s ústupem místních kultů a nástupem křesťanské víry. Ze 4. st. n. l. se dochoval pouze jeden dedikační nápis {\em I Aeg Thrace} 19 z Abdéry, který věnovala Sab(b)ais Diovi {\em Hypsistovi}, tedy nejvyššímu bohu.

\subsection[veřejné-nápisy-17]{Veřejné nápisy}

Celkem se dochovalo šest veřejných nápisů, z nichž dva jsou milníky\footnote{{\em Perinthos-Herakleia} 292 z Hérakleie a Velkov 2005 58 z Mesámbrie.}, dva stavební nápisy, dokumentující stavbu brány v Byzantiu (Konstantinopoli) a stavbu věže v Kaliakře\footnote{{\em SEG} 53:651, {\em IG Bulg} 1, 2 12bis.}, a dva honorifikační nápisy věnované tehdejším císařům vysokými městskými úředníky.\footnote{{\em SEG} 51:916 datovaný na poč. 4. st. n. l., a {\em SEG} 52:695 datovaný mezi roky 324-337 n. l., oba z Augusty Traiany.} Nápisy pocházejí z doby tetrarchie, zejména z vlády císaře Konstantina. Ač je jejich celkový počet velmi omezený, i přesto se jedná o tradiční projevy suverenity římské říše a císaře. Primární funkcí těchto nápisů bylo informovat o stavebních aktivitách státního aparátu, ale i šířit pověst a postavení římského císaře v rámci provincie. Malý počet veřejných nápisů může dosvědčovat, že státní aparát byl v této době v krizi, a tudíž i celková produkce nápisů byla velmi nízká, a omezené pouze do první poloviny 4. st. n. l.\footnote{V polovině 4. st. n. l. došlo k dokončení administrativních reforem, které začaly již za císaře Diokleciána: území Thrákie bylo přeměněno na diecéze {\em Thracia} a {\em Dacia}, z nichž Thracia se dále dělila na šest provincií se svými hlavními městy, která se staly administrativními centry nejbližších regionů (Dumanov 2015, 91).} Státní instituce do jisté míry i nadále fungovaly, což dokumentují dekrety pocházející z Augusty Traiany či milníky z Hérakleie (dříve Perinthu), obecně však produkce nápisů sloužících veřejnému zájmu končí zhruba po polovině 4. st. n. l. O této doby se vyskytují nápisy pouze pro soukromou potřebu jednotlivců či uzavřených komunit.

\subsection[shrnutí-21]{Shrnutí}

Dochované nápisy ze 4. st. n. l. nasvědčují, že epigrafická produkce byla zhruba od konce 3. st. n. l. na ústupu. Došlo však nezbytně k přeměně formy a obsahu publikovaných nápisů tak, aby více odpovídala tehdejším společenským potřebám. Křesťanská víra se stala jednotícím prvkem nápisů produkovaných v Thrákii během 4. a následně i během 4. až 5. st. n. l. Nápisy téměř zcela vymizely z veřejné sféry, kde došlo k poměrně radikálním přeměnám administrativního uspořádání, což mělo za následek změny i v epigrafické produkci. V soukromé sféře se publikace nápisů udržela pouze v rámci křesťanské komunity.

\section[charakteristika-epigrafické-produkce-ve-4.-st.-n.-l.-až-5.-st.-n.-l.]{Charakteristika epigrafické produkce ve 4. st. n. l. až 5. st. n. l.}

Nápisy datované do 4. až 5. st. n. l. pocházejí převážně z velkých měst na pobřeží. Pokračuje úbytek jak celkového počtu nápisů, tak i epigraficky aktivních komunit. Komunity jsou spíše uzavřené, nedochází ke kulturní výměně na úrovni předcházejících století. Zcela převládají soukromé nápisy funerální funkce, na nichž je patrný silný vliv křesťanství, který ovlivňuje jak podobu, tak obsah nápisů. Dochází k opětovné individualizaci nápisů, ač rozsah nápisů se nijak nemění.

\placetable[none]{}
\starttable[|l|]
\HL
\NC {\em Celkem:} 35 nápisů

{\em Region měst na pobřeží:} Byzantion 4, Maróneia 13, Perinthos (Hérakleia) 13 (celkem 30 nápisů)

{\em Region měst ve vnitrozemí:} Augusta Traiana 2, Serdica 2, Traianúpolis 1 (celkem 5 nápisů)

{\em Celkový počet individuálních lokalit}: 8

{\em Archeologický kontext nálezu:} funerální 2, sídelní 2, náboženský 1, sekundární 6, neznámý 24

{\em Materiál:} kámen 32 (mramor 23; jiný 1, neznámý 8), kov 2 (olovo 2), jiný 1

{\em Dochování nosiče}: 100 \letterpercent{} 4, 75 \letterpercent{} 3, 50 \letterpercent{} 3, 25 \letterpercent{} 10, oklepek 1, nemožno určit 14

{\em Objekt:} stéla 32, architektonický prvek 1, mozaika 1, jiný 1

{\em Dekorace:} reliéf 18, jiná 1, bez dekorace 16; reliéfní dekorace figurální 2 nápisy (vyskytující se motiv: zvíře 1, scéna lovu 1, skupina lidí 1, stojící osoba 1, socha 1), architektonické prvky 17 nápisů (vyskytující se motiv: naiskos 9, florální motiv 1, jiný 9 (kříž 9)

{\em Typologie nápisu:} soukromé 30, veřejné 1, neurčitelné 4

{\em Soukromé nápisy:} funerální 28, jiný 2 (proklínací destička 2)

{\em Veřejné nápisy:} jiný 1 (z toho hraniční kámen 1)

{\em Délka:} aritm. průměr 6,67 řádku, medián 6, max. délka 22, min. délka 1

{\em Obsah:} latinský text 1 nápis, písmo římského typu 6; hledané termíny (administrativní termíny 8 - celkem 12 výskytů, epigrafické formule 7 - 23 výskytů, honorifikační 0 - 0 výskytů, náboženské 0 - 0 výskytů, epiteton 0 - počet výskytů 0)

{\em Identita:} vymizení náboženské terminologie, včetně vymizení lokálních kultů z nápisů, regionální epiteton 0, subregionální epiteton 0, kolektivní identita 0 termínů; celkem 28 osob na nápisech, 15 nápisů s jednou osobou; max. 5 osob na nápis, aritm. průměr 0,8 osoby na nápis, medián 1; komunita řeckého a římského charakteru, thrácký prvek zastoupen minimálně, jména pouze řecká (17,14 \letterpercent{}), pouze thrácká (0 \letterpercent{}), pouze římská (8,57 \letterpercent{}), kombinace řeckého a thráckého (2,85 \letterpercent{}), kombinace řeckého a římského (14,28 \letterpercent{}), kombinace thráckého a římského (0 \letterpercent{}), kombinovaná řecká, thrácká a římská jména (0 \letterpercent{}), jména nejistého původu (14,27 \letterpercent{}), beze jména (42,85 \letterpercent{}); geografická jména z oblasti Thrákie 0, mimo Thrákii 0;

\NC\AR
\HL
\HL
\stoptable

Celková produkce je ve 4. až 5. st. n. l. zhruba desetinová při porovnání s dobou největší epigrafické aktivity, tedy 2. a 3. st. n. l. Epigrafická produkce se do jisté míry vrátila na úroveň 1. či 2. st. př. n. l., kdy většina nápisů byla produkována v pobřežních oblastech a ve vnitrozemí se nápisy objevily pouze sporadicky. Většina z 35 nápisů z daného období pochází z přímořských oblastí, a to zejména z regionu Maróneie a Hérakleie, jak je patrné na mapě 6.10 v Apendixu 2. Nápisy ve vnitrozemí pocházejí převážně z urbánních center, jako je Serdica či Augusta Traiana, kde dochází k velmi omezenému přežívání epigrafické kultury.

Nejčastější formou objektu nesoucí nápis je i nadále mramorová stéla, z jiných materiálů se dochovaly dvě proklínací destičky psané na olovo, jeden nápis na mozaice a jeden nápis na nástěnné malbě uvnitř hrobky.

\subsection[funerální-nápisy-18]{Funerální nápisy}

Funerálních nápisů se celkem dochovalo 28 a i nadále pocházejí převážně z křesťanské komunity, jak je patrné z obsahu i dekorace náhrobků samotných.\footnote{Místo spočinutí označují slova jako {\em mnémeion} či {\em mnéma} označující jak hrob, tak stélu samotnou. Dále se začíná objevovat termín {\em thesis}, typický právě pro křesťanské nápisy, označující místo posledního spočinutí. Termín {\em latomeion} se vyskytuje celkem pětkrát a označuje sarkofág, který nese nápis a zároveň slouží jako místo posledního odpočinku. Sarkofágy pocházejí převážně z křesťanské komunity z Hérakleie a sloužily pro rodinné pohřby, v jednom případě až pro šest lidí. Texty na sarkofágu typicky ztotožňovaly nebožtíka jako řádného obyvatele Hérakleie. V šesti případech nápisy obsahovaly formuli, které zakazovala jejich další používání pod pokutou, vymahatelnou v rámci samosprávy či církve právě v Hérakleii. Podobné nápisy existovaly v několika městech již od 1. st. n. l. Podle množství dochovaných nápisů s identickým, či velmi podobným textem se dá usuzovat, že se jednalo o poměrně častý problém, který ve 4. st. n. l. přetrval zejména v Hérakleii, např. na nápise {\em Perinthos-Herakleia} 180.} I nadále si uchovaly poměrně informativní a interaktivní charakter: v osmi případech nápisy promlouvají k náhodně procházejícímu poutníkovi ({\em chaire parodeita}) a sdělují mu životní osudy zemřelého. Zemřelý je identifikován pomocí osobního jména a jména rodiče, případně jeho přináležitost ke křesťanské komunitě. Poměrně dlouhý rozsah nápisů poskytuje řadu informací nejen o nebožtíkovi, ale i o jeho rodině.\footnote{Ve dvou případech se jednalo o vojáky z povolání, konkrétně o legionáře a {\em centenaria}, v jednom případě o lékaře, v jednom případě o námezdního dělníka, dále o architekta, stříbrotepce, mincovního mistra a výběrčího daní. Dozvídáme se o manželkách zesnulých, jejich potomcích a někdy i o věku, jehož se dožili. Výjimku tvoří nápis {\em SEG} 49:871 nalezený na nástěnné malbě uvnitř hrobky číslo 252 v regionu města Augusta Traiana s typickou formulí přející štěstí ({\em agathé týché}).}

Identita osob byla jasně dána křesťanskou vírou, a nebylo tedy nutné uvádět geografický původ. K prolínání onomastických tradic téměř již nedocházelo, převaha osobních jmen byla řeckého původu, s malým podílem jen římského původu. Thrácký prvek se až na jednu výjimku z dochovaného epigrafického materiálu zcela vytratil.\footnote{Nápis {\em Perinthos-Herakleia} 183 patřil stříbrotepci s původně thráckým jménem, Mókiánem z Hérakleie, který přijal křesťanskou víru, a text nápisu obsahoval stejnou právní formulku poskytující ochranu před novým použitím sarkofágu, která se vyskytovala v oblasti již od 1. st. n. l.} Tímto zásadním způsobem proměnilo křesťanství podobu a obsah epigrafické produkce v Thrákii, ale i mimo ni. Přestala být oceňována identita politická či vojenská kariéra a společenská prestiž v rámci státního aparátu, ale namísto nich na důležité místo ve společnosti nastoupila sounáležitost s křesťanskou obcí.

\subsection[dedikační-nápisy-18]{Dedikační nápisy}

Ve své tradiční podobě se dedikační nápis nedochoval ani jeden, nicméně následující čtyři nápisy je možné zahrnout do široce pojaté kategorie dedikací: dva nápisy na podstavcích soch, jejichž text se s největší pravděpodobností vztahoval k nedochovaným sochám\footnote{Nápisy {\em SEG} 46:845 a 845,2 zmiňují bohyni Hekaté a pravděpodobně řečníka Aischína, a byly objeveny v rámci archeologických vykopávek na Istanbulském {\em hippodromu}.}, a dále dvě proklínací tabulky.\footnote{{\em SEG} 60:747 a 748, které jsou datované do 4. až 5. st. n. l. Jejich text se sestává z magických formulí a relativně běžného palindromu {\em ablanathanalba} (Gager 1999, 136). Účelem těchto proklínacích destiček bylo získat sílu, lásku, zdraví, peníze či moc, či naopak jejich magickou mocí uškodit nepříteli. Jejich text je často nesrozumitelný a obsahuje magické formule, které měly přimět nadpřirozenou sílu vykonat přání pisatele. Jejich výskyt je vcelku běžný v průběhu celé antiky, jejich dochování z oblasti Thrákie je však poměrně vzácné. Tyto dva texty ({\em SEG} 60:747 a 748) byly nalezeny v průběhu archeologických vykopávek v roce 2010 v moderním Istanbulu, a je možné, že v budoucnosti bude objeveno více podobných nálezů.}

\subsection[veřejné-nápisy-18]{Veřejné nápisy}

Veřejný nápis ze 4. až 5. st. n. l. se dochoval pouze jeden, a to velmi poškozený hraniční kámen {\em I Aeg Thrace} 343 z Maróneie.

\subsection[shrnutí-22]{Shrnutí}

Skupina nápisů datovaných do 4. až 5. st. n. l. vykazuje stejné charakteristiky jako nápisy datované do 4. st. n. l., poukazující na poměrně značné změny ve společenském uspořádání. Tyto změny měly za následek nejen téměř úplné vymizení epigrafické produkce ve službách politické autority, ale i ze soukromého sektoru. Proměna kulturních a náboženských projevů společnosti je dobře patrná na pozměněném obsahu a formě nápisů, kde hlavní roli přejímá křesťanská víra a thrácký element se opět vytrácí.

\section[charakteristika-epigrafické-produkce-v-5.-st.-n.-l.]{Charakteristika epigrafické produkce v 5. st. n. l.}

V 5. st. n. l. pokračuje útlum epigrafických aktiv, komunity se ještě více uzavírají. Produkce do omezené míry přežívá na pobřeží a ve vnitrozemských městech, nicméně se snižuje variabilita formy a obsahu nápisů: soukromé funerální nápisy jsou takřka jediným typem nápisů, převládá jednoduchá dekorace s křesťanskou tematikou. Vliv křesťanství narůstá, jak v dekoraci, tak i funkci nápisů, podobně jako v předcházejícím období.

\placetable[none]{}
\starttable[|l|]
\HL
\NC {\em Celkem:} 12 nápisů

{\em Region měst na pobřeží:} Byzantion 3, Maróneia 4, Perinthos (Hérakleia) 1 (celkem 8 nápisů)

{\em Region měst ve vnitrozemí:} Filippopolis 2, Plótinúpolis 1, Traianúpolis 1 (celkem 4 nápisy)

{\em Celkový počet individuálních lokalit}: 6

{\em Archeologický kontext nálezu:} funerální 3, náboženský 2, sekundární 2, neznámý 5

{\em Materiál:} kámen 12 (mramor 7; vápenec 2, neznámý 3)

{\em Dochování nosiče}: 100 \letterpercent{} 2, 75 \letterpercent{} 1, 50 \letterpercent{} 1, 25 \letterpercent{} 2, nemožno určit 6

{\em Objekt:} stéla 10, architektonický prvek 1, mozaika 1

{\em Dekorace:} reliéf 5, bez dekorace 7; reliéfní dekorace figurální 0, architektonické prvky 5 nápisů (vyskytující se motiv: naiskos 1, sloup 1, jiný 4 (kříž 4))

{\em Typologie nápisu:} soukromé 11, neurčitelné 1

{\em Soukromé nápisy:} funerální 10, neznámý 1

{\em Veřejné nápisy:} 0

{\em Délka:} aritm. průměr 3,91 řádku, medián 3, max. délka 14, min. délka 1

{\em Obsah:} latinský text 0, písmo římského typu 0; hledané termíny (administrativní 1 - celkem 1 výskyt, epigrafické formule 3 - 3 výskyty, honorifikační 0, náboženské 0, epiteton 0)

{\em Identita:} narůstající vliv křesťanství, regionální epiteton 0, subregionální epiteton 0, kolektivní identita 0 termínů; celkem 12 osob na nápisech, 10 nápisů s jednou osobou; max. 1 osoba na nápis, aritm. průměr 0,8 osoby na nápis, medián 1; komunita řeckého a omezeně římského charakteru, jména pouze řecká (41,67 \letterpercent{}), pouze thrácká (0 \letterpercent{}), pouze římská (0 \letterpercent{}), kombinace řeckého a thráckého (0 \letterpercent{}), kombinace řeckého a římského (8,33 \letterpercent{}), kombinace thráckého a římského (0 \letterpercent{}), kombinovaná řecká, thrácká a římská jména (0 \letterpercent{}), jména nejistého původu (33,2 \letterpercent{}), beze jména (16,6 \letterpercent{}); geografická jména z oblasti Thrákie 0, mimo Thrákii 0;

\NC\AR
\HL
\HL
\stoptable

Celková epigrafická produkce v 5. st. n. l. poklesla o 52 \letterpercent{} oproti 4. st. př. n. l. a o 93 \letterpercent{} oproti 3. st. n. l. Jedná se o pokračující trend poklesu epigrafické produkce, který začal již na konci 3. st. n. l., a který je pozorovatelný i v dalších částech bývalé římské říše. Nápisy pocházejí především z Maróneie a Byzantia (Konstantinopole), jak je patrné na mapě 6.11 v Apendixu 2.

Nápisy z 5. st. n. l. jsou převážně vyrobeny z kamene, jen jeden nápis se zachoval na dřevě. Jejich charakter je soukromý a slouží výhradně funerální funkci, celkem v 11 případech. Velký vliv na formu i obsah nápisu má křesťanství, které se stalo výhradním tématem spojujícím dochované nápisy. Časté je vyobrazení kříže či christogramu, se zmínkami o křesťanském bohu, případně Kristovi. Funerální nápisy se stávají individuálním monumentem, zmiňujícím pouze zemřelého, a nikoliv jeho rodinu. Jména na nápisech jsou převážně řeckého původu, ale do určité míry dochází k uchování římské onomastické tradice.\footnote{Např. na nápise {\em SEG} 56:823 vystupuje muž jménem Flavios Eutychés.} Thrácká jména, stejně tak jako kulty thráckého původu zcela chybí. Nevyskytují se ani zmínky o kontaktech s jinými komunitami, ať už z regionu či mimo region a nedochází k epigraficky zaznamenané imigraci lidí, tak jak bylo běžné např. ve 2. a 3. st. př. n. l.

\subsection[shrnutí-23]{Shrnutí}

V 5. st. n. l. se komunity uzavírají a opouštějí dříve nastolenou společenskou organizaci a zvyky, mezi něž patřil například i zvyk publikovat nápisy. Zcela dochází k vymizení institucionální epigrafické produkce. Vzniklé křesťanské nápisy jsou poslední přežívající tradicí, amalgámem pohanského a křesťanského pohřebního ritu, který se udržel v místech se silnou historickou tradicí publikace nápisů, a na mnoha jiných místech zanikl.

\section[charakteristika-epigrafické-produkce-v-5.-st.-n.-l.-až-6.-st.-n.-l.]{Charakteristika epigrafické produkce v 5. st. n. l. až 6. st. n. l.}

U nápisů datovaných do 5. až 6. st. n. l. je i nadále možné sledovat celkový pokles jejich produkce. Většina z nich pochází z pobřežních oblastí a má soukromý charakter. Obsah nápisů úzce souvisí s projevy křesťanské víry a funerální funkcí nápisů. Epigraficky aktivní komunity jsou spíše uzavřeného charakteru, epigraficky nejsou zaznamenány kontakty dokonce ani s nejbližším okolím.

\placetable[none]{}
\starttable[|l|]
\HL
\NC {\em Celkem:} 16 nápisů

{\em Region měst na pobřeží:} Apollónia Pontská 1, Byzantion 3, Maróneia 3, Mesámbria 3, Topeiros 1 (celkem 11 nápisů)

{\em Region měst ve vnitrozemí:} Nicopolis ad Istrum 2 (celkem 2 nápisy)\footnote{Celkem tři nápisy nebyly nalezeny v rámci regionu známých měst, editoři korpusů udávají jejich polohu vzhledem k nejbližšímu modernímu sídlišti, či uvádí muzeum, v němž se nachází.}

{\em Celkový počet individuálních lokalit}: 8

{\em Archeologický kontext nálezu:} sekundární 3, neznámý 13

{\em Materiál:} kámen 15 (mramor 13; jiný 2), jiný 1 (z toho dřevo 1)

{\em Dochování nosiče}: 50 \letterpercent{} 5, 25 \letterpercent{} 3, kresba 1, nemožno určit 7

{\em Objekt:} stéla 11, architektonický prvek 2, jiný 1, neznámý 2

{\em Dekorace:} reliéf 5, bez dekorace 11; reliéfní dekorace figurální 1 nápis (vyskytující se motiv: funerální scéna/symposion 1), architektonické prvky 4 nápisy (vyskytující se motiv: kříž 7, christogram 1)

{\em Typologie nápisu:} soukromé 13, veřejné 0, neurčitelné 3

{\em Soukromé nápisy:} funerální 10, dedikační 1, jiný 1, neznámý 1

{\em Veřejné nápisy:} 0

{\em Délka:} aritm. průměr 4,75 řádku, medián 5, max. délka 9, min. délka 1

{\em Obsah:} latinský text 0, písmo řím. typu 1; hledané termíny (administrativní termíny 0, epigrafické formule 1 - 1 výskyt, honorifikační 0, náboženské 2 - 3 výskyty, epiteton 0)

{\em Identita:} křesťanská náboženská terminologie, vymizení lokálních kultů z nápisů, regionální epiteton 0, subregionální epiteton 0, kolektivní identita 0 termínů; celkem 8 osob na nápisech, 8 nápisů s jednou osobou; max. 1 osoba na nápis, aritm. průměr 0,5 osoby na nápis, medián 0,5; uzavřené komunity, bez prolínání onomastických tradic, jména pouze řecká (25 \letterpercent{}), pouze thrácká (0 \letterpercent{}), pouze římská (12,5 \letterpercent{}), kombinace řeckého a thráckého (0 \letterpercent{}), kombinace řeckého a římského (0 \letterpercent{}), kombinace thráckého a římského (0 \letterpercent{}), kombinovaná řecká, thrácká a římská jména (0 \letterpercent{}), jména nejistého původu (12,5 \letterpercent{}), beze jména (50 \letterpercent{}); geografická jména z oblasti Thrákie 1, mimo Thrákii 0;

\NC\AR
\HL
\HL
\stoptable

Většina nápisů z 5. až 6 st. n. l. pochází z pobřežních oblastí. Epigraficky aktivní komunity se udržují pouze na několika místech ve třech regionech na mořském pobřeží, zejména v okolí Byzantia (Konstantinopole), Perinthu (Hérakleie), Maróneie a Mesámbrie, podobně jako v předcházejícím období, což dobře ilustruje Mapa 6.11 v Apendixu 2.

Nápisy mají výhradně soukromý charakter, z nichž 10 nápisů mělo funerální funkci a pocházelo z křesťanské komunity, a jeden sloužil jako dedikace věnovaná křesťanskému Bohu.\footnote{Nápis Velkov 2005 54 z Mesámbrie.} Nápisy jsou většinou velmi špatně dochované, nicméně z jejich výzdoby a fragmentárního textu lze soudit, že pocházejí výhradně z křesťanského kontextu, z komunit v Maróneii, Byzantiu, Mesámbrii a Apollónii. Dochovaná jména jsou převážně řeckého původu a setkáváme se i s uváděním funkcí, které zemřelý zastával v rámci křesťanské komunity.\footnote{Např. nápis {\em I Aeg Thrace} 96 nebo {\em SEG} 60:735.} Zcela naopak chybí nápisy veřejné povahy, a dále nápisy v nichž by figurovalo obyvatelstvo nesoucí thrácká jména.

\subsection[shrnutí-24]{Shrnutí}

Nápisy datované do 5. až 6. st. n. l. poukazují i nadále na zásadní vliv křesťanství na společnost tehdejší Thrákie, která se projevila i na podobě epigrafické produkce. Dochované funerální nápisy pocházejí především z oblastí se silnou křesťanskou komunitou. Velmi malý počet dochovaných nápisů nasvědčuje, že zvyk publikovat nápisy do jisté míry přežíval v komunitách, které přeměnily podobu funerálních nápisů tak, aby jejich forma a obsah vyhovovaly potřebám křesťanské víry.

\section[nápisy-z-6.-až-8.-st.-n.-l.]{Nápisy z 6. až 8. st. n. l.}

Vytvořená HAT databáze obsahuje celkem 12 nápisů ze 6. st. n. l., které odpovídají kritériím datace, tedy nápisů s koeficientem 1. Dále databáze obsahuje i dva nápisy ze 7. až 8. st. n. l., tedy nápisy s koeficientem 0,5. Tato skupina nápisů vykazuje stejné rysy jako nápisy z 5. a 5. až 6. st. n. l., a proto je zmíním alespoň ve stručnosti.

Nápisy pocházejí převážně pobřežních oblastí, pouze dva nápisy pocházejí z vnitrozemí. Produkční centra jsou prakticky stejná jako v 5. st. př. n. l., jak je patrné z Mapy 6.11 v Apendixu 2. Objekty nesoucí nápis jsou převážně z mramoru s vyobrazením křesťanského kříže a křesťanská tematika se odráží i v textu nápisů, které jsou převážně funerálního zaměření. Zbylé čtyři nápisy označují výrobce keramických cihel, nesoucí vyobrazení kříže a pořečtěnou verzi termínu {\em koubikoularios}, tedy eunuch sloužící v předpokojích římského a později byzantského císaře. Složení epigraficky aktivní společnosti je dle původu osobních jmen z třetiny řecké, s odkazy na místní komunity jako je Filippopolis či na obyvatele řeckých měst z Malé Asie z oblasti Galatie a Ankýry. Tato skupina nápisů tedy nijak nevybočuje z trendu epigrafické produkce nastavené již na konci 4. st. n. l., a zejména pak v průběhu 5. st. n. l., kde se křesťanství, a zejména jeho projevy v rámci funerálního ritu, staly jediným přeživším motivem epigrafické produkce v Thrákii.

\section[epigrafická-produkce-a-proměny-společnosti]{Epigrafická produkce a proměny společnosti}

Dochovaný epigrafický materiál, který bylo možné datovat s přesností do jednoho až dvou století, poskytuje poměrně ucelený soubor dat o vývoji společnosti antické Thrákie v průběhu více než 11 století, přičemž ale z některých staletí máme k dispozice více dochovaného materiálu. Soubor analyzovaných 2036 nápisů umožňuje sledovat vývoj epigrafické produkce v jednotlivých časových obdobích a navzájem je srovnávat se známými společensko-politickými změnami.

Podíváme-li se na celkový počet nápisů, jak ho zobrazuje graf 6.01 v Apendixu 2, skupiny nápisů datovaných s přesností do jednoho a do dvou století vykazují velmi podobné trendy. K výraznému nárůstu produkce dochází v 5. a 4. st. př. n. l. Další období růstu počtu nápisů jsou ve 2. st. n. l. a pak zejména v 3. st. n. l. Ze srovnání se známými historickými jevy a událostmi tak plyne, že období růstu epigrafické produkce mohou být spojována s dobou intenzivního působení řeckých obcí na území Thrákie v klasické době, kdy došlo k rozšíření epigrafického zvyku i mimo bezprostřední území řecký {\em poleis}, a kdy se začalo ve velmi omezené míře na nápisech objevovat i thrácké obyvatelstvo, zejména aristokratického původu. Další období růstu souvisí s dobou relativní politické stability řeckých obcí ve 2. st. př. n. l., kdy byla velká část Thrákie součástí makedonského království. Nejvýraznější nárůst epigrafické produkce nastává ve 2. a 3. st. n. l., kdy dochází k velkému množství administrativních reforem v rámci římské říše a s tím spojeným větším zapojení běžného obyvatelstva do státních a vojenských institucí.

Naopak období poklesu epigrafické produkce oproti předcházejícímu století nastalo na přelomu 4. a 3. st. př. n.l., výrazněji pak v 1. st. př. n. l. a nejstrmější propad pochází z přelomu 3. a 4. st. n. l. Doby poklesu epigrafické aktivity mohou být přičítány období ekonomické a společenské krize, vyčerpání a následné transformace, která vede i k odlišnému pojetí epigrafické produkce. K poklesu produkce dochází zejména na přelomu 1. st. př. n. l. a 1. st. n. l., kdy Thrákie prochází reformou politické moci a vazalské krále thráckého původu střídá autorita římské říše. Tento propad epigrafické produkce nicméně není tak markantní jako téměř ukončení epigrafické produkce na konci 3. st. př. n. l., kdy dochází k poklesu produkce o více než 90 \letterpercent{}, což souvisí s celkovou destabilizací poměrů v římské říši, vleklou ekonomickou krizí, zvyšujícími se nájezdy nepřátelských kmenů a v neposlední řadě i narůstající společenskou rolí křesťanské víry.

Opakující se trendy růstu v době prosperity a stability a poklesu v době nejistoty a úpadku se projevují i napříč typologickými skupinami nápisů. Změny vnitřní infrastruktury se projeví se zpožděním určité doby na celkovém počtu veřejných nápisů, které jsou produktem organizované administrativy a aktivity fungující politické autority. Jejich nárůst je patrný ve 3. st. př. n. l. jakožto důsledek zvýšených aktivit makedonských králů, avšak i přesto stále převažují nápisy soukromé povahy. Další nárůst produkce veřejných nápisů je pozorovatelný ve 2. st. n. l., kdy dochází k nárůstu regulace v souvislosti s administrací římské provincie a stavebních aktivit financovaných státním aparátem. Tento růst dále pokračuje i v 3. st. n. l., kdy se římská říše dlouhodobě potýká s krizí, nárůstem byrokratického aparátu, zvyšování moci armády a centralizací řízení rozsáhlé říše. Rostoucí komplexita politické a společenské organizace vede k nutnosti efektivněji uchovávat a předávat informace, nutné k řízení stratifikované společnosti o velikosti větší než několik set členů. Proto v komplexních společnostech dochází k vytvoření systému uchovávání a předávání informací, jehož částečným projevem jsou i dochované veřejné nápisy (Johnson 1973, 3-4). V 2. a 1. st. př. n. l. je naopak možné pozorovat snížení celkového počtu nápisů, které je nejmarkantnější ve 4. st. n. l., kdy dochází k destabilizaci politické autority, která využívala nápisy pro účely organizace a správy území a obyvatelstva. Probíhající reformy organizace říše, ekonomická a politická nestabilita vedou ve 4. st. n. l. ke konečnému úpadku byrokratického aparátu v jeho původní podobě, a tedy i k razantnímu opuštění epigrafické produkce.

Nápisy soukromé povahy tvořily většinu nápisů po celou sledovanou dobu a do jisté míry u nich probíhaly podobné změny růstu a poklesu produkce jako u veřejných nápisů, jak je patrné z grafu 6.02 v Apendixu 2. Tento jev souvisí s propojeností základní infrastruktury nutné k vytvoření nápisů, kterou sdílí jak nápisy veřejné, tak i soukromé povahy. V dobách prosperity, kdy dochází k nárůstu společenské komplexity, současně se totiž objevuje i infrastruktura nutná ke zhotovení nápisů, jakou je profesní specializace, intenzifikace výroby, navýšení gramotnosti obyvatelstva a akumulace ekonomického potenciálu osobami podílejícími se na produkci nápisů. Pokud jsou tyto podmínky naplněny, tím pádem může snadněji dojít k většímu zapojení obyvatelstva na produkci nápisů, a tedy i nárůstu produkce soukromých nápisů.

Prudké nárůsty produkce soukromých nápisů se objevují v 5. st. př. n. l., dále ve 2. st. n. l., tedy v době prosperity a stability, kdy soukromé osoby mají dostatek prostředků na zhotovení nápisu. Naopak ve 3. st. př. n. l., 1. st. př. n. l. a zejména ve 4. st. n. l. dochází k úpadku publikační činnosti soukromých osob. Tento jev je možné spojovat s obdobím společenské a ekonomické nejistoty, kdy lidé upouští od publikování nápisů a dávají přednost zajištění primárních životních potřeb. V těchto dobách také může klesat počet obyvatel schopných psát, spolu s tím, jak v dobách krize mizí i nutná infrastruktura potřebná pro vznik gramotné vrstvy obyvatel, případně pro vytváření nápisů samých (Tainter 1988, 118-123, 137).

Zajímavým jevem je současný nárůst veřejných nápisů a současný pokles nápisů funerálních, ke kterému došlo ve 3. st. př. n. l. a také ve 3. st. n. l., jak je patrné v grafu 6.02 v Apendixu 2. Možné vysvětlení tohoto trendu je krize tehdejšího politického a společenského uspořádání a snaha politické autority o reformy, navýšení regulace a kontroly byrokratického aparátu. Přímým důsledkem je zvýšené finanční zatížení středních vrstev společnosti, což v konečném důsledku vede k relativně okamžitému poklesu produkce soukromých nápisů. Ve 3. st. př. n. l. tento jev může souviset se snahou původně autonomních řeckých měst se vyrovnat s působením makedonských králů v Thrákii, které je možné datovat již od poloviny 4. st. n. l. s výraznějšími projevy zejména ve století následujícím. Ve 3. st. n. l. je to naopak reakce na krizi římské říše, způsobenou jak vojenskými převraty a konflikty, nebezpečím na vnějších hranicích říše, finanční zátěží a přebujelým administrativním aparátem (Tainter 1988, 137-140). Administrativní aparát se za každou cenu snaží udržet v chodu i za cenu zvyšujících se nákladů, které dopadají především na střední vrstvu. To může vést až ke snižování vzdělanosti a schopnosti obyvatelstva publikovat a financovat zhotovování nápisů, což má za následek úbytek soukromých nápisů ve 3. st. n. l. O století později tento trend vyústí v konečný úpadek epigrafické produkce ve veřejné sféře a přeměnu publikačních zvyků v soukromé sféře, vedoucí k přežívání epigrafické funerální produkce v křesťanských komunitách.

Trendy v chování vyšší a střední vrstvy, tedy těch, kteří si mohli dovolit publikovat nápisy, dobře dokumentují trendy zobrazené v grafu 6.03 v Apendixu 2. Po většinu doby převládaly funerální nápisy, jejichž funkce byla označovat hrob a připomínat zemřelého. Jejich celkové počty reagovaly na aktuální změny ve společnosti, tj. pokles v době krize a nárůst v době stability. Změna nastává na konci 1. st. n. l., kdy se Thrákie stává součástí římské říše a objevuje se prudký nárůst dedikačních nápisů, doprovázený pomalým poklesem funerálních nápisů. Ve 3. st. n. l. jsou dedikační nápisy dokonce četnější než nápisy funerální, což svědčí o proměnách chování tehdejší populace. S tím souvisí i nárůst epigrafické produkce související s lokálními kulty, které vykazují prvky jak místní thrácké víry, tak řeckého náboženství. Došlo tak k unikátnímu spojení a přeměně zvyků tehdejší společnosti, kdy se funkce převážné části epigrafické produkce změnila z připomínání zemřelého směrem k vyjádření náboženského přesvědčení. Ve 4. st. n. l. dochází k všeobecnému úpadku a navrácení se k funerální funkci dochovaných nápisů.

\subsection[epigrafická-produkce-v-jednotlivých-obdobích]{Epigrafická produkce v jednotlivých obdobích}

V zásadě je možné epigrafickou produkci rozdělit podle míry celkové epigrafické produkce a charakteru materiálu do několika období, v nichž dochází k zásadnějším proměnám společnosti, způsobených jak změnami vnitřního uspořádání společnosti, tak i ovlivněním nově příchozími kulturami.

Období 6. a 5. st. př. n. l. je charakterizováno soustředěním epigrafické produkce v řeckých městech na pobřeží, a pouze sporadickým užitím písma v thráckém aristokratickém kontextu ve vnitrozemí. Použití nápisů v thráckém kontextu je odlišné od použití v řecky mluvících komunitách na pobřeží a nic nenasvědčuje přenosu kulturních zvyklostí v počátečních staletích. Ke kontaktu dochází v omezené míře na diplomatické úrovni v řadách aristokratů a dále na úrovni obyvatelstva žijícího v bezprostřední blízkosti řeckých měst, avšak ani zde nedochází k významnějšímu přenosu kultury a zvyklostí. Výskyt thráckých osobních jmen v řeckém kontextu poukazuje na existenci smíšených svazků, ale vzhledem k tomu, že se nejedná se více než o 3 \letterpercent{} z celkového počtu dochovaných osobních jmen, pak i smíšené svazky byly záležitostí spíše výjimečnou. Tuto dobu je možné ohraničit objevením prvních epigrafických památek v 6. st. př. n. l. a polovinou 4. st. př. n. l., kdy dochází k nárůstu moci thrácké aristokracie spolu s výraznějším vstupem Makedonie na scénu.

Ve 4. st. př. n. l. pokračuje epigrafická produkce v řeckých městech na pobřeží beze změny, pouze se rozšiřuje do více míst a zvyšuje se i celková produkce. Převahu mají nápisy soukromé povahy, ale objevují se i veřejné nápisy, jejichž charakter je do velké míry ovlivněn tradicí a normou obvyklou pro daný typ nápisu. Epigrafická produkce se v omezené míře objevuje i v nově vzniklých smíšených makedonsko-thráckých osídleních ve vnitrozemí, nicméně ani zde nedochází k prolínání onomastických zvyklostí či náboženských představ a obě komunity si udržují spíše tradiční charakter, soudě dle epigrafické produkce. Podobný trend nárůstu celkové epigrafické produkce pokračuje i ve 3. a 2. st. př. n. l., kdy však dochází k poklesu produkce soukromých nápisů v řeckých městech na pobřeží ve 3. st. př. n. l., ale zároveň dochází k nárůstu produkce veřejných nápisů, což může svědčit o nárůstu institucionální regulace. Pobřežní komunity se taktéž v této době více otevírají multinárodnostnímu hellénistickému společenství, což vede k objevení neřeckých náboženství a osob s neřeckými jmény. Zároveň se v řeckých městech objevují i božstva thráckého původu, avšak podíl thráckých jmen se stále pohybuje pod úrovní 3 \letterpercent{} obyvatelstva. Thrácká aristokracie se na epigrafické produkci podílí zcela minimálně, a to zejména v okolí smíšených (řecko-)makedonsko-thráckých osídlení.

Skupina nápisů z 1. st. př. n. l. a 1. st. n. l. poukazuje na úpadek epigrafické produkce jako důsledek společensko-politické krize a nestálosti poměrů v oblasti. V této době začíná do politické situace silně zasahovat Řím, ať už přímo, či pomocí nepřímé intervence. Od poloviny 1. st. n. l. je oblast oficiálně připojena pod římskou říši jako provincie {\em Thracia} a {\em Moesia Inferior}, avšak nedochází k významnému kulturnímu předělu či k zásadním změnám v epigrafické produkci.

Významnější změny naopak přináší 2. st. n. l., kdy dochází k prudkému nárůstu jak soukromé, tak veřejné epigrafické produkce. Reformy administrativního uspořádání provincie, nárůst byrokratické zátěže a centrální organizace má za následek zintenzivnění epigrafické produkce, objevení se nových institucí a specializovaných profesí. Zapojení thráckého obyvatelstva v armádě a městské samosprávě s sebou nese i nárůst gramotnosti, povědomosti o epigrafických zvyklostech, a tedy i nárůst produkce soukromých nápisů.

Nápisy z 2., ale i z 3. st. n. l. nabízejí větší různorodost obsahu nápisů, která je odrazem smíšeného složení epigraficky aktivní společnosti. Zhruba jedna pětina epigraficky aktivní populace jsou osoby nesoucí thrácké jméno, kteří však v mnoha případech přijali i jména římská pravděpodobně jako důkaz dosaženého společenského postavení a vykonaných skutků. Spolu s rozšířením epigrafické produkce mezi thrácké obyvatelstvo se objevují ve větší míře i thrácké kulty, které se smísily s řeckými, ale i dalšími středozemními kulty. Přítomnost lidí z jiných částí římské říše poukazuje na zvýšený pohyb obyvatelstva, a to zejména z Malé Asie. Dedikační nápisy se ve 2. a 3. st. n. l. stanou vůbec nejhojnější skupinou nápisů, což svědčí o zařazení epigrafické produkce k náboženským zvyklostem. Dochází i k rozšíření nových zvyklostí, jako je udávání věku na funerálních nápisech či pravidelné uvádění skutků a společenského postavení, stejně tak jako nejbližší rodiny a přátel, kteří si tak pravděpodobně zajišťovali dědická práva. V závislosti na společenském uspořádání římské říše se proměňují i onomastické zvyklosti obyvatel, ač tyto trendy byly omezeně pozorovatelné už i v druhé polovině 1. st. n. l. Uplatňuje se systém tří jmen, z nichž alespoň jedno jméno má římský původ a poukazuje tak na společenské postavení svého nositele. Tento trend je však narušen po roce 212 n. l., kdy právo nosit římské jméno má každý obyvatel římské říše a v průběhu času tudíž ztrácí na původní prestiži.

Zároveň s narůstající variabilitou obsahu se však ve 2. a 3. st. n. l. projevuje i relativní ustálení formy, a to jak ve vnější podobě epigrafických monumentů, tak i v případě opakujících se formulí a ustálených slovních spojení. V případě veřejných nápisů je vidět jasně centralizovaná role římské administrativy, která do jisté míry předepisuje výslednou podobu a obsah nápisů, avšak za ponechání prostoru pro vytvoření lokálních variant místními samosprávami. V případě soukromých nápisů se spíše než o vliv předepsaných norem, jedná o věc osobního vkusu kombinovanou s nápodobou již existujících zvyklostí, která nepřímo vytváří unifikovanou podobu soukromých nápisů.

Následné 4. a 5. st. n. l. jsou znamením úpadku epigrafické produkce a přeměny struktury společnosti. Reformy organizace společnosti vedou k jiným formám uplatňování politické autority, v nichž vydávání nápisů nehraje žádnou roli a od tohoto zvyku se upouští. Nárůst významu křesťanství v životě soukromých osob je možné pozorovat téměř na všech dochovaných nápisech soukromého charakteru. Z nápisů zcela mizí náboženství jiného typu než křesťanství a stírají se jak etnické rozdíly, tak i zaniká původní hierarchie společnosti a vytváří se nová struktura, provázaná úzce s křesťanskou církví. V následujících stoletích je možné spatřovat přežívající pozůstatky epigrafické produkce, ale ve velmi omezeném množství a pouze v soukromé sféře.

\chapter{Mapování kulturních změn}
V této kapitole se zabývám zeměpisným uspořádáním nálezových míst jednotlivých nápisů na území Thrákie. Zvláštní pozornost věnuji produkčním centrům, tedy místům se zvýšeným výskytem nápisů, a vzájemným vztahům, které tato centra mezi sebou měla. Sleduji povahu epigrafické produkce se zaměřením na veřejné a soukromé nápisy, jejich rozmístění v krajině a vztah s nejbližšími sídly, existující infrastrukturou a zeměpisnými podmínkami, ve snaze dokázat souvislost mezi rozšířením epigrafické produkce a rostoucí společenskou komplexitou a mírou politické organizace.

\section[prostorová-analýza-epigrafické-produkce]{Prostorová analýza epigrafické produkce}

Prostorovou analýzu vytvářím na základě souboru 4453 nápisů, což představuje 95 \letterpercent{} všech nápisů obsažených v databázi {\em Hellenization of Ancient Thrace}. Tento soubor nápisů byl zvolen na základě dostupných informací o jejich místě nálezu, respektive o přesné lokalizaci místa nálezu uváděné v epigrafických korpusech a původních zdrojích. Tuto informaci o přesnosti místa nálezu v databázi zastupuje tzv. {\em position certainty index}, s hodnotami přesného určení místa nálezu do 1 km s 1540 nápisy, do 5 km s 2323 nápisy a do 20 km s 590 nápisů. Celkem 212 nápisů s přesností určení nad 20 km do současné analýzy nezahrnuji, vzhledem k nízké výpovědní hodnotě o místě vzniku i místě nálezu.\footnote{Pro podrobnosti o zvolené metodě výběru nápisů a zaznamenávání přesnosti místa nálezu více v kapitole 4 věnované metodologii práce.}

\subsection[epigrafická-produkční-centra-1]{Epigrafická produkční centra}

Oblast Thrákie se počty dochovaných nápisů řadí mezi střední producenty, a to zejména od 1. st. n. l., kdy se stala součástí římské říše. Srovnáme-li území, které Thrákie zhruba zaujímala a celkový počet dochovaných nápisů dostaneme se na hodnoty epigrafické produkce srovnatelné s římskou provincií {\em Gallia} či {\em Germania} (Woolf 1998, 81-82).

Bereme-li průměrnou velikost území Thrákie v době římské zhruba 130,000 km\high{2}, a celkový počet dochovaných nápisů je 4665, hustota nápisů činí 3,58 nápisu na 100 km\high{2}. Toto číslo se samozřejmě liší v čase. V době římské sledujeme zhruba 2,6x vyšší hustotu nápisů než v předcházejících obdobích. Skupina nápisů od 6. do 1. st. př. n. l. na stejně velkém území zaujímá hustotu zhruba 0,75 nápisu na 100 km\high{2}. V případě nápisů od 1. do 5. st. n. l. se jedná o hustotu 1,98 nápisů na 100 km\high{2}, nedatované nápisy mají hodnotu 0,85 nápisu na 100 km\high{2}. Tato čísla poukazují na výrazný nárůst epigrafické produkce v době, kdy se Thrákie stala součástí římské říše. V žádném případě ale Thrákie nedosahovala úrovní provincií na Apeninském poloostrově, kde byla hustota nápisů 13 na 100 km\high{2} a v případě {\em Latia} i 55 nápisů na 100 km\high{2}. Thrákie v době římské se tak řadí spíše na úroveň provincie {\em Gallia Comata, Belgica} či {\em Germania Inferior} s hustotou dva nápisy na 100 km\high{2} (Woolf 1998, 81-83).

Nárůst epigrafické produkce však nebyl na všech místech rovnoměrný, ale je možné sledovat shlukování nápisů v okolí měst, pohřebišť, svatyní a dále v okolí komunikací. Vzhledem k tomu, že nápisy sloužily zejména pro místní trh a nestaly se komoditou dálkového obchodu ve větším měřítku, se dá předpokládat, že produkční centra se nacházela nedaleko od místa nálezu nápisu, zpravidla v řádu několika kilometrů. Na základě srovnání archeologických dat Bekker-Nielsen (1989, 30-32) došel k závěru, že vzdálenosti mezi centry a jejich ekonomicky a společensky závislými regiony se pohybují od 10 do 37 km dle zastávané funkce. Dle Bekker-Nielsena je průměrnou vzdáleností od města k hranicím závislého regionu 37 km, což průměrná délka maximálního denního pochodu v římské době, která odpovídala 25 římským mílím. Délka denního pochodu se stala se jednou ze základních délkových jednotek udávaných např. při přesunech římského vojska (Madzharov 2009, 51). Z těchto čísel odvozuje délku půl-denního pochodu, tedy vzdálenost 18,5 km, která by umožňovala obyvatelům žijícím v okolí centra cestu do města a návrat domů v témže dni. Vzdálenost 10 km pak používá jako oblast ze které se obyvatelé mohli uchylovat pod ochranu centra v případě nebezpečí, a to v řádu dvou hodin chůze či jedné hodiny jízdy.\footnote{Částečně vycházím i údajů projektu Orbis vytvořeného na Stanford University, \useURL[url22][http://orbis.stanford.edu/][][{\em http://orbis.stanford.edu/}]\from[url22]. Tento projekt modeluje a mapuje rychlost přepravy v římském světě na základě geografických dat a známých historických a literárních zdrojů, nejčastěji tzv. itinerářů, o způsobech a rychlostech přepravy v římské světě. Autoři projektu zahrnují různé způsoby přepravy a k nim doplňují i průměrnou vzdálenost jako bylo možné tímto druhem přepravy urazit. Pro pěší pochod počítají 30 km za den, pro přepravu pomocí vozu taženého oslem 12 km za den, pro přepravu s větším vozem 36 km za den, pro jízdu na koni 56 km za den, zrychlenou jízdu na koni až 250 km za den, na lodi apod. Autoři berou v úvahu i zrychlený přesun vojenských jednotek, kterému stanovili průměrnou hodnotu 60 km za den (Scheidel {\em et al.} 2012, \useURL[url23][http://orbis.stanford.edu/orbis2012/ORBIS_v1paper_20120501.pdf][][{\em http://orbis.stanford.edu/orbis2012/ORBIS_v1paper_20120501.pdf}]\from[url23]). Pro vypočtení konkrétní trasy je možné zadat celou řadu parametrů, které zohledňují jak roční a denní dobu, rychlost přepravy, způsob přepravy, zda se jednalo o vojenskou či soukromou přepravu. Na základě těchto parametrů pak Orbis spočítá nejen dobu trvání cesty, ale i její finanční nákladnost v římských denárech. Příkladem může být např. cesta z Byzantia do Filippopole, která v červenci konaná pěšky s rychlostí 30 km za den trvá 11,7 dní. Pokud použijeme rychlost zrychleného vojenského přesunu 60 km za den, dostaneme se na 6,2 dní. Cesta mezi městy Serdica a Byzantion trvala 16,5 dní pěšího pochodu o rychlosti 30 km za den a 8,6 dne zrychlené vojenského přesunu o rychlosti 60 km za den. Odéssos a Byzantion jsou ve vzdálenosti 2,3 dní plavby na moři.}

Vzhledem k tomu, že náročnost terénu se na území Thrákie navzájem liší, zaokrouhlila jsem vzdálenosti půldenního pochodu na 20 km a celodenního pochodu na 40 km. Území v okruhu do 20 km považuji za oblast spadající pod přímý ekonomicky a společensky vliv daného centra. Území do vzdálenosti do 40 km představuje oblast maximálního dosahu vlivu daného centra, avšak s nižší mírou interakce mezi centrem a regionem. Pro účely této práce se držím střední hodnoty 20 km pro vyznačení regionu daného centra s přímým ekonomickým a kulturním vlivem, což také odpovídá hodnotám 1 až 3 koeficientu určení přesnosti místa nálezu. Pro srovnání udávám i počty nápisů nalezených v širším regionu 40 km, abych srovnala bezprostřední okolí centra s jeho regionem, a z nich vyplývající trendy v rozmístění epigrafické produkce.

Na základě rozmístění zvýšených koncentrací nápisů v terénu je možné určit produkční centra, tedy místa, kde byly nápisy s největší pravděpodobností vytvářeny a určeny pro ekonomicky závislý region. Tato produkční centra byla v Thrákii nerovnoměrně rozmístěna v závislosti na geografických podmínkách, ale i na demografickém uspořádání oblasti.

\subsection[faktory-ovlivňující-rozmístění-nápisů]{Faktory ovlivňující rozmístění nápisů}

Rozmístění nálezových míst nápisů ovlivňují v první řadě geografické podmínky. Pokud se podíváme na mapu nálezových míst a jejich pozici v krajině, téměř tři čtvrtiny nápisů se nalézají v nížinách do 249 m. n. m. Na úpatí hor a v nižších horských polohách se nachází něco málo přes 20 \letterpercent{} nápisů. Zbylých 6 \letterpercent{} nápisů pochází z horských oblastí zejména v severovýchodní části Thrákie. Obecně je v horských oblastech méně nápisů než ve vnitrozemí, nicméně podél toků řek se nápisy vyskytují i v polohách nad 1000 m. n. m. Jak je dále patrné z mapy 7.01 v Apendixu 2, nápisy se mají tendence shlukovat jednak na mořském pobřeží a ve vnitrozemí v okolí velkých řek, které v antice sloužily jako hlavní komunikační tepny (Bouzek 1996, 221-222; Bravo a Chankowski 1999, 310-311; Archibald 2002).

Zeměpisné podmínky však nejsou jediným faktorem ovlivňujícím rozmístění nápisů. Pravděpodobně ještě důležitější roli hraje umístění lidských sídel, případně míst určených pro vybranou aktivitu, jako např. pohřebiště či svatyně. Z mapy 7.02 v Apendixu 2 je patrné, že velká část nápisů se nachází ve vzdálenosti 20 km od města, tedy ve vzdálenosti, kterou bylo možné ujít v jednom dni. V okruhu do 20 km od města se nachází 67 \letterpercent{} nápisů. Zbývajících 33 \letterpercent{} pochází z oblastí vzdálených od města více než 20 km. Pokud okruh okolo města rozšíříme na délku maximálního denního pochodu, do vzdálenosti 40 km od města spadá 83,5 \letterpercent{} nápisů. Zbývajících 16,5 \letterpercent{} nápisů se nachází ve větší vzdálenosti než 40 km od regionálních produkčních center.

Důležitou roli hraje nejen vzdálenost od nejbližšího města, ale i pozice vůči cestám, jak je dobře vidět na mapě 7.03 v Apendixu 2. Jako komunikace byly v antice využívány i velké řeky, které byly pravděpodobně splavné minimálně na některých úsecích. Cesty existovaly již v předřímských dobách, ale k jejich systematické stavbě, rozšiřování silniční sítě a údržbě docházelo až od 1., ale zejména ve 2. a 3. st. n. l., jak dosvědčují dochované milníky či související stavby (Madzharov 2009, 29-40). Z blízkosti několika kilometrů od cest pocházela převážná část nápisů: ve vzdálenosti do 20 km od cest se našlo 95 \letterpercent{} nápisů, nad 20 km pak zbývajících 5 \letterpercent{} nápisů. Ve vzdálenosti do 10 km od cesty se našlo 83 \letterpercent{} nápisů, nad 10 km pak 17 \letterpercent{} nápisů. Ve vzdálenosti do 5 km od cest bylo lokalizováno 75 \letterpercent{} nápisů, 25 \letterpercent{} pak ve vzdálenosti větší než 5 km.\footnote{Se zmenšující se vzdáleností od cesty se zmenšovala i počet nápisů. Jinými slovy, oblast do 5 km okolo cest obsahovala méně nápisů než oblast pokrývající území v okruhu 10 či 20 km okolo cesty.} Z toho vyplývá, že více jak dvě třetiny nápisů byly nalezeny v bezprostředním okolí cest. Tato existující infrastruktura v podobě silnic do velké míry usnadňovala pohyby nejen vojsk, ale i běžného obyvatelstva, které tak mohlo např. snadněji navštěvovat svatyně ve vzdálenějších oblastech. V oblasti se schůdným terénem jako např. v okolí Filippopole se vzdálenost od města, kterou bylo možné ujít v jednom dni, prodlužuje. V případě pohybu po {\em Via Diagonalis} a nápisů nalezených v okolí města Filippopolis to může být až 60 km po západo-východní ose.

Zásadní roli na rozmístění nápisů v krajině tedy hrály jednak příznivé přírodní podmínky a přístupný terén v kombinaci s blízkostí lidských sídel a rozmístění sítě komunikací, ať už v podobě řek či pozemních cest. Hustota nápisů byla největší ve městech, případně ve vybraných svatyních a s narůstající vzdáleností od města počet nápisů klesal. Tento trend částečně narušovaly komunikace, v jejichž bezprostřední blízkosti se nápisy taktéž nacházely, a to pravděpodobně díky zvýšenému pohybu obyvatelstva a usazování v menších sídlech a stanicích, které se staraly o údržbu a bezpečnost cest (Madzharov 2009, 43-57).

\subsection[skupiny-produkčních-center-nápisů]{Skupiny produkčních center nápisů}

Produkční centra je možné charakterizovat jako místa se zvýšenou koncentrací nálezů nápisů. Mapa 7.04 v Apendixu 2 dokumentuje hustotu nalezených nápisů na území Thrákie v podobě tzv. teplotní mapy. Místa s tmavší barvou představují místa s vyšší koncentrací nápisů v okruhu 20 km. Nejtmavší místa jsou velká města jak na pobřeží, tak ve vnitrozemí na důležitých cestách či křižovatkách cest. V horských oblastech je minimum míst s nejvyšší koncentrací nápisů, ale vyskytují se zde lokality se středními hodnotami počtu nápisů. Většina míst s největší koncentrací nápisů je v nejbližším okolí měst, ale v několika případech se setkáváme i s vysokou koncentrací nápisů v lokalitách neměstského charakteru.

\subsubsection[produkční-centra-městského-charakteru]{Produkční centra městského charakteru}

Jak plyne z výše řečeného, největším producentem nápisů jsou města a jejich nejbližší regiony ve vzdálenosti do 20 km. Města je možné rozdělit podle počtu nalezených nápisů na nadregionální producenty s počtem okolo 300 a více nápisů, velké regionální producenty s počty od 150 do 250 nápisů, menší regionální centra s 50 až 149 nápisy a malá produkční centra s 49 a méně nápisy. Jejich polohu v Thrákii a kategorii, do níž dané produkční místo spadalo ilustruje mapa 7.05 v Apendixu 2.

Do kategorie producentů s nadregionálním významem patří města, která svým významem překračovala území Thrákie či zastávala významnou pozici v rámci nadregionální samosprávy. Do této kategorie patří města Byzantion, Odéssos a Filippopolis. Ač nejvíce nápisů bylo nalezeno v černomořském Odéssu, a to celkem 359, toto číslo je ve skutečnosti o pár desítek nápisů nižší, protože do 20 km okruhu okolo Odéssu zasahuje území Marcianopole a Dionýsopole a program QGIS započítal tyto nápisy na pomezí ke všem městům stejně. Nicméně i přesto z regionu Odéssu pochází zhruba 300 nápisů a řadí se tak k jedněm z největších producentů nápisů v Thrákii. Z Byzantia a jeho regionu pochází 352 nápisů, což ho tak řadí na první místo epigrafické produkce.\footnote{Byzantion zastával významnou pozici v době vlády Říma, a to zejména od 4. st. n. l., kdy se stal pod jménem Konstantinopol hlavním městem římské říše (Jones 1971, 23).} Filippopolis s 296 nápisy spadá do kategorie nadregionálních produkčních center, jakožto administrativní a kulturní centrum provincie {\em Thracia}.

Do skupiny velkých regionálních center spadají čtyři města na pobřeží Černého, Marmarského a Egejského moře, která hrála roli administrativní a ekonomického centra daného přímořského regionu. Největším producentem z této skupiny je Perinthos s 248 nápisy, dále Maróneia s 234 nápisy, Apollónia Pontská s 218 nápisy a Mesámbria se 172 nápisy. Tato města zaujímala pozici regionálních center zejména ve stoletích př. n. l., nicméně si jistou míru autonomie a politické moci udržela i v římské době.

Skupina středních regionálních center zahrnuje města jak na mořském pobřeží, tak města vnitrozemská, která se nacházela většinou na křižovatkách cest či významných komunikacích a sloužila jako administrativní a ekonomické centrum pro nejbližší region. Do této kategorie patří Parthicopolis se 147 nápisy, Nicopolis ad Istrum se 136 nápisy, Hérakleia Sintská se 128 nápisy, Augusta Traiana se 124 nápisy, Pautália se 122 nápisy, Serdica se 112 nápisy a Marcianopolis se 122 nápisy.

Skupina malých regionálních center s méně než 30 nápisy zaujímala pouze marginální roli v epigrafické produkci, což však nereflektuje její postavení ve společnosti jako ekonomické či politické centrum okolní oblasti, jak je tomu např. u Ainu či Nicopolis ad Nestum či Bizyé. Tento stav spíše reflektuje nedostatečný stav prozkoumání regionu těchto měst, zejména z důvodu moderní zástavby. V budoucnu se dá tak v těchto městech dají očekávat nálezy nápisů, které by tato města posunula do kategorie středních regionálních producentů.

\subsubsection[produkční-centra-neměstského-charakteru]{Produkční centra neměstského charakteru}

Z již zmiňované teplotní mapy plyne, že místa s velkou hustotou nápisů nebyla pouze městského charakteru, ale dochovalo se i několik lokalit mimo region města s vysokou koncentrací nápisů. V těchto případech se jedná o svatyně umístěné ve volné přírodě, kde bylo nalezené velké množství dedikací. Společnou charakteristikou těchto svatyní byla dobrá dostupnost, a tedy i blízkost cest v případě horských lokalit či nenáročnost terénu v případě svatyň v nížinách a v podhůří, jak je též patrné z mapy 7.06 v Apendixu 2.

Mezi svatyně s největším počtem nápisů patří svatyně v Batkunu s 194 nápisy nalezenými ve vzdálenosti 5 km od místa svatyně Asklépia.\footnote{Tsonchev (1941).} Dále sem patří svatyně Asklépia z lokality Slivnica se 72 nápisy, Glava Panega se 77 nápisy, Daskalovo se 76 nápisy.\footnote{Boteva (1985); Gocheva (1992), Oppermann (2006, 147-154).} Svatyně Apollóna s místními epitety pochází z lokalit Kiril Metodievo s 33 nápisy a Kran taktéž s 33 nápisy.\footnote{Tabakova (1959, 1961), Tabakova-Tsanova (1980).} Svatyně Nymf a Asklépia je známá z lokality Búrdapa, kde bylo nalezeno 47 nápisů.\footnote{Janouchová (2013, 14).} V lokalitě Mezdra bylo nalezeno 17 nápisů věnovaných různým božstvům, mimo jiné Démétér či Diovi. S lokalitě Skaptopara se pak jedná převážně o funerální nápisy náležející k blízkému osídlené - thrácké vesnici Skaptopara.

Většina těchto svatyní pochází ze severozápadní Thrákie z horských oblastí. Nejníže položená lokalita se nachází ve výšce 209 m. n. m a nejvýše položená ve výšce 754 m. n. m. Průměrná nadmořská výška všech devíti lokalit je 404 m. n. m. (aritmetický průměr; medián je 345 m. n. m.). Lokality jsou v těsné blízkosti cest a řek, což usnadňovalo věřícím přístup a pravděpodobně tento fakt hrál roli v jejich oblíbenosti mezi věřícími. Svatyně s nejvíce nápisy se nachází ve vzdálenosti 28, respektive 50 km od města Filippopolis, které po většinu římské doby zastávalo roli hlavního města provincie a jehož obyvatelé s největší pravděpodobností navštěvovali tuto svatyni Nymf v Búrdapě a Asklépia v Batkunu.

Ať už se jedná o nápisy nalezené v regionu měst či v lokalitách neměstského charakteru, zásadní roli na rozmístění lokalit měla existence místní infrastruktury a systému komunikací. Nápisy se objevují zejména ve vzdálenosti dosažitelné v rámci jednoho dne od hlavních administrativních center a podél cest. Čím bylo dané sídlo důležitější a koncentrovaly se v něm politické a administrativní instituce, tím více se v jeho okolí objevilo i nápisů. Ekonomická aktivita jednotlivých měst na celkovou výši epigrafické produkce zásadní vliv neměla, příkladem může být město Ainos, jeden z hlavních říčních a přímořských přístavů s až překvapivě nízkým počtem nápisů. Obecný trendem zůstává, že s přibývající vzdáleností od lidských sídel, narůstající nadmořskou výškou a vzdáleností od cest počet nápisů klesá.

\section[rozmístění-nápisů-v-závislosti-na-jejich-typologii]{Rozmístění nápisů v závislosti na jejich typologii}

Rozmístění nápisů a jejich zařazení dle společenské funkce přináší nový pohled na roli, jakou nápisy v antické Thrákii hrály. Rozmístění veřejných nápisů odpovídá roli, jakou hrály byrokratické instituce a infrastruktura vytvářená komplexní společností v době existence římské provincie. Soukromé nápisy mohou existovat nezávisle na tomto uspořádání, nicméně k jejich rozšíření ve větším měřítku dochází právě v koexistenci s prostředím raného státu s centralizovanou mocí a organizovaným řízením. Proto se i soukromé nápisy nacházejí v okolí center či oblastí se zvýšenou mírou společenské a politické provázanosti jakou obyvatelům poskytují města. Naopak venkov je epigraficky málo aktivní, a to i v římské době, a aktivity souvisí spíše s iniciativou jedinců než obecným přístupem venkovské populace.

\subsection[veřejné-nápisy-jako-produkt-komplexní-společnosti]{Veřejné nápisy jako produkt komplexní společnosti}

Rozmístění veřejných nápisů odpovídá funkci, kterou hrály v rámci organizace společnosti, jako např. uchovávání a rozšiřování informací důležitých pro chod společnosti a udržení pořádku. Veřejné nápisy sloužily nejen jako zdroj informací, ale také samy byly projevem politické moci daného subjektu, a proto se vyskytují v hojné míře v administrativních a politických centrech, ať už regionálního či nadregionálního charakteru (Tainter 1986, 99-106).

Mapa 7.07 v Apendixu 2 velmi dobře ilustruje rozmístění veřejných nápisů a jejich vzdálenosti od městských center, případně cest. Z celkem 695 veřejných nápisů pochází 79 \letterpercent{} ze vzdálenosti do 20 km, a zbývajících 21 \letterpercent{} nápisů ze vzdálenosti nad 20 km. Skupina nápisů nalezených na území ve vzdálenosti maximálního denního pochodu, tedy do 40 km od měst, představuje 92 \letterpercent{} všech nápisů. Zbylých 8 \letterpercent{} nápisů pochází ze vzdálenosti větší než 40 km od měst a nachází se většinou v přímé blízkosti silnic. Obsah této menší skupiny milníků a stavebních nápisů nejčastěji souvisí právě s údržbou římských silnic a vytyčováním vzdáleností.\crlf
Jedním z příkladů projevů politické autority a existující infrastruktury v krajině je rozmístění milníků, které označovaly vzdálenost k významnému městu v římské době (Hollenstein 1975, 23-45; Madzharov 2009, 57-59). Zpravidla milník udával vzdálenost v mílích k městu, do jehož regionu spadala správa cesty. Nápis byl umístěn viditelně vedle cesty tak, aby si každý cestující zjistil jednak jak velkou vzdálenost mu zbývá ujít či ujet do daného města, ale nápis také nesl informace o tom, kdo cestu spravuje, případně se postaral o její postavení či opravy. Na mapě 7.08 v Apendixu 2 je velice dobře vidět, že nálezová místa milníků kopírují trasu známých římských cest.\footnote{Madzharov (2009, 57-59) udává celkový počet řeckých i latinských milníků z území dnešního Bulharska na 180 exemplářů. Mapa 7.08 v Apendixu 2 zobrazuje pouze datované milníky, z nichž většina je psána řecky. Latinsky psané milníky nejsou z větší části zahrnuty do HAT databáze.} Nejvíce milníků pochází z tzv. {\em Via Diagonalis}, což byla jedna z nejvýznamnějších cest Balkánu, protože spojovala východní provincie se západem a sloužila k častému přesunu vojsk mezi lokalitou Singidunum, dnešním Bělehradem, a Byzantiem, dnešním Istanbulem (Jireček 1877; Madzharov 2009, 70-131). Desítky milníků se dochovaly z okolí města Serdica, Filippopolis a Augusta Traiana, pod jejichž správu vybrané úseky {\em Via Diagonalis} spadaly. První milník se objevil již v 1. st. n. l. v oblasti {\em Via Egnatia} na egejském pobřeží, nicméně stavební aktivity ve vnitrozemské Thrákii jsou dobře dokumentované až z 2. st. n. l. Nejvíce milníků pochází z 3. st. n. l., kdy docházelo k úpravám {\em Via Diagonalis}, vzhledem k jejímu častému využití římskými vojsky (Hollenstein 1975, 27-41). Na přelomu 3. a 4. st. n. l. docházelo k omezení stavebních aktivit a odpovídá tomu i menší počet dochovaných milníků. Poslední milník se dochoval ze 4. st. n. l. z {\em Via Egnatia} v okolí Perinthu.

Z některých úseků pochází velké množství milníků, např. v okolí města Serdica, ale z velké části cest se nedochoval ani jeden milník. Tento fakt je možné přisuzovat stavu archeologických výzkumů, které se zaměřují na zkoumání osídlení, což vede k relativně málo známému systému římských cest v Thrákii. Výzkumy posledních let se začínají zaměřovat i na síť silnic, jako na důležitou součást provinciálního uspořádání, a v budoucnosti se tak dají očekávat nové objevy, které mohou přinést i nové milníky a související stavební nápisy.

\subsection[soukromé-nápisy-a-projevy-kulturních-zvyklostí]{Soukromé nápisy a projevy kulturních zvyklostí}

Rozmístění soukromých nápisů do velké míry sleduje podobné trendy jako veřejné nápisy: větší část nápisů se nachází ve vzdálenosti na úrovni jednodenního pochodu od měst, případně v lokalitách podél cest, avšak ve srovnání s veřejnými nápisy je to až o 15 \letterpercent{} nápisů méně. Ve vzdálenosti do 20 km od měst se totiž nachází 64 \letterpercent{} nápisů, zbývajících 36 \letterpercent{} je ve vzdálenosti větší než 20 km. Pokud bereme v potaz vzdálenost do 40 km, pak se v tomto rozsahu nachází 81 \letterpercent{} z 3440 soukromých nápisů a zbylých 19 \letterpercent{} je vzdáleno od měst více než 40 km.

Soukromé nápisy se objevují v úzkém pobřežním pásu do vzdálenosti zhruba 20 km na pobřeží Egejského a Marmarského moře, jak je patrné z mapy 7.09 v Apendixu 2. Na pobřeží Černého moře nápisy plynule přechází z pobřeží do vnitrozemí, což může být vysvětleno absencí pohoří, které by bránilo jejich rozšíření podobně jako v případě egejské oblasti. Ve vnitrozemí se nápisy nachází zejména v okolí {\em Via Diagonalis} a na ní ležících měst Filippopolis a Serdica, nicméně zvýšené koncentrace soukromých nápisů můžeme pozorovat téměř v okolí všech měst s výjimkou Bizyé. V případě Bizyé lze však absenci nápisů vysvětlit nedostatkem publikovaného materiálu, a nikoliv negativním postojem obyvatelstva vůči zhotovování nápisů pro soukromé účely. Na rozdíl od veřejných nápisů soukromé nápisy pocházejí i z venkovských oblastí, které nesousedí s městem a ani v jejich blízkosti neprochází řádná z hlavních římských cest. Přítomnost soukromých nápisů v rurálním kontextu je tak možné vysvětlit jako důsledek pohybu obyvatelstva či adopce epigrafických zvyklostí v omezené míře i ve venkovském prostředí.

Dvě nejčastější společenské funkce, jakou soukromé nápisy zastávaly byla funkce funerální a dedikační. Rozmístění soukromých nápisů dle jejich typologie přináší zajímavé poznatky o chování tehdejšího obyvatelstva a šíření kulturních zvyklostí mezi jednotlivými komunitami.

\subsubsection[funerální-nápisy-19]{Funerální nápisy}

Rozmístění funerálních nápisů reflektuje odlišné zvyklosti, respektive odlišnou demografickou strukturu obyvatelstva pobřežní a vnitrozemské Thrákie. Jak je patrné z mapy 7.10 v Apendixu 2, v případě pobřežní Thrákie se funerální nápisy nacházejí převážně přímo ve městech či v nejbližším okolí. Ve vnitrozemí pak nalézáme funerální nápisy jak ve městech, ale i v oblastech podél cest mimo region měst, případně na venkově mimo dosah cest. Z celkem 1631 funerálních nápisů se jich 87 \letterpercent{} nachází ve vzdálenosti do 20 km od měst a 13 \letterpercent{} ve vzdálenosti větší než 20 km. Do skupiny nápisů nalezených ve vzdálenosti od měst v délce denního pochodu, tedy 40 km, spadá 97 \letterpercent{} nápisů a jen 3 \letterpercent{} byla nalezena ve vzdálenosti větší než 40 km. Z toho plyne, že většina funerálních nápisů se nacházela v okolí měst či přímo ve městech. V případě vzdálenosti nálezových míst od trasy cest bylo 80 \letterpercent{} nápisů nalezeno ve vzdálenosti do 5 km od cesty, 20 \letterpercent{} nápisů ve vzdálenosti větší než 5 km. Pokud tuto vzdálenost změníme na 10 km od cesty, počet nápisů naroste na 84 \letterpercent{} všech nápisů nalezených do vzdálenosti 10 km a 16 \letterpercent{} ve vzdálenosti nad 10 km. V případě vzdálenosti 20 km tento poměr naroste na 97 \letterpercent{} nápisů ve vzdálenosti do 20 km a 3 \letterpercent{} ve vzdálenosti nad 20 km. Z toho plyne, že většina funerálních nápisů se našla v bezprostřední blízkosti cesty.

Pokud se podíváme na umístění funerálních nápisů v krajině, celkem 93 \letterpercent{} nápisů pochází z nížin do 273 m. n. m., čemuž odpovídá i průměrná nadmořská výška nálezových míst všech funerálních nápisů 81 m. n. m.\footnote{Pro srovnání průměrná nadmořská výška nálezových míst všech dedikačních nápisů je 311 m. n. m., viz níže.} Funerální nápisy tedy byly nalézány především v dobře přístupném terénu, v blízkosti měst a podél tras hlavních cest. Funerální nápisy zpravidla pocházejí z okolí lidských sídel, tedy z míst s vyšší mírou zalidnění. Lidé byli pohřbíváni mimo centrum měst, nejčastěji v bezprostředním okolí hlavních cest vedoucích směrem z města ven. Jedná se o zcela přirozený jev umisťovat pohřebiště v blízkosti lidských sídel, a nikoliv do odlehlých horských oblastí (Kurtz a Boardman 1973, 49-51; Damyanov 2010, 270).

Z četnosti výskytu funerálních nápisů na mapě 7.11 v Apendixu 2, zcela jasně vyplývá, že zvyk zhotovovat funerální nápisy převažuje v pobřežních oblastech, konkrétně ve městech Byzantion, Apollónia Pontská, Odéssos, Perinthos, Maróneia a Mesámbria, původně řecké kolonie. Z vnitrozemských center je to původně makedonské sídlo Hérakleia Sintská a Makedonci založená Filippopolis. Hustota výskytu funerálních nápisů ve vnitrozemí má však daleko nižší hodnoty, z čehož plyne, že zvyklost vytvářet a veřejně vystavovat nápisy patří spíše do řecké kulturní okruhu a pouze částečně se rozšířila z řeckých měst na pobřeží do vnitrozemské Thrákie. Tomuto faktu odpovídá i datace funerálních nápisů. Z 6. až 1. st. př. n. l. pocházejí funerální nápisy převážně z pobřežních oblastí či z oblasti Hérakleie Sintské a lokality Didymoteichon, zatímco z 1. až 5. st. n. l. pocházejí vnitrozemské nápisy jak z pobřeží, oblasti toku řeky Strýmón, tak i z oblasti centrální Thrákie.

\subsubsection[dedikační-nápisy-19]{Dedikační nápisy}

Dedikační nápisy jsou projevem náboženského přesvědčení jednotlivců i celých skupin a jejich rozmístění reflektuje normy chování v rámci jednotlivých komunit. Pobřežní Thrákie se vyznačuje velmi nízkým počtem dedikací, zatímco ve vnitrozemí se nachází několik oblastí s vysokou koncentrací dedikačních nápisů.

Jak je patrné z mapy 7.12 v Apendixu 2 dedikační nápisy se vyskytují jak v blízkosti měst a podél cest, podobně jako funerální nápisy. Ve vzdálenosti do 20 km od měst se nachází 43 \letterpercent{} nápisů, zbylých 57 \letterpercent{} nápisů pochází ze vzdálenosti vyšší než 20 km. Do skupiny nápisů nalezených ve vzdálenosti do 40 km od měst spadá 68 \letterpercent{} nápisů, zbylých 32 \letterpercent{} bylo nalezeno ve vzdálenosti větší než 40 km od měst. V porovnání s funerálními nápisy se dedikační nápisy vyskytují v okolí měst v menší míře a zejména pocházejí z venkovských a horských oblastí, které jsou vzdáleny od hlavních městských center.\footnote{Funerální nápisy mají poměr 87:13 u vzdálenosti 20 km (do 20 km vs. nad 20 km), a u vzdálenosti 40 km se tento poměr mění na 97:3.}

Největší koncentrace dedikací pocházejí z podhůří pohoří Rodopy a Pirin a Stara Planina v centrální Thrákii a z horských oblastí severozápadní Thrákie. Průměrná nadmořská výška místa nálezu dedikačního nápisu je 311 m. n. m., ale 21 nápisů bylo nalezeno dokonce ve výškách nad 764 m. n. m.\footnote{Pro srovnání průměrná nadmořská výška nálezových míst funerálních nápisů je 81 m. n. m., z čehož plyne, že dedikační nápisy byly nalézány ve vyšších horských polohách, případně na úpatí hor.} Téměř 80 \letterpercent{} nápisů pochází z oblastí do 509 m. n. m., představuje jednak oblast nížin, ale i na úpatí hor a v údolí řeky Strýmónu. Dedikace pocházející z horských oblastí s nadmořskou výškou vyšší než 510 m. n. m. pocházejí zejména z oblasti okolo měst Serdica a Pautália s pohořími Vitoša a Stara Planina.

Dedikační nápisy a jejich výskyt v horských oblastech a podhůří poukazuje na udržení tradičního charakteru thráckého náboženství, které bylo spojeno s přírodními silami a ve vztahu s okolní krajinou (Janouchová 2013, 10). Svatyně původně thráckých božstev byly nejčastěji umístěny ve volné přírodě a tento trend pokračoval i v době římských, z níž pochází většina dochovaných dedikačních nápisů.

\section[srovnání-epigrafické-produkce-s-archeologickými-daty]{Srovnání epigrafické produkce s archeologickými daty}

Ač se na první pohled může zdát, že dochované nápisy nám poskytují informace o interakci jednotlivých komunit v plném rozsahu, opak je často pravdou a dochování nápisů může do velké míry dílem náhody. Nápisy se dochovaly pouze z části tehdy existujících lokalit, a dokonce i ne ze všech řeckých kolonií té doby. Archeologické výzkumy neprobíhají na všech místech stejnou měrou a některé lokality jsou lépe prozkoumané. Pokud by produkce nápisů byla rovnoměrná po celém území, pak by z lépe prozkoumaných lokalit mělo také pocházet více nápisů, čemuž tak je pouze v určitých případech.\footnote{Např. v případě Apollónie Pontské ve 4. st. př. n. l. či v Byzantiu ve 2. a 1. st. př. n. l., více v sekcích věnovaných jednotlivým stoletím v kapitole 6.} Jaký je poměr epigraficky aktivních měst vůči těm, které nápisy neprodukovaly? Nebo jinými slovy, na kolik jsou nápisy relevantním historickým zdrojem pro studium společnosti jako celku? Co tedy stojí za faktem, že z určitých lokalit pochází velké množství nápisů a z podobných lokalit, které jsou i do stejné míry prozkoumané, se dochovalo nápisů velmi málo či dokonce žádné? Na tyto otázky se pokusím alespoň nastínit odpověď na dvou příkladech srovnávajících epigrafická a archeologická data z území Thrákie.

Zásadním problémem, na nějž toto a podobná srovnání naráží, je nekompletní povaha archeologických, ale i epigrafických dat. Naše současné znalosti postihují pouze zlomek tehdy existujících lokalit, který je podobně nahodilý jako v případě nápisů. Do značné míry tak srovnání založené na souborech archeologických a epigrafických dat může být nepřesné a částečně zkreslené charakterem dostupných dat. Proto v žádném případě nelze zde uváděná čísla brát jako kompletní a neměnná, ale spíše jako orientační a shrnující aktuální stav našich znalostí. Hlavním cílem tohoto srovnání je ukázat, že epigraficky aktivní byla pouze velmi malá část tehdejší společnosti a nápisy pocházejí pouze ze zlomku tehdy známých lokalit, ač by jejich relativně velké počty na první pohled mohly svědčit o opaku.

\subsection[příklad-řeckých-měst-v-thrákii-v-7.-až-4.-st.-př.-n.-l.]{Příklad řeckých měst v Thrákii v 7. až 4. st. př. n. l.}

Částečné srovnání poměru epigraficky aktivních komunit na úrovni měst a komunit bez epigrafické aktivity mezi 7. a 4. st. př. n. l. umožňují data získaná z nedávno publikovaných kompendií archeologických lokalit jako Hansen {\em et al.} (2004) {\em An Inventory of Archaic and Classical Poleis}, které zaznamenává především řecká města na pobřeží a Talbert (2000) {\em Barrington Atlas of the Greek and Roman World}, zaznamenávající archeologické lokality jak na pobřeží, tak ve vnitrozemí.\footnote{Hansen {\em et al.} (2004) se primárně zaměřuje jen na řecká města archaického a klasického období. Celkové počty řeckých kolonií udávaných kolektivem autorů nereprezentují celkový počet všech existujících lokalit v daném století, a to již z podstaty historiografických a archeologických dat, které obsahují velkou míru nejistoty a nekompletnosti. Nicméně i přesto se jedná o jeden ze současných nejucelenějších souborů informací, který lze použít pro celkové srovnání. Jako další kontrolní soubor používám lokality z {\em Barrington Atlas of Greek and Roman World} (Talbert 2000), který je však obecně považován za méně přesný než Hansen {\em et al.} (2004). Talbert (2000) uvádí jak řecká osídlení, tak i osídlení thráckého původu, avšak nerozlišuje mezi jednotlivými stoletími, ale lokality zasazuje dohromady v rámci období, a proto se celkové číslo jeví jako dvojnásobně vyšší v případě 5. a 4. st. př. n. l., tedy klasického období.} Počet lokalit z těchto dvou zdrojů je nicméně vhodné chápat nikoliv jako kompletní výčet, ale spíše měřítko aktivit v dané oblasti a indikátor objevujících se trendů.

Tabulka 7.01 v Apendixu 1 poskytuje přehled počtu lokalit udávaných Hansenem {\em et al.} (2004) a Talbertem (2000) v jednotlivých stoletích a srovnává je s celkovým počtem epigrafických lokalit. Odhlédneme-li od problematiky konečných čísel lokalit a zaměříme se spíše na obecné trendy, je možné sledovat, že a) narůstající počet sídlišť odpovídá i narůstajícímu počtu epigrafických lokalit, b) sídlišť bez epigrafické produkce je několikanásobně více než sídlišť s dochovanou epigrafickou produkcí, a to zejména v 7. a 6. st. př. n. l. a v kontextu vnitrozemských lokalit. Z tohoto srovnání plyne, že epigraficky aktivní byla v nejlepším případě jen třetina všech známých sídlišť a zhruba dvě třetiny měst řeckého původu. Oproti původnímu očekávání nelze zvyk publikovat nápisy automaticky považovat za jednotnou charakteristiku všech řeckých osídlení na území Thrákie, ale i v rámci řeckých {\em poleis} se setkáváme s odlišným přístupem k epigrafické produkci, který se měnil v průběhu staletí.

Jednou z hlavních myšlenek hellénizačního přístupu v tradičním slova smyslu je, že se hellénizace společnosti projevuje jako uniformní ve všech řecky mluvících komunitách a jejich okolí. Epigrafická produkce je tradičně považována za jeden z projevů „hellénství” a je vnímána jako nedílný projev společensko-kulturního uspořádání řecky mluvících komunit. Jak je patrné z výše uvedeného, epigrafická produkce se v případě řeckých komunit na území Thrákie objevovala zhruba ve dvou třetinách komunit existujících v období od 7. do 4. st př. n. l., a proto nelze tyto dva fenomény spojovat bez dalšího vysvětlení a zařazení místních specifických podmínek v jednotlivých obcích.\footnote{Podrobněji se vybranými regiony se zabývám v rámci sekcí věnovaným jednotlivým stoletím v kapitole 6.} Pro období 7. až 4. st. př. n. l. tak epigrafická produkce figuruje spíše jako doplňující zdroj informací o tehdejší společnosti a je nutné přihlížet k archeologickým poznatkům, případně k historiografickým pramenům.

I přesto, že nápisy představují značně selektivní zdroj informací, v celkovém pohledu nabízí jedinečné srovnání jednotlivých regionů, a to i v období od 7. do 4. st. př. n. l. (Woolf 1998, 80-82; Bodel 2001, 38-39). Regionální rozdíly v epigrafické produkci mohou značit jednak výrazně jiný přístup komunity k epigrafické kultuře, související se situací v mateřském městě, nestabilní politickou situaci, nedostatek materiálu, či pouze reflektují náhodnost dochování nápisů. Nápisy se zpravidla nedochovaly z prvních let po založení kolonie, ale pochází z následujících let, ne-li desetiletí, kdy došlo ke stabilizaci politické situace, zajištění základních potřeb a bezpečí obyvatelstva. Stabilizace podmínek, za nichž mohly vznikat nápisy ve větší míře, tak přímo může souviset s nárůstem společenské komplexity a míry společenské organizace. Pokud totiž dojde k upevnění politické moci, dochází i zpravidla k ustálení produkce a zajištění nutné infrastruktury (Tainter 1988, 106-118).\footnote{O roli institucí, byrokratického aparátu na rozšíření nápisů podrobněji hovořím v kapitole 3.} V období prosperity je tak možné sledovat nárůst celkového počtu soukromých nápisů, a naopak v době nestability narůstají ve snaze regulovat tehdejší společnost celkové počty nápisů veřejných.

Tabulka 7.02 v Apendixu 1 dokazuje, že 5. a 4. st. př. n. l. představovalo dle narůstajících počtů archeologických i epigrafických lokalit a celkových počtů epigrafické produkce dobu relativní stability, zatímco ve 3. st. př. n. l. došlo k poklesu celkové epigrafické produkce, markantnímu nárůstu veřejných nápisů na třetinu celkového počtu a poklesu soukromých nápisů. Stejně tak došlo i k poklesu epigrafických lokalit, což nasvědčuje na destabilizaci situace v regionu v průběhu 3. st. př. n. l. Z historických zdrojů víme, že 3. st. př. n. l. byla Thrákie svědkem několika vojenských tažení hellénistických panovníků, invaze Keltů ze střední Evropy, která s sebou nesla i destrukce archeologických lokalit a vedla k přeměně společenského uspořádání Thrákie. Ve 2. st. př. n. l. naopak dochází k určití stabilizaci poměrů, což se projevuje i na zvýšené epigrafické produkci, nárůstu počtu lokalit a nárůstu soukromých nápisů. 1. st. př. n. l. naopak zaznamenává propad epigrafické produkce, pokles počtu epigrafických lokalit a soukromých nápisů téměř na poloviční hodnoty, zatímco veřejné nápisy se udržují na podobné úrovni jako ve 3. st. př. n. l. a představují zhruba čtvrtinu všech nápisů. V Thrákii v této době dochází k přeměně společenské struktury směrem k vazalskému království v područí Říma, což s sebou neslo míru nejistoty, která se odrazila i na epigrafické produkci.

\subsection[příklad-z-thráckého-vnitrozemí-kazanlacké-údolí-v-době-římské]{Příklad z thráckého vnitrozemí: Kazanlacké údolí v době římské}

Jiný druh srovnání archeologických a epigrafických dat nabízí následující příklad z vybraného regionu vnitrozemské Thrákie, Kazanlackého údolí. Srovnání na mikro-regionální úrovni umožňuje porovnat poměr všech známých archeologických lokalit s epigrafickými lokalitami, což je jen velmi těžko dosažitelné na úrovni makro-regionální jako v případě srovnání všech známých lokalit ze 7. až 4. st. př. n. l. Proto jsem jako příklad zvolila dobře ohraničené území ve vnitrozemské Thrákii a zaznamenala všechny známé archeologické lokality a porovnala je se známými místy nálezů nápisů.\footnote{Pro tyto účely jsem zvolila oblast Kazanlackého údolí ve střední části Bulharska. Oblast okolo Kazanlaku je známá jako kulturní a historické centrum thráckých panovníků, kteří udržovali čilé kontakty s řeckými obcemi. V letech 2009-2011 zde probíhaly povrchové sběry projektu {\em The Tundzha Regional Archaeological Project} (TRAP; Sobotkova {\em et al.} 2010; Ross {\em et al.} v přípravě, vyjde 2017), během nichž se podařilo zmapovat osídlení ve větší části Kazanlackého údolí. Výsledky těchto sběrů tak představují výborný výchozí soubor dat, pokrývající větší část vybraného regionu v jeho komplexnosti a zaznamenávají jak viditelné monumenty, tak i koncentrace keramiky a architektonických prvků na povrchu. Pro úplnost porovnávám data i se soupisy archeologických lokalit, které byly pořízeny pro Kazanlak v roce 1991 (Domaradzki 1991; Tabakova-Tsanova 1991) a výstupy archeologických vykopávek na zkoumaném území (Chichikova, Dimitrov a Alexieva 1978; Tabakova 1959; Tabakova-Tsanova 1961, 1980; Dinchev 1997; Nekhrizov {\em et al.} 2013). Toto srovnání vychází ze studie, která vyjde v roce 2017 rámci sborníku z projektu {\em The Tundzha Regional Archaeological Project} (Janouchová v přípravě, vyjde 2017).}

Projekt {\em The Tundzha Regional Archaeological Project}, jehož závěry používám jako výchozí data, zaznamenal v průběhu let 2009 až 2011 celkem 82 archeologických lokalit a 773 mohyl na území o rozloze 85 km\high{2} (Sobotková v přípravě, vyjde 2017). Pro dobu pozdně železnou (500 - 0 př. n. l.) bylo nalezeno 38 lokalit a pro dobu římskou (1 - 400 n. l.) 23 lokalit, což v průměru znamená výskyt jedné lokality na 2,23 km\high{2} v době pozdně železné a na 3,69 km\high{2} v době římské. Jak je patrné z tabulky 7.03 v Apendixu 1, z vybraného území se dochovalo celkem 43 nápisů, z nichž osm spadalo do doby pozdně železné a bylo nalezeno ve čtyřech lokalitách s centrem v Seuthopoli, hellénistické rezidenci odryského panovníka Seutha (Dimitrov, Chichikova a Alexieva 1978, 3-5). Tyto nápisy poukazují na prominentní roli Seuthopole a zcela ojedinělý přístup k publikaci nápisů a řecké kultuře obecně, který panovník Seuthés zaujímal (Janouchová v přípravě, vyjde 2017). Nápisy z Kazanlackého údolí v hellénistické době pocházely pouze z kontextů spojených s panovníkem Seuthem a zvyk veřejně vystavovat nápisy tesané do kamene, jak je obvyklé v řeckých komunitách, po jeho smrti postupně vymizel z thráckého prostředí na několik dalších století. Nicméně povědomí o užívání písma přetrvalo alespoň v prostředí thrácké aristokracie v mírně změněné formě, která předměty nesoucí nápisy využívala pro soukromé účely, jakožto prestižní předmět, zdůrazňující jejich společenský status (Sahlins 1963; Whitley 1991, 349-350).

Z doby římské pocházelo 35 nápisů z pěti lokalit, z čehož většina nápisů pocházela ze svatyní Apollóna, nesoucího místní přízviska {\em Teradéenos} a {\em Zerdénos}, umístěných v okolí moderní vesnice Kran (Tabakova 1959, 97-104; Tabakova-Tsanova 1980, 173-194). Z toho plyne, že ze známých lokalit doby pozdně železné 10,5 \letterpercent{} obsahovalo minimálně jeden nápis. V době římské se tento poměr zvýšil na dvojnásobek na 21,7 \letterpercent{}. Pokud vezmeme v potaz charakter jednotlivých epigrafických lokalit, v římské době většina nápisů z oblasti pozemních sběrů v rámci projektu TRAP pochází ze svatyní Apollóna nesoucího místní přízvisko. Většina nápisů má soukromý charakter a jejich zhotoviteli jsou osoby nesoucí převážně thrácká či kombinovaná římská a thrácká jména, což poukazuje na jejich zapojení v římské armádě a samosprávě měst (Janouchová v přípravě, vyjde 2017). V římské době tedy nárůst počtu epigrafických lokalit odpovídá i většímu zapojení místního obyvatelstva do epigrafické produkce, která i přesto zůstávala nadále poměrně nízká.

\section[shrnutí-25]{Shrnutí}

Nápisy v antické Thrákii pocházejí především z produkčních center umístěných ve městech a jejich bezprostředním okolí, což je trend podobný i v jiných částech antického světa, např. v římské provincii {\em Gallia}. Zvýšená koncentrace nápisů se nachází i podél cest, ať už z menších osídlení v jejich okolí či související se správou cesty.

Zásadní role politické autority a organizace společnosti je vidět i na rozmístění veřejných nápisů, které se koncentrují v okolí velkých administrativních center, případně podél cest v podobě milníků a stavebních nápisů. Na veřejné nápisy navazují nepřímo i nápisy soukromé, které vykazují podobné rysy: velký počet soukromých nápisů pochází z okolí měst, kde žila velká část epigraficky aktivních obyvatel v římské době, a dále z menších sídel v okolí cest, jak je patrné z rozmístění funerálních nápisů. Soukromé nápisy se, na rozdíl od nápisů veřejných, vyskytují i mimo dosah cest a často v těžko přístupném terénu, jak je možné vidět zejména v případě dedikačních nápisů.

Rozmístění funerálních nápisů poukazuje, že zvyk zřizovat náhrobky se rozšířil zejména v řecké komunitě na pobřeží, kde si udržel výsadní postavení jak v předřímské, tak i v římské době. Do vnitrozemí se tato zvyklost rozšířila pouze částečně, a to do původně makedonských hellénistických sídel v Héraklei Sintské či Filippopoli.

Dedikační nápisy oproti tomu pocházejí převážně z thráckého vnitrozemí. Vzhledem k jejich rozmístění v krajině se dá usuzovat, že převážně rurální charakter svatyní typický pro thrácké náboženství přetrvával i v době římské. Svatyně se nalézaly v blízkosti cest a jejich největší koncentrace pochází z podhůří Rodop v okolí města Filippopolis a horských oblastí okolo města Serdica. Nápisy pocházejí jak z malých venkovských svatyněk, tak i z lokalit až s téměř 200 nápisy. Božstva na nápisech nesou řecká jména, jako je např. Asklépios, Apollón či Nymfy, nicméně velmi často mají i místní jméno, které s největší pravděpodobností poukazuje na lokální tradici a náboženský synkretismus mezi thráckými a řeckými náboženskými představami.

Srovnání s dostupnými archeologickými daty nabízí zcela novou perspektivu a ukazuje, že přístup k publikaci nápisů nebyl vždy stejný. Obecně více nápisů pochází z řeckých komunit než z thráckého vnitrozemí, nicméně i v rámci řeckých měst se setkáváme zhruba s jednou třetinou lokalit bez epigrafických nálezů. Proto nelze zcela jednoznačně považovat zvyk publikovat nápisy za univerzální projev řecké kultury a společenského uspořádání, ale k vysvětlení důvodů rozšíření nápisné kultury je nutné se zaměřit i na jiné druhy motivací, jako je snaha využití nápisů k šíření státem podporované ideologie a řízení státního celku či osobní pohnutky spojené s projevem víry a dosažení společenské prestiže.

\chapter{Závěr}
Epigrafické památky sloužily v pracích zaměřených na hellénizaci neřecky mluvících obyvatel jako měřítko a zároveň prostředek naprostého pořečtění obyvatelstva, a to bez další analýzy. Kritické zhodnocení epigrafických památek za účelem identifikace společenských změn byla v~minulosti reflektována jen velice okrajově (Mihailov 1977, 343-344; Sharankov 2011, 135-145). Pouhá přítomnost řecky psaných textů na území Thrákie sloužila jako důkaz adopce řecké kultury, společenské organizace, a v souvislosti s tím i řecké identity. Hellénizace obyvatelstva byla vnímána jako nevyhnutelný proces, který začal s příchodem řeckých kolonistů v 7. a 6. st. př. n. l., pokračoval v hellénistické době s aktivitami místní aristokracie a nabyl ještě větší intenzity pod římskou nadvládou. Zvyk vydávat nápisy byl v tehdejší společnosti natolik zakořeněn, že nevymizel ani s oslabením politického a kulturního vlivu řeckých obcí v 1. st. př. n. l. a 1. st. n. l., ale naopak se změnou politického uspořádání za římské nadvlády v následujících stoletích ještě zesílil. Tento jev bývá někdy souhrnně nazýván termínem římská hellénizace, tedy jakési zintenzivnění hellénizačního procesu za pomoci infrastruktury římské říše (Vranič 2014, 39).

\section[zhodnocení-vlivu-hellénizace-na-epigrafickou-produkci-v-thrákii]{Zhodnocení vlivu Hellénizace na epigrafickou produkci v Thrákii}

V této práci jsem na základě studia dochovaného epigrafického materiálu došla k závěru, že v dnešní době je teoretický koncept hellénizace do značné míry překonaný a jako interpretační rámec vysvětlující rozšíření epigrafické produkce v Thrákii je nedostačující. Hellénizační přístup totiž představuje jednostranně zaměřený model, zatížený množstvím předsudků, který nereflektuje pestrost mezikulturních kontaktů v celé jejich šíři, a nenabízí prostor pro aktivní zapojení thrácké populace v celém procesu kulturní změny. V duchu hellénizace je rozšíření epigrafické produkce v Thrákii interpretováno jako bezvýhradné přijetí zvyku publikovat nápisy místním obyvatelstvem se současným opuštěním vlastní identity a její postupné nahrazení identitou řeckou. Pokud by došlo k úplné a bezvýhradné hellénizaci, tak jak tento model navrhuje, všechny nápisy by byly produkovány Thráky, kteří by postupem času přijali i řeckou identitu a řecká jména a stali se tak v epigrafickém prostředí nerozlišitelnými od Řeků. Postupem času by thrácký prvek zcela vymizel nejen z onomastických záznamů, ale i z vyjádření příslušnosti k místním komunitám a projevům víry typických pro thrácké obyvatelstvo. Thrácká identita, spolu s jmény, by tak v pozdních fázích hellénizace neměla na nápisech vůbec figurovat. Univerzálním motivem rozšíření nápisů by byla přirozená touha po všem řeckém, po řecké kultuře a civilizaci. Neexistovaly by lokální varianty, místa s větší či menší mírou rezistence. Nápisy by byly rozmístěny rovnoměrně po celém území a zachovávaly by si uniformní charakter a formu. Jak je patrné z analyzovaného materiálu v jednotlivých časových obdobích a specifických regionech, tento uplatněný hellénizační model neodpovídá skutečné situaci. Naopak, prvky dokumentující existenci místní identity se vyskytují po celou dobu existence epigrafických památek na území Thrákie, a ve 2. a 3. st. n. l. dochází k intenzifikaci tohoto trendu. Řecká identita na epigrafických památkách tak nenahrazuje přináležitost k thrácké komunitě, ale jsme svědky adaptace nového společenského uspořádání a zvyklostí na všech zúčastněných stranách probíhající mezikulturní výměny. Jak dokazuje podrobná analýza nápisů, zvyk vytvářet a publikovat nápisy byl přijat různými vrstvami a skupinami společnosti odlišně, v různou dobu a s různorodými projevy, a proto je nutné jej jednoznačně interpretovat jako důsledek uniformního vlivu řecké kultury a civilizace.

Nápisy a písmo obecně v tehdejší společnosti sloužily jako prostředek vyjadřování a zaznamenávání informací, který se rozšířil z řeckých měst na pobřeží směrem do progresivních komunit ve vnitrozemí v klasické a hellénistické době, avšak k jeho rozšíření do prakticky všech komunit došlo až pod vlivem sjednocující autority římské říše. Pouhá kulturní dominance řeckého světa tak nemůže sloužit jako jediné vysvětlení epigrafické produkce v Thrákii, ale spíše je vhodné na rozšíření nápisů nahlížet jako na jeden z průvodních jevů rozvíjející se společensko-politické organizace a struktury.

\section[charakter-epigrafické-produkce-v-thrákii-v-předřímské-době]{Charakter epigrafické produkce v Thrákii v předřímské době}

Epigrafická kultura se v Thrákii v plné míře prosadila až v době římské, jako jeden z projevů rozvinuté komplexní společnosti a státní organizace. Do té doby můžeme sledovat dva zcela odlišné přístupy k nápisné kultuře, které do jisté míry vycházely z rozdílného kulturního pozadí pobřežních a vnitrozemských komunit. Na pobřeží Černého, Marmarského a Egejského moře byla epigrafická produkce ovlivněná přímo řeckou přítomností na tomto území a vykazovala rysy typické pro řeckou kulturu, zatímco obyvatelé thráckého vnitrozemí přistupovali k nápisné kultuře a písmu obecně velmi odlišně.

\subsection[pobřežní-komunity]{Pobřežní komunity}

V období od 7. do 1. st. př. n. l. se epigrafická produkce soustředila zejména v bezprostředním okolí řeckých měst na pobřeží Egejského, Černého a Marmarského moře a v menší míře v okolí řecky mluvících komunit ve vnitrozemí Thrákie v okolí řek Strýmón, Hebros a Tonzos. Většina nápisů z okolí těchto komunit byla soukromého charakteru a sloužila k označení místa pohřbu, v případě funerálních nápisů, a jako součást věnování božstvu v případě dedikačních nápisů. Charakter soukromých nápisů, dochovaná osobní jména a vyjádření identity poukazují na téměř výlučně řecký původ obyvatelstva, které se podílelo na epigrafické produkci. Thrácká osobní jména se na nápisech z pobřeží vyskytovala pouze ve 2 \letterpercent{}, což může poukazovat na velmi malou míru interakce a zapojování thrácké populace do chodu řeckých komunit, či se může jednat o výsledek nezájmu thráckého obyvatelstva podílet se na zvyklostech zhotovovat nápisy soukromého charakteru.

Veřejné nápisy z řecky mluvících komunit dokládají, že řecká města na pobřeží Thrákie byla organizována velmi podobně jako ve zbytku tehdejšího řeckého světa a zvyk vydávat veřejná nařízení a ustanovení byl v těchto komunitách plně rozvinutý. Veřejné nápisy taktéž dokazují existenci samosprávních institucí, stratifikaci společenských rolí a rozdělení moci, stejně tak fungující systém procedur a ustálených norem chování. Regionální varianty těchto prvků však poukazují na absenci sjednocující politické autority. Zásadní roli tedy hrála regionální centra jakožto autonomní politické jednotky, které vycházely z podobného kulturního pozadí, avšak jednaly v zájmů svých občanů.

\subsection[vnitrozemské-komunity]{Vnitrozemské komunity}

V případě thrácké komunity se v období od 7. do 1. st. př. n. l. jednalo o ojedinělé a omezené pokusy o zavedení řeckých epigrafických zvyklostí, typické pro komunity řízené autoritativními jedinci z řad aristokracie, jako v případě Seuthopole, či obchodní komunity se smíšeným obyvatelstvem, které však měly omezený vliv a trvání, jako v případě Pistiru či Hérakleie Sintské. Thráčtí aristokraté využívali písmo utilitárně, tedy k označení vlastnictví a účelu souvisejícího objektu či jako způsob komunikační strategie směrem k řecky mluvícím komunitám.

První skupina nápisů z vnitrozemí je spojena s aktivitami thrácké aristokracie a nachází se většinou na předmětech z drahých kovů či souvisejících s funerálním ritem. Thráčtí aristokraté nepřijali zvyk zhotovování a vystavování náhrobních stél tak, jak bylo běžné v řeckých komunitách na pobřeží, ale většina funerálních nápisů pocházela z interiéru hrobek, či z předmětů, které se staly součástí pohřební výbavy. Oproti řeckým komunitám byly tyto nápisy určeny jen nejbližšímu kruhu aristokratů, nebyly veřejně přístupné a funerální funkci získaly až druhotně. Tento fakt vypovídá jednak o odlišném společenském uspořádání thrácké a řecké společnosti, ale i o odlišném přístupu k písmu a zaznamenávání informací. Písmo zde nesloužilo k šíření informací mezi širokou veřejnost jako v případě nápisů z řeckých měst, ale bylo využíváno čistě pro potřeby úzkého okruhu aristokratů a sloužilo jako prostředek zvýšení společenského postavení.

Druhá skupina nápisů z thráckého vnitrozemí pochází z kontextů řeckých, makedonských či smíšených komunit, jako v případě řeckého {\em emporia} v Pistiru, makedonského {\em emporia} v Héraklei Sintské, hellénistické rezidence v Seuthopoli či hellénistického města v Kabylé. V těchto případech se užití písma podobá více zvyklostem zaznamenaných v řeckých komunitách na pobřeží. Vyskytují se zde nápisy na kamenných stélách, které byly veřejně vystavované a byly jak soukromého, tak veřejného charakteru. Navíc z těchto míst pochází i graffiti na keramice, dokazující určitý stupeň gramotnosti místního obyvatelstva. Osobní jména na nápisech jsou převážně řeckého původu či se jedná o thrácká jména pravděpodobně aristokratického původu. Charakter dochovaných nápisů naznačuje přejímání určitých zvyklostí typických pro řeckou kulturu a společenskou organizaci, avšak také naznačuje jejich adaptaci na místní podmínky. Epigrafická produkce tak v případě soukromých nápisů sloužila pro interní potřeby dané komunity a nedocházelo k rozšíření zvyklostí mimo bezprostřední sféru vlivu daného osídlení. V případě veřejných nápisů, které sloužily jako prostředek kodifikace diplomatických vztahů, je pak konkrétní forma ovlivněna zvolenou komunikační strategií všech zúčastněných stran a snahou nalézt společné vyjadřovací prostředky. Vzhledem k propracovanosti systému smluv a ustanovení v řeckém světě je pak zcela logické, že se zúčastněné strany rozhodly využít ustálenou formu a částečně i obsah těchto smluv. Z těchto důvodů tzv. pistirský či seuthopolský nápis vycházejí z řeckých vzorů, avšak reagují na konkrétní politickou situaci a tomu odpovídá i jejich unikátní obsah, který do značné míry zdůrazňuje autonomii a suverenitu thrácké aristokracie vůči řeckým komunitám.

Z podrobné analýzy dochovaných nápisů plyne, že epigrafická produkce v předřímské době pochází z řeckých komunit a ve velmi omezené míře i z kontextu thrácké aristokracie, která adoptovala prvky nápisné kultury pro své vlastní účely a dala jim unikátní charakter. Důsledkem kontaktu s řeckou kulturou bylo objevení nápisů v thráckém kontextu jakožto prostředku komunikace vně i uvnitř komunity, avšak za zachování tradičního charakteru thrácké společnosti a platných norem chování.

\section[charakter-epigrafické-produkce-v-thrákii-v-římské-době]{Charakter epigrafické produkce v Thrákii v římské době}

Epigrafická produkce pocházející z období římské nadvlády vykazuje oproti předcházejícímu období prudký nárůst, zejména v průběhu 2. a 3. st. n. l. Navíc se zcela stírají rozdíly mezi odlišným užitím nápisů na pobřeží a ve vnitrozemí, nápisy tak dostávají velmi podobný charakter a epigrafická produkce se z pobřeží přesouvá do okolí center městského charakteru ve vnitrozemí. Politický a kulturní vliv řeckých měst je však natolik oslaben, že tuto proměnu epigrafické kultury není možné spojovat s civilizační tendencí řecké kultury, známou pod pojmem hellénizace, či tzv. římská hellénizace.

Ze studia dochovaných nápisů vyplývá, že k hlavnímu rozvoji epigrafické produkce došlo nikoliv v souvislosti s řeckou přítomností v Thrákii, ale v přímé souvislosti s nárůstem společenské organizace a rozvinutím potřebné infrastruktury v době římské. V předřímské době nápisy pocházely převážně z řeckých měst na pobřeží, případně z ekonomických a kulturních center ve vnitrozemí. V době římské epigrafická produkce pocházela z přímého okolí městských center, která se v této době začala objevovat ve zvýšené míře i ve vnitrozemí, a dále v okolí římských silnic, které sloužily pro přesuny vojsk i civilního obyvatelstva a výraznou měrou přispěly k propojení vzdálených regionů a zintenzivnění kulturních kontaktů. Urbanizace thráckého vnitrozemí měla na projevy epigrafické produkce přímý vliv, stejně tak jako centralizace politické moci, jevy obecně spojené s růstem společenské komplexity. V této době zároveň došlo k rozdělení práce, zintenzivnění produkce a zajištění potřebné infrastruktury nutné k produkci nápisů ve velkém měřítku.

Jedním z hlavních důvodů nárůstu epigrafické produkce v římské době bylo zvýšené zapojení místních obyvatel do služeb římské armády, pozorovatelné již od 1. st. n. l. Veteráni, kteří se po dlouhé vojenské službě vraceli do Thrákie, s sebou přinesli nově získané kulturní zvyklosti související s jejich službou v armádě a pobytem na územích se zcela odlišnou kulturou. Je více než pravděpodobné, že za dobu služby vojáci získali alespoň základní stupeň gramotnosti a seznámili se se zvykem publikovat nápisy, což se po konci služby projevilo i ve změně přístupu k zhotovování nápisů. Na nápisech v římské době se taktéž objevuje větší zastoupení thráckých jmen než v době předřímské, což je přímý důsledek většího zapojení Thráků do epigrafické produkce. S větším zapojením Thráků souvisí i nárůst počtu dedikací věnovaných místním božstvům, zejména ve 2. a 3. st. n. l. Tento jev by mohl představovat nárůst uvědomění si thrácké identity, nicméně taktéž se může jednat o pouhý epigrafický záznam již existujícího trendu, který se podařilo zachytit právě díky většímu zapojení thrácké populace na epigrafické produkci. Tím, že se Thrákové více zapojovali do chodu římské říše, zejména službou v armádě a civilní správě, se jim dostalo náležitého vzdělání a tím více se následně mohli zapojovat i do produkce nápisů, čemuž odpovídá i větší zastoupení thráckého prvku na dochovaných nápisech.

\subsection[proměny-identity-a-vliv-řecké-kultury]{Proměny identity a vliv řecké kultury}

Jeden ze základních rysů hellénizace v podobě proměny původní identity osob a její postupné nahrazení řeckou identitou není na nápisech z Thrákie pozorovatelný. Naopak, po celou dobu dochází ke kladení důrazu na lokální identitu a kontinuitu tradičních hodnot. Zejména v římské době se osoby více ztotožňují s politickou autoritou na úrovni měst a vesnických samospráv, státních i místních institucí a v neposlední řadě i vojska.

Identifikace s thráckým etnikem na nápisech v předřímské době téměř neexistuje a v římské době se objevuje v souvislosti se systémem verbování a tvoření vojenských jednotek římské armády, ale soudě dle nápisů, v životě civilního obyvatele nehraje etnicita takřka žádnou roli{\bf .} Naopak identifikace a snaha zařazení osob do nejbližších komunit se vyskytuje jako jednotící prvek po celou dobu, a to zejména na nápisech soukromého charakteru, kde se setkáváme s uváděním biologického původu a odkazů na předky.

Osobní jména v předřímské době nevykazují velkou míru prolínání onomastických tradic a jednotlivé komunity spíše dodržují konzervativní charakter předávání osobní jmen z generace na generaci. Nicméně Thrákové přecházejí na systém identifikace s uváděním jmen rodičů, jak je zvykem v řecky mluvícím prostředí. V římské době dochází k určité proměně onomastických zvyklostí, kdy jsou k tradičním jménům řeckého či thráckého původu přidávána jména římská. Přijmutí tradičního římského onomastického systému poukazuje na dosažené společenského postavení nositele, jakožto symbolu získání římského občanství, jehož vlastnictví s sebou nese společenské výhody, ale nedokazuje proměnu identity. Naopak fakt, že se většinou římská jména vyskytují v kombinaci se jmény thráckého či řeckého původu spíše poukazuje na adaptaci obyvatel na nové společensko-politické podmínky za současného uchování původní identity.

Řečtina je i nadále využívána jako hlavní jazyk nápisů, ale spíše, než o projev hellénizace společnosti se jedná o projev převzetí fungujícího systému komunikace a zaznamenávání informací bez nutnosti změny jeho formy. Tomu nasvědčuje i velká konzervativnost epigrafického jazyka a obsahu nápisů, kde tradiční epigrafické formule zůstávají neměnné i po několik století. Nicméně proměňující se dekorace a provedení nosiče nápisu naznačují změnu vkusu a poptávky zhotovitelů. V římské době se objevuje více variant dekorace nosičů nápisů, které nicméně vycházejí z předřímských vzorů.

Nápisy z římské doby vykazují několik společných rysů, které se nevyskytovaly v předcházejícím období a které jsou pozorovatelné i mimo Thrákii na území ostatních římských provincií. Společným rysem funerálních nápisů z římské doby je tendence spolu s identifikací zemřelého uvádět i členy rodiny, či přátele, kteří nápis nechali zhotovit. V předřímské době je na nápisech uváděn pouze zemřelý či rodiče, případně partner. V římské době se tento okruh lidí značně rozšiřuje i na sourozence, vnuky a přátele, případně kolegy z armády. Je více než pravděpodobné, že se tak dělo z důvodů zajištění dědických práv a z povinností dědiců vycházejících z tehdy platných právních norem. Dalším prvkem, pozorovatelným v době římské na celém území Thrákie, ale i mimo něj, je snaha jedince prezentovat na nápisech dosažené společenské postavení a zapojení do institucionálních struktur tehdejší společnosti. Děje se tak vědomou prezentací na nápisech ve formě konkrétní podoby osobního jména, dále ve formě uvádění dosažených životních úspěchů a společenského postavení, ale i ve formě identifikace s politickou či náboženskou komunitou. Podobně se objevuje ve velké míře na nápisech z této doby i zvyk udávat věk zemřelého, což je zvyk typický pro nápisy římské doby. Tyto trendy nejsou typické pouze pro nápisy z Thrákie, ale vyskytují se téměř na celém území římské říše a naznačují ovlivnění podoby epigrafické produkce tehdy platnými společenskými normami a společenskou organizací, kterou představovala centrální autorita římské říše.

\section[zhodnocení-použité-metodologie]{Zhodnocení použité metodologie}

Použitá metodologie kvantitativní a kvalitativní studie epigrafické produkce v místě a čase může být aplikována i na jiná místa antického světa, kde docházelo k setkávání řecké, římské a místní kultury a není pevně vázána pouze na oblast Thrákie. Tato metodologie v sobě kombinuje tradiční přístup k epigrafickému materiálu s moderními teoretickými přístupy napříč příbuznými disciplínami. Syntetické práce srovnávající velké množství epigrafických dat na regionální úrovni nejsou příliš časté, a proto je použitá metodologie pokusem uchopit data získaná z nápisů se všemi jejich nedostatky, jako je například míra nejistoty časového zařazení či nejistota při určování provenience nápisu. Inspirací po metodologické stránce jsou podobné postupy užívané běžně v archeologii, které své uplatnění v historii a epigrafice teprve pomalu hledají. Použitá metodologie tak umožňuje zařadit velké množství nápisů do konkrétního kontextu a sledovat trendy celospolečenského charakteru a jejich vývoj v místě a čase. Podobné metody je možné použít nejen pro nápisy, ale i pro další součásti materiální kultury a případně navzájem tato data srovnávat.

Nespornou výhodou, kterou současná práce disponuje, je široký záběr moderních technologií, které umožňují nápisy analyzovat ve větším měřítku a zapojovat nové přístupy a metody, jako například statistické zhodnocení sledovaných jevů, srovnávání jednotlivých regionů či mapování změn společnosti na časové ose, tak i zapojení do kontextu krajiny a vztahu s lidským osídlením. Epigrafickou produkci je tak možno velmi srozumitelně prezentovat v sérii chronologických map, nebo se zaměřením na jednotlivé aspekty nápisů ve vztahu s dalšími lidskými aktivitami na území Thrákie.

Praktické využití pro spřízněné obory nabízí vytvořená elektronická databáze nápisů {\em Hellenization of Ancient Thrace}, která sjednocuje data z mnoha zdrojů a převádí je do jednotné koherentní formy{\em .} V rámci současných vědeckých principů jsou data z databáze v neupravené podobě k dispozici volně na internetu a případní zájemci je mohou použít v rámci vlastního výzkumu, či k ověření uváděných interpretací. Přímé praktické využití této databáze vidím v rámci jednotlivých archeologických projektů, které si tak mohou poměrně snadno opatřit data o epigrafických nálezech z blízkosti hledané lokality či regionu v již digitalizované podobě a připravené pro další analýzy. V takto kompletní podobě žádná jiná volně dostupná databáze neposkytuje data o místě nálezu nápisu, jeho obsahu a z něj vycházejících interpretacích, jako je datace, funkce a další relevantní informace.

\section[nastínění-dalšího-vývoje]{Nastínění dalšího vývoje}

V rámci studia proměn thrácké společnosti a epigrafické produkce v Thrákii se nabízí rozšíření výzkumu na latinsky psané nápisy z oblasti, případně rozšíření zkoumaného území na oblasti severně od Dunaje či do oblasti Bíthýnie. V případě zapojení latinských nápisů se nabízí srovnání přístupů společnosti ve vztahu ke zvolenému publikačnímu nápisu a konkrétní podobě a obsahu nápisů. Zajímavé by mohlo být i srovnání přístupu obyvatelstva dle etnicity či společenského postavení a volby publikačního jazyka. Zapojení nápisů z oblasti severně od Dunaje a z oblasti maloasijské Bíthýnie, tedy regionů do jisté míry taktéž obývaných thráckými kmeny, by pomohlo zhodnotit nakolik jsou si obyvatelé podobní, jednak co se týče přístupu k epigrafické kultuře v průběhu staletí, ale i nakolik se podobají složením společnosti a vývojem společenského uspořádání.

Velmi zajímavým rozšířením této práce by bylo srovnávat epigrafická data s archeologickými daty z téhož regionu. Pokud bude v budoucnosti dostupný zdroj koherentních archeologických dat, toto srovnání by do velké míry přispělo k naší znalosti o složení a proměnách společenského uspořádání a zvyklostí obyvatelstva antické Thrákie.

\section[shrnutí-26]{Shrnutí}

Podoba epigrafické produkce a její rozšíření v antické Thrákii souvisí spíše než s civilizačním vlivem řecké kultury s rozvojem tehdejšího společenského uspořádání. Většina nápisů je sice psána řecky a užitá terminologie taktéž pochází z řeckého prostředí, ale charakter epigrafické produkce nasvědčuje, že spíše než k nevědomému přebírání kulturních zvyklostí a transformaci thrácké společnosti směrem ke společnosti řecké docházelo k adopci písma a epigrafických zvyklostí jako propracovaného systému vyjadřování za uchování kulturní integrity a identity jednotlivých komunit. V návaznosti na rozvoj politické organizace v době římské dochází k intenzifikaci epigrafické produkce, a to především v souvislosti s vytvořením potřebné infrastruktury a s tím souvisejícím větším zapojením obyvatelstva.
