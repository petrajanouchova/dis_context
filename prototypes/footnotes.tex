\environment env_dis.tex
\usepath[{./}]

\setuppagenumbering[state=start,alternative=singlesided,location={footer,right},style={\ss\tfx},way=bytext]

\starttext

V reakci na politický vývoj 20. století a snahu o vyrovnání se s rozpadem evropských koloniálních velmocí se i v akademickém světě objevila celá řada nových teorií a směrů\index{postkolonialismus}. Viděno novou optikou, dříve aplikované akulturační teorie nevystihovaly komplexnost studií zabývajících se mezikulturními vztahy, a tak bylo nutné hledat nové alternativní přístupy.\footnote{Postkoloniální teoretické přístupy jsou charakterizovány svou vzájemnou provázaností a mnohostranností. Není tedy možné určit jeden převládající teoretický směr, ale jedná se spíše o amalgám několika přístupů, jehož složení se liší dle konkrétního badatele a jeho záměru. Následující výčet charakterizuje pouze nejvýznamnější z hlavních směrů současnosti. Odbornou literaturou poslední doby rezonuje proces formování místní identity jako reakce na vliv nadregionálních uskupení, jako je římské impérium (Mattingly 2010). Podílem římské říše na globalizaci společnosti a následnými projevy na místní kulturu se zabývá směr taktéž známý jako tzv. {\em glokalizace} (Pitts 2008). Dále sem patří směry zabývající se prolínání kultur a vytváření nových forem mezikomunitní komunikace a společenského uspořádání jako je hybridizace, kreolizace (Liebmann 2013) či {\em middle-ground theory} (White 1991; Antonaccio 2013). Formováním nových identit na základě kontaktů s jinými kulturami a reprezentací nově vzniklých identit v materiální kultuře se zabývá celá řada badatelů (Hodos 2006; 2010), stejně tak kontextualizací společenských role v rámci původní a nové kultury (Appadurai 1986; Dietler 1997).} Celková nespokojenost s teoretickými přístupy, které se snažily vysvětlit historické procesy jedním možným modelem, vyústila v rozvoj několika paralelních, navzájem se prolínajících teoretických směrů, jejichž zastánci se ve velké míře inspirovali v současné antropologické teorii, sociologii a v exaktních vědách. Hlavní charakteristikou těchto směrů je důraz, který kladou na multidimenzionalitu a variabilitu mezikulturních kontaktů. Více prostoru badatelé věnují národům a společnostem, které byly dříve vnímány pouze v dualistické opozici kolonizátor - kolonizovaný, a jejich různým reakcím na kontakty s jinými kulturami (Dommelen 1998, 25-26; Silliman 2013, 495).\footnote{Silliman (2013, 495) popisuje jeden z postkoloniálních směrů, nicméně jeho definice může být použita jako definice celého teoretického směru: „{\em Hybridity in a postcolonial sense tends to be a direct critique of previous versions of colonial theory that considered the effects of colonialism on indigenous people to be those of assimilation, acculturation, or even the more neutrally termed culture change. Hybridity offers a counterclaim of cultural creativity and agency, and it lends more subversion, nuance, and ambiguity than traditional assessments of the effects of colonialism.}”} Vznikly tak zcela nové a unikátní modely a přístupy k archeologickému materiálu, zaměřující se jednak na dynamiku celého procesu, ale i na důsledky kulturních interakcí všech zúčastněných stran. Dřívější čistě ekonomická vysvětlení, či motivy kulturní převahy nejsou zcela zavrženy, ale figurují pouze jako jedno z možných paralelních vysvětlení.


\stoptext