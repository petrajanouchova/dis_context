\environment env_dis

\usepath[{/home/petra/Github/dis_context/thesisBySection/Apendix2}]
\startcomponent Apendix2
\startchapter[title={Apendix 2 - Mapy a grafy}, reference={Apendix2}, marking={Apendix 2}]


\setuplocalinterlinespace[line=2.0ex]

\placefigure[here][1.01a]{Přibližné území antické Thrákie dle literárních pramenů, geografický reliéf, nejvýznamnější řeky a antická města.}{\externalfigure[1.01a_ThraceOverview.png][width=\textwidth,frame=on]}

\placefigure[here][4.01a]{Ukázka identifikačních záznamů v databázi Hellenization of Ancient Thrace, vytvořené v softwaru Heurist Scholar. Nápis SEG 49:992 = 100405.}{\externalfigure[4.01a_Identification_100405.png][width=0.8\textwidth,frame=on]}

\placefigure[here][4.02a]{Ukázka geografického záznamu v Heurist Digitizer, nápis SEG 49:992 = 100405.}{\externalfigure[4.02a_Heurist_Digitizer_100405.png][width=0.6\textwidth,frame=on]}

\placefigure[here][4.03a]{Ukázka záznamu geografických dat v databázi Hellenization of Ancient Thrace, vytvořené v softwaru Heurist Scholar. Nápis IG Bulg 3,2 1612 = 51612.}{\externalfigure[4.03a_Geography_51612.png][width=0.6\textwidth,frame=on]}

\placefigure[here][4.04a]{Ukázka záznamu dat o nosiči nápisu v databázi Hellenization of Ancient Thrace, vytvořené v softwaru Heurist Scholar. Nápis Gyuzelev 2002 3 = 100105.}{\externalfigure[4.04a_Object_metadata_100105.png][width=0.5\textwidth,frame=on]}

\placefigure[here][4.05a]{Dochování nosiče nápisu spolu s ilustrativními příklady jednotlivých kategorií.}{\externalfigure[4.05a_Preservation.pdf][width=\textwidth,frame=off]}

\placefigure[here][4.06a]{Architektonická dekorace nosičů nápisů spolu s ilustrativními příklady jednotlivých kategorií.}{\externalfigure[4.06a_ArchitecturalDecor.pdf][width=\textwidth,frame=off]}

\placefigure[here][4.07a]{Figurální dekorace nosičů nápisů spolu s ilustrativními příklady jednotlivých kategorií.}{\externalfigure[4.07a_FiguralDecor.pdf][width=\textwidth,frame=off]}

\placefigure[here][4.08a]{Poměrné zastoupení délky intervalů datací všech 2276 datovaných nápisů.}{\externalfigure[4.08a_HAT_Delka_Intervalu_BW.png][width=0.95\textwidth,frame=on]}

\placefigure[here][4.09a]{Poměr datovaných nápisů v jednotlivých stoletích s přihlédnutím k šíři jejich datace (vyjádřenou pomocí jejich koeficientů 1 - 0,12), celkem 2276 nápisů.}{\externalfigure[4.09a_HAT_Sire_Datace_procentualni_new.png][width=0.95\textwidth,frame=on]}

\placefigure[here][4.10a]{Celková epigrafická produkce v závislosti na použití metody datace nápisů. Srovnání normalizované metody datace spolu s nenormalizovanou metodou datace nápisů.}{\externalfigure[4.10a_MethodDating.png][width=0.95\textwidth,frame=on]}

\placefigure[here][5.01a]{Mapa rozmístění nálezových míst nápisů obsažených v databázi {\em Hellenization of Ancient Thrace}. Oranžová barva ohraničuje oblast pokrývající území, z nějž pocházejí nápisy. Bílá místa vyznačují oblast, z níž nejsou dostupná epigrafická data.}{\externalfigure[5.01a_HAT_all_inscriptions_CZ.png][width=0.95\textwidth,frame=on]}

\placefigure[here][6.01a]{Mapa nálezových míst nápisů datovaných do 6. a do 6. až 5. st. př. n. l.}{\externalfigure[6.01a_HAT_6thcBC_CZ.png][width=0.95\textwidth,frame=on]}

\placefigure[here][6.02a]{Mapa nálezových míst nápisů datovaných do 5. a do 5. až 4. st. př. n. l.}{\externalfigure[6.02a_HAT_5thcBC_CZ.png][width=0.95\textwidth,frame=on]}

\placefigure[here][6.03a]{Mapa nálezových míst nápisů datovaných do 4. a do 4. až 3. st. př. n. l.}{\externalfigure[6.03a_HAT_4thcBC_CZ.png][width=0.95\textwidth,frame=on]}

\placefigure[here][6.04a]{Mapa nálezových míst nápisů datovaných do 3. a do 3. až 2. st. př. n. l.}{\externalfigure[6.04a_HAT_3rdcBC_CZ.png][width=0.95\textwidth,frame=on]}

\placefigure[here][6.05a]{Mapa nálezových míst nápisů datovaných do 2. a do 2. až 1. st. př. n. l.}{\externalfigure[6.05a_HAT_2ndcBC_CZ.png][width=0.95\textwidth,frame=on]}

\placefigure[here][6.06a]{Mapa nálezových míst nápisů datovaných do 1. a do 1. st. př. n. l. až 1. st. n. l.}{\externalfigure[6.06a_HAT_1stcBC_CZ.png][width=0.95\textwidth,frame=on]}

\placefigure[here][6.07a]{Mapa nálezových míst nápisů datovaných do 1. st. n. l. a do 1. až 2. st. n. l.}{\externalfigure[6.07a_HAT_1stcAD_CZ.png][width=0.95\textwidth,frame=on]}

\placefigure[here][6.08a]{Mapa nálezových míst nápisů datovaných do 2. st. n. l. a do 2. až 3. st. n. l.}{\externalfigure[6.08a_HAT_2ndcAD_CZ.png][width=0.95\textwidth,frame=on]}

\placefigure[here][6.09a]{Mapa nálezových míst nápisů datovaných do 3. st. n. l. a do 3. až 4. st. n. l.}{\externalfigure[6.09a_HAT_3rdcAD_CZ.png][width=0.95\textwidth,frame=on]}

\placefigure[here][6.10a]{Mapa nálezových míst nápisů datovaných do 4. st. n. l. a do 4. až 5. st. n. l.}{\externalfigure[6.10a_HAT_4thcAD_CZ.png][width=0.95\textwidth,frame=on]}

\placefigure[here][6.11a]{Mapa nálezových míst nápisů datovaných do 5. st. n. l. a do 5. až 8. st. n. l.}{\externalfigure[6.11a_HAT_5thcAD_CZ.png][width=0.95\textwidth,frame=on]}

\placefigure[here][6.12a]{Přehled epigrafické produkce v jednotlivých stoletích dle šíře datace nápisů (nápisy datované do jednoho vs. dvou století).}{\externalfigure[6.12a_Prehled_produkce_stoleti_koeficienty.png][width=0.95\textwidth,frame=on]}

\placefigure[here][6.13a]{Srovnání epigrafické produkce v jednotlivých stoletích v závislosti na typologii nápisu.}{\externalfigure[6.13a_Epi_produkce_typologie.png][width=0.9\textwidth,frame=on]}

\placefigure[here][6.14a]{Srovnání epigrafické produkce soukromých nápisů v jednotlivých stoletích v závislosti na šíři datace.}{\externalfigure[6.14a_Epi_produkce_soukrome.png][width=0.9\textwidth,frame=on]}

\placefigure[here][7.01a]{Rozmístění nálezových míst nápisů v závislosti na jejich nadmořské výšce (DEM - digital elevation model).}{\externalfigure[7.01a_HAT_all_DEM.png][width=0.9\textwidth,frame=on]}

\placefigure[here][7.02a]{Rozmístění nálezových míst všech nápisů a jejich vzdálenost od městských center (do 20 km, do 40 km od daného centra).}{\externalfigure[7.02a_HAT_all_buffer_cities.png][width=0.9\textwidth,frame=on]}

\placefigure[here][7.03a]{Rozmístění nálezových míst všech nápisů a jejich vzdálenost od římských cest - silnic (do 10 km od dané silnice).}{\externalfigure[7.03a_HAT_all_buffer_roads_names.png][width=0.9\textwidth,frame=on]}

\placefigure[here][7.04a]{Hustota výskytu nálezových míst nápisů a jejich relativní poloha vůči městským centrům a římským cestám - silnicím.}{\externalfigure[7.04a_HeatMap_20km_NEW.png][width=0.9\textwidth,frame=on]}

\placefigure[here][7.05a]{Mapa hlavních produkčních center nápisů dle celkového počtu nalezených nápisů v okruhu do 20 km od daného produkčního centra městského typu.}{\externalfigure[7.05a_Productioncenters_20kmbuffer_scale.png][width=0.9\textwidth,frame=on]}

\placefigure[here][7.06a]{Mapa hlavních produkčních center nápisů dle celkového počtu nalezených nápisů v okruhu do 20 km od daného produkčního centra jiného než městského typu.}{\externalfigure[7.06a_NonUrban_Productioncenters_5kmbuffer_scale.png][width=0.9\textwidth,frame=on]}

\placefigure[here][7.07a]{Rozmístění nálezových míst veřejných nápisů a jejich vzdálenost od městských center (do 20 km, do 40 km od daného centra).}{\externalfigure[7.07a_Cities_buffer_public_inscriptions_NEW.png][width=0.9\textwidth,frame=on]}

\placefigure[here][7.08a]{Rozmístění nálezových míst datovaných milníků a jejich vzdálenost od městských center (do 20 km od daného centra).}{\externalfigure[7.08a_Milestones_date_20km_name.png][width=0.9\textwidth,frame=on]}

\placefigure[here][7.09a]{Rozmístění nálezových míst soukromých nápisů a jejich vzdálenost od městských center (do 20 km, do 40 km od daného centra).}{\externalfigure[7.09a_Cities_buffer_private_inscriptions_NEW.png][width=0.9\textwidth,frame=on]}

\placefigure[here][7.10a]{Rozmístění nálezových míst funerálních nápisů a jejich vzdálenost od městských center (do 20 km od daného městského centra a do 10 km od římských cest - silnic).}{\externalfigure[7.10a_Funerary_citiesroads.png][width=0.9\textwidth,frame=on]}

\placefigure[here][7.11a]{Hustota výskytu nálezových míst funerálních nápisů a jejich relativní poloha vůči městským centrům a římským cestám - silnicím. }{\externalfigure[7.11a_Funerary_heatmap20km.png][width=0.9\textwidth,frame=on]}

\placefigure[here][7.12a]{Rozmístění nálezových míst dedikačních nápisů v závislosti na jejich nadmořské výšce (DEM - digital elevation model).}{\externalfigure[7.12a_Dedications_elevations_dem.png][width=0.9\textwidth,frame=on]}

\stopchapter
\stopcomponent
