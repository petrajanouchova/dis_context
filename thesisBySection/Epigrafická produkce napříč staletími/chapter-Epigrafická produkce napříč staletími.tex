
\environment ../env_dis
\startcomponent chapter-Epigrafická produkce napříč staletími
\chapter{Epigrafická produkce napříč staletími}
V této kapitole se věnuji podrobné charakteristice epigrafické produkce v Thrákii v rámci jednotlivých staletí. Zaměřuji se především na rozšíření epigrafické produkce a její proměny v čase. Důraz kladu na projevy celospolečenských trendů a změn zvyklostí v epigraficky aktivních komunitách. Specificky mě zajímá rozšíření epigrafického zvyku v prostředí urbánních a venkovských komunit, dále těchto komunit, společenská hierarchie a šíření kulturních prvků typických pro řecky mluvící komunity v rámci osídlení thráckého pobřeží i thráckého vnitrozemí a v neposlední řadě funkce, jakou nápisy zastávaly v dané komunitě. Předmětem chronologické analýzy je celkem 2036 nápisů, u nichž je možné zařadit dobu jejich vzniku do konkrétního časového bodu či intervalu.\footnote{Celkový soubor 2036 nápisů s koeficientem 1 a 0,5 tvoří zhruba 89,5 \letterpercent{} všech 2276 datovaných nápisů (nenormalizovaná čísla), což představuje statisticky signifikantní vzorek nápisů nesoucích patřičnou míru výpovědní hodnoty pro dané století. Jako součást chronologické studie pracuji jen s nápisy datovanými s přesností do jednoho století s koeficientem 1 (1291 nápisů) a s nápisy datovaných do dvou po sobě následujících století, tedy nápisy s koeficientem 0,5 (745 nápisů). Metoda výběru nápisů je podrobně vysvětlena v kapitole 4.} Dataci nápisů přebírám tak, jak ji stanovili editoři v jednotlivých korpusech.\footnote{Seznamy nápisů přináležející do daného souboru jednoho či dvou století jsou součástí Apendixu 3.}

\stopcomponent