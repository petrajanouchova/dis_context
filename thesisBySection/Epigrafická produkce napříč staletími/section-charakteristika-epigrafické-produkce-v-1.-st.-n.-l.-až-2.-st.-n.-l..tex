
\environment ../env_dis
\startcomponent section-charakteristika-epigrafické-produkce-v-1.-st.-n.-l.-až-2.-st.-n.-l.
\section[charakteristika-epigrafické-produkce-v-1.-st.-n.-l.-až-2.-st.-n.-l.]{Charakteristika epigrafické produkce v 1. st. n. l. až 2. st. n. l.}

Nápisy datované do 1. a 2. st. n. l. pocházejí i nadále z pobřežních oblastí, s hlavním produkčním centrem v Perinthu. Soukromé nápisy funerálního typu i stále převažují, prodlužuje se průměrná délka nápisu a stejně tak i celkový poměr přítomnosti římských jmen v epigraficky aktivních komunitách. Osoby nesoucí thrácká jména se dostávají do izolace, nedochází k jejich kombinaci s řeckými či římskými jmény v rámci jednoho nápisu.

\startDelimitedTable
{\em Celkem}~ 94 nápisů

{\em Region měst na pobřeží}~ Abdéra 2, Anchialos 1, Apollónia Pontská 1, Bizóné 1, Byzantion 6, Lýsimacheia 1, Maróneia 12, Mesámbriá 3, Odéssos 1, Perinthos (Hérakleia) 53, Sélymbria 1, Topeiros 1, Zóné 1 (celkem 84 nápisů)

{\em Region měst ve vnitrozemí}~ Augusta Traiana 2, Hérakleia Sintská 1, Filippopolis 2, údolí středního toku řeky Strýmónu 3 (celkem 8 nápisů)\footnote{Celkem dva nápisy nebyly nalezeny v rámci regionu známých měst, editoři korpusů udávají jejich polohu vzhledem k nejbližšímu modernímu sídlišti.}

{\em Celkový počet individuálních lokalit}~ 26

{\em Archeologický kontext nálezu}~ funerální 1, sídelní 2, náboženský 2, sekundární 6, neznámý 83

{\em Materiál}~ kámen 89 (mramor 81, jiný 1, neznámý 7), kov 2, neznámý 3

{\em Dochování nosiče}~ 100 \letterpercent{} 16, 75 \letterpercent{} 3, 50 \letterpercent{} 13, 25 \letterpercent{} 13, kresba 5, nemožno určit 43

{\em Objekt}~ stéla 51, architektonický prvek 17, socha 1, nádoba 1, jiný 19 (z toho sarkofág 16), neznámý 2

{\em Dekorace}~ reliéf 43, bez dekorace 51; reliéfní dekorace figurální 20 nápisů (vyskytující se motiv: jezdec 3, stojící osoba 4, skupina lidí 1, zvíře 3, obětní scéna 1, Héraklés 1, Artemis 1, tři Nymfy 2, funerální scéna/symposion 1, funerální portrét 1, jiný 2), architektonické prvky 31 nápisů (vyskytující se motiv: naiskos 11, sloup 1, báze sloupu či oltář 7, architektonický tvar/forma 7, geometrický motiv 2, florální motiv 12, věnec 1, jiný 2)

{\em Typologie nápisu}~ soukromé 73, veřejné 15, neurčitelné 6

{\em Soukromé nápisy}~ funerální 57, dedikační 15, jiný 1

{\em Veřejné nápisy}~ seznamy 1, honorifikační dekrety 4, státní dekrety 1, funerální na náklady obce 2, náboženský 3, jiný 4 (z toho veřejné dedikace 3)

{\em Délka}~ aritm. průměr 6,05 řádku, medián 5, max. délka 18, min. délka 1

{\em Obsah}~ dórský dialekt 1, latinský text 6 nápisů, písmo římského typu 14; hledané termíny (administrativní termíny 17 - celkem 44 výskytů, epigrafické formule 18 - 109 výskytů, honorifikační 2 - 2 výskytů, náboženské 20 - 35 výskytů, epiteton 5 - počet výskytů 5)

{\em Identita}~ řecká božstva 10, egyptská božstva 1, pojmenování míst a funkcí typických pro řecké náboženské prostředí, místní thrácká božstva, regionální epiteton 3, subregionální epiteton 2, kolektivní identita 7 termínů, celkem 7 výskytů - obyvatelé řeckých obcí z oblasti Thrákie 7, mimo ni 0; celkem 132 osob na nápisech, 40 nápisů s jednou osobou; max. 8 osob na nápis, aritm. průměr 1,4 osoby na nápis, medián 1; komunita multikulturního charakteru se zastoupením řeckého, římského a thráckého prvku, jména pouze řecká (23,4 \letterpercent{}), pouze thrácká (2,12 \letterpercent{}), pouze římská (18,08 \letterpercent{}), kombinace řeckého a thráckého (0 \letterpercent{}), kombinace řeckého a římského (22,34 \letterpercent{}), kombinace thráckého a římského (0 \letterpercent{}), kombinovaná řecká, thrácká a římská jména (5,31 \letterpercent{}), jména nejistého původu (9,55 \letterpercent{}), beze jména (19,14 \letterpercent{}); geografická jména z oblasti Thrákie 2, geografická jména mimo Thrákii 4;
\stopDelimitedTable


Nápisů datovaných do období 1. až 2. st. n. l. je přibližně o 67 \letterpercent{} víc než v předcházejícím období. Většina nápisů pochází z oblasti na pobřeží, z okolí původně řeckých měst. Nápisy ve vnitrozemí se objevují podél cest, které jsou pravidelně udržovány a spravovány od druhé poloviny 1. st. n. l., a zejména v okolí rozvíjejících se regionálních urbánních center, jako je Filippopolis či Kabylé, a dále v oblasti středního toku řeky Strýmónu, jak ilustruje mapa 6.07a v Apendixu 2. Největším epigrafickým producentem je od druhé poloviny 1. st. n. l. Perinthos (Hérakleia) odkud pochází 56 \letterpercent{} celkové produkce.\footnote{Nárůst epigrafické produkce Perinthu souvisí s faktem, že se v polovině 1. st. n. l. stalo hlavním městem nově vytvořené provincie {\em Thracia} a je logické, že se zde soustředila většina produkce.} Pozici středního producenta si i nadále udržuje Maróneia, zatímco Byzantion, které hrálo roli největšího epigrafického producenta v předchozích dvou stoletích, nyní produkuje zhruba pouze 6 \letterpercent{} nápisů.

Materiálem nosiče nápisů je z 94 \letterpercent{} kámen, z převážné části je to mramor, dále vápenec a místní zdroj kamene. Nápisy na kovových předmětech se dochovaly pouze dva.\footnote{Do skupiny nápisů datovaných do 1. až 2. st. n. l. patří vojenský diplom AE 2007, 1260 psaný na bronzové destičce, který uděluje římské občanství, právo nosit římské jméno, pořídit si manželku a právo na vlastní pozemek. Text je psaný latinsky a pochází z okolí vesnice Trapoklovo v jihovýchodním Bulharsku. Bohužel text je velmi poškozený, a tudíž není možné dělat žádné další závěry. Nicméně se jedná o jeden z prvních projevů udělení římského občanství vojákovi za jeho dlouholeté služby, který byl nalezen přímo na území Thrákie. Zároveň se jedná o další důkaz narůstajícího vlivu římské armády nejen na proměňující se složení obyvatelstva, ale zejména na měnící se onomastické zvyky v důsledku zvýšené přítomnosti veteránů na území Thrákie (Dana 2013, 239-264).} Nejoblíbenější formou kamenného nápisu je i nadále stéla, nicméně populární jsou i funerální nápisy na sarkofázích, pocházejících v 15 případech z Perinthu (Hérakleii) a v jednom z Byzantia.

\subsection[funerální-nápisy-12]{Funerální nápisy}

Celkem 59 nápisů je interpretováno jako funerální\footnote{57 nápisů je soukromého charakteru a dva jsou veřejného charakteru.}, což představuje zhruba dvě třetiny celého souboru, přibližně stejně jako v 1. st. př. n. l. Většina nápisů je zhotovena na stéle, ale spadá sem i skupina 16 sarkofágů, které nesly nápis, a zároveň sloužily jako úložiště tělesných pozůstatků zemřelého. Hlavní produkční centra funerálních nápisů jsou Perinthos s 42 nápisy, Byzantion a Maróneia s pěti nápisy.

Sarkofágy z Perinthu užívají téměř totožné formule a termíny a na většině z nich se vyskytuje formule o ochraně sarkofágu a pozůstatků zemřelého politickou autoritou města, jako ve skupině nápisů z 1. st. př. n. l. Osobní jména na sarkofázích nesou v polovině případů římská jména a ve zbývající polovině jména řecká. Thrácká jména se na sarkofázích vůbec nevyskytují, což může poukazovat na kulturní provázanost tohoto zvyku s komunitami římského, případně řeckého původu.

Převaha funerálních nápisů stále pochází z řeckých komunit, které si i nadále udržují relativně konzervativní charakter. Nejčastěji se na nápisech objevují řecká jména, zhruba v polovině příkladů. Dále ve 40 \letterpercent{} jména římská a zbylých 10 \letterpercent{} připadá jménům thráckým a jménům nejistého původu. Thrácká jména se objevují pouze na dvou nápisech, a to jak samotná, tak v kombinaci s řeckým jménem, výhradně v souvislosti s funerálním nápisem zemřelého zastávajícího funkci stratéga.\footnote{Nápis Manov 2008 199 a nápis {\em I Aeg Thrace} 87.} Geografický původ uvádí pouze jeden nápis, odkazující na bíthýnskou Nikomédii.\footnote{{\em Perinthos-Herakleia} 144.}

Obsah nápisů si udržoval tradiční formu, nicméně se vyskytují termíny popisují nově zavedené součásti funerálního ritu, jako sarkofág, ale objevují se i termíny pro nově vzniklá povolání a funkce.\footnote{Invokační formule oslovující okolo jdoucího čtenáře {\em (chaire}) se objevily celkem 26krát, ({\em parodeita}) 21krát. Funerální nápisy sloužily ve většině případů pro rodinné pohřby: celkem 24krát se objevuje text zhotovený jedním z partnerů pro sebe a pro manžela či manželku, či jiného člena nejbližší rodiny. Pro popis hrobu samotného sloužily termíny {\em mnémeion} jednou, {\em latomeion} v devíti případech a {\em soros} v 11 případech, popisující sarkofág, dále {\em stéla} se sedmi výskyty jako označení hrobu a {\em bómos} se dvěma výskyty a po jednom výskytu {\em tymbos} a {\em séma} označují hrob samotný.} Text nápisů nese více detailů se života zemřelého: setkáváme se s šesti vojáky různých hodností, jako například legionář, jezdec, {\em stratégos} či {\em pragmatikos}. Dále se setkáváme se stavitelem domů ({\em domotektón}), ale i notářem ({\em notários}). V 17 případech se dozvídáme věk, kterého se zemřelý dožil, obyvkle zaokrouhlený na pět let. Jedná se o typický římský kulturní zvyk, který se na území Thrákie vyskytl již v 1. st. n. l. ve velmi omezeném počtu, nyní však u celé třetiny funerálních nápisů.

\subsection[dedikační-nápisy-12]{Dedikační nápisy}

Dedikačních nápisů se dochovalo celkem 19, z čehož 16 je soukromé povahy a tři jsou na pomezí soukromého a veřejného nápisu, což oproti minulému období představuje dvojnásobný nárůst. Polovina nápisů pochází z Perinthu a na dalších místech na pobřeží a ve vnitrozemí se vyskytují nápisy od jednoho do tří exemplářů. Nápisy jsou zhotoveny převážně z kamene, jen v jednom případě se jedná o zlatý amulet z Perinthu.

Věnování jsou určena jak řeckým, tak lokálním božstvům, která částečně využívají řecká jména v kombinaci s místním epitetem. Dedikace jsou určeny Apollónovi {\em Iatrovi}, Asklépiovi {\em Zylmyzdriénovi}, Diovi s přízviskem {\em Lofeités}, Sarápidovi, Héře a anonymním božstvům. Osobní jména dedikantů poukazují na smíšené složení věřících. U nápisů věnovaných božstvům s místním (subregionálním) epitetem se vyskytují thrácká jména ve vyšším poměru než u ostatních dedikací, což naznačuje, že lokální kulty byly oblíbené především u místního thráckého obyvatelstva. Přítomnost jmen smíšených řecko-thráckých a římsko-řeckých ukazuje, že lokální kulty byly přístupné i lidem mimo thráckou komunitu.

\subsection[veřejné-nápisy-12]{Veřejné nápisy}

Celkem se dochovalo 15 veřejných nápisů: čtyři z Maróneie a Perinthu a po jednom nápise z ostatních měst na pobřeží. Politickou autoritu představují instituce řeckých {\em poleis}, jako {\em búlé} a {\em démos} v případě Maróneie, a {\em démos} v případě Odéssu a Perinthu, a dále římský císař a vysoce postavení úředníci římské říše. Thráčtí králové se na veřejných nápisech nevyskytují, stejně tak jako stratégové. Nápisy mají nejčastěji povahu honorifikačních dekretů či seznamů věřících a textů náboženské povahy. Tradiční forma honorifikačních dekretů tak, jak ji známe např. z 3. st. př. n. l., se nedochovala, z čehož se dá usuzovat, že došlo k proměně celé procedury udílení poct, a tedy i k proměně podoby honorifikačních textů, podobně jako např. v Malé Asii (Van Nijf 2015, 240).\footnote{Bohužel dochovaný soubor honorifikačních nápisů je natolik fragmentární, že není možné blíže popsat jednotlivé změny.}

\subsection[shrnutí-16]{Shrnutí}

S větší participací thráckého prvku na epigrafické produkci může souviset zapojení Thráků v provinciální administrativě a v římské armádě, což s sebou neslo jednak přijetí nových zvyků a jednak i nárůst gramotnosti, a tedy i změny přístupu k publikaci nápisů. Postupné proměny onomastických záznamů svědčí o aktivním zapojení thráckých mužů v římské armádě a následné proměny onomastických zvyků u této části populace. S tím může nepřímo i souviset i mírný nárůst počtu místních kultů, které však zůstávají otevřené všem bez rozdílu původu. Zvyky civilní části obyvatelstva zůstávají podobné, jako v období před vznikem provincie {\em Thracia}, což svědčí o plynulém přechodu bez znatelného kulturního zlomu.

\stopcomponent