
\environment ../env_dis
\startcomponent section-charakteristika-epigrafické-produkce-v-1.-st.-n.-l.
\section[charakteristika-epigrafické-produkce-v-1.-st.-n.-l.]{Charakteristika epigrafické produkce v 1. st. n. l.}

Většina nápisů z 1. st. n. l. i nadále pochází z pobřežních oblastí, kde však dochází k proměně komunit směrem k větší otevřenosti a multikulturalitě a narůstající roli institucionální epigrafické produkce. Celková délka nápisů se prodlužuje, stejně tak se zvyšuje i počet osob vystupujících na nápisech. Římský kulturní prvek se začíná prosazovat jak na poli proměňujících se osobních jmen, tak i přítomností latinsky psaných nápisů či jejich částí.


\startCSV
{\em Celkem}~ 68 nápisů

{\em Region měst na pobřeží}~ Abdéra 2, Abritus 1, Anchialos 1, Apollónia Pontská 1, Byzantion 15, Ferai 1, Koila 1, Madytos 1, Maróneia 13, Mesámbria 1, Odéssos 2, Perinthos (Hérakleia) 13, Topeiros 3 \cite[celkem55]

{\em Region měst ve vnitrozemí}~ Filippopolis 4, Neiné 1, Nicopolis ad Nestum 1, Serdica 1, údolí středního toku řeky Strýmón 1 \cite[celkem8]\footnote{Celkem čtyři nápisy nebyly nalezeny v rámci regionu známých měst, editoři korpusů udávají jejich polohu vzhledem k nejbližšímu modernímu sídlišti či muzeu, kde se v současnosti nacházejí. Jeden nápis byl nalezen mimo území Thrákie, avšak editoři uvádějí jeho původ jako thrácký na základě širšího kontextu.}

{\em Celkový počet individuálních lokalit}~ 27

{\em Archeologický kontext nálezu}~ sídelní 5, náboženský 3, sekundární 8, neznámý 52

{\em Materiál}~ kámen 64 (mramor 55, z toho z Thasu 1, z okolí Maróneii 2; vápenec 1; jiný 4, z toho póros 1; neznámý 4), kov kombinovaný s kamenem 1, neznámý 3

{\em Dochování nosiče}~ 100 \letterpercent{} 6, 75 \letterpercent{} 7, 50 \letterpercent{} 7, 25 \letterpercent{} 10, oklepek 2, kresba 2, ztracený 1, nemožno určit 33

{\em Objekt}~ stéla 46, architektonický prvek 13, socha 1, jiný 2, neznámý 6, \cite[celkem3]

{\em Dekorace}~ reliéf 25, malovaná 1, bez dekorace 42; reliéfní dekorace figurální 13 nápisů (vyskytující se motiv: jezdec 1, stojící osoba 2, lovecká scéna 1, funerální scéna/symposion 2, obětní scéna 2, jiný 1), architektonické prvky 11 nápisů (vyskytující se motiv: naiskos 3, sloup 2, báze sloupu či oltář 4, architektonický tvar/forma 4, florální motiv 2, jiný 2)

{\em Typologie nápisu}~ soukromé 37, veřejné 28, neurčitelné 3

{\em Soukromé nápisy}~ funerální 24, dedikační 10, jiný 1, neznámý 1

{\em Veřejné nápisy}~ seznamy 3, honorifikační dekrety 10, státní dekrety 2, náboženské 5, nařízení 1, funerální na náklady obce 1, jiný 4, neznámý 2

{\em Délka}~ aritm. průměr 7,53 řádku, medián 5, max. délka 94, min. délka 1

{\em Obsah}~ latinský text 12 nápisů, písmo římského typu 1; hledané termíny (administrativní termíny 29 nápisů - celkem 52 výskytů, epigrafické formule 13 - 24 výskytů, honorifikační 5 - 7 výskytů, náboženské 25 - 45 výskytů, epiteton 5 - počet výskytů 9)

{\em Identita}~ řecká božstva 11, egyptská božstva 2, pojmenování míst a funkcí typických pro řecké náboženské prostředí, objevující se místní thrácká božstva, regionální epiteton 1, subregionální epiteton 4, kolektivní identita 6 termínů, celkem 10 výskytů - obyvatelé řeckých obcí z oblasti Thrákie 3, mimo ni 0, kolektivní pojmenování Thráx 4, Rómaios 2, Kimbros 1; celkem 152 osob na nápisech, 27 nápisů s jednou osobou; max. 34 osob na nápis, aritm. průměr 2,35 osoby na nápis, medián 1; komunita multikulturního charakteru se zastoupením řeckého, thráckého prvku a narůstající pozicí římského prvku, jména pouze řecká (13,23 \letterpercent{}), pouze thrácká (4,4 \letterpercent{}), pouze římská (25 \letterpercent{}), kombinace řeckého a thráckého (8,82 \letterpercent{}), kombinace řeckého a římského (16,17 \letterpercent{}), kombinace thráckého a římského (4,4 \letterpercent{}), kombinovaná řecká, thrácká a římská jména (5,88 \letterpercent{}), jména nejistého původu (7,31 \letterpercent{}), beze jména (14,7 \letterpercent{}); geografická jména z oblasti Thrákie 18, z toho pojmenování stratégií v Thrákii 11, geogr. jména mimo Thrákii 0;



\stopCSV

Do 1. st. n. l. bylo celkem datováno 68 nápisů, což je zhruba stejně jako v 1. st. př. n. l. Jak je dobře vidět na mapě 6.07 v Apendixu 2, nápisy pocházejí převážně z pobřežních oblastí, nicméně osm nápisů bylo nalezeno i ve vnitrozemí. Zatímco velké produkční centrum v této době chybí, produkční centra střední velikosti se nacházejí i nadále v Byzantiu, dále v Perinthu a Maróneii.

Téměř výhradně je použitým materiálem kámen, nejčastější formou nosiče jsou i nadále stély, avšak v malé míře se začínají objevovat i sarkofágy. Přes polovinu nápisů představují funerální texty, nicméně počet veřejných nápisů narostl na více než 40 \letterpercent{}, což naznačuje větší míru institucionálního zapojení na epigrafické produkci. Dělo se tak zejména v okolí Perinthu, který se stal v druhé polovině století hlavním městem nově vzniklé provincie {\em Thracia} \cite[righttext={{, 521},{, 74}}][Sharankov2005, Sayar1998].

\subsection[funerální-nápisy-11]{Funerální nápisy}

Celkem se dochovalo 25 funerálních nápisů, z nichž většina pochází z Byzantia, Maróneie a Perinthu. V této době se poprvé objevily funerální sarkofágy, které nesly nápis, a zároveň sloužily jako úložiště tělesných pozůstatků zemřelého. Sarkofágy pocházely z Istanbulu, Perinthu a Maróneie, tedy měst, v němž měl Řím silné politické postavení.

Převaha osobních jmen řeckého původu poukazuje na přetrvávající charakter produkčních center, avšak se silným římským vlivem na podobu osobních jmen. Celkem 27 osob neslo řecká jména, 18 jména římská, šest jmen thráckých a tři jména neznámého či cizího původu. K mísení řeckých a římských, případně thráckých jmen dochází zcela výjimečně, přijímání systému tří římských jmen není ještě rozšířené. Thrácká jména se vyskytují pouze na třech nápisech a jejich nositelé pocházejí z vyšších společenských vrstev.\footnote{Jako např. stratégos Rhoimetalkás, syn Diasenea se svou ženou Besoulou, dcerou Moukaporida, a dětmi Kaproubebou a Daroutourme na nápise {\em I Aeg Thrace} 387.} Poprvé se objevují osobní jména (Titos) Flavios/Flavia a (Titos) Klaudios.\footnote{{\em I Aeg Thrace} 317, {\em Perinthos-Herakleia} 72, {\em Perinthos-Herakleia} 128.} Přijímání jmen po rodových jménech římských císařů byla typická praxe při udílení římského občanství, kdy se přidávaly k původnímu jménu nositele jako ocenění za služby ve správě provincie či v armádě \cite[righttext={, 13}][Topalilov2012]. V těchto konkrétních případech je jméno tvořeno i dalšími osobními jmény římského původu a nasvědčuje to spíše faktu, že se nejednalo o případy nově uděleného občanství, ale o osoby, které jména získali po svých předcích, a šlo tedy o Římany, nikoliv o Řeky či Thráky, kteří přijali nová jména.\footnote{Vyjádření kolektivní identity se vyskytují pouze jednou na podstavci, na němž byla pravděpodobně umístěna socha thráckého gladiátora. Nápis {\em I Aeg Thrace} 484 označuje etnický původ (Thráx) a označení druhu gladiátora ({\em mormillón}, lat. {\em mirmillo}).}

Jazyk funerálních nápisů si udržuje do jisté míry tradiční charakter, jak naznačují vyskytující se formule, ale dochází i k určitým inovacím ve funerálním ritu a jeho projevech v epigrafice.\footnote{Invokační formule {\em chaire} je použita třikrát, nebožtík je titulován jako {\em hérós} čtyřikrát, okolo jdoucí je oslovován jednou ({\em parodeita}).} Poprvé se setkáváme s formulí, která poskytuje právní ochranu hrobu, či spíše sarkofágu, před jeho dalším použitím.\footnote{{\em Perinthos-Herakleia} 88. Pokud by došlo k tomu, že budou do hrobu „{\em vloženy ostatky někoho jiného, zaplatí viník částku {[}x{]} městu/pokladníkovi}", do jehož regionu nápis spadá. Výše částky se lišila v dalších stoletích město od města, stejně tak i přesné znění formule.} Kdo by toto nařízení překročil, hrozí mu finanční postih, který je vymahatelný autoritami města. Tento text je typicky na konci funerální nápisu, a vyskytuje se výhradně na nápisech z Perinthu, tedy místě se silnou římskou přítomností. V dalších stoletích dojde k rozšíření této formule i do jiných míst, nicméně její výskyt pravděpodobně souvisí s administrativní změnou a regulací funerálního ritu v oblasti jihovýchodní Thrákie.

\subsection[dedikační-nápisy-11]{Dedikační nápisy}

Římská politická přítomnost v regionu se začíná projevovat i na měnícím se složení epigraficky aktivní části populace a dochází k opětovnému zapojení místních elit. Zvyk věnovat nápisy božstvům thráckého i řeckého původu se v této době rozšířil zejména mezi thráckou aristokracii, zastávající vysoké úřednické funkce v provinciální správě, a částečně i mezi veterány římské armády. Dedikačních nápisů se dochovalo celkem 10, z nichž tři pocházejí z vnitrozemí a sedm z pobřežních oblastí. Vnitrozemské dedikace jsou ve dvou případech věnovány stratégy nesoucí římská a thrácká jména, což poukazuje na jejich thrácký aristokratický původ\footnote{{\em SEG} 54:639, {\em IG Bulg} 4 2338.} a v jednom případě veteránem římské armády, který nese výhradně římská jména.\footnote{Nápis {\em IG Bulg} 3,1 1410 je psaný z větší čísti latinsky, nicméně jméno božstva a věnování je psáno řecky. Jméno veterána a jeho vojenské zařazení je psáno latinsky, stejně tak jako časové určení, které udává dobu vlády císaře Vespasiána a jeho sedmý konzulát, tedy rok 76 n. l.} Věnování těchto vnitrozemských dedikací patří Héře {\em Sonkéténé}, božstvu {\em Médyzeovi} a Artemidě Kyperské. Dedikace z pobřeží pak náleží císaři Vespasiánovi, Apollónovi {\em Karsénovi}, Diovi {\em Patróovi}, Diovi {\em Zbelsúrdovi}, Hygiei, Nymfám a Néreovi. Tato věnování pocházejí od stratégů nesoucího římská, řecká i thrácká jména, propuštěnce nesoucího řecká jméno a od nejvyššího kněze Dionýsova kultu nesoucího řecká jména.

Nápisy zmiňující stratégy pocházejí převážně z jihovýchodní části Thrákie, z okolí Perinthu a Anchialu, kde pravděpodobně docházelo ke kulturnímu transferu ve zvýšené míře, alespoň na úrovni nejvyšších představitelů politické organizace. Jména římsko-thrácká se vyskytují pouze na nápise {\em IG Bulg} 4 2338, kde figuruje Flavios Dizalas, syn Esbenida a jeho partnerka Reptaterkos, dcera Hérakleidova.\footnote{Nápis pochází z regionu města Nicopolis ad Nestum, datován do poslední čtvrtiny 1. st. n. l.} Jak prozrazuje text nápisu, thrácký Dizalás byl stratégem v Thrákii a občanství a s ním i právo nosit jméno Flavios si získal ve službách římské říši. Jedná se tak o jeden z prvních potvrzených případů, kdy Thrák získal římské jméno za své zásluhy, a nikoliv jako výsledek smíšených manželství.\footnote{Vzhledem k tomu, že Dizalás zastával úřad stratéga, jednalo se pravděpodobně o potomka thráckých aristokratů, který byl za svou loajalitu odměněn římským občanstvím, a tedy i právem nosit jméno Flavios.} Zvyklosti dedikovat nápisy se uplatňují v thrácké komunitě pouze v nejvyšších vrstvách aristokratů, případně veteránů, a stále nepronikly mezi běžné obyvatelstvo.

\subsection[veřejné-nápisy-11]{Veřejné nápisy}

Veřejných nápisů se dochovalo celkem 28, z čehož šest pochází z Perinthu a Byzantia a tři z Filippopole. Perinthos se po polovině 1. st. n. l. stal sídlem místodržícího provincie {\em Thracia}, a tak není výskyt celkem 10 nápisů překvapivý \cite[righttext={, 76}][Lozanov2015].\footnote{Hlavním městem provincie {\em Thracia} se stává Perinthos, kde vznikla i většina institucí a město se stalo jak sídlem místodržícího, tak zde sídlila vojenská posádka, a město se tak stalo přirozeným produkčním centrem veřejných nápisů \cite[righttext={{, 521},{, 74}}][Sharankov2005, Sayar1998].} V 10 případech se jedná o honorifikační dekrety, v pěti případech o texty náboženského charakteru a další kategorie jsou zastoupeny v řádech kusů nápisů. Politickou autoritu představuje římský císař, případě vysocí provinciální úředníci, kteří ho zastupují. Instituce jednotlivých měst mohou vydávat nařízení pod patronátem římské říše či thráckého krále, avšak převážně se veřejné nápisy vydané městskými institucemi omezují na honorifikační dekrety.\footnote{Např. nápis {\em SEG} 55:752 z Filippopole či {\em I Aeg Thrace} 83 z Abdéry. Politická autorita se proměňuje v závislosti na změně politické situace. Do poloviny 1. st. n. l. jako autority vystupuje jednak lid, instituce řeckých měst a thráčtí panovníci, kteří však byli smluvně zavázání Římu. Tito panovníci z kmene Odrysů, Astů a Sapaiů jsou známí jako vazalští králové Říma, kteří si částečně udržují autonomii, ale ve velké míře podléhají vlivu a politickým rozhodnutím Říma \cite[righttext={, 78-80}][Lozanov2015]. Pravděpodobně na začátku 1. st. n. l. dochází k rozdělení Thrákie na tzv. stratégie, tedy administrativní a vojenské regiony, které byly oficiálně řízeny thráckými stratégy z řad thráckých aristokratů. Gabriella Parissaki navrhuje, že se jednalo o mezistupeň mezi tradičním kmenovým uspořádáním a centralizovanou samosprávou římské provincie, kde thráčtí aristokraté sehráli významnou roli \cite[righttext={, 320-328}][Parissaki2009].} V této době se taktéž objevuje první milník, informující o době vzniku daného úseku cesty, a první nápisy informující o císařském provinciálním stavebním programu.\footnote{{\em I Aeg Thrace} 453 z lokality Ferai, datovaný do doby vlády císaře Nerona a {\em IG Bulg} 5 5691 z města Serdica, což představuje první důkaz o císařem organizované stavbě komunikací v Thrákii na pobřeží Egejského moře i ve vnitrozemí. Budování cest pravděpodobně probíhalo již v dřívějších stoletích, ale z té doby se nám nedochovaly milníky či jiné epigrafické záznamy \cite[righttext={{, 76},{, 63-65}}][Lozanov2015, Madzhahov2009].; {\em IG Bulg} 1,2 57 z Odéssu a {\em IK Sestos} 29 ze Séstu slouží jako doložení stavebních aktivit.} Zmínky o thrácké aristokracii jako takové a thráckých králích z epigrafických záznamů zcela mizí po polovině 1. st. n. l., ač systém stratégií se ještě udržuje několik desetiletí \cite[righttext={, 78-81}][Lozanov2015]. Thrácká aristokracie se tak pravděpodobně adaptovala na nové podmínky a zaujala roli v provinciální samosprávě, právě v roli stratégů.

Dochází také k pozvolné proměně použitého jazyka veřejných nápisů a instituce spojené s chodem římské provincie se začínají objevovat ve zvýšené míře.\footnote{Celkem 24 administrativních termínů se vyskytuje ve 37 případech. Většina termínů se již objevila v minulých stoletích, mezi nové termíny však patří {\em hegémón}, {\em kaisar} jako termíny označující římského císaře, dále {\em búleutés} jako člen {\em búlé}, dále {\em agoranomos}, {\em nauarchos}, a nakonec termín pro hlavní město {\em métropolis}. Mezi nejčastěji se opakující termín patří i nadále {\em démos} s 6 výskyty, a {\em gymnasiarchés} se čtyřmi výskyty.} Osobní jména poukazují na proměňující se trendy při výběru jmen a přijímání římských onomastických tradic zejména mezi Thráky sloužícími v římské armádě a vykonávajícími vysoké úřednické funkce. Lidé zastávající funkci stratéga běžně nesou tři jména, kombinující římská jména, jako Titos, Flavios, Tiberios, Gaios, Ioulios nebo Klaudios, spolu s osobními jmény thráckého původu.\footnote{Např. na nápise {\em I Aeg Thrace} 84 z Topeiru.}

\subsection[shrnutí-15]{Shrnutí}

Z dostupných nápisů je patrné, že připojení Thrákie k Římu s sebou neslo celou řadu změn. V reakci na zapojení thráckého obyvatelstva do římské armády se proměnily onomastické zvyky vojáků a veteránů. Epigrafická produkce se opět navrací do vnitrozemí, nicméně s příchodem autority římského císaře se proměňují projevy thrácké aristokracie. Projevy zvyšujícího vlivu římské administrativy jsou patrné zejména v druhé polovině 1. st. n. l., kdy dochází k budování provinciální infrastruktury a spolu s tím i k nárůstu počtu veřejných nápisů. Celkové počty dochovaných nápisů jsou však poměrně nízké, z čehož se dá usuzovat, že nárůst vlivu Říma byl postupný a reformy byly spíše dlouhodobého charakteru.

\stopcomponent