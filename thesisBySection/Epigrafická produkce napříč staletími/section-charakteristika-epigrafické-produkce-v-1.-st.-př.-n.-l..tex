
\environment ../env_dis
\startcomponent section-charakteristika-epigrafické-produkce-v-1.-st.-př.-n.-l.
\section[charakteristika-epigrafické-produkce-v-1.-st.-př.-n.-l.]{Charakteristika epigrafické produkce v 1. st. př. n. l.}

Celková epigrafická produkce v 1. st. př. n. l. poměrně výrazně klesá. Epigraficky aktivní komunity jsou i nadále z poloviny řecké, avšak i nadále dochází k jejich postupnému otevírání a mísení onomastických a náboženských tradic. Veřejné nápisy představují až třetinu celkové produkce a poprvé se objevují nápisy obsahující latinský text. V malé míře narůstají i počty dochovaných římských jmen.


\startframedbox
{\em Celkem}~ 69 nápisů

{\em Region měst na pobřeží}~ Abdéra 2, Apollónia Pontská 1, Byzantion 36, Dionýsopolis 1, Maróneia 10, Mesámbria 11, Odéssos 4, Perinthos (Hérakleia) 1, Sélymbria 1, Séstos 1 (celkem 68 nápisů)

{\em Region měst ve vnitrozemí}~ Didymoteichon (Plótinúpolis) 1

{\em Celkový počet individuálních lokalit}~ 13

{\em Archeologický kontext nálezu}~ funerální 2, sídelní 1, náboženský 1, sekundární 6, neznámý 59

{\em Materiál}~ kámen 67 (mramor 67, z toho mramor z Thasu 1), neznámý 2

{\em Dochování nosiče}~ 100 \letterpercent{} 4, 75 \letterpercent{} 5, 50 \letterpercent{} 5, 25 \letterpercent{} 12, kresba 2, nemožno určit 41

{\em Objekt}~ stéla 64, architektonický prvek 3, neznámý 2

{\em Dekorace}~ reliéf 48, bez dekorace 21; reliéfní dekorace figurální 34 nápisů (vyskytující se motiv: jezdec 1, stojící osoba 2, sedící osoba 1, funerální scéna/symposion 3), architektonické prvky 16 nápisů (vyskytující se motiv: naiskos 5, báze sloupu či oltář 3, věnec 1, florální motiv 3, architektonický tvar/forma 5)

{\em Typologie nápisu}~ soukromé 50, veřejné 17, neurčitelné 2

{\em Soukromé nápisy}~ funerální 42, dedikační 11, jiný 1\footnote{Jeden nápis měl vzhledem ke své nejednoznačnosti kombinovanou funkci funerálního a zároveň dedikačního nápisu, proto je součet nápisů obou typů vyšší než celkový počet soukromých nápisů.}

{\em Veřejné nápisy}~ náboženské 2, seznamy 3, honorifikační dekrety 2, státní dekrety 8, jiné 1, neznámý 1

{\em Délka}~ aritm. průměr 5,4 řádku, medián 2, max. délka 49, min. délka 1

{\em Obsah}~ dórský dialekt 9, latinský text 1 nápis; hledané termíny (administrativní termíny 18 - celkem 44 výskytů, epigrafické formule 9 - 28 výskytů, honorifikační 13 - 17 výskytů, náboženské 23 - 35 výskytů, epiteton 4 - počet výskytů 5)

{\em Identita}~ řecká božstva 10, egyptská božstva 3, pojmenování míst a funkcí typických pro řecké náboženské prostředí, regionální epiteton 4, kolektivní identita 4 termíny, celkem 6 výskytů - obyvatelé řeckých obcí z oblasti Thrákie 1, ale i mimo ni 1, kolektivní pojmenování Thráx 3, Rómaios 1; celkem 141 osob na nápisech, 41 nápisů s jednou osobou; max. 20 osob na nápis, aritm. průměr 2,04 osoby na nápis, medián 1; komunita řeckého charakteru s částečným zastoupením římského a thráckého prvku, jména pouze řecká (46,47 \letterpercent{}), pouze thrácká (1,44 \letterpercent{}), pouze římská (4,34 \letterpercent{}), kombinace řeckého a thráckého (4,34 \letterpercent{}), kombinace řeckého a římského (5,79 \letterpercent{}), kombinace thráckého a římského (1,44 \letterpercent{}), kombinovaná řecká, thrácká a římská jména (5,79 \letterpercent{}), jména nejistého původu (21,71 \letterpercent{}), beze jména (8,69 \letterpercent{}); geografická jména z oblasti Thrákie 3, geografická jména mimo Thrákii 2;



\stopframedbox

Do 1. st. př. n. l. bylo datováno 69 nápisů, což znamená pokles epigrafické produkce o 40 \letterpercent{} oproti předcházejícímu období. Jak je patrné na mapě 6.06 v Apendixu 2, nápisy pocházejí téměř výhradně z pobřežních oblastí s výjimkou lokality Didymoteichon, která se nachází v regionu pozdějšího města Plótinúpolis na řece Tonzos. Hlavní produkční centrum je i nadále v Byzantiu, odkud pochází 43 \letterpercent{} nápisů. Další středně velká produkční centra se nachází v Maróneii, v Mesámbrii, a částečně i v Odéssu, podobně jako v předcházejícím období.

Archeologický kontext míst nálezu je bohužel opět z velké části neznámý, případně sekundární a materiál použitý na výroby nosičů nápisů je opět výhradně kámen, zejména mramor. Nejrozšířenějším nosičem je kamenná stéla a opět zcela chybí nápisy na keramice či na kovových předmětech, které se v Thrákii vyskytovaly mezi 5. a 3. st. př. n. l.

\subsection[funerální-nápisy-9]{Funerální nápisy}

Celkem se dochovalo 42 funerálních nápisů, z nichž všechny sloužily jako primární funerální nápisy, tedy k označení hrobu a předání informace o zemřelém jeho nejbližšímu okolí. Všechny nápisy pocházely z pobřežních oblastí, s největší koncentrací 33 funerálních nápisů pocházejících z Byzantia. Pravděpodobnosti v souvislosti s politickým a ekonomickým rozvojem se Byzantion stal již koncem 2. st. př. n. l. epigrafickým centrem Thrákie a tento trend pokračuje i v průběhu 1. st. př. n. l. a na přelomu letopočtu.

Text funerálních nápisů si udržuje i nadále tradiční podobu\footnote{Invokační formule {\em chaire} se objevuje 12krát, v jednom případě se text nápisu obrací na podsvětní božstva, Háda a Acherón. Rozsah většinou nepřesahuje tři řádky, pouze v jednom případě dosahuje maximální délky 10 řádků. Tento obsáhlejší nápis {\em IG Bulg} 1,2 344 z Mesámbrie mluví o dosažených úspěších jistého Aristóna, který udatně bojoval s nepřátelským kmenem Bessů. Bessové představovali jeden z thráckých kmenů, který se dostal do popředí zájmu pramenů zejména v 1. st. př. n. l. v souvislosti s politickým vývojem v regionu a nástupem aristokratů právě z kmene Bessů k moci.} a identita zemřelých a jejich nejbližších ukazuje na převážně řecký původ nápisů. Jedná se o společnost, v níž se potkává několik kulturních tradic, což se projevuje zejména na proměňujících se onomastických zvyklostech. Zhruba dvě třetiny jmen jsou původem řecká, nicméně 13 \letterpercent{} jmen je možné zařadit jako jména římská a 8 \letterpercent{} jako jména thrácká, zbývajících 16 \letterpercent{} představuje jména cizího či neznámého původu. V omezené míře dochází k mísení řeckých jmen se jmény římskými, což je pravděpodobně důsledek smíšených sňatků a zvýšené přítomnosti římských občanů na území Byzantia.\footnote{Příkladem může být nápis {\em IK Byzantion} 269, který představuje náhrobní kámen členů jedné rodiny, kde dochází ke smíšenému sňatku mezi mužem s řeckými jmény a ženou se jmény římského původu: Hadys, syn Poseidónia a jeho manželka Veigellia Katyla a Dionysis, syn Hadyův.} Poprvé se také začínají ve velmi malé míře objevovat řecká či thrácká jména v kombinaci s římskou nomenklaturou.\footnote{Např. Markos Antonios Dadas z nápisu {\em IK Byzantion} 198.}

\subsection[dedikační-nápisy-9]{Dedikační nápisy}

Celkem se dochovalo 11 dedikačních nápisů, které pocházejí převážně z Mesámbrie, Maróneie a Byzantia, s výjimkou jednoho nápisu z lokality Didymoteichon na dolním toku řeky Tonzos. Čtyři nápisy byly věnované egyptským božstvům Sarápidovi, Ísidě, Anúbidovi a Harpokratiónovi. Egyptské kulty se objevují v Byzantiu, Maróneii a Mesámbrii, tedy ve všech hlavních produkčních centrech té doby, což poukazuje důležitost nejen pro jako místa setkávání kultur, ale i regionální centra té doby. Dále se objevila věnování vždy po jednom nápisu Diovi {\em Aithriovi}, Athéně a Neikonemesis {\em Sóteiře} a Hérakleovi {\em Sótérovi}.

Osobní jména dedikantů poukazují na jejich řecký původ, nicméně římský prvek začíná hrát důležitou roli i na dedikačních nápisech.\footnote{Celkem 26 jmen mělo řecký původ, 11 římský, dva thrácký a sedm nebylo možné s jistotou určit. Nejvíce jmen se vyskytlo na nápise {\em IK Byzantion} 19 z Byzantia, kde je možné napočítat až 26 jmen různého původu: jedná se o věnování Diovi {\em Aithriovi} obyvateli neznámé vesnice, kde funkce kněžích zastávají muži se třemi římskými jmény ({\em tria nomina}). Zhruba polovina obyvatel na této dedikaci má taktéž římské jméno, které je v několika případech kombinované se jménem řeckým (Lajtar 2000,50-51).} Na míru zapojení thráckých elit do praxe věnování nápisů božstvům, tedy zvyklostí do této doby omezené převážně na řeckou komunitu. Nápis {\em I Aeg Thrace} 458 věnovaný původně řeckému božstvu Hérakleovi Sótérovi z lokality Didymoteichon věnoval thrácký král Kotys, syn Rháskúporida mezi lety 42 a 31 př. n. l. \footnote{Tento nápis je na pomezí soukromého a veřejného nápisu, vzhledem k tomu, že byl věnován jménem krále Kotya, tedy svrchované politické autority.}

\subsection[veřejné-nápisy-9]{Veřejné nápisy}

Veřejných nápisů se dochovalo celkem 17, což představuje mírný pokles oproti 2. st. př. n. l. Nápisy pocházejí výhradně z pobřežních komunit, nejvíce jich bylo nalezeno na černomořském a egejském pobřeží: čtyři v Odéssu a Maróneii, tři v Mesámbrii a dále vždy po jednom nápise. Místa s největším počtem veřejných nápisů jsou shodná s největšími producenty soukromých nápisů.

Nápisy mohou být nepřímým důkazem o politických událostech a o zvyšujícím se vlivu Říma v regionu, avšak stále za udržení tradičních zvyklostí a procedur spojených s vydáváním veřejných nápisů. Nápisy obsahují 14 hledaných termínů v 35 výskytech, což je představuje výrazný pokles oproti předcházejícímu období. Nejvíce se vyskytuje termín {\em démos} v 10 případech, dále {\em búlé} a {\em polis} se čtyřmi výskyty a {\em basileus} a {\em pséfisma} se třemi výskyty. Dochází i k poklesu celkového počtu termínů označujícího společenské funkce a instituce. Většinu veřejných nápisů představují dekrety vydávané politickou autoritou, což v tomto případě byly instituce řecké {\em polis}, nicméně nápisy zmiňují představitele římské říše a thrácké krále jako rovnocenné partnery a uznávají jejich autoritu.\footnote{Příkladem mohou být nápisy {\em IG Bulg} 1,2 13, {\em IG Bulg} 1,2 43, {\em IG Bulg} 1,2 314a. Jedním takovýmto příkladem je dekret {\em IG Bulg} 1,2 314 z Mesámbrie, díky němuž se dozvídáme, že makedonský místodržící M. Terentius Lucullus v roce 72/71 př. n. l. ustanovil vojenskou posádku na území Thrákie a upevnil tak římský vliv v oblasti (Lozanov 2015, 77). O délce trvání římské přítomnosti a o míře vlivu však nemáme žádné další informace, nicméně je to jeden z prvních náznaků římského zájmu v oblasti, který se projevil i v epigrafice.}

Thrácká aristokracie i nadále hraje důležitou politickou roli v regionu: na nápisech se vyskytují celkem čtyři muži, nesoucí označení král Thráků: Kotys, syn Rháskúporida, Rhoimetalkás, Sadalás a Burebista. Z historických zdrojů víme, že v 1. st. př. n. l. došlo ke sjednocení Thráků pod kmenem Sapaiů, a tedy i k upevnění pozice thráckého panovníka se sídlem v Bizyi v jihovýchodní Thrákii (Lozanov 2015, 78). Rozmístění nápisů odpovídá i přibližné rozmístění sféry vlivu jednotlivých dynastií: Sapaiové na jihovýchodě, Odrysové ve středu a Getové na severu Thrákie.\footnote{Do dynastie Sapaiů spadají Rhaskúporis a Kotys s nápisy pocházejícími z Bizóné, Maróneie a lokality Dydimoteichon. Sadalás pravděpodobně patří do dynastie Odrysů a vyskytuje se na nápise {\em IG Bulg} 1,2 43 z Odéssu a Burebista, zmíněný na nápise {\em IG Bulg} 1,2 13 z Dionýsopole pochází z kmene Getů.}

\subsection[shrnutí-13]{Shrnutí}

Počty dochovaných nápisů z 1. st. př. n. l. poukazují na celkové snížení produkce nápisů a jejich vymizení z vnitrozemí, pravděpodobně související s celkovým úpadkem ekonomické prosperity thráckého vnitrozemí (Lozanov 2015, 84). Hlavním ekonomickým a produkčním centrem je i nadále Byzantion, ale nápisy se v menší míře vydávají i v dalších městech na pobřeží. Řecký prvek hraje i nadále důležitou roli, ale je doplněn jak thráckým, tak římským i dalšími prvky. Vzhledem k narůstající moci Říma se začíná proměňovat i složení epigraficky aktivní populace a její zvyky, ať už se jedná o projevy náboženství či nové onomastické trendy, případně o detailnější obsah funerálních nápisů.

Dochází k proměně na politické scéně, kdy se v 1. st. př. n. l. se opět objevuje thrácká aristokracie, tentokrát v podobě dynastie Sapaiů, Odrysů a Getů. Její přítomnost a projevy v epigrafice se však nepodobají nápisům 5. až 3. st. př. n. l., nicméně mají formu obvyklou spíše u politické autority typu řecké {\em polis}. Je tedy možné říci, že tato nově reformovaná thrácká aristokracie přistoupila na společenské normy a zvyklosti svých nejbližších sousedů a partnerů a jako komunikační strategii zvolila formu veřejných dekretů a usnesení.

\stopcomponent