
\environment ../env_dis
\startcomponent section-charakteristika-epigrafické-produkce-v-5.-až-4.-st.-př.-n.-l.
\section[charakteristika-epigrafické-produkce-v-5.-až-4.-st.-př.-n.-l.]{Charakteristika epigrafické produkce v 5. až 4. st. př. n. l.}

Převážná většina produkce nápisů datovaných do 5. a 4. st. př. n. l. pocházela i nadále z pobřežních oblastí, pouze s jedním nápisem pocházejícím z thráckého vnitrozemí. Nápisy i nadále sloužily převážně funerální funkci. Vzhled i obsah nápisů poukazoval na řecký původ nápisů, stejně tak i dochovaná osobní jména potvrzují, že epigrafická produkce se i nadále soustředila v rámci řeckých komunit.

\startDelimitedTable
 {\em Celkem}~ 124 nápisů

{\em Region měst na pobřeží}~ Abdéra 2, Apollónia Pontská 96, Chersonésos Molyvótés 1, Mesámbriá 2, Odéssos 8, Strýmé 9, Zóné 1; (celkem 119 nápisů)

{\em Region měst ve vnitrozemí}~ Beroé (pozdější Augusta Traiana) 1\postponenotes\footnote{Celkem čtyři nápisy pocházely z blíže neznámého místa v Thrákii: dva z nich pocházely z pobřeží Egejského moře a dva z území Bulharska.}

{\em Celkový počet individuálních lokalit}~ 15 (76 \letterpercent{} všech nápisů z Apollónie Pontské)

{\em Archeologický kontext nálezu}~ funerální 70, sekundární 7, neznámý 47

{\em Materiál}~ kámen 123 (mramor 29, vápenec, 51, jiný 37, z čehož je pískovec 31, a dále místní kámen; neznámý 6), kov 1 (stříbro)

{\em Dochování nosiče}~ 100 \letterpercent{} 62, 75 \letterpercent{} 13, 50 \letterpercent{} 16, 25 \letterpercent{} 5, oklepek 1, kresba 10, ztracený 3, nemožno určit 14

{\em Objekt}~ stéla 119, architektonický prvek 4, jiný 1

{\em Dekorace}~ reliéf 27, malovaná dekorace 11 (písmena červenou barvou), bez dekorace 87; reliéfní dekorace figurální 2 nápisy (vyskytující se motiv: jezdec 1, funerální scéna 1, stojící osoba 1, funerální portrét 1), architektonické prvky 17 nápisů (vyskytující se motiv: naiskos 22, sloup 3, báze sloupu či oltář 14, florální motiv 6, architektonický tvar 29, jiný 1)

{\em Typologie nápisu}~ soukromé 120, veřejné 1, neurčitelné 3

{\em Soukromé nápisy}~ funerální 119, vlastnictví 1

{\em Veřejné nápisy}~ náboženské 1

{\em Délka}~ aritm. průměr 2,25 řádku, medián 2, max. délka 7, min. délka 1

{\em Obsah}~ použitý dialekt atticko-iónský 1, stoichédon 1; řecká božstva 1, epiteton regionální 1, epigrafické formule funerální 2 - celkem 2 výskyty, ostatní hledané termíny 0

{\em Identita}~ místa zmíněná v nápisech z Thrákie (Perinthos 1), celkem 123 osob na nápisech, 104 nápisů s jednou osobou, max. 2 osoby na nápisech, aritm. průměr 0,99 osoby na nápis, medián 1; komunita převládajícího řeckého charakteru, osobní jména řecká (85 \letterpercent{}), thrácká (0,8 \letterpercent{}), kombinace řeckého a thráckého (1,61 \letterpercent{}), kombinace řeckého a jména nejistého původu (5,64 \letterpercent{}) jména nejistého původu (1,61 \letterpercent{}), bez jména (5,64 \letterpercent{}); vyjádření kolektivní identity 1 (občan řeckého města Hérakleia)
\stopDelimitedTable
\flushnotes

Celkem se dochovalo 124 nápisů a podobně jako v předcházejících obdobích pocházejí z okolí řeckých měst na mořském pobřeží, jak je patrné z \in{Mapy}[Apendix2:::6.02a] v \in{Apendixu}[Apendix2:::Apendix2]. Největším producentem nápisů je Apollónia Pontská, odkud pochází 96 nápisů, což představuje 77 \letterpercent{} všech nápisů z daného období.\footnote{Za vysoký počet nápisů pocházejících právě z Apollónie může pravděpodobně stav archeologického výzkumu v oblasti a nedávné objevení několika nekropolí přímo na území antického města (Velkov 2005; Gyuzelev 2002, 2005, 2013; Baralis a Panayotova 2013, 2015; Hermary {\em et al}. 2010). V této době byla Apollónia ekonomické centrum a měla četné kontakty i s vnitrozemím, jak dokazují i např. mincovní nálezy (Paunov 2015, 268-269). Apollónia Pontská, kolonie Mílétu\index{řecká kolonizace}, byla založená koncem 7. st. př. n. l. Poměrně záhy se stala obchodním centrem regionu a svou výsadní pozici si udržela i v průběhu 5. st. př. n. l., kdy dokonce začala razit vlastní mince, které se nacházejí nejen v Thrákii, ale i ve Středomoří (Isaac 1986, 242-246). Apollónia\index{Apollónia Pontská} tak pravděpodobně sloužila jako středisko obchodu a nevyhnutelně zde docházelo i ke kulturní výměně mezi řeckým světem a thráckým obyvatelstvem.} Pouze jeden nápis pochází z vnitrozemí z okolí moderního města Kazanlak, spadajícího do regionu antického města {\em Beroé}, později známého pod římským jménem {\em Augusta Traiana}. Lešnikova Mogila, v níž byl nápis nalezen jako součást pohřební výbavy, bývá interpretována jako místo pohřbu bohatého aristokrata thráckého původu, viz níže (Kitov 1995, 19-21).

Nápisy datované do 5. až 4. st. př. n. l. mají velmi podobný charakter jako nápisy datované do 5. st. př. n. l. Většina nápisů je zhotovena z místního kamene a má podobu jednoduché stély s minimem dekorací, jako jsou florální motivy či červeně malovaná písmena. Hlavní funkcí nápisů je stále označení místa pohřbu v řeckých komunitách a v thráckém vnitrozemí slouží nápisy i nadále úzkému okruhu aristokratů.

\subsection[funerální-nápisy-2]{Funerální nápisy}

Skupina nápisů datovaných do 5. a 4. st. př. n. l. obsahuje 119 primárních funerálních nápisů tesaných do kamene. Až 95 z nich pochází z Apollónie Pontské, což je pravděpodobně důsledek nedávného archeologického výzkumu ve městě, při němž byly objeveny rozsáhlé nekropole Kalfata a Budžaka (Gyuzelev 2002, 2005, 2013; Velkov 2005).

Oproti předcházejícímu období je možné pozorovat jen velmi pozvolný nárůst výskytu thráckých jmen na funerálních nápisech. Řecká a thrácká jména se společně vyskytují na třech funerálních nápisech, což představuje pouhých 2,4 \letterpercent{} funerálních nápisů z daného období. Vyskytující se thrácká jména patří výhradně ženám a všechny tři nápisy pocházejí z nekropole Kalfata ve městě Apollónia Pontská, kde pravděpodobně docházelo ke kontaktu mezi Thráky a Řeky a k navazování partnerských vztahů mezi jedinci s odlišným původem.\footnote{Žena thráckého původu partnerka či dcera muže nesoucí jméno řeckého původu se jménem Paibiné Augé, partnerka/dcera Hermaia z nápisu {\em IG Bulg} 1,2 430; tak jako žena se jménem řeckého původu partnerka či dcera muže se jménem thráckého původu: Faniché, partnerka/dcera Kerzea (Gyuzelev 2002, 20), a dále Dioskoridé, partnerka/dcera Basstakilea z nápisu {\em IG Bulg} 1,2 440. Fakt, že dochované nápisy dokumentují vytváření příbuzenských kontaktů mezi Thráky a Řeky, znamená, že obě mezi oběma komunitami docházelo ke kontaktu již delší dobu a vznikly zde vztahy, které epigrafické prameny nepostihují vůbec, či nepřímo a se zpožděním i desítek let.}

Tzv. sekundárně funerální se dochoval pouze jeden nápis zhotovený na stříbrné nádobě. Tento nápis taktéž pochází z kontextu bohaté pohřební výbavy hrobky thráckého aristokrata, která je známá pod názvem Lešnikova Mogila a nachází se u moderního města Kazanlak ({\em SEG} 55:742; Kitov 1995, 19-21). Podobně jako u stejné skupiny nápisů datovaných do 5. st. př. n. l. se jedná o stříbrnou nádobu sloužící za života majitele, jehož jméno pravděpodobně nápis nese.\footnote{O přesné podobě jména a znění nápisu se badatelé nemohou shodnout, vzhledem k tomu, že se jedná o jediný výskyt tohoto jména. Dimitrov (2009, 31-32) navrhuje interpretaci „(nádoba) Dynta, syna Zeila”. Theodossiev (1997, 174) navrhuje znění „(fiálé) Dynta, syna Zemya” a Dana (2015, 247) navrhuje znění „(majetek) Dyntozelmia”.} V okruhu thráckých aristokratů vlastnictví předmětu nesoucího nápis poukazovalo na společenskou prestiž majitele, podobně jako tomu je i u skupiny nápisů na kovových nádobách datovaných do 5. st. př. n. l.

\subsection[dedikační-nápisy-2]{Dedikační nápisy}

Z této skupiny nápisů se nedochoval žádný dedikační nápis.

\subsection[veřejné-nápisy-2]{Veřejné nápisy}

Z této skupiny nápisů se dochoval pouze jeden nápis {\em IG Bulg} 1,2 398, který Georgi Mihailov označuje jako {\em res sacrae}, ale svou povahou se nápis nachází na pomezí veřejného, dedikačního a stavebního textu (Mihailov 1970, 365-366). Tento nápis sloužil k označení místa v regionu Apollónie Pontské, kde stál chrám, {\em megaron}, řecké bohyně Gé {\em Chthonios}. Nejedná se tedy v pravém slova smyslu o nápis vytvořený politickou autoritou, který by souvisel s chodem státu či jiné politické organizace, ale spíše o nápis dokumentující rozčlenění půdy a existenci chrámu řecké bohyně na území řecké obce.

\subsection[shrnutí-6]{Shrnutí}

Nápisy datované do 5. až 4. st. př. n. l. vykazují velkou míru podobnosti se skupinou nápisů datovaných do 5. st. př. n. l. Převahu mají funerální nápisy, které pocházejí pouze z prostředí řeckých měst na pobřeží a které si udržují tradiční formu i obsah. Z vnitrozemí pochází pouze jeden nápis zhotovený na stříbrné nádobě, který je možné spojovat s thráckou aristokracií a jehož účel byl převážně utilitární povahy s významem pro okruh nejbližších aristokratů. Zvyk vytvářet nápisy nesoucí sdělení určené pro širší komunitu i nadále zůstává znakem řecké kultury a neprojevuje sena zvyklostech thráckých obyvatel.

\stopcomponent