
\environment ../env_dis
\startcomponent section-charakteristika-epigrafické-produkce-v-5.-st.-n.-l.-až-6.-st.-n.-l.
\section[charakteristika-epigrafické-produkce-v-5.-st.-n.-l.-až-6.-st.-n.-l.]{Charakteristika epigrafické produkce v 5. st. n. l. až 6. st. n. l.}

U nápisů datovaných do 5. až 6. st. n. l. je i nadále možné sledovat celkový pokles jejich produkce. Většina z nich pochází z pobřežních oblastí a má soukromý charakter. Obsah nápisů úzce souvisí s projevy křesťanské víry a funerální funkcí nápisů. Epigraficky aktivní komunity jsou spíše uzavřeného charakteru, epigraficky nejsou zaznamenány kontakty dokonce ani s nejbližším okolím.


\startCSV
{\em Celkem}~ 16 nápisů

{\em Region měst na pobřeží}~ Apollónia Pontská 1, Byzantion 3, Maróneia 3, Mesámbria 3, Topeiros 1 (celkem 11 nápisů)

{\em Region měst ve vnitrozemí}~ Nicopolis ad Istrum 2 (celkem 2 nápisy)\footnote{Celkem tři nápisy nebyly nalezeny v rámci regionu známých měst, editoři korpusů udávají jejich polohu vzhledem k nejbližšímu modernímu sídlišti, či uvádí muzeum, v němž se nachází.}

{\em Celkový počet individuálních lokalit}~ 8

{\em Archeologický kontext nálezu}~ sekundární 3, neznámý 13

{\em Materiál}~ kámen 15 (mramor 13; jiný 2), jiný 1 (z toho dřevo 1)

{\em Dochování nosiče}~ 50 \letterpercent{} 5, 25 \letterpercent{} 3, kresba 1, nemožno určit 7

{\em Objekt}~ stéla 11, architektonický prvek 2, jiný 1, neznámý 2

{\em Dekorace}~ reliéf 5, bez dekorace 11; reliéfní dekorace figurální 1 nápis (vyskytující se motiv: funerální scéna/symposion 1), architektonické prvky 4 nápisy (vyskytující se motiv: kříž 7, christogram 1)

{\em Typologie nápisu}~ soukromé 13, veřejné 0, neurčitelné 3

{\em Soukromé nápisy}~ funerální 10, dedikační 1, jiný 1, neznámý 1

{\em Veřejné nápisy}~ 0

{\em Délka}~ aritm. průměr 4,75 řádku, medián 5, max. délka 9, min. délka 1

{\em Obsah}~ latinský text 0, písmo řím. typu 1; hledané termíny (administrativní termíny 0, epigrafické formule 1 - 1 výskyt, honorifikační 0, náboženské 2 - 3 výskyty, epiteton 0)

{\em Identita}~ křesťanská náboženská terminologie, vymizení lokálních kultů z nápisů, regionální epiteton 0, subregionální epiteton 0, kolektivní identita 0 termínů; celkem 8 osob na nápisech, 8 nápisů s jednou osobou; max. 1 osoba na nápis, aritm. průměr 0,5 osoby na nápis, medián 0,5; uzavřené komunity, bez prolínání onomastických tradic, jména pouze řecká (25 \letterpercent{}), pouze thrácká (0 \letterpercent{}), pouze římská (12,5 \letterpercent{}), kombinace řeckého a thráckého (0 \letterpercent{}), kombinace řeckého a římského (0 \letterpercent{}), kombinace thráckého a římského (0 \letterpercent{}), kombinovaná řecká, thrácká a římská jména (0 \letterpercent{}), jména nejistého původu (12,5 \letterpercent{}), beze jména (50 \letterpercent{}); geografická jména z oblasti Thrákie 1, mimo Thrákii 0;



\stopCSV

Většina nápisů z 5. až 6 st. n. l. pochází z pobřežních oblastí. Epigraficky aktivní komunity se udržují pouze na několika místech ve třech regionech na mořském pobřeží, zejména v okolí Byzantia (Konstantinopole), Perinthu (Hérakleie), Maróneie a Mesámbrie, podobně jako v předcházejícím období, což dobře ilustruje Mapa 6.11 v Apendixu 2.

Nápisy mají výhradně soukromý charakter, z nichž 10 nápisů mělo funerální funkci a pocházelo z křesťanské komunity, a jeden sloužil jako dedikace věnovaná křesťanskému Bohu.\footnote{Nápis Velkov 2005 54 z Mesámbrie.} Nápisy jsou většinou velmi špatně dochované, nicméně z jejich výzdoby a fragmentárního textu lze soudit, že pocházejí výhradně z křesťanského kontextu, z komunit v Maróneii, Byzantiu, Mesámbrii a Apollónii. Dochovaná jména jsou převážně řeckého původu a setkáváme se i s uváděním funkcí, které zemřelý zastával v rámci křesťanské komunity.\footnote{Např. nápis {\em I Aeg Thrace} 96 nebo {\em SEG} 60:735.} Zcela naopak chybí nápisy veřejné povahy, a dále nápisy v nichž by figurovalo obyvatelstvo nesoucí thrácká jména.

\subsection[shrnutí-24]{Shrnutí}

Nápisy datované do 5. až 6. st. n. l. poukazují i nadále na zásadní vliv křesťanství na společnost tehdejší Thrákie, která se projevila i na podobě epigrafické produkce. Dochované funerální nápisy pocházejí především z oblastí se silnou křesťanskou komunitou. Velmi malý počet dochovaných nápisů nasvědčuje, že zvyk publikovat nápisy do jisté míry přežíval v komunitách, které přeměnily podobu funerálních nápisů tak, aby jejich forma a obsah vyhovovaly potřebám křesťanské víry.

\stopcomponent