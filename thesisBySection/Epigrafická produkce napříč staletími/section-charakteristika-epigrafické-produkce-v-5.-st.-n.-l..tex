
\environment ../env_dis
\startcomponent section-charakteristika-epigrafické-produkce-v-5.-st.-n.-l.
\section[charakteristika-epigrafické-produkce-v-5.-st.-n.-l.]{Charakteristika epigrafické produkce v 5. st. n. l.}

V 5. st. n. l. pokračuje útlum epigrafických aktiv, komunity se ještě více uzavírají. Produkce do omezené míry přežívá na pobřeží a ve vnitrozemských městech, nicméně se snižuje variabilita formy a obsahu nápisů: soukromé funerální nápisy jsou takřka jediným typem nápisů, převládá jednoduchá dekorace s křesťanskou tematikou. Vliv křesťanství narůstá, jak v dekoraci, tak i funkci nápisů, podobně jako v předcházejícím období.


\startframedbox
{\em Celkem}~ 12 nápisů

{\em Region měst na pobřeží}~ Byzantion 3, Maróneia 4, Perinthos (Hérakleia) 1 (celkem 8 nápisů)

{\em Region měst ve vnitrozemí}~ Filippopolis 2, Plótinúpolis 1, Traianúpolis 1 (celkem 4 nápisy)

{\em Celkový počet individuálních lokalit}~ 6

{\em Archeologický kontext nálezu}~ funerální 3, náboženský 2, sekundární 2, neznámý 5

{\em Materiál}~ kámen 12 (mramor 7; vápenec 2, neznámý 3)

{\em Dochování nosiče}~ 100 \letterpercent{} 2, 75 \letterpercent{} 1, 50 \letterpercent{} 1, 25 \letterpercent{} 2, nemožno určit 6

{\em Objekt}~ stéla 10, architektonický prvek 1, mozaika 1

{\em Dekorace}~ reliéf 5, bez dekorace 7; reliéfní dekorace figurální 0, architektonické prvky 5 nápisů (vyskytující se motiv: naiskos 1, sloup 1, jiný 4 (kříž 4))

{\em Typologie nápisu}~ soukromé 11, neurčitelné 1

{\em Soukromé nápisy}~ funerální 10, neznámý 1

{\em Veřejné nápisy}~ 0

{\em Délka}~ aritm. průměr 3,91 řádku, medián 3, max. délka 14, min. délka 1

{\em Obsah}~ latinský text 0, písmo římského typu 0; hledané termíny (administrativní 1 - celkem 1 výskyt, epigrafické formule 3 - 3 výskyty, honorifikační 0, náboženské 0, epiteton 0)

{\em Identita}~ narůstající vliv křesťanství, regionální epiteton 0, subregionální epiteton 0, kolektivní identita 0 termínů; celkem 12 osob na nápisech, 10 nápisů s jednou osobou; max. 1 osoba na nápis, aritm. průměr 0,8 osoby na nápis, medián 1; komunita řeckého a omezeně římského charakteru, jména pouze řecká (41,67 \letterpercent{}), pouze thrácká (0 \letterpercent{}), pouze římská (0 \letterpercent{}), kombinace řeckého a thráckého (0 \letterpercent{}), kombinace řeckého a římského (8,33 \letterpercent{}), kombinace thráckého a římského (0 \letterpercent{}), kombinovaná řecká, thrácká a římská jména (0 \letterpercent{}), jména nejistého původu (33,2 \letterpercent{}), beze jména (16,6 \letterpercent{}); geografická jména z oblasti Thrákie 0, mimo Thrákii 0;



\stopframedbox

Celková epigrafická produkce v 5. st. n. l. poklesla o 52 \letterpercent{} oproti 4. st. př. n. l. a o 93 \letterpercent{} oproti 3. st. n. l. Jedná se o pokračující trend poklesu epigrafické produkce, který začal již na konci 3. st. n. l., a který je pozorovatelný i v dalších částech bývalé římské říše. Nápisy pocházejí především z Maróneie a Byzantia (Konstantinopole), jak je patrné na mapě 6.11 v Apendixu 2.

Nápisy z 5. st. n. l. jsou převážně vyrobeny z kamene, jen jeden nápis se zachoval na dřevě. Jejich charakter je soukromý a slouží výhradně funerální funkci, celkem v 11 případech. Velký vliv na formu i obsah nápisu má křesťanství, které se stalo výhradním tématem spojujícím dochované nápisy. Časté je vyobrazení kříže či christogramu, se zmínkami o křesťanském bohu, případně Kristovi. Funerální nápisy se stávají individuálním monumentem, zmiňujícím pouze zemřelého, a nikoliv jeho rodinu. Jména na nápisech jsou převážně řeckého původu, ale do určité míry dochází k uchování římské onomastické tradice.\footnote{Např. na nápise {\em SEG} 56:823 vystupuje muž jménem Flavios Eutychés.} Thrácká jména, stejně tak jako kulty thráckého původu zcela chybí. Nevyskytují se ani zmínky o kontaktech s jinými komunitami, ať už z regionu či mimo region a nedochází k epigraficky zaznamenané imigraci lidí, tak jak bylo běžné např. ve 2. a 3. st. př. n. l.

\subsection[shrnutí-23]{Shrnutí}

V 5. st. n. l. se komunity uzavírají a opouštějí dříve nastolenou společenskou organizaci a zvyky, mezi něž patřil například i zvyk publikovat nápisy. Zcela dochází k vymizení institucionální epigrafické produkce. Vzniklé křesťanské nápisy jsou poslední přežívající tradicí, amalgámem pohanského a křesťanského pohřebního ritu, který se udržel v místech se silnou historickou tradicí publikace nápisů, a na mnoha jiných místech zanikl.

\stopcomponent