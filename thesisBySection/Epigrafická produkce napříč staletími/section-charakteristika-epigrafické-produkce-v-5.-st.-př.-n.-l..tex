
\environment ../env_dis
\startcomponent section-charakteristika-epigrafické-produkce-v-5.-st.-př.-n.-l.
\section[charakteristika-epigrafické-produkce-v-5.-st.-př.-n.-l.]{Charakteristika epigrafické produkce v 5. st. př. n. l.}

V 5. st. př. n. l. dochází ke znatelnému nárůstu epigrafické produkce a rozšíření nápisů do většího počtu řeckých komunit na pobřeží, kde i nadále převládají funerální nápisy. V thráckém vnitrozemí se poprvé objevují nápisy použité v rámci thráckého funerálního kontextu, patrně patřící thrácké aristokracii a související s prominentní pozicí, kterou aristokraté ve společnosti zastávali.


\startframedbox
{\em Celkem}~ 60 nápisů

{\em Region měst na pobřeží}~ Abdéra 10, Apollónia Pontská 9, Mesámbria 4, Perinthos (Hérakleia) 2, Strýmé 20, Zóné 9 \cite[celkem54]

{\em Region měst ve vnitrozemí}~ Pulpudeva (pozdější Filippopolis) 5\footnote{Jeden nápis byl nalezen mimo region známých měst, a není ho možné zařadit do regionu měst na pobřeží, ani ve vnitrozemí.}

{\em Celkový počet individuálních lokalit}~ 13

{\em Archeologický kontext nálezu}~ funerální 12, sídelní 1, nábož. 1, sekundární 11, neznámý 35

{\em Materiál}~ kámen 54 (mramor 26, z toho mramor z Thasu 1, vápenec 6, jiný 20, z čehož je pískovec 1, póros 5; neznámý 2), kov 3 (stříbro, zlato), keramika 3

{\em Dochování nosiče}~ 100 \letterpercent{} 10, 75 \letterpercent{} 10, 50 \letterpercent{} 23, 25 \letterpercent{} 7, kresba 2, nemožno určit 8

{\em Objekt}~ stéla 49, architektonický prvek 7, nádoba 3, jiné 1

{\em Dekorace}~ reliéf 12, malovaná dekorace 0, bez dekorace 48; reliéfní dekorace figurální celkem 2 nápisy (vyskytující se motiv: stojící osoba 2, sedící osoba 1, obětní scéna 1), architektonické prvky 14 (vyskytující se motiv: naiskos 3, sloup 2, báze sloupu či oltář 5, florální motiv 1, architektonický tvar 5)

{\em Typologie nápisu}~ soukromé 55, veřejné 3, neurčitelné 2

{\em Soukromé nápisy}~ funerální 43, dedikační 6, vlastnictví 4, jiný 1, neznámý 1

{\em Veřejné nápisy}~ nařízení 1, jiný 2

{\em Délka}~ aritm. průměr 2,68 řádku, medián 2, max. délka 16, min. délka 1

{\em Obsah}~ dórský dialekt 2, iónsko-attický 4; bústrofédon 1, stoichédon 1; hledané termíny (administrativní 4 - celkem 4 výskyty, epigrafické formule 3 - celkem 7 výskytů, honorifikační 0, náboženské 6 - celkem 8 výskytů, epiteton 6 - celkem 6 výskytů)

{\em Identita}~ řecká božstva, regionální epiteta, kolektivní identita 4 - pouze obyvatelé řeckých obcí mimo oblast Thrákie, celkem 56 osob na nápisech, 46 nápisů s jednou osobou; max. 2 osoby na nápisech, aritm. průměr 0,93 osoby na nápis, medián 1; komunita převládajícího řeckého charakteru, jména pouze řecká (65 \letterpercent{}), thrácká (1,67 \letterpercent{}), kombinace řeckého a thráckého (1,67 \letterpercent{}), jména nejistého původu (10 \letterpercent{})



\stopframedbox

Do 5. století celkem spadá 60 nápisů a k celkovému rozšíření nálezových lokalit na 13 oproti třem lokalitám z 6. st. př. n. l., a s tím i spojeným nárůstem epigrafické produkce.\footnote{Nálezové lokality nesou jméno dle nejbližšího moderního osídlení, pokud není známo jejich antické jméno. V databázi jsou vedeny jako „{\em Modern Location}". V regionu antického města je zpravidla více nálezových lokalit, které mohou, ale nutně nemusí korespondovat s archeologickou lokalitou.} Mapa 6.02 v Apendixu 2 ilustruje rozložení nápisů na území Thrákie v 5. st. př. n. l. Většina nálezových lokalit pochází z území řeckých kolonií na pobřeží Egejského, Černého a Marmarského moře, v maximální vzdálenosti do 20 km od pobřeží. Největším producentem s 20 nápisy je řecké osídlení Strýmé poblíž moderního Molyvoti na egejském pobřeží.\footnote{Z dalších měst, která v té době existovala, se nápisy do dnešní doby nedochovaly, či ještě nebyly objeveny. Možné je však také, že tato města v 5. st. př. n. l. nápisy neprodukovala, ať už z důvodů nestabilních podmínek, či odlišného přístupu tamních obyvatel k publikaci nápisů.} Z thráckého vnitrozemí se dochovaly nápisy v řádu pěti kusů z okolí moderní vesnice Duvanlij. Nápisy byly objeveny jako součást bohaté pohřební výbavy thráckých aristokratů v monumentálních hrobkách severně od řeky Hebros v regionu thráckého osídlení {\em Pulpudeva}, pozdější Filippopole. Tyto vnitrozemské nápisy se neodlišují pouze svou polohou, ale i použitým materiálem a funkcí, kterou objekt zastával v rámci společnosti.

Podobně jako v 6. st. př. n. l. je převážná většina 90 \letterpercent{} nosičů nápisu zhotovena z kamene, nicméně zbývajících 10 \letterpercent{} nápisů se nachází na kovových předmětech a na keramických nádobách.\footnote{Z nápisů tesaných do kamene má výrazná většina 82 \letterpercent{} charakter soukromého nápisu, tedy nápisu sloužícího pro soukromé účely jedince či skupiny lidí. Téměř ze tří čtvrtin převládají funerální nápisy se 43 exempláři, dedikační nápisy představují zhruba 10 \letterpercent{} a zbývající nápisy není možné přesněji určit.} Hlavní role nápisů psaných na kameni byla předat a sdělit specifickou zprávu v rámci lokální komunity, proto byly z převážné většiny určené k veřejnému vystavení, přístupné všem. Na lokální původ nápisů tesaných do kamene poukazuje i použitý materiál jako je vápenec, pískovec či porézní kámen, tzv. póros.\footnote{Tento fakt poukazuje na regionální charakter místních komunit, které primárně využívaly místně dostupný materiál a dále podporuje teorii, že nápisy byly produkovány místně a nestávaly se předmětem dálkového obchodu, alespoň v 5. st. př. n. l.} Oproti tomu primární funkce nápisů na kovových předmětech a keramických nádobách bylo v pěti z šesti nápisů označení vlastnictví či autorství a v jednom případě dedikace božstvu. Tato skupina nápisů sloužila primárně pro interní potřebu majitele a zpravidla se nejednalo o nápisy veřejně přístupné komukoliv, ale pouze vybraným členům z okolí majitele předmětu. Použitý materiál byl často drahý kov, jako je zlato či stříbro, což naznačuje na vysoké společenské postavení majitele. Mnohdy měly tyto předměty po smrti majitele využití jako sekundární funerální nápisy, a to zejména ve společnostech s kmenovým uspořádáním založených na charismatu a společenské prestiži jedince (Whitley 1991, 354-361; Bliege Bird a Smith 2005, 221-222, 233-234).

\subsection[funerální-nápisy-1]{Funerální nápisy}

Nápisy označené jako funerální je možné rozdělit do dvou kategorií dle jejich původní funkce a vztahu ke známému archeologickému kontextu: primární funerální nápisy, které byly vytvořeny pro účel pohřebního ritu, tedy např. vnitřní vybavení hrobky, architektonické součásti hrobky, a dále sekundární funerální nápisy, jejichž funkce byla původně jiná, ale pro svou sentimentální a společenskou hodnotu předměty nesoucí nápis tvořily součást pohřební výbavy \cite[Janouchová a Weissová2015].\footnote{Dosud nepublikovaný příspěvek na konferenci Symposium on Mediterranean Archaeology, 12. - 14. listopadu 2015, Kemer-Antalya, Turecko.}

\subsubsection[primární-funerální-nápisy]{Primární funerální nápisy}

Primární funerální nápisy jsou typicky kamenné funerální stély či jiné předměty, které sloužily k označení místa pohřbu a připomínání zemřelého. Dochovalo se jich celkem 37 a pocházejí výhradně z území řeckých měst v pobřežních oblastech.

Přes 91 \letterpercent{} primárních funerálních nápisů nese pouze řecká jména, s výjimkou jednoho nápisu z Apollónie, kde spolu figurují řecká a thrácká jména. Na černomořském pobřeží bylo možno u dvou nápisů určit použití dórského dialektu, a to u nápisů z Mesámbrie, která byla založená jako dórská kolonie, a dále i použití dialektu iónsko-attického u dvou nápisů z původně iónských kolonií Perinthu a Apollónie. Tento fakt souvisí s dialektem užívaným v rámci řeckých obcí, které oblasti osídlily a s nimiž je pojilo silné kulturně-historické pouto. Vyjádření identity na funerálních nápisech se objevuje celkem na čtyřech nápisech: vždy se jedná o vyjádření příslušnosti k řeckému městskému státu a většinou se nachází v kombinaci s řeckým osobním jménem.\footnote{Vyskytující se termíny: Aigínétés, Athénaios, Kyzikénos a Paroités.} Geografické jméno se na funerálních nápisech vyskytuje pouze jednou a jedná se o město Perinthos, tedy o řecké osídlení z území Thrákie samotné. Co se týče hledaných termínů a vyjádření identity na funerálních nápisech se vyskytuje pouze jediný administrativní termín, popisující identitu ženy nesoucí jméno řeckého původu jako propuštěnou otrokyni.\footnote{Nápis {\em IG Bulg} 1,2 334octies z~Mesámbrie.}

Z dochovaných funerálních nápisů je patrné, že místní řecké komunity byly i v 5. st. př. n. l. poměrně uzavřené a ke kontaktu mezi thráckým obyvatelstvem docházelo v minimální míře v okolí Apollónie Pontské. Naopak ke kontaktu s dalšími částmi řecky mluvícího světa docházelo na pobřeží Egejského moře, které bylo v této době místem zvýšeného zájmu řeckých obcí. Nedostatek interakcí na epigrafickém materiále však nutně nemusí znamenat neexistenci kontaktů, ale spíše poukazuje na charakter nápisů jako na velmi selektivní médium, zachycující pouze malou část tehdejší společnosti. Archeologické výzkumy naopak dokazují, že mezi řeckými a thráckými komunitami docházelo v 5. st. př. n. l. k vzájemné interakci na každodenní bázi, která se ale bohužel neprojevila na nápisech (Kostoglou 2010, 180-185; Ilieva 2007, 212-221).

\subsubsection[sekundární-funerální-nápisy]{Sekundární funerální nápisy}

Sekundární funerální nápisy se nacházejí na kovových či keramických předmětech, které primárně nebyly vyrobeny pro pohřební ritus, ale do dnešní doby se dochovaly právě jako součást pohřební výbavy. Celkem se jedná o tři nápisy na kovových nádobách a tři na keramice, které pocházejí převážně z thráckého vnitrozemí z kontextu aristokratických pohřbů.

Nápisy na předmětech z drahých kovů, případně na importované keramice kvalitního provedení\footnote{Tři nápisy na keramice byly taktéž nalezeny na thráckém území v monumentálních hrobkách: dva z nápisů pocházejí taktéž z nekropole u vesnice Duvanlij ({\em SEG} 47:1061,4 a {\em SEG} 47:1061,5) a jeden nápis pochází z nekropole města Apollónia Pontská na černomořském pobřeží \cite[{\em SEG}54]. Ve všech případech se jednalo o keramické nádoby řecké provenience, jako je attická hydria, dále střep keramického talíře nesoucí mužské jméno řeckého původu, a nakonec skyfos nesoucí řecké jméno a věnování Afrodíté. Předměty mohly původně sloužit jako obchodní artikl, či dar, a text nápisu nemusí mít žádnou souvislost se sekundární depozicí v hrobce, ani s majitelem hrobky. I přesto, že tyto nápisy nemají přímou výpovědní hodnotu k průběhu funerálního ritu samotného, jedná se o důkaz cirkulace předmětů mezi thráckým vnitrozemím a černomořským pobřežím.}, a tedy i vysoké hodnoty, byly nalezeny ve funerálním kontextu v thráckém vnitrozemí: předměty pocházejí z monumentálních hrobek, které se v 5. st. př. n. l. nacházely na území kmene Odrysů, nalezených v blízkosti moderní vesnice Duvanlij \cite[Filov {\em et al.}1934]. Monumentalita hrobek, kterou je možné spatřit ještě dnes, a nákladnost nalezené pohřební výbavy dává soudit, že se jednalo o významné jedince, pravděpodobně elitní členy kmene Odrysů (Archibald 1998, 154-171). Krátké nápisy na kovových nádobách jsou psány řeckou alfabétou a mají charakter soukromého nápisu. Protože předměty byly zhotoveny ze stříbra a ze zlata, pravděpodobně si je mohli dovolit jen nejbohatší členové tehdejší společnosti. Dochovaná jména jsou thráckého původu a bývají interpretována jako jména majitele hrobky a pohřební výbavy ({\em SEG} 46:871; Filov {\em et al.} 1934).\footnote{Konkrétně se jedná se o dva zlaté pečetní prsteny, stříbrné nádoby, součásti luxusního picího servisu.} Charakter dochovaných nápisů na kovových předmětech napovídá, že se jednalo předměty sloužící již za života majitele, které byly po jeho smrti přeneseny do hrobky jako hodnotný předmět, ukazatel společenského postavení majitele a důkaz prestiže v rámci komunity (Sahlins 1963; Whitley 1991, 354-361; Bliege Bird and Smith 2005, 221-222, 233-234).

Využití písma ve funerálním ritu se thráckém kulturním prostředí v 5. st. př. n. l. se poměrně zásadně odlišovalo od funkcí, které písmo zastávalo v řeckých komunitách na thráckém pobřeží. Docházelo-li ke kontaktu za účelem obchodní výměny zboží, nedocházelo ještě v této době k prolínání kulturních zvyklostí a společenského uspořádání, jak by se dalo očekávat. Pokud k nim přesto docházelo, výrazně se ale neprojevily na výsledné podobě a využití funerálních nápisů.

\subsection[dedikační-nápisy-1]{Dedikační nápisy}

Dedikačních nápisů je celkem pět a pocházejí z pobřeží Egejského moře: tři z Abdéry a dva ze Zóné. Nápisy jsou datovány do druhé poloviny 5. st. př. n. l. a jedná se výhradně o věnování řeckým božstvům, jako Afrodíté {\em Syria}, Hermés {\em Agoráios}, Pýthia a Hestiá. Věnování obsahují tradiční dedikační formule užívané v řecké epigrafické tradici, avšak pouze jeden nápis obsahuje výlučně řecká jména. Další nápisy obsahují jména velmi špatně dochovaná, u nichž bohužel není možné zjistit jejich původ. V jednom případě je nápis psán iónsko-attickým dialektem a v jednom případě je použita epichórická alfabéta z Thasu/Paru. Z těchto nepřímých důkazů je tak možné usuzovat, že dedikace pocházely převážně z řecky mluvících komunit v egejské oblasti. Dochované dedikační nápisy nasvědčují, že v 5. st. př. n. l. nedocházelo k prolínání řeckých a thráckých náboženských představ, ale tehdejší náboženství a jeho projevy na nápisech si udržovaly poměrně konzervativní přístup.

\subsection[veřejné-nápisy-1]{Veřejné nápisy}

Dochované tři veřejné nápisy reprezentují pouze 5 \letterpercent{} nápisů z daného souboru. Ve dvou případech se jedná o nápis vymezující hranice náboženského okrsku řeckých božstev Dia, Athény a {\em héróů} se jmény {\em Podalirios}, {\em Machaón} a {\em Periéstos}. Tyto nápisy pocházejí regionu řeckého města Strýmé na pobřeží Egejského moře. Nápis z Abdéry představuje velmi fragmentárně dochované nařízení vydané blíže neznámou politickou autoritou mezi lety 485 a 475 př. n. l. Na žádném z nápisů nebylo možné určit kontext komunity, z které nápis pocházel, vzhledem k chybějícím osobním jménům a vyjádřením identity. Místa nálezů pocházejí z regionu řeckých měst Abdéra, Strýmé, a tudíž se dá předpokládat, že byly vytvořeny zde žijícím řeckým obyvatelstvem pro interní potřeby řecké komunity. Použitý lokální materiál, jako je vápenec a mramor, více než nasvědčují omezení produkce veřejných nápisů pouze na řeckou komunitu v pobřežních oblastech.

\subsection[shrnutí-5]{Shrnutí}

Epigrafické památky z 5. st. př. n. l. pocházející z řeckých měst na egejském a černomořském pobřeží ukazují určitou míru konzervativismu vůči thráckým obyvatelům, alespoň dle obsahu dochovaných nápisů. Zcela odlišné pojetí funkce písma mezi řeckou a thráckou komunitou však nasvědčuje, že kontakty v 5. st. př. n. l. zůstávaly spíše na obchodní úrovni, a přenos kulturních zvyklostí byl i nadále minimální a omezoval se především na oblasti bezprostředně sousedící s řeckými městy na pobřeží. Nápisy v thráckém kontextu figurovaly pouze v elitních kruzích a jejich užití nasvědčuje o využití ryze pro soukromé účely prominentních jedinců. Oproti tomu v řeckém kontextu je epigrafická produkce rozšířena do větší části populace, nápisy jsou součástí života komunity a stejně tak tomu odpovídá i jejich obsah.

K vzájemným kontaktům Řeků a Thráků docházelo dle epigrafického materiálu ve velmi omezené míře v okolí Apollónie, která se nacházela v sousedství území thráckých kmenů a pravděpodobně sloužila jako středisko obchodní výměny mezi řeckým světem a thráckým obyvatelstvem. Není tedy vůbec překvapivé, že se thrácké obyvatelstvo objevuje i v onomastických záznamech pocházejících z Apollónie, což může dokazovat nově vznikající příbuzenské vztahy, či mísení onomastických tradicí obou komunit. Jedná se nicméně o první epigraficky postihnuté kontakty Thráků a Řeků na nearistokratické úrovni.

\stopcomponent