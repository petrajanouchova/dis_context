
\environment ../env_dis
\startcomponent section-charakteristika-epigrafické-produkce-ve-2.-až-1.-st.-př.-n.-l.
\section[charakteristika-epigrafické-produkce-ve-2.-až-1.-st.-př.-n.-l.]{Charakteristika epigrafické produkce ve 2. až 1. st. př. n. l.}

Nápisy datované do 2. a 1. st. př. n. l. pocházejí i nadále převážně z řeckého prostředí, nicméně se začíná projevovat vliv římské přítomnosti v regionu. To s sebou nese zvýšenou přítomnost římských jmen a stále pokračující nárůst počtu veřejných nápisů, v nichž se Řím objevuje jako mocný spojenec. Nápisy pocházejí převážně z pobřežních oblastí, avšak je možné pozorovat nárůst epigrafické aktivity v oblasti středního toku Strýmónu. Zapojení thrácké aristokracie na produkci epigrafických pramenů je pozorovatelné v malé míře.

\placetable[none]{}
\starttable[|l|]
\HL
\NC {\em Celkem:} 125 nápisů

{\em Region měst na pobřeží:} Abdéra 3, Apollónia Pontská 1, Byzantion 79, Maróneia 10, Mesámbriá 7, Odéssos 14, Perinthos (Hérakleia) 1, Sélymbria 3 (celkem 118 nápisů)

{\em Region měst ve vnitrozemí:} Beroé (Augusta Traiana) 2, (Marcianopolis) 1, údolí střední toku řeky Strýmónu 3, Hérakleia Sintská 1

{\em Celkový počet individuálních lokalit}: 21

{\em Archeologický kontext nálezu:} funerální 3, sídelní 1, sekundární 9, neznámý 112

{\em Materiál:} kámen 125 (mramor 118, z toho mramor z Prokonnésu 1, místní mramor 1, jiné 1; z čehož je varovik 1)

{\em Dochování nosiče}: 100 \letterpercent{} 8, 75 \letterpercent{} 14, 50 \letterpercent{} 10, 25 \letterpercent{} 8, kresba 1, nemožno určit 84

{\em Objekt:} stéla 119, architektonický prvek 4, jiné 1

{\em Dekorace:} reliéf 103, malovaná dekorace 1, bez dekorace 22; reliéfní dekorace figurální 81 nápisů (vyskytující se motiv: jezdec 2, stojící osoba 5, sedící osoba 6, skupina lidí 2, zvíře 5, funerální scéna/symposion 15, scéna oběti 4, jiný 2), architektonické prvky 24 nápisů (vyskytující se motiv: naiskos 6, sloup 1, báze sloupu či oltář 1, věnec 1, florální motiv 11, architektonický tvar/forma 2, jiný 2)

{\em Typologie nápisu:} soukromé 107, veřejné 14, neurčitelné 4

{\em Soukromé nápisy:} funerální 98, dedikační 11, jiné 1\footnote{Součet nápisů jednotlivých typů je vyšší než počet veřejných nápisů vzhledem k možným kombinacím jednotlivých typů v rámci jednoho nápisu.}

{\em Veřejné nápisy:} náboženské 1, seznamy 1, honorifikační dekrety 1, státní dekrety 6, neznámý 2

{\em Délka:} aritm. průměr 3,92 řádku, medián 3, max. délka 60, min. délka 1

{\em Obsah:} dórský dialekt 11; hledané termíny (administrativní termíny 16 - celkem 44 výskytů, epigrafické formule 13 - 46 výskytů, honorifikační 14 - 19 výskytů, náboženské 24 - 35 výskytů, epiteton 2 - počet výskytů 2)

{\em Identita:} řecká božstva, pojmenování míst a funkcí typických pro řecké náboženské prostředí, místní thrácká božstva, regionální epiteton 2, kolektivní identita 3 termíny, celkem 3 výskyty - obyvatelé řeckých obcí z oblasti Thrákie 1, ale i mimo ni 1, kolektivní pojmenování barbaroi 1; celkem 217 osob na nápisech, 86 nápisů s jednou osobou; max. 63 osob na nápis, aritm. průměr 1,73 osoby na nápis, medián 1; komunita řeckého charakteru se zastoupením římského a thráckého prvku, jména pouze řecká (56 \letterpercent{}), pouze thrácká (2,4 \letterpercent{}), pouze římská (1,6 \letterpercent{}), kombinace řeckého a thráckého (3,2 \letterpercent{}), kombinace řeckého a římského (8 \letterpercent{}), kombinace thráckého a římského (0,8 \letterpercent{}), jména nejistého původu (15,2 \letterpercent{}), beze jména (11,2 \letterpercent{}){\bf ;} geografická jména z oblasti Thrákie 0, geografická jména mimo Thrákii 7;

\NC\AR
\HL
\HL
\stoptable

Oproti nápisům datovaným do 3. až 2. st. př. n. l. je u skupiny nápisů datovaných do 2. až 1. st. př. n. l. pozorovatelný téměř 200 \letterpercent{} nárůst celkového počtu nápisů. Většina produkčních center se nachází na pobřeží, nicméně individuální nápisy byly nalezeny i v thráckém vnitrozemí, zejména v okolí řeky Strýmón, jak je možné vidět na mapě 6.05a v Apendixu 2. Hlavním produkčním centrem je Byzantion, avšak pozici menších produkčních center si i nadále udržují Odéssos a Maróneia.

Materiálem, z nějž jsou nápisy zhotovovány, je výhradně kámen a většina nápisů má tvar stély. Převládající funkce nápisů je funerální a v malé míře i dedikační. Objevují se i nápisy veřejné, ač v menším počtu než v předcházejících obdobích.

\subsection[funerální-nápisy-8]{Funerální nápisy}

Nejvíce z 98 funerálních nápisů pochází z Byzantia, celkem 78, což představuje téměř 80 \letterpercent{} všech funerálních nápisů z daného období. Z vnitrozemí pocházejí pouze tři nápisy, a to z údolí středního toku Strýmónu z regionu Hérakleie Sintské, podobně jako v předcházejícím období. Celkově dochází k poklesu počtů funerálních nápisů napříč celou Thrákií, s výjimkou řeckého Byzantia, kde dochází k poměrně markantnímu nárůstu. Podobně jako ve 2. st. př. n. l. zcela chybí sekundární funerální nápisy, které by bylo možné spojovat s thráckou aristokracií, což může svědčit o oslabení politické a ekonomické moci thráckých aristokratů či o proměnách přístupu Thráků k užití písma a k funerálním ritu obecně.

Skupina tří nápisů z thráckého vnitrozemí z regionu Hérakleie Sintské zaznamenává jména řeckého původu. Jedná se o funerální stély tří žen, jejichž otcové nesou taktéž jména řeckého (či makedonského) původu. Jak je již patrné v předcházejícím období, tato komunita si uchovala tradiční hodnoty, alespoň co se týče epigrafického projevu.

\subsection[dedikační-nápisy-8]{Dedikační nápisy}

Dedikačních nápisů se dochovalo celkem 11 a většina z nich pochází z pobřežních oblastí, až na jeden nápis z lokality Madara, která leží ve vnitrozemí zhruba ve vzdálenosti 70 km východně od Odéssu. Jména dedikantů jsou výhradně řeckého původu, až na jednu výjimku z regionu Topeiru, odkud pochází nápis {\em I Aeg Thrace} 105 se jménem pravděpodobně thráckého původu.

Vyskytující se epigrafické formule dosvědčují udržení řeckých tradic typických pro dedikační nápisy, avšak v menší míře než v předcházejícím období. Formule typické pro věnování nápisů se objevují zřídka: {\em charistérion} pouze v jednom případě, {\em euchén} pouze dvakrát. V jednom případě se dochovalo {\em enkómion} {\em I Aeg Thrace 205} určené egyptské bohyně Ísidě s délkou přes 44 řádků a původem z Maróneie. Jedná se na svou dobu o neobvyklý nápis jak formou, tak obsahem a poukazuje na trvající vliv řeckého náboženství v jižních oblastech Thrákie (Loukopoulou {\em et al.} 2005, 385).

Poprvé se v nápisech objevují i místní thrácké kulty, a i nadále se vyskytují věnování božstvům egyptského původu. Výskyt nápisů věnovaných místním božstvům je omezen na jeden region, či dokonce jednu svatyni v Thrákii, jako např. {\em hérós} {\em Karabasmos} či {\em Perkón} z Odéssu. Dále se zde objevují božstva řecká, jako např. Zeus {\em Hypsistos} či samothrácká božstva. Objevuje se opět i bohyně {\em Fosforos}, nejčastěji ztotožňovaná s Artemidou, Hekaté či Bendidou (Janouchová 2013a, 103-104). Podobně jako v předcházejícím století se vyskytují i dedikace původně egyptským božstvům Sarápidovi, Ísidě, Anúbidovi a Harpokratiónovi, a dále zbožštělým egyptským vládcům Ptolemaiovi (VI. nebo VII.) a Kleopatře. Obecně dochází k většímu prolínání náboženských systémů a představ a jejich zaznamenávání na permanentní médium nápisu, což může nasvědčovat na větší otevřenost společnosti, zvýšenou míru kulturního kontaktu a změny ve společnosti, které vyústily v proměnu náboženského systému, tedy obecně jedné z nejkonzervativnějších částí kultury.

\subsection[veřejné-nápisy-8]{Veřejné nápisy}

Celkem se dochovalo 14 veřejných nápisů, což značí propad oproti předcházejícím obdobím. Krom jednoho nápisu z thráckého Kabylé všechny nápisy pocházejí z okolí řeckých měst na pobřeží. Klesající počet nápisů může naznačovat na pokles moci publikujících politických autorit, či poukazuje na nedostatečnou míru prozkoumání kulturních vrstev 2. a 1. st. př. n. l. Nejčastější termíny jsou stále {\em démos}, {\em búlé}, {\em politai} a {\em pséfisma} a i nadále fungují dříve ustanovené procedury spojené se zhotovováním a vystavováním nápisů, nicméně klesající počet termínů označujících instituce může naznačovat jejich postupný úpadek či pozbytí významu v rámci fungování obce. Forma honorifikačních dekretů z Odéssu si udržela stejnou formu jako u nápisů datovaných do 2. st. př. n. l. a pravděpodobně vycházela ze stejných předpisů a pravidel. Honorifikační nápisy z Mesámbrie a Apollónie mají zcela jinou formu a používají jiné formule, což svědčí o nadále trvající autonomii řeckých měst a regionalismu.

Nápis {\em I Aeg Thrace} 212 představuje seznam věřících kultu Sarápida a Ísidy z Maróneie a objevuje se na něm až 75 jmen převážně řeckého původu, avšak i se sedmi jmény římskými a dvěma thráckými. Z přítomnosti osobních jmen je patrné, že kult byl oblíben převážně u mužů nesoucí řecká jména, nicméně byl přístupný i Thrákům a Římanům. Funkce knězů zastávali muži nesoucí řecká jména a z velké části nedocházelo k mísení onomastických tradic, tj. římská jména tvořila osobní jméno výhradně s dalším jedním či dvěma římskými jmény a nedocházelo k jejich prolínání s řeckými či thráckými jmény. Relativní izolovanost římských jmen svědčí o tom, že zvyk přijímat římská jména se v tomto období ještě neprosadil v podobě, jaká bude běžná v následujících stoletích.

\subsection[shrnutí-12]{Shrnutí}

Nápisy 2. až 1. st. př. n. l. dokumentují omezení publikačních aktivit thrácké aristokracie pro potřeby udržení společenského postavení v rámci komunity. Do regionu poprvé vstupuje se vší silou Řím, což se projevuje jak na obsahu a formě veřejných nápisů, narůstající variabilitě náboženských systémů, ale i na proměňující se skladbě osobních jmen. Římané jsou členy kultů a jsou na území Thrákie pohřbíváni, zejména pak v okolí Byzantia. Nedochází však k mísení onomastických tradic, ale Římané a Řekové si i nadále udržují kulturní odstup.

\stopcomponent