
\environment ../env_dis
\startcomponent section-charakteristika-epigrafické-produkce-ve-2.-st.-n.-l.-až-3.-st.-n.-l.
\section[charakteristika-epigrafické-produkce-ve-2.-st.-n.-l.-až-3.-st.-n.-l.]{Charakteristika epigrafické produkce ve 2. st. n. l. až 3. st. n. l.}

Nápisy datované do 2. a 3. st. př. n. l. pocházejí ze dvou třetin z vnitrozemí s největší produkcí v okolí Augusty Traiany, středního toku řeky Strýmónu a Filippopole. Poprvé dedikační nápisy převažují nad funerálními nápisy, což pravděpodobně souvisí i se zvyšujícím se počtem místních kultů a variabilitou vyobrazení božstev na nápisech. Složení společnosti je i nadále multikulturní, s dominantním římským prvkem, nicméně celkově dochází k většímu zapojení thrácké populace do epigrafické produkce.

\startDelimitedTable
 {\em Celkem}~ 182 nápisů

{\em Region měst na pobřeží}~ Anchialos 2, Byzantion 7, Dionýsopolis 2, Madytos 1, Maróneia 8, Mesámbriá 3, Odéssos 9, Perinthos (Hérakleia) 18, Sélymbria 3, Zóné 1 (celkem 54 nápisů)

{\em Region měst ve vnitrozemí}~ Augusta Traiana 49, Filippopolis 16, Hadriánopolis 2, Marcianopolis 5{\bf ,} Nicopolis ad Nestum 2, Nicopolis ad Istrum 11, Pautália 3, Plótinúpolis 4, Serdica 8, Traianúpolis 3, údolí středního toku řeky Strýmónu 21 (celkem 124 nápisů)\footnote{Celkem dva nápisy nebyly nalezeny v rámci regionu známých měst, editoři korpusů udávají jejich polohu vzhledem k nejbližšímu modernímu sídlišti, či uvádí muzeum, v němž se nachází. Další dva nápisy pocházejí z neznámého místa v Bulharsku.}

{\em Celkový počet individuálních lokalit}~ 61

{\em Archeologický kontext nálezu}~ funerální 1, sídelní 14 (z toho obchodní 7), náboženský 36, sekundární 22, neznámý 109

{\em Materiál}~ kámen 181 (mramor 138; vápenec 28, jiný 3, z toho syenit 2, varovik 1; neznámý 12), neznámý 1

{\em Dochování nosiče}~ 100 \letterpercent{} 21, 75 \letterpercent{} 27, 50 \letterpercent{} 27, 25 \letterpercent{} 44, kresba 13, ztracený 2, nemožno určit 48

{\em Objekt}~ stéla 136, architektonický prvek 36, socha 2, mozaika 1, jiný 5, neznámý 2

{\em Dekorace}~ reliéf 120, malovaná 2, bez dekorace 58; reliéfní dekorace figurální 73 nápisů (vyskytující se motiv: jezdec 33, sedící osoba 1, stojící osoba 5, skupina lidí 1, zvíře 1, Artemis 3, Asklépios 2, Héraklés 2, Zeus a Héra 1, scéna lovu 2, funerální scéna/symposion 7, funerální portrét 8, jiný 8), architektonické prvky 48 nápisů (vyskytující se motiv: naiskos 8, sloup 5, báze sloupu či oltář 21, architektonický tvar/forma 13, florální motiv 6, věnec 4, jiný 7)

{\em Typologie nápisu}~ soukromé 126, veřejné 46, neurčitelné 10

{\em Soukromé nápisy}~ funerální 54, dedikační 76, jiný 1\footnote{Několik nápisů mělo vzhledem ke své nejednoznačnosti kombinovanou funkci, proto je součet nápisů obou typů vyšší než celkový počet soukromých nápisů.}

{\em Veřejné nápisy}~ seznamy 3, honorifikační dekrety 30, státní dekrety 3, náboženský 4, jiný 3, neznámý 3

{\em Délka}~ aritm. průměr 5,0 řádku, medián 4, max. délka 48, min. délka 1

{\em Obsah}~ dórský dialekt 0, latinský text 4 nápisy, písmo římského typu 74; hledané termíny (administrativní termíny 31 - celkem 139 výskytů, epigrafické formule 24 - 128 výskytů, honorifikační 3 - 3 výskyty, náboženské 42 - 112 výskytů, epiteton 26 - počet výskytů 40)

{\em Identita}~ řecká božstva 19, egyptská božstva 4, římská božstva 2, pojmenování míst a funkcí typických pro řecké náboženské prostředí, nárůst počtu lokálních kultů, regionální epiteton 12, subregionální epiteton 14, kolektivní identita 13 termínů, celkem 24 výskytů - obyvatelé řeckých obcí z oblasti Thrákie 7, mimo ni 2; kolektivní pojmenování etnik či kmenů (Thráx 2, Rómaios 3, Bíthýnos 1, Acháios 1); celkem 256 osob na nápisech, 85 nápisů s jednou osobou; max. 24 osob na nápis, aritm. průměr 1,41 osoby na nápis, medián 1; komunita multikulturního charakteru se zastoupením řeckého, římského a thráckého prvku, se silnou přítomností římského prvku, jména pouze řecká (11,53 \letterpercent{}), pouze thrácká (6,59 \letterpercent{}), pouze římská (24,72 \letterpercent{}), kombinace řeckého a thráckého (4,39 \letterpercent{}), kombinace řeckého a římského (12,08 \letterpercent{}), kombinace thráckého a římského (2,74 \letterpercent{}), kombinovaná řecká, thrácká a římská jména (4,94 \letterpercent{}), jména nejistého původu (6,02 \letterpercent{}), beze jména (26,92 \letterpercent{}); geografická jména z oblasti Thrákie 12, geografická jména mimo Thrákii 4;
\stopDelimitedTable


V případě skupiny 182 nápisů datovaných do 2. a 3. st. n. l. se jedná o nárůst o 93 \letterpercent{} oproti skupině nápisů z 1. až. 2. st. n. l. Nápisy pocházejí z většiny z thráckého vnitrozemí z okolí hlavních městských center a okolí hlavních komunikačních tepen, jak je patrné z mapy 6.08a v Apendixu 2. Největším producentem se stává Augusta Traiana, jakožto hlavní město provincie {\em Thracia}. Producenty střední velikosti jsou další velká centra městského typu, jako Perinthos, Nicopolis ad Istrum a města z oblasti středního toku řeky Strýmónu.

Převážná část nosičů nápisů je z kamene, dva nápisy jsou zhotovené z kovu a jeden je na mozaice.\footnote{Nápisy na kovu představují vojenský diplom ({\em AE} 2007, 1259) a {\em instrumentum domesticum} čili předmět běžné denní potřeby ({\em AE} 2004, 1302, {\em instrumentum domesticum}, Chaniotis 2005, 92). Nápis {\em SEG} 54:652 se nachází na mozaice. Užitá osobní jména poukazují jak na řecký původ, tak na římské onomastické tradice u osob, které na nápisech figurují, ať už jako majitelé, či zhotovitelé. Nápisy pocházejí z okolí vojenského tábora v Kabylé a z vojenského tábora ve městě Nicopolis ad Istrum na Dunaji. Nápisy pocházejí nepřímo z kontextu spojeného s přítomností a projevy armádních struktur na území Thrákie. Vzhledem k jejich malému počtu a krátkému rozsahu však není možné vyvozovat nic dalšího.}

\subsection[funerální-nápisy-14]{Funerální nápisy}

Funerálních nápisů se dochovalo celkem 54, z čehož 52 má povahu soukromého nápisu a dva byly zhotoveny na náklady města.\footnote{{\em SEG} 48:894 a {\em I Aeg Thrace} 217.} Nápisy pocházejí z celého území Thrákie, tedy jak z pobřeží, tak z vnitrozemí. Hlavními produkčními centry jsou města v údolí středního toku Strýmónu, odkud pochází 14 nápisů, dále Perinthos s devíti nápisy a Maróneia se šesti nápisy.

Texty nápisů se pozvolna proměňují a dochází k opouštění tradičně používaných formulí. Stejně jako u skupiny nápisů z 2. st. n. l. se objevují nové termíny, které popisují nově vzniklé skutečnosti a předměty, jako termíny pro sarkofág a formule zajišťující právní ochranu jejich obsahu.\footnote{Invokační formule {\em chaire} se objevuje pouze čtyřikrát, oslovení okolojdoucího ({\em parodeita}) pouze čtyřikrát. Termín označující hrob ({\em tymbos}) se objevuje jednou, termín {\em stélé} čtyřikrát, {\em mnémeion} jednou, termín pro sarkofág celkem pětkrát ({\em soros, latomeion, diathéké}). Třikrát se objevuje invokační formule vzývající podsvětní bohy v latině ({\em Dis Manibus}) a ani jednou v řečtině. Celkem čtyři nápisy uvádí věk zemřelého, což je typický zvyk pro římské nápisy. Podobně jako u nápisů datovaných do 2. st. n. l. pochází skupina devíti sarkofágů zejména z lokalit v okolí Propontidy (Perinthos, Byzantion), z nichž čtyři nesou ochrannou formuli, která zakazuje nové použití sarkofágu pod peněžní pokutou. Osobní jména na těchto sarkofázích jsou převážně řeckého původu.} Obsah textů poukazuje na nadále se proměňující složení společnosti, či alespoň narůstající manifestaci nejrůznějších životních drah a profesí na nápisech. Stejně tak se objevují zvyklosti typické pro nápisy z římské doby, stejně jako u skupiny nápisů z 1. a zejména z 2. st. n. l.\footnote{Texty jsou zhotovovány členy nejbližší rodiny a hrobky slouží k pohřbům více členů rodiny. Celkem 26 nápisů zmiňuje společný hrob s partnerem zemřelého či zmiňuje členy rodiny. Vojáci se na nápisech objevují celkem třikrát, konkrétně jde o legionáře, jezdce a {\em carceraria}, hlídače vojenského vězení, a jednou o veterán. Jiná, než vojenská povolání zmiňují funkce vždy po jednom výskytu kněze ({\em archiereus}, {\em hiereus}), člena městské rady ({\em búleutés}), námořního obchodníka ({\em naukléros}) či patrona. Geografický původ zemřelého nápisy neudávají.}

Vyskytující se osobní jména jsou nadále převážně řecká, nicméně je možné pozorovat nárůst jak jmen římských, tak především i jmen thráckých. Řecká jména představují 38 \letterpercent{}, což představuje zhruba 12 \letterpercent{} pokles oproti nápisům z 1. až 2. st. n. l. K mísení thráckých a římských onomastických tradic dochází na funerálních nápisech zcela minimálně. Římská jména představují 30 \letterpercent{}, což značí 10 \letterpercent{} pokles. Thrácká jména naopak zaznamenala nárůst ze zhruba 3 \letterpercent{} až na 20 \letterpercent{}. Největší koncentrace devíti nápisů nesoucích thrácká jména pochází v údolí středního toku řeky Strýmónu a dále z okolí Augusty Traiany a Filippopole po dvou nápisech. Osoby nesoucí thrácká jména většinou odkazují na svůj původ i pomocí thráckého jména rodiče či dalších členů rodiny, případně i prohlášením o svém původu.\footnote{Příkladem je nápis {\em IG Bulg} 3,2 1794, který patří Apollónidovi, synovi Aulozénida ze Sapaiké, který je však pochován v thráckém vnitrozemí v lokalitě známé jako Dodoparon nedaleko středního toku řeky Tonzos (Janouchová 2016).}

\subsection[dedikační-nápisy-14]{Dedikační nápisy}

Dedikačních nápisů se dochovalo celkem 76, což představuje čtyřnásobný nárůst oproti nápisům z 1. až 2. st. n. l. Nejvíce nápisů pochází z regionu Augusty Traiany celkem se 33 nápisy, dále sem patří Serdica s osmi nápisy, obce z údolí středního toku řeky Strýmónu se sedmi nápisy a Filippopolis s šesti nápisy.

Dedikace jsou věnovány jak božstvům nesoucím řecká jména, tak božstvům místním či smíšeným.\footnote{Celkem se dochovalo 24 dedikací Apollónovi, šest Asklépiovi, tři Diovi a Artemidě, dvě Hygiei a Sabaziovi, po jedné Héře a Sarápidovi, Ísidě, Anúbidovi, Harpokratiónovi, Músám, Horám, Athéně, Héfaistovi, Matce bohů, Hérakleovi, Déméter a Télesforovi. Z místních božstev je to především {\em hérós}/{\em theos} nesoucí lokální přízviska či božstvo s lokálním přízviskem, jako {\em Marón, Poénos, Kersénos, Dabatopiénos, Zerdénos, Estrakénos, Raniskelénos, Zbelsúrdos, Aularkénos, Téradéenos} či {\em Kendreisos}.} Podle počtu výskytů na nápisech představuje Apollón nejpopulárnější božstvo, které bylo dle množství místních epitet rozšířeno do celé řady míst a do velké míry splynulo s místními kulty. Apollón se stal nejoblíbenějším božstvem a je mu určeno celkem 24 dedikací pocházejících z oblastí jižně od pohoří Haimos, zejména v okolí řeky Tonzos a Hebros, a dále v Sélymbrii a Serdice. Mezi jeho nejčastěji užívaná místní epiteta patří: {\em Foibos}, {\em Poénos}, {\em Kersénos}, {\em Zerdénos}, {\em Epékoos}, {\em Genikos}, {\em Geniakos}, {\em Patróos}, {\em Raniskelénos}, {\em Aularkénos}, {\em Lykios}, {\em Kendreisos} a {\em Téradéenos}, ale bývá oslovován též jako {\em hérós}, {\em theos} či {\em kyrios} (Goceva 1992; Bujukliev 1997). Největší počet dedikací pochází ze dvou velkých svatyní Apollóna s místním přízviskem {\em Téradéenos} a {\em Zerdénos}, nalézajících se u vesnice Kran v Kazanlackém údolí, nedaleko místa dřívějšího hellénisticko sídla Seuthopole (Tabakova 1959; Tabakova-Tsanova 1980).\footnote{Ač Seuthopolis zanikla již na začátku 3. st. př. n. l., je možné, že určité povědomí o řeckém náboženství zůstalo zakořeněno v místní populaci a postupem času se transformovalo do podoby místních kultů.} Jak dokazují dochovaná osobní jména dedikantů, kult Apollóna byl přístupný všem a relativně vysoký počet thráckých jmen může naopak naznačovat oblíbenost kultu Apollóna mezi místní populací.\footnote{Celkem se dochovala jména 12 mužů, z nichž čtyři nesli thrácká jména, čtyři pouze římská jména a tři řecká jména, jedno jméno nebylo možné určit vzhledem ke špatnému stavu dochování. Dedikanti byli ve dvou případech vojáci, v jednom případě člen {\em búlé} a v jednom případě se jednalo o dohlížitele nad agorou z Plotínopole.} Teorii o splynutí kultu Apollóna s lokálními kulty taktéž nahrává forma nosiče nápisů a jeho dekorace: 16 nápisů neslo reliéfní dekoraci zobrazující jezdce na koni, taktéž známého jako fenomén tzv. thráckého jezdce, který byl oblíben i mezi thráckými vojáky a veterány (Kazarow 1938; Dimitrova 2002; Boteva 2002; 2005; 2007; Oppermann 2006).

Podobný trend většího zapojení místní populace je možné sledovat i v případě dalších kultů božstev nesoucích řecká jména, jako je kult Asklépia.\footnote{Asklépiovi je věnováno celkem šest dedikací a božstvo je s oslovováno jako {\em kyrios}, {\em theos} či přízviskem {\em Liménios}. Asklépios je uctíván dohromady s Hygieí a Télesforem, a to lidmi se jmény řeckého i římského původu, nikoliv však thráckého. Nápisy pocházejí především ze svatyně u vesnice Slivnica v regionu města Serdica (Boteva 1985).} Pokud srovnáme poměr osobních jmen na dedikačních nápisech vůči poměru osobních jmen na všech nápisech z daného období, je možné sledovat nárůst jmen thráckého původu o 16,5\letterpercent{}. Thrácká jména na všech nápisech z 2. st. n. l. představují 12,5\letterpercent{} osobních jmen a na na dedikačních nápisech mají thrácká jména zastoupení až 29 \letterpercent{}.\footnote{U jmen řeckých je to pokles o 10 \letterpercent{} a u jmen římských pokles o 1,5 \letterpercent{}, nicméně i přesto římská jména představují přes 40 \letterpercent{} všech jmen na dedikačních nápisech a řecká jména jednu čtvrtinu.} Z čehož plyne, že oproti jiným druhům nápisů a s nimi spojených aktivit, Thrákové byli více epigraficky aktivní, pokud šlo o náboženské zvyklosti. Tomuto faktu odpovídá i současně pozorovatelný nárůst množství lokálních kultů, u nichž se předpokládá, že vznikly spojením místní tradice a řeckého náboženského systému již v dřívějších dobách, ale první epigrafické záznamy o jejich existenci pochází až z 2. st. n. l.

\subsection[veřejné-nápisy-14]{Veřejné nápisy}

Celkem se dochovalo 46 veřejných nápisů datovaných do 2. až 3. st. n. l., což představuje trojnásobný nárůst oproti skupině nápisů datovaných do 1. až 2. st. n. l. Nejvíce nápisů pochází z Augusty Traiany, celkem 10 nápisů, z Perinthu a Nicopolis ad Istrum, sedm nápisů a z Filippopole, pět nápisů. Nejčastějším typem s 29 výskyty jsou i nadále honorifikační nápisy vydávané městskými institucemi. Dochází k pokračujícím změnám epigrafického jazyka a standardizaci veřejných nápisů, stejně tak k proměně formulí typických pro veřejné nápisy. I nadále dekrety vydává {\em búlé} a {\em démos} pod patronátem císaře\footnote{Císař je označován termíny {\em kaisar} sedmkrát a {\em autokratór} 17krát.}, nicméně nyní se setkáváme i s přídomky {\em kratistos, hieros, hierotatos} či {\em lamprotatos}, které jsou spojovány s jednotlivými městskými institucemi. Podobného označení se městským institucím dostávalo i v římských provinciích Malé Asie, a to přibližně ve stejné době, což je příznakem jisté standardizace epigrafického jazyka napříč římskými provinciemi (Heller 2015, 250-253). K určitému sjednocení formy nápisů napříč městy dochází i pokud jde o uvádění císařských titulů, císařovy rodiny a zařazení nápisů do daného roku císařovy vlády.\footnote{Např. na nápise {\em IG Bulg} 2 617 a 624 z Nicopolis ad Istrum.} Na nápisech jsou často uváděni i vysocí provinciální úředníci, za jejichž služby došlo k vydání nápisu či např. byly prováděny stavební aktivity.\footnote{Termín {\em presbeutés} a {\em antistratégos} se vyskytuje sedmkrát, {\em thrakarchés} jednou, {\em búleutés} dvakrát, {\em efébarchos} jednou, {\em synedrion} jednou, {\em métropolis} jednou.}

Z dalších druhů nápisů se dochoval pouze jeden milník, stejně tak jako jeden nápis dokumentující stavební aktivity veřejného charakteru, či označení hranic regionu města\footnote{{\em Perinthos-Herakleia} 38, 40 a {\em IG Bulg} 5 5540.} a dva seznamy osob, které dobře zachycují onomastické zvyky ve vztahu ke kolektivní identitě. Na nápisech {\em IG Bulg} 1,2 50 a 51 obsahujících seznam věřících boha Dionýsa a seznam efébů z Odéssu zcela převládají řecká a thrácká jména. To může být důsledek kulturněspolečenských norem, které měly za následek, že osoby nesoucí římská jména do těchto náboženských skupin nevstupovaly, případně v daném kontextu vystupovaly pod jinou součástí své identity a záměrně vypoustěly římské jméno.

\subsection[shrnutí-18]{Shrnutí}

Ač u nápisů 2. až 3. st. n. l. epigraficky aktivní populace zůstává převážně řecká, dochází k nárůstu římského elementu, ale zejména i k sebeuvědomění a zapojení thráckého obyvatelstva. Thrákové se objevují zejména v souvislosti s vojenskou službou a nápisy využívají jako formu prezentace společenského statutu. Thrácký prvek se objevuje zejména v místních svatyních, které jsou ale přístupné všem obyvatelům. Nadále dochází i k upevňování zvyklostí a jejich epigrafických projevů, které se poprvé objevily až s římskou přítomností, jako např. rozšíření funerálních sarkofágů či uvádění věku zemřelého. Veřejné nápisy pak poukazují na míru regionalismu a autonomie jednotlivých městských samospráv, avšak zaštítěného všudypřítomnou autoritou římského císaře a provinciálních institucí.

\stopcomponent