
\environment ../env_dis
\startcomponent section-charakteristika-epigrafické-produkce-ve-2.-st.-n.-l.
\section[charakteristika-epigrafické-produkce-ve-2.-st.-n.-l.]{Charakteristika epigrafické produkce ve 2. st. n. l.}

Ve 2. st. n. l. dochází k prudkému nárůstu epigrafické aktivity, a to zejména ve vnitrozemí, odkud pochází více nápisů než z pobřeží. Institucionální uspořádání a politický vliv Říma mají zásadní vliv na rozšíření epigrafické produkce. Veřejné nápisy představují téměř polovinu celého souboru, častější je i výskyt hledaných termínů a latinského textu. Římská jména se vyskytují na polovině všech nápisů, ať už samostatně, či v kombinaci. Dochází k prolínání kulturních a náboženských tradic, objevují se ale i nové prvky a jména kultů. Politická příslušnost a status nabírají na důležitosti, a proto narůstá i počet honorifikací.

\placetable[none]{}
\starttable[|l|]
\HL
\NC {\em Celkem:} 254 nápisů

{\em Region měst na pobřeží:} Abdéra 5, Anchialos 2, Apollónia Pontská 1, Byzantion 22, Dionýsopolis 1, Kallipolis 2, Madytos 1, Maróneia 19, Mesámbriá 1, Odéssos 17, Perinthos (Hérakleia) 21, Sélymbria 1, Topeiros 4, Zóné 1 (celkem 98 nápisů)

{\em Region měst ve vnitrozemí:} Augusta Traiana 23, Carasura 1, Discoduraterae 3, Filippopolis 29, Hadriánopolis 1, Hérakleia Sintská 2, Marcianopolis 4, Neiné 1, Nicopolis ad Istrum 28, Pautália 5, Plótinúpolis 2, Serdica 18, Traianúpolis 2, údolí středního toku řeky Strýmónu (Parthicopolis a okolí) 30 (celkem 149 nápisů)\footnote{Celkem sedm nápisů nebylo nalezeno v rámci regionu známých měst, editoři korpusů udávají jejich polohu vzhledem k nejbližšímu modernímu sídlišti, či uvádí muzeum, v němž se nachází.}

{\em Celkový počet individuálních lokalit}: 69

{\em Archeologický kontext nálezu:} funerální 5, sídelní 43 (z toho obchodní 10), náboženský 5, sekundární 25, jiný 3, neznámý 173

{\em Materiál:} kámen 246 (mramor 169, z Chalkedónu 1, z Prokonnésu 1; vápenec 49, jiný 10, z toho syenit 5, póros 1; neznámý 18), kov 2, neznámý 6

{\em Dochování nosiče}: 100 \letterpercent{} 36, 75 \letterpercent{} 35, 50 \letterpercent{} 44, 25 \letterpercent{} 43, oklepek 2, kresba 8, ztracený 8, nemožno určit 78

{\em Objekt:} stéla 148, architektonický prvek 89, socha 2, jiný 8 (z toho {\em instrumentum domesticum} 1, vojenský diplom 1) neznámý 7

{\em Dekorace:} reliéf 144, bez dekorace 110; reliéfní dekorace figurální 44 nápisů (vyskytující se motiv: jezdec 8, stojící osoba 6, skupina lidí 1, zvíře 1, Artemis 1, scéna lovu 1, funerální scéna/symposion 10, funerální portrét 9, jiný 3), architektonické prvky 99 nápisů (vyskytující se motiv: naiskos 9, sloup 9, báze sloupu či oltář 45, architektonický tvar/forma 18, geometrický motiv 4, florální motiv 18, věnec 2, jiný 7)

{\em Typologie nápisu:} soukromé 130, veřejné 117, neurčitelné 7

{\em Soukromé nápisy:} funerální 84, dedikační 49, vlastnictví 1, jiný 2\footnote{Několik nápisů mělo vzhledem ke své nejednoznačnosti kombinovanou funkci, proto je součet nápisů obou typů vyšší než celkový počet soukromých nápisů.}

Veřejné nápisy: seznamy 2, honorifikační dekrety 63, státní dekrety 6, nařízení 5, náboženský 13, jiný 24, neznámý 4

Délka: aritm. průměr 6,52 řádku, medián 5, max. délka 62, min. délka 1

{\em Obsah:} dórský dialekt 2, latinský text 22 nápisů{\bf ,} písmo římského typu 85; hledané termíny (administrativní termíny 40 - celkem 284 výskytů, epigrafické formule 29 - 199 výskytů, honorifikační 10 - 13 výskytů, náboženské 46 - 156 výskytů, epiteton 26 - počet výskytů 33)

{\em Identita:} řecká božstva 19, egyptská božstva 2, římská božstva 2, thrácká božstva 2, pojmenování míst a funkcí typických pro řecké náboženské prostředí{\bf ,} regionální epiteton 15, subregionální epiteton 11, kolektivní identita 18 termínů, celkem 55 výskytů - obyvatelé řeckých obcí z oblasti Thrákie 12, mimo ni 0; kolektivní pojmenování etnik či kmenů (Thráx 13, Rómaios 9, barbaros 1, Asianos 1, Kappadox 1), člen fýly 1; celkem 511 osob na nápisech, 94 nápisů s jednou osobou; max. 29 osob na nápis, aritm. průměr 2,02 osoby na nápis, medián 1{\bf ;} komunita multikulturního charakteru se zastoupením řeckého, římského a thráckého prvku, se silnou přítomností římského prvku, jména pouze řecká (11,06 \letterpercent{}), pouze thrácká (1,18 \letterpercent{}), pouze římská (31,25 \letterpercent{}), kombinace řeckého a thráckého (3,95 \letterpercent{}), kombinace řeckého a římského (15,81 \letterpercent{}), kombinace thráckého a římského (3,95 \letterpercent{}), kombinovaná řecká, thrácká a římská jména (6,32 \letterpercent{}), jména nejistého původu (12,23 \letterpercent{}), beze jména (14,22 \letterpercent{}); geografická jména z oblasti Thrákie 15, geografická jména mimo Thrákii 4;

\NC\AR
\HL
\HL
\stoptable

Do 2. st. n. l. bylo datováno celkem 254 nápisů, což představuje nárůst o 272 \letterpercent{} oproti předcházejícímu období. Jak je patrné z mapy 6.08a v Apendixu 2, poprvé v tomto období převládá epigrafická produkce ve vnitrozemí, a nikoliv na pobřeží. Většina lokalit s nápisy ve vnitrozemí se nacházela v povodí velkých řek či na trase {\em Via Diagonalis}, římské cesty spojující severozápad s jihovýchodem, procházející skrz města Serdicu, Filippopolis, Perinthos a Byzantion (Jireček 1877; Madzharov 2009, 70-31). Nápisy pocházely převážně z níže položených sídel, v horských oblastech se našly jednotlivé nápisy pouze v oblasti cest a průsmyků či v okolí vojenských posádek. Hlavními produkčními centry byla oblast údolí středního toku Strýmónu s městy Hérakleia Sintská, Neiné a Parthicopolis\footnote{Parthicopolis je známá též jako Paroikopolis a nachází se pod moderním městem Sandanski v Bulharsku. Dle Mitreva (2017, 106-108) na tomtéž místě existovala thrácká vesnice Desudaba, která byla později transformována do makedonské kolonie Alexandropolis. Vztah mezi Alexandropolí a Parthicopolí není zcela jasný, nicméně první archeologické nálezy a nápisy pocházejí až z 1. a zejména z 2. st. n. l., tedy z římské doby. Parthicopolis byla epigraficky aktivní i v průběhu 3. st. n. l. (Mihailov 1997, 401).}, a dále v oblasti střední Thrákie s městy Filippopolis, Nicopolis ad Istrum, Augusta Traiana, a na pobřeží Propontidy ve městech Byzantion a Perinthos. Oproti předcházejícím obdobím dochází k přesunu epigrafické produkce z jednoho dominantního administrativního centra do většího počtu měst. Oproti 1. st. n. l. upadá celková produkce Byzantia a Perinthu, což souvisí s přesunem hlavního města provincie do Filippopole, kde je naopak možné pozorovat markantní nárůst nápisů (Topalilov 2012, 13).

Použitým materiálem je výhradně kámen, a to zejména mramor ze 70 \letterpercent{}, vápenec z 18 \letterpercent{} a syenit z 2 \letterpercent{}. Kamenné nosiče nápisů mají ze dvou třetin tvar stély, v necelé třetině tvar architektonického prvku. Ve zbylých případech se čtyři nápisy dochovaly na sochách, jeden na mozaice a 19 na sarkofázích či jejich fragmentech.

\subsection[funerální-nápisy-13]{Funerální nápisy}

Funerálních nápisů se dochovalo celkem 84, z čehož 80 má povahu soukromého nápisu a čtyři byly zhotoveny na náklady města, konkrétně Maróneie. Oproti předcházejícímu období počet funerálních nápisů narostl více než třikrát, což je možné spojovat s nárůstem celkového počtu obyvatel, ale i s proměnami přístupu obyvatelstva k zřizování funerálních nápisů. Nápisy pocházejí z celého území Thrákie, tedy jak z pobřeží, tak z vnitrozemí. Hlavními produkčními centry je údolí středního toku Strýmónu s městem Hérakleia Sintská, Neiné a Parthicopolis, odkud pochází 25 nápisů, dále Byzantion s 18 nápisy, Perinthos a Maróneia s 11 nápisy. Obecně funerální nápisy pocházejí z okolí velkých městských center, kde žilo nejvíce lidí.

Texty funerálních nápisů poukazují na proměňující se složení společnosti a nárůst důležitosti římského vojska. Texty nesou standardní formule, podobně jako v předcházejících stoletích\footnote{Invokační formule {\em chaire} se objevuje 16krát, oslovení okolojdoucího ({\em parodeita}) 11krát. termín označující hrob ({\em tymbos}) se objevuje jednou, termín {\em stélé} třikrát, termín {\em mnémé} 15krát, {\em mnémeion} čtyřikrát, termín pro sarkofág celkem třikrát ({\em soros}). Také se objevuje termín {\em chamosorion}, popisující pravděpodobně sarkofág umístěný na podstavci či přímo na zemi. Poprvé se čtyřikrát objevuje invokační formule vzývající podsvětní bohy v latině ({\em Dis Manibus}) a dvakrát v řečtině ({\em Theoi Katachthonioi}).} a jsou zhotovovány členy nejbližší rodiny a hrobky slouží k pohřbům více členů rodiny.\footnote{Celkem 22 nápisů vyjmenovává několik členů rodiny, ve čtyřech případech nápisy zmiňují potomky a sourozence, v jednom případě rodiče a ve dvou přítele. Ve dvou případech se setkáváme s nápisem zhotoveným propuštěným otrokem pro svého bývalého pána.} Celkem 20 nápisů, pocházejících zejména z Byzantia a Perinthu, uvádí věk zemřelého, což je zvyk typický pro římské nápisy. Latinský text se objevil na šesti nápisech, většinou u nápisů patřícím vojákům či veteránům, kteří nesli římská jména. Vojáků se na nápisech objevuje celkem pět, od běžných vojáků až po legionáře a jezdce. Jiná, než vojenská povolání zmiňují funkce vždy po jednom výskytu, a to kněze ({\em hiereus}), člena městské rady ({\em búleutés}), člena {\em gerúsie} ({\em gerúsiastés}). Na nápisech figurují taktéž čtyři otroci, z čehož dva jsou propuštěnci. Sarkofágy, podobně jako v 1. st. n. l., pocházejí převážně z Perinthu (Hérakleii) a Byzantia a dalších pobřežních lokalit, ale dva pocházejí i z thráckého vnitrozemí. Z osmi sarkofágů z Perinthu (Hérakleie) a Byzantia pouze dva nesou ochrannou formuli, která zakazuje nové použití sarkofágu pod peněžní pokutou\footnote{Tato formule se objevila již na sarkofázích z Perinthu (Hérakleii) v 1. st. n. l. a tento zvyk se rozšířil do nedalekého Byzantia.}, ale téměř na všech se udává věk zemřelého, případně jeho původ a kdo nechal sarkofág zhotovit.

Geografický původ zemřelého dokládá zvýšenou migraci z oblastí Malé Asie, což je patrné zejména u nápisů pocházejících z Filippopole (Topalilov 2012, 13; Sharankov 2011, 143).\footnote{Celkem sedm nápisů udává geografický původ zemřelého, a to jako pocházejícího z Byzantia, Abdéry (dvakrát), Hérakleie, Filippopole, Kappadokie, Níkaie a Apameie v Bíthýnii.} Vyskytující se osobní jména i jsou nadále převážně řecká, nicméně je možné pozorovat nárůst jak jmen římských, tak především i jmen thráckých. Řecká jména představují 42 \letterpercent{}, thrácká jména představují 18 \letterpercent{}, římská jména představují necelou třetinu všech jmen. Thrácká jména se vyskytují jak samostatně, tak v malé míře i v kombinaci s řeckými i římskými jmény. Celé dvě třetiny nápisů s thráckým jménem pocházejí z údolí středního toku Strýmónu, kde se thrácká populace zapojovala do zvyku zhotovovat veřejně vystavené funerální nápisy zejména v první polovině 2. st. n. l.\footnote{Z celkem 22 nápisů v nichž se vyskytuje alespoň jedno thrácké jméno jich 14 pochází z okolí řeky Strýmónu. Celkem se zde vyskytuje 40 osob, z nichž osm jsou ženy} Římská jména se vyskytují převážně samostatně, v menší míře v kombinaci s řeckými a téměř minimálně v kombinaci s thráckými jmény. V kombinaci s thráckými jmény se vyskytují většinou u jedinců, kteří sloužili v římské armádě a přijali nové jméno, nicméně i nadále odkazují na svůj thrácký původ.\footnote{Např. na nápise {\em IG Bulg} 5 5462 z Filippopole Ailios Polemón, {\em benefikarios}, věnoval svým předkům Beithytraleiovi, synovi Taséa(?) a Kouété, dceři Dydéa(?). Nápis Manov 2008 82 dokumentuje případ veterána se jmény Gaios Valerios Púdens {[}Poudens{]}, jehož otec nesl jméno Gaios Virginios {[}Ourginios{]} Púdens {[}Poudens{]} a matka jméno Múkasoké {[}Moukasoké{]}. Veteránova žena nesla jméno Severa, syn jméno Ioulios Maximos, a sestra Zaikaidenthé.} Kombinovaná jména, která by odkazovala na nedávné přijetí římského jména a s ním i spojených privilegií, se na rozdíl od 1. st. n. l. dochovala pouze v minimální míře. To může značit, že Thrákové nedostávali za svou službu právo nosit římská jména, či se plně adaptovali na systém tří jmen, zcela opustili užívání původních thráckých jmen, a stali se tak nerozlišitelnými od Římanů. Nicméně vzhledem k udržení thráckých jmen na nápisech ze 3. st. n. l. je toto vysvětlení nedostačující.

\subsection[dedikační-nápisy-13]{Dedikační nápisy}

Dedikačních nápisů se dochovalo celkem 49, z čehož 44 bylo soukromé povahy a pět bylo zhotoveno jako veřejný text, tj. dedikace císaři či text zhotovený na náklady obce. Nejvíce nápisů pochází z regionu Augusty Traiany, celkem s osmi nápisy, dále z údolí středního toku Strýmónu sedm nápisů, poté je Serdica s pěti nápisy.

Dedikace jsou věnovány jak božstvům nesoucím řecká jména, tak božstvům místním, případně božstvům, jež vznikla smíšením řeckých a místních tradic.\footnote{Objevuje se osm Asklépiovi, sedm Diovi, šest Dionýsovi a Artemidě, pět Apollónovi, jedna Hygiei, Sabaziovi, Sarápidovi, Héře, Athéně a Semelé.} Místní epiteta vznikla pravděpodobně ze jména místa, kde se vyskytovala svatyně.\footnote{Z místních božstev je to především hérós/theos {\em Karabasmos}, {\em Manimazos}, {\em Marón}, {\em Pyrmerúlás}, {\em Zbelsúrdos} a dále místní přízviska spojená s Apollónem a Asklépiem a Artemidou, jako např. {\em Aulariokos}, {\em Epékoos}, {\em Tadénos}, {\em Beeuchios}, {\em Patróos}, {\em Suidénos} (Janouchová 2016 k rozšíření přízviska {\em Patróos}).} Lokální thrácké kulty se vyskytovaly na více než jednom místě pouze v ojedinělých případech, většinou se jednalo kult pevně svázaný s daným místem.

Přes polovinu osobních jmen na dedikačních nápisech tvořila římská jména, třetinu jména řecká a zhruba 10 \letterpercent{} jména thrácká. Nápisy s thráckými jmény pocházely výhradně z vnitrozemských oblastí z okolí Augusty Traiany a středního toku Strýmónu a byly věnovány jak řeckým božstvům, tak řeckým božstvům nesoucím místní přízviska.\footnote{Příkladem nápisů věnovaných řeckému božstvu s místním přízviskem jsou dva nápisy z Kabylé, kde byl ve 2. st. n. l. umístěný vojenský tábor pomocných jednotek a pravděpodobně zde žili Thrákové společně s Řeky. Nápisy {\em Kabyle} 18 a 19 nesou věnování Apollónovi {\em Tadénovi}, provedené členy {\em cohors II Lucensium}, kteří sami nesli thrácká osobní jména (Velkov 1991, 23-24).} Zhruba polovina těchto dedikací byla věnována Thráky sloužícími v římské armádě, kteří zároveň nesli římská a thrácká jména.

\subsection[veřejné-nápisy-13]{Veřejné nápisy}

Poprvé větší část veřejných nápisů pochází z vnitrozemí, a to zejména z velkých městských center a jejich nejbližšího okolí. Tento fakt jistě souvisí s urbanizačními aktivitami a posílením politické moci a významu městské samosprávy za vlády Trajána a Hadriána (Topalilov 2012, 14-15). Celkem se dochovalo 117 veřejných nápisů, což představuje oproti 1. st. n. l. zhruba čtyřnásobný nárůst celkového počtu. Tento trend je pozorovatelný zejména v oblasti vnitrozemské Thrákie v okolí měst Nicopolis ad Istrum a Filippopolis, odkud pochází 21 nápisů a Augusta Traiana se 14 nápisy. Nejvíce se dochovalo honorifikačních nápisů, ale objevují se ve větší míře i milníky a nápisy dokumentující stavební aktivity.

Honorifikační nápisy představují nejvýznamnější skupinu veřejných nápisů. Oproti předcházejícímu období se jedná o téměř osminásobný nárůst z osmi na 63 honorifikačních nápisů, což dokazuje oblíbenost tohoto druhu nápisů. Honorifikační nápisy pocházejí z měst jak na pobřeží, tak zejména ve vnitrozemí z regionu města Nicopolis ad Istrum, odkud pochází 18 textů, a z regionu Filippopole, kde bylo nalezeno 14 textů. Honorifikační dekrety jsou i nadále vydávány politickou autoritou města, a to pod patronátem římského císaře.\footnote{Nejčastějšími termíny reprezentujícími politickou autoritu měst jsou {\em búlé} s 24 výskyty, {\em démos} s 25, {\em polis} se 7 výskyty.} Tradiční forma honorifikačních nápisů, kdy se dobrodinci ({\em euergetés}) udělují konkrétní privilegia za vykonané dobrodiní, je nicméně vystřídána novou formou, kdy namísto významného jedince je hlavní pozornost věnována císaři, na jehož počet většina nápisů a s nimi spojených monumentů a budov vzniká.\footnote{Texty obsahují tradiční invokační formuli ve 27 případech ({\em agathé týché}) a jsou převážně psány řecky. Dva texty jsou psány latinsko-řecky, kde latinský text je uváděn první a řecký text je překladem latinského textu a hlavní autoritou těchto nápisů je vždy římský císař. Další dva nápisy jsou čistě latinské a jsou určeny přímo římskému císaři.} Tato proměna honorifikačních nápisů je patrná již v průběhu 1. st. n. l. i v jiných částech římské říše (Van Nijf 2015, 240). Důraz je kladen převážně na osobní kvality a zásluhy jedinců, jejich pověst a společenskou prestiž, což může odrážet i větší míru stratifikace tehdejší společnosti.

Milníky a nápisy dokumentující stavební aktivity organizované provinciální samosprávou poukazují na nárůst stavebních aktivit v 2. st. n. l., zejména po roce 116 n. l. Celkem devět milníků nesoucí řecké nápisy pochází jak z okolí významných měst ve vnitrozemí, tak na pobřeží. Silnice {\em Via Diagonalis} byla v této době doplněna dalšími menšími silnicemi spojujícími významná města právě s touto nejdůležitější dopravní tepnou celého Balkánu. Dochované milníky poukazují na existenci cest v druhé polovině 2. st. n. l. v okolí Marcianopole a Odéssu, dále na probíhající opravy {\em Via Diagonalis}, která spojovala Serdicu a Byzantion. Silnice využívala primárně římská armáda, nicméně z této infrastruktury těžilo i místní obyvatelstvo a silnice umožnily lepší propojení regionu a pohyb obyvatel. \footnote{Císař mohl udělit speciální práva a ochranu konkrétní skupině obyvatel, jak je tomu na nápise {\em I Aeg Thrace} 185 z Maróneie. Císař Hadrián tímto dekretem určeným obyvatelům Maróneie z roku 131 n. l. mimo jiné udělil právo k užívání a ochranu po dobu putování po cestě z Maróneii do Filipp, čímž se pravděpodobně myslela {\em Via Egnatia}, jedna z nejvýznamnějších římských cest. Na civilní využití menších cest a mostů v okolí Pautálie poukazuje nápis {\em SEG} 54:648 z poloviny 2. st. n. l.}

Další nápisy zaznamenávají stavbu akvaduktů v Odéssu, městského opevnění ve Filippopoli a Serdice a lázní v Pautálii a Augustě Traianě či rekonstrukci divadla ve Filippopoli. Dva dochované nápisy označují hranice území Abdéry jako samosprávné jednotky pod autoritou císaře.\footnote{{\em I Aeg Thrace} 78 a 79. Předpokládá se, že hraniční kameny se vyskytovaly na pomezí území měst vcelku často, do dnešních dnů se jich dochovalo bohužel pouze několik exemplářů.} Stavební aktivity většinou financovali místodržící či vysocí úředníci na počest císaře a soudě dle osobní jmen, byli tito vysoce postavení muži zejména římského původu, ale vyskytovali se i jedinci nesoucí kombinaci thráckých a římských jmen.\footnote{Jako např. Titus Vitrasius Pollio z nápisu {\em IG Bulg} 1,2 59 a Poplios Ailios Aulouporis z nápisu {\em IG Bulg} 5 5334.}

Veřejným textům náboženského charakteru zcela dominuje císařský kult, v seznamech věřících převládají řecká a thrácká jména.\footnote{Na nápise {\em IG Bulg} 4 1925 se setkáváme se seznamem příslušnic mystérií Velké Matky a Attida, kde se vyskytují jak řecká, tak římská ženská jména, avšak text je poměrně špatně dochovaný.} Osoba císaře hraje zásadní roli i v rozložení politické moci.\footnote{Termíny označující autoritu císaře jsou s 30 výskyty {\em autokratór}, dále s 26 výskyty {\em kaisar}, s 12 {\em hypatos} a se 17 {\em hegémón}. Císař bývá někdy titulován i jako otec vlasti, {\em patér patridos}, a {\em archiereus}, což je řecká verze titulu {\em pontifex maximus} (Mason 1974, 115-116).} Autoritu císaře přímo v provincii reprezentuje místodržící a jeho zástupce, na nápisech často titulovaný jako {\em presbeutés} a {\em antistratégos}, latinsky {\em legatus Augusti pro praetore} (Mason 1974, 153-155), a to konkrétně ve 24 případech. Jako zástupce císařské moci vystupuje v několika případech i {\em thrakarchés}, tedy vysoký úředník s náboženským zaměřením, podobným nejvyššímu knězi celé provincie (Lozanov 2015, 82-83). Úřad {\em thrakarcha} obvykle zastupoval muž nesoucí kombinovaná thrácká a římská jména, což nepřímo poukazuje na jeho aristokratický thrácký původ. Jak je však patrné, thrácký původ v této době nebyl překážkou kariéry v rámci provinciálních institucí i na poměrně vysokých pozicích (Sharankov 2005b, 532).

S dobou vlády Trajána se taktéž spojují administrativní reformy, díky nimž narostla autorita měst ve vnitrozemí na úkor dřívějšího uspořádání založeného na kmenovém principu, což se projevilo i na výskytu termínů užívaných na nápisech.\footnote{Parissaki (2013, 83-84) se domnívá, že dříve fungující systém stratégií byl za doby vlády Trajána nahrazen novým administrativním systémem, v němž hrála zásadní roli městská samospráva.} Z jednotlivých úřadů či osob zastávajících funkce v rámci samosprávy se objevuje {\em búlé} ve 34 případech {\em démos} ve 42, {\em polis} ve 24, dále {\em archón} na čtyřech nápisech, na dvou nápisech {\em fýlé} a {\em pontarchés}\footnote{Úřad {\em pontarcha} je znám z císařské doby a jedná se o nejvyššího úředníka uskupení šesti měst na pobřeží Černého moře, který zastupoval císařskou autoritu a zároveň vykonával úřad nejvyššího kněze (Cook 1987, 30). Toto uskupení černomořské Hexapole zahrnovalo pobřežní města převážně z provincie {\em Moesia Inferior} jako Histria, Tomis, Kallatis, Dionýsopolis, Odéssos a Mesámbriá a usuzuje se, že vzniklo v době vlády císaře Trajána či Hadriána, tedy před polovinou 2. st. n. l. Vytvořením této organizace původně řecká města fakticky ztratila svou autonomii, či její pozůstatky, a spadala přímo pod autoritu císaře (Lozanov 2015, 83).} a na jednom nápise {\em agaranomos} a {\em gerúsia}. V rámci posílení pozice se města z Thrákie společně sdružovala v tzv. {\em koinon tón Thrákón}, s Filippopolí jako hlavním městem a sídlem spolkového sněmu ({\em métropolis}; Sharankov 2005, 518-531).\footnote{Přesná funkce tohoto uskupení měst není jasná, nicméně jeho vznik byl pravděpodobně inspirován vzorem ze sousední Bíthýnie pro usnadnění provinciální administrativy (Lozanov 2015, 82).} Politické centrum tehdejší provincie se tak přesunulo z Perinthu do vnitrozemské Filippopole, čemuž odpovídá i přesun jednoho z největších producentů veřejných, tak i soukromých nápisů.

\subsection[shrnutí-17]{Shrnutí}

Na nápisech z 2. st. n. l. lze velmi dobře pozorovat rostoucí vliv Říma nejen na celkovou epigrafickou produkci, ale i na proměňující se strukturu společnosti. Zvyšující se epigrafická produkce souvisí s narůstající společenskou komplexitou, která se mimo jiné projevuje narůstající administrativou a institucionální zátěží spojenou s vedením velké říše. To se projevuje jak větším počtem veřejných nápisů regulujících společenské uspořádání římské provincie, ale i výsadní pozicí císaře, coby svrchované politické autority. Rostoucí role vojska je patrná jak na obsahu nápisů, často publikovaných vojáky či veterány, proměňujícími se zvyklostmi a přejímání nových vzorců chování, ale i samotným rozmístěním epigraficky aktivních komunit. Patrná je koncentrace epigrafické produkce v okolí městských center, podél vojenských cest a v okolí vojenských a polovojenských osídlení.

\stopcomponent