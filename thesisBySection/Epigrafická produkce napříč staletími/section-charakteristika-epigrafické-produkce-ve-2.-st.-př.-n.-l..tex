
\environment ../env_dis
\startcomponent section-charakteristika-epigrafické-produkce-ve-2.-st.-př.-n.-l.
\section[charakteristika-epigrafické-produkce-ve-2.-st.-př.-n.-l.]{Charakteristika epigrafické produkce ve 2. st. př. n. l.}

Nápisy datované do 2. st. př. n. l. pocházejí výhradně z měst na pobřeží. Mezi oblasti, kde se nápisy začínají objevovat nově, patří Thrácký Chersonésos a lokality na pobřeží Marmarského moře. Téměř pětinu nápisů představují nápisy veřejné, a to zejména dekrety vydávané politickou autoritou. Dochází také k nárůstu výskytů hledaných slov, nápisy se stávají delší a obsahově komplexnější. Celkově dochází k většímu otevírání původně řeckých komunit na pobřeží a k výskytu nových prvků. Zároveň s tím ale dochází k útlumu epigrafických aktivit thrácké aristokracie ve vnitrozemí.

\startDelimitedTable
 {\em Celkem}~ 115 nápisů

{\em Region měst na pobřeží}~ Abdéra 7, Ainos 1, Anchialos 1, Apollónia Pontská 2, Bisanthé 1, Bizóné 3, Byzantion 69, Dionýsopolis 1, Lýsimacheia 1, Maróneia 15, Mesámbriá 5, Odéssos 2, Perinthos (Hérakleia) 1, Sélymbria 1, Séstos 1, Topeiros 1 (celkem 112)

{\em Region měst ve vnitrozemí}~ 0\footnote{Celkem tři nápisy byly nalezeny mimo území Thrákie, avšak editoři korpusů je vzhledem k jejich obsahu zařadili mezi nápisy pocházející z Thrákie.}

{\em Celkový počet individuálních lokalit}~ 23

{\em Archeologický kontext nálezu}~ sídelní 3, náboženský 4, sekundární 12, neznámý 96

{\em Materiál}~ kámen 113 (mramor 108, z toho mramor z Prokonnésu 2, vápenec 1, jiné 1), keramika 1, neznámý 1

{\em Dochování nosiče}~ 100 \letterpercent{} 6, 75 \letterpercent{} 8, 50 \letterpercent{} 12, 25 \letterpercent{} 11, oklepek 1, kresba 1, nemožno určit 76

{\em Objekt}~ stéla 106, architektonický prvek 5, socha 1, nádoba 1;

{\em Dekorace}~ reliéf 78, bez dekorace 37; reliéfní dekorace figurální 55 nápisů (vyskytující se motiv: jezdec 1, stojící osoba 1, sedící osoba 1, skupina lidí 1, funerální scéna/symposion 8, jiné 1), architektonické prvky 29 nápisů (vyskytující se motiv: naiskos 3, sloup 1, báze sloupu či oltář 4, věnec 1, florální motiv 11, geometrický motiv 0, architektonický tvar/forma 4, jiné 1)

{\em Typologie nápisu}~ soukromé 86, veřejné 25, neurčitelné 4

{\em Soukromé nápisy}~ funerální 80, dedikační 8, jiné (jméno autora) 1\footnote{V určitých případech může docházet ke kumulaci jednotlivých typů textů v rámci jednoho nápisu, či jejich nejednoznačnost neumožňuje rozlišit mezi několika typy. V těchto případech pak součet všech typů nápisů může přesahovat celkové číslo nápisů.}

{\em Veřejné nápisy}~ náboženské 1, seznamy 2, honorifikační dekrety 7, státní dekrety 15, funerální 2\footnote{Možnost kombinace kategorií, součet všech typů může přesahovat celkové číslo pro danou kategorii.}

{\em Délka}~ aritm. průměr 7,93 řádku, medián 2, max. délka 107, min. délka 1

{\em Obsah}~ dórský dialekt 10; hledané termíny (administrativní termíny 38 - celkem 125 výskytů, epigrafické formule 16 - 44 výskytů, honorifikační 27 - 91 výskytů, náboženské 26 - 54 výskytů, epiteton 7 - počet výskytů 7)

{\em Identita}~ řecká božstva 12 pojmenování, egyptská božstva 2 pojmenování, římská božstva 2 pojmenování, dále názvy míst a funkcí typických pro řecké náboženské prostředí, regionální epiteton 7, kolektivní identita 15 termínů, celkem 25 výskytů - obyvatelé řeckých obcí z oblasti Thrákie, ale i mimo ni, kolektivní pojmenování kmenové příslušnosti (Bíthýnos, Thráx), kolektivní pojmenování Římanů (Rómaios), celkem 186 osob na nápisech, 78 nápisů s jednou osobou; max. 25 osob na nápis, aritm. průměr 1,61 osoby na nápis, medián 1; komunita převládajícího řeckého charakteru, jména pouze řecká (60 \letterpercent{}), pouze thrácká (1,73 \letterpercent{}), pouze římská (4,34 \letterpercent{}), kombinace řeckého a thráckého (4,34 \letterpercent{}), kombinace řeckého a římského (2,6 \letterpercent{}), jména nejistého původu (19,97 \letterpercent{}), beze jména (6,95 \letterpercent{}); geografická jména z oblasti Thrákie 7, geografická jména mimo Thrákii 6;
\stopDelimitedTable


Celkový počet nápisů datovaných do 2. st. zůstává přibližně na stejné úrovni jako ve 3. st. př. n. l. Naprostá většina nápisů pochází z pobřežních oblastí, z bezprostřední blízkosti řeckých měst, jak ukazuje mapa 6.05a v Apendixu 2. Hlavní produkční centrum pro celý region se přesunulo z Mesámbrie a Maróneie do Byzantia, což je pravděpodobně odrazem nárůstu politického postavení Byzantia, co by spojence Říma (Jones 1973, 7).

Materiálem nesoucím nápisy je téměř výhradně kámen, s jednou výjimkou nápisu {\em SEG} 54:633 na střepu keramické nádoby.\footnote{Na pomezí soukromého a veřejného nápisu je dochovaný nápis {\em SEG} 54:633 na střepu keramické nádoby, která pochází z Apollónie Pontské a nese řecké jméno Aris{[}s{]}teidés, a bývá editory interpretován jako ostrakon. Pravděpodobněji se ale jedná o označení vlastníka nádoby.} Nosiče nápisů jsou převážně zhotovovány z místního zdroje kamene a dá se tedy i nadále předpokládat, že materiál pro zhotovování nápisů byl získáván v nejbližším okolí produkčních center a nebyl předmětem dálkového obchodu.

\subsection[funerální-nápisy-7]{Funerální nápisy}

Ze 2. st. př. n. l. pochází 80 funerálních nápisů, jejichž primární funkcí bylo sloužit jako náhrobní kámen a označovat hrob. Oproti předcházejícím obdobím se nedochovaly nápisy na předmětech osobní potřeby, které se staly součástí pohřební výbavy. Tento fakt může poukazovat na stav prozkoumání kulturních vrstev z této doby, či na upadající vliv thráckých aristokratů, a tudíž i na pokles epigrafických aktivit spojených a jejich aktivitami.

Celkově dochází k poklesu počtů funerálních nápisů napříč celou Thrákií, s výjimkou řeckého Byzantia, kde dochází k poměrně markantnímu nárůstu. Nápisy z pobřežních oblastí nesou ze 73 \letterpercent{} jména řeckého původu, nicméně zejména v oblasti Byzantia se setkáváme i s jmény thráckými, zastoupenými zhruba v 5 \letterpercent{}, a římskými, zastoupenými zhruba 7 \letterpercent{}, zbylých 15 \letterpercent{} jmen je cizího či nejistého původu.\footnote{Nápisů s pouze řeckými jmény je celkem 53, nápisy s řeckým a římským jménem jsou dva, a nápisy s řeckým a thráckým jménem jsou čtyři, všechny z Byzantia. Nápis {\em IK Byzantion} 214 jako jediný prokazatelně patří muži nesoucí thrácké jméno, jehož otec nese jméno řeckého původu: Mokaporis, syn Moscha. Thrácká jména se vyskytují celkem na šesti nápisech, z nichž pouze na jednom se setkáváme s kombinací čistě thráckých jmen. Nápis {\em IK Byzantion} 340 patřil Mokazoiré, dceři Dinea.} Necelých 30 \letterpercent{} nápisů je určeno ženám, které jsou dále identifikovány jako dcery či partnerky. Kombinace osobních jmen je možné interpretovat jako důsledek smíšených sňatků, či přejímání onomastických zvyků dané kultury. Tento jev je omezený na oblast Byzantia a nelze tedy hovořit o celospolečenském fenoménu.

Geografický původ zemřelých poukazuje vždy po jednom nápise na původ z Galatie a Bíthýnie, Apameie, tedy z Malé Asie z oblastí sousedících s Thrákií, což poukazuje na přesuny obyvatel mezi Evropou a Asií, ač na omezené úrovni.\footnote{Nápis {\em IK Byzantion} 120 patřil Theodórovi, jehož otec pocházel z města Bíthýnion v Bíthýnii, ale sám Theodóros se cítil být občanem Byzantia, kde také byl pochován.} Nápisy jsou většinou krátké v rozsahu jednoho až tří řádků. Výjimku tvoří několik veršovaných nápisů z Maróneie a Byzantia o rozsahu až 13 řádek, které podrobně líčí životní osudy a vykonané skutky zemřelého tak, jak bývá obvyklé zejména v pozdějších dobách.\footnote{Ač jsou všechna jména na těchto delších nápisech až na jednu výjimku řeckého původu, jedná se spíše o zvyk, který je pozorovatelný v římské době.} Vyskytující se epigrafické formule i nadále odpovídají funerálním nápisům tak, jak jsme je mohli vidět v řeckých komunitách v 5. - 3. st. př. n. l.\footnote{Celkem 11krát nápisy zdraví čtenáře. Místo pohřbu je v jednom případě označeno termínem {\em tafos}, jednou je použit termín {\em mnéma}, které označuje jak hrob, tak zároveň i náhrobní kámen.}

\subsection[dedikační-nápisy-7]{Dedikační nápisy}

Celkem osm dedikačních nápisů pochází výhradně z okolí řeckých kolonií na pobřeží. Věnování provedli muži nesoucí jména řeckého původu a nápisy byly dedikovány božstvům řeckého původu, tj. Dioskúrům, Diovi, Hermovi, Afrodíté s místním přízviskem {\em Ainijská} a samothráckým božstvům, dále božstvům egyptského původu, tj. Sarápidovi a Ísidě. Poprvé se objevuje věnování Diovi a Rómé, personifikovanému božstvu představujícímu autoritu města Říma.\footnote{{\em I Aeg Thrace} 187 pochází z egejské Maróneie.}

\subsection[veřejné-nápisy-7]{Veřejné nápisy}

Celkem se dochovalo 25 veřejných nápisů, což oproti minulému století představuje mírný propad v celkové produkci. Nápisy pocházejí výhradně z řeckých měst na pobřeží, případně byly nalezeny mimo oblast Thrákie, ale jejich obsah je zcela jasně s Thrákií spojuje. Nejvíce textů pochází z Maróneie s devíti exempláři. Celkem 80 \letterpercent{} veřejných nápisů představují dekrety vydané institucemi řecké {\em polis}, jako je {\em búlé} a {\em démos}.

Honorifikační nápisy jsou určeny význačným mužům řeckého i římského původu. Řím se objevuje v nápisech jako silný hráč na poli mezinárodní politiky, v mnoha případech vystupuje i jako spojenec řeckých států na egejském pobřeží, např. na nápise {\em I Aeg Thrace} 168. Pokud jde o zmínky o Thrácích, velmi záleží na konkrétním případě a záměru daného nápisu. Konkrétně na nápise {\em IK Sestos} 1 jsou Thrákové zmiňováni jako sousedé, kteří mohou ohrožovat bezpečí obyvatel Séstu. V případě dalšího nápisu {\em I Aeg Thrace} 5 je thrácký panovník Kotys vnímán jako suverénní politická autorita, stojící na podobné úrovni jako Řím a abdérský lid. V kontextu veřejných nápisů tedy nefigurují Thrákové jako barbaři či nepřátelé, ale spíše jako mocní spojenci a nevyzpytatelní sousedé autonomních řeckých měst.

Dochované administrativní termíny poukazují na nárůst vyskytujících se termínů na 35, které se dohromady objevují 117krát.\footnote{Nejpoužívanějšími termíny byly pojmy bezprostředně reprezentující politickou autoritu a vztahující k proceduře vydávání nařízení jako {\em démos}, {\em búlé}, {\em polis} a {\em pséfisma}. Objevují se nové funkce a instituce, které se nápisech dříve nevyskytovaly, jako například {\em synedrion}, {\em symmachiá}, {\em gymnasion}, {\em grammateus}, {\em chorégiá}, {\em efébos} a {\em basilissa.} Existence {\em gymnasia} na území Thrákie je poprvé potvrzena epigraficky, a to v Apollónii a v Séstu, instituce {\em efébie} je potvrzena také v Séstu. Dále se zde vyskytují termíny s finanční a obchodní tematikou jako je {\em emporion}, {\em chrémata}, {\em analóma}, {\em trápeza}, {\em chóra}, {\em oikos} a {\em katoikia}. Zcela poprvé se objevuje termín {\em autokratór}, který je v následujících stoletích používán výhradně pro římského císaře, ale v tomto případě na nápise {\em IG Bulg} 1,2 388bis označuje velitele námořnictva z Istru, který pomohl Apollónii v době války mezi Apollónií a Mesámbrií.} Nárůst celkového počtu termínů a objevení nových institucí a funkcí související s organizací politické autority a jejím výkonem naznačuje, že docházelo k proměnám společenské organizace v rámci řeckých komunit na pobřeží. Nárůst dochovaných ujednání mezi politickými autoritami regionu naznačuje i zintenzivnění diplomatických kontaktů, nebo alespoň jejich kodifikace a častější zaznamenávání na trvalé médium.

\subsection[shrnutí-11]{Shrnutí}

V průběhu 2. st. př. n. l. dochází k prvním projevům mísení řeckého a římského kulturního prostředí, avšak se znatelnou převahou řeckého elementu. Thrákové se do produkce soukromých nápisů zapojují minimálně, a to pouze v regionu Byzantia a na úrovni běžného obyvatelstva, nikoliv thrácké aristokracie, jak bylo zvykem v předcházejících stoletích. Obsah a forma veřejných nápisů naznačují objevení nových institucí, intenzifikaci diplomatických kontaktů mezi jednotlivými politickými autoritami a jejich kodifikaci v epigrafické produkci. Thráčtí králové jsou na veřejných nápisech pocházejících z řeckého kontextu vnímáni jako rovnocenní spojenci, nikoliv jako primitivní barbaři.

\stopcomponent