
\environment ../env_dis
\startcomponent section-charakteristika-epigrafické-produkce-ve-3.-až-2.-st.-př.-n.-l.
\section[charakteristika-epigrafické-produkce-ve-3.-až-2.-st.-př.-n.-l.]{Charakteristika epigrafické produkce ve 3. až 2. st. př. n. l.}

Nápisy datované do 3. až 2. st. př. n. l. pocházejí převážně z řeckého kulturního prostředí. Dochází k nárůstu produkce veřejných nápisů, stejně tak k výskytu hledaných termínů. Prolínání řeckých a thráckých onomastických tradic je možné sledovat v omezené míře, dále i rozšíření lokálních, řeckých a egyptských náboženských tradic.


\startframedbox
{\em Celkem}~ 59 nápisů

{\em Region měst na pobřeží}~ Abdéra 5, Apollónia Pontská 4, Byzantion 12, Dionýsopolis 1, Lýsimacheia 1, Maróneia 10, Mesámbria 9, Odéssos 7, Perinthos (Hérakleia) 3 (celkem 52 nápisů)

{\em Region měst ve vnitrozemí}~ Beroé (Augusta Traiana) 1, Filippopolis 1, údolí středního toku řeky Strýmónu 3\footnote{Celkem dva nápisy nebylo nalezeny v rámci regionu známých měst, editoři korpusů udávají jejich polohu vzhledem k nejbližšímu modernímu sídlišti (jedna lokalita s jedním nápisem), či uvádějí jejich původ jako blíže neznámé místo v Thrákii (jeden nápis).}

{\em Celkový počet individuálních lokalit}~ 14

{\em Archeologický kontext nálezu}~ funerální 4, sídelní 1, náboženský 1, sekundární 4, neznámý 48

{\em Materiál}~ kámen 56 (mramor 48, jiný 2, neznámý 6), keramika 1, kov 1, neznámý 1

{\em Dochování nosiče}~ 100 \letterpercent{} 3, 75 \letterpercent{} 10, 50 \letterpercent{} 8, 25 \letterpercent{} 14, kresba 1, ztracený 2, nemožno určit 21

{\em Objekt}~ stéla 50, socha 1, nádoba 1, architektonický prvek 4, jiné 2, neznámý 1

{\em Dekorace}~ reliéf 29, malovaná dekorace 1, bez dekorace 29; figurální 8 nápisů (vyskytující se motiv: skupina lidí 1, sedící postava 2, stojící postava 1, funerální scéna 2, jezdec 1), architektonické prvky 21 nápisů (vyskytující se motiv: naiskos 9, florální motiv 7, sloup 1, báze sloupu či oltář 3, věnec 1, architektonický tvar/forma 3)

{\em Typologie nápisu}~ soukromé 37, veřejné 20, neurčitelné 2

{\em Soukromé nápisy}~ funerální 30, dedikační 6, vlastnictví 1, jiné (jméno autora) 1\footnote{Vzhledem ke kumulaci typů je celkový součet vyšší než počet soukromých nápisů.}

{\em Veřejné nápisy}~ náboženské 1, seznamy 1, honorifikační dekrety 5, státní dekrety 11, funerální na náklady obce 1, neznámý 1

{\em Délka}~ aritm. průměr 5,08 řádku, medián 3, max. délka 25, min. délka 1

{\em Obsah}~ dórský dialekt 7, graffiti 1; hledané termíny (administrativní termíny 11 - celkem 39 výskytů, epigrafické formule 7--20 výskytů, honorifikační 11--22 výskytů, náboženské 14--20 výskytů, epiteton 1 - počet výskytů 1)

{\em Identita}~ řecká božstva, místní božstva, egyptská božstva, kolektivní identita 3 - obyvatelé řeckých obcí, z toho z oblasti Thrákie 1, mimo Thrákii 2, celkem 71 osob na nápisech, 34 nápisů s jednou osobou; max. 6 osob na nápis, aritm. průměr 1,2 osoby na nápis, medián 1; komunita převládajícího řeckého charakteru, jména pouze řecká (52 \letterpercent{} - celkem 31 nápisů), thrácká (3,38 \letterpercent{} - celkem 2 nápisy), kombinace řeckého a thráckého (5,08 \letterpercent{} - 3 nápisy), pouze římská jména (1,69 \letterpercent{} - 1 nápis), jména nejistého původu (16,94 \letterpercent{}); geografická jména míst v Thrákii 1;



\stopframedbox

Celkem se dochovalo 59 nápisů, které i nadále pocházejí převážně z řeckých komunit na pobřeží a jen minimum nápisů je nalezeno v thráckém vnitrozemí, jak dokazuje mapa 6.04 v Apendixu 2. Nápisy z vnitrozemí pocházejí z okolí Kazanlackého údolí a středního toku Strýmónu, tedy oblastí spojovaných se zvýšenou přítomností makedonských či thráckých vojáků v řeckých službách (Nankov 2012; 2015).

Převládajícím materiálem je již tradičně kámen, nápisy na kovu a na keramice se dochovaly vždy po jednom exempláři. V obou případech nápisů na jiném nosiči, než na kameni se jedná o identifikaci autora předmětu či vnitřní dekorace hrobky v kombinaci se jménem majitele předmětu, tedy podobné použití jako v 5. až 3. st. př. n. l. Přes polovinu nápisů představují funerální nápisy, ale setkáváme se i s označením vlastnictví z prostředí thrácké aristokracie.\footnote{Do této kategorie spadá nápis {\em IG Bulg} 5 5638bis nalezený na amfoře v antickém městě Kabylé na středním toku řeky Tonzos. Kabylé bylo známo jako makedonské vojensko-obchodní osídlení založené na místě dřívějšího thráckého sídliště (Handzhijska a Lozanov 2010, 260-263). Nápis představuje pravděpodobně jméno majitele či objednatele amfory, Sadalás, syn Téreův. Obě jména jsou typicky thrácká, většinou patřící thrácké aristokracii (Dana 2014, 298-301; 355-361).}

\subsection[funerální-nápisy-6]{Funerální nápisy}

Celkem se dochovalo 30 funerálních nápisů datovaných do 3. až 2. st. př. n. l., které pocházejí převážně z řecké komunity na pobřeží a z řecké či makedonské komunity v okolí Hérakleie Sintské, jak dokazuje přítomnost zejména řeckých jmen zemřelých osob.\footnote{Jediné jméno, které je možné průkazně spojovat s thráckým původem, je jméno Amatokos z {\em IK Byzantion} 325, které je použito jako jméno rodiče Hermia.} Charakteristika nápisů je totožná s funerálními nápisy pocházející z řeckých komunit 5. až 3. st. př. n. l. Za zvláštní pozornost nicméně stojí skupina tří funerálních nápisů z Hérakleie Sintské, na nichž se dochovala jména celkem pěti osob, z čehož byly čtyři ženy.\footnote{Jednotlivé osoby byly identifikovány nejen pomocí osobního jména, ale i pomocí údajů o rodičích a partnerech a veškerá dochovaná jména jsou jména řeckého či makedonského původu. Jiná vyjádření identity, či podoby jazyka, která by pomohla komunitu lépe zařadit, se nedochovala.} Nosiče nápisů byly ve dvou případech vyrobeny z místně dostupného materiálu jako je tuf, vápenec a varovik, ale uchovávaly si tradiční vzhled jednoduchých funerálních stél s akroteriem, figurální funerální scénou a v jednom případě písmeny malovanými červenou barvou. Zdroj materiálu byl sice místní, nicméně provedení odkazuje na techniky a motivy tradičně používané v rámci řecké či makedonské komunity. Z archeologických zdrojů však víme, že v době hellénismu byla v Hérakleii Sintské založena pravděpodobně makedonská vojensko-obchodní stanice, která se později rozrostla na město (Nankov 2015, 7-10, 22-27). Není zcela jasné, zda se jednalo o osídlení čistě makedonské, či bylo obývané jak Makedonci, tak Thráky, jak bývalo obvyklé u měst zakládaných Filippem II. (Adams 2007, 9-11). V současné době zde neustále probíhají archeologické výzkumy, a tak je možné, že se do budoucna objeví ještě více důkazů. Zatím je však zřejmé, že mimo pobřežní oblasti byl zvyk stavění náhrobních kamenů ve 3. až 2. st. př. n. l. rozšířen pouze v oblasti obývané řeckými či makedonskými osadníky, a nevyskytoval se v čistě thrácké komunitě.

Podobně jako v 5. a 4. st. př. n. l. je písmo v kontextu thrácké aristokracie využíváno pro velmi specifický účel a v okruhu velmi omezeného počtu lidí. Hlavním účelem je ztotožnit majitele, který patřil do okruhu thrácké aristokracie, či zhotovitele, který mohl být jak thráckého, případně řeckého původu. V případě nápisu na kovovém předmětu {\em SEG} 59:759 se jedná o jméno tvůrce na zlatém diadému, který se nalezl uvnitř hrobky patřící pravděpodobně ženě. Jméno zhotovitele předmětu je řeckého původu a používá typicky řeckou formuli {\em epoi{[}é{]}sen}, tedy zhotovil Démétrios (Manov 2009, 27-30).\footnote{Na témže diadému se nachází ještě pravděpodobně thrácké jméno Kortozous v genitivu singuláru, u nějž není jisté, zda patřilo muži či ženě. Manov usuzuje, že je to jméno majitele diadému, a pokud jím byla žena pohřbená v hrobce, kde byl předmět nalezen, pak diadém mohl patřit právě jí. Nejedná se tedy o primárně funerální nápis, ale o předmět osobní potřeby, který byl uložen do hrobu po smrti majitele.} Pořizování předmětů s nápisy nicméně stále nepatřilo k běžnému standardu ani mezi thráckými aristokraty, natož mezi běžnou thráckou populací a přístup k písmu byl odlišný v rámci řecké a thrácké komunity, což dokazuje i nadále pokračující absence pohřební stél z thráckého kontextu.

\subsection[dedikační-nápisy-6]{Dedikační nápisy}

Zvyk věnovat stély s nápisy se i nadále na přelomu 3. a 2. st. př. n. l. vyskytoval v řeckých komunitách na pobřeží. Celkem se dochovalo šest dedikačních nápisů, které pocházely z území řeckých měst a věnování provedly osoby nesoucí téměř výhradně řecká jména. Věnování byla určena božstvům řeckého a egyptského původu, jako je Ísis, Sarápis a Anúbis.\footnote{Věnování Ísidě, Sarápidovi a Anúbidovi na nápise {\em IG Bulg} 1,2 322ter z černomořské Mesámbrie, a Ísidě a Afrodíté na nápise {\em Perinthos-Herakleia} 42 z Perinthu.} Rozšíření kultu egyptských božstev v Thrákii bývá vysvětlováno zvýšenou přítomností hellénistických vojsk a politického vlivu Ptolemaiovců v oblasti Perinthu (Tacheva-Hitova 1983, 54-58; Barrett a Nankov 2010, 17). I nadále nemáme důkazy o zapojení thráckého obyvatelstva a thráckého náboženství do procesu epigrafické produkce.

\subsection[veřejné-nápisy-6]{Veřejné nápisy}

Celkem se dochovalo 20 nápisů spadající do kategorie veřejných nápisů: převážná většina z nich byly dekrety vydávané v rámci řeckých městských států na pobřeží, přičemž šest dekretů pochází z Odéssu a pět z Maróneie. Politickou autoritu jednotlivých měst zastupovaly orgány jako {\em démos} a {\em búlé} a nejčastější druhem dokumentu jsou honorifikační dekrety vystavené pro význačné osoby či osoby, které se výjimečným způsobem zasloužili o udělení poct. Dle dochovaných osobních jmen se většinou jednalo o muže řeckého původu. V několika případech známe jejich mateřskou obci: Athény, Kallatis, Chersonésos a Antiocheia.\footnote{{\em I Aeg Thrace} 172, {\em IG Bulg} 1,2 13ter, {\em IG Bulg} 1,2 39, {\em IG Bulg} 1,2 41. V případě seznamu osob {\em Perinthos-Herakleia} 62 z Perinthu se dochovala jména šesti mužů řeckého původu. Svou identitu udávají pomocí osobního jména a jména otce, a dále specifikují svůj původ: tři muži pocházejí z fýly {\em Théseis}, dva z fýly {\em Basileis} a jeden z města Byzantion.}

Texty honorifikačních nápisů vycházely ze stejného základu, nicméně každý nápis byl přizpůsoben konkrétní situaci. Konkrétní znění dekretů se lišilo město od města, což poukazuje na jejich politickou samostatnost a nezávislý vývoj epigrafických formulí, které nicméně vycházejí ze společného základu. Součástí textu byly občas i podmínky zveřejnění nápisu, což pro řecké komunity zpravidla bývalo umístění veřejného nápisu do svatyně patrona daného města, jak je typické i pro jiné řecké komunity.\footnote{Text nápisů {\em IG Bulg} 1,2 41 a 42 z Odéssu nařizuje umístit honorifikační nápis v podobě sochy do svatyně samothráckých božstev a do svatyně anonymního božstva.}

\subsection[shrnutí-10]{Shrnutí}

Nápisy datované do 3. a 2. st. př. n. l. nezaznamenávají změnu trendu nastaveného v předchozích stoletích. Řecká a thrácká komunita interagují pouze v omezené míře a udržují si své tradiční zvyklosti, alespoň soudě dle dochovaných nápisů. Poprvé se v této době objevují projevy rozšíření egyptského náboženství do Thrákie, což můžeme vidět zejména na Řeky osídleném pobřeží a v jihovýchodní části Thrákie, kde získali určitý politický vliv Ptolemaiovci.

\stopcomponent