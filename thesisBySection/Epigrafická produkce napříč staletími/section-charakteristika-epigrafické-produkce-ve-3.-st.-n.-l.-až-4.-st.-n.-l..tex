
\environment ../env_dis
\startcomponent section-charakteristika-epigrafické-produkce-ve-3.-st.-n.-l.-až-4.-st.-n.-l.
\section[charakteristika-epigrafické-produkce-ve-3.-st.-n.-l.-až-4.-st.-n.-l.]{Charakteristika epigrafické produkce ve 3. st. n. l. až 4. st. n. l.}

V epigrafické produkci dochází na přelomu 3. a 4. st. n. l. k velkému propadu. Zcela dochází k vymizení thráckých osobních jmen a místních kultů z dochovaných nápisů. Komunity se opět uzavírají a pokles epigrafické produkce je velmi patrný na všech místech. Veřejné nápisy v této době poprvé převažují nad nápisy soukromými a polovinu z nich tvoří milníky, které se nacházely podél významných silnic.

\startDelimitedTable
{\em Celkem}~ 24 nápisů

{\em Region měst na pobřeží}~ Anchialos 1, Maróneia 7, Perinthos (Hérakleia) 8 (celkem 16 nápisů)

{\em Region měst ve vnitrozemí}~ Augusta Traiana 1, Filippopolis 1, Serdica 2, Traianúpolis 1, údolí středního toku řeky Strýmónu 1 (celkem 6 nápisů)\footnote{Celkem dva nápisy nebyly nalezeny v rámci regionu známých měst, editoři korpusů udávají jejich polohu vzhledem k nejbližšímu modernímu sídlišti.}

{\em Celkový počet individuálních lokalit}~ 12

{\em Archeologický kontext nálezu}~ sídelní 3 (z toho obchodní 1), sekundární 3, neznámý 19

{\em Materiál}~ kámen 24 (mramor 14; vápenec 3, jiný 2; neznámý 5)

{\em Dochování nosiče}~ 100 \letterpercent{} 3, 75 \letterpercent{} 4, 50 \letterpercent{} 2, 25 \letterpercent{} 4, kresba 1, ztracený 1, nemožno určit 9

{\em Objekt}~ stéla 11, architektonický prvek 11, jiný 1, neznámý 1

{\em Dekorace}~ reliéf 7, bez dekorace 17; reliéfní dekorace figurální 0 nápisů, architektonické prvky 7 nápisů (vyskytující se motiv: sloup 6, jiný 1)

{\em Typologie nápisu}~ soukromé 11, veřejné 12, neurčitelné 1

{\em Soukromé nápisy}~ funerální 9, dedikační 2, vlastnictví 1, jiný 1\postponenotes\footnote{Několik nápisů mělo vzhledem ke své nejednoznačnosti kombinovanou funkci, proto je součet nápisů obou typů vyšší než celkový počet soukromých nápisů.}

{\em Veřejné nápisy}~ honorifikační dekrety 4, jiný 7 (z toho milník 4, hraniční kámen 3)

{\em Délka}~ aritm. průměr 9,2 řádku, medián 10, max. délka 21, min. délka 1

{\em Obsah}~ latinský text 5 nápisů, písmo římského typu 3; hledané termíny (administrativní termíny 5 - celkem 16 výskytů, epigrafické formule 9 - 21 výskytů, honorifikační 1 - 2 výskyty, náboženské 2 - 2 výskyty, epiteton 24 - počet výskytů 55)

{\em Identita}~ řecká božstva 1, egyptská božstva 0, římská božstva 0, křesťanství 0, prudký pokles náboženské terminologie, včetně vymizení lokálních kultů z nápisů, regionální epiteton 0, subregionální epiteton 0{\bf ,} kolektivní identita 3 termíny, celkem 9 výskytů {\bf -} obyvatelé řeckých obcí z oblasti Thrákie 1, mimo ni 0; kolektivní pojmenování etnik či kmenů (Thráx 2), obyvatelé thráckých vesnic 1; celkem 50 osob na nápisech, 5 nápisů s jednou osobou; max. 6 osob na nápis, aritm. průměr 2,08 osoby na nápis, medián 2; komunita se zastoupením řeckého a římského, thrácký prvek zcela chybí, jména pouze řecká (4 \letterpercent{}), pouze thrácká (0 \letterpercent{}), pouze římská (52 \letterpercent{}), kombinace řeckého a thráckého (0 \letterpercent{}), kombinace řeckého a římského (12 \letterpercent{}), kombinace thráckého a římského (0 \letterpercent{}), kombinovaná řecká, thrácká a římská jména (4 \letterpercent{}), jména nejistého původu (4 \letterpercent{}), beze jména (24 \letterpercent{}); geografická jména nejistého původu 1;
\stopDelimitedTable
\flushnotes

U skupiny nápisů datovaných do 3. a 4. st. n. l. je pozorovatelný prudký pokles epigrafické produkce až o 86 \letterpercent{} oproti skupině nápisů datovaných do 2. až 3. st. n. l. Tento jev odpovídá dění i v jiných částech římské říše, kdy po prudkém nárůstu koncem 2. st. a na začátku 3. st. dochází ve druhé polovině 3. a na začátku 4. st. n. l. k výraznému snížení počtu dochovaných nápisů (MacMullen 1982, 245; Meyer 1990, 82-94). Produkční centra se v malé míře uchovávají v pobřežních oblastech, zejména v Maróneii a Perinthu, jak je patrné na \in{Mapě}[Apendix2:::6.09a] v \in{Apendixu}[Apendix2:::Apendix2].

\subsection[funerální-nápisy-16]{Funerální nápisy}

Funerálních nápisů se dochovalo celkem devět, což představuje takřka šestinásobný propad v produkci oproti nápisům datovaným do 2. až 3. st. n. l. Celkem čtyři nápisy pocházejí z Maróneie, dva z Perinthu. V jednom případě se jedná o sarkofág, jehož neoprávněné použití bylo chráněno institucemi v Maróneii, podobně jako u stejných nápisů z 1. až 3. st. n. l.\footnote{Nápis {\em I Aeg Thrace} 312.} Většina dochovaných osobních jmen je řeckého a římského původu, pouze v údolí středního toku Strýmónu se dochovalo šest osob se jmény pravděpodobně thráckého původu.\footnote{Nápis Manov 2008 168.} Úbytek počtu nápisů je jedinou zásadní změnou, celkový charakter funerálních nápisů zůstává stejný jako v předcházejícím období, nakolik je možné usuzovat z omezeného vzorku nápisů.

\subsection[dedikační-nápisy-16]{Dedikační nápisy}

Dedikační nápis se dochoval pouze jeden, což představuje velký propad oproti předcházejícímu období. {\em IG Bulg} 3,2 1835 je psán řecky a latinsky a je věnován císaři Diokletiánovi, Maximiánovi, Konstantinu Chlorovi a Galeriovi vojákem nesoucí jména Aurelios Iúlianos.

\subsection[veřejné-nápisy-16]{Veřejné nápisy}

Nápisů datovaných do 3. až 4. st. n. l. se dochovalo dohromady 12, což představuje téměř 50 \letterpercent{} všech nápisů z dané skupiny. Celkem se jedná o čtyři honorifikační dekrety, čtyři milníky, tři hraniční kameny a jeden nápis s věnováním císařům. Všechny nápisy pochází z období tetrarchie a jsou úzce spojeny s osobnostmi císařů.

Všechny honorifikační dekrety pocházejí z Hérakleie (původního Perinthu), která v této době hrála roli regionálního centra, než se jím v roce 330 n. l. stala Konstantinopol, bývalé Byzantion. Všechny čtyři nápisy jsou věnované obyvateli Hérakleie tehdejším císařům Diokletiánovi, Maximiánovi, Konstantinu Chlorovi a Galeriovi. Stejně tak i dochované tři hraniční kameny jsou věnovány tetrarchům, kteří tak vyměření hranic území daného sídla propůjčují patřičnou legitimitu.\footnote{Ve všech případech se jedná o hraniční kámen místní thrácké vesnice, neznámé z dalších historických zdrojů. {\em I Aeg Thrace} 398 jedná se o vyznačení hranic osídlení ({\em chórióma}) Barilos (?). {\em I Aeg Thrace} 382 vyznačení hranic vesnice Eresén{[}-{]}. {\em I Aeg Thrace} 383 vyznačení hranic neznámé vesnice.} Dochované čtyři milníky pocházejí z okolí {\em Via Diagonalis} (Serdica, Augusta Traiana a Perinthos, tehdy již jako Hérakleia) a jsou datované na přelom 3. a 4. st. n. l., kdy docházelo k stavebním aktivitám a úpravám této významné silnice, sloužící zejména k přepravě římských vojsk.

\subsection[shrnutí-20]{Shrnutí}

Na přelomu 3. a 4. st. n. l. dochází k útlumu epigrafických aktivit a výraznému poklesu počtu dochovaných nápisů. To může souviset s celospolečenskou krizí římské říše a ekonomickým úpadkem, způsobeným dlouholetými vojenskými spory pocházejícími jak z nitra říše samotné, tak i nájezdy nepřátel na severních hranicích. S úpadkem institucí tak, jak je známe dosud a omezením jejich činnosti dochází i k vymizení veřejných nápisů a přeměně soukromých nápisů, jak je patrné dále ve 4. st. n. l.

\stopcomponent