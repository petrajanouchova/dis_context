
\environment ../env_dis
\startcomponent section-charakteristika-epigrafické-produkce-ve-3.-st.-př.-n.-l.
\section[charakteristika-epigrafické-produkce-ve-3.-st.-př.-n.-l.]{Charakteristika epigrafické produkce ve 3. st. př. n. l.}

Nápisy v této době pocházejí převážně z pobřežních oblastí, nicméně nápisy se objevují i v kontextu thrácké aristokracie ve vnitrozemí stejně jako v 5. a 4. st. př. n. l. V řeckých komunitách na pobřeží dochází ke kontaktu s ostatními řecky mluvícími komunitami mimo Thrákii a v malé míře i s thráckým kulturním prostředím. Nárůst počtu veřejných nápisů a výskyt hledaných termínů svědčí o nárůstu společenské komplexity, a to zejména v řeckém prostředí.

\startDelimitedTable
 {\em Celkem}~ 117 nápisů

{\em Region měst na pobřeží}~ Abdéra 5, Agathopolis 1, Anchialos 2, Apollónia Pontská 5, Byzantion 16, Dionýsopolis 6, Doriskos 1, Maróneia 21, Mesámbriá 42, Naulochos 1, Odéssos 5, Perinthos (Hérakleia) 1 (celkem 106 nápisů)

{\em Region měst ve vnitrozemí}~ Beroé (Augusta Traiana) 4, (Marcianopolis) 1, (Plótinúpolis) 1\postponenotes\footnote{Celkem pět nápisů nebylo nalezeno v rámci regionu známých měst, editoři korpusů udávají jejich polohu vzhledem k nejbližšímu modernímu sídlišti (jedna lokalita s jedním nápisem), či uvádějí jejich původ jako blíže neznámé místo v Thrákii (dva nápisy), či jako nápisy pocházející z území mimo Thrákii (dva nápisy).}

{\em Celkový počet individuálních lokalit}~ 18

{\em Archeologický kontext nálezu}~ funerální 10, sídelní 3, náboženský 2, sekundární 10, neznámý 92

{\em Materiál}~ kámen 116 (mramor 103, z toho mramor z Thasu 1, vápenec 3, jiné 1; z čehož je syenit 1; neznámý 9), jiný materiál 1

{\em Dochování nosiče}~ 100 \letterpercent{} 19, 75 \letterpercent{} 6, 50 \letterpercent{} 35, 25 \letterpercent{} 28, oklepek 1, kresba 2, nemožno určit 26

{\em Objekt}~ stéla 110, architektonický prvek 5, socha 1, nástěnná malba 1

{\em Dekorace}~ reliéf 61, malovaná dekorace 2, bez dekorace 54; reliéfní dekorace figurální 13 nápisů (vyskytující se motiv: jezdec 1, stojící osoba 3, sedící osoba 8, skupina lidí 4), architektonické prvky 44 nápisů (vyskytující se motiv: naiskos 22, sloup 5, báze sloupu či oltář 1, florální motiv 10, geometrický motiv 1, architektonický tvar/forma 5)

{\em Typologie nápisu}~ soukromé 79, veřejné 36, neurčitelné 2

{\em Soukromé nápisy}~ funerální 71, dedikační 5, jiný (jméno autora) 1, neznámý 2

{\em Veřejné nápisy}~ nařízení 1, náboženské 2, seznamy 2, honor. dekrety 9, státní dekrety 23\postponenotes\footnote{V určitých případech může docházet ke kumulaci jednotlivých typů textů v rámci jednoho nápisu, či jejich nejednoznačnost neumožňuje rozlišit mezi několika typy a těmto nápisů jsou přiřazeny několikanásobné hodnoty (např. nápis náboženský a zároveň seznam). V těchto případech pak součet všech typů nápisů může přesahovat celkové číslo nápisů.}

{\em Délka}~ aritm. průměr 5,56 řádku, medián 3, max. délka 37, min. délka 1

{\em Obsah}~ dórský dialekt 26, graffiti 1; hledané termíny (administrativní termíny 25 - celkem 77 výskytů, epigrafické formule 11 - 46 výskytů, honorifikační 19 - 65 výskytů, náboženské 18 - 42 výskytů, epiteton 1 - počet výskytů 1)

{\em Identita}~ řecká božstva, pojmenování míst a funkcí typických pro řecké náboženské prostředí, regionální epiteton 1, kolektivní identita 15 termínů, celkem 18 výskytů - obyvatelé řeckých obcí z oblasti Thrákie, ale i mimo ni, kolektivní pojmenování kmenové příslušnosti (Thessalos, Aitólos, Krés), celkem 166 osob na nápisech, 70 nápisů s jednou osobou; max. 12 osob na nápis, aritm. průměr 1,41 osoby na nápis, medián 2; komunita převládajícího řeckého charakteru, jména pouze řecká (75 \letterpercent{}), thrácká (3,41 \letterpercent{}), kombinace řeckého a thráckého (2,56 \letterpercent{}), jména nejistého původu (14,52 \letterpercent{}); geografická jména z oblasti Thrákie 5, geografická jména mimo Thrákii 3;
\stopDelimitedTable
\flushnotes

Celkem se dochovalo 117 nápisů, zejména z řeckých měst či jejich bezprostředního okolí. Z vnitrozemí pocházejí nápisy z lokalit na největších řekách spojujících vnitrozemskou Thrákii s Egejskou oblastí, konkrétně z povodí řek Tonzos, Hebros a Strýmón. \in{Mapa}[Apendix2:::6.04a] v \in{Apendixu}[Apendix2:::Apendix2] zachycuje konkrétní rozložení lokalit v nichž byly nalezeny nápisy. Apollónia Pontská přestává být hlavním producentem nápisů a na její místo nastupuje její jižní soused Mesámbriá, odkud pochází 36 \letterpercent{} nápisů z daného období. Dalšími významnými producenty nápisů jsou Maróneia, Byzantion a Odéssos. Archeologický kontext nálezů nápisů je většinou neznámý, nicméně zhruba u 14 nápisů byl tento kontext určen jako funerální, tj. pochází z pohřebiště či mohyly. Archeologické lokality dosvědčují náhlé změny poměrů, pravděpodobně související s příchodem keltských kmenů do Thrákie, které měly za následek destrukci některých lokalit, případně jejich úplný zánik, jako v případě {\em emporia} Pistiros (Bouzek {\em et al.} 2016).

Převládajícím materiálem je z 99 \letterpercent{} i nadále kámen: z téměř 91 \letterpercent{} mramor a dále kámen místního původu, jako je vápenec, syenit či varovik. Nápisy tesané do kamene pocházely z velké části z řeckých měst či jejich bezprostředního okolí. Výjimku tvoří několik nápisů pocházejících z thráckého vnitrozemí, a to z údolí středního Strýmónu, dále z oblasti střední Thrákie s lokalitami Seuthopolis a Kabylé. V těchto místech se předpokládá buď přítomnost řeckého či makedonského obyvatelstva nebo přinejmenším velmi intenzivní kontakty thrácky a řecky mluvících komunit. Dochází zde k náznakům podobného využití písma pro regulaci společenských vztahů ve formě, na jakou jsme zvyklí z řeckých obcí na pobřeží.\footnote{V jednom případě se dochoval nápis na jiném materiále než na kameni, a to na fresce uvnitř hrobky z thráckého prostředí.} Podobně jako v předcházejících dvou stoletích se setkáváme s odlišným pojetím písma a jeho kulturně-společenské funkce v rámci řecké a thrácké komunity v malé míře i ve 3. st. př. n. l.

\subsection[funerální-nápisy-5]{Funerální nápisy}

Funerální nápisy tvoří nejpočetnější skupinu soukromých nápisů, podobně jako v předcházejících stoletích. Celkem se dochovalo 71 primárních funerálních nápisů datovaných do 3. st. př. n. l. Do skupiny sekundárních funerálních nápisů patří jeden nápis vyrytý do fresky uvnitř mohylové hrobky.

\subsubsection[primární-funerální-nápisy-2]{Primární funerální nápisy}

Do této skupiny již tradičně patří funerální stély či jiné předměty, jejichž hlavní funkcí bylo označovat místo pohřbu a připomínat zemřelého. Celkem se jich dochovalo 71 a pocházely především z řeckých obcí na mořském pobřeží. Největší koncentrace 23 nápisů pochází z černomořské Mesámbrie a dále z Byzantia a z egejské Maróneie se 17 nápisy. Mimo pobřežní oblasti byl nalezen jeden náhrobních nápis z okolí moderní vesnice Červinite Skali v oblasti středního toku Strýmónu.

Skupina nápisů pocházejících z pobřežních oblastí naznačuje, že nápisy z řeckých měst a jejich bezprostředního okolí nadále udržují funerální ritus v podobě, v jaké se projevoval v předcházejících staletích.\footnote{Texty funerálních nápisů byly převážně jednoduché, v rozsahu dvou až tří řádků. Nejdelší text měl 11 řádků, ale to se jednalo spíše o výjimku. Typický nápis obsahoval jméno zemřelého a jméno jeho rodiče či partnera. Pouze výjimečně nápis obsahoval informace z nebožtíkova života či ve čtyřech případech vyjádření zármutku pozůstalých zhotovené metricky. Texty tedy spíše následovaly tradici jednoduchých textů poukazujících pouze na osobu zemřelého, jak bylo obvyklé v předcházejících stoletích.} Analýza osobních jmen potvrdila, že 84 \letterpercent{} osobních jmen je řeckého původu, 13 \letterpercent{} není možné určit a pouze 3 \letterpercent{} jmen jsou pravděpodobně thráckého původu. Thrácká jména se vyskytovala celkem na třech nápisech, z nichž dva pravděpodobně pocházely z Odéssu a jeden z Byzantia. V jednom případě se jednalo o ženu, v jednom případě o manželský pár-sourozence a v jednom případě o muže. Z toho plyne, že funerální stély byly využívány výhradně obyvateli se jmény řeckého původu. Další vyjádření identity jako kolektivní pojmenování odkazují na řecký původ zemřelých či jejich rodinných příslušníků s kořeny v okolí thráckého regionu.\footnote{Patří sem například termíny jako Filippeus, Lýsimacheus, Hérakleótés, a pak dále Krés a Rhodios. Geografické pojmy se vyskytují jen v jednom případě a poukazují na původ jistého Apollónia z Babylónu. Celkem šest lidí udávalo svůj geografický původ, který byl z více než poloviny mimo oblast Thrákie, nicméně jejich jména byla řeckého původu.} Ač nečetné, jedná se o zdokumentované případy migrace části obyvatel. Nelze však tvrdit, že by ve 3. st. př. n. l. migrace obyvatelstva byla častější než ve 4. st. př. n. l. pouze na základě většího výskytu geografických termínů. Kladení důrazu na geografický původ může souviset s narůstající propojeností hellénistického světa, která je jedním z průvodních znaků hellénismu a se snahou udržet si svou identitu i v rámci nové komunity, a proto se tato vyjádření začala objevovat častěji.

Nejen osobní jména a vyjádření identity, ale i hledané termíny poukazují na udržování tradic řeckého kulturního prostředí i v rámci funerálního ritu.\footnote{V osmi případech vyskytují se zde typické invokační formule ({\em chaire}). Pro popis hrobky se ve dvou případech používá taktéž zcela typické vyjádření ({\em tymbos}), z čehož v jednom případě na nápise {\em IK Byzantion} 305 byl termín upřesněn jako navršený hrob čili mohyla ({\em chóstos tymbos}). Dále se zde vyskytuje jednou termín pro hrob samotný ({\em tafos}).} Všechny dochované termíny jsou použity v jejich původním významu a v kontextu v jakém bychom je mohli najít i ve zbytku řecky mluvícího světa té doby. Veškeré důkazy tedy poukazují na přetrvávání tradičních funerálních zvyklostí a jejich epigrafických projevů ve 3. st. př. n. l. pouze v rámci řecké, případně makedonské komunity. Zapojení thrácké komunity je pozorovatelné maximálně pouze na úrovni jedinců, žijících v prostředí řeckých komunit.

\subsubsection[sekundární-funerální-nápisy-2]{Sekundární funerální nápisy}

Nápis pochází z thrácké aristokratické hrobky z Kazanlackého údolí na řece Tonzos, z území tradičně ovládaného kmenem Odrysů. Dle bohaté pohřební výbavy se usuzuje, že se mohlo jednat o členy thrácké aristokracie, podobně jako v podobných případech v předcházejících stoletích (Sharankov 2005, 29-35). {\em Dipinti} {\em SEG} 58:703 bylo nalezeno na fresce v hrobové komoře a má dvě části. První části nese jméno Seutha, syna Rhoigova a souvisí s ním kresba mladého muže, která je umístěna v bezprostřední blízkosti. Druhá část je tzv. autorský podpis malíře jménem Kozimazés, který je znám jako zhotovitel další fresky {\em SEG} 58:674 v hrobce v Alexandrovu, která je datovaná do 4. až 3. st. př. n. l. a o níž jsem hovořila dříve. Charakterem tento nápis odpovídá podobným skupinám funerálních nápisů z 5. a 4. st. př. n. l., kdy thrácká aristokracie využívala písmo zejména utilitárně pro svou soukromou potřebu vně komunity, tedy odlišným způsobem, než můžeme vidět v řeckých komunitách.

\subsection[dedikační-nápisy-5]{Dedikační nápisy}

Dedikačních nápisů se dochovalo celkem pět a, až na jednu výjimku ze Seuthopole, pocházejí všechny z regionu řeckých měst na pobřeží. Věnování byla určena převážně řeckým božstvům jako je Zeus, Dionýsos a Démétér, případně tehdejším významným panovníkům.\footnote{V jednom případě byl nápis věnován Diovi a králi Filippovi, s titulem zachránce ({\em sótér}). Tento nápis {\em I Aeg Thrace} 186 pochází z Maróneie a králem může být myšlen jak makedonský Filip II., tak Filip V. (Loukopoulou {\em et al.} 2005, 372). Další podobná dedikace {\em IK Sestos} 39 určená králi Filippovi pochází z Lýsimachei na Thráckém Chersonésu, nicméně v tomto případě se jedná spíše o makedonského krále Filippa V. (Krauss 1980, 92).}

Dedikační nápisy se vyskytují převážně v kontextu řeckých komunit. Jedinou výjimkou je nápis pocházející z vnitrozemské Seuthopole, která je považována za původně thrácké osídlení se silnou přítomností řeckého či makedonského prvku (Nankov 2012, 120). Výskyt osobních jmen poukazuje na převahu řeckého prvku: řeckých osobních jmen se vyskytlo celkem šest, zatímco thrácké jméno se dochovalo pouze jedno, a to na nápise {\em IG Bulg} 3,2 1732, který pochází právě ze Seuthopole.\footnote{Nápisy {\em IG Bulg} 3,2 1731 a 1732 dosvědčují existenci řeckých kultů na území Seuthopole, nebo kultů nesoucí řecká jména. Nápis {\em IG Bulg} 3,2 1732 věnoval věnoval Amaistas, syn Medista, kněz Dionýsova kultu, což poukazuje na zapojení thrácké populace do chodu původně řeckého kultu v rámci rezidence thráckého vladaře Seutha III. Epigraficky je v Seuthopoli doložen jak kult Dionýsa, tak svatyně Velkých božstev ze Samothráké, nicméně archeologicky se existenci těchto kultů nepodařilo zcela prokázat, ač zde byla nalezena terakotová soška Kybelé a sošky inspirované uměním východního Středomoří (Nankov 2007, 63; Barrett a Nankov 2010, 17).} Tento fakt svědčí o existenci relativně zavedených náboženských praktik, a to zejména na území řeckých měst, ale v případě Seuthopole i mimo ně. Soudě dle dochovaných nápisů, Seuthopole je příkladem komunity, která nebyla čistě thrácká, ale docházelo zde ke kulturnímu a náboženskému kontaktu, ač v relativně malém měřítku.

\subsection[veřejné-nápisy-5]{Veřejné nápisy}

Ve 3. st. př. n. l. je možné sledovat narůstající využití psaného slova za účelem uplatňování autority existujících autonomních politických jednotek na území Thrákie. Celkem se dochovalo 36 veřejných nápisů, reprezentujících až 31 \letterpercent{} všech nápisů z daného období.\footnote{Oproti 4. st. př. n. l. je možné sledovat nárůst až na čtyřnásobek celkového počtu veřejných nápisů.} Veřejné nápisy až na tři výjimky pocházejí z řeckých měst na pobřeží, a z Mesámbrie na černomořském pobřeží dokonce až 18 exemplářů.\footnote{Tento vysoký pomčet veřejných, ale i soukromých nápisů z Mesámbrie z 3. st. př. n. l. může být dán jak stavem archeologických výzkumů, které se odehrávaly zejména v 60. až 80. létech 20. století, ale částečně i ekonomickým a politickým významem, který Mesámbriá v průběhu 3. st. př. n. l. měla (Velkov 1969; 2005; Venedikov 1969; Ognenova-Marinova {\em et al.} 2005).} Dva nápisy z vnitrozemí pocházejí z původně thráckých osídlení, která hrála důležitou roli i za makedonských místodržících, konkrétně ze Seuthopole a Kabylé. Typologicky se jedná 31 dekretů vydaných politickou autoritou ({\em pséfisma}), z čehož devět bylo honorifikačních nápisů.

Dekrety jsou nejčastějších dochovaným typem veřejného nápisu, protože sloužily jako jeden z projevů suverenity politické autority, která je vydala.\footnote{Veřejné nápisy byly tesány do kamene a byly určeny pro veřejné vystavení. Jejich primárním účelem bylo jednak informovat o daném nařízení či usnesení politické autority, kterou v případě řeckých {\em poleis} byl lid, zastoupený termíny {\em démos} a {\em búlé,} a v případě thrácké aristokracie kmenový vůdce, označovaný termínem {\em basileus}. Předpokládá se, že každý si mohl nařízení přečíst, pokud toho byl schopen, a případně se na něj odvolat, pokud by docházelo k jeho porušování.} Zajišťovaly tak osobám, které spadaly pod vliv dané autority ochranu, základní práva a výměnou za jejich loajalitu určitou společenskou prestiž. Dále se částečně dochovalo jedno nařízení z Abdéry o platbách za dodávání pravdivých informací státu ({\em I Aeg Thrace} 2), jeden veřejný nápisy s náboženskou tematikou, seznam osob nejistého významu a jeden blíže neurčený nápis veřejného charakteru.

Ve zvýšené míře se na nápisech nachází administrativní termíny, a to celkem 25 termínů v 77 výskytech.\footnote{Předně se jedná o instituce a funkce zajišťující chod státního aparátu či symbolizující politickou autoritu samotnou jako je {\em démos}, {\em búlé}, {\em archón}, {\em stratégos}, {\em basileus} atd. Nejčastěji se opakujícím termínem byl {\em démos}, který se objevil 17krát, dále {\em polis} s 11 výskyty a {\em búlé} s šesti výskyty.} Častý výskyt tradičních termínů vychází z ustálené formy dekretů a formulí, které se používaly v celém řeckém světě ve velmi podobné formě. Texty nápisů jsou často velmi popisné a detailně zmiňují veškeré situace, v nichž je daný text nápisů platný a k čemu konkrétně dává pravomoci. Dále se zde vyskytuje celá řada specializovaných povolání, funkcí a referencí na existující společenskou hierarchii, což svědčí o narůstající komplexitě komunit, které nápisy vydávaly (Tainter 1988, 106-108).\footnote{Mezi objevující se nová povolání a funkce patří zejména funkce spojené s výkonem chodu státu a zajištění dodržování publikovaných nařízení, jako například {\em tamiás}, {\em argyrotamiás}, {\em gymnasiarchés}, {\em polítés}, {\em oikonomos}, {\em archón}, {\em stratégos}, {\em basileus}, {\em kéryx}, {\em theóros} či {\em presbys}.} Celá řada specializovaných termínů a formulí odkazuje na komplexní procedury spojené s pořizováním nápisů a jejich veřejným vystavováním. Jednalo se tedy pravděpodobně o ustanovenou proceduru, organizovanou obcí, podobně jak ji známe i z jiných řeckých komunit té doby. Zejména u honorifikačních nápisů z řeckých obcí je možné pozorovat výskyt formulí, které reflektují ustálené administrativní procedury a existující instituce pověřené epigrafickou produkcí. Ač je obsah formulí stejný či velmi podobný, jejich konkrétní forma se liší město od města.\footnote{Text dekretů se typicky sestává z uvedení instituce udělující privilegia ({\em búlé}, {\em démos}, {\em polis} či kolektivní pojmenování občanů města), komu jsou privilegia a pocty uděleny s uvedením důvodů vedoucích k udělení poct. Dále často následuje výčet a specifika udělených privilegií a praktické informace o financování a veřejném vystavení nápisu. V dochovaných textech existuje mnoho variant počínaje pořadím privilegií, konkrétními udělenými poctami, způsobem veřejného vystavení, institucemi vydávajícím dekret. Dále se zde vyskytují i místní varianty použití určitých slovesných tvarů či infinitní větné konstrukce ({\em edoxe} vs. {\em dedochthai}, {\em edókan} vs. {\em dedosthai}), kde v jednotlivých {\em poleis} převládá použití jednoho nebo druhého tvaru.} Nejjednotnější formu mají nápisy z Odéssu, kde většina nápisů vycházela ze stejného vzoru, což poukazuje na zavedený systém byrokratických procedur. Nápisy z ostatních řeckých měst obsahují lehce pozměněné formulace a lokální varianty textu, což nasvědčuje na velmi podobné politické uspořádání a totožné procedury. Tento fakt nicméně poukazuje i na absenci sjednocující autority a normy, která by skoordinovala formální stránku nápisů. Řecká města tedy i ve 3. st. př. n. l. vystupovala a jednala jako autonomní politické jednotky, jejichž procedurální postupy byly do značné míry konzervativní.

Veřejné nápisy pocházející z thráckého kontextu přejímají jazyk a formu řeckých usnesení, ač jejich obsah nemá příliš paralel v celém antickém světě. Konkrétně se jedná o nápisy {\em IG Bulg} 3,2 1731 ze Seuthopole a {\em Kabyle} 2 z Kabylé, které byly v této době původně thráckými komunitami se silnou makedonskou přítomností, a to se odráží i na projevu politické autority na nápisech.\footnote{Nápis {\em IG Bulg} 3,2 1731 ze Seuthopole vydala vdova po thráckém panovníkovi Seuthovi III. Bereníké, která tak formou přísahy řeší nastalou situaci s paradynastou Spartokem z Kabylé, které se nachází zhruba 90 km po toku řeky Tonzos. Bereníké uzavírá se Spartokem dohodu, pravděpodobně za účelem uchování postavení a moci pro své syny Hebryzelma, Térea, Satoka a Sadala (Ognenova-Marinova 1980, 47-49; Velkov 1991, 7-11; Calder 1996, 172-173; Tacheva 2000; Archibald 2004, 886). Částečně dochovaný nápis {\em Kabyle} 2 (Velkov 1991, 11-12) byl vydán pravděpodobně v polovině 3. st. př. n. l. politickou autoritou řecké obce. Velkov navrhuje jako nejpravděpodobnější místo vzniku Mesámbrii a jednalo by i se tak o vzájemnou smlouvu mezi Kabylé a blízkou Mesámbrií na černomořském pobřeží. Po polovině století Kabylé i Mesámbriá pravděpodobně spadaly pod vliv keltského panovníka Kavara, který v obou městech nechal razit mince pro svou potřebu (Draganov 1993, 75-86, 107).} Podobně jako v případě nápisu z Pistiru ze 4. st. př. n. l. se mohlo jednat o komunikační strategii thrácké aristokracie vůči řecky mluvícímu obyvatelstvu, či o snahu o zavedení zvyklostí typických pro řeckou komunitu, které však neměly dlouhého trvání. Vzhledem k ojedinělému užití nápisů ve veřejné funkci nejde příliš mluvit o dlouhodobém přejímání zvyků či organizace, ale spíše o dokumentech vzniklých jako reakce na aktuální politickou situaci a snahu se s ní vyrovnat způsobem obvyklým pro jednu zúčastněnou stranu. Absence institucí a úřadů typických pro řecké {\em poleis} poukazuje na specificitu thrácké politické autority a společenské organizace, kde hlavní roli hrál panovník a nejbližší okruh aristokratů, a nikoliv lid a státní instituce.

\subsection[shrnutí-9]{Shrnutí}

Ve 3. st. př. n. l. i nadále většina epigrafické produkce pochází z řeckých měst na pobřeží a vykazuje všechny charakteristické rysy typické pro řecký svět té doby. Zvýšení celkového počtu veřejných nápisů z těchto komunit může nasvědčovat na zavádění nových procedur a institucí do měst, či alespoň větší míru využívání nápisů pro účely politické organizace a vedení administrativy. Naopak snížení počtu soukromých nápisů může poukazovat na míru nejistoty, které museli obyvatelé čelit, což mohlo souviset s invazí keltských kmenů na počátku 3. st. př. n. l. Na poměrně bouřlivý charakter této doby poukazují i destrukční vrstvy z četných archeologických nalezišť, mimo jiné i z Pistiru.

Ve vnitrozemí se nápisy objevují pouze v kontextu aristokratických kruhů a slouží zejména k upevnění společenské pozice v rámci komunity, ale v ojedinělých případech i jako prostředek komunikace s řeckými či makedonskými partnery. Výjimečnou roli zaujímají multikulturní komunity v Seuthopoli a Kabylé, které vycházejí z thráckých kořenů, avšak v určité míře přijímají i prvky tradičně označované jako řecké či makedonské. Dalším potenciálním místem kontaktu kultur je Hérakleia Sintská, původně makedonská vojensko-obchodní stanice u středního toku řeky Strýmónu. Epigrafické důkazy však i nadále poukazují spíše na uzavřený charakter komunit a přetrvávání tradičních kulturních hodnot v jejich rámci.

\stopcomponent