
\environment ../env_dis
\startcomponent section-charakteristika-epigrafické-produkce-ve-4.-až-3.-st.-př.-n.-l.
\section[charakteristika-epigrafické-produkce-ve-4.-až-3.-st.-př.-n.-l.]{Charakteristika epigrafické produkce ve 4. až 3. st. př. n. l.}

Nápisy datované od 4. do 3. st. př. n. l. pocházejí převážně z řeckého prostředí na pobřeží. I nadále má největší počet nápisů funerální funkci, avšak začínají se objevovat i dedikace věnované božstvům řeckého původu. Nečetné nápisy z vnitrozemí i nadále slouží potřebám thrácké aristokracie a jejich společenská funkce se odlišuje od využití nápisů v řeckých městech. Komunity si i nadále udržují konzervativní charakter a k jejich prolínání dochází ve velmi omezené míře.


\startframedbox
{\em Celkem}~ 25 nápisů

{\em Region měst na pobřeží}~ Abdéra 2, Apollónia Pontská 4, Byzantion 3, Maróneia 3, Mesámbria 3, Perinthos (Hérakleia) 1, Zóné 1 (celkem 17 nápisů)

{\em Region měst ve vnitrozemí}~ Beroé (Augusta Traiana) 1, Filippopolis 3, (Hadriánúpolis) 1\footnote{Celkem tři nápisy nebyly nalezeny v rámci regionu známých měst, editoři korpusů udávají jejich polohu vzhledem k nejbližšímu modernímu sídlišti (tři lokality ve vnitrozemí).}

{\em Celkový počet individuálních lokalit}~ 14

{\em Archeologický kontext nálezu}~ funerální 5, sídelní 3, sekundární 3, neznámý 14

{\em Materiál}~ kámen 23 (mramor 16, vápenec 3, jiný 2, neznámý 4), jiný 1, neznámý 1

{\em Dochování nosiče}~ 100 \letterpercent{} 3, 75 \letterpercent{} 1, 50 \letterpercent{} 6, 25 \letterpercent{} 2, kresba 4, nemožno určit 9

{\em Objekt}~ stéla 23, nástěnná malba 1

{\em Dekorace}~ reliéf 8, bez dekorace 17; figurální dekorace 0, architektonické prvky 10 nápisů (vyskytující se motiv: naiskos 7, florální motiv 2, neznámý 1)

{\em Typologie nápisu}~ soukromé 20, veřejné 3, neurčitelné 2

{\em Soukromé nápisy}~ funerální 14, dedikační 4, vlastnictví 1, jiný 1

{\em Veřejné nápisy}~ honorifikační dekrety 1, státní dekrety 1, neznámý 1

{\em Délka}~ aritm. průměr 4,24 řádku, medián 2, max. délka 20, min. délka 1

{\em Obsah}~ dórský dialekt 1; stoichédon 1; hledané termíny (administrativní 4 - celkem 4 výskyty, epigrafické formule 6 - celkem 6 výskytů, honorifikační 7 - celkem 7 výskytů, náboženské 3 - celkem 4 výskyty, epiteton 0)

{\em Identita}~ řecká božstva 2, kolektivní identita 2 - obyvatelé řeckých obcí z oblasti Thrákie, celkem 23 osob na nápisech, 17 nápisů s jednou osobou; max. 2 osob na nápis, aritm. průměr 1,15 osoby na nápis, medián 1; komunita převládajícího řeckého charakteru, jména pouze řecká (60 \letterpercent{} - celkem 15 nápisů), thrácká (8 \letterpercent{} - celkem 2 nápisy), kombinace řeckého a thráckého (0 \letterpercent{}), jména nejistého původu (12 \letterpercent{}); geografická jména 0;



\stopframedbox

Nápisů datovaných do 4. až 3. st. př. n. l. se dochovalo 25, což představuje úbytek o 80 \letterpercent{} oproti podobné skupině nápisů datovaných do 5. až 4. st. př. n. l. Skupina nápisů ze 4. až 3. st. př. n. l. vykazuje mírný nárůst lokalit v thráckém vnitrozemí v okolí řeckých a makedonských sídel v Pistiru a Seuthopoli a podél hlavních toků, nicméně i přesto se většina epigrafické produkce nachází na pobřeží Černého, Marmarského a Egejského moře, jak dokazuje následující mapa 6.03 v Apendixu 2.

Nápisy datované do 4. až 3. st. př. n. l. pokračují v tradicích ustanovených v předcházejících obdobích a nedochází k zásadní kulturní změně zaznamenané na epigrafické produkci.\footnote{V 58 \letterpercent{} se jedná o funerální nápisy z okolí řeckých měst, ve 32 \letterpercent{} o nápisy dedikační a ve 24 \letterpercent{} o nápisy veřejné. Nosiče nápisů jsou převážně zhotoveny z kamene, sloužící soukromým účelům a veřejnému vystavení, nicméně dva nápisy pocházejí z interiéru hrobky thráckého aristokrata a typologicky odpovídají podobným utilitárním nápisům na cenných předmětech z 5. a 4. st. př. n. l. Nápis {\em SEG} 39:653 nespadá ani do kategorie funerálních či dedikačních nápisů, nicméně je důležitý vzhledem k tomu, že pochází z vnitrozemí, z regionu pozdější Hadriánopole. Nápis byl publikován pouze částečně, a je znám v podstatě pouze jeho text, který zní „{\em Hebryzelmis, syn Seutha, Prianeus}". Hebryzelmis i Seuthés jsou thrácká aristokratická jména, nicméně badatelé si nejsou jisti původem etnického jména Prianeus, ale obecně ho považují za jméno thráckého kmene z egejské oblasti (Veligianni 1995, 158). Bohužel více informací o nápisu není dostupných, což znemožňuje jakékoliv další závěry.}

\subsection[funerální-nápisy-4]{Funerální nápisy}

Celkem se z této doby dochovalo 13 funerálních nápisů patřící do kategorie primárních funerálních nápisů na stélách. Těchto 13 nápisů pochází výhradně z řeckých měst na mořském pobřeží a vyskytují se na nich pouze řecká jména, podobně jako v předcházejících staletích.

Dva nápisy pocházejí z vnitřní architektury mohylových hrobek, které obě pravděpodobně patřily thráckým aristokratům, soudě dle bohaté pohřební výbavy a jejich umístění na území ovládaném kmenem Odrysů. Jeden nápis pochází z hrobky u moderní vesnice Alexandrovo na dolním toku řeky Hebros a druhý z hrobky u moderní vesnice Kupinovo u Veliko Turnova. Texty jsou vytesány či vyškrábány do stěn vnitřní komory hrobek. V případě hrobky z Alexandrova poukazují na majitele hrobky či zhotovitele hrobky či její výzdoby - nástěnných maleb zobrazující lov divokého kance, do nichž je právě vyryto zmíněné {\em graffito} {\em SEG} 54:628 (Kitov 2004, 45-46). Tento malíř je znám ještě z nápisu z hrobky v Kazanlaku, která bývá datována do 3. st. n. l. (Sharankov 2005, 29-35). Druhý nápis 46:852 se nachází na říčním kameni vestavěném do vnitřní konstrukce hrobky a nese nápis σκιάς nejasného významu. Z výše uvedeného plyne, že použití písma v kontextu thrácké aristokracie je identické s nápisy z 5. a 4. st. př. n. l. a nedochází k zásadním proměnám kulturních zvyklostí.

\subsection[dedikační-nápisy-4]{Dedikační nápisy}

V tomto období se poprvé objevily i tři dedikační nápisy ve vnitrozemí, z celkově čtyř dochovaných nápisů. Tyto nápisy nalezené ve vnitrozemí pocházejí komunit tradičně označovaných jako řecké, jako je tomu v případě nápisu z Pistiru či řecko-thrácké se silnými řeckými konotacemi a aristokratickými vazbami na řeckou kulturu, v případe nápisu ze Sborjanova či z okolí Seuthopole.\footnote{Seuthopolis byla sídlem odryského panovníka Seutha III., kterou si nechal postavit na březích řeky Tonzos v dnešním Kazanlackém údolí ve střední Thrákii. Jednalo se o opevněné sídlo o velikosti 4 ha, postavené dle vzorů rezidencí hellénistických vladařů. Byly zde nalezeny domy řecké typu, pravoúhlé uspořádání domů a ulic, množství řeckých importů, graffit a mimo jiné i řecky psaný nápis, tzv. seuthopolský nápis {\em IG Bulg} 3, 2 1731, o němž podrobněji hovořím níže (Dimitrov a Chichikova 1978; Čičikova 1984; Domaradzka 2005, 299). Nedaleko vesnice Sborjanovo v severovýchodním Bulharsku se našlo sídlo gétských panovníků, často označované jako hlavní město kmene Getů a ztotožňované se sídlem panovníka Dromichaita, Chelis (Stoyanov 1997; 2001, 207-219). I odsud pocházejí graffiti s thráckými jmény (Domaradzka 2005, 298) a dedikační nápis {\em SEG} 55:739.} Nápis {\em SEG} 55:739 z lokality poblíž moderního Sborjanova obsahoval tradiční formuli dedikační nápisů ({\em euchén}) a byl věnován bohyni Fosforos, která je nejčastěji spojována s Artemidou.\footnote{V dalším thráckém městě Kabylé se bohyně Fosforos stala dokonce patronkou města a hlavním vyobrazením na mincích ražených v hellénismu v Kabylé (Janouchová 2013, 103-104). Kult Artemis {\em Fosforos} je znám i z Byzantia, kde se konaly slavnosti Bosporií, doprovázené průvodem s pochodněmi, podobně jako ve 4. st. př. n. l. v Athénách (Janouchová 2013, 97; Lajtar 2000, 39-41: {\em IK Byzantion} 11, nedatovaný nápis z Byzantia).} Zbylé texty neobsahují věnování božstvu, ale pouze krátkou identifikaci dedikanta.\footnote{Dochovaná jména jsou pouze řeckého původu, a to jak v případě jmen dedikantů, tak i jejich rodičů. Krátké texty jsou psány řecky a nevykazují žádné odchylky a nepravidelnosti v užití řečtiny.} Ač byly tyto dedikační nápisy nalezeny v thráckém vnitrozemí, vše nasvědčuje na udržení kontinuity řeckých náboženských tradic a nedokazuje ovlivnění místními thráckými náboženskými představami.

\subsection[veřejné-nápisy-4]{Veřejné nápisy}

Do této skupiny patří celkem tři nápisy, z nichž dva jsou honorifikační dekrety vydané autoritou řecké {\em polis}\footnote{Nápis {\em I Aeg Thrace} 400 pochází z antického města Drys na egejském pobřeží a jedná se o honorifikační dekret pro Polyárata, syna Histiáiova, kterému byla udělena privilegium proxénie, společně s udělením občanství, odpuštěním daní a právo volného vstupu do města. Obyvatele města Drys zastupují v tomto případě archónti, kteří reprezentují politickou autoritu daného města.} a jeden z nich pochází z thráckého vnitrozemí.\footnote{Honorifikační dekret {\em IG Bulg} 3,1 1114 pochází přímo z vnitrozemí z okolí moderní vesnice Batkun v regionu Filippopole, nedaleko řeckého {\em emporia} Pistiros. Text nápisu je znám bohužel jen částečně, nicméně z dostupného textu plyne, že občané neznámého města nechali neznámým bratrům postavit sochu a umístit jí ve svatyni Apollóna, a navíc se zavázali, že bratři budou vyznamenáni věncem při každé slavnosti konané pravděpodobně v dané svatyni. Nápis byl pravděpodobně nalezen u vesnice Batkun v kontextu venkovské svatyně věnované Asklépiovi {\em Zymdrénovi}, která se nachází na svazích pohoří Rodopy nedaleko Filippopole (Tsonchev 1941). Bravo a Chankowski (1990, 296-299) zpochybňují svatyni v Batkunu jako původní místo, kde nápis stál a jako alternativu navrhují kontext řecké komunity v nedalekém řecké {\em emporiu} Pistiros.} Není zcela jasné, kdo nápis vydal, ale usuzuje se, že se mohlo jednat o instituci reprezentující lid nedaleké Filippopole či Seuthopole (Archibald 2004, 886-889). Jedná se tak o první dochovaný dekret z vnitrozemí vydaný jinou politickou autoritou, než byl thrácký panovník, což může poukazovat na pomalu se měnící politické uspořádání v Thrákii.

\subsection[shrnutí-8]{Shrnutí}

Nápisy datované do 4. až 3. st. př. n. l. poukazují na narůstající pronikání epigrafické produkce do thráckého vnitrozemí, které je patrné již u skupiny nápisů datovaných do 4. st. př. n. l. Ve vnitrozemí se nápisy objevují zejména v okolí řeckých a makedonských osídlení, v nichž mohli žít i místní obyvatelé či minimálně se s nimi museli stýkat na každodenní bázi. Nicméně i v okolí těchto sídel si epigrafická produkce udržuje tradiční řecký charakter a dle osobních jmen je do publikační činnosti zapojena pouze populace nesoucí řecká jména a dodržující řecké zvyklosti.

\stopcomponent