
\environment ../env_dis
\startcomponent section-charakteristika-epigrafické-produkce-ve-4.-st.-n.-l.-až-5.-st.-n.-l.
\section[charakteristika-epigrafické-produkce-ve-4.-st.-n.-l.-až-5.-st.-n.-l.]{Charakteristika epigrafické produkce ve 4. st. n. l. až 5. st. n. l.}

Nápisy datované do 4. až 5. st. n. l. pocházejí převážně z velkých měst na pobřeží. Pokračuje úbytek jak celkového počtu nápisů, tak i epigraficky aktivních komunit. Komunity jsou spíše uzavřené, nedochází ke kulturní výměně na úrovni předcházejících století. Zcela převládají soukromé nápisy funerální funkce, na nichž je patrný silný vliv křesťanství, který ovlivňuje jak podobu, tak obsah nápisů. Dochází k opětovné individualizaci nápisů, ač rozsah nápisů se nijak nemění.

\placetable[none]{}
\starttable[|l|]
\HL
\NC {\em Celkem:} 35 nápisů

{\em Region měst na pobřeží:} Byzantion 4, Maróneia 13, Perinthos (Hérakleia) 13 (celkem 30 nápisů)

{\em Region měst ve vnitrozemí:} Augusta Traiana 2, Serdica 2, Traianúpolis 1 (celkem 5 nápisů)

{\em Celkový počet individuálních lokalit}: 8

{\em Archeologický kontext nálezu:} funerální 2, sídelní 2, náboženský 1, sekundární 6, neznámý 24

{\em Materiál:} kámen 32 (mramor 23; jiný 1, neznámý 8), kov 2 (olovo 2), jiný 1

{\em Dochování nosiče}: 100 \letterpercent{} 4, 75 \letterpercent{} 3, 50 \letterpercent{} 3, 25 \letterpercent{} 10, oklepek 1, nemožno určit 14

{\em Objekt:} stéla 32, architektonický prvek 1, mozaika 1, jiný 1

{\em Dekorace:} reliéf 18, jiná 1, bez dekorace 16; reliéfní dekorace figurální 2 nápisy (vyskytující se motiv: zvíře 1, scéna lovu 1, skupina lidí 1, stojící osoba 1, socha 1), architektonické prvky 17 nápisů (vyskytující se motiv: naiskos 9, florální motiv 1, jiný 9 (kříž 9)

{\em Typologie nápisu:} soukromé 30, veřejné 1, neurčitelné 4

{\em Soukromé nápisy:} funerální 28, jiný 2 (proklínací destička 2)

{\em Veřejné nápisy:} jiný 1 (z toho hraniční kámen 1)

{\em Délka:} aritm. průměr 6,67 řádku, medián 6, max. délka 22, min. délka 1

{\em Obsah:} latinský text 1 nápis, písmo římského typu 6; hledané termíny (administrativní termíny 8 - celkem 12 výskytů, epigrafické formule 7 - 23 výskytů, honorifikační 0 - 0 výskytů, náboženské 0 - 0 výskytů, epiteton 0 - počet výskytů 0)

{\em Identita:} vymizení řecké a thrácké náboženské terminologie, včetně vymizení lokálních kultů z nápisů, křesťanství jediným náboženstvím, regionální epiteton 0, subregionální epiteton 0, kolektivní identita 0 termínů; celkem 28 osob na nápisech, 15 nápisů s jednou osobou; max. 5 osob na nápis, aritm. průměr 0,8 osoby na nápis, medián 1; komunita řeckého a římského charakteru, thrácký prvek zastoupen minimálně, jména pouze řecká (17,14 \letterpercent{}), pouze thrácká (0 \letterpercent{}), pouze římská (8,57 \letterpercent{}), kombinace řeckého a thráckého (2,85 \letterpercent{}), kombinace řeckého a římského (14,28 \letterpercent{}), kombinace thráckého a římského (0 \letterpercent{}), kombinovaná řecká, thrácká a římská jména (0 \letterpercent{}), jména nejistého původu (14,27 \letterpercent{}), beze jména (42,85 \letterpercent{}); geografická jména z oblasti Thrákie 0, mimo Thrákii 0;

\NC\AR
\HL
\HL
\stoptable

Celková produkce je ve 4. až 5. st. n. l. zhruba desetinová při porovnání s dobou největší epigrafické aktivity, tedy 2. a 3. st. n. l. Epigrafická produkce se do jisté míry vrátila na úroveň 1. či 2. st. př. n. l., kdy většina nápisů byla produkována v pobřežních oblastech a ve vnitrozemí se nápisy objevily pouze sporadicky. Většina z 35 nápisů z daného období pochází z přímořských oblastí, a to zejména z regionu Maróneie a Hérakleie, jak je patrné na mapě 6.10a v Apendixu 2. Nápisy ve vnitrozemí pocházejí převážně z urbánních center, jako je Serdica či Augusta Traiana, kde dochází k velmi omezenému přežívání epigrafické kultury.

Nejčastější formou objektu nesoucí nápis je i nadále mramorová stéla, z jiných materiálů se dochovaly dvě proklínací destičky psané na olovo, jeden nápis na mozaice a jeden nápis na nástěnné malbě uvnitř hrobky.

\subsection[funerální-nápisy-18]{Funerální nápisy}

Funerálních nápisů se celkem dochovalo 28 a i nadále pocházejí převážně z křesťanské komunity, jak je patrné z obsahu i dekorace náhrobků samotných.\footnote{Místo spočinutí označují slova jako {\em mnémeion} či {\em mnéma} označující jak hrob, tak stélu samotnou. Dále se začíná objevovat termín {\em thesis}, typický právě pro křesťanské nápisy, označující místo posledního spočinutí. Termín {\em latomeion} se vyskytuje celkem pětkrát a označuje sarkofág, který nese nápis a zároveň slouží jako místo posledního odpočinku. Sarkofágy pocházejí převážně z křesťanské komunity z Hérakleie a sloužily pro rodinné pohřby, v jednom případě až pro šest lidí. Texty na sarkofágu typicky označovaly nebožtíka jako řádného obyvatele Hérakleie. V šesti případech nápisy obsahovaly formuli, které zakazovala jejich další používání pod pokutou, vymahatelnou v rámci samosprávy či církve právě v Hérakleii. Podobné nápisy existovaly v několika městech již od 1. st. n. l. Podle množství dochovaných nápisů s identickým, či velmi podobným textem se dá usuzovat, že se jednalo o poměrně častý problém, který ve 4. st. n. l. přetrval zejména v Hérakleii, např. na nápise {\em Perinthos-Herakleia} 180.} I nadále si uchovaly poměrně informativní a interaktivní charakter: v osmi případech nápisy promlouvají k náhodně procházejícímu poutníkovi ({\em chaire parodeita}) a sdělují mu životní osudy zemřelého. Zemřelý je identifikován pomocí osobního jména a jména rodiče, případně jeho přináležitost ke křesťanské komunitě. Poměrně dlouhý rozsah nápisů poskytuje řadu informací nejen o nebožtíkovi, ale i o jeho rodině.\footnote{Ve dvou případech se jednalo o vojáky z povolání, konkrétně o legionáře a {\em centenaria}, v jednom případě o lékaře, v jednom případě o námezdního dělníka, dále o architekta, stříbrotepce, mincovního mistra a výběrčího daní. Dozvídáme se o manželkách zesnulých, jejich potomcích a někdy i o věku, jehož se dožili. Výjimku tvoří nápis {\em SEG} 49:871 nalezený na nástěnné malbě uvnitř hrobky číslo 252 v regionu města Augusta Traiana s typickou formulí přející štěstí ({\em agathé týché}).}

Identita osob byla jasně dána křesťanskou vírou, a nebylo tedy nutné uvádět geografický původ. K prolínání onomastických tradic téměř již nedocházelo s tím, že převaha dochovaných osobních jmen byla řeckého původu, pouze s malým podílem jmen římského původu, nakolik je možné v této době přisuzovat jednotlivým jménům kulturní původ. Thrácký prvek se až na jednu výjimku z dochovaného epigrafického materiálu zcela vytratil.\footnote{Nápis {\em Perinthos-Herakleia} 183 patřil stříbrotepci s původně thráckým jménem, Mókiánem z Hérakleie, který přijal křesťanskou víru, a text nápisu obsahoval stejnou právní formulku poskytující ochranu před novým použitím sarkofágu, která se vyskytovala v oblasti již od 1. st. n. l.} Tímto zásadním způsobem proměnilo křesťanství podobu a obsah epigrafické produkce v Thrákii, ale i mimo ni. Přestala být oceňována identita politická či vojenská kariéra a společenská prestiž v rámci státního aparátu, ale namísto nich na důležité místo ve společnosti nastoupila sounáležitost s křesťanskou obcí.

\subsection[dedikační-nápisy-18]{Dedikační nápisy}

Ve své tradiční podobě se dedikační nápis nedochoval ani jeden, nicméně následující čtyři nápisy je možné zahrnout do široce pojaté kategorie rituálně zaměřených textů: dva nápisy na podstavcích soch, jejichž text se s největší pravděpodobností vztahoval k nedochovaným sochám\footnote{Nápisy {\em SEG} 46:845 a 845,2 zmiňují bohyni Hekaté a pravděpodobně řečníka Aischina, a byly objeveny v rámci archeologických vykopávek na Istanbulském {\em hippodromu}.}, a dále dvě proklínací tabulky.\footnote{{\em SEG} 60:747 a 748, které jsou datované do 4. až 5. st. n. l. Jejich text se sestává z magických formulí a relativně běžného palindromu {\em ablanathanalba} (Gager 1999, 136). Účelem těchto proklínacích destiček bylo získat sílu, lásku, zdraví, peníze či moc, či naopak jejich magickou mocí uškodit nepříteli. Jejich text je často nesrozumitelný a obsahuje magické formule, které měly přimět nadpřirozenou sílu vykonat přání pisatele. Jejich výskyt je vcelku běžný v průběhu celé antiky, jejich dochování z oblasti Thrákie je však poměrně vzácné. Tyto dva texty ({\em SEG} 60:747 a 748) byly nalezeny v průběhu archeologických vykopávek v roce 2010 v moderním Istanbulu, a je možné, že v budoucnosti bude objeveno více podobných nálezů.}

\subsection[veřejné-nápisy-18]{Veřejné nápisy}

Veřejný nápis ze 4. až 5. st. n. l. se dochoval pouze jeden, a to velmi poškozený hraniční kámen {\em I Aeg Thrace} 343 z Maróneie.

\subsection[shrnutí-22]{Shrnutí}

Skupina nápisů datovaných do 4. až 5. st. n. l. vykazuje stejné charakteristiky jako nápisy datované do 4. st. n. l. a poukazuje na poměrně značné změny ve společenském uspořádání. Tyto změny měly za následek nejen téměř úplné vymizení epigrafické produkce ve službách politické autority, ale i ze soukromého sektoru. Proměna kulturních a náboženských projevů společnosti je dobře patrná na pozměněném obsahu a formě nápisů, kde hlavní roli přejímá křesťanská víra a thrácký element se opět vytrácí.

\stopcomponent