
\environment ../env_dis
\startcomponent section-charakteristika-epigrafické-produkce-ve-4.-st.-n.-l.
\section[charakteristika-epigrafické-produkce-ve-4.-st.-n.-l.]{Charakteristika epigrafické produkce ve 4. st. n. l.}

Nápisy 4. st. n. l. následují trend prudkého poklesu produkce, který začal již na přelomu 3. a 4. st. n. l. Dochází k uzavírání komunit, snižuje se i variabilita forem i obsahu nápisů. Spolu s tím se objevují i jasné projevy křesťanské tematiky, a to zejména v oblasti Perinthu, Abdéry a Byzantia. Opět převládá funerální funkce nápisů, jejich obsah je však podstatně odlišný.


\startCSV
{\em Celkem}~ 23 nápisů

{\em Region měst na pobřeží}~ Abdéra 2, Bizóné 1, Byzantion 5, Kallipolis 1, Mesámbria 2, Perinthos (Hérakleia) 7 \cite[celkem18]

{\em Region měst ve vnitrozemí}~ Augusta Traiana 3, Nicopolis ad Istrum 1, Traianúpolis 1, \cite[celkem5]

{\em Celkový počet individuálních lokalit}~ 11

{\em Archeologický kontext nálezu}~ funerální 2, sídelní 4, náboženský 1, sekundární 2, neznámý 14

{\em Materiál}~ kámen 22 (mramor 12; vápenec 3, neznámý 7), jiný 1

{\em Dochování nosiče}~ 100 \letterpercent{} 3, 50 \letterpercent{} 3, 25 \letterpercent{} 3, nemožno určit 15

{\em Objekt}~ stéla 17, architektonický prvek 4, nástěnná malba 1, jiný 1

{\em Dekorace}~ reliéf 6, malovaná 1, bez dekorace 16; reliéfní dekorace figurální 1 nápis \cite[vyskytující se motiv: zvíře1], architektonické prvky 7 nápisů (vyskytující se motiv: naiskos 2, sloup 1, báze sloupu či oltář 1, geometrický motiv 1, florální motiv 1, jiný 2)

{\em Typologie nápisu}~ soukromé 14, veřejné 6, neurčitelné 3

{\em Soukromé nápisy}~ funerální 10, dedikační 1, jiný 3 \cite[z toho popis sochy2]

{\em Veřejné nápisy}~ honorifikační dekrety 2, jiný 4 \cite[z toho milník2]

{\em Délka}~ aritm. průměr 8,34 řádku, medián 6, max. délka 33, min. délka 1

{\em Obsah}~ latinský text 2 nápisy, písmo římského typu 4; hledané termíny (administrativní termíny 13 - celkem 18 výskytů, epigrafické formule 11 - 22 výskytů, honorifikační 0 - 0 výskytů, náboženské 5 - 5 výskytů, epiteton 1 - počet výskytů 1)

{\em Identita}~ řecká božstva 1, egyptská božstva 0, římská božstva 0, křesťanská tematika 9, pokles náboženské terminologie, včetně vymizení lokálních kultů z nápisů, regionální epiteton 1, subregionální epiteton 0, kolektivní identita 3 termíny, celkem 3 výskyty - obyvatelé řeckých obcí z oblasti Thrákie 3, mimo ni 0; celkem 34 osob na nápisech, 11 nápisů s jednou osobou; max. 8 osob na nápis, aritm. průměr 1,48 osoby na nápis, medián 1; komunita řeckého a římského charakteru, thrácký prvek zcela chybí, jména pouze řecká (26,08 \letterpercent{}), pouze thrácká (0 \letterpercent{}), pouze římská (4,34 \letterpercent{}), kombinace řeckého a thráckého (0 \letterpercent{}), kombinace řeckého a římského (39,13 \letterpercent{}), kombinace thráckého a římského (0 \letterpercent{}), kombinovaná řecká, thrácká a římská jména (0 \letterpercent{}), jména nejistého původu (4,34 \letterpercent{}), beze jména (26,08 \letterpercent{}); geografická jména z oblasti Thrákie 3, mimo Thrákii 0;



\stopCSV

Epigrafická produkce ve 4. st. n. l. podstupuje velké změny. V první řadě se jedná o celkový prudký úpadek produkce o 94 \letterpercent{} ze 390 na 23 nápisů. Příčiny tohoto procesu je možné sledovat již v druhé polovině 3. st. n. l. Tehdejší společensko-politická krize vyústila v nestabilitu říše, způsobenou jak vnitřními rozbroji, ekonomickou krizí, ale i hrozícím nebezpečím za hranicemi římského impéria \cite[righttext={, 87-88}][Lozanov2015]. Tyto jevy se samozřejmě podepsaly i na úpadku epigrafické produkce, což vyústilo v poměrně radikální proměnu jejího charakteru. Většina nápisů pochází z pobřežních oblastí, s největší produkcí v Hérakleii, býv. Perinthu, jak je patrné na mapě 6.10 v Apendixu 2.

Nejčastější formou objektu nesoucí nápis je již tradičně mramorová stéla, na níž začínají převládat křesťanské motivy, jako je kříž či christogram.

\subsection[funerální-nápisy-17]{Funerální nápisy}

Zvyk zhotovovat funerální nápisy se tak ve 4. st. n. l. uchoval v Thrákii pouze v rámci křesťanské komunity, převážně z okolo Hérakleie. Funerálních nápisů se celkem dochovalo 10 a přestavují tak nejčetnější skupinu nápisů ze 4. st. n. l. Devět nápisů nese jasné znaky křesťanské víry nebožtíků či pozůstalých, ať už je to slovní vyjádření, zobrazení christogramu či vyobrazení křížů.\footnote{Např. na nápise {\em Perinthos-Herakleia} 219 se setkáváme s frází „Χρειστιανοὶ δὲ πάντες ἔνεσμεν", tedy vědomé přihlášení se všech zmíněných osob ke křesťanské víře.} Fakt, že většina nápisů nese textové či vizuální konotace na křesťanskou víru, není nijak překvapivý, vzhledem k tomu, že hlavních produkční centrum Hérakleia, odkud pochází šest nápisů, byla zároveň jedním ze hlavních sídel křesťanské komunity pro region jihovýchodní Thrákie a křesťanství se v průběhu 4. st. n. l. stalo uznávaným náboženstvím a dokonce oficiální vírou římské říše \cite[righttext={, 92-96}][Dumanov2015]. Osoby na nápisech i nadále nesou jména vzniklá kombinací římského a řeckého jména, a i nadále se v mírně pozměněné formě udržují kulturní zvyklosti předcházejících období, jako je např. fráze o ochraně hrobu či texty nápisů promlouvající ke kolemjdoucím.\footnote{Na nápise {\em Perinthos-Herakleia} 177 vystupuje Aurelios Afrodeiseios spolu s manželkou Aurelií Deionoisií. V nápisu je také uvedeno, že kdokoliv se odváží či pokusí neoprávněně použít hrobku, musí zaplatit pokutu právoplatným dědicům. Celkem se tato fráze objevuje na čtyřech sarkofázích z Hérakleie. Na závěr nápisu je taktéž ve čtyřech případech uvedena tradiční formule {\em chaire parodeita}, avšak s mírně pozměněnou ortografií, která pravděpodobně reflektovala tehdejší výslovnost.}

\subsection[dedikační-nápisy-17]{Dedikační nápisy}

Dedikační nápisy ve 4. st. n. l. téměř vymizely, což může souviset i s ústupem místních kultů a nástupem křesťanské víry. Ze 4. st. n. l. se dochoval pouze jeden dedikační nápis {\em I Aeg Thrace} 19 z Abdéry, který věnovala Sab(b)ais Diovi {\em Hypsistovi}, tedy nejvyššímu bohu.

\subsection[veřejné-nápisy-17]{Veřejné nápisy}

Celkem se dochovalo šest veřejných nápisů, z nichž dva jsou milníky\footnote{{\em Perinthos-Herakleia} 292 z Hérakleie a Velkov 2005 58 z Mesámbrie.}, dva stavební nápisy, dokumentující stavbu brány v Byzantiu (Konstantinopoli) a stavbu věže v Kaliakře\footnote{{\em SEG} 53:651, {\em IG Bulg} 1, 2 12bis.}, a dva honorifikační nápisy věnované tehdejším císařům vysokými městskými úředníky.\footnote{{\em SEG} 51:916 datovaný na poč. 4. st. n. l., a {\em SEG} 52:695 datovaný mezi roky 324-337 n. l., oba z Augusty Traiany.} Nápisy pocházejí z doby tetrarchie, zejména z vlády císaře Konstantina. Ač je jejich celkový počet velmi omezený, i přesto se jedná o tradiční projevy suverenity římské říše a císaře. Primární funkcí těchto nápisů bylo informovat o stavebních aktivitách státního aparátu, ale i šířit pověst a postavení římského císaře v rámci provincie. Malý počet veřejných nápisů může dosvědčovat, že státní aparát byl v této době v krizi, a tudíž i celková produkce nápisů byla velmi nízká, a omezené pouze do první poloviny 4. st. n. l.\footnote{V polovině 4. st. n. l. došlo k dokončení administrativních reforem, které začaly již za císaře Diokleciána: území Thrákie bylo přeměněno na diecéze {\em Thracia} a {\em Dacia}, z nichž Thracia se dále dělila na šest provincií se svými hlavními městy, která se staly administrativními centry nejbližších regionů \cite[righttext={, 91}][Dumanov2015].} Státní instituce do jisté míry i nadále fungovaly, což dokumentují dekrety pocházející z Augusty Traiany či milníky z Hérakleie (dříve Perinthu), obecně však produkce nápisů sloužících veřejnému zájmu končí zhruba po polovině 4. st. n. l. O této doby se vyskytují nápisy pouze pro soukromou potřebu jednotlivců či uzavřených komunit.

\subsection[shrnutí-21]{Shrnutí}

Dochované nápisy ze 4. st. n. l. nasvědčují, že epigrafická produkce byla zhruba od konce 3. st. n. l. na ústupu. Došlo však nezbytně k přeměně formy a obsahu publikovaných nápisů tak, aby více odpovídala tehdejším společenským potřebám. Křesťanská víra se stala jednotícím prvkem nápisů produkovaných v Thrákii během 4. a následně i během 4. až 5. st. n. l. Nápisy téměř zcela vymizely z veřejné sféry, kde došlo k poměrně radikálním přeměnám administrativního uspořádání, což mělo za následek změny i v epigrafické produkci. V soukromé sféře se publikace nápisů udržela pouze v rámci křesťanské komunity.

\stopcomponent