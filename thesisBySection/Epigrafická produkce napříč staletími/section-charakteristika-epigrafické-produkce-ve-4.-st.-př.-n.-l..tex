
\environment ../env_dis
\startcomponent section-charakteristika-epigrafické-produkce-ve-4.-st.-př.-n.-l.
\section[charakteristika-epigrafické-produkce-ve-4.-st.-př.-n.-l.]{Charakteristika epigrafické produkce ve 4. st. př. n. l.}

Většina epigrafické produkce ve 4. st. př. n. l. pochází z řeckých komunit podél mořského pobřeží. I nadále převládají nápisy funerální, ale pozvolna narůstá i počet veřejných nápisů, sloužících převážně k organizaci a uplatnění politické svrchovanosti řeckých komunit, ale i jako prostředek vyjádření diplomatických vztahů mezi řeckými obcemi a thráckými aristokraty. Vyskytující se jména na nápisech nasvědčují na omezené mísení řecké a thrácké onomastické tradice v bezprostředním okolí řeckých měst. Nápisy se objevují ve velmi omezené míře i v thráckém vnitrozemí, kde slouží především potřebám thrácké aristokracie, stejně jako v 5. st. př. n. l.

\startframedbox

 {\em Celkem:} 168 nápisů

{\em Region měst na pobřeží}~ Abdéra 16, Apollónia Pontská 55, Byzantion 7, Chersonésos Molyvótés 1, Dionýsopolis 1, Maróneia 9, Mesámbria 11, Odéssos 5, Perinthos (Hérakleia) 2, Strýmé 29, Zóné 18 \cite[celkem154]

{\em Region měst ve vnitrozemí}~ Beroé (Augusta Traiana) 2, Pistiros 1\footnote{Celkem 11 nápisů nebylo nalezeno v rámci regionu známých měst, editoři korpusů udávají jejich polohu vzhledem k nejbližšímu modernímu sídlišti (čtyři lokality s celkem pěti nápisy), či uvádějí jejich původ jako blíže neznámé místo v Thrákii (šest nápisů).}

{\em Celkový počet individuálních lokalit}~ 26

{\em Archeologický kontext nálezu}~ funerální 58, sídelní 3, nábož. 2, sekundární 21, neznámý 84

{\em Materiál}~ kámen 162 (mramor 92, z toho mramor z Thasu 3, vápenec 44, jiné 20, z čehož je pískovec 4, póros 2, vulkanický kámen 3; neznámý 6), kov 4 (stříbro 2, zlato 1, neznámý kov 1), jiný materiál 1, neznámý 1

{\em Dochování nosiče}~ 100 \letterpercent{} 54, 75 \letterpercent{} 33, 50 \letterpercent{} 34, 25 \letterpercent{} 16, kresba 6, ztracený 1, nemožno určit 24

{\em Objekt}~ stéla 142, architektonický prvek 18, nádoba 2, socha 1, nástěnná malba 1, jiný 2, neznámý 2

{\em Dekorace}~ reliéf 67, malovaná dekorace 6, jiná dekorace 1, bez dekorace 94; reliéfní dekorace figurální 6 nápisů (vyskytující se motiv: jezdec 1, stojící osoba 2, sedící osoba 2, funerální scéna 1), architektonické prvky 61 nápisů (vyskytující se motiv: naiskos 24, sloup 4, báze sloupu či oltář 14, florální motiv 5, architektonický tvar/forma 19, jiný 1)

{\em Typologie nápisu}~ soukromé 153, veřejné 9, neurčitelné 6

{\em Soukromé nápisy}~ funerální 142, dedikační 5, vlastnictví 4, jiný 1, neznámý 1

{\em Veřejné nápisy}~ nařízení 2, náboženské 1, seznamy 1, honorifikační dekrety 1, státní dekrety 2, jiný 1, neznámý 1\footnote{Součet nápisů jednotlivých typů je vyšší než počet veřejných nápisů vzhledem k možným kombinacím jednotlivých typů v rámci jednoho nápisu.}

{\em Délka}~ průměr 2,96 řádku, medián 2, max. 46, min. 1

{\em Obsah}~ dórský dialekt 12, iónsko-attický dialekt 3; stará attická alfabéta 2, stoichédon 3, graffiti 1; hledané termíny (administrativní 11 - celkem 16 výskytů, epigrafické formule 3 - celkem 3 výskyty, honorifikační 3 - celkem 4 výskyty, náboženské 8 - celkem 10 výskytů, epiteton 2)

{\em Identita}~ řecká božstva 6, subregionální hérós 2, kolektivní identita 10 - obyvatelé řeckých obcí, Thrákové jako kolektivní pojmenování, celkem 174 osob na nápisech, 121 nápisů s jednou osobou; max. 7 osob na nápis, aritm. průměr 1,03 osoby na nápis, medián 1; komunita převládajícího řeckého charakteru, jména pouze řecká (72 \letterpercent{}), thrácká (2,97 \letterpercent{}), kombinace řeckého a thráckého (0,59 \letterpercent{}), jména nejistého původu (14,28 \letterpercent{}); geografická jména z oblasti Thrákie 4, geografická jména mimo Thrákii 1;

\stopframedbox

Do 4. st. př. n. l. bylo datováno 168 nápisů, což znamená nárůst o 180 \letterpercent{} oproti skupině nápisů datovaných do 5. st. př. n. l. Ve 4. st. př. n. l. i nadále narůstá počet individuálních lokalit, v nichž byly nápisy nalezeny. Jak dokazuje mapa 6.03 v Apendixu 2, většina nápisů stále pochází z řeckých měst na pobřeží Černého, Marmarského a Egejského moře. Téměř jedna třetina nápisů pochází z řeckého města Apollónia Pontská, která v této době patří k hlavním kulturním a ekonomickým centrům regionu, ale zároveň také k nejlépe archeologicky prozkoumaným městům.\footnote{V posledních několika desetiletích zde byly nalezené nové nekropole s velkým počtem funerálních nápisů (Isaac 1986, 246; Avram, Hind a Tsetkhladze 2004, 931-932; Velkov 2005; Gyuzelev 2002, 2005, 2013). Oproti nápisům z 5. až 4. st. př. n. l. jsou nicméně celková čísla dochovaných nápisů z Apollónie nicméně zhruba o 40 \letterpercent{} nižší, což může být jednak důsledek lehkého útlumu postavení Apollónie, či výsledek do značné míry náhodných archeologických nálezů a široké datace nápisů.} Mezi producenty střední velikosti patří řecká města z pobřežních oblastí jako je Abdéra, Byzantion, Maróneia, Mesámbria, Odéssos, Strýmé a Zóné. Oproti 5. st. př. n. l. je možné zaznamenat objevení nových lokalit ve vnitrozemské Thrákii, a to zejména v okolí dvou hlavních toků, Hebros a Tonzos. Lokality sídelního a funerálního charakteru ve vnitrozemí jsou, dle obecně přijímaného konsenzu, jak řeckého, tak thráckého původu.\footnote{Lokalita Pistiros je archeology považována za řeckou obchodní stanici a říční přístav (Bouzek {\em et al.} 1996, 9). Lokalita Seuthopolis je interpretována jako sídlo thráckého panovníka Seutha III. (Dimitrov, Chichikova a Alexieva 1978, 3-5). Lokality Sborjanovo, Smjadovo, Alexandrovo, Kupino, Naip jsou považovány za hrobky bohatých thráckých aristokratů z kmene Odrysů (Stoyanov 2001, 207-218; Atanasov 2006, 6; Kitov 2001, 15-29; Gergova 1995, 385-392; Delemen 2006, 261-263). Archeologický kontext míst nálezu nápisů je bohužel z poloviny neznámý, dále z jedné třetiny funerální, což znamená, že nápisy byly nalezeny buď v konkrétní hrobce, její blízkosti, či v blízkosti pohřebiště. Ve třech případech pochází nápisy ze sídelního archeologického kontextu, a ve dvou případech bylo místo jejich nálezu určeno jako náboženského původu, např. u nápisů pocházejících ze svatyně. Sekundární archeologický kontext je znám celkem u 21 nápisů, což je zhruba osmina souboru.}

Podobně jako v 5. st. př. n. l. je možné sledovat odlišné charakteristiky nápisů zhotovených na kameni a na kovu či keramice. Jednak se jedná o jejich odlišnou kulturně-společenskou funkci, ale také i jejich výskyt v rámci prostorově odlišných komunit. nápisy na kameni (96 \letterpercent{}) se vyskytovaly převážně na pobřeží, nápisy na jiném médiu pocházejí převážně z vnitrozemí (3 \letterpercent{}).\footnote{U jednoho nápisu editor korpusu neudává materiál nosiče a je tedy neznámý.} Nápisy na kameni jsou datovány do 4. st. př. n. l. v počtu 162 nápisů, z nichž 59 \letterpercent{} nápisů je zhotoveno z mramoru, 28 \letterpercent{} z vápence a zbývající část z místně dostupného kamene.\footnote{V případě pobřeží Egejského moře zdroj mramoru pochází např. z lomů na Thasu.} Podobně jako v 5. st. př. n. l. dochází k využívání místních zdrojů a nemáme důkazy o tom, že by se nápisy staly předmětem dálkového obchodu. \footnote{Stejně jako v 5. st. př. n. l. má převážná část nápisů na kameni charakter soukromého nápisu, s podílem 91 \letterpercent{} nápisů z daného období. Dále jsou to nápisy veřejné se zastoupením 5 \letterpercent{} a nápisy, které nebylo možné určit, které představují zhruba 4 \letterpercent{}.} Oproti tomu nápisy z kovu a nápisy vyryté do stěny hrobky pocházejí z thráckého vnitrozemí či z oblastí na pobřeží Egejského moře tradičně ovládanými thráckým kmenem Odrysů v okolí hory Ganos, mezi řeckými městy Ainos a Perinthos (Archibald 1998, 109-111).\footnote{Stejně jako v případě nápisů na jiném materiálu než kameni, datovaných do 5. st př. n. l., i tato skupina čtyř nápisů pochází z kontextu bohatých pohřbů, patřícím pravděpodobně thrácké aristokracii, jako je Dalakova Mogila u Kazanlaku, lokalita Naip u hory Ganos na pobřeží Marmarského moře, neznámá mohyla u Kazanlaku (Delemen 2006, 251-273; Kitov 1995, 17; Kitov a Dimitrov 2008, 25-32). Jedná se jak o funerální nápisy primární, tak sekundárně umístěné.}

\subsection[funerální-nápisy-3]{Funerální nápisy}

Primární funerální nápisy si i ve 4. st. př. n. l. uchovávají stejný charakter jako v 5. st. př. n. l., doplněné o nápisy na vnitřní architektuře hrobek. Primární funerální nápisy i nadále pocházejí ze stejných či charakteristikou velmi podobných komunit na pobřeží, stejně tak jako sekundární funerální nápisy převážně pocházejí z kontextu thrácké aristokracie ve vnitrozemí.

\subsubsection[primární-funerální-nápisy-1]{Primární funerální nápisy}

Celkem se jedná o 142 nápisů tesaných do kamene, zhotovených nejčastěji ve tvaru stély.\footnote{Typický text funerálního nápisu představuje jméno zemřelého se jménem otce či partnera v genitivu. Pouze v jednom případě tento jednoduchý text doplňují okolnosti smrti a detaily z života zemřelého (Gyuzelev 2013, 20). Průměrná délka funerálního nápisu je 2,3 řádku, nicméně nejdelší text má až šest řádků.} Funerální nápisy pocházejí opět z řeckých komunit podél mořského pobřeží: největší skupina 55 funerálních nápisů pochází z Apollónie Pontské na pobřeží Černého moře. Další významnou skupinu tvoří 29 nápisů ze Strýmé na egejském pobřeží.\footnote{V sedmi případech je dokonce dochováno použití dórského dialektu v původně dórských koloniích Mesámbria s pěti nápisy, Byzantion s jedním; jeden nápis pochází z původně nedórského Odéssu.}

Osobní jména na funerálních nápisech jsou z 85 \letterpercent{} řeckého původu. V jednom případě nápisu {\em I Aeg Thrace} 153 ze Strýmé na egejském pobřeží je zaznamenáno mísení onomastických tradic, kdy jedna osoba nesoucí řecké jméno může mít thráckého otce.\footnote{Jméno otce Dadás není výhradně thrácké, tudíž není možné na jeho thrácký původ poukázat se 100 \letterpercent{} jistotou.} Vyjádření identity na funerálních nápisech na kameni poukazuje na čtyři případy příslušnosti k řecky mluvícím komunitám, naznačené užitím termínů Kyrenáios, Abdérítés, Amfipolítés a Kyzikénos, což poukazuje na stále relativně konzervativní společnost, kde imigrace nebyla častá, či nebylo nutné uvádět svůj geografický původ v rámci funerálních nápisů.\footnote{Geografická jména zmiňují pouze v jednom případě Olynthos, avšak text nápisu {\em I Aeg Thrace} 214 je příliš poškozen, abychom z něj mohli usuzovat něco dalšího.} Hledané termíny poukazují opět na převážně řeckou komunitu, která udržovala tradiční zvyky, nicméně tradiční formule náležící funerálním nápisům se vyskytla pouze jednou. Nesetkáváme se ani s individualizovanou prezentací jednotlivých zemřelých a poukazování na jejich společenský status, výjimkou je jeden nápis propuštěné otrokyně na nápise {\em IG Bulg} 1,2 334novies a.

Dva nápisy zhotovené na stěně vnitřní komory mohylové hrobky pocházejí z okolí hory Ganos\footnote{Text nápisu SEG 56:827bis: ΚΑΘΑΘΑ. Z téže hrobky dále pochází stříbrná nádoba {\em SEG} 56:828, označující majitele nádoby, thráckého aristokrata Térea (Delemen 2006, 261-262). Z historických pramenů víme, že jméno Térés tradičně patřilo panovníkům z rodu thráckých Odrysů, avšak nedokážeme s přesností posoudit, zda se jednalo o téhož Térea, majitele této hrobky (Hdt. 4.80, 7.137; Thuc. 2.29, 2.67, 2.95; Dem. 12.8).} na pobřeží Marmarského moře a z lokality u vesnice Smjadovo v severní části Thrákie.\footnote{Nápis {\em SEG} 52:712 ze Smjadova byl nalezen na překladovém kameni, který byl součástí vnitřní architektury hrobky, patřící pravděpodobně členovi či člence thrácké aristokracie. Vzhledem k tomu, že se jedná o nápis pocházející z vnitřního prostoru hrobky, jednalo se nejspíše o jméno majitelky hrobky: Gonimase(ze), ženy Seutha. Přesné znění nápisu {\em SEG} 52:712 není zcela jasné. Dimitrov (2009, 17-18) navrhuje překlad „Gonimaseze, Seuthova žena", zatímco Dana (2015, 246-247) text čte následovně „Gonimase, Seuthova žena, (i nadále) žije!".} Z uvedeného charakteru nápisů plyne, že přístup do samotných hrobek či k předmětům na kovových předmětech, měla velmi omezené skupina lidí z nejbližšího okolí majitele. Dle původu dochovaných jmen a charakteru pohřební výbavy je možné soudit, že se jednalo o členy thrácké aristokracie. Užití písma v tomto případě bylo omezeno převážně na utilitární funkci označení majitele či zhotovitele a obsah textu se nevázal k funerálnímu ritu samotnému. Svým charakterem se tak nápisy z vnitřní architektury hrobek řadí spíše k nápisům na kovových nádobách a jiných materiálech, nalezených v thráckém vnitrozemí v kontextu aristokratických hrobek v 5. až 3. st. př. n. l.

\subsubsection[sekundární-funerální-nápisy-1]{Sekundární funerální nápisy}

Do skupiny sekundárních funerálních nápisů patří celkem čtyři nápisy na kovových předmětech, zhotovené z drahých kovů: jeden zlatý pečetní prsten {\em SEG} 58:699 nese jméno a podobiznu pravděpodobně svého majitele Seutha, syna Térea. Dále do této skupiny patří stříbrná nádoba nesoucí nápis {\em SEG} 56:828 se jménem Térea, označující pravděpodobně majitele nádoby, zmíněná výše. Další nápis {\em SEG} 53:706 na stříbrné nádobě s textem „{\em Kotys, z Ergiské}", který taktéž označuje majitele a geografický termín Ergiské označuje buď původ majitele či původ nádoby samotné. Bohužel u této nádoby není zcela možné určit její přibližné místo nálezu, protože pochází z aukce, avšak podobá se nádobám pocházejícím z území Thrákie (Loukopoulou 2008, 158-159). Poslední nápis {\em SEG} 46:851 na kovovém kratéru nese nápis označující pravděpodobně její obsah.\footnote{Nápis naznačuje, že je obsah jsou čtyři kyliky, tedy číše na víno.}

Ač byly všechny předměty nalezeny ve funerálním kontextu, obsah textu se nevtahuje specificky k pohřebnímu ritu. Nápisy představují součást pohřební výbavy uložené do hrobu společně se zemřelým a mají čistě utilitární funkci. Původ majitele, alespoň na kolik je možné soudit dle osobních jmen, byl thrácký.\footnote{Seuthés, Térés, Kotys jsou tradiční jména thrácké aristokracie, vyskytující se i v literárních pramenech (Thuc. 2. 29, 2.67, 2.95, 2. 97; Strabo 12.3.29; Tac. {\em Ann}. 2.64-65).} Naopak obsah nádoby je označen typicky řeckým způsobem, což může poukazovat na její řecký původ.\footnote{Písmeno delta označuje číslo 4, a slovo kylix pochází z řečtiny a označuje číši na pití vína.} Nicméně tento nápisy mohl vzniknout daleko dříve, než se nádoba dostala do thráckého prostředí, a její přítomnost dokládá pouze obchodní či diplomatické kontakty a výměnu zboží mezi thráckým vnitrozemím a řeckým pobřežím.

\subsection[dedikační-nápisy-3]{Dedikační nápisy}

Celkový počet dedikačních nápisů se oproti 5. st. př. n. l. výrazně nezměnil. Dochovalo se celkem pět nápisů z oblastí na pobřeží Egejského a Černého moře. Nápisy obsahují tradiční formule dedikační nápisů ({\em epoiésen}, {\em euchén}) a byly věnovány převážně řeckým božstvům Démétér, Kybelé a {\em héróům}, s přízviskem {\em Epénór} a {\em Mesopolités}. Nic nenasvědčuje, že by se Thrákové aktivně podíleli na vydávání dedikačních nápisů či že by nápisy pocházely z čistě thráckého prostředí. Texty jsou krátké a psány řecky, gramaticky a ortograficky nevykazují žádné zvláštnosti oproti standardnímu užití řečtiny a obsahují pouze řecká jména. Osobní jména, jména božstev, forma a obsah nápisů navíc poukazují na řecký původ dedikantů a převážně řecký kontext komunit, z nichž nápisy pocházely.

\subsection[veřejné-nápisy-3]{Veřejné nápisy}

Veřejných nápisů se dochovalo celkem devět, což představuje oproti 5. st. př. n. l. mírný nárůst. Narůstající číslo odpovídá i upevňování pozice politických autorit v regionu v důsledku stabilizace podmínek a nárůstu počtu nových osídlení.\footnote{Jedním z efektů dlouhodobé stabilizace politické situace může být i nárůst společenské komplexity. Tento jev se v rámci epigrafiky může projevit nejen nárůstem celkové epigrafické produkce, ale i vznikem nových funkcí a povolání, a tedy i jejich následnému objevení v textu nápisů.} Typologicky se jedná o šest dekretů, z čehož tři jsou honorifikační udílející pocty významným jedincům. Dále sem patří dvě nařízení, regulující obchodní výměnu, jeden seznam obsahující záznam pravděpodobně dlužných částek či vynaložených nákladů, jeden text náboženského charakteru na pomezí soukromého textu a dva nápis jiného či blíže neznámého charakteru.

Ve 4. st př. n. l. vidíme první náznaky využití nápisů v rámci regulace veřejného politického a ekonomického života: objevují se termíny jako je {\em polis}, {\em démos}, {\em búlé} apod. které poukazují na existenci samosprávných institucí. Tyto instituce dosud nebyly epigraficky potvrzeny z území Thrákie, nicméně pravděpodobně existovaly v daných městech již dříve. Termíny se objevují na nápisech pocházejících z řeckých komunit v Mesámbrii, Dionýsopoli, Zóné, Perinthu a Byzantiu, což svědčí o tradičním uspořádání těchto řeckých měst, o jejich politické autonomii a o fungujícím státním aparátu, schopném vydávat veřejná nařízení.\footnote{Příkladem může být nařízení {\em I Aeg Thrace} 3 z Abdéry datované do poloviny století regulující obchod s otroky a hospodářskými zvířaty.} O existenci diplomatických a ekonomických vztahů řeckých měst a thráckých aristokratů svědčí nápis {\em SEG} 49:911.\footnote{Tento nápis pochází pravděpodobně z prostředí řeckého {\em emporia} Pistiros v thráckého vnitrozemí a jedná se dohodu, v jejímž rámci se reguluje obchod mezi odryským panovníkem Kotyem a řeckými městy na pobřeží Egejského moře, a to konkrétně Maróneiou, Apollónií a Thasem. Dana (2015, 247-248) poukazuje na několik desítek {\em graffit} nalezených v okolí Pistiru, často nesoucí řecká jména či věnování řeckým božstvům, což může naznačovat trvalý pobyt osob řeckého původu (Domaradzka 1996, 2002, 2005, 2013). V okolí Pistiru se našlo několik dalších nápisů psaných osobami s řeckými jmény, mimo jiné i slavný dekret z Batkunu {\em IG Bulg} 3,1 1114, o němž hovořím níže. Vše tedy nasvědčuje, že {\em emporion} Pistiros bylo skutečně řeckou obchodní stanicí, jejíž část, nebo možná celá populace, byla řeckého původu.} Jedná se vůbec o první smlouvu ekonomického charakteru pocházející z území Thrákie, kde thrácký panovník vystupuje jako svrchovaná politická autorita, rovná autoritě řeckých měst na pobřeží. Dle současných interpretací šlo o regulaci již probíhajících ekonomických kontaktů, reagující na aktuální či minulé problémy ve snaze o nápravu a kodifikaci vzájemných vztahů (Velkov a Domaradzka 1996, 209-215; Bravo a Chankovski 1999, 279-290; Graninger 2013, 108-109). O vnitřním uspořádání území ovládaného Odrysy však z nápisu není bohužel možné vyčíst nic dalšího, ale vzhledem o ojedinělému výskytu podobného textu je možné říci, že thrácká aristokracie ve 4. st. n. l. nevyužívala nápisy k uplatnění politické moci a veřejné prezentaci svrchované autority stejným způsobem jako bylo obvyklé v řeckých městech na pobřeží. Naopak řecký charakter osídlení {\em emporia} Pistiros nasvědčuje, že thrácký panovník přistoupil na formu komunikace obvyklou pro řeckou komunitu, ať už pro usnadnění srozumitelnosti sdělení, či jako diplomatické gesto. Nápis je psán řecky, upravuje podmínky vzájemné interakce a zajišťuje ochranu řeckých obchodníků na území Thrákie, která je však garantována autoritou odryského panovníka. V případě tzv. pistirského nápisu se tedy jedná o vzájemnou dohodu řecké a thrácké politické autority, k jejíž prezentaci slouží výrazové a kulturně-společenské prvky pouze jedné ze zmíněných stran, a to strany řecké.

\subsection[shrnutí-7]{Shrnutí}

Z uvedeného je patrné, že nápisy soukromého charakteru ve 4. st. př. n. l. sledovaly podobné trendy jako nápisy v 5. st. př. n. l., a není zde patrný žádný společensko-kulturní předěl. Většina epigrafické produkce se koncentrovala na území řeckých komunit podél mořského pobřeží, nicméně se objevují nápisy i ve vnitrozemí v prostředí thrácké aristokracie.

Thrácká komunita využívala nápisy převážně utilitárně pro označení vlastníka, autora či v souvislosti s funkcí předmětu. Tyto nápisy byly zhotoveny pro velmi úzkou skupinu a jejich vlastnictví či vlastnictví předmětu nesoucí nápis poukazovalo na vysoký společenský status majitele. Jejich výsadní postavení v rámci společnosti nasvědčovalo i jejich uložení v hrobce jako součást pohřební výbavy. Oproti tomu nápisy pocházející z řeckých komunit poukazují na uchovávání kulturních tradic a zvyklostí a pouze omezený kontakt s thráckým obyvatelstvem. Převažují veřejně vystavené funerální nápisy, napomáhající vytváření povědomí o jednotné komunitě a kolektivní paměti.

V případě veřejných nápisů se setkáváme s mírným nárůstem jejich celkového počtu. Jednou z příčin rozšíření epigrafické produkce i mimo soukromou sféru může být jak nárůst společenské komplexity, jak v prostředí řeckých komunit, tak i nárůst politické moci kmene Odrysů. Veřejné nápisy v thráckém prostředí slouží jako prostředek komunikační strategie vůči Řekům, a pravděpodobně neslouží pro komunikaci uvnitř thrácké komunity.

\stopcomponent