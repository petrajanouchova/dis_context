
\environment ../env_dis
\startcomponent section-epigrafická-produkce-a-proměny-společnosti
\section[epigrafická-produkce-a-proměny-společnosti]{Epigrafická produkce a proměny společnosti}

Dochovaný epigrafický materiál, který bylo možné datovat s přesností do jednoho až dvou století, poskytuje poměrně ucelený soubor dat o vývoji společnosti antické Thrákie v průběhu více než 11 století. Soubor analyzovaných 2036 nápisů umožňuje sledovat vývoj epigrafické produkce v jednotlivých časových obdobích a navzájem je srovnávat se známými společensko-politickými změnami.

\subsection[proměny-epigrafické-produkce-v-závislosti-na-společenské-organizaci-a-komplexitě]{Proměny epigrafické produkce v závislosti na společenské organizaci a komplexitě}

Podíváme-li se na celkový počet nápisů, jak ho zobrazuje \in{Graf}[Apendix2:::6.12a] v \in{Apendixu}[Apendix2:::Apendix2], skupiny nápisů datovaných s přesností do jednoho a do dvou století vykazují velmi podobné trendy. K výraznému nárůstu produkce \index{epigrafická produkce} dochází v 5. a 4. st. př. n. l. Další období růstu počtu nápisů jsou ve 2. st. n. l. a pak zejména v 3. st. n. l. Ze srovnání se známými historickými jevy a událostmi tak plyne, že období růstu epigrafické produkce mohou být spojována s dobou intenzivního působení řeckých obcí na území Thrákie v klasické době, kdy došlo k rozšíření epigrafického zvyku i mimo bezprostřední území řeckých {\em poleis}, a kdy se začalo ve velmi omezené míře na nápisech objevovat i thrácké obyvatelstvo, zejména aristokratického původu\index{thrácká aristokracie}. Další období růstu souvisí s dobou relativní politické stability řeckých obcí ve 2. st. př. n. l., kdy byla velká část Thrákie součástí makedonského království. Nejvýraznější nárůst epigrafické produkce nastává ve 2. a 3. st. n. l., kdy dochází k velkému množství administrativních reforem v rámci římské říše a s tím spojeným větším zapojení běžného obyvatelstva do státních a vojenských institucí.

Naopak období poklesu epigrafické produkce oproti předcházejícímu století nastalo na přelomu 4. a 3. st. př. n.l., výrazněji pak v 1. st. př. n. l. a nejstrmější propad pochází z přelomu 3. a 4. st. n. l. Doby poklesu epigrafické aktivity mohou být přičítány období ekonomické a společenské krize, vyčerpání a následné transformace, která vede i k odlišnému pojetí epigrafické produkce. K poklesu produkce dochází zejména na přelomu 1. st. př. n. l. a 1. st. n. l., kdy Thrákie prochází reformou politické moci a vazalské krále thráckého původu střídá autorita římské říše. Tento propad epigrafické produkce nicméně není tak markantní jako téměř ukončení epigrafické produkce na konci 3. st. př. n. l., kdy dochází k poklesu produkce o více než 90 \letterpercent{}, což souvisí s celkovou destabilizací poměrů v římské říši, vleklou ekonomickou krizí, zvyšujícími se nájezdy nepřátelských kmenů a v neposlední řadě i narůstající společenskou rolí křesťanské víry.

Opakující se trendy růstu v době prosperity a stability a poklesu v době nejistoty a úpadku se projevují i napříč typologickými skupinami nápisů. Změny vnitřní infrastruktury se projeví s určitým zpožděním na celkovém počtu veřejných nápisů, které jsou produktem organizované administrativy a aktivity fungující politické autority. Jejich nárůst je patrný ve 3. st. př. n. l. jakožto důsledek zvýšených aktivit makedonských králů, avšak i přesto stále převažují nápisy soukromé povahy. Další nárůst produkce veřejných nápisů je pozorovatelný ve 2. st. n. l., kdy dochází k nárůstu regulace v souvislosti s administrací římské provincie a stavebních aktivit financovaných státním aparátem. Tento růst dále pokračuje i v 3. st. n. l., kdy se římská říše dlouhodobě potýká s krizí, nárůstem byrokratického aparátu\index{administrativa}, zvyšování moci armády\index{římská armáda} a centralizací\index{centralizace} řízení rozsáhlé říše. Rostoucí komplexita politické a společenské organizace vede k nutnosti efektivněji uchovávat a předávat informace, nutné k řízení stratifikované společnosti o velikosti větší než několik set členů. Proto v komplexních společnostech dochází k vytvoření systému uchovávání a předávání informací, jehož částečným projevem jsou i dochované veřejné nápisy (Johnson 1973, 3-4). V 2. a 1. st. př. n. l. je naopak možné pozorovat snížení celkového počtu nápisů, které je nejmarkantnější ve 4. st. n. l., kdy dochází k destabilizaci politické autority, která využívala nápisy pro účely organizace a správy území a obyvatelstva. Probíhající reformy organizace říše, ekonomická a politická nestabilita vedou ve 4. st. n. l. ke konečnému úpadku byrokratického aparátu v jeho původní podobě, a tedy i k razantnímu opuštění epigrafické produkce.

\subsection[soukromé-vs.-veřejné-nápisy-a-role-komplexní-společnosti]{Soukromé vs. veřejné nápisy a role komplexní společnosti}

Nápisy soukromé povahy tvořily většinu nápisů po celou sledovanou dobu a do jisté míry u nich probíhaly podobné změny růstu a poklesu produkce jako u veřejných nápisů, jak je patrné z \in{Grafu}[Apendix2:::6.13a] v \in{Apendixu}[Apendix2:::Apendix2]. Tento jev souvisí s propojeností základní infrastruktury nutné k vytvoření nápisů, kterou sdílí jak nápisy veřejné, tak i soukromé povahy. V dobách prosperity, kdy dochází k nárůstu společenské komplexity, současně se totiž objevuje i infrastruktura nutná ke zhotovení nápisů, jakou je profesní specializace, intenzifikace výroby, navýšení gramotnosti obyvatelstva a akumulace ekonomického potenciálu osobami podílejícími se na produkci nápisů. Pokud jsou tyto podmínky naplněny, tím pádem může snadněji dojít k většímu zapojení obyvatelstva na produkci nápisů, a tedy i nárůstu produkce soukromých nápisů.

Prudké nárůsty produkce soukromých nápisů se objevují v 5. st. př. n. l., dále ve 2. st. n. l., tedy v době prosperity a stability, kdy soukromé osoby mají dostatek prostředků na zhotovení nápisu. Naopak ve 3. st. př. n. l., 1. st. př. n. l. a zejména ve 4. st. n. l. dochází k úpadku publikační činnosti soukromých osob. Tento jev je možné spojovat s obdobím společenské a ekonomické nejistoty, kdy lidé upouští od publikování nápisů a dávají přednost zajištění primárních životních potřeb. V těchto dobách také může klesat počet obyvatel schopných psát, spolu s tím, jak v dobách krize mizí i nutná infrastruktura potřebná pro vznik gramotné vrstvy obyvatel, případně pro vytváření nápisů samých (Tainter 1988, 118-123, 137).

Zajímavým jevem je současný nárůst veřejných nápisů a současný pokles nápisů funerálních, ke kterému došlo ve 3. st. př. n. l. a také ve 3. st. n. l., jak je patrné v \in{Grafu}[Apendix2:::6.13a] v \in{Apendixu}[Apendix2:::Apendix2]. Možné vysvětlení tohoto trendu je krize tehdejšího politického a společenského uspořádání a snaha politické autority o reformy, navýšení regulace a kontroly byrokratického aparátu. Přímým důsledkem je zvýšené finanční zatížení středních vrstev společnosti, což v konečném důsledku vede k relativně okamžitému poklesu produkce soukromých nápisů. Ve 3. st. př. n. l. tento jev může souviset se snahou původně autonomních řeckých měst vyrovnat se s působením makedonských králů v Thrákii, které je možné datovat již od poloviny 4. st. n. l. s výraznějšími projevy zejména ve století následujícím. Ve 3. st. n. l. je to naopak reakce na krizi římské říše, způsobenou jak vojenskými převraty a konflikty, nebezpečím na vnějších hranicích říše, finanční zátěží a přebujelým administrativním aparátem (Tainter 1988, 137-140). Administrativní aparát se snaží udržet v chodu i za cenu zvyšujících se nákladů, které dopadají především na střední vrstvu. To může vést až ke snižování vzdělanosti a schopnosti obyvatelstva publikovat a financovat zhotovování nápisů, což má za následek úbytek soukromých nápisů ve 3. st. n. l. O století později tento trend vyústí v konečný úpadek epigrafické produkce ve veřejné sféře a přeměnu publikačních zvyků v soukromé sféře, vedoucí k přežívání epigrafické funerální produkce v křesťanských komunitách.

Trendy v chování vyšší a střední vrstvy, tedy těch lidí, kteří si mohli dovolit publikovat nápisy, dobře dokumentují trendy zobrazené v \in{Grafu}[Apendix2:::6.14a] v \in{Apendixu}[Apendix2:::Apendix2]. Po většinu doby převládaly funerální nápisy, jejichž funkcí bylo označovat hrob a připomínat zemřelého. Jejich celkové počty reagovaly na aktuální změny ve společnosti, tj. pokles v době krize a nárůst v době stability. Změna nastává na konci 1. st. n. l., kdy se Thrákie stává součástí římské říše a objevuje se prudký nárůst dedikačních nápisů, doprovázený pomalým poklesem funerálních nápisů. Ve 3. st. n. l. jsou dedikační nápisy dokonce četnější než nápisy funerální, což svědčí o proměnách chování tehdejší populace. S tím souvisí i nárůst epigrafické produkce související s lokálními kulty, které vykazují prvky jak místní thrácké víry, tak řeckého náboženství. Došlo tak k unikátnímu spojení a přeměně zvyků tehdejší společnosti, kdy se funkce převážné části epigrafické produkce změnila z připomínání zemřelého směrem k vyjádření náboženského přesvědčení. Ve 4. st. n. l. dochází k všeobecnému úpadku a navrácení se k funerální funkci dochovaných nápisů.

\subsection[epigrafická-produkce-v-jednotlivých-obdobích]{Epigrafická produkce v jednotlivých obdobích}

V zásadě je možné epigrafickou produkci rozdělit podle její celkové míry a rozsahu, případně charakteru materiálu do několika období, v nichž dochází k zásadnějším proměnám společnosti, způsobených jak změnami vnitřního uspořádání společnosti, tak i ovlivněním nově příchozími kulturami.

Období 6. a 5. st. př. n. l. je charakterizováno soustředěním epigrafické produkce v řeckých městech na pobřeží a pouze sporadickým užitím písma v thráckém aristokratickém kontextu ve vnitrozemí\index{řecká kolonizace}. Použití nápisů v thráckém prostředí je odlišné od použití v řecky mluvících komunitách na pobřeží a nic nesvědčí o přenosu kulturních zvyklostí v počátečních staletích. Ke kontaktu dochází v omezené míře na diplomatické úrovni v řadách aristokratů a dále na úrovni obyvatelstva žijícího v bezprostřední blízkosti řeckých měst, avšak ani zde nedochází k významnějšímu přenosu kultury a zvyklostí. Výskyt thráckých osobních jmen v řeckém kontextu poukazuje na existenci smíšených svazků, ale vzhledem k tomu, že se nejedná se více než o 3 \letterpercent{} z celkového počtu dochovaných osobních jmen, pak i smíšené svazky byly záležitostí spíše výjimečnou.\footnote{Dle studia osobních jmen se thrácká populace v předřímské době podílela na epigrafické produkci jen minimální měrou, a to zejména v bezprostředním okolí řeckých osídlení. Loukopoulou (1989, 185-217) udává, že v oblasti Propontidy se thrácká jména na nápisech z archaické a klasické doby takřka nevyskytovala. V době hellénismu a v době římské celkové zastoupení thráckých jmen na nápisech nepřesahovalo 3 - 6 \letterpercent{} dle konkrétního regionu, s tím, že nejvíce jmen thráckého původu bylo na nápisech z okolí Byzantia.} Tuto dobu je možné ohraničit objevením prvních epigrafických památek v 6. st. př. n. l. a polovinou 4. st. př. n. l., kdy dochází k nárůstu moci thrácké aristokracie spolu s výraznějším vstupem Makedonie na scénu. To sebou nese proměnu poměru soukromých a veřejných nápisů v rámci celkové produkce s až třetinovým nárůstem počtu veřejných nápisů, což se projeví zejména později ve 3. st. př. n. l., jak dobře ilustruje \in{Graf}[Apendix2:::6.13a] v \in{Apendixu}[Apendix2:::Apendix2].

Ve 4. st. př. n. l. pokračuje epigrafická produkce v řeckých městech na pobřeží beze výraznějších změn, pouze dochází k jejímu rozšíření do více míst a zvyšuje se i celková produkce. Převahu mají nápisy soukromé povahy, ale objevují se i veřejné nápisy, jejichž charakter je do velké míry ovlivněn tradicí a normou obvyklou pro daný typ nápisu. Epigrafická produkce se v omezené míře objevuje i v nově vzniklých smíšených makedonsko-thráckých osídleních\index{makedonská kolonizace} ve vnitrozemí, nicméně ani zde nedochází k prolínání onomastických zvyklostí či náboženských představ a obě komunity si udržují spíše tradiční charakter, soudě dle epigrafické produkce.

Podobný trend nárůstu celkové epigrafické produkce pokračuje i ve 3. a 2. st. př. n. l., kdy však dochází k poklesu produkce soukromých nápisů v řeckých městech na pobřeží ve 3. st. př. n. l., ale zároveň dochází k nárůstu produkce veřejných nápisů, což může svědčit o nárůstu institucionální regulace. Pobřežní komunity se taktéž v této době více otevírají multinárodnostnímu hellénistickému společenství, což vede k objevení neřeckých náboženství a osob s neřeckými jmény. Zároveň se v řeckých městech objevují i božstva thráckého původu, avšak podíl thráckých jmen se stále pohybuje pod úrovní 3 \letterpercent{} obyvatelstva. Thrácká aristokracie se na epigrafické produkci podílí zcela minimálně, a to zejména v okolí smíšených (řecko-)makedonsko-thráckých osídlení.

Skupina nápisů z 1. st. př. n. l. a 1. st. n. l. poukazuje na úpadek epigrafické produkce jako důsledek společenskopolitické krize a nestálosti poměrů v oblasti. V této době začíná do politické situace silně zasahovat Řím, ať už přímo, či pomocí nepřímé intervence. Od poloviny 1. st. n. l. je oblast oficiálně připojena pod římskou říši jako provincie {\em Thracia} a {\em Moesia Inferior}, avšak nedochází k významnému kulturnímu předělu či k zásadním změnám v epigrafické produkci.

Významnější změny naopak přináší 2. st. n. l., kdy dochází k strmému nárůstu jak soukromé, tak veřejné epigrafické produkce. Reformy administrativního uspořádání provincie, nárůst byrokratické zátěže a centrální organizace má za následek zintenzivnění epigrafické produkce, objevení se nových institucí a specializovaných profesí. Zapojení thráckého obyvatelstva v armádě\index{římská armáda} a městské samosprávě s sebou nese i nárůst gramotnosti, povědomosti o epigrafických zvyklostech, a tedy i nárůst celkového počtu soukromých nápisů.

Nápisy z 2., ale i z 3. st. n. l. nabízejí větší různorodost obsahu nápisů, která je odrazem smíšeného složení epigraficky aktivní společnosti. Zhruba jedna pětina epigraficky aktivní populace jsou osoby nesoucí thrácké jméno, kteří však v mnoha případech přijali i jména římská pravděpodobně jako důkaz dosaženého společenského postavení a vykonaných skutků. Spolu s rozšířením epigrafické produkce mezi thrácké obyvatelstvo se objevují ve větší míře i thrácké kulty, které se smísily s řeckými, ale i dalšími středomořskými kulty. Přítomnost lidí z jiných částí římské říše poukazuje na zvýšený pohyb obyvatelstva, a to zejména v případě lidí pocházejících z Malé Asie \index{Bíthýnie} a usídlených v oblasti velkých měst jako je Filippopole a Serdica (Slawisch 2007, 171). Zvýšená mobilita osob je typická pro multikulturní prostředí římské říše a projevuje se nejen v oblasti Thrákie (Carroll 2006, 281). Dedikační nápisy se ve 2. a 3. st. n. l. stanou vůbec nejhojnější skupinou nápisů, což svědčí o zařazení epigrafické produkce k náboženským zvyklostem. Dochází i k rozšíření nových zvyklostí, jako je udávání věku na funerálních nápisech či pravidelné uvádění skutků a společenského postavení, stejně tak jako nejbližší rodiny a přátel, kteří si tak pravděpodobně zajišťovali dědická práva (Carroll 2006, 281). V závislosti na společenském uspořádání římské říše se proměňují i onomastické zvyklosti obyvatel, ač tyto trendy byly omezeně pozorovatelné už i v druhé polovině 1. st. n. l. Uplatňuje se systém tří jmen\index{tria nomina}, z nichž alespoň jedno jméno má římský původ a poukazuje tak na společenské postavení svého nositele. Tento trend je však narušen po roce 212 n. l., kdy právo nosit římské jméno má každý obyvatel římské říše a v průběhu času tudíž ztrácí na původní prestiži.

Zároveň s narůstající variabilitou obsahu se však ve 2. a 3. st. n. l. projevuje i relativní ustálení formy, a to jak ve vnější podobě epigrafických monumentů, tak i v případě opakujících se formulí a ustálených slovních spojení. V případě veřejných nápisů je vidět jasně centralizovaná role římské administrativy\index{administrativa}, která do jisté míry předepisuje výslednou podobu a obsah nápisů, avšak za ponechání prostoru pro vytvoření lokálních variant místními samosprávami. V případě soukromých nápisů se spíše než o vliv předepsaných norem jedná o věc osobního vkusu kombinovanou s nápodobou již existujících zvyklostí, která nepřímo vytváří unifikovanou podobu soukromých nápisů.

Následné 4. a 5. st. n. l. jsou znamením úpadku epigrafické produkce a přeměny struktury společnosti. Reformy organizace společnosti vedou k jiným formám uplatňování politické autority, v nichž vydávání nápisů nehraje žádnou roli a od tohoto zvyku se upouští. Nárůst významu křesťanství v životě soukromých osob je možné pozorovat téměř na všech dochovaných nápisech soukromého charakteru. Z nápisů zcela mizí náboženství jiného typu než křesťanství\index{křesťanství} a stírají se jak etnické rozdíly, tak i zaniká původní hierarchie společnosti a vytváří se nová struktura, provázaná úzce s křesťanskou církví. V následujících stoletích je možné spatřovat přežívající pozůstatky epigrafické produkce, ale ve velmi omezeném množství a pouze v soukromé sféře.

\stopcomponent