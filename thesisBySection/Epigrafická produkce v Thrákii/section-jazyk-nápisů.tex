
\environment ../env_dis
\startcomponent section-jazyk-nápisů
\section[jazyk-nápisů]{Jazyk nápisů}

Analyzovaný soubor nápisů obsahuje převážně řecky psané nápisy, které tvoří až 98 \letterpercent{} všech nápisů. Databáze také obsahuje nápisy nesoucí zároveň řecký a latinský text (1 \letterpercent{}), a v neposlední řadě nápisy nesoucí pouze latinsky psaný text (1 \letterpercent{}).\footnote{Takto nízké číslo latinských nápisů nereflektuje jejich skutečný stav dochování, ale je výsledkem omezení daného výzkumu a zaměření práce pouze na řecky psané texty. V budoucnosti je možné práci rozšířit i o latinsky psané texty, což v současné době přesahuje možnosti a omezení dané doktorským projektem. Doplnění současného projektu o latinsky psané texty by vyžadovalo vytvoření badatelského týmu a jednalo by se o projekt dlouhodobého charakteru, jehož přínos v rámci současného stavu poznání by byl zcela jistě neoddiskutovatelný.} Nápisy, které bývají někdy označovány jako thrácké, neřadím do samostatné kategorie, vzhledem k jejich problematické povaze, ale zahrnuji je do nápisů řeckých s patřičným komentářem (Dimitrov 2009, 3-19; Dana 2015, 244-245).\footnote{Badatelé se nemohou shodnout, zda se skutečně jedná o thrácké nápisy, či pouze o nesrozumitelně psané nápisy řecké (např. Dimitrov 2009; Dana 2015). Jedná se o několik (méně než 10, konkrétní číslo je záležitostí definice „thráckého” nápisu jednotlivých badatelů) velmi krátkých textů psaných alfabétou, sestávajících z jednoho, či několika málo slov. Význam těchto nápisů zůstává nejasný, a tedy i jejich interpretační hodnota je značně omezená. I přesto jsem se rozhodla je zařadit do kategorie řeckých nápisů, avšak brát v potaz jejich speciální charakter.}

Velká část nápisů z Thrákie je psána v řečtině, a to i v době římské nadvlády. Převaha epigrafických textů v řečtině zcela odpovídá vývoji i v jiných částech Balkánského poloostrova a východních částí římského impéria obecně. Řečtina se stala oficiálním publikačním jazykem poměrně záhy po objevení prvních řeckých osídlení na thráckém pobřeží. První nápisy pocházely z čistě řeckého kontextu, nicméně v průběhu staletí se i thrácké obyvatelstvo naučilo používat alfabétu a později řečtinu jako prostředek písemné komunikace. V římské době nedošlo k zásadnímu kulturnímu přelomu v užívání publikačního jazyka a řečtina zůstala preferovaným jazykem epigrafických památek, alespoň na území provincie {\em Thracia}. Na hranicích mezi provincií {\em Thracia} a {\em Moesia Inferior} se v římské době setkávaly dvě jazykové tradice, latinská a řecká, které zásadně ovlivnily jazyk publikovaných nápisů. Přes území Balkánu vedla v době římské říše tzv. Jirečkova linie, která od sebe oddělovala území, kde byla valná většina textů psána řecky, od území s převahou latinsky psaných epigrafických památek (Jireček 1911, 36-39). Jirečkova linie procházela přibližně na místě dnešního pohoří Stara Planina a pomyslně oddělovala řecky píšící jih od latinsky píšícího severu. Přesné statistiky řecky vs. latinsky psaných textů je velmi obtížné uvést, vzhledem k nedostupnosti jednotného zdroje nápisů.\footnote{V roce 2006 začal vznikat digitální korpus nápisů z Bulharska pod patronátem Univerzity {\em Sv. Kliment Ohridski} v Sofii, který měl obsahovat zhruba 3500 řecky psaných nápisů. Bohužel i v roce 2017 je tento projekt stále ve fázi vývoje a oficiální webová stránka je veřejnosti nedostupná, http://telamon.proclassics.org/index.php (navštíveno 7. března 2017). Databáze {\em Packard Humanities Institute} s názvem {\em Searchable Greek Inscriptions} (PHI) obsahuje pouze řecky psané nápisy z území Thrákie v počtu okolo 4000 exemplářů, avšak s metadaty omezenými na dataci, jméno nálezové lokality a text nápisu. K poslední úpravě databáze došlo v září 2015, což znamená, že neobsahuje nové nápisy (\useURL[url19][http://inscriptions.packhum.org/][][{\em http://inscriptions.packhum.org/}]\from[url19], navštíveno 5. března 2017). Databáze {\em Epigraphic Database Heidelberg} (EDH) obsahuje jak latinské, tak řecké nápisy, nicméně není ani zdaleka kompletní. Ač dochází k neustálému doplňování nápisů, 5. března 2017 databáze obsahovala 397 nápisů z provincie {\em Thracia} a 1938 z provincie {\em Moesia Inferior}. (\useURL[url20][http://edh-www.adw.uni-heidelberg.de/home/][][{\em http://edh-www.adw.uni-heidelberg.de/home/}]\from[url20], navštíveno 5. března 2017). Databáze latinských nápisů {\em Corpus Inscriptionum Latinarum} (CIL) obsahuje k 5. březnu 2017 79 latinsky psaných nápisů z provincie {\em Thracia} a 387 z provincie {\em Moesia Inferior} (\useURL[url21][http://cil.bbaw.de/cil_en/dateien/datenbank_eng.php][][{\em http://cil.bbaw.de/cil_en/dateien/datenbank_eng.php}]\from[url21], navštíveno 5. března 2017).} Obecně se předpokládá, že tento poměr je pro Thrákii kolem 80:20 ve prospěch řecky psaných nápisů.\footnote{Přesné číslo latinsky psaných nápisů pocházejících z území Thrákie není známé. Jednotlivé nápisy jsou publikovány v několika zdrojích, a pravděpodobně některé nápisy zůstávají nepublikované. Odhady badatelů se tak do velké míry různí: Milena Minkova (2000, 1-7) poskytuje statistiku pouze pro území moderního Bulharska, kde předpokládá existenci zhruba 1200-1300 latinsky psaných nápisů. Nicolay Sharankov (2011, 145) uvádí, že řeckých nápisů je zhruba 20krát více než latinských, avšak tento odhad je pravděpodobně příliš nízký. Pokud vezmeme 4600 řeckých nápisů z {\em Hellenization of Ancient Thrace} databáze jako výchozí číslo, dle Sharankova by latinských nápisů bylo pouze 230. Pouze z okolí samotného města Novae pochází 447 latinsky psaných nápisů (Gerov 1989, 207).} V okolí větších měst a sídel vojenských jednotek je přítomnost latinsky psaných textů vyšší vzhledem k přítomnosti byrokratického a vojenského aparátu než na thráckém venkově, kde převládají řecky psané texty. V případě Thrákie se však latinský a řecký svět do značné míry prolínaly a docházelo k vzájemnému ovlivňování a lingvistickým výpůjčkám (Dana 2015, 253). \footnote{Agniezka Tomas (2007, 31-47; 2016) zpracovala výskyt řeckých a latinských nápisů v okolí města Novae na Dunaji v provincii {\em Moesia Inferior} na pomezí tzv. Jirečkovy linie. Novae bylo sídlem římské legie již od roku 45 n. l. a vzhledem k trvalé přítomnosti římského vojska se dá předpokládat převaha latinských nápisů nad nápisy psanými řecky. To platí pro bezprostřední okolí Novae a pro sídla ve vnitrozemí, která měla na starosti zásobování města a vojenských jednotek, jako např. {\em villa} v Pavlikeni. Ve venkovských sídlech se však objevují i nečetné řecky psané nápisy, a to zejména ve svatyních poblíž vesnic Paskalevec, Butovo a Obedinenie (Tomas 2007, 44). V okolí města Nicopolis ad Istrum, které se nachází zhruba 80 km jihovýchodně, je již převaha dochovaných nápisů psaná řecky, a to jak z města, tak z venkovských oblastí.}

\subsection[identita-a-volba-jazyka]{Identita a volba jazyka}

Volba řečtiny jako publikačního jazyka nápisů byla do nedávné doby brána jako jeden ze znaků hellénizace thráckého obyvatelstva (Sharankov 2011, 139-141). Podobně pak volba latiny jako publikačního jazyka byla brána jako znak romanizace, která však na území Thrákie neproběhla v tak úspěšné míře, jako hellénizace, a to zejména vzhledem k celkovému většímu počtu dochovaných řecky psaných nápisů (Tomas 2016, 119-120). Musíme vzít v potaz, že místní thrácké obyvatelstvo bylo před příchodem řeckých osadníků na území Thrákii negramotné, a s největší pravděpodobností byla řecká alfabéta prvním písemný systémem, s nímž se setkalo a jemuž bylo vystaveno v dlouhodobém časovém horizontu. Převzetí písemného systému pak může v tomto případě představovat způsob komunikační strategie, a tedy snahu o nalezení společného dorozumívacího systému. Tzv. „thrácké” nápisy, tedy nápisy psané alfabétou, které dosud nebyly přesvědčivě interpretovány, mohou představovat mezistupeň vývoje, kdy Thrákové používali písmo ve velmi omezené podobě, avšak kontext použití byl specifický pro thrácké prostředí (Dana 2015, 244-245).

Hellénizace jakožto přijetí řečtiny jako jazyka písemné komunikace s sebou nese celou řadu problémů. Jedním z nich je fakt, že řečtinu není možné brát jako jednolitý jazyk, ale spíše jako komplex příbuzných dialektů, vycházejících, ze společného základu. To alespoň platí v archaické a klasické době, kdy se na nápisech objevují charakteristické prvky náležící např. dórskému či ionsko-attickému dialektu.\footnote{Podrobněji v kapitole 6 sekce věnované 6. až 4. st. př. n. l.} Pokud se na nápisech dochovaly konkrétní dialektální zvláštnosti, pak nápisy s největší pravděpodobností pocházely z řecké komunity, která tak chtěla poukázat na své vazby s mateřskou obcí, jak dosvědčují mnohé příklady z thráckého pobřeží. Celkem bylo v Thrákii zaznamenáno 97 nápisů nesoucích charakteristické znaky dórského dialektu, které pocházely převážně z řeckých kolonií založených dórskou Megarou jako je Mesámbria s 52 nápisy, Byzantion s 34 nápisy, a Sélymbria se 7 nápisy. Z dalších nedórských lokalit pocházejí pouze náhodné nálezy vždy po jednom nápise v Apollónii Pontské, Marcianopoli, Odéssu a lokalitě Karon Limen.\footnote{Lokalita Karon Limen, též známá jako Caron Limen či Portus Caria.} Většina (86,46 \letterpercent{}) nápisů psaných dórským dialektem pochází z 5. - 1. st. př. n. l., a největší část z nich pochází z 3. st. př. n. l. z černomořské Mesámbrie.\footnote{Z Mesámbrie 3. st. př. n. l. pochází dvojnásobně velký počet veřejných nápisů oproti ostatním obcím, a tyto texty jsou navíc psány dórským dialektem. Převážně se jedná o dekrety vydávané lidem Mesámbrie ({\em búlé} a {\em démos}), čili o nařízení regulující společenskou organizaci, dále o honorifikační dekrety a o seznamy občanů. Více v kapitole 6 v sekcích věnovaných 3. st. př. n. l.} Texty dialektálních nápisů si uchovávají velmi konzervativní charakter, a to nejen volbou dialektu mateřského města, ale i svou formou. Na těchto nápisech se objevuje velký počet standardních epigrafických formulí, stejně tak jako referencí tradičních institucí řeckých {\em poleis}, regionálních řeckých božstev. Co se týče osobních jmen a vyjádření identity, nápisy pocházejí převážně z čistě řeckého prostředí (68 nápisů, 70 \letterpercent{}), thrácká jména se vyskytují pouze na 4 nápisech (4,12 \letterpercent{}), a téměř vždy v kombinaci s řeckým jménem. Volba dórského dialektu na nápisech tak poukazuje na kontinuitu společensko-kulturních vazeb a norem i několik století po prvotním založení řeckých kolonií.

\subsection[postavení-řečtiny-v-římské-době]{Postavení řečtiny v římské době}

V případě římské doby konkrétní volba jazyka představovala volbu mezi tradičním jazykem, který se v oblasti používal jako jazyk epigrafické publikace již několik století, či mezi oficiálním jazykem římského impéria, jeho byrokratického a vojenského aparátu (Zgusta 1980, 135-137; Gerov 1980, 155-164). Volbou publikačního jazyka zhotovitel situoval sdělení daného nápisu do konkrétního kulturního prostředí, které mělo předem stanovená pravidla a očekávání. Volba jazyka v sobě nesla očekávání čtenářské obce, konkrétní komunity, jíž bylo sdělení určeno. Obsah nápisu naznačuje, že se jednalo spíše o vědomou volbu než o projev nekritického přijímání určitého jazyka, což v určitých podmínkách naznačují právě hellénizační či romanizační teorie (Sharankov 2011, 139-141; Tomas 2016, 119-120).

Práce, která by srovnávala řecky a latinsky psanou epigrafickou produkci v jejich kompletním rozsahu, by jistě byla velmi záslužná, nicméně natolik časově náročná vzhledem k současnému stavu znalostí, že se svou povahou hodí spíše pro výzkumný tým. Proto v současné práci vycházím z několika místních studií, které však vykazují podobné rysy. Srovnání řeckých a latinských nápisů ve vztahu k demografii a funkci nápisu v několika regionech Thrákie ukazují, že nápisy, v nichž se vyskytuje latina, pocházejí z prostředí vojenské a politické samosprávy, úzce související s aktivitami římského impéria. Řečtina naopak převládá ve venkovských oblastech a zejména ve funkci dedikačních nápisů, ale setkáme se s ní i v rámci veřejných nápisů (Gerov 1980, 158; Sharankov 2011, 145; Tomas 2007, 44-45; Janouchová 2017, v tisku).

Zhruba 1 \letterpercent{} (40) všech analyzovaných nápisů v databázi jsou nápisy obsahující řecký a latinský text. Ve více jak polovině textů (24) se jedná o identický text v řečtině a v latině. V těchto případech nelze říci, že by jeden z jazyků měl přednost: oba dva figurují na pozici prvního uvedeného textu rovnoměrně, pravděpodobně dle preferencí zhotovitele a očekávané komunity čtenářů. V případě nápisů s odlišnými latinskými a řeckými texty (16) vidíme jasné rozdělení na komunity, jimž byl text primárně určen, případně z jaké komunity pocházel zhotovitel. Nápisy určené latinsky mluvící komunitě jsou psány primárně latinsky se sekundárním textem řecky. Sekundární text je jednak standardní epigrafická formule, jako např. invokační formule vzývající božstvo, či se může jednat o podpis zhotovitele předmětu nesoucí nápis. V neposlední řadě do této kategorie patří i dedikace římskému císaři psané latinsky doplněné označením místní komunity psaným taktéž řecky. Tyto sekundární texty poukazují na společensko-kulturní pozadí, které bylo pro zhotovitele obvyklé a cítil potřebu se na něj odkázat i v rámci nápisu, který byl jinak určen primárně latinské komunitě.\footnote{Např. {\em IG Bulg} 2 749; {\em Perinthos-Herakleia} 292.} Zhotovitelé nápisů jsou tak důkazem, že bylo možné přestupovat mezi jednotlivými jazykovými komunitami a panovala mezi nimi vzájemná kulturní tolerance. Malý počet těchto nápisů v celkovém korpusu však poukazuje na fakt, že většina epigraficky aktivní populace si zvolila jeden či druhý epigrafický jazyk. Dle dostupných dat se na většině území Thrákie jednalo o řečtinu, a to i v době, kdy se území stalo součástí římského impéria.

\subsection[styl-jazyka]{Styl jazyka}

Kulturní a politické proměny tehdejší společnosti významnou měrou neovlivnily jazykovou formu a styl použitého jazyka. Zdá se, že styl epigrafického jazyka zachovával poměrně konzervativní podobu a vyplýval spíše ze společenské funkce nápisu, než že by podléhal aktuálním trendům (Petzl 2012, 52-58).

Celkem 121 textů nápisů, či alespoň jejich část, byla psána metricky. Jednalo se výhradně o nápisy soukromého charakteru: přes tři čtvrtiny, celkem 94 nápisů, mělo funkci funerálních nápisů. Dedikační nápisy se vyskytovaly v 17 případech, což představovalo zhruba 14 \letterpercent{}, zbytek nápisů nebylo možné určit. Poměr metricky psaných nápisů, u nichž byla známá datace, vůči celkovému počtu datovaných nápisů mírně narůstal od 1. st. př. n. l., a to zhruba o 0,5 \letterpercent{}.\footnote{Celkový počet datovaných nápisů ze 7. až 1. století př. n. l. (975,58; normalizovaná datace, více o metodě normalizované datace v kapitole 4) představuje 27,47 \letterpercent{} všech nápisů. Celkem 19 nápisů psaných metricky (normalizovaná datace) z téhož období, představuje 1,95 \letterpercent{} ze všech datovaných nápisů. U nápisů datovaných do 1. až 8. st. n. l. se jedná celkem o 2575,09 nápisů (normalizovaná datace), což představuje 72,51 \letterpercent{} všech datovaných nápisů. U nápisů psaných metricky a datovaných do 1. až 8. st. n. l. se jedná o 62 nápisů (normalizovaná datace), což představuje 2,41 \letterpercent{} všech datovaných nápisů.} Tento nárůst je statisticky nesignifikantní, a mohl být způsoben stavem a nahodilostí dochování nápisů. Celkově je tedy možné říci, že tendence publikovat metrické nápisy zůstává konstantní po celou dobu epigrafické produkce v Thrákii a nemění se v závislosti na proměnách složení populace, či v reakci na politicko-kulturní události. Metricky psané nápisy tvoří v průměru 2,2 \letterpercent{} celkové produkce, tedy její marginální část, která tvořila velmi konzervativní a neměnnou součást epigrafické produkce.

\stopcomponent