
\environment ../env_dis
\startcomponent section-onomastické-zvyklosti-a-identita-osobního-jména
\section[onomastické-zvyklosti-a-identita-osobního-jména]{Onomastické zvyklosti a identita osobního jména}

Osobní jména, která se vyskytovala na nápisech, odhalují, že epigrafická produkce byla zaměřená především na mužskou část populace. Mužská individuální jména, tedy jména osoby, která vystupovala jako primární {\em agens} nápisu, ať už zemřelý, dedikant či honorovaný člověk, tvořila 81,04 \letterpercent{} všech jmen, což zcela odpovídá zastoupení jednotlivých pohlaví na nápisech z dalších částí řeckého světa (Parissaki 2007, 267). V rámci onomastických zvyků se primární identita\index{identita} člověka skládala z několika individuálních jmen, dále ze jmen rodičů, případně prarodičů a partnerů. V rolích rodičů se z 92,32 \letterpercent{} vyskytovala mužská jména. Stejně tak v rolích prarodičů figurovala v 91,53 \letterpercent{} mužská jména. Tento fakt poukazuje na silnou tradici identifikace pomocí {\em patronymik} nebo {\em paponymik} v rámci epigraficky aktivní společnosti Thrákie. Co se týče jmen partnerů, pak se obě pohlaví vyskytují přibližně stejně často (42,98 \letterpercent{} oproti. 52,05 \letterpercent{}). Ženská jména se celkově vyskytovala na 12,26 \letterpercent{} nápisů a jako primární {\em agens} nápisu figurovala ve 13,84 \letterpercent{}, podrobněji v \in{Tabulce}[Apendix1:::5_05] v \in{Apendixu}[Apendix1:::Apendix1].

\subsection[onomastické-tradice]{Onomastické tradice}

\index{onomastické tradice}Podíváme-li se na tradiční kulturněspolečenský původ jmen, celková čísla ukazují, že většina jmen, která se na nápisech vyskytují, pochází z řeckého kulturního prostředí (41,36 \letterpercent{}), následují římská jména (35,79 \letterpercent{}) a thrácká jména (13,57 \letterpercent{}). \in{Tabulka}[Apendix1:::5_06] v \in{Apendixu}[Apendix1:::Apendix1] poskytuje souhrnné počty jmen dle jejich původu ze všech časových období.\footnote{V \in{kapitole}[chapter-Epigrafická-produkce-napříč-staletími:::chapter-Epigrafická-produkce-napříč-staletími] je rámci jednotlivých století vždy jednotlivě pojednáno o měnících se poměrech původu jmen a případných demografických změnách epigraficky aktivní populace.} Pokud obě dvě statistiky srovnáme dohromady, můžeme jasně vidět rozdíly mezi onomastickými zvyky jednotlivým kulturních prostředí. Jak je patrné z \in{Tabulky}[Apendix1:::5_07] v \in{Apendixu}[Apendix1:::Apendix1], u řeckých jmen je to silná tradice uvádění {\em patronymik} (37,78 \letterpercent{}) a {\em paponymik} (0,61 \letterpercent{}) oproti římskému kulturnímu prostředí. V římském prostředí se užívání kombinace tří individuálních jmen pro jednotlivce ({\em tria nomina}\index{tria nomina}) odráží ve vysokém počtu individuálních jmen (90,07 \letterpercent{}). Thrácké prostředí sleduje spíše řeckou onomastickou tradici, a to v užívání {\em patronymik} a {\em paponymik}. Z analýzy thráckých onomastických zvyků však víme, že dochází i k následování římské tradice, a to zejména v užívání kombinace několika individuálních jmen i v kombinaci se jménem římským, které se taktéž podílí na vysokém počtu římských individuálních jmen. Je tedy zřejmé, že osoby nesoucí thrácká osobní jména poměrně záhy přijaly zvyk uvádění osobního jména, doplněného jménem otce či prarodiče, a tento způsob identifikace přetrval i v době římské. Naopak, v době římské byl tento zvyk doplněn zvykem novým, a to přidáním původně římského jména či několika jmen k jménu původně thráckému, za nímž i nadále následovalo jméno otce či prarodiče. V thráckém prostředí se tak v této době setkáváme se s kombinací několika onomastických tradic a vytvářením jedinečného systému identifikace osob thráckého původu\index{thrácká populace}.

Právo nosit římská jména úzce souviselo s udělením římského občanství. V případě Thrákie se římské občanství udílelo v 1. st. n. l. převážně vysoce postaveným aristokratům, kteří zároveň hráli důležitou roli v místní samosprávě. V 2. a na počátku 3. st. n. l. se římské občanství udílelo za zásluhy a službu v římské armádě. Nositelé římského jména v tomto období pocházeli převážně ze středních společenských vrstev, kteří si na rozdíl od běžného obyvatelstva udržovali vyšší společenské postavení, nicméně se nejednalo výhradně o elity, jako v případě 1. st. n. l. (Camia 2013, 187-197). Ve 3. st. n. l. se právo nosit římské jméno se po roce 212 n. l. rozšířilo na všechny svobodné obyvatele římské říše a společenská prestiž a výsady spojené s římskými onomastickými zvyklostmi ztratili ve srovnání s předchozím obdobím své výsadní postavení (Beshevliev 1970, 31-32).

\subsection[zastoupení-pohlaví-v-závislosti-na-kulturním-prostředí]{Zastoupení pohlaví v závislosti na kulturním prostředí}

Pokud se podíváme na zastoupení pohlaví v jednotlivým kulturních prostředích, v řeckém prostředí je zastoupeno největší procento ženských jmen oproti mužským jménům (13,85 \letterpercent{}). Mužská jména však představují více jak 85 \letterpercent{} všech jmen v řeckém, římském i thráckém prostředí \index{onomastické tradice}(Parissaki 2007, 267).

Jak je patrné z dat v \in{Tabulce}[Apendix1:::5_08] a \in{Tabulce}[Apendix1:::5_09] v \in{Apendixu}[Apendix1:::Apendix1], ženská jména v řeckém prostředí figurují na místě rodičů častěji (1,67 \letterpercent{}) než v případě římského a thráckého prostředí. To poukazuje na větší důležitost mateřské linie v rámci řecké společnosti. Otcovská linie však i přesto bývá pro srovnání uváděna v rámci identifikace jednotlivce zhruba 20krát častěji, než je tomu v případě matky (36,26 \letterpercent{}). Naopak jako partnerská jména jsou v řeckém prostředí uváděna častěji mužská jména, než ženská (2,81 \letterpercent{} oproti 1,58 \letterpercent{}), což opět poukazuje na důležitější roli muže v rámci uspořádání řecké společnosti. Vysoký počet jmen rodičů poukazuje na fakt, že udávání {\em patronymika} bylo v řeckém prostředí velmi časté, vzhledem k tomu, že 78 \letterpercent{} všech individuálních jmen bylo doplněno právě jménem otce.

V římském prostředí roli při identifikaci jednotlivce hraje osobnost konkrétního člověka ({\em tria nomina}\index{tria nomina}) a linie předků nemá tak význačnou pozici jako v řeckém prostředí (0,41 \letterpercent{} pro ženy a 7,26 \letterpercent{} pro muže). Role partnera při identifikace jednotlivce je v římském prostředí taktéž nižší než v řeckém prostředí (1 \letterpercent{} pro ženy a 1,03 \letterpercent{} pro muže). Thrácké onomastické prostředí\index{onomastické tradice} do velké míry přejímá zvyky jak řeckého, tak římského prostředí: poměry užívání individuálních jmen a jmen předků při identifikaci osoby mají v řeckém a thráckém prostředí velmi podobné hodnoty - jedinci nesoucí thrácká jména udávají svůj původ pomocí jmen otců a prarodičů. Stejně tak ale jedinci identifikují sama sebe pomocí kombinace tří jmen, z nichž jedno, případně dvě jsou římského původu. Výjimku tvoří jména partnerů, kdy thrácká ženská jména jsou ze všech kulturních prostředí nejčastěji užívána v roli partnerek (2,38 \letterpercent{}), což může být důsledek smíšených partnerských svazků mezi Thrákyněmi a Řeky.

\subsection[smíšené-svazky-a-prolínání-tradic]{Smíšené svazky a prolínání tradic}

\index{smíšená manželství}Údaje o smíšených svazcích mezi Řeky a Thráky se na nápisech dochovaly celkem ve 43 případech, a to v rámci identifikace jednotlivce ve formě jména primární osoby a jména partnera. Nejčastější formou jsou svazky mezi mužem z řeckého prostředí a ženou z thráckého prostředí (33 výskytů, 76,7 \letterpercent{}). Celkem je možné rozlišit šest variant v závislosti na původu jména a pohlaví osoby vystupující na nápisech:

\startitemize[packed]
\item muž se jménem řeckého původu, partnerka se jménem thráckého původu (19 výskytů)
\item žena se jménem řeckého původu, partner se jménem thráckého původu (4 výskyty)
\item žena se jménem thráckého původu, partner se jménem řeckého původu (13 výskytů) 
\item muž nesoucí jméno thráckého a římského původu, partnerka se jménem řeckého původu (2 výskyty)
\item žena nesoucí jméno thráckého a římského původu, partner se jménem řeckého původu (1 výskyt).
\stopitemize

Smíšené svazky se vyskytují zejména na nápisech pocházejících z regionu městských center a z oblastí, kde se dá předpokládat zvýšená míra kontaktů mezi thráckou a řeckou, případně římskou populací, tj. v okolí {\em emporií} či vojenských lokalit. Největší koncentrace nápisů se smíšenými svazky pochází z Odéssu s 23 nápisy, převážně z římské doby. Dále se nápisy se smíšenými svazky v době římské objevovaly v údolí středního toku řeky Strýmónu (10) a v menší míře ve městech Byzantion (3), Nicopolis ad Nestum (2), Pautália (2), Augusta Traiana (2), Filippopolis (1) a Seuthopolis (1). Nápisy jsou většinou římské, až na tři hellénistické nápisy, které pocházejí ve dvou exemplářích z Odéssu a jeden ze Seuthopole.\footnote{Právě hellénistický nápis {\em IG Bulg} 3,2 1731 ze Seuthopole dokumentuje aristokratický svazek thráckého krále Seutha III. a pravděpodobně řecké či makedonské Bereníké a jedná se o první zmínku o existenci smíšených svazků thrácké aristokracie (Tacheva 2000, 30-35; Calder 1996, 172-173). Ostatní nápisy dokumentují svazky mimo aristokratické kruhy, nakolik je možné tak soudit z mnohdy lakonického textu.} Údaje o smíšených svazcích pochází převážně z funerálních nápisů, u nichž se zejména v římské době předpokládá uvádění rodinných příslušníků a vzájemných vztahů, mimo jiné i z dědických důvodů (Saller a Shaw 1984, 145; Meyer 1990, 95).

\stopcomponent