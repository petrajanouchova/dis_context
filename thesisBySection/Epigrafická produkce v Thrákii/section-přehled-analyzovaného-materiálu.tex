
\environment ../env_dis
\startcomponent section-přehled-analyzovaného-materiálu
\section[přehled-analyzovaného-materiálu]{Přehled analyzovaného materiálu}

Soubor nápisů přestavuje jedinečný komparativní soubor, reflektující společensko-kulturní změny a vývoj společnosti antické Thrákie. Databáze HAT tak krom informací o nápisech samých obsahuje i data o 5464 epigrafických osobách, které figurují v textech nápisů, jejichž jméno a identita se skládají z celkem 3956 osobních jmen a 178 termínů vyjádření kolektivní identity. Dále databáze obsahuje informace o 678 místech, kde byly nápisy nalezeny, ať už na území Thrákie či z nejbližšího okolí. V neposlední řadě se zde nacházejí data vztahující se k obsahu nápisů: celkem databáze zaznamenává 134 zeměpisných jmen, 233 epitet jednotlivých božstev, 137 termínů z náboženské oblasti, 112 termínů z oblasti organizace společnosti, 49 termínů vztahujících se k epigrafickým zvyklostem, a 37 termínů vztahující se k udílení poct v rámci epigrafické kultury.\footnote{Podrobný komentář a vysvětlení termínů se nachází v kapitole 4, věnované metodologii práce.}

\subsection[zeměpisné-uspořádání-nápisů]{Zeměpisné uspořádání nápisů}

Nápisy pocházejí z oblasti jihovýchodní části dnešního Balkánského poloostrova, a to konkrétně z území dnešních států Bulharska, Řecka a Turecka. Velká část nápisů byla nalezena, či editory zařazena do oblastí ve vzdálenosti do 20 km od mořského pobřeží, a dále z vnitrozemských městských center, která se nacházejí poblíž toků řek Hebros, Strýmón a Tonzos, případně v blízkosti hlavních cest v nížinatých oblastech. Z hornatých oblastí vnitrozemské Thrákie pochází jen menší část nápisů, a to zejména z oblasti pohoří Stara Planina, Rodopy, Pirin a Rila.\footnote{Více o zeměpisném rozložení nápisů hovořím v kapitole 7.}

Dochované nápisy jsou do značné míry výsledkem náhodných nálezů v průběhu posledních dvou století a systematického archeologického výzkumu zhruba od poloviny 20. století. Archeologické výzkumy neprobíhaly na všech místech stejnou měrou a ve stejném rozsahu, a proto i počty dochovaných nápisů mnohdy odpovídají stavu archeologického výzkumu na území Thrákie. Příkladem může být dobrý stav poznání v černomořských řeckých koloniích, které bylo možné prozkoumat velmi detailně v posledních 50 letech zejména díky prudkému nárůstu turistického ruchu a s ním spojených záchranných výzkumů (Velkov 1969; Ognenova-Marinova {\em et al.} 2005; Baralis a Panayotova 2015). Opačným příkladem je oblast turecké Thrákie, která je prozkoumána jen velmi málo. Vzhledem k složité politické a ekonomické situaci jsou data z části evropského Turecka nedostupná, až na oblast okolo Perinthu, Byzantia a řeckých měst na Thráckém Chersonésu, které se podařilo získat díky výzkumu a spolupráci německých a tureckých vědců (Krauss 1980; Lajtar 2000; Sayar 1998). Ucelená publikace nápisů z egejské Thrákie je taktéž poměrně nedávnou záležitostí \cite[Loukopoulou {\em et a}l.2005], avšak zde jsou nálezy nápisů doplněny dlouhodobými archeologickými výzkumy.

Mapa 5.01 v Apendixu 2 vyznačuje oranžovou barvou oblast, kterou pokrývají v databázi zpracované nápisy. Naopak místa bílá jsou zároveň místy, pro něž nejsou dostupné systematicky zpracované korpusy nápisů. Tato oblast bez nápisů se vyznačuje poměrně nepřístupným terénem a pravděpodobně byla málo zalidněná jak v antice, tak dnes. Obecně se předpokládá, že z této části Thrákie pochází jen velmi malý počet nápisů, které by tak výrazněji neměly zasáhnout do celkového obrazu epigraficky aktivní společnosti antické Thrákie, pokud dojde v budoucnu k jejich publikování.

Nápisy se podařilo lokalizovat dle dat dostupných v epigrafických korpusech, a to s následující přesností: do 1 km 1540 nápisů (33 \letterpercent{})\footnote{Následující statistiky vycházejí z celkového počtu 4665 nápisů, což představuje 100 \letterpercent{} analyzovaného souboru. V určitých případech může dojít při konečném součtu všech položek k hodnotě větší či menší než 100 \letterpercent{}. Pokud je součet větší než 100 \letterpercent{}, daná položka mohla mít více hodnot, tj. nápis mohl být určen například jako funerální a honorifikační zároveň, pokud svými charakteristickými rysy spadal do obou kategorií, V tomto případě pak konečný součet má hodnotu více jak 100 \letterpercent{}, protože položky obsahují více než jednu hodnotu. V případě statistik, kde je konečná hodnota menší než 100 \letterpercent{}, pak tento rozdíl náleží položkám, které se nepodařilo určit, jak z důvodu jejich fragmentárnosti, nedostupnosti informace apod.}, s přesností do 5 km 2323 nápisů (50 \letterpercent{}), s přesností do 20 km 590 nápisů (13 \letterpercent{}), a s přesností nad 20 km 212 nápisů (4 \letterpercent{}). Udávaná poloha většiny nápisů byla editory označena jako místo primárního nálezu (4305 nápisů, tedy 92 \letterpercent{}), tedy místo, kde byl nápis umístěn, když sloužil své primární funkci. Archeologický kontext místa, v němž byl nápis nalezen, udávají autoři korpusů zhruba u třetiny nápisů: funerální 251 nápisů, sídelní 222 nápisů, z čehož 53 bylo nalezeno v rámci obchodního kontextu daného osídlení, např. na agoře, na foru, v emporiu, v přístavu apod.; jiný kontext 16, rituální/náboženský kontext 779 nápisů. Pouze u 380 nápisů (8 \letterpercent{}) udávají místo nálezu jako sekundární kontext, což znamená, že nápis byl použit při stavbě mladší stavby, či autor korpusu uvádí, že byl nápis přesunut z původního místa, a přestal sloužit primární funkci.

\subsection[použitý-materiál-a-nosič-nápisu]{Použitý materiál a nosič nápisu}

Převážná většina nápisů byla zhotovena z kamene (4518 nápisů, což představuje 97 \letterpercent{}), a to především z mramoru (3292 nápisů), vápence (652 nápisů), syenitu (53 nápisů), pískovce (45 nápisů) a jiných, vesměs lokálních variant použitého kamene.\footnote{Agniezka Tomas (2016, 37-40) poukazuje na využití místního kamene na nápisech z okolí římského tábora v Novae. Převážně jsou využívány místní zdroje vápence a pískovce, pocházející jednak z bezprostředního okolí Novae a jednak z lokalit dostupných proti proudu řeky Jantra směrem na jih. Místní zdroj kamene byl tak využíván nejen pro epigrafické aktivity, ale zejména pro stavbu budov a jiných staveb nejen od druhé poloviny 1. st. n. l., jak naznačují dochované nápisy, ale pravděpodobněji již dříve. Tomas zaznamenává šest míst v regionu Novae, kde byly nápisy vyráběny ve větším měřítku, kde umísťuje i produkční dílny: Butovo, Novae, Dimum, Nicopolis ad Istrum, Karaisen a Pejčinovo - vše lokality zhruba v okruhu 40 km s výjimkou Nicopolis ad Istrum.} Z materiálů jiných než kámen se jednalo o 24 nápisů na kovu,\footnote{Jako je zlato, stříbro, olovo a editorem blíže neurčený kov.} dále 15 nápisů bylo na keramice, jeden nápis na dřevě, a šest na jiném druhu materiálu, jako jsou například mozaiky, či jako součást nástěnné malby. Téměř dvě třetiny všech objektů nesoucích nápis byly dekorované (2904 nápisů, 65,04 \letterpercent{}). Převládající typ dekorace byla reliéfní výzdoba (64,32 \letterpercent{} všech objektů nesoucích nápis, tedy 98,89 \letterpercent{} objektů s dekorací), nápisy s dochovanou malbou či jiným typem dekorace tvořily dohromady pouze 0,72 \letterpercent{} všech nápisů. Mezi nejčastější typy reliéfní výzdoby patřila výzdoba figurální (39,14 \letterpercent{} ze všech objektů nesoucích nápis), dále pak architektonická výzdoba či výzdoba ornamentální (23,88 \letterpercent{} ze všech objektů nesoucích nápis).\footnote{Vysvětlení jednotlivých pojmů a jejich obrazová galerie je součástí Apendixu 1. Jednotlivým druhům dekorace a jejich výskytu se věnuji podrobněji v následujících chronologicky řazených sekcích v této kapitole.} Nápisy tesané do kamene a jejich dekorace prochází v průběhu staletí vývojem směrem od prostých stél s jednoduchou florální či malovanou dekorací z klasické a hellénistické doby, až po složité reliéfní dekorace a nejrůznější motivy, nacházející se jak na stélách, tak na architektonických prvcích jako jsou sloupy, oltáře či dokonce sarkofágy v době římské.\footnote{Paradoxně k největšímu rozšíření reliéfních vyobrazení řeckých božstev a scén z řecké mytologie dochází až v době římské, nikoliv v době hellénismu, jak by se dalo očekávat.} Stupeň a způsob dochování objektů nesoucí nápis, jak je uvádějí autoři korpusu, či jak bylo patrné z přiložené vizuální dokumentace, jako jsou fotografie, kresby, oklepky apod. podrobně dokumentuje tabulka 5.01 v Apendixu 1. Je zřejmé, že výpovědní hodnota a přesnost interpretace je vyšší u nápisů, které se dochovaly v co možná nejkompletnější podobě. Bohužel u čtvrtiny nápisů nebylo možné určit stupeň dochování, a to zejména kvůli chybějící vizuální dokumentaci, a nejasnému charakteru textu. I přesto se však téměř 70 \letterpercent{} objektů nesoucí nápis dochovalo v takové podobě, že je možné z nich získat data potřebná k další analýze obsahu textu.

\stopcomponent