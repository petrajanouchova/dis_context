
\environment ../env_dis
\startcomponent section-rozsah-a-typologie-nápisů
\section[rozsah-a-typologie-nápisů]{Rozsah a typologie nápisů}

Dochovaná délka textů nápisů se pohybuje v rozmezí 1 až 270 řádků, s průměrnou délkou nápisu 4,6 řádků (aritmetický průměr 4,6 řádků; medián 3 řádky).\footnote{Aritmetický průměr ({\em mean}) je součtem všech hodnot, vydělený celkovým počtem prvků. Medián udává střední hodnotu souboru vzestupně seřazených hodnot a dělí tak soubor na dvě stejně početné poloviny. Hodnota mediánu není zpravidla ovlivněna extrémními hodnotami, např. velmi se odlišujícími maximálními či minimálními hodnotami, na rozdíl od aritmetického průměru, a lépe tak poukazuje na střední hodnoty daného souboru. Pro srovnání udávám vždy aritmetický průměr a medián daného souboru.} U 784 nápisů se nepodařilo zjistit ani jejich přibližnou délku, a to zejména kvůli jejich špatnému stavu dochování. Z celkové analýzy délky dochovaného textu je zřejmé, že tři čtvrtiny nápisů mají délku do 5 řádků (3503 nápisů), z čehož jedna třetina nápisů obsahuje právě dva řádky textu (1206 nápisů). Skupina nápisů s rozsahem od 6 do 10 řádků obsahuje 777 textů, skupina nápisů s rozsahem 11-20 řádků 319 textů, skupina nápisů s rozsahem textu 21-50 řádků obsahuje 55 textů, a nad 51 řádků se celkem vyskytuje pouhých 11 nápisů. Celkově převládají texty krátkého charakteru a nápisy delší než 20 řádků se vyskytují spíše výjimečně.

Celkem 3609 nápisů bylo publikováno jednotlivými osobami či skupinami lidí pro soukromé účely, což představuje 77 \letterpercent{} všech nápisů. Nápisy veřejné povahy, vydané politickou autoritou či samosprávní jednotkou, tvoří zhruba 15 \letterpercent{} všech nápisů, zbytek nebylo možné přesně určit. Velký počet soukromých nápisů plně odpovídá převaze kratších textů, které ve velké většině mají charakter textů publikovaných pro osobní potřebu jednotlivců či skupin lidí. Jak plyne z dat uvedených v \in{Tabulce}[Apendix1:::5_02] v \in{Apendixu}[Apendix1:::Apendix1], průměrná délka soukromého nápisu je 3,71 řádku (aritm. průměr). Oproti tomu veřejné nápisy jsou velmi často popisnějšího charakteru: jejich průměrná délka je 10,93 řádku (aritm. průměr).

Soukromé nápisy je možné podle jejich obsahu a primární funkce rozdělit na několik základních skupin. Jak je patrné z \in{Tabulky}[Apendix1:::5_03] v \in{Apendixu}[Apendix1:::Apendix1], nejpočetnějšími dvěma skupinami soukromých nápisů jsou dedikační a funerální nápisy. Tyto dvě skupiny dohromady tvoří přes dvě třetiny všech nápisů (37 \letterpercent{} pro nápisy funerální a 38 \letterpercent{} pro nápisy dedikační) a přes 97 \letterpercent{} všech soukromých nápisů. Zbývající kategorie soukromých nápisů, jako jsou vlastnické nápisy či soukromé nápisy nespadající do žádné z výše zmíněných kategorií či nápisy, které nebylo možné do konkrétní kategorie přiřadit, tvoří zbývající 2,6 \letterpercent{} soukromých nápisů.

\subsection[funerální-nápisy]{Funerální nápisy}

\index{funerální zvyklosti}Funerální nápisy představují spolu s dedikačními nápisy nejčastější kategorii nápisů, a to nejen z Thrákie, ale i z celého antického světa (Bodel 2001, 30-35). Databáze obsahuje 1740 funerálních nápisů, což představuje druhou nejpočetnější skupinu nápisů právě po dedikačních nápisech se 1777 exempláři. Do kategorie funerálních nápisů tak spadají nápisy vydávané pozůstalými členy rodiny, či blízkými přáteli zemřelého. Jejich primární funkcí bylo označit místo pohřbu a připomínat život zemřelé osoby v nejbližší komunitě (Sourvinou-Inwood 1996, 140-142). Funerální nápisy byly velmi často umístěny přímo nad místem pohřbu, ať už se jednalo o tzv. plochý hrob, či o vyvýšenou mohylu ve formě hliněného násypu, postavenou nad místem pohřbu. Podobné hroby jako v Thrákii můžeme vidět v archaické době v Athénách, ale i na jiných místech Řecka (Kurtz a Boardman 1971, 68-90). U nápisů z Thrákie je jejich archeologický funerální kontext znám u 251 nápisů, z toho 146 nápisů je možné identifikovat s pohřební mohylou či jejím bezprostředním okolím, nicméně celkové počty funerálních nápisů jsou mnohonásobně vyšší, avšak bez známého kontextu nálezu.

První funerální nápisy se objevují již v 6. st. př. n. l. v prostředí řeckých kolonií na pobřeží, kde se tato tradice uchovala po celou dobu antiky. V římské době se náhrobní kameny rozšířily i do vnitrozemí, kde se nacházejí zejména v okolí měst a podél cest.\footnote{Více o vývoji funerálních nápisů v jednotlivých stoletích v \in{kapitole}[chapter-Epigrafická-produkce-napříč-staletími:::chapter-Epigrafická-produkce-napříč-staletími]. O rozmístění funerálních nápisů v krajině více v \in{kapitole}[chapter-Mapování-kulturních-změn:::chapter-Mapování-kulturních-změn].} Obecně se soudí, že funerální nápisy na náhrobcích byly v rámci řecké a římské společnosti veřejně přístupné a každý, kdo uměl číst, se mohl dozvědět více o zemřelé osobě, ale i nejbližší komunitě (Kurtz a Boardman 1971, 86; Saller 2001, 97-107). Obsahem funerálních textů jsou nejen zmínky o nebožtíkovi, jeho původu, dosažených životních úspěších a společenské prestiži, ale podobný druh informací je možné z textu vyčíst i o nejbližších, kteří nechali nápis zhotovit. V určitých případech, a to zejména v rámci thrácké komunity, skupina funerálních nápisů mohla obsahovat i nápisy umístěné uvnitř hrobu samotného, ať už jako součást pohřební výbavy, nebo hrobové architektury. V takových případech nápisy mohou nést výpovědní hodnotu i o průběhu pohřebního ritu samotného, jakožto o zesnulém a o komunitě, z níž pocházel.\footnote{Více o konkrétním obsahu a vývoji sdělení v rámci jednotlivých komunit v \in{kapitole}[chapter-Epigrafická-produkce-napříč-staletími:::chapter-Epigrafická-produkce-napříč-staletími].}

Na funerálních nápisech se dochovaly záznamy o celkem 2294 osobách, a to v podobě osobního jména či kombinace osobních jmen, případně jmen označujících rodiče či partnery. Celkem je zaznamenáno 4678 osobních jmen, což znamená, že na jednom funerálním nápise figurovala v průměru 1,31 osoby, a jednalo se tedy primárně o náhrobky patřící spíše jednotlivcům než rozvětveným rodinám. Tento poměr se proměňuje v jednotlivých stoletích, ale obecně je možné sledovat odklon od individualismu klasické a hellénistické doby směrem k uvádění většího počtu osob na náhrobcích v římské době. Tento jev je pozorovatelný napříč dalšími oblastmi římské říše, kdy bylo zvykem uvádět na nápisech i členy širší rodiny, případně přátele, a to pravděpodobně z dědických důvodů (Saller a Shaw 1984, 1445).

\index{funerální zvyklosti}Na funerálních nápisech převládají mužská jména řeckého původu. V 71 \letterpercent{} případů se jedná o jméno patřící muži, ve 21 \letterpercent{} o ženě a zhruba 5 \letterpercent{} není možné jméno přiřadit ani do jedné skupiny, ať už pro nejednoznačnost, případně špatný stav dochování. Poměr původu osobních jmen vyznívá ve prospěch jmen řeckého původu. Řeckých jmen se dochovalo 60 \letterpercent{}, thráckých jmen 10 \letterpercent{}, římských jmen 21 \letterpercent{} a jmen nejasného původu bylo 9 \letterpercent{}. Kolektivní vyjádření identit na funerálních nápisech se dochovala u 60 textů. Převážně se jedná o uvedení původu zemřelého či člena rodiny. U 48 případů je to odkaz na město se silnou řecky mluvící populací převážně v Thrákii, případně také v Malé Asii. Odkaz na region se objevuje pouze třikrát, na příslušnost ke konkrétnímu kmeni šestkrát, dále výraz {\em barbaros} pouze dvakrát a původ označovaný místní vesnicí pouze jednou. Na 57 nápisech se dochovaly dialektální znaky, což ve většině případů byly znaky dórského dialektu a tyto nápisy pocházely z původně dórských kolonií Mesámbriá a Byzantion. Celkem 628 nápisů si uchovalo tradiční epigrafické formule spojené s funerálními nápisy, jako např. {\em mnémé}, {\em theois katachthoniois, chaire parodeita} téměř po celou dobu antiky. Pro popis místa pohřbu se používaly ustálené termíny {\em tymbos}, {\em tafos}, {\em bómos} a {\em stélé} pro označení náhrobního kamene. V římské době se objevují sarkofágy a spolu s nimi i termíny {\em latomeion}, {\em soros.} Jazyk užívaný pro účely funerálních nápisů si zachoval poměrně stabilní formu a tradiční terminologii, nicméně s příchodem Říma došlo k proměně v rámci ritu samotného, kdy hrobky na určitou dobu doplnily i sarkofágy a urny, jak dokazují četné archeologické nálezy (Tomas 2016, 84-87).

\subsection[dedikační-nápisy]{Dedikační nápisy}

\index{náboženské zvyklosti}Dedikační nápisy představují nejčastější typ dochovaných nápisů v HAT databázi s celkem 1777 exempláři. Dedikačními nápisy jsou označovány předměty nesoucí věnování určitému božstvu. Zpravidla bývají nalézány ve společnosti dalších votivních předmětů na místě svatyně daného božstva, ale mohou být nalezeny i mimo kultovní kontext (Janouchová 2013b; 2016; v tisku 2018).\footnote{V určitých případech může docházet k cyklické kategorizaci, kdy je svatyně označena svatyní pouze na základě přítomnosti dedikačních nápisů, či naopak je nápis označen jako dedikační na základě nálezu v rámci archeologického kontextu svatyně. Abych tomuto předešla, přidržuji se informací poskytovaných editory jednotlivých epigrafických korpusů, avšak primárně přihlížím k obsahu a charakteristice nápisu, a automaticky neřadím jako dedikace nápisy pocházející z lokality charakterizované jako svatyně.} Dedikační nápisy věnují konkrétnímu božstvu či božstvům, jak jednotlivci, tak i skupiny lidí, a to s prosbou o pomoc či jako díky za již vykonané skutky a dobrodiní. Zhotovitelé nápisu často uvádějí nejen důvod věnování nápisu, ale i svou identitu, která v kontextu dané komunity hrála relevantní roli. Na nápisy tak zaznamenávají své osobní jméno, původ, zařazení do určité rodiny a komunity, ale mnohdy zdůrazňují i své životní úspěchy, případně společenský status.\footnote{Tato sdělení a jejich vývoj v rámci komunit podrobněji rozebírám v chronologickém přehledu jednotlivých století pak v \in{kapitole}[chapter-Epigrafická-produkce-napříč-staletími:::chapter-Epigrafická-produkce-napříč-staletími].}

Celkem bylo možné na dedikačních nápisech dle osobních jmen identifikovat 1649 osob, což představuje průměrně 1,077 osoby na nápis. Dedikační nápisy byly spíše individuální záležitostí a se skupinovými dedikacemi se setkáváme méně často než například s rodinnými funerálními nápisy. Dedikant se na nápisech zpravidla identifikoval jedním až dvěma osobními jmény, nejčastěji svým jménem a jménem rodiče. Celkem se dochovalo 3112 osobních jmen, z čehož jména řeckého původu představují 29,5 \letterpercent{}, jména thráckého původu 24,5 \letterpercent{}, jména římského původu 36 \letterpercent{} a jména nejistého původu 10 \letterpercent{}. Většinu dedikantů představovali muži, a to celkem v 88 \letterpercent{} případů. Ženy dedikovaly nápisy jen v 7 \letterpercent{} případů a u 5 \letterpercent{} osobních jmen nešlo jednoznačně určit pohlaví dedikanta.

Na dedikačních nápisech zaznamenáváme větší míru zapojení thráckého obyvatelstva, než je tomu u funerálních nápisů. Dedikace byly z převážné většiny určeny božstvům nesoucím kombinovaná řecká jména a lokální epiteta, poukazující na udržování místních tradic zároveň s adaptací nových kulturních a náboženských prvků. Nejoblíbenějšími božstvy byl Asklépios s 249 dedikacemi, Apollón se 139, Zeus se 131, Héra se 72, a dále Dionýsos a Nymfy se 38 dedikacemi, Hygieia s 31, Héraklés s 28 a Artemis s 24. Mezi božstva z východního Středomoří patří Sarápis s 12 dedikacemi, Ísis s 10, Anúbis se třemi, Harpokratión se čtyřmi, Mithra s osmi, Kybelé s jednou, syrská bohyně se dvěma, Velká Matka se sedmi, Sabazios s pěti dedikacemi.

\index{náboženské zvyklosti}U 586 nápisů se dochovalo božské epiteton, které u dvou třetin nápisů poukazovalo na místní původ kultu. Mezi místní {\em héróy} nesoucí lokální přízvisko patřil {\em Zylmyzdriénos} se 47 nápisy\footnote{Dochované varianty jména jsou {\em Zylmyzdriénos, Zymdrénos, Zymedrénos, Zymlydrénos, Zymydrénos, Zymyzdros, Zysdrénos}.}, {\em Salsúsénos} s 32 nápisy\footnote{Dochované varianty jména jsou {\em Saldénos, Saldobysénos, Saldokelénos, Saldoúissénos, Saldoúsenos, Saldoússénos, Saldoúusénos, Saldúisénos, Saldúsénos, Saldúusénos, Salénos, Saltobysénos, Saltobyssénos, Saltúusénos}.}, {\em Keiladeénos} s 20 nápisy\footnote{Dochované varianty jména jsou {\em Keiladeénos, Kiladeénos, Deiladebénos, Keilaiskénos, K{[}eilade{]}énos}.}, {\em Karistorénos} se 17 nápisy, {\em Karabasmos} s 11 nápisy a {\em Zbelsúrdos} s devíti nápisy.\footnote{Dochované varianty jména jsou {\em Zbelsúrdos} a {\em Zbelthiúrdos}.} U těchto nápisů je nepatrně vyšší zastoupení jmen thráckého původu s 26,5 \letterpercent{}, zatímco jména řeckého původu mírně klesla na 25,5 \letterpercent{}, jména římská zůstávají na přibližně stejné úrovni 37 \letterpercent{} a neurčená jména na 11 \letterpercent{}.\footnote{Podobné srovnání pro oblast okolo Novae ukazují, že zhruba 24 \letterpercent{} nápisů je věnováno lidmi s řeckým jménem, 14 \letterpercent{} se jménem thráckého původu a 62 \letterpercent{} se jménem latinského původu čili římského (Tomas 2016, 84).} Vyšší počet thráckých jmen může poukazovat na oblíbenost místních božstev u thrácké populace, a tedy i kontinuitu nábožesnkých představ.

Dedikační nápisy se staly velmi oblíbené ve vnitrozemských komunitách, kde můžeme sledovat vznik svatyní s velkými koncentracemi nápisů zejména ve 2. a 3. st. n. l. Z těchto svatyní pochází velká část dedikací, avšak některé lokality jsou známy i díky jedinému, někdy i náhodnému nálezu votivního předmětu či nápisu.\footnote{Více o dedikačních nápisech v proměnách času v \in{kapitole}[chapter-Epigrafická-produkce-napříč-staletími:::chapter-Epigrafická-produkce-napříč-staletími] a o jejich rozmístění v krajině v \in{kapitole}[chapter-Mapování-kulturních-změn:::chapter-Mapování-kulturních-změn].}

\subsection[veřejné-nápisy]{Veřejné nápisy}

\index{administrativa}Celkem se dochovalo 717 veřejných nápisů, což představuje 15 \letterpercent{} všech dochovaných nápisů. Veřejné nápisy, tedy nápisy zhotovené politickou a administrativní autoritou za účelem organizace společnosti a udržení společenského řádu, je taktéž možné rozdělit podle jejich obsahu do několika částečně se prolínajících skupin. Jak plyne z dat uvedených v \in{Tabulce}[Apendix1:::5_04] v \in{Apendixu}[Apendix1:::Apendix1], nejpočetnější skupinou jsou honorifikační nápisy, tedy nápisy vydané v rámci samosprávné jednotky k poctě jednotlivce či skupiny lidí, které představují 41,56 \letterpercent{} veřejných nápisů a 6,39 \letterpercent{} všech nápisů. Další početnou skupinou jsou státní dekrety, tedy nařízení vydaná politickou autoritou, která mohou mít normativní charakter, či se jedná o veřejně vydaná ustanovení a smlouvy (2,35 \letterpercent{} všech nápisů). Dále do kategorie veřejných nápisů spadají i seznamy nejrůznějšího charakteru (1,37 \letterpercent{} všech nápisů) a náboženské texty (1,14 \letterpercent{} všech nápisů), veřejné nápisy nespadající do předchozích kategorií (2,94 \letterpercent{} všech nápisů) a veřejné nápisy, které nebylo možné přesněji určit (1,74 \letterpercent{} všech nápisů).

Jazyk veřejných nápisů je do značné míry konzervativní: ustálené formule a termíny se opakují v nápisech stejné funkce. Celkem 532 nápisů obsahuje hledané administrativní termíny\index{administrativa}, a to celkem v 1362 výskytech, což představuje zhruba 2,6 termínu na nápis. Mezi nejčastější termíny patří {\em démos} s 217 výskyty, {\em polis} se 158, {\em búlé} se 153, {\em autokratór} se 113, {\em kaisar} s 84, {\em presbeutés} a {\em antistratégos} se 79, {\em hégémón} se 41 a {\em hypatos} se 41 výskyty. V průběhu doby dochází k nárůstu používání těchto termínů, což souvisí jednak s narůstají mírou epigrafické produkce, ale i zintenzivněním společenské organizace a vytvořením nových administrativních institucí v době římské.

I přes jistou ustálenost formy je možné sledovat vývoj regionálních variant textu, což by poukazovalo na míru autonomie jednotlivých oblastí, a to zejména v předřímské době. Vliv jednotné autority, který by měl za následek sjednocování formy veřejných nápisů, není v této době patrný. Pokud docházelo k ovlivňování a výpůjčkám v rámci společenské organizace, dělo se tak místně a v relativně malém měřítku. Naopak v římské době dochází k určitému sjednocení obsahu i formy veřejných nápisů na celém území Thrákie, patrně pod vlivem jednotné administrativy a rozsáhlého byrokratického aparátu.\footnote{Podrobně se vývojem veřejných nápisů v jednotlivých stoletích zabývám v \in{kapitole}[chapter-Epigrafická-produkce-napříč-staletími:::chapter-Epigrafická-produkce-napříč-staletími], o jejich rozmístění v krajině pak v \in{kapitole}[chapter-Mapování-kulturních-změn:::chapter-Mapování-kulturních-změn].} Příkladem mohou být například milníky, které si zachovaly téměř identickou podobu nejen v Thrákii, ale například i v sousední Bíthýnii a i v jiných místech římské říše (pro Bíthýnii: French 2013).

\stopcomponent