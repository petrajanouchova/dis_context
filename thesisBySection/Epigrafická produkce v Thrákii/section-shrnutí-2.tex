
\environment ../env_dis
\startcomponent section-shrnutí-2
\section[shrnutí-2]{Shrnutí}

Epigrafická produkce v antické Thrákii měla z velké části charakter soukromých sdělení, která z převážné většiny souvisela s pohřebním ritem a náboženskými aktivitami. Funerální nápisy pocházely většinově z řecké a v později i z římské komunity, a osobami, které texty nápisů zmiňují, jsou převážně muži, ač ženy byly zastoupeny z jedné pětiny. Dedikační nápisy poukazují na větší zapojení thráckého obyvatelstva mužského pohlaví, a naopak pokles počtu ženských dedikantů. S větším zapojením místního obyvatelstva souvisí i rostoucí popularita lokálních kultů, tedy kultů nesoucích nejčastěji řecké jméno božstva doplněné lokální epitetem.

Převládajícím publikačním jazykem byla v předřímské době řečtina, v římské době v provincii {\em Thracia} a v provincii {\em Moesia Inferior} latina v kombinaci s řečtinou. Latina se udržovala jako jazyk vojenské komunikace administrativy spojené s chodem římské říše, avšak občas docházelo i ke kombinaci obou jazyků na jednom médiu v souvislosti s očekávanými čtenáři a jejich jazykovými znalostmi. Řečtina se prosadila spíše v kontextu náboženských textů a ve venkovských oblastech, zatímco latinské nápisy pocházejí z okolí vojenských táborů a administrativních center. Volba daného jazyka byla do velké míry záležitostí individuální volby zhotovitele nápisu, zejména vzhledem k původu zhotovitele a publiku, jemuž bylo sdělení určeno.

Osobní jména na nápisech naznačují, že epigrafická produkce byla zaměřená především na mužskou část populace, a ženy vystupovaly spíše v roli partnerek a dcer než jako primární adresát nápisu. Větší zastoupení žen se vyskytuje na funerálních nápisech z řeckých kolonií na pobřeží, kde ženy tvořily asi jednu pětinu adresátů nápisů. Tento trend se nicméně nepřenesl do vnitrozemských oblastí a na dedikační nápisy, kde je zastoupení žen nižší. Co se týče onomastických zvyklostí, thrácká populace přijmula jak zvyky řeckého prostředí, ale i později římský způsob uvádění jména. V thráckém prostředí se typicky udávalo jméno otce, případně prarodiče, podobně jako v řeckých komunitách. Nezáleželo na tom, zda jména stejného či různého původu čili zda byla thrácká, či řecko-thrácká, a stejně tak i jména rodičů. Tento způsob identifikace Thrákové zřejmě přijali od Řeků, pokud ho však již nepoužívali v dobách před výskytem prvních dochovaných nápisů.

V případě přejímání římských onomastických zvyků se setkáváme s užíváním tří jmen ({\em tria nomina}), stejně jako bylo obvyklé v jiných částech římské říše, např. v {\em Gallii} (Carroll 2006, 129-130; 253-255). Thrácké jméno je obvykle kombinované s jedním či dvěma římskými jmény a poukazuje tak na vyšší společenský status nositele, případně na jeho právní postavení. I nadále je však zvyk uvádět tři osobní jména kombinován s udáváním původu v podobě jmen otce či prarodiče a setkáváme se tak s komplikovaným systémem identifikace jednotlivců za zdůraznění thráckého původu nositele.

\stopcomponent