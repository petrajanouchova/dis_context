
\environment ../env_dis
\startcomponent chapter-Epigrafická produkce v Thrákii
\chapter{Epigrafická produkce v Thrákii}
V této kapitole charakterizuji dochovaný epigrafický materiál vzhledem k jeho rozmístění v Thrákii, fyzické podobě objektu nesoucí nápis a typovému zařazení a funkci nápisů. Dále se věnuji jednotlivým aspektům epigrafické produkce v souvislosti s prezentací identity osob, jako je třeba volba publikačního jazyka, či proměny onomastických zvyků. Výchozím souborem je 4665 nápisů, které jsem shromáždila v rámci databáze {\em Hellenization of Ancient Thrace} (HAT), zahrnující materiál celkem z 11 století, konkrétně od 6. st. př. n. l. do 5. st. n. l. (Janouchová 2014).\footnote{Digitální apendix, spolu s kompletním obsahem databáze je volně dostupný na GitHubu na adrese \useURL[url18][https://github.com/petrajanouchova/hat_project][][{\em https://github.com/petrajanouchova/hat_project}]\from[url18], stav k 9. listopadu 2016. Tabulka 5.00 v Apendixu 1 shrnuje celkový počet záznamů v databázi rozdělený dle jednotlivých druhů záznamů.}

\stopcomponent