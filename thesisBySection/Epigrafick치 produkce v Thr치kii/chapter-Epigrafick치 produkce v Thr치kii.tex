
\environment ../env_dis
\startcomponent chapter-Epigrafická-produkce-v-Thrákii
\startchapter[title={Epigrafická produkce v Thrákii}, reference={chapter5}, marking={Epigrafická produkce v Thrákii}]

V této kapitole charakterizuji dochovaný epigrafický materiál jako celek, se zaměřením na obecné trendy prolínající se napříč celým souborem. Věnuji se zejména jednotlivým aspektům epigrafické produkce v souvislosti s prezentací identity osob, jako je volba publikačního jazyka či proměny onomastických zvyků. Výchozím souborem je 4665 nápisů, které jsem shromáždila v rámci databáze {\em Hellenization of Ancient Thrace} (HAT), zahrnující materiál celkem z více než 11 století a pocházející z jihovýchodní části balkánského polostrova (Janouchová 2014).\footnote{Digitální apendix, spolu s kompletním obsahem databáze je volně dostupný na GitHubu na adrese \useURL[url19][https://github.com/petrajanouchova/hat_project][][{\em https://github.com/petrajanouchova/hat_project}]\from[url19], stav k 9. listopadu 2016. Tabulka 5.00 v Apendixu \in[Apendix1:::Apendix1] shrnuje celkový počet záznamů v databázi rozdělený dle jednotlivých druhů záznamů.}

\stopchapter
\stopcomponent