
\environment ../env_dis
\startcomponent section-prostorová-analýza-epigrafické-produkce
\section[prostorová-analýza-epigrafické-produkce]{Prostorová analýza epigrafické produkce}

Prostorovou analýzu vytvářím na základě souboru 4453 nápisů, což představuje 95 \letterpercent{} všech nápisů obsažených v databázi {\em Hellenization of Ancient Thrace}. Tento soubor nápisů byl zvolen na základě dostupných informací o jejich místě nálezu, respektive o přesné lokalizaci místa nálezu uváděné v epigrafických korpusech a původních zdrojích. Tuto informaci o přesnosti místa nálezu v databázi zastupuje tzv. {\em position certainty index}, s hodnotami přesného určení místa nálezu do 1 km s 1540 nápisy, do 5 km s 2323 nápisy a do 20 km s 590 nápisů. Celkem 212 nápisů s přesností určení nad 20 km do současné analýzy nezahrnuji, vzhledem k nízké výpovědní hodnotě o místě vzniku i místě nálezu.\footnote{Pro podrobnosti o zvolené metodě výběru nápisů a zaznamenávání přesnosti místa nálezu více v kapitole 4 věnované metodologii práce.}

\subsection[epigrafická-produkční-centra-1]{Epigrafická produkční centra}

Oblast Thrákie se počty dochovaných nápisů řadí mezi střední producenty, a to zejména od 1. st. n. l., kdy se stala součástí římské říše. Srovnáme-li území, které Thrákie zhruba zaujímala a celkový počet dochovaných nápisů dostaneme se na hodnoty epigrafické produkce srovnatelné s římskou provincií {\em Gallia} či {\em Germania} (Woolf 1998, 81-82). Bereme-li průměrnou velikost území Thrákie v době římské zhruba 130,000 km\high{2}, a celkový počet dochovaných nápisů je 4665, hustota nápisů činí 3,58 nápisu na 100 km\high{2}. Toto číslo se samozřejmě liší v čase. V době římské sledujeme zhruba 2,6x vyšší hustotu nápisů než v předcházejících obdobích. Skupina nápisů od 6. do 1. st. př. n. l. na stejně velkém území zaujímá hustotu zhruba 0,75 nápisu na 100 km\high{2}. V případě nápisů od 1. do 5. st. n. l. se jedná o hustotu 1,98 nápisů na 100 km\high{2}, nedatované nápisy mají hodnotu 0,85 nápisu na 100 km\high{2}. Tato čísla poukazují na výrazný nárůst epigrafické produkce v době, kdy se Thrákie stala součástí římské říše. V žádném případě ale Thrákie nedosahovala úrovní provincií na Apeninském poloostrově, kde byla hustota nápisů 13 na 100 km\high{2} a v případě {\em Latia} i 55 nápisů na 100 km\high{2}. Thrákie v době římské se tak řadí spíše na úroveň provincie {\em Gallia Comata, Belgica} či {\em Germania Inferior} s hustotou dva nápisy na 100 km\high{2} (Woolf 1998, 81-83).

Nárůst epigrafické produkce však nebyl na všech místech rovnoměrný, ale je možné sledovat shlukování nápisů v okolí měst, pohřebišť, svatyní a dále v okolí komunikací. Vzhledem k tomu, že nápisy sloužily zejména pro místní trh a nestaly se komoditou dálkového obchodu ve větším měřítku, se dá předpokládat, že produkční centra se nacházela nedaleko od místa nálezu nápisu, zpravidla v řádu několika kilometrů. Na základě srovnání archeologických dat Bekker-Nielsen (1989, 30-32) došel k závěru, že vzdálenosti mezi centry a jejich ekonomicky a společensky závislými regiony se pohybují od 10 do 37 km dle zastávané funkce. Dle Bekker-Nielsena je průměrnou vzdáleností od města k hranicím závislého regionu 37 km, což je průměrná délka maximálního denního pochodu v římské době, která odpovídala 25 římským mílím. Délka denního pochodu se stala jednou ze základních délkových jednotek udávaných např. při přesunech římského vojska (Madzharov 2009, 51). Z těchto čísel Bekker-Nielsen (1989, 30-32) dále odvozuje délku půl-denního pochodu, tedy vzdálenost 18,5 km, která by umožňovala obyvatelům žijícím v okolí centra cestu do města a návrat domů v témže dni. Vzdálenost 10 km pak používá jako oblast ze které se obyvatelé mohli uchylovat pod ochranu centra v případě nebezpečí, a to v řádu dvou hodin chůze či jedné hodiny jízdy.\footnote{Částečně vycházím z údajů projektu Orbis vytvořeného na Stanford University, \useURL[url23][http://orbis.stanford.edu/][][{\em http://orbis.stanford.edu/}]\from[url23]. Tento projekt modeluje a mapuje rychlost přepravy v římském světě na základě geografických dat a známých historických a literárních zdrojů, nejčastěji v podobě tzv. itinerářů. Autoři projektu zahrnují různé způsoby přepravy a k nim doplňují i průměrnou vzdálenost v kilometrech, jakou bylo možné tímto druhem přepravy urazit za jeden den. Pro pěší pochod počítají 30 km za den, pro přepravu pomocí vozu taženého oslem 12 km za den, pro přepravu s větším vozem 36 km za den, pro jízdu na koni 56 km za den, zrychlenou jízdu na koni až 250 km za den, na lodi apod. Autoři berou v úvahu i zrychlený přesun vojenských jednotek, kterému stanovili průměrnou hodnotu 60 km za den (Scheidel {\em et al.} 2012, \useURL[url24][http://orbis.stanford.edu/orbis2012/ORBIS_v1paper_20120501.pdf][][{\em http://orbis.stanford.edu/orbis2012/ORBIS_v1paper_20120501.pdf}]\from[url24]). Pro vypočtení konkrétní trasy je možné zadat celou řadu parametrů, které zohledňují jak roční a denní dobu, rychlost přepravy, způsob přepravy, zda se jednalo o vojenskou či soukromou přepravu. Na základě těchto parametrů pak Orbis spočítá nejen dobu trvání cesty, ale i její finanční nákladnost v římských denárech. Příkladem může být např. cesta z Byzantia do Filippopole, která v červenci konaná pěšky s rychlostí 30 km za den trvá 11,7 dní. Pokud použijeme rychlost zrychleného vojenského přesunu 60 km za den, dostaneme se na 6,2 dní. Cesta mezi městy Serdica a Byzantion trvala 16,5 dní pěšího pochodu o rychlosti 30 km za den a 8,6 dne zrychlené vojenského přesunu o rychlosti 60 km za den. Odéssos a Byzantion jsou ve vzdálenosti 2,3 dní plavby na moři.}

Vzhledem k tomu, že náročnost terénu se na území Thrákie navzájem liší, zaokrouhlila jsem vzdálenosti půldenního pochodu na 20 km a celodenního pochodu na 40 km. Území v okruhu do 20 km považuji za oblast spadající pod přímý ekonomicky a společensky vliv daného centra. Území do vzdálenosti do 40 km představuje oblast maximálního dosahu vlivu daného centra, avšak s nižší mírou interakce mezi centrem a regionem. Pro účely této práce se držím střední hodnoty 20 km pro vyznačení regionu daného centra s přímým ekonomickým a kulturním vlivem, což také odpovídá hodnotám 1 až 3 koeficientu určení přesnosti místa nálezu. Pro srovnání udávám i počty nápisů nalezených v širším regionu 40 km, abych srovnala bezprostřední okolí centra s jeho regionem a z nich vyplývající trendy v rozmístění epigrafické produkce.

Na základě rozmístění zvýšených koncentrací nápisů v terénu je možné určit produkční centra, tedy místa, kde byly nápisy s největší pravděpodobností vytvářeny a určeny pro ekonomicky závislý region. Tato produkční centra byla v Thrákii nerovnoměrně rozmístěna v závislosti na geografických podmínkách, ale i na demografickém uspořádání oblasti.

\subsection[faktory-ovlivňující-rozmístění-nápisů]{Faktory ovlivňující rozmístění nápisů}

Rozmístění nálezových míst nápisů ovlivňují v první řadě geografické podmínky. Pokud se podíváme na mapu nálezových míst a jejich pozici v krajině, téměř tři čtvrtiny nápisů se nalézají v nížinách do 249 m. n. m. Na úpatí hor a v nižších horských polohách se nachází něco málo přes 20 \letterpercent{} nápisů. Zbylých 6 \letterpercent{} nápisů pochází z horských oblastí zejména v severovýchodní části Thrákie. Obecně je v horských oblastech méně nápisů než ve vnitrozemí, nicméně podél toků řek se nápisy vyskytují i v polohách nad 1000 m. n. m. Jak je dále patrné z mapy 7.01a v Apendixu \in[Apendix2:::Apendix2], nápisy mají tendenci se shlukovat jednak na mořském pobřeží a ve vnitrozemí v okolí velkých řek, které v antice sloužily jako hlavní komunikační tepny (Bouzek 1996, 221-222; Bravo a Chankowski 1999, 310-311; Archibald 2002).

Zeměpisné podmínky však nejsou jediným faktorem ovlivňujícím rozmístění nápisů. Pravděpodobně ještě důležitější roli hraje umístění lidských sídel, případně míst určených pro vybranou aktivitu, jako např. pohřebiště či svatyně. Z mapy 7.02a v Apendixu \in[Apendix2:::Apendix2] je patrné, že velká část nápisů se nachází ve vzdálenosti 20 km od města, tedy ve vzdálenosti, kterou bylo možné ujít v jednom dni. V okruhu do 20 km od města se nachází 67 \letterpercent{} nápisů. Zbývajících 33 \letterpercent{} pochází z oblastí vzdálených od města více než 20 km. Pokud okruh okolo města rozšíříme na délku maximálního denního pochodu, do vzdálenosti 40 km od města spadá 83,5 \letterpercent{} nápisů. Zbývajících 16,5 \letterpercent{} nápisů se nachází ve větší vzdálenosti než 40 km od regionálních produkčních center.

Důležitou roli hraje nejen vzdálenost od nejbližšího města, ale i pozice vůči cestám, jak je dobře vidět na mapě 7.03a v Apendixu \in[Apendix2:::Apendix2]. Jako komunikace byly v antice využívány i velké řeky, které byly pravděpodobně splavné minimálně na některých úsecích. Cesty existovaly již v předřímských dobách, ale k jejich systematické stavbě, rozšiřování silniční sítě a údržbě docházelo až od 1., ale zejména ve 2. a 3. st. n. l., jak dosvědčují dochované milníky či související stavby (Madzharov 2009, 29-40). Z blízkosti několika kilometrů od cest pocházela převážná část nápisů: ve vzdálenosti do 20 km od cest se našlo 95 \letterpercent{} nápisů, nad 20 km pak zbývajících 5 \letterpercent{} nápisů. Ve vzdálenosti do 10 km od cesty se našlo 83 \letterpercent{} nápisů, nad 10 km pak 17 \letterpercent{} nápisů. Ve vzdálenosti do 5 km od cest bylo lokalizováno 75 \letterpercent{} nápisů, 25 \letterpercent{} pak ve vzdálenosti větší než 5 km.\footnote{Se zmenšující se vzdáleností od cesty se zmenšoval i počet nápisů. Jinými slovy, oblast do 5 km okolo cest obsahovala méně nápisů než oblast pokrývající území v okruhu 10 či 20 km okolo cesty.} Z toho vyplývá, že více jak dvě třetiny nápisů byly nalezeny v bezprostředním okolí cest. Tato existující infrastruktura v podobě silnic do velké míry usnadňovala pohyby nejen vojsk, ale i běžného obyvatelstva, které tak mohlo např. snadněji navštěvovat svatyně ve vzdálenějších oblastech. V oblasti se schůdným terénem jako např. v okolí Filippopole se vzdálenost od města, kterou bylo možné ujít v jednom dni, prodlužuje. V případě pohybu po {\em Via Diagonalis} a nápisů nalezených v okolí města Filippopolis to může být až 60 km po západo-východní ose.

Zásadní roli na rozmístění nápisů v krajině tedy hrály jednak příznivé přírodní podmínky a přístupný terén v kombinaci s blízkostí lidských sídel a rozmístění sítě komunikací, ať už v podobě řek či pozemních cest. Hustota nápisů byla největší ve městech, případně ve vybraných svatyních a s narůstající vzdáleností od města počet nápisů klesal. Tento trend částečně narušovaly komunikace, v jejichž bezprostřední blízkosti se nápisy taktéž nacházely, a to pravděpodobně díky zvýšenému pohybu obyvatelstva a usazování v menších sídlech a stanicích, které se staraly o údržbu a bezpečnost cest (Madzharov 2009, 43-57).

\subsection[skupiny-produkčních-center-nápisů]{Skupiny produkčních center nápisů}

Produkční centra je možné charakterizovat jako místa se zvýšenou koncentrací nálezů nápisů. Mapa 7.04a v Apendixu \in[Apendix2:::Apendix2] dokumentuje hustotu nalezených nápisů na území Thrákie v podobě tzv. teplotní mapy. Místa s tmavší barvou představují místa s vyšší koncentrací nápisů v okruhu 20 km. Nejtmavší místa jsou velká města jak na pobřeží, tak ve vnitrozemí na důležitých cestách či křižovatkách cest. V horských oblastech je minimum míst s nejvyšší koncentrací nápisů, ale vyskytují se zde lokality se středními hodnotami počtu nápisů. Většina míst s největší koncentrací nápisů je v nejbližším okolí měst, ale v několika případech se setkáváme i s vysokou koncentrací nápisů v lokalitách neměstského charakteru.

\subsubsection[produkční-centra-městského-charakteru]{Produkční centra městského charakteru}

Jak plyne z výše řečeného, největším producentem nápisů jsou města a jejich nejbližší regiony ve vzdálenosti do 20 km. Města je možné rozdělit podle počtu nalezených nápisů na nadregionální producenty s počtem okolo 300 a více nápisů, velké regionální producenty s počty od 150 do 250 nápisů, menší regionální centra s 50 až 149 nápisy a malá produkční centra s 49 a méně nápisy. Jejich polohu v Thrákii a kategorii, do níž dané produkční místo spadalo ilustruje mapa 7.05a v Apendixu \in[Apendix2:::Apendix2].

Do kategorie producentů s nadregionálním významem patří města, která svým významem překračovala území Thrákie či zastávala významnou pozici v rámci nadregionální samosprávy. Do této kategorie patří města Byzantion, Odéssos a Filippopolis. Ač nejvíce nápisů bylo nalezeno v černomořském Odéssu, a to celkem 359, toto číslo je ve skutečnosti o pár desítek nápisů nižší, protože do 20 km okruhu okolo Odéssu zasahuje území Marcianopole a Dionýsopole a program QGIS započítal tyto nápisy na pomezí ke všem městům stejně. Nicméně i přesto z regionu Odéssu pochází zhruba 300 nápisů a řadí se tak k jedněm z největších producentů nápisů v Thrákii. Z Byzantia a jeho regionu pochází 352 nápisů, což ho tak řadí na první místo epigrafické produkce.\footnote{Byzantion zastával významnou pozici v době vlády Říma, a to zejména od 4. st. n. l., kdy se stal pod jménem Konstantinopol hlavním městem římské říše (Jones 1971, 23).} Filippopolis s 296 nápisy spadá do kategorie nadregionálních produkčních center, jakožto administrativní a kulturní centrum provincie {\em Thracia}.

Do skupiny velkých regionálních center spadají čtyři města na pobřeží Černého, Marmarského a Egejského moře, která hrála roli administrativního a ekonomického centra daného přímořského regionu. Největším producentem z této skupiny je Perinthos s 248 nápisy, dále Maróneia s 234 nápisy, Apollónia Pontská s 218 nápisy a Mesámbriá se 172 nápisy. Tato města zaujímala pozici regionálních center zejména v 6. až 3. st. př. n. l. a v případě Perinthu až 1. st. př. n. l., nicméně si jistou míru autonomie a politické moci udržela i v římské době.

Skupina středních regionálních center zahrnuje města jak na mořském pobřeží, tak města vnitrozemská, která se nacházela většinou na křižovatkách cest či významných komunikacích a sloužila jako administrativní a ekonomická centra pro nejbližší region. Do této kategorie patří Parthicopolis\footnote{Též známá jako Paroikopolis, nejčastěji ztotožňovaná s osídlením nacházejícím se pod základy současného města Sandanski v Bulharsku (Mitrev 2017, 106-108).} se 147 nápisy, Nicopolis ad Istrum se 136 nápisy, Hérakleia Sintská se 128 nápisy, Augusta Traiana se 124 nápisy, Pautália se 122 nápisy, Serdica se 112 nápisy a Marcianopolis se 122 nápisy.

Skupina malých regionálních center s méně než 30 nápisy zaujímala pouze marginální roli v epigrafické produkci, což však nereflektuje její postavení ve společnosti jako ekonomické či politické centrum okolní oblasti, jak je tomu např. u Ainu či Nicopolis ad Nestum či Bizyé. Tento stav spíše reflektuje nedostatečný stav prozkoumání regionu těchto měst, zejména z důvodu moderní zástavby. V budoucnu se dá tak v těchto městech dají očekávat nálezy nápisů, které by tato města posunula do kategorie středních regionálních producentů.

\subsubsection[produkční-centra-neměstského-charakteru]{Produkční centra neměstského charakteru}

Z již zmiňované teplotní mapy 7.04 v Apendixu \in[Apendix2:::Apendix2] plyne, že místa s velkou hustotou nápisů nebyla pouze městského charakteru, ale dochovalo se i několik lokalit mimo region města s vysokou koncentrací nápisů. V těchto případech se jedná o svatyně umístěné ve volné přírodě, kde bylo nalezené velké množství dedikací. Společnou charakteristikou těchto svatyní byla dobrá dostupnost, a tedy i blízkost cest v případě horských lokalit či nenáročnost terénu v případě svatyň v nížinách a v podhůří, jak je též patrné z mapy 7.06a v Apendixu \in[Apendix2:::Apendix2].

Mezi svatyně s největším počtem nápisů patří svatyně v Batkunu s 194 nápisy nalezenými ve vzdálenosti do 5 km od místa svatyně Asklépia.\footnote{Tsonchev (1941). Většina nápisů byla nalezena v bezprostření blízkosti svatyně, nicméně nekteré nápisy byly sekundárně přemístěny do nedalekého kostela. To je ostatně případ velkého počtu nápisů, které bývaly často použity jako stavební materiál pozdějších staveb, zejména pak kostelů.} Dále sem patří svatyně Asklépia z lokality Slivnica se 72 nápisy, Glava Panega se 77 nápisy, Daskalovo se 76 nápisy.\footnote{Boteva (1985); Gocheva (1992), Oppermann (2006, 147-154).} Svatyně Apollóna nesoucího místní epiteta pochází z lokalit Kiril Metodievo s 33 nápisy a Kran taktéž s 33 nápisy.\footnote{Tabakova (1959), Tabakova-Tsanova (1961; 1980).} Svatyně Nymf a Asklépia je známá z lokality Búrdapa, kde bylo nalezeno 47 nápisů.\footnote{Janouchová (2013b, 14).} V lokalitě Mezdra bylo nalezeno 17 nápisů věnovaných různým božstvům, mimo jiné Démétér či Diovi. S lokalitě Skaptopara se pak jedná převážně o funerální nápisy náležející k blízkému osídlení - thrácké vesnici Skaptopara.

Většina těchto svatyní pochází ze severozápadní Thrákie z horských oblastí. Nejníže položená lokalita se nachází ve výšce 209 m. n. m a nejvýše položená ve výšce 754 m. n. m. Průměrná nadmořská výška všech devíti lokalit je 404 m. n. m. (aritmetický průměr; medián je 345 m. n. m.). Lokality jsou v těsné blízkosti cest a řek, což usnadňovalo věřícím přístup a pravděpodobně tento fakt hrál roli v jejich oblíbenosti mezi věřícími. Svatyně s nejvíce nápisy se nachází ve vzdálenosti 28, respektive 50 km od města Filippopolis, které po většinu římské doby zastávalo roli hlavního města provincie a jehož obyvatelé s největší pravděpodobností navštěvovali tuto svatyni Nymf v Búrdapě a Asklépia v Batkunu (Janouchová 2013b).

Ať už se jedná o nápisy nalezené v regionu měst či v lokalitách neměstského charakteru, zásadní roli na rozmístění lokalit měla existence místní infrastruktury a systému komunikací. Nápisy se objevují zejména ve vzdálenosti dosažitelné v rámci jednoho dne od hlavních administrativních center a podél cest. Čím bylo dané sídlo důležitější a koncentrovaly se v něm politické a administrativní instituce, tím více se v jeho okolí objevilo i nápisů. Ekonomická aktivita jednotlivých měst na celkovou výši epigrafické produkce zásadní vliv neměla, příkladem může být město Ainos, jeden z hlavních říčních a přímořských přístavů s až překvapivě nízkým počtem nápisů. Obecným trendem zůstává, že s přibývající vzdáleností od lidských sídel, narůstající nadmořskou výškou a vzdáleností od cest počet nápisů klesá.

\stopcomponent