
\environment ../env_dis
\startcomponent section-rozmístění-nápisů-v-závislosti-na-jejich-typologii
\section[rozmístění-nápisů-v-závislosti-na-jejich-typologii]{Rozmístění nápisů v závislosti na jejich typologii}

Rozmístění nápisů a jejich zařazení dle společenské funkce přináší nový pohled na roli, jakou nápisy v antické Thrákii hrály. Rozmístění veřejných nápisů odpovídá roli, jakou hrály byrokratické instituce a infrastruktura vytvářená komplexní společností v době existence římské provincie. Soukromé nápisy mohou existovat nezávisle na tomto uspořádání, nicméně k jejich rozšíření ve větším měřítku dochází právě v koexistenci s prostředím raného státu s centralizovanou mocí a organizovaným řízením. Proto se i soukromé nápisy nacházejí v okolí center či oblastí se zvýšenou mírou společenské a politické provázanosti, jakou obyvatelům poskytují města. Naopak venkov je epigraficky málo aktivní, a to i v římské době, a epigrafické aktivity souvisí spíše s iniciativou jedinců než obecným přístupem venkovské populace.

\subsection[veřejné-nápisy-jako-produkt-komplexní-společnosti]{Veřejné nápisy jako produkt komplexní společnosti}

Rozmístění veřejných nápisů odpovídá funkci, kterou hrály v rámci organizace společnosti, jako např. uchovávání a rozšiřování informací důležitých pro chod společnosti a udržení pořádku. Veřejné nápisy sloužily nejen jako zdroj informací, ale také samy byly projevem politické moci daného subjektu, a proto se vyskytují v hojné míře v administrativních a politických centrech, ať už regionálního či nadregionálního charakteru (Tainter 1988, 99-106).

\in{Mapa}[Apendix2:::7.07a] v \in{Apendixu}[Apendix2:::Apendix2] velmi dobře ilustruje rozmístění veřejných nápisů a jejich vzdálenosti od městských center, případně cest. Z celkem 695 veřejných nápisů pochází 79 \letterpercent{} ze vzdálenosti do 20 km, a zbývajících 21 \letterpercent{} nápisů ze vzdálenosti nad 20 km. Skupina nápisů nalezených na území ve vzdálenosti maximálního denního pochodu, tedy do 40 km od měst, představuje 92 \letterpercent{} všech nápisů. Zbylých 8 \letterpercent{} nápisů pochází ze vzdálenosti větší než 40 km od měst a nachází se většinou v přímé blízkosti silnic. Obsah této menší skupiny milníků a stavebních nápisů nejčastěji souvisí právě s údržbou římských silnic a vytyčováním vzdáleností.

Jedním z příkladů projevů politické autority a existující infrastruktury v krajině je rozmístění milníků, které označovaly vzdálenost k významnému městu v římské době (Hollenstein 1975, 23-45; Madzharov 2009, 57-59). Zpravidla milník udával vzdálenost v mílích k městu, do jehož regionu spadala správa daného úseku silnice. Nápis byl umístěn viditelně vedle silnice tak, aby si každý cestující zjistil jak velkou vzdálenost mu zbývá ujít či ujet do daného města, ale nápis také nesl informace o tom, kdo cestu spravuje, případně se postaral o její postavení či opravy. Na \in{Mapě}[Apendix2:::7.07a] v \in{Apendixu}[Apendix2:::Apendix2] je velice dobře vidět, že nálezová místa milníků kopírují trasu známých římských cest.\footnote{Madzharov (2009, 57-59) udává celkový počet řeckých i latinských milníků z území dnešního Bulharska na 180 exemplářů. Mapa \in{Mapa}[Apendix2:::7.08a] v \in{Apendixu}[Apendix2:::Apendix2] zobrazuje pouze datované milníky, z nichž většina je psána řecky. Latinsky psané milníky nejsou z větší části zahrnuty do HAT databáze.} Nejvíce milníků pochází z tzv. {\em Via Diagonalis}, což byla jedna z nejvýznamnějších cest Balkánu, protože spojovala východní provincie se západem a sloužila k častému přesunu vojsk mezi lokalitou Singidunum, dnešním Bělehradem, a Byzantiem, dnešním Istanbulem (Jireček 1877; Madzharov 2009, 70-131). Desítky milníků se dochovaly z okolí města Serdica, Filippopolis a Augusta Traiana, pod jejichž správu vybrané úseky {\em Via Diagonalis} spadaly. První milník se objevil již v 1. st. n. l. v oblasti {\em Via Egnatia} na egejském pobřeží, nicméně stavební aktivity ve vnitrozemské Thrákii jsou dobře dokumentované až z 2. st. n. l. Nejvíce milníků pochází z 3. st. n. l., kdy docházelo k úpravám {\em Via Diagonalis} vzhledem k jejímu častému využití římskými vojsky (Hollenstein 1975, 27-41). Na přelomu 3. a 4. st. n. l. docházelo k omezení stavebních aktivit a odpovídá tomu i menší počet dochovaných milníků. Poslední milník se dochoval ze 4. st. n. l. z {\em Via Egnatia} v okolí Perinthu.

Z některých úseků pochází velké množství milníků, např. v okolí města Serdica, ale z velké části cest se nedochoval ani jeden milník. Tento fakt je možné přisuzovat stavu archeologických výzkumů, které se zaměřují na zkoumání osídlení, což vede k relativně málo známému systému římských cest v Thrákii. Výzkumy posledních let se začínají zaměřovat i na síť silnic, jako na důležitou součást provinciálního uspořádání, a v budoucnosti se tak dají očekávat nové objevy, které mohou přinést i nové milníky a související stavební nápisy.

\subsection[soukromé-nápisy-a-projevy-kulturních-zvyklostí]{Soukromé nápisy a projevy kulturních zvyklostí}

Rozmístění soukromých nápisů do velké míry sleduje podobné trendy jako veřejné nápisy: větší část nápisů se nachází ve vzdálenosti na úrovni jednodenního pochodu od měst, případně v lokalitách podél cest, avšak ve srovnání s veřejnými nápisy je to až o 15 \letterpercent{} nápisů méně. Ve vzdálenosti do 20 km od měst se totiž nachází 64 \letterpercent{} nápisů, zbývajících 36 \letterpercent{} je ve vzdálenosti větší než 20 km. Pokud bereme v potaz vzdálenost do 40 km, pak se v tomto rozsahu nachází 81 \letterpercent{} z 3440 soukromých nápisů a zbylých 19 \letterpercent{} je vzdáleno od měst více než 40 km.

Soukromé nápisy se objevují v úzkém pobřežním pásu do vzdálenosti zhruba 20 km na pobřeží Egejského a Marmarského moře, jak je patrné z \in{Mapy}[Apendix2:::7.09a] v \in{Apendixu}[Apendix2:::Apendix2]. Na pobřeží Černého moře nápisy plynule přechází z pobřeží do vnitrozemí, což může být vysvětleno absencí pohoří, které by bránilo jejich rozšíření dále do vnitrozemí, podobně jako v případě nápisů pocházejících z území podél egejského pobřeží. Ve vnitrozemí se nápisy nachází zejména v okolí {\em Via Diagonalis} a na ní ležících měst Filippopolis a Serdica, nicméně zvýšené koncentrace soukromých nápisů můžeme pozorovat téměř v okolí všech měst s výjimkou Bizyé. V případě Bizyé lze však absenci nápisů vysvětlit nedostatkem publikovaného materiálu, a nikoliv negativním postojem obyvatelstva vůči zhotovování nápisů pro soukromé účely. Na rozdíl od veřejných nápisů soukromé nápisy pocházejí i z venkovských oblastí, které nesousedí s městem a ani v jejich blízkosti neprochází žádná z hlavních římských cest. Přítomnost soukromých nápisů v rurálním kontextu je tak možné vysvětlit jako důsledek pohybu obyvatelstva či adopce epigrafických zvyklostí v omezené míře i ve venkovském prostředí.

Dvě nejčastější společenské funkce, jakou soukromé nápisy zastávaly byla funkce funerální a dedikační. Rozmístění soukromých nápisů dle jejich typologie přináší zajímavé poznatky o chování tehdejšího obyvatelstva a šíření kulturních zvyklostí mezi jednotlivými komunitami.

\subsubsection[funerální-nápisy-19]{Funerální nápisy}

Rozmístění funerálních nápisů reflektuje odlišné zvyklosti, respektive odlišnou demografickou strukturu obyvatelstva pobřežní a vnitrozemské Thrákie. Jak je patrné z \in{Mapy}[Apendix2:::7.10a] v \in{Apendixu}[Apendix2:::Apendix2], v případě pobřežní Thrákie se funerální nápisy nacházejí převážně přímo ve městech či v nejbližším okolí. Ve vnitrozemí pak nalézáme funerální nápisy nejen ve městech, ale i v oblastech podél cest mimo region měst, případně na venkově mimo dosah cest. Z celkem 1631 funerálních nápisů se jich 87 \letterpercent{} nachází ve vzdálenosti do 20 km od měst a 13 \letterpercent{} ve vzdálenosti větší než 20 km. Do skupiny nápisů nalezených ve vzdálenosti od měst v délce denního pochodu, tedy 40 km, spadá 97 \letterpercent{} nápisů a jen 3 \letterpercent{} byla nalezena ve vzdálenosti větší než 40 km. Z toho plyne, že většina funerálních nápisů se nacházela v okolí měst či přímo ve městech. V případě vzdálenosti nálezových míst od trasy cest bylo 80 \letterpercent{} nápisů nalezeno ve vzdálenosti do 5 km od cesty, 20 \letterpercent{} nápisů ve vzdálenosti větší než 5 km. Pokud tuto vzdálenost změníme na 10 km od cesty, počet nápisů naroste na 84 \letterpercent{} všech nápisů nalezených do vzdálenosti 10 km a 16 \letterpercent{} ve vzdálenosti nad 10 km. V případě vzdálenosti 20 km tento poměr naroste na 97 \letterpercent{} nápisů ve vzdálenosti do 20 km a 3 \letterpercent{} ve vzdálenosti nad 20 km. Z toho plyne, že většina funerálních nápisů se našla v bezprostřední blízkosti cesty. Jako příklad mohou sloužit archeologicky prozkoumané nekropole v okolí Nicopolis ad Istrum, které se nacházely v přímém sousedství hlavních cest vedoucích z města (Vladkova 2016, 410-411).

Tomuto faktu odpovídá i obsah celé řady nápisů nalezených v okolí cest. Nápisy promlouvají k okolo jdoucím poutníkům či kupcům ({\em parodeita}), což předpokládá umístění náhrobních kamenů v přímé blízkosti frekventovaných cest. Nápis {\em IG Bulg} 3,2 1022 dokonce promlouvá specificky k neznámému poutníkovi, který projíždí v blízkosti hrobky s osly naloženými zbožím, aby uctil a oplakal Kainia, syna Mithridatova a Chrésté tak, jak se sluší mezi lidmi.\footnote{Podobná reflexe pohřební zvyklostí a rozmístění hrobek v přímé blízkosti silnic se objevuje v latinsky psané literatuře (Propertius 2.1.75-8; 2.11.5-6; 3.16.25-30; Ovidius {\em Tristia} 3.3.73-6 užívají shodně termín {\em viator}ů; Varro {\em De lingua Latina} 6.49), což naznačuje na běžné rozšíření tohoto zvyku v římské době napříč římským impériem.} Zvyk umísťovat náhrobky podél silnic je dobře znám i z jiných míst římské říše, např. v okolí Říma byly nekropole umísťovány podél známé {\em Via Flaminia} (Carroll 2006, 48-58). Termín {\em parodeita} (či jeho podoba {\em parodita}) se objevuje celkem na 116 nápisech, z čehož 106 nápisů se nachází ve vzdálenosti do 5 km od nejbližší známé trasy římské silnice. Většina nápisů s termínem {\em parodeita} se nachází na předpokládané trase silnic vedoucích z velkých městských center, a to ve vzdálenosti do 20 km od města. V Thrákii se náhrobní kameny obsahující termín {\em parodeita} nejčastější vyskytují na pobřeží Marmarského moře v okolí měst Perinthos, kde bylo nalezeno 53 nápisů. Dále nápisy pocházely zejména z Byzantia v téže oblasti a z pobřeží Černého moře z okolí Apollónie Pontské, Mesámbrie, Odéssu a Marcianopole. Až na jednu výjimku z 2. st. př. n. l. ({\em IK Byzantion} 102) pocházejí všechny datované nápisy promlouvající ke kolemjdoucím z římské doby. Rozmístění a obsah nápisů tak nepřímo reflektují existující zvyklosti spojené s pohřbíváním, které se v Thrákii rozšířily až v římské době.\footnote{Formule {\em chaire} či v plurále {\em chairete}, která přímo oslovuje čtenáře, se v Thrákii objevuje celkem na 332 nápisech. Poprvé se dochovala na nápisech datovaných do 3. až 2. st. př. n. l. a to ve čtyřech případech pocházejících z řeckých měst na pobřeží, které byly v kontaktu s římským světem a celým Středomořím (Abdéra, Maróneia, Byzantion). Ze 2. až 1. st. př. n. l. se dochovalo 44 nápisů s touto invokační formulí, které z převážné většiny pocházejí z Byzantia. V římské době došlo k rozšíření této formule na nápisy pocházející z celého území Thrákie.}

Pokud se podíváme na umístění funerálních nápisů v krajině, celkem 93 \letterpercent{} nápisů pochází z nížin do 273 m. n. m., čemuž odpovídá i průměrná nadmořská výška nálezových míst všech funerálních nápisů 81 m. n. m.\footnote{Pro srovnání průměrná nadmořská výška nálezových míst všech dedikačních nápisů je 311 m. n. m., viz níže.} Funerální nápisy tedy byly nalézány především v dobře přístupném terénu, v blízkosti měst a podél tras hlavních cest. Lidé byli pohřbíváni mimo centrum měst, nejčastěji v bezprostředním okolí hlavních cest vedoucích směrem z města ven. Rozmístění funerálních nápisů zcela reflektuje přirozený jev umisťovat pohřebiště v blízkosti lidských sídel, a nikoliv do odlehlých horských oblastí (Kurtz a Boardman 1973, 49-51; Damyanov 2010, 270).

Z četnosti výskytu funerálních nápisů na \in{Mapě}[Apendix2:::7.11a] v \in{Apendixu}[Apendix2:::Apendix2] zcela jasně vyplývá, že zvyk zhotovovat funerální nápisy převažuje v pobřežních oblastech, konkrétně ve městech Byzantion, Apollónia Pontská, Odéssos, Perinthos, Maróneia a Mesámbriá, původně řeckých koloniích. Z vnitrozemských center je to původně makedonské sídlo Hérakleia Sintská a Makedonci založená Filippopolis. Hustota výskytu funerálních nápisů ve vnitrozemí má však daleko nižší hodnoty, z čehož plyne, že zvyklost vytvářet a veřejně vystavovat nápisy je možné spojovat spíše se zvyklostmi řeckého kulturní okruhu a pouze částečně se rozšířila z řeckých měst na pobřeží do vnitrozemské Thrákie. Tomuto faktu odpovídá i datace funerálních nápisů. Z 6. až 1. st. př. n. l. pocházejí funerální nápisy převážně z pobřežních oblastí či z oblasti Hérakleie Sintské a lokality Didymoteichon, zatímco v období mezi 1. až 5. st. n. l. pocházejí funerální nápisy jak z pobřeží, oblasti toku řeky Strýmónu, tak i z oblasti centrální Thrákie. V době římské většina funerálních nápisů pochází z okolí silnic v blízkosti velkých městských center, což odpovídá i charakteru rozmístění náhrobních nápisů v jiných částech římské říše (Carroll 2006, 48-53).

\subsubsection[dedikační-nápisy-19]{Dedikační nápisy}

Dedikační nápisy jsou projevem náboženského přesvědčení jednotlivců i celých skupin a jejich rozmístění reflektuje normy chování v rámci jednotlivých komunit. Pobřežní Thrákie se vyznačuje velmi nízkým počtem dedikací, zatímco ve vnitrozemí se nachází několik oblastí s vysokou koncentrací dedikačních nápisů.

Jak je patrné z \in{Mapy}[Apendix2:::7.12a] v \in{Apendixu}[Apendix2:::Apendix2] dedikační nápisy se vyskytují v blízkosti měst a podél cest, podobně jako funerální nápisy. Ve vzdálenosti do 20 km od měst se nachází 43 \letterpercent{} nápisů, zbylých 57 \letterpercent{} nápisů pochází ze vzdálenosti vyšší než 20 km. Do skupiny nápisů nalezených ve vzdálenosti do 40 km od měst spadá 68 \letterpercent{} nápisů, zbylých 32 \letterpercent{} bylo nalezeno ve vzdálenosti větší než 40 km od měst. V porovnání s funerálními nápisy se dedikační nápisy vyskytují v okolí měst v menší míře a zejména pocházejí z venkovských a horských oblastí, které jsou vzdáleny od hlavních městských center.\footnote{Funerální nápisy mají poměr 87:13 u vzdálenosti 20 km (do 20 km vs. nad 20 km), a u vzdálenosti 40 km se tento poměr mění na 97:3.}

Největší koncentrace dedikací pocházejí z podhůří pohoří Rodopy, Pirin a Stara Planina v centrální Thrákii a z horských oblastí severozápadní Thrákie. Průměrná nadmořská výška místa nálezu dedikačního nápisu je 311 m. n. m., ale 21 nápisů bylo nalezeno dokonce ve výškách nad 764 m. n. m.\footnote{Pro srovnání průměrná nadmořská výška nálezových míst funerálních nápisů je 81 m. n. m., z čehož plyne, že dedikační nápisy byly nalézány ve vyšších horských polohách, případně na úpatí hor, viz dříve.} Téměř 80 \letterpercent{} nápisů pochází z oblastí do 509 m. n. m., tedy z oblasti nížin ve střední části Thrákie podél řeky Hebros, ale i z oblastí na úpatí hor a v údolí řeky Strýmónu. Dedikace pocházející z horských oblastí s nadmořskou výškou vyšší než 510 m. n. m. pocházejí zejména z oblasti okolo měst Serdica a Pautália s pohořími Vitoša a Stara Planina.

Dedikační nápisy a jejich výskyt v horských oblastech a podhůří poukazuje na udržení tradičního charakteru thráckého náboženství, které se vyznačovalo silným propojením s přírodními silami a často bylo v souladu s okolní krajinou (Janouchová 2013b, 10). Svatyně původně thráckých božstev byly nejčastěji umístěny ve volné přírodě a tento trend pokračoval i v době římské, z níž pochází většina dochovaných dedikačních nápisů.

\stopcomponent