
\environment ../env_dis
\startcomponent section-srovnání-epigrafické-produkce-s-archeologickými-daty
\section[srovnání-epigrafické-produkce-s-archeologickými-daty]{Srovnání epigrafické produkce s archeologickými daty}

Ač se na první pohled může zdát, že dochované nápisy nám poskytují informace o interakci jednotlivých komunit v plném rozsahu, opak je často pravdou a dochování nápisů může do velké míry dílem náhody. Nápisy se dochovaly pouze z části tehdy existujících lokalit, a dokonce i ne ze všech řeckých kolonií té doby. Archeologické výzkumy neprobíhají na všech místech stejnou měrou a některé lokality jsou lépe prozkoumané. Pokud by produkce nápisů byla rovnoměrná po celém území, pak by z lépe prozkoumaných lokalit mělo také pocházet více nápisů, čemuž tak je pouze v určitých případech.\footnote{Např. v případě Apollónie Pontské ve 4. st. př. n. l. či v Byzantiu ve 2. a 1. st. př. n. l., více v sekcích věnovaných jednotlivým stoletím v kapitole 6.} Jaký je poměr epigraficky aktivních měst vůči těm, které nápisy neprodukovaly? Nebo jinými slovy, na kolik jsou nápisy relevantním historickým zdrojem pro studium společnosti jako celku? Co tedy stojí za faktem, že z určitých lokalit pochází velké množství nápisů a z podobných lokalit, které jsou i do stejné míry prozkoumané, se dochovalo nápisů velmi málo či dokonce žádné? Na tyto otázky se pokusím alespoň nastínit odpověď na dvou příkladech srovnávajících epigrafická a archeologická data z území Thrákie.

Zásadním problémem, na nějž toto a podobná srovnání naráží, je nekompletní povaha archeologických, ale i epigrafických dat. Naše současné znalosti postihují pouze zlomek tehdy existujících lokalit, který je podobně nahodilý jako v případě nápisů. Do značné míry tak srovnání založené na souborech archeologických a epigrafických dat může být nepřesné a částečně zkreslené charakterem dostupných dat. Proto v žádném případě nelze zde uváděná čísla brát jako kompletní a neměnná, ale spíše jako orientační a shrnující aktuální stav našich znalostí. Hlavním cílem tohoto srovnání je ukázat, že epigraficky aktivní byla pouze velmi malá část tehdejší společnosti a nápisy pocházejí pouze ze zlomku tehdy známých lokalit, ač by jejich relativně velké počty na první pohled mohly svědčit o opaku.

\subsection[příklad-řeckých-měst-v-thrákii-v-7.-až-4.-st.-př.-n.-l.]{Příklad řeckých měst v Thrákii v 7. až 4. st. př. n. l.}

Částečné srovnání poměru epigraficky aktivních komunit na úrovni měst a komunit bez epigrafické aktivity mezi 7. a 4. st. př. n. l. umožňují data získaná z nedávno publikovaných kompendií archeologických lokalit jako Hansen {\em et al.} (2004) {\em An Inventory of Archaic and Classical Poleis}, které zaznamenává především řecká města na pobřeží a Talbert (2000) {\em Barrington Atlas of the Greek and Roman World}, zaznamenávající archeologické lokality jak na pobřeží, tak ve vnitrozemí.\footnote{Hansen {\em et al.} (2004) se primárně zaměřuje jen na řecká města archaického a klasického období. Celkové počty řeckých kolonií udávaných kolektivem autorů nereprezentují celkový počet všech existujících lokalit v daném století, a to již z podstaty historiografických a archeologických dat, které obsahují velkou míru nejistoty a nekompletnosti. Nicméně i přesto se jedná o jeden ze současných nejucelenějších souborů informací, který lze použít pro celkové srovnání. Jako další kontrolní soubor používám lokality z {\em Barrington Atlas of Greek and Roman World} \cite[Talbert2000], který je však obecně považován za méně přesný než Hansen {\em et al.} (2004). Talbert (2000) uvádí jak řecká osídlení, tak i osídlení thráckého původu, avšak nerozlišuje mezi jednotlivými stoletími, ale lokality zasazuje dohromady v rámci období, a proto se celkové číslo jeví jako dvojnásobně vyšší v případě 5. a 4. st. př. n. l., tedy klasického období.} Počet lokalit z těchto dvou zdrojů je nicméně vhodné chápat nikoliv jako kompletní výčet, ale spíše měřítko aktivit v dané oblasti a indikátor objevujících se trendů.

Tabulka 7.01 v Apendixu 1 poskytuje přehled počtu lokalit udávaných Hansenem {\em et al.} (2004) a Talbertem (2000) v jednotlivých stoletích a srovnává je s celkovým počtem epigrafických lokalit. Odhlédneme-li od problematiky konečných čísel lokalit a zaměříme se spíše na obecné trendy, je možné sledovat, že a) narůstající počet sídlišť odpovídá i narůstajícímu počtu epigrafických lokalit, b) sídlišť bez epigrafické produkce je několikanásobně více než sídlišť s dochovanou epigrafickou produkcí, a to zejména v 7. a 6. st. př. n. l. a v kontextu vnitrozemských lokalit. Z tohoto srovnání plyne, že epigraficky aktivní byla v nejlepším případě jen třetina všech známých sídlišť a zhruba dvě třetiny měst řeckého původu. Oproti původnímu očekávání nelze zvyk publikovat nápisy automaticky považovat za jednotnou charakteristiku všech řeckých osídlení na území Thrákie, ale i v rámci řeckých {\em poleis} se setkáváme s odlišným přístupem k epigrafické produkci, který se měnil v průběhu staletí.

Jednou z hlavních myšlenek hellénizačního přístupu v tradičním slova smyslu je, že se hellénizace společnosti projevuje jako uniformní ve všech řecky mluvících komunitách a jejich okolí. Epigrafická produkce je tradičně považována za jeden z projevů „hellénství” a je vnímána jako nedílný projev společensko-kulturního uspořádání řecky mluvících komunit. Jak je patrné z výše uvedeného, epigrafická produkce se v případě řeckých komunit na území Thrákie objevovala zhruba ve dvou třetinách komunit existujících v období od 7. do 4. st př. n. l., a proto nelze tyto dva fenomény spojovat bez dalšího vysvětlení a zařazení místních specifických podmínek v jednotlivých obcích.\footnote{Podrobněji se vybranými regiony se zabývám v rámci sekcí věnovaným jednotlivým stoletím v kapitole 6.} Pro období 7. až 4. st. př. n. l. tak epigrafická produkce figuruje spíše jako doplňující zdroj informací o tehdejší společnosti a je nutné přihlížet k archeologickým poznatkům, případně k historiografickým pramenům.

I přesto, že nápisy představují značně selektivní zdroj informací, v celkovém pohledu nabízí jedinečné srovnání jednotlivých regionů, a to i v období od 7. do 4. st. př. n. l. \cite[righttext={{, 80-82},{, 38-39}}][Woolf1998, Bodel2001]. Regionální rozdíly v epigrafické produkci mohou značit jednak výrazně jiný přístup komunity k epigrafické kultuře, související se situací v mateřském městě, nestabilní politickou situaci, nedostatek materiálu, či pouze reflektují náhodnost dochování nápisů. Nápisy se zpravidla nedochovaly z prvních let po založení kolonie, ale pochází z následujících let, ne-li desetiletí, kdy došlo ke stabilizaci politické situace, zajištění základních potřeb a bezpečí obyvatelstva. Stabilizace podmínek, za nichž mohly vznikat nápisy ve větší míře, tak přímo může souviset s nárůstem společenské komplexity a míry společenské organizace. Pokud totiž dojde k upevnění politické moci, dochází i zpravidla k ustálení produkce a zajištění nutné infrastruktury \cite[righttext={, 106-118}][Tainter1988].\footnote{O roli institucí, byrokratického aparátu na rozšíření nápisů podrobněji hovořím v kapitole 3.} V období prosperity je tak možné sledovat nárůst celkového počtu soukromých nápisů, a naopak v době nestability narůstají ve snaze regulovat tehdejší společnost celkové počty nápisů veřejných.

Tabulka 7.02 v Apendixu 1 dokazuje, že 5. a 4. st. př. n. l. představovalo dle narůstajících počtů archeologických i epigrafických lokalit a celkových počtů epigrafické produkce dobu relativní stability, zatímco ve 3. st. př. n. l. došlo k poklesu celkové epigrafické produkce, markantnímu nárůstu veřejných nápisů na třetinu celkového počtu a poklesu soukromých nápisů. Stejně tak došlo i k poklesu epigrafických lokalit, což nasvědčuje na destabilizaci situace v regionu v průběhu 3. st. př. n. l. Z historických zdrojů víme, že 3. st. př. n. l. byla Thrákie svědkem několika vojenských tažení hellénistických panovníků, invaze Keltů ze střední Evropy, která s sebou nesla i destrukce archeologických lokalit a vedla k přeměně společenského uspořádání Thrákie. Ve 2. st. př. n. l. naopak dochází k určití stabilizaci poměrů, což se projevuje i na zvýšené epigrafické produkci, nárůstu počtu lokalit a nárůstu soukromých nápisů. 1. st. př. n. l. naopak zaznamenává propad epigrafické produkce, pokles počtu epigrafických lokalit a soukromých nápisů téměř na poloviční hodnoty, zatímco veřejné nápisy se udržují na podobné úrovni jako ve 3. st. př. n. l. a představují zhruba čtvrtinu všech nápisů. V Thrákii v této době dochází k přeměně společenské struktury směrem k vazalskému království v područí Říma, což s sebou neslo míru nejistoty, která se odrazila i na epigrafické produkci.

\subsection[příklad-z-thráckého-vnitrozemí-kazanlacké-údolí-v-době-římské]{Příklad z thráckého vnitrozemí: Kazanlacké údolí v době římské}

Jiný druh srovnání archeologických a epigrafických dat nabízí následující příklad z vybraného regionu vnitrozemské Thrákie, Kazanlackého údolí. Srovnání na mikro-regionální úrovni umožňuje porovnat poměr všech známých archeologických lokalit s epigrafickými lokalitami, což je jen velmi těžko dosažitelné na úrovni makro-regionální jako v případě srovnání všech známých lokalit ze 7. až 4. st. př. n. l. Proto jsem jako příklad zvolila dobře ohraničené území ve vnitrozemské Thrákii a zaznamenala všechny známé archeologické lokality a porovnala je se známými místy nálezů nápisů.\footnote{Pro tyto účely jsem zvolila oblast Kazanlackého údolí ve střední části Bulharska. Oblast okolo Kazanlaku je známá jako kulturní a historické centrum thráckých panovníků, kteří udržovali čilé kontakty s řeckými obcemi. V letech 2009-2011 zde probíhaly povrchové sběry projektu {\em The Tundzha Regional Archaeological Project} (TRAP; Sobotkova {\em et al.} 2010; Ross {\em et al.} v přípravě, vyjde 2017), během nichž se podařilo zmapovat osídlení ve větší části Kazanlackého údolí. Výsledky těchto sběrů tak představují výborný výchozí soubor dat, pokrývající větší část vybraného regionu v jeho komplexnosti a zaznamenávají jak viditelné monumenty, tak i koncentrace keramiky a architektonických prvků na povrchu. Pro úplnost porovnávám data i se soupisy archeologických lokalit, které byly pořízeny pro Kazanlak v roce 1991 (Domaradzki 1991; Tabakova-Tsanova 1991) a výstupy archeologických vykopávek na zkoumaném území (Chichikova, Dimitrov a Alexieva 1978; Tabakova 1959; Tabakova-Tsanova 1961, 1980; Dinchev 1997; Nekhrizov {\em et al.} 2013). Toto srovnání vychází ze studie, která vyjde v roce 2017 rámci sborníku z projektu {\em The Tundzha Regional Archaeological Project} (Janouchová v přípravě, vyjde 2017).}

Projekt {\em The Tundzha Regional Archaeological Project}, jehož závěry používám jako výchozí data, zaznamenal v průběhu let 2009 až 2011 celkem 82 archeologických lokalit a 773 mohyl na území o rozloze 85 km\high{2} (Sobotková v přípravě, vyjde 2017). Pro dobu pozdně železnou \cite[500 -0] bylo nalezeno 38 lokalit a pro dobu římskou \cite[1 -400] 23 lokalit, což v průměru znamená výskyt jedné lokality na 2,23 km\high{2} v době pozdně železné a na 3,69 km\high{2} v době římské. Jak je patrné z tabulky 7.03 v Apendixu 1, z vybraného území se dochovalo celkem 43 nápisů, z nichž osm spadalo do doby pozdně železné a bylo nalezeno ve čtyřech lokalitách s centrem v Seuthopoli, hellénistické rezidenci odryského panovníka Seutha (Dimitrov, Chichikova a Alexieva 1978, 3-5). Tyto nápisy poukazují na prominentní roli Seuthopole a zcela ojedinělý přístup k publikaci nápisů a řecké kultuře obecně, který panovník Seuthés zaujímal (Janouchová v přípravě, vyjde 2017). Nápisy z Kazanlackého údolí v hellénistické době pocházely pouze z kontextů spojených s panovníkem Seuthem a zvyk veřejně vystavovat nápisy tesané do kamene, jak je obvyklé v řeckých komunitách, po jeho smrti postupně vymizel z thráckého prostředí na několik dalších století. Nicméně povědomí o užívání písma přetrvalo alespoň v prostředí thrácké aristokracie v mírně změněné formě, která předměty nesoucí nápisy využívala pro soukromé účely, jakožto prestižní předmět, zdůrazňující jejich společenský status (Sahlins 1963; Whitley 1991, 349-350).

Z doby římské pocházelo 35 nápisů z pěti lokalit, z čehož většina nápisů pocházela ze svatyní Apollóna, nesoucího místní přízviska {\em Teradéenos} a {\em Zerdénos}, umístěných v okolí moderní vesnice Kran \cite[righttext={{, 97-104},{, 173-194}}][Tabakova1959, Tabakova-Tsanova1980]. Z toho plyne, že ze známých lokalit doby pozdně železné 10,5 \letterpercent{} obsahovalo minimálně jeden nápis. V době římské se tento poměr zvýšil na dvojnásobek na 21,7 \letterpercent{}. Pokud vezmeme v potaz charakter jednotlivých epigrafických lokalit, v římské době většina nápisů z oblasti pozemních sběrů v rámci projektu TRAP pochází ze svatyní Apollóna nesoucího místní přízvisko. Většina nápisů má soukromý charakter a jejich zhotoviteli jsou osoby nesoucí převážně thrácká či kombinovaná římská a thrácká jména, což poukazuje na jejich zapojení v římské armádě a samosprávě měst (Janouchová v přípravě, vyjde 2017). V římské době tedy nárůst počtu epigrafických lokalit odpovídá i většímu zapojení místního obyvatelstva do epigrafické produkce, která i přesto zůstávala nadále poměrně nízká.

\stopcomponent