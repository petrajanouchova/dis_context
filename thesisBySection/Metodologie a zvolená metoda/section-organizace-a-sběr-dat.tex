
\environment ../env_dis
\startcomponent section-organizace-a-sběr-dat
\section[organizace-a-sběr-dat]{Organizace a sběr dat}

V roce 2012 jsem dělala průzkum dostupných elektronických databází a nástrojů, které by mi pomohly zpracovat vybrané téma. Databáze dostupné na internetu byly tehdy relativně omezené a nenabízely tak široké spektrum funkcí jako v roce 2017, zejména co se týče exportu, souvisejících geografických dat. Dalším zásadním omezením bylo, že data zvolená k výzkumu pocházela z mnoha zdrojů, zpracovaných více či méně detailně, a vyskytovala se především pouze v tištěné formě. Rozhodla jsem se tedy, že data budu sbírat sama, a to metodou digitalizace tištěných korpusů a jiných zdrojů relevantních dat. Kde bylo možné využívat již digitalizovaná data, jako např. {\em Barrington Atlas of Greek and Roman World}, {\em Searchable Greek Inscriptions}, {\em Supplementum Epigraphicum Graecum} atp., postupně jsem je zapojovala do návrhu projektu, v závislosti na jejich dostupnosti.\footnote{V pilotní fází projektu jsem sbírala data do několika jednoduchých tabulek v rámci MS Excel, nicméně jsem brzy zjistila, že je tento přístup neefektivní a velice náchylný k chybovosti, zejména v rámci týmové spolupráce. V roce 2012 jsem tedy vytvořila relační databází v MS Access. V podstatě se jednalo o jednoduchý model tabulek spojených pomocí relací (klíčů), který pomohl zaznamenávat data v normované podobě (tedy neduplikující se záznamy). V roce 2013 jsem obdržela dvouletou finanční podporu od Grantové Agentury Univerzity Karlovy \cite[GAUK2013] a mohla jsem zapojit do projektu studenty magisterského a doktorského studia na FFUK (Markéta Kobierská, Jan Ctibor, a Barbora Weissová). V té době jsem byla na studijní stáži v Austrálii na University of New South Wales, a ostatní členové týmu byli taktéž studijně v různých částech Evropy. Databáze v MS Access v roce 2013 ještě nepodporovala cloudové řešení, a nebylo v mých tehdejších technických možnostech zajišťovat její serverový hosting. Proto jsem se rozhodla pro jiné řešení, které by umožnilo přístup přes internet, a které by navíc nebylo závislé na používaném operačním systému (Windows, Linux, Mac).}

\stopcomponent