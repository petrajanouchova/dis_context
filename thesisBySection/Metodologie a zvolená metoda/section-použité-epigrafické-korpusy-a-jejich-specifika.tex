
\environment ../env_dis
\startcomponent section-použité-epigrafické-korpusy-a-jejich-specifika
\section[použité-epigrafické-korpusy-a-jejich-specifika]{Použité epigrafické korpusy a jejich specifika}

Epigrafické památky nalezené na území, které je tradičně označováno jako Thrákie, tedy oblast dnešního Bulharska, severního Řecka a evropského Turecka, jsou poměrně hojné. Nalézají se zde jak nápisy psané řecky, latinsky\footnote{Celkový počet latinsky psaných nápisů nalezených na území Thrákie není přesně zmapován, nicméně se odhaduje, že jich je zhruba třikrát méně než nápisů psaných řecky. Milena Minkova (2000, 1-7) předpokládá existenci zhruba 1200-1300 latinských nápisů na území dnešního Bulharska. V porovnání se zhruba 3000 řecky psanými nápisy ze stejného území dojdeme k poměru 70:30 právě ve prospěch řeckých nápisů \cite[righttext={, 437}][Janouchová2016]. Podrobněji se vyjadřuji k problematice zvoleného publikačního jazyka v kapitole 5.}, bilingvně řecko-latinsky\footnote{Diskuze k řecko-latinským nápisům je součástí kapitoly 5.}, tak existuje i několik málo nápisů klasifikovaných jako psaných thrácky\footnote{Tomaschek 1893; Detschew 1952; 1957; Dimitrov 2009. S největší pravděpodobností thráčtina představovala indoevropský jazyk, podobný řečtině, nicméně nedochoval se dostatek písemného materiálu a pramenů, aby bylo možné thráčtinu blíže charakterizovat. Do dnešní doby bylo nalezeno celkem 6 nápisů \cite[righttext={, 1-19}][Dimitrov2009], a zhruba 1400 osobních a místních jmen, které se vyskytují v řeckých a latinských nápisech \cite[righttext={, 25}][Dana2011]. I když probíhají pokusy o překlad těchto nápisů, přesvědčivá a všemi přijímaná interpretace stále ještě čeká na své objevení.}. Předmětem této studie jsou zejména nápisy psané řecky, jakožto nejpočetnější dochovaný soubor nápisů, který představuje statisticky nejrelevantnější vzorek s nejvyšší reprezentativní hodnotou, ač k nápisům psaným jinými jazyky v daných situacích přihlížím též.

Epigrafické prameny jsem pro usnadnění zpracování velkého počtu dat shromáždila do databáze {\em Hellenization of Ancient Thrace}. Databáze obsahuje data z 10 ucelených epigrafických korpusů řecky psaných nápisů, které pocházejí či byly nalezeny na území Thrákie. Dále databáze obsahuje epigrafická data z menších korpusů a jednotlivých článků, které souhrnně spadají pod jedenáctou položku {\em Other}, tedy jiný zdroj.\footnote{Kompletní seznam všech nápisů, které jsou součástí databáze se nachází v Apendixu 3}

Většina korpusů je vydávána v rámci jednotlivých moderních států a nezaměřuje se na oblast Thrákie jako celku. Pro oblast moderního Bulharska používám jako hlavní zdroj informací v latině publikovaná nápisná korpora {\em Inscriptiones Graecae in Bulgaria repertae} ({\em IG Bulg}; Mihailov 1956, 1958, 1961, 1964, 1966, 1970, 1997), které zahrnují pouze území moderního Bulharska. Tato korpora představují nejucelenější sbírku řeckých nápisů z~mnou zkoumané oblasti až s~téměř 3000 exempláři. Nápisy jsou opatřeny dobrou fotografickou dokumentací a relativně dobrým popisem. Vzhledem k~faktu, že některé knihy již byly vydány před více jak 50 lety, jsou některé informace, dle moderních principů epigrafických publikací zastaralé, či nedostatečně detailní. Tento fakt jim však neubírá na výpovědní hodnotě a stále představují cenný zdroj historických informací. Mihailov editoval většinu nápisů sám, a tak není divu, že při tak velkém počtu nápisů došlo i k~chybnému vydání některých z~nich. Tyto nedostatky se však sám autor snažil napravit v~5. svazku \cite[Mihailov1997], a ve většině případů se mu to i zdařilo.

Nápisy publikované v Bulharsku po roce 1997 spadají do kategorie {\em Other}, která obsahuje jak nápisy publikované v rámci {\em Supplementum Epigraphicum Graecum} ({\em SEG}) od r. 1996 do r. 2010 \cite[vydané v roce2015], tak v rámci menších korpusů (Velkov 1991; Manov 2008), či v rámci jednotlivých článků (Velkov 2005; Gyuzelev 2002; 2005; 2013).\footnote{Nápisy bývají často publikovány v lokálních časopisech, které jsou obtížné dostupné, a proto je možné, že některé nápisy v databázi zatím zaznamenány nejsou, avšak maximálně v řádech několika desítek exemplářů. Tento fakt by však neměl nijak ubrat na hodnotě již dosud sesbíraných nápisů a jejich celkové relevanci pro vybraný druh analýzy.}

Oblast egejské části severního Řecka pokrývá nedávno v řečtině publikovaný soubor {\em Epigrafes tis Thrakis tou Aigaiou} ({\em I Aeg Thrace}; Loukoupoulou {\em et al.} 2005). Tato publikace pokrývá oblast mezi řekami Hebros a Néstos a jedná se o soubornou publikaci epigrafických památek jižní části antické Thrákie, rozkládající se na území dnešního Řecka. Soubor 500 nápisů pokrývá nápisy z~hlavních řeckých měst regionu, jako je Abdéra, Maróneia, Topeiros, Zóné, Drys a jejich bezprostřední okolí. Publikace zprostředkovává i kvalitní obrazovou dokumentaci, nutnou pro účely disertační práce. Korpus zahrnuje nalezené nápisy do r. 2005, novější nápisy je taktéž možno nalézt v {\em SEG}.

Další zdroje nápisů pokrývají oblast jižní Thrákie nacházející se na území dnešního Turecka s~celkovým počtem okolo 700 nápisů. Jedná se o jediné tři ucelené soubory nápisů z~této oblasti, které zpracovávají nápisy z řeckých měst na pobřeží a jejich bezprostředního okolí ({\em IK Sestos}, Krauss 1980 pro oblast Séstu a Thráckého Chersonésu; {\em IK Byzantion}, Lajtar 2000 pro oblast Byzantia; {\em Perinthos-Herakleia}, Sayar 1998 pro oblast Perinthu a okolí). Některé publikace jsou až 30 let staré a pro nově nalezené nápisy v~těchto oblastech je opět nutné konzultovat SEG. Problémem napříč všemi moderními zeměmi zůstávají nápisy nepublikované, které nemohu do své analýzy zahrnout. Jednak je to případ tureckých měst, kde velká část nápisů zůstává nepublikovaná v muzeích, či v sekundárním kontextu, ale jedná se například i o objekty z vykopávek, jako příklad graffit z~antické Seuthopole či Kabylé, které na svou zevrubnou publikaci stále ještě čekají mnohdy i 60 let po svém objevení.\footnote{Informace poskytnutá L. Domaradzkou v roce 2012.}

\stopcomponent