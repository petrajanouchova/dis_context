
\environment ../env_dis
\startcomponent section-statistické-zpracování-epigrafických-dat
\section[statistické-zpracování-epigrafických-dat]{Statistické zpracování epigrafických dat}

Nápisy představují nejen výborný sociologický a antropologický materiál pro studium antické společnosti, ale pokud se k nim přistupuje komplexně, mohou odhalit nové pohledy na tehdejší společnost, které by bylo jen velmi obtížné získat studiem jednotlivých nápisů. Statistické zpracování většího počtu nápisů s sebou nese celou řadu metodologických problémů. Epigrafická data jsou svou podstatou nedokonalá, často nekoherentní a tudíž velmi specifická. Z tohoto důvodu dochází ke statistickému zpracování nápisů relativně málo často (MacMullen 1982; Meyer 1990; Saller a Shaw 1984; Prag 2002; Prag 2013; Korhonen 2011).

I přes neoddiskutovatelný potenciál statistického zpracování nápisů existuje velké množství přístupů, které nereflektují problematickou a často nekoherentní povahu informací získaných z nápisů, a může tak docházet k jejich zkreslení, či nepochopení. V následující pasáži se nejdříve věnuji míře nejistoty datace, která bývá často opomíjená v rámci epigrafických studií, a dále metodě výběru statisticky signifikantního souboru nápisů vhodných k temporální analýze celospolečenských trendů. V neposlední řadě poukazuji na problém nadhodnocování celkových čísel epigrafické produkce při statistickém zpracování nápisů v jednotlivých stoletích a nabízím řešení za využití metody normalizované datace.

\subsection[pravděpodobnost-a-míra-nejistoty-datace-nápisů]{Pravděpodobnost a míra nejistoty datace nápisů}

Jednotlivé nápisy bývají datovány s přesností na konkrétní roky poměrně zřídka, ale spíše častěji bývají datovány s přesností několika století či spadají pouze do časového období přesahující několik staletí, jako např. hellénismus, klasická doba, římská doba atp. Tento nejednotný styl datování, který se různí i dle osobního přístupu editorů epigrafických korpusů, velmi znesnadňuje jakékoliv statistické výpočty a srovnávání epigrafické produkce v průběhu několika století. V rámci hledání řešení této nelehké situace jsem se inspirovala v oblasti archeologických polních sběrů, nabízejících možná metodologická řešení míry nejistoty datace.

Jednotlivé lokality bývají datovány na základě studia nalezené keramiky. Ta je zařazena do chronologických skupin dle výskytu charakteristických tvarových prvků a použitých technologií výroby a zpracování. V případě, že je keramika nekvalitní, nediagnostická, či špatně dochovaná, a tedy charakteristické prvky neumožňují jasnou dataci, archeologové přišli s řešením tohoto problému nejistoty datace (Dewar 1991; Bevan a Conolly 2006; Crema 2012; Bevan {\em et al.} 2013; Sobotková 2018). Při statistickém zpracování přiřazují jednotlivým střepům procentuální vyjádření pravděpodobnosti dané datace, tedy jakýsi koeficient pravděpodobnosti, zda daný střep spadá do dané chronologické skupiny (např. Bevan {\em et al.} 2013; Crema 2012). Na základě analýzy pravděpodobnosti datace keramických střepů a zohlednění těchto výsledků pak mohou archeologové vypočítat přibližnou dobu trvání a celkový počet lokalit v daném období. Jednotlivá chronologická období jsou vnímána jako na sebe navazující kontinuum, a nikoliv striktně oddělené kategorie, a na prvek nejednoznačnosti a nejistoty datace je nahlíženo jako na jedno z hodnotících kritérií. Tímto způsobem je možné daný předmět analyzovat komplexněji a sledovat provázanost mezi jednotlivými chronologickými kategoriemi. Tato v mnohém inovativní metoda navíc umožňuje zahrnout materiál, který by byl dříve pro svou nejednoznačnost z celkové analýzy vyloučen, a tím pádem by by mohlo dojít ke ztrátě dat a ke zkreslení interpretací.

Studie zabývající se statistickým zpracováním nápisů v dlouhém časovém intervalu jsou poměrně málo časté, vzhledem k nedostatku relevantních korpusů a problematické povaze datace nápisů. Při statistickém zpracování nápisů totiž nevyhnutelně dochází k určité redukci intervalu datace, což je nezbytné vzhledem k nutnosti zpracování velkého počtu dat a jejich prezentace v jednotném a koherentním formátu. Na rozdíl od archeologické aplikace pravděpodobnosti a temporální nejistoty, která především řeší dobu trvání jednotlivých lokalit a vztahy s ostatními lokalitami datovanými do téže doby, použití této metody v epigrafice se v několika zásadních hlediscích odlišuje. Jednak tím, že datace nápisu totiž udává nejpravděpodobnější interval, v němž nápis vznikl, a neudává dobu existence nápisu (která by ve valné většině případů trvala až do současnosti). Nápis tak má, na rozdíl od archeologických lokalit, výpovědní hodnotu vztahující se k době vzniku, nikoliv však k době své existence, a proto se částečně liší i přístup k nejistotě jeho datace. Dalším zásadním rozdílem je, že v archeologickém použití této metody jde především o odhad počtu současně existujících lokalit, avšak už se většinou dále nepracuje na úrovni jednotlivých lokalit a s nimi spojených kvalitativních prvků, jako je tomu u studia nápisů (Dewar 1991, 604). Nápisy navíc nebývají v epigrafických korpusech datovány s procentuálním vyjádřením pravděpodobnosti datace do jednotlivých století, ale interval datace má stejnou pravděpodobnost po celou dobu trvání, ať už se jedná o rok, deset let či pět století. Povaha samotných dat tedy neumožňuje použít metodu ve formě jakou navrhuje např. Bevan {\em et al.} 2013; Crema 2012, ale je nutné ji upravit a částečně zjednodušit tak, aby odpovídala povaze epigrafických dat a metodologii datování nápisů.

Metoda definování pravděpodobnosti datace upravená pro epigrafické prostředí umožňuje nápisy rozdělit do jednotlivých časových období s přihlédnutím k možné variabilitě a poměrně širokému rozptylu jejich intervalu datace. Jinými slovy, metoda zohledňuje míru nejistoty datace a pravděpodobnost doby vzniku nápisu v rámci jednotlivých století za účelem vytvoření nejrelevantnějšího souboru nápisů pro studium konkrétních trendů v rámci daného století.\footnote{Tato metoda vznikala nezávisle na statistických studiích zabývajících se epigrafickými prameny z jiných regionů (Prag 2002; Prag 2013; Korhonen 2011), avšak základní principy a řešení nejistoty datace se do velké míry shodují.} V oblasti epigrafiky podobnou metodologii uplatňuje např. Jonathan Prag (2002; 2013) u statistických studií nápisů nalezených na Sicílii, kde rovněž postihuje období delší než deset století. Tato metoda umožňuje zařadit nápisy do kategorie jednotlivých století s přihlédnutím k celkovému intervalu doby jejich vzniku ({\em terminus post quem} a {\em terminus ante quem}). Dle dostupných údajů o dataci jsem vypočetla tzv. koeficient pravděpodobnosti datace, který zaznamenává míru pravděpodobnosti, s níž daný nápis vznikl v průběhu daného století.\footnote{Pro lepší pochopení udám konkrétní příklad: nápis je editorem datován do 2.-3. st. n. l., což znamená, že v rámci tradičního přístupu by byl nápis započítán jak v rámci statistik pro 2. st. n. l., tak v rámci 3. st. n. l. Stejně tak by ve statistikách nebyl rozdíl mezi nápisem datovaným do 2.-3. st. n. l. a nápisem datovaným do 2. st. n. l. (tj. nápis datovaný do 2. a 3. st. n. l. by měl stejnou váhu pro interpretaci epigrafické produkce ve 2. st.n. l. jako např. nápis datovaný pouze do 2. st. n. l.). Tyto rozdíly v šíři datace jsem považovala za nutné oddělit a proto jsem vytvořila tzv. koeficient pravděpodobnosti datace. U nápisů datovaných do více než jednoho století jsem tedy plnou hodnotu pravděpodobnosti (1=100 \letterpercent{}) vydělila počtem století, do nichž byly nápisy datovány a vyšel mi koeficient pravděpodobnosti datace pro dané století. U nápisů datovaných do dvou století byl tak koeficient 0,5 pro každé z obou století, u nápisů datovaných do tří století byl koeficient 0,33 pro každé ze třech století. U nápisů datovaných do čtyř století byl koeficient 0,25, atd.} Tímto způsobem jsem rozlišila nápisy, které jsou datovány do jednoho století, a obsahují tak statisticky přesnější data o trendech jednotlivých století, od nápisů jejichž datace přesahuje do několika století, u nichž je tak těžší sledovat vývoj měnících se parametrů v čase, a jejichž výpovědní hodnota ke konkrétnímu časovému okamžiku je relativně nízká.

Průměrná délka časového intervalu datace všech datovaných nápisů je 114,6 let (aritmetický průměr); medián je 99 let, což přibližně odpovídá dělení na století, které jsem zvolila jako základní chronologické dělení pro další statistické analýzy.\footnote{Použitý R skript je součástí digitálního Apendixu, dostupného na adrese \useURL[url18][https://github.com/petrajanouchova/hat_project][][{\em https://github.com/petrajanouchova/hat_project}]\from[url18]. Aritmetický průměr ({\em mean}) je součtem všech hodnot, vydělený celkovým počtem prvků. Medián udává střední hodnotu souboru vzestupně seřazených hodnot a dělí tak soubor na dvě stejně početné poloviny. Hodnota mediánu není zpravidla ovlivněna extrémními hodnotami, např. velmi se odlišujícími maximálními či minimálními hodnotami, na rozdíl od aritmetického průměru, a lépe tak poukazuje na střední hodnoty daného souboru. Pro srovnání udávám vždy aritmetický průměr a medián daného souboru.} Z grafu 4.08a v Apendixu \in[Apendix2:::Apendix2] můžeme vidět jasné tendence přiřazování datace napříč všemi datovanými nápisy (2276 nápisů): a) zařadit nápis co nejpřesněji, v rámci intervalu menšího než 20 let, b) s přesností na jedno století, tedy intervalu menšímu či rovnajícímu se 100 letům, c) s přesností na dvě století, tedy intervalu menšímu či rovnajícímu se 200 letům. Omezený počet nápisů je datován s přesností na tři století, případně na pět, avšak tato datace je již příliš široká pro studii vývoje v rámci jednotlivých století. Procento všech datovaných nápisů, jejichž interval datace, tedy období vzniku nápisu ohraničené StartYR a EndYR v databázi, byla pro interval 1-50 let 35,15 \letterpercent{}, pro interval 51-100 let 30,13 \letterpercent{}, a pro interval 101-704 let 34,63 \letterpercent{}. Pro účely temporální studie v kapitole 6 jsem vybrala pouze nápisy, které jsou datovány v rozmezí jednoho století (jimž jsem přidělila koeficient 1), a nápisy datovány do rozmezí dvou století (koeficient 0,5).\footnote{Nápisy, které se podařilo datovat jako relativně římské, do analýzy v kapitole 6 nezařazuji, zejména vzhledem k jejich vágní dataci, a tedy malé výpovědní hodnotě v rámci temporální studie. Stejně tak nápisy, jejichž šíře datace přesahuje dvě století.} Tyto dvě skupiny dohromady reprezentují 89,5 \letterpercent{} (2036 nápisů) všech datovaných nápisů, které však zcela neodpovídá kategoriím jednotlivých staletí, ale navzájem se částečně překrývají, a proto o nich pojednávám v kapitole 6 jako o navzájem oddělených kategoriích.

Pokud však chceme srovnat celkovou míru epigrafické produkce v jednotlivých staletích, je nutné přistoupit k metodě normalizované datace, která zobrazuje nejpravděpodobnější procentuální zastoupení epigrafické produkce v daném století. Graf 4.09a v Apendixu \in[Apendix2:::Apendix2] ilustruje poměr datovaných nápisů v jednotlivých stoletích s přihlédnutím k šíři jejich datace (vyjádřenou pomocí jejich koeficientů 1 - 0,125). Ve 4. st. př. n. l. představují nápisy s koeficientem 1 až 0,5 téměř 95 \letterpercent{} všech datovaných nápisů, což značí velmi malou míru nejistoty přesné datace nápisů. Naopak ve 4. st. n. l. se jedná pouze o 70 \letterpercent{}, což značí relativně velkou míru nejistoty datace nápisů přiřazených do daného století. Důvodem může být nízký počet charakteristických datačních prvků, s nímž se potýkali autoři epigrafických korpusů v kombinaci se všeobecným poklesem publikační aktivity ve 4. st. n. l. Medián, tedy střední hodnota pro všechna sledovaná století u skupiny nápisů s koeficientem 1 až 0,5 je 85 \letterpercent{}. Celkem v osmi z jedenácti sledovaných časových období je jejich poměr vyšší než 85 \letterpercent{}. Korpus nápisů datovaných do rozmezí dvou století, tedy nápisy s koeficientem 1 a 0,5, představuje statisticky dostatečně signifikantní vzorek pro všechna měřená století a jeho analýza umožňuje sledovat měnící se celospolečenské trendy s přesností na 100, respektive 200 let. Vzhledem k tomu, že epigrafická produkce je obecně svou povahou poměrně konzervativní, společenskokulturní změny se na formě a obsahu nápisů mohou objevit ve větším měřítku až se zpožděním několika desetiletí. Proto jsem zvolila časový interval jednoho století jako výchozí jednotku pro sledování vývoje a proměn epigrafické produkce v čase.

\subsubsection[problém-nadhodnocování-epigrafické-produkce]{Problém nadhodnocování epigrafické produkce}

V rámci analýzy celkové epigrafické produkce v jednotlivých století jsem řešila problém nadhodnocování celkového čísla datovaných nápisů do konkrétního století. Každý nápis, který byl totiž datován do více než jednoho století, může být započítán pro každé z těchto století se stejnou váhou, a může tak dojít ke zkreslení epigrafické produkce směrem nahoru. Společnost v Thrákii by se tak mohla zdát více epigraficky aktivní než jak tomu nasvědčuje celkový počet dochovaných nápisů.\footnote{Součet počtu nápisů pro jednotlivá století byl vyšší než celkový počet dochovaných nápisů.} Jako možné řešení této situace, které by lépe reflektovalo poměrné rozložení nápisů v rámci jednotlivých století, jsem se rozhodla uplatnit koeficient pravděpodobnosti datace k získání celkového počtu nápisů datovaných do daného století. Namísto abych započítávala každý nápis s hodnotou 1 pro všechna století, kam byly nápisy datovány, jsem započítávala pouze hodnotu jejich koeficientu pravděpodobnosti, a součtem koeficientů všech nápisů jsem došla ke konečnému číslu epigrafické produkce v daném století, které lépe reflektuje poměrné rozložení nápisů a nejistotu jejich datace (normalizovaná datace).

Graf 4.10a v Apendixu \in[Apendix2:::Apendix2] nabízí porovnání obou metod uplatněných na soubor datovaných nápisů z Thrákie. Výsledná linie představuje celkový počet datovaných nápisů v rámci jednotlivých století v závislosti na použité metodě (normalizovaná datace versus nenormalizovaná datace).\footnote{Výsledné číslo datace v daném století je v případě použití normalizované metody součtem koeficientů pravděpodobnosti všech nápisů. U nenormalizované datace každý nápis má hodnotu 1 pro každé století, do nějž je nápis datován (konečný součet je číslo 1 u nápisů datovaných do jednoho století, či násobky čísla jedna u nápisů datovaných do více století).} Ze srovnání obou křivek je patrné, že trendy měnícího se počtu nápisů zůstávají stejné, v závislosti na uplatnění koeficientu pravděpodobnosti datace se však mění celkové počty nápisů. Při uplatnění nenormalizované datace tak dochází k nárůstu počtu nápisů, který však neodpovídá počtu reálně existujících nápisů, viz výše. Rozdíl mezi výslednými čísly může být poměrně dramatický zejména ve stoletích, kde byly nápisy datované šířeji, například ve 2. st. n. l. tento rozdíl činí až 238 nápisů.\footnote{Obě dvě křivky zaznamenávají podobný vývoj epigrafické produkce, až na nápisy datované do 1. st. n. l., kde se trendy rozcházejí. Tento fakt může poukazovat na určitou nejistotu datace nápisů právě do zvoleného období, které je označováno jako období nástupu římské moci a transformace oblasti do římské provincie {\em Thracia}. Celkově však nenormalizovaná datace nápisů ukazuje obecně vyšší číslo nápisů, a to až o 60 \letterpercent{}, tj. o 1368 nápisů, než odpovídá skutečnosti, tj. 2276 oproti 3644 nápisům. Dochází tak ke zkreslení povahy epigrafické produkce v Thrákii, které může vést k nadhodnocování určitých trendů, zejména ve 2. a 3. st. n. l, kdy je nárůst celkového počtu nápisů markantní. Z tohoto důvodu jsem se rozhodla používat metodu normalizované datace, pokud hovořím o celkových trendech v rámci datovaných nápisů.} Z tohoto důvodu v kapitole 6. věnované chronologickému přehledu nápisů dále pojednávám zvlášť o nápisech datovaných s přesností do jednoho a do dvou staletí, aby nedocházelo k nárůstu celkového počtu nápisů a zároveň i inflaci epigrafické produkce v daném časovém období.

\subsection[nejistota-rozmístění-nápisů-v-krajině]{Nejistota rozmístění nápisů v krajině}

Dalším problémem, s nímž bylo nutné se vypořádat, je nejistota spojená s určením místa nálezu nápisu. Ve velkém množství případů epigrafické korpusy udávají pouze přibližnou polohu, kde byl nápis nalezen, a není ho tak možné spojovat s konkrétní archeologickou lokalitou. Pro provedení analýzy rozmístění nápisů, která bere v potaz vzájemné prostorové vztahy a zasazení epigrafické produkce do kontextu zeměpisných podmínek, bylo nutné se vypořádat s mírou nejistoty při určování nejpravděpodobnějšího místa nálezu nápisu.

Za tímto účelem jsem vytvořila tzv. {\em position certainty index}, tedy koeficient přesnosti určení míst nálezu, který určuje velikost území, na němž byl nápis nejpravděpodobněji nalezen.\footnote{Tabulka 4.03 v Apendixu \in[Apendix1:::Apendix1] přehledně shrnuje hodnoty koeficientu a celkového počtu nápisů jednotlivých skupin.} Při stanovování velikosti území jsem vycházela z informací poskytovaných editory jednotlivých autorů, zejména pak {\em IG Bulg}. Pokud bylo možné místo spojit s konkrétní archeologickou lokalitou, přesnost míry určení místa jsem stanovila ve vzdálenosti do 1 km, nesoucí koeficient 1.\footnote{Příkladem udávání místa nálezu, které bylo možno spojit s konkrétní archeologickou lokalitou je např. {\em IG Bulg} 118: „{\em Odessi reperta in via Prespa}.”; {\em I Aeg Thrace} 195: „Προέρχεται άπο τήν θέση Μάρμαρα της αρχαίας Μαρώνειας.”} Pokud autor korpusu udal místo nálezu v okolí moderního sídla s udáním konkrétních vzdáleností a směru, kde byl nápis nalezen, místo nálezu se s největší pravděpodobností nacházelo v okruhu 5 km od tohoto moderního sídla.\footnote{Příkladem relativně přesného místa nálezu je např. {\em IG Bulg} 3,2 1843: „{\em Repertus in agro quodam ad vicum nunc Dobrinovo, olim Hasbeglij dicto}.”; {\em IG Bulg} 4 2014: „{\em Reperta 1 km orientem versus a vico Gurmazovo, conservabatur penes vicanum eiusdem vici Pane Gjorev}.”} Pokud editor korpusu udal pouze všeobecnou informaci o nálezovém místě a jeho poloze vůči moderním sídlům, oblast nejpravděpodobnějšího místa nálezu byla stanovena do vzdálenosti 20 km od uvedeného moderního sídla.\footnote{Příkladem obecného udávání místa nálezu je např. {\em IG Bulg 4} 2034: „{\em Reperta ad vicum Dragoman}.”; {\em IK Byzantion} 159: „{\em Gefunden in Istanbul.}”} Skupina nápisů, jejichž místo nálezu je známé jen velmi obecně, či vůbec, ale editoři nápisu udávají, že pochází z Thrákie, nese koeficient 4, což značí velmi malou míru pravděpodobnosti konkrétního zeměpisného určení.\footnote{Příkladem nápisu s neznámou lokalitou je {\em IG Bulg} 5 5927: „{\em Thrace, région indéterminée}.”} Nápisy z této poslední skupiny vynechávám ze všech analýz rozmístění v kapitole 7, vzhledem k velmi malé výpovědní hodnotě o místě nálezu a prostorovém uspořádání těchto nápisů v krajině.

\stopcomponent