
\environment ../env_dis
\startcomponent section-technické-řešení-heurist
\section[technické-řešení-heurist]{Technické řešení: Heurist}

V rámci studijního pobytu v Sydney v roce 2012 jsem se seznámila s platformou {\em Heurist Scholar}, tehdy vyvíjenou na University of Sydney a rozhodla jsem se jí využít pro sběr dat v digitální podobě.\footnote{Heurist vznikl v roce 2005 pod vedením Dr. Iana Johnsona z University of Sydney. \useURL[url2][http://heuristnetwork.org/][][{\em http://heuristnetwork.org/}]\from[url2]. Od května 2013 je projekt veden jako Open Source a zájemci ho mohou využívat pro výzkumné účely bezplatně. Heurist je online databáze specificky navržená pro potřeby digitálního zaznamenávání historických dat v akademickém prostředí. Nejedná se tedy o databázi v pravém smyslu slova (například při porovnání s MySQL, PostgreSQL atp.), ale spíše o platformu umožňující sbírat, analyzovat a zveřejňovat data bez předchozích programátorských znalostí. Hlavní výhodou je přístupnost a flexibilita: kdokoliv s přístupem na internet a přístupem k databázi může přes webové rozhraní zadávat a analyzovat data. Heurist navíc v sobě kombinuje i určité funkčnosti geomapovacího softwaru, což je jedna z prerekvizit mého výzkumu. V roce 2013 epigrafické databáze ještě nezaznamenávaly geografická data a nebyly propojeny s geoinformatickými systémy (GIS) do té míry, jako je tomu v roce 2017, a tudíž tato funkčnost Heuristu měla zásadní roli na mé rozhodnutí používat právě Heurist.} Databázi jsem vytvořila v programu Heurist 3.1.0 v červnu 2013. Po prvotní fází zadávání dat jsem provedla revizi a úpravu struktury v říjnu 2013. Hlavní fáze zadávání dat probíhala od června 2013 do září 2016.\footnote{Celková doba nutná k vytvoření databáze: zhruba dva měsíce studia principů vytváření databází, jeden měsíc věnovaný testování a vylepšování databáze. Celkové zadávání nápisů do databáze bylo rozděleno mezi tři uživatele v poměru Janouchová (48 \letterpercent{}, 2362 nápisů), Kobierská (27 \letterpercent{}, 1348 nápisů), Ctibor (25 \letterpercent{}, 1270 nápisů); celkem 293 nápisů má dva autory z důvodu např. jeho několikanásobné opakované publikace, či re-editace v epigrafickém korpusu. Průměrná doba zadávání jednoho nápisu je zhruba 15 minut, s tím, že komplexnější nápisy trvaly až několik hodin, jednodušší naopak kratší dobu. Prostým vynásobením 15 minut x počet nápisů (4667) se dostaneme na šest měsíců nepřetržitého zadávání. Zadávání bylo vzhledem k studijnímu a pracovními vytížení členů týmu rozloženo do dvou let (hlavní zadávací fáze červen 2013-prosinec 2014; březen 2016-červen 2016). Po ukončení zadávání jsem jakožto administrátor kontrolovala kvalitu záznamů a jejich přesnost, a případně nápisy editovala, kde bylo nutné, což zabralo zhruba třetinu času oproti zadávání, tedy zhruba dva měsíce času. Tato fáze skončila v září 2016.} K 9. listopadu 2016 databáze obsahuje 4665 nápisů které díky své digitalizaci získaly jednotnou strukturu, ač pochází z různých epigrafických zdrojů, různé kvality a detailnosti zpracování. Data jsou volně dostupná komukoliv na internetu v několika digitálních formátech.\footnote{Heurist poskytuje možnost data především prohlížet a filtrovat, vyhledávat v nich, nicméně pro komplexnější analýzu je nutné data vyexportovat a využívat další specializované programy. Data získaná zadáváním informací do Heuristu je možné exportovat, a následně analyzovat, několika možnými způsoby. Geografická data je možné exportovat ve formátu KML ({\em keyhole markup language}) či {\em shapefile}, který je možné zobrazit v geoinformačním softwaru (ArcGIS, QGIS) či v Google Earth. Geografická data je možné exportovat ve formátu CSV ({\em comma separated value}) a transformovat do požadovaného formátu přímo v geoinformačním softwaru. Tabulární data (data textového charakteru) je možné exportovat ve formátu CSV a dále zpracovávat pomocí analytických programů jako je MS Excel, Google Spreadsheet, R. Databázi je možné jako celek exportovat jako XML formát, MySQL, PostgreSQL avšak tento způsob je vhodný spíše pro zálohování dat, než pro jejich analýzu a je vhodný pro zkušené uživatele. Data v neupraveném formátu společně s výsledky analýz a mapami jsou volně dostupná na adrese: \useURL[url3][https://github.com/petrajanouchova/hat_project][][{\em https://github.com/petrajanouchova/hat_project}]\from[url3]. V současné době HAT databáze používá verzi Heurist 4.2.8. (od 14. září 2016) a po přihlášení je dostupná na adrese: \useURL[url4][http://heurist.sydney.edu.au/h4/?db=HAT_Hellenization_of_Ancient_Thrace][][{\em http://heurist.sydney.edu.au/h4/?db=HAT_Hellenization_of_Ancient_Thrace}]\from[url4].}

\subsection[struktura-databáze-a-použitá-metodologie]{Struktura databáze a použitá metodologie}

Databáze je navržena v angličtině, pro větší přístupnost akademické veřejnosti a další použitelnost dat. Původní jazyky tištěných epigrafických korpusů jsou bulharština, řečtina, latina, němčina, francouzština a angličtina, což mnohdy znesnadňovalo práci s nimi. V současné podobě v HAT databázi jsou tak data přístupná i té části badatelů, která neovládá jeden či více z těchto jazyků. Údaje v databázi jsou jednak digitální podobou epigrafického {\em lemmatu} tak, jak ho vydali editoři korpusů, a jednak se jedná o interpretace odvozené z epigrafického {\em lemmatu}, textů nápisů a doprovodné fotografické a kresebné dokumentace.\footnote{Schéma databáze se nachází v tabulce 4.00 v Apendixu 1. Celkový přehled zpracovaných nápisů je součástí Apendixu 3.} V následující sekci se věnuji vybraným součástem struktury databáze a vysvětlení použitých termínů pro větší srozumitelnost dat obsažených v databázi. Tato struktura byla vytvořena na základě organizace tištěných korpusů a vychází především z uspořádání korpusů {\em IG Bulg.}\footnote{Podrobné vysvětlení jednotlivých položek databáze považuji za základní předpoklad práce s daty v režimu {\em Open Source}, tedy v režimu, který předpokládá šiřitelnost dat a srozumitelnost obsahu databáze pro další uživatele.}

\subsubsection[identifikační-čísla]{Identifikační čísla}

Každý nápis představuje jedinečný záznam, jemuž Heurist automaticky přidělí unikátní identifikační číslo ({\em Heurist ID}). Toto číslo není příliš intuitivní pro práci s nápisy, a proto jsem vytvořila ještě jiné identifikační číslo, které nápis rovnou přiřazuje ke korpusu, v němž byl publikován ({\em Corpus ID Numeric}). Každý korpus má daný číselný kód, a je tak možné na první pohled určit, do jakého korpusu nápis náleží. V podstatě se jedná o číslování použité v původním korpusu, ke kterému se přidá určitá hodnota dle zvoleného korpusu (10 000, 20 000,...). Korpus jiných zdrojů ({\em Other}) obsahuje nápisy z několika zdrojů s nestejným číslováním, a proto jsem zde zvolila posloupné číslování (100 000 + číslo předcházejícího nápisu + 1).\footnote{Přehled jednotlivých korpusů a přidělených řad identifikačních čísel nápisů se nachází v tabulce 4.01 v Apendixu 1, a celkový přehled všech použitých nápisů v Apendixu 3.} Každý nápis má celkem tři identifikační čísla: číslo přidělené Heuristem ({\em Heurist ID}), identifikační číslo s kódovanými hodnotami ({\em Corpus ID numeric}) a identifikační číslo uvedené v původním korpusu ({\em Corpus ID number}). Některé nápisy mají uvedené ještě identifikační číslo ze {\em Supplementum Epigraphicum Graecum} (SEG number), avšak toto číslo není uváděno systematicky u všech nápisů.\footnote{Příkladem nápisu se všemi identifikačními čísly je nápis SEG 49:992, který spadá do korpusu {\em Other}, jeho {\em Numeric ID number} je 100405, a {\em Heurist ID} je 16577. Ukázka záznamu v programu Heurist se nachází na obrázku 4.01 v Apendixu 2.}

\subsubsection[geografické-údaje]{Geografické údaje}

Většina korpusů uvádí původ nápisu pouze velmi obecně, a tudíž určit konkrétní místo nálezu může být v určitých případech obtížné. Navíc se přístup jednotlivých editorů při popisu místa nálezu liší, a tím pádem i charakter dostupných informací. Pro systematické zpracování nápisů bylo nutné geografické informace sjednotit, abych bylo možné je později souhrnně analyzovat a zobrazit formou map. K tomuto účelu jsem využívala program Heurist a jeho mapovací součást {\em Heurist Digitizer}.\footnote{Heurist využívá mapovací software {\em Google Maps}, který umožňuje manipulací s mapou či satelitním snímkem zaznamenávat zeměpisné koordináty (zeměpisnou šířku a zeměpisnou délku, v sekci Geolocation). Obrázek 4.02 v Apendixu 2 zobrazuje rozhraní programu {\em Heurist Digitizer}, v němž se zadávalo místo nálezu každého nápisu.}

Konkrétní bod míst nálezu nápisu je zvolen dle informací z informací poskytnutých editorem nápisů, uvedených v nápisném {\em lemmatu}. Editoři však udávají údaje o místě nálezu velmi obecně, a proto bylo nutné přijít s řešením zahrnujícím míru nejistoty určení místa nálezu. Na základě těchto specifik byl každému bodu přiřazen tzv. koeficient přesnosti místa nálezu ({\em Position certainty}), jehož hodnoty udávají hodnotu přesnosti místa nálezu.\footnote{Tzv. {\em buffer zone}, čili nárazníková zóna okolo daného bodu se vzdáleností do 1 km, do 5 km, do 20 km a nad 20 km, viz Tabulka 4.02 v Apendixu 1. Program Heurist udává každému místu nálezu jeden konkrétní bod na planetě, který je vyznačený zeměpisnými koordináty. Editoři nápisů naopak udávají spíše oblast, v níž byl nápis nalezen a nikoliv konkrétní bod. Zvolená metoda se snaží oba dva přístupy spojit a zároveň zaznamenat data o místu náletu v jednotné formě, díky níž je možné data vzájemně srovnávat a analyzovat. Podrobněji zvolenou metodologii rozebírám níže v této kapitole.} Nápisy, jejichž místo nálezu bylo možné určit s přesností do 20 km, využívám v rámci analýzy rozmístění nápisů v krajině v kapitole 7.

Následující položky vysvětlují způsob zaznamenávání geografických údajů do databáze:\footnote{Ukázka záznamu v programu Heurist zaznamenávající zeměpisné údaje se nachází na obrázku 4.03 v Apendixu 2.}

\startitemize
\item
  \startblockquote
  Místní údaje ({\em Location}) zaznamenávají místní jména tak, jak je obsahují epigrafické korpusy. Zaznamenává se jméno nejbližšího místa či osídlení, dále jméno antické lokality, pokud je známé, a region antického města, pod které nálezové místo spadalo ({\em Modern Location, Ancient Site, Ancient site - region}). Pokud se jedná o místní jméno, které již dnes neexistuje, pak je o tom poznámka v položce {\em Geography notes}. Do položky {\em Geography notes} se zaznamenávají veškeré údaje dostupné o místě nálezu, které není možné zadat do žádného jiného pole.
  \stopblockquote
\item
  \startblockquote
  Položka {\em Reuse} zaznamenává, zda byl nápis sekundárně použitý, např. nalezený ve zdech budovy, či přenesený do kostela. Pokud je položka prázdná, nemáme informace o sekundárním použití nápisu.\footnote{Příkladem zmínky o sekundárním použití může být {\em IG Bulg} 1,2 393: „{\em Apolloniae, olim conservatum erat in aedibus quibusdam}.”}
  \stopblockquote
\item
  \startblockquote
  Položka {\em Archaeological context} zaznamenává známý archeologický kontext nalezeného nápisu tak, jak ho udává epigrafický korpus ({\em Funerary} pro lokality s vyskytujícími se hroby, hrobkami, či archeologicky identifikovanými pohřebišti; {\em Habitation} pro lokality s archeologicky identifikovanými sídlišti, budovami a místy lidského osídlení; {\em Commercial} jako podkategorie {\em Habitation}, pro lokality s identifikovaných obchodním charakterem, jako například emporion, přístav, agora apod.; {\em Other} pro typy archeologické lokality nespadající do žádné z uvedených kategorií, s detaily uvedenými v poznámce; {\em Religious/Ritual} pro svatyně; {\em Secondary} pro nápisy nalezené v sekundárním kontextu).
  \stopblockquote
\item
  \startblockquote
  Položka {\em Mound} zaznamenává, zda editor udává, že byl nápis nalezen v pohřební mohyle, či její bezprostřední blízkosti.
  \stopblockquote
\stopitemize

\subsubsection[údaje-o-nosiči-nápisu]{Údaje o nosiči nápisu}

Tato sekce zaznamenává údaje o nosiči, na němž se nápis nachází. Informace pochází jak z epigrafického {\em lemmatu}, tak z doprovodného vizuálního materiálu, pokud byl k dispozici, jako je fotografie, kresba, fotografie oklepku. Podle charakteru epigrafického objektu je možné alespoň relativně určit, zda se jednalo o místně produkovaný předmět, či se mohlo jednat o import.\footnote{Ukázka záznamu dat o nosiči nápisu v programu Heurist se nachází na obrázku 4.04 v Apendixu 2.}

\startitemize
\item
  \startblockquote
  Položka {\em Material category} zaznamenává, z jakého materiálu byl předmět vyroben ({\em Metal} - kov; {\em Other} - jiný druh materiálu, detaily se uvádějí do poznámek {\em Decoration notes}, např. nápis vyrytý do omítky, či součástí mozaiky; {\em Perishable} - netrvanlivý materiál jako dřevo aj.; {\em Pottery} - keramika, terracotta; {\em Stone} - kámen). Pokud byl předmět vyroben z kamene, zajímá mě, z jakého druhu ({\em Stone}), případně zda je možné určit jeho původ ({\em Origin of stone}).
  \stopblockquote
\item
  \startblockquote
  Položka {\em Object category} zaznamenává, o jaký druh předmětu se jedná: {\em Architectural feature} pro architektonické prvky nesoucí nápis, {\em Mosaic} pro mozaiku obsahující nápis, {\em Other} pro předmět nespadající ani do jedné z uvedených kategorií. Případné detaily o dekoraci nosiče jsou uvedené do poznámky {\em Decoration notes}, {\em Stele} pro stélu, či tabulku nesoucí nápis; {\em Sculpture} pro sochu či její součást nesoucí nápis, {\em Vessel} pro nádobu či její součást nesoucí nápis; {\em Wall} pro stěnu stavby, či zeď nesoucí nápis.
  \stopblockquote
\stopitemize

Dále databáze zaznamenává celou řadu dat získaných jednak z textu epigrafického {\em lemmatu}, ale zejména z doprovodných vizuálních příloh. Editoři často doplnili publikaci fotografií, kresbou, fotografií oklepku, či méně často textovým popisem stavu dochování, či dekorace. Pro usnadnění zaznamenávání a následné analýzy jsem vytvořila základní typologii popisu předmětu, doplněnou o konkrétní příklady s fotografiemi.

\startitemize
\item
  \startblockquote
  Položka {\em Preservation} zaznamenává stav dochování předmětu v době vydání korpusu. Stav dochování je odhadován na základě procentuálního dochování původního tvaru a velikosti nadepsaného předmětu (100 \letterpercent{}, 75 \letterpercent{}, 50 \letterpercent{}, 25 \letterpercent{}), či zda se předmět dochoval pouze ve formě oklepku, překresby, či zda je předmět ztracený, či není možné stav jeho dochování určit.\footnote{Obrazová galerie příkladů je součástí tabulky 4.03 v Apendixu 1.}
  \stopblockquote
\item
  \startblockquote
  Položka {\em Decoration} zaznamenává typy použitých dekoračních technik - zejména mě zajímá reliéf a malovaná dekorace; další techniky se popisují v {\em Decoration Notes}. Položka {\em Relief decoration} zaznamenává konkrétní typy reliéfní dekorace. Položka {\em Architectural relief} je první typ reliéfní dekorace, zaznamenávající všechny druhy použité architektonické dekorace a její možné kombinace.\footnote{Jejich seznam a obrazová galerie je součástí tabulky 4.04 v Apendixu 1.} Položka {\em Figural relief} zaznamenává konkrétní druhy figurálního reliéfu, které se na předmětu vyskytují, a to i jejich kombinace.\footnote{Jejich seznam a obrazová galerie je součástí tabulky 4.05 v Apendixu 1.} Položka {\em Decoration notes} zaznamenává informace nespadající ani do jedné z výše uvedených kategorií či podrobnější informace uvedené editory korpusu, případně zpozorované na vizuálním materiálu.
  \stopblockquote
\item
  \startblockquote
  Položka {\em Visual record} availability zaznamenává, zda editoři korpusu přiložili též vizuální dokumentaci nápisu, jako fotografii, kresbu, fotografii oklepku.
  \stopblockquote
\stopitemize

\subsubsection[údaje-získané-z-textu]{Údaje získané z textu}

Tato sekce zaznamenává informace získané jednak z epigrafického {\em lemmatu}, jako je např. datace, ale i z formy a obsahu samotného nápisu. Přidržuji se standardní typologie řeckých nápisů, jak je známá z {\em Inscriptiones Graecae}, či jak je uvádí McLean (2002, 181-210).

\subsubsection[datace]{Datace}

Datace nápisu patří ve velké míře o interpretaci autora korpusu na základě referencí v textu, provedení nápisu a na základě analogií. Datace nápisu je zaznamenána tak, jak jí uvádějí editoři korpusu, s výjimkou, že všechna data jsou převedena do číselného intervalu pro jejich snadnější zpracování a vzájemnou konzistentnost dat. Roky před naším letopočtem se zapisují jako záporné hodnoty, a roky po našem letopočtu jako plusové hodnoty. {\em Start Year} udává maximální možné datum v minulosti ({\em terminus post quem}), do nějž může být nápis dle editora datován. {\em End Year} udává minimální možné datum v minulosti ({\em terminus ante quem}), do nějž může být nápis dle editora datován. Editoři mohou používat dataci ohraničenou konkrétními roky, případně stoletími, ale mnohdy používají popisnou dataci, zařazení do období, bez jasného vymezení jeho hranic. Pokud není stanoveno jinak, držím se v těchto případech standardního členění dle řeckých dějin (archaická doba, klasická doba, hellénismus atp.). Pokud tedy editor datuje nápis do klasické doby, pak použitá datace odpovídá maximálnímu rozsahu standardní datace řeckých dějin (479 - 338 př. n. l.).\footnote{Přehledná chronologická tabulka 4.06 s použitou metodologií převodu různých forem datace do jednotných čísel je uvedena v Apendixu 1.}

Každý editor epigrafického korpusu přistupoval k dataci jiným způsobem: někdo se snažil datovat všechny nápisy, ač někdy v rozmezí na několik století, a jiný editor dával přednost datovat jen nápisy, u nichž byl schopen přiřadit konkrétní datum, a nejisté nápisy ponechat bez datace. Tím ale vznikla poměrně velká skupina nedatovaných nápisů, kterou se mi však povedlo aplikací následujících pravidel zmenšit zhruba o třetinu, a přiřadit nápisům alespoň relativní dataci. Položka {\em Relative Date} obsahuje relativní dataci založenou na interpretaci výskytu osobních jmen, na vizuální kontrole užitého písma a výskytu osobních jmen, či jiných termínů, díky nimž je možné nápis alespoň relativně datovat jako římský, tj. spadajících do 1. - 5. st. n. l. Určujícími prvky jsou:

\startitemize
\item
  \startblockquote
  Přítomnost jmen, které je možné označit jako římská, či ovlivněná římskými onomastickými zvyky, např. jména jako Aurelios, Gaios, Ailios, Klaudios apod.
  \stopblockquote
\item
  \startblockquote
  Formy písma typicky užívané v době římské, a to zejména se zaměřením na písmena omega, lunární sigma, či na přítomnost ligatur. Proměnu stylu jsem zaznamenala u nápisů datovaných do římské doby a zpětně jsem tyto nové poznatky zpětně aplikovala na nedatované nápisy, v nichž se však vyskytovala stejná forma písma.
  \stopblockquote
\stopitemize

Položka {\em Century} je pak pouhým výčtem všech století, do nichž byl nápis datovaný. Tato položka byla vytvořena z praktického důvodu snadnějšího vyhledávaní a filtrování nápisů datovaných do hledaných století.

\subsubsection[typologie-nápisu]{Typologie nápisu}

\startitemize
\item
  \startblockquote
  Položka {\em Dialect} zaznamenává použitý řecký dialekt, alespoň nakolik ho bylo možné rozlišit z textu nápisu, či ve výjimečných případech z epigrafického {\em lemmatu}.
  \stopblockquote
\item
  \startblockquote
  Položka {\em Latin} zaznamenává, zda se v textu nápis vyskytuje latinský text či nikoliv.
  \stopblockquote
\item
  \startblockquote
  Položka {\em Language Form} zaznamenává literární formu jazyka, v němž je převážná většina textu napsána, tj. zda je psána ve verši či v próze.
  \stopblockquote
\item
  \startblockquote
  Položka {\em Script} zaznamenává druh použitého písma. Písmo je interpretováno na základě studia vizuálního materiálu. Pokud vizuální materiál chybí, analýza nebyla provedena a položka zůstala nevyplněná, či přebírá informace poskytnuté editorem korpusu.
  \stopblockquote
\item
  \startblockquote
  Položka {\em Layout} zaznamenává rozvržení nápisu a jeho pravidelnost. Zejména se jedná o zachycení datovatelných forem rozmístění textu, jako je např. {\em bústrofédon} či {\em stoichédon} (dle McLean 2002).
  \stopblockquote
\item
  \startblockquote
  Položka {\em Document Typology} zaznamenává druh dokumentu na základě studia textu a základních charakteristik předmětu nesoucí nápis. Jedná se o nápisy veřejné, soukromé, či o nápisy neurčitelné. Položka {\em Public Documents} obsahuje základní typologii veřejných nápisů (dle McLean 2002). Položka {\em Private Documents} obsahuje základní typologii soukromých nápisů (dle McLean 2002). Položka {\em Document Typology Notes} zaznamenává poznámky k typologii nápisu a obecně k intepretacím založeným na čtení textu.
  \stopblockquote
\item
  \startblockquote
  Položka {\em Extent of Lines} zaznamenává maximální dochovaný počet řádků textu tak, jak je text publikován v epigrafickém korpusu.\footnote{V případě nápisů rozdělených do několika sloupců se počet řádků sčítá.}
  \stopblockquote
\stopitemize

\subsubsection[textová-analýza-jednotlivých-termínů]{Textová analýza jednotlivých termínů}

V této sekci se zaznamenává výskyt specifických termínů ({\em keywords}) v textu nápisu. Tyto termíny postihují základní struktury společenského uspořádání typického pro komplexní společnost, jako jsou názvy institucí, názvy funkcí státního aparátu, specializovaná povolání, názvy lokalit a aktivit spojených s fungováním komplexní společnosti apod. Při analýze nápisů mě zajímá nakolik se tyto termíny vyskytovaly v konkrétních společenských kontextech, zda je možné jejich výskyt spojit s určitou společenskou skupinou. Dále mě zajímá, zda se termíny postupem času rozšiřovaly jak místně, tak i v rámci hierarchie společnosti, a zda je možné sledovat posun v jejich použití a významu. U všech hledaných termínů se určuje i jejich společensko-kulturní původ, na kolik je možné ho stanovit.\footnote{Vzhledem k faktu, že se jedná o řecky psané nápisy, které se drží řeckých vzorů i v pozdějších dobách, pak i většina termínů nutně pochází původně z řeckého prostředí.}

Hledané termíny jsou rozděleny do několika tematických skupin:

\startitemize[n][stopper=.]
\item
  \startblockquote
  Termíny týkající se fungování komplexní společnosti, administrativy a aparátu, který celou společnost udržoval v chodu ({\em administrative keywords}). Termíny se dále dělí do několika podskupin dle svého zaměření na termíny spojené s určitou lokalitou, konkrétní funkcí, institucí, s finančním, vojenským či kulturním zaměřením termínu, společenským postavením a v neposlední řadě specializací povolání.\footnote{Jejich kompletní seznam je uveden v tabulce 4.07 v Apendixu 1.}
  \stopblockquote
\item
  \startblockquote
  Opakující se formulace typické pro řeckou epigrafickou produkci, jako jsou invokační formule, terminologie spojená s publikačními aktivitami státního aparátu, spojení slov typická pro dedikační či funerální nápisy, formulace zabývající se publikováním nápisu.\footnote{Jejich kompletní seznam je uveden v tabulce 4.08 v Apendixu 1.}
  \stopblockquote
\item
  \startblockquote
  Termíny týkající se honorifikačních aktivit, jako jednoho z typických projevů epigrafické aktivity v rámci řeckého kulturního prostředí, probíhající pod patronátem státního aparátu. Termíny jsou rozděleny do několika podkategorií, jako jsou důvody pro udělení pocty, konkrétní obsah poct, opatření zabývající se publikováním poct.\footnote{Jejich kompletní seznam je uveden v tabulce 4.09 v Apendixu 1.}
  \stopblockquote
\item
  \startblockquote
  Termíny týkající se náboženství a kultovních aktivit, jednotlivých božstev, dále míst a funkcí spojených s výkonem náboženských rituálů ({\em religious keywords}).\footnote{Jejich kompletní seznam je uveden v tabulce 4.10 v Apendixu 1.}
  \stopblockquote

  \startitemize[a]
  \item
    \startblockquote
    Jako zvláštní podskupinu jsem zaznamenávala jednotlivá božská epiteta tak, jak se vyskytovala v textu nápisů. Spadají sem i pravopisné varianty jmen, u nichž rozlišuji epiteta vázající se k místním božstvům a božstvům všeobecně rozšířeným.\footnote{Jejich kompletní seznam je uveden v tabulce 4.11 v Apendixu 1.}
    \stopblockquote
  \stopitemize
\stopitemize

\subsubsection[identita-a-identifikace]{Identita a identifikace}

V této části se snažím určit společensko-kulturní kontext, v němž nápis vznikl. Na základě analýzy textu nápisu zaznamenávám, jak se lidé prezentovali na nápisech a pod jakou identitou se rozhodli vystupovat, či se k ní nějakým způsobem vyjádřit.\footnote{Podrobnější teoretický úvod k prezentaci identity na nápisech se nachází v kapitole 3.} Rozlišuji několik stupňů prezentace identity, a to jsou osobní jména a vazby na nejbližší okolí, jako např. rodinu, předky, partnery, přátele. Dále mě zajímají vazby na skupiny lidí a komunity, a společensko-politické jednotky jako jsou vesnice, města, regiony, státy apod.

\startitemize
\item
  \startblockquote
  V sekci {\em Names} zaznamenávám všechny osoby, které je možné identifikovat dle osobního jména, či jejich kombinací (osobní jméno + {\em patronymikum}, jméno matky, partnera, předků, jména potomků, sourozenců apod.). U jednotlivých jmen zaznamenávám i jejich původ, tedy kontext, v němž se jména vyskytovala původně a nejčastěji, a tradičně bývají považována za řecká, thrácká, římská či jiného původu/není možné jejich původ určit. Dále rozlišuji, zda se jedná o jméno ženské, či mužské, či zda není možné jeho příslušnost určit dle dochovaných textů.
  \stopblockquote
\item
  \startblockquote
  V rámci {\em Collective Group Names} zaznamenávám jména popisující skupinu lidí. Může se jednat jak o skupinu lidí založenou na etnickém principu, společném zájmu, či společensko-politické příslušnosti, či společném náboženském přesvědčení).\footnote{Jejich kompletní seznam je uveden v tabulce 4.12 v Apendixu 1.}
  \stopblockquote
\item
  \startblockquote
  V sekci {\em Geographic Names} zaznamenávám místní jména, tak, jak jsou uvedené v textu nápisu (osídlení, regiony, pohoří, řeky apod.).\footnote{Jejich kompletní seznam je uveden v tabulce 4.13 v Apendixu 1.}
  \stopblockquote
\stopitemize

\subsubsection[text-nápisu]{Text nápisu}

Text nápisu získávám z online databáze {\em Searchable Greek Inscriptions}\footnote{\useURL[url5][http://noapplet.epigraphy.packhum.org/][][{\em http://noapplet.epigraphy.packhum.org/}]\from[url5], 20. září 2016}, {\em Supplementum Epigraphicum Graecum}\footnote{\useURL[url6][http://referenceworks.brillonline.com/browse/supplementum-epigraphicum-graecum][][{\em http://referenceworks.brillonline.com/browse/supplementum-epigraphicum-graecum}]\from[url6], navštíveno 20. září 2016}, {\em Epigraphic Database Heidelberg}\footnote{\useURL[url7][http://edh-www.adw.uni-heidelberg.de/home][][{\em edh-www.adw.uni-heidelberg.de/home}]\from[url7], navštíveno 20. září 2016}, případně manuálním přepisem textů epigrafických korpusů. Heurist uchovává text ve formátu WYSIWYG\footnote{WYSIWYG je akronym anglické věty „{\em What you see is what you get”}, česky „co vidíš, to dostaneš”. Tato zkratka označuje způsob záznamu dokumentů v počítači, při kterém je verze zobrazená na obrazovce vzhledově totožná s výslednou verzí dokumentu, která je uložena v databázi.}, který se nejvíce podobá výsledné formě nápisu. Text užívám pro vnitřní potřebu projektu, jehož cílem není provádět nové edice nápisů, ale již vydané nápisy systematicky zpracovat a vzájemně porovnat zdroje, které nebylo pro svou nekompatibilnost nemožné konzistentně srovnávat. Texty samotné slouží jako doprovodné informace, vlastní data určená k analýze, extrahovaná z textů, jsou obsažena ve strukturované podobě databáze samotné.

\subsubsection[přílohy]{Přílohy}

V sekci {\em Attachments} se nacházejí obrazové přílohy tak, jak byly otištěny v jednotlivých korpusech. Tato sekce slouží čistě pro interní potřebu projektu a není zcela kompletní.

\stopcomponent