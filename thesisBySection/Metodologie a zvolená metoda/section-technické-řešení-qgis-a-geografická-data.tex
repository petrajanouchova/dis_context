
\environment ../env_dis
\startcomponent section-technické-řešení-qgis-a-geografická-data
\section[technické-řešení-qgis-a-geografická-data]{Technické řešení: QGIS a geografická data}

Geografická data získaná exportem z Heuristu mají formát jednak KML, ale i CSV se zeměpisnými koordináty, které následně převádím na formát {\em shapefile}. Pro zpracování dat a následnou vizualizaci formou map používám {\em open source} program QGIS, verze 2.8 Wien.\footnote{Více informací o softwaru QGIS a verzi 2.8, oficiální stránka společnosti \useURL[url8][http://www.qgis.org/en/site/index.html][][{\em http://www.qgis.org/en/site/index.html}]\from[url8] \cite[26. září2016].} QGIS je geoinformatický software určený ke zpracování, úpravám, analýze a přehledné prezentaci zeměpisných dat. Hlavní výhodou programu QGIS je jeho volná dostupnost a relativně malá hardwarová náročnost. Výsledným produktem práce s programem QGIS jsou analýzy v kapitole 7 a mapy obsažené v této práci.

Zdigitalizovaná data o poloze a rozmístění nápisů uchovávám a analyzuji ve formátu CSV, KML a {\em shapefile}. Jednotlivé nápisy obsahují informace o zeměpisné šířce a délce nejpravděpodobnějšího místa jejich nálezu. Dále nápisy obsahují informaci o přesnosti daného místa nálezu (tedy koeficient přesnosti místa nálezu, {\em Position certainty}, viz výše), který určuje, jak velký okruh v řádech kilometrů od daného geografického bodu připadá na nejpravděpodobnější místo nálezu daného nápisu (tzv. {\em buffer zone}). Dále každý záznam obsahuje kompletní tabulární data, dle nichž je možné nápisy filtrovat, analyzovat a zobrazovat na mapě.

Další geografická data, která pro projekt používám, jsou digitalizovaná data z {\em Barrington Atlas of the Greek and Roman World} \cite[Talbert2000], částečně upravená data získaná z projektu {\em Pleiades}\footnote{\useURL[url9][https://pleiades.stoa.org/home][][{\em https://pleiades.stoa.org/home}]\from[url9] \cite[26. září2016]}, data získaná z projektu {\em Tundzha Regional Archaeological Project} (Ross {\em et al.}, v přípravě, vyjde 2017), data získaná z projektu {\em Burial Mound}s (Weissova 2013; 2016), data z projektu {\em Stroyno Archeological Project} (Tušlová {\em et al.} 2015; 2016) a satelitní snímky poskytnuté nadací {\em Digital Globe Foundation}. Jako výchozí geografická data a podkladové vrstvy používám volně dostupné mapy {\em OpenStreetMap}\footnote{\useURL[url10][https://wiki.openstreetmap.org/wiki/Cs:Main_Page][][{\em https://wiki.openstreetmap.org/wiki/Cs:Main_Page}]\from[url10] \cite[26. září2016]}, data získaná v průběhu mapování vodních zdrojů Bulharska společností {\em Japan International Cooperation Agency}\footnote{\useURL[url11][http://www.jica.go.jp/english/our_work/social_environmental/archive/pro_asia/bulgaria_1.html][][{\em http://www.jica.go.jp/english/our_work/social_environmental/archive/pro_asia/bulgaria_1.html}]\from[url11] a \useURL[url12][http://open_jicareport.jica.go.jp/pdf/11878667_02.pdf][][{\em http://open_jicareport.jica.go.jp/pdf/11878667_02.pdf}]\from[url12] \cite[26. září2016]}, data získaná digitalizací topografických map Bulharska s měřítkem 1:50 000 a 1:100 000, data ve formátu {\em shapefile} obsahující administrativní hranice moderních států a silniční sítě\footnote{\useURL[url13][http://www.vdstech.com/osm-data.aspx][][{\em http://www.vdstech.com/osm-data.aspx}]\from[url13] \cite[26. září2016]} a další volně dostupná vektorová data.\footnote{\useURL[url14][http://www.naturalearthdata.com/][][{\em http://www.naturalearthdata.com/}]\from[url14] \cite[26. září2016]}

\stopcomponent