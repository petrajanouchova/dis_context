
\environment ../env_dis
\startcomponent section-technické-řešení-r-a-tabulární-data
\section[technické-řešení-r-a-tabulární-data]{Technické řešení: R a tabulární data}

Tabulární data o jednotlivých nápisech získaná ve formátu CSV zpracovávám a analyzuji v statistickém programu R verze 3.3.1 pro Windows OS a R Studio verze 0.99.903.\footnote{\useURL[url15][https://www.r-project.org/][][{\em https://www.r-project.org/}]\from[url15] a \useURL[url16][http://www.r-project.cz/about.html][][{\em http://www.r-project.cz/about.html}]\from[url16] \cite[26. září2016]. Za tabulární data považuji strukturovaná data získaná exportem z databáze v Heuristu, ve formátu CSV. Může se jednak jak o numerická data, zeměpisné koordináty, či o text. Data pocházejí z jednotlivých epigrafických korpusů, či se jedná o interpretace odvozené z dat publikovaných v daných korpusech (relativní datace, relativní poloha místa nálezu, vizuální podoba předmětu atp.).} R je volně dostupný program, který se specializuje na statistické zpracování dat a jejich prezentaci v podobě grafů a diagramů. Výsledným produktem jsou statistické údaje grafy obsažené v této práci a skripty jednotlivých analýz, které jsou součástí digitálního Apendixu.\footnote{https://github.com/petrajanouchova/hat_project.}

\stopcomponent