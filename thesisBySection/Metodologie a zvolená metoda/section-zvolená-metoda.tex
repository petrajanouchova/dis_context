
\environment ../env_dis
\startcomponent section-zvolená-metoda
\section[zvolená-metoda]{Zvolená metoda}

Při analýze materiálu se držím principu falzifikovatelnosti a testovatelnosti použitých metod, ve snaze ověřit či vyvrátit z nich vycházející interpretace \cite[righttext={, 57-73}][Popper2005]. Abych mohla nápisy z různých zdrojů a různé povahy vzájemně srovnávat, musela jsem přistoupit k jejich kategorizaci na základě interpretace jejich obsahu a formy. Tato míra jisté abstrakce umožňuje zjednodušený a kvantifikovatelný náhled do jednotlivých komunit s cílem je vzájemně porovnat v rámci několika časových úseků (O'Shea a Barker 1996, 13-19; Bodel 2001, 80-82). Při interpretaci nápisů se držím tradičních principů a kategorií řecké a latinské epigrafiky, stanovených již autory {\em Inscriptiones Graecae} a {\em Corpus Inscriptionum Latinarum}. Tyto principy, ač vznikly před více než 200 lety, tvoří i dnes pevné základy epigrafické disciplíny (Bodel 2001, 153-174; McLean 2002, 22, 181-182).

Pro usnadnění zpracování velkého množství nápisů jsem vytvořila elektronickou databázi, v níž jsem shromáždila přes 4600 nápisů z oblasti Thrákie. K analýze tak velkého množství nápisů jsem využívala moderní metody a nástroje, známé spíše z přírodních věd, z archeologie a z oblasti tzv. {\em digital humanities} \cite[righttext={, 275-293}][Bodel2012]. Konkrétní podoba užitých postupů je však přizpůsobena specifikům epigrafického materiálu, a je reakcí na nutnost analyzovat množství materiálu s poměrně velkým počtem informací nejisté povahy, jako je například datace či umístění nápisu. Do velké míry se však jedná o inovativní přístup, který staví na několika pilotních studiích a kombinuje dohromady přístupy několika disciplín (Benefiel 2010, 45-65; Feraudi-Gruenais 2010; 14-16; Witschel 2010, 77-86; Janouchová 2014).

Ke studiu společnosti antické Thrákie na základě studia dochovaných nápisů přistupuji na třech vzájemně provázaných úrovních: a) na úrovni jednotlivých nápisů a funkce, jakou ve společnosti zastávaly; b) na úrovni společenských trendů; c) a na úrovni epigrafických produkčních center. Zajímají mě nejen geografické vzorce a relativní rozmístění nápisů vůči kulturním centrům a komunikacím, ale i všeobecné proměny společnosti v závislosti na známých společensko-politických událostech.

\subsection[funkce-nápisů]{Funkce nápisů}

Prvním bodem analýzy na úrovni jednotlivých nápisů je stanovení role, jakou nápisy v dané společnosti hrály, a jak se tato role proměňovala v závislosti na čase a místě. Funkce nápisů, kterou je možné odvodit z jejich obsahu a formy, umožňuje hodnotit povahu epigrafické produkce a roli jakou nápisná kultura hrála v životě jednotlivců, ale i celé komunity \cite[righttext={, 181-182}][McLean2002]. Z povahy nápisů je možné odvodit celou řadu informací o existujícím politickém uspořádání, společenské hierarchii a infrastruktuře, která umožňuje jak fungování samotného politické uspořádání, tak stojí i za samotnou epigrafickou produkcí (Johnson 1973, 3-4; Tainter 1988, 99-106, 111-115).

\subsection[celospolečenské-trendy-a-složení-společnosti]{Celospolečenské trendy a složení společnosti}

Dále se při analýze nápisů zaměřuji na proměňující se složení populace, která se aktivně zapojovala do epigrafické produkce. V tomto ohledu mě zajímá především konkrétní způsob sebeidentifikace individuálních osob na médiu permanentního charakteru a její proměny. Zejména se zaměřuji na identifikaci s rodinou a nejbližší komunitou, dále s většími společensko-politickými celky, a případně s celky etnickými (Jenkins 2008; Derks 2009, 239-244; Schuler 2012, 63-67). V neposlední řadě mě zajímá, jaký byl vztah epigraficky aktivní populace k řeckému etniku a řeckým kulturním zvyklostech, zda docházelo k jejich přejímání či adaptaci na nové podmínky a ke vzniku zcela nového symbolického systému hodnot a dorozumívacích prostředků.

S tím velmi úzce souvisí i analýza norem chování v závislosti na společenském postavení, pohlaví a v neposlední řadě i původu. Zajímám se o konkrétní projevy a šíření inovací a kulturních prvků v pobřežních a vnitrozemských komunitách v průběhu jednotlivých století.\footnote{Podrobněji v kapitole 6 pro analýzu nápisů v jednotlivých stoletích a v kapitole 7 pro analýzu rozmístění epigrafických produkčních center.} Sleduji jednak proměny složení epigraficky aktivní populace, ale i proměny epigrafické produkce obecně. Dále se zaměřuji na výskyt konkrétních prvků společenské organizace a projevů kultury a náboženství, a zasazuji roli a význam daných prvků do kontextu sledované společnosti v daném časovém období \cite[righttext={, 66-68}][Dietler2005].

\subsection[epigrafická-produkční-centra]{Epigrafická produkční centra}

Místa se zvýšenou epigrafickou aktivitou, tedy místa v jejichž blízkosti byly nápisy nalezeny, poukazují s největší pravděpodobností i na epigrafickou aktivitu v tamní komunitě.\footnote{Rozmístění nápisů a vztah k jednotlivým produkčním centrům podrobněji rozebírám v kapitole 7.} V rámci místně zaměřené studie mě zajímá vztah rozmístění nálezových míst nápisů v krajině, a zda je z nich možné vypozorovat obecně platné trendy. Zajímá mě například, zda se nápisy objevují pouze v okolí řeckých měst, či se nalézají i v čistě thráckém kontextu a zda je například možné pozorovat rozdíly mezi umístěním nápisů v krajině v závislosti na jejich společenské funkci. Dále mě zajímá, jaké je rozmístění nápisů vůči místům s lidskou aktivitou, jakou jsou např. města či cesty, a zda je možné pozorovat vliv této infrastruktury na rozmístění nápisů, případně na jejich zvýšené koncentrace na určitých místech v závislosti na demografii regionu (Woolf 1998, 77-105; Woolf 2004, 157-164; Bodel 2001, 80-82). Navzájem srovnávám jednotlivé oblasti Thrákie s cílem postihnout opakující se vzorce chování na základě rozmístění nápisů. V neposlední řadě srovnávám epigrafická produkční centra se známými archeologickými prameny. Interpolací a statistickým zhodnocením jednotlivých epigrafických nálezů z vybraných lokalit se snažím zhodnotit míru relevance a výpovědní hodnoty epigrafických památek pro studium antické společnosti jako celku.

\stopcomponent