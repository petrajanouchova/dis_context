
\environment ../env_dis
\startcomponent chapter-Mezikulturní-kontakty-v-antickém-světě
\startchapter[title={Mezikulturní kontakty v antickém světě}, reference={chapter-Mezikulturní-kontakty-v-antickém-světě}, marking={Mezikulturní kontakty v antickém světě}]
Thrákie tvořila spojnici mezi Asií a Evropou, a jak dokládají archeologické nálezy, docházelo na tomto území k přesunům obyvatelstva v obou směrech dávno před objevením prvních písemných památek. V souvislosti s řeckou kolonizací regionu a následnou přítomností řecky mluvícího obyvatelstva se badatelé věnovali ve zvýšené míře právě kontaktům řeckého a místního thráckého obyvatelstva, avšak zejména z perspektivy příchozích Řeků. Tyto kontakty bývaly v minulosti hodnoceny jednotným interpretačním modelem hellénizace, o němž se dnes soudí, že je příliš jednostranně zaměřen a dostatečně nepostihuje mnohoznačnou realitu mezikulturních kontaktů. V rámci hledání nového interpretačního rámce došlo v posledních několika desetiletích ke změnám v přístupu k materiálu. Archeologové ve spolupráci s kolegy z příbuzných disciplín vyvinuli nové teoretické přístupy, zkoumající mezikulturní kontakty společností na základě dochované materiální kultury. Hlavním cílem této kapitoly je poskytnout přehled teoretických přístupů zabývajících se mezikulturními kontakty a zhodnotit jejich dopad na současnou podobu archeologie, historie a v neposlední řadě i epigrafiky. Především se zaměřuji na tzv. koloniální přístupy v čele s akulturací a hellénizací, které měly největší vliv na interpretaci dochovaného materiálu, a dále na tzv. post-koloniální přístupy a jejich obecný dopad na současné historické bádání. Závěrem hodlám představit zvolený teoretický přístup, z nějž do velké míry vycházím při interpretaci pramenů v této práci.
\stopchapter
\stopcomponent