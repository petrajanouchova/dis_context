
\environment ../env_dis
\startcomponent section-koloniální-teoretické-modely
\section[koloniální-teoretické-modely]{Koloniální teoretické modely}

Takzvané koloniální teoretické modely vycházejí z tradice evropského kolonialismu 19. a první poloviny 20. století, který vnímá západní kulturu jako kulturní a společenskou normu vzešlou z antické tradice \cite[righttext={{, 33-47},{, 33}}][Dietler2005, Jones1997]. Národy a společnosti, které se od této normy odlišují, jsou vnímány jako podřadné, a jako objekt vhodný k civilizování západní řecko-římskou kulturou.

Nejdříve se zaměřím na koncepty akulturace a hellénizace, které, ač koncepty 19. století, hluboce rezonovaly v archeologických a antropologických publikacích po většinu 20. století, a v případě Thrákie mají ze specifických důvodů vliv na interpretaci materiálu i dodnes. I přes svůj velký dopad na současné bádání mají tyto přístupy celou řadu nedostatků. Hlavními body kritiky je jejich jednostranné zaměření a interpretace mezikulturních kontaktů, dále pak monolitická neměnnost a opomíjení často bohatých kontextů a situací vznikajících z reality společenských a kulturních interakcí \cite[righttext={, 33-47}][Dietler2005].

\subsection[akulturace]{Akulturace}

Pojem akulturace nikdy nepředstavoval jeden ucelený model, ale spíše teoretický směr vysvětlující a popisující proces kulturní změny a důsledky kontaktu odlišných kultur, a to jak společensky, kulturně, tak materiálně (Redfield {\em et al}. 1936, 149)\footnote{Redfield {\em et al.} 1936, 149: „{\em Acculturation compherehends those phenomena which result when groups of individuals having different cultures come into continuous first-hand contact, with subsequent changes in the original patterns of either or both groups}”.}. Akulturace se stala jedním z hlavních interpretačních přístupů pojednávající o kulturních změnách v první polovině 20. století, tedy doby silně ovlivněném evropským a americkým kolonialismem a imigrací \cite[righttext={, 24}][Cusick1998]. Akulturace našla své uplatnění v psychologii, sociologii, či archeologii, vždy se ale jednalo o vysvětlení procesu změny u jedince či ve společnosti na základě kontaktu s jinou kulturou.

Akulturační teorie předpokládá adopci či úplné nahrazení kulturou s propracovanější společenskou strukturou v prostředí méně vyvinuté kultury. Evropská, či západní, kultura, a tedy i kultura starověkého Řecka a Říma, bývá v rámci akulturačních teorií chápána jako nadřazená kultuře původní, a ta je naopak chápána pouze jako pasivní příjemce a kultura méně vyspělá. Reakce na mezikulturní kontakt bývají v rámci akulturační teorie interpretovány různě, počínaje úplnou adopcí nové kultury a odvržením staré, adaptací a splynutím s původní kulturou, úplným odmítnutím, či tzv. {\em code-switching}, čili používáním jedné či druhé kulturní normy dle aktuální situace \cite[righttext={, 29}][Cusick1998].\footnote{V publikaci {\em Memorandum for the Study of Acculturation} z roku 1936 (Redfield {\em et al.} 1936, 149-152) autoři sumarizují tři možné druhy reakcí přijímající kultury jako následující: a) přijetí nové kultury za částečné či úplné ztráty kultury původní, b) adaptaci nové kultury a spojení s kulturou původní, c) reakci a odmítnutí nové kultury.}

Autoři zmíněného {\em Memoranda} (Redfield {\em et al.} 1936, 150-151) poukazují na možné způsoby studia akulturace. Jako první bod uvádějí určení druhu a charakteru kontaktů mezi oběma kulturami\footnote{Zda kontakty probíhají mezi celými skupinami, či vybranými jednotlivci se speciálním posláním, zda jsou kontakty přátelské či nepřátelské, zda probíhají mezi skupinami stejné velikosti, na stejném stupni společenské komplexity, zda probíhají na území obývané jednou ze stran, či na zcela novém území.}, dále analýzu situací, v nichž ke kontaktům dochází\footnote{Zda byly elementy nové kultury představeny silou, či se jednalo o dobrovolnou akci, zda mezi oběma stranami panovala společenská nerovnost, a pokud existovala, zda se jednalo o politickou či společenskou dominanci jedné strany.}, zhodnocení samotného procesu akulturace\footnote{Jaké prvky společnosti či jedince se proměňují ve vztahu k druhu kontaktu a vázané na konkrétní situaci; dále k jakým změnám dochází pod nátlakem a jaké prvky jsou odmítnuty a z jakého důvodu. Dalším bodem je oboustranné zhodnocení praktických výhod akulturace, např. ekonomického prospěchu či politické dominance, a jiných motivů, jako je např. získaní společenské prestiže, či návaznost na prvky v kultuře již dříve existující.} a nakonec integraci nových kulturních prvků do původní kultury.\footnote{Autoři hodnotí dobu, jaká uplynula od představení prvku, obtížnost přijetí nového prvku v rámci původní kultury, dále míru nutnosti přizpůsobení původní kultury novému uspořádání.} Autoři v {\em Memorandu} (Redfield {\em et al.} 1936) nastínili základní metodologii a přístupy ke studiu změn ve společnosti, které jsou přínosné i v dnešní době, avšak i přesto zůstává akulturační model příliš obecný a jednostranně zaměřený pouze na změnu kultury příjemce. Mezikulturní kontakty probíhají však oběma směry a dochází ke změnám v kultuře obou zúčastněných stran, nikoliv pouze v kultuře s méně propracovaným systémem ideových hodnot. Tento jednostranný a částečně diskriminující přístup byl kritizován celou řadou vědců pro své zjednodušování skutečnosti a opomíjení skutečného stavu, kde se kultury ovlivňují navzájem, a proto se dnes od jeho užívání upouští \cite[righttext={, 23}][Cusick1998].

\subsubsection[akulturace-v-archeologii-a-historických-vědách]{Akulturace v archeologii a historických vědách}

Uplatnění akulturačního přístupu v archeologii a historických vědách se do velké míry zaobírá pouze adopcí a adaptací materiální kultury. Přítomnost materiálních projevů typických pro jednu kulturu mimo obvyklé území bývá pak často vysvětlována právě akulturací tamního obyvatelstva; jinými slovy čím větší množství cizího materiálu je nalezeno, tím větší bývá přisuzován akulturační vliv příchozí kultury na tamní obyvatelstvo. Často bývají změny v materiální kultuře vykládány pouze jako jevy vedoucí ke změně identity osob, ač existují i jiná možná vysvětlení, jako například změna trendu a poptávky nebo změna technologie \cite[righttext={, 31}][Cusick1998]. Materiální památky samy o sobě většinou neposkytují dostatek informací k odlišení akulturace, tedy adopcí či adaptací cizí kultury, a pouhého používání předmětů pocházejících z jiné kultury, a proto se upouští od užívání akulturace jako interpretačního rámce kulturního kontaktu \cite[righttext={, 29-39}][Jones1997].\footnote{Terminologie používaná pro vysvětlení akulturačních procesů se stala natolik běžnou, že je mnohdy užívána bez přímé souvislosti s akulturačním přístupem či bez reflexe a znalosti problematiky.}

\subsection[hellénizace]{Hellénizace}

Teoretický koncept hellénizace vychází z akulturačního přístupu, který aplikuje specificky na kontakty s řeckým světem (Boardman 1980; Tsetskladze 2006). Akulturace i hellénizace sdílejí velké množství charakteristik, jako je nerovné rozdělení moci a kulturní dominance mezi hlavními aktéry a dále jednosměrnost kulturní výměny. Základním procesem kulturní změny u obou přístupů je difuze, kdy se materiální objekty, myšlenky a koncepty šíří od společensky vyvinutější společnosti k méně vyvinuté kultuře. Ač neexistuje jednotná definice hellénizace, většina autorů se shodne, že se jedná kontinuální progresivní proces adopce řeckého jazyka, materiální i nehmotné kultury, způsobu života, a v neposlední řadě řecké identity a příslušnosti k řecké komunitě neřeckým obyvatelstvem.\footnote{Vranič 2014, 33: „{\em It is widely believed that the ancient “Hellenization” process, traditionally perceived as a simple and unilateral “spreading of Greek influences", without any recognition of reciprocity, resistance, and non-Greek agency in the Mediterranean, begins with the initial colonial encounters in the Archaic period}.”} Hellénizace je vnímána jako nevratný proces se stoupající tendencí, tj. pokud je někdo hellénizován, nemůže se „odhellénizovat” a navrátit se k původnímu stavu \cite[righttext={, 9}][Vlassopoulos2013].\footnote{ Termín je odvozen od řeckého slovesa „Ἑλληνίζω”, což primárně znamená mluvit řeckým jazykem (LSJ). V přeneseném slova smyslu a v pasivní formě se používal tento termín již v antice, a to ve smyslu pořečťovat, činit řeckým ve smyslu jazyka a kultury. Opačný význam neslo sloveso „βαρβᾰρίζω”, které mohlo znamenat jak „mluvit jiným jazykem, než řecky”, tak i „chovat se jinak, než je obvyklé pro Řeka”, případně „stranit Neřekům”. Původní význam termínu „βάρβᾰρος” byl „člověk mluvící jinou řečí než řečtina” a jeho použití nemělo negativní nádech \cite[righttext={, 345}][Malkin2004]. Pejorativní význam získalo slovo až druhotně v kontextu řecko-perských válek. Moderní použití slova barbar, či barbarský se odvozuje právě od negativních zkušeností Řeků s multikulturní a multinárodnostní perskou říší v průběhu 5. století př. n. l.}

Nejstarší definice hellénizace pochází od Hérodota, řeckého autora 5. st. př. n. l. který se pokusil se definovat podstatu {\em hellénicity} na základě reflexe společných rysů, které spolu Řekové sdíleli.\footnote{Hdt. 8.144.2: τὸ Ἑλληνικὸν.} Na základě srovnání mnoha řeckých komunit té doby Hérodotos formuloval čtyři kritéria příslušnosti k řecké komunitě, která se do velké míry shodují s definicí hellénizace tak, jak jí chápou moderní badatelé. Společnými rysy, dle Hérodota, byla sdílená minulost a fakt, že všichni Řekové mají společné předky a pojí je krevní pouto. Dalším charakteristickými rysy byly společná řeč, a fakt, že vedou podobný způsob života a vyznávají stejnou víru (8.144.2).\footnote{V současnosti je tato definice považována za příliš zobecňující, jednostranný pohled, který opomíjí variabilitu etnické identity v antickém světě (Zacharia 2008, 34-36; Janouchová 2016). Podobná kritika se objevila nejen vůči Hérodotově pojetí řecké příslušnosti, ale i vůči modernímu teoretickému konceptu hellénizace, jak byl chápán v 19. a 20. st. n. l.} Jakékoliv odchýlení od kulturní normy, ať už kulturní, tak morální, je prezentováno v ostrém kontrastu „těch druhých” (Hartog 1988 {[}1980{]}, 212-259; Vlassopoulos 2013b, 56-57). Hérodotův dualismus Řeků a „těch druhých” barbarů se ve velké míře odráží i v moderní sekundární literatuře zabývající se neřecky mluvícími národy antického starověku (Harrison 2002; Zacharia 2008). Proto i západní akademický svět a interpretace materiálu vycházející ze studia řecky psané literatury, jsou svou podstatou hellénocentrické a hellénofilské a přijímají hellénizaci jako jediné možné východisko kontaktu řecké a neřecké populace.\footnote{Jak je možné si povšimnout v rámci úvodní kapitoly, řečtí autoři, jako Hérodotos, Thúkýdidés či Xenofón, se nezaměřovali specificky na popis neřecky mluvících národů, ale zmiňovali se o nich v souvislosti s vysvětlením širších historických skutečností a pak zejména v komparaci s popisem řeckého etnika. Cílem mnoha těchto prací nebylo zaznamenávat etnografickou skutečnost o neřeckých národech, ale spíše bavit řeckého posluchače výčtem kuriozit a neobvyklých norem chování (Hartog 1988 {[}1980{]}, 230-237). Pro lepší ilustraci slouží například Hérodotovo dílo, jehož jednotlivé části, {\em logoi,} jsou sbírkou etnografických zajímavostí a faktů o neřeckých národech tak, jak byly prezentovány řeckému posluchači.}

Moderní teoretický koncept hellénizace je založen do značné míry na studiu dochovaných literárních pramenů, převážně psaných řecky, určených pro řecké publikum. Michael Dietler správně podotkl, že západní kultura vychází z hellénofilství 19. století, a je do značné míry postavena na studiu těchto textů, což by vysvětlovalo popularitu konceptu hellénizace v 19. a 20. století \cite[righttext={, 35-51}][Dietler2005]. Řecká kultura byla považována v antice, tak i v badatelských kruzích za kulturní normu, a cokoliv odlišného bylo považováno za podřadné, barbarské v moderním slova smyslu.

Zcela v duchu akulturačních teorií i hellénizační přístup předpokládá kulturní nadřazenost jedné společnosti nad druhou \cite[righttext={, 16}][Stein2005]. Neřecká společnost automaticky považována za kulturu vystav absorbující výdobytky kolonizující řecké společnosti, jako je společenská organizace či propracovaný náboženský systém (Whitehouse a Wilkins 1989, 103; Owen 2005, 12-13; Dietler 2005, 55-61).\footnote{Whitehouse a Wilkins 1995, 103: „{\em Equally invidious is the strongly pro-Greek prejudice of most scholars, which leads them to regard all things Greek as inherently superior. It follows that Greekness is seen as something that other societies will acquire through simple exposure---like measles (but nicer!)}.”} Jak akulturační, tak hellénizační přístup většinou ignoruje podíl místního neřeckého obyvatelstva na vytváření společné kultury, případně neuznává, že k procesu kulturní výměny mohlo docházet a často i docházelo z obou stran \cite[righttext={, 126}][Antonaccio2001]. Místní populace je vnímána jako kulturně vyprázdněné území, jehož obyvatelé jsou příchozími Řeky civilizováni na vyšší úroveň a pasivně přijímají novou kulturu.

Whitehouse a Wilkins (2005, 103) udávají, že koncept hellénizace je možné použít jako interpretační rámec například pro srovnávání stylistických změn ve výrobě keramiky, či architektury, ale považují ho za zcela nevhodný pro studium změn ve struktuře společnosti. Hellénizační přístup bývá aplikován {\em en bloc} na celou populaci, bez reflexe lokálních variant, území větší či menší rezistence, či naopak otevřenosti nově příchozí kultuře. Hellénizace zcela opomíjí místní společenský, politický a ekonomický kontext, který hrál zásadní roli při kontaktu s novou kulturou. Kritik Hellénizace Michael Dietler (1997, 296) dokonce podotýká, že model sám osobně nenese interpretační hodnotu, pouze dokumentuje a popisuje situaci bez skutečného porozumění mezikulturních kontaktů a jejich důsledku pro všechny zúčastněné komunity. Tím se stává příliš obecným a jednostranně zaměřeným modelem i pro studium epigrafických památek jakožto produktů dané společnosti.

\subsubsection[srovnání-teoretických-konceptů-hellénizace-a-romanizace]{Srovnání teoretických konceptů hellénizace a romanizace}

Pro lepší pochopení vývoje teoretického konceptu hellénizace a jeho velké obliby a rozšíření mezi badateli je nutné stručně pohovořit o teoretickém pozadí, z nějž hellénizace vychází. Koncept hellénizace se totiž do velké míry inspiroval tematicky spřízněnou romanizací, tedy velmi oblíbeným teoretickým přístupem 19. a počátku 20. století, který si nicméně udržel své příznivce i do současné doby (Woolf 1994, 116; Freeman 1997, 27-50; Millett 1990). Koncept romanizace se zabývá taktéž adopcí materiální kultury, jazyka, zvyklostí, způsobu života a přijetí nové identity původními obyvateli nově získaných území římského impéria \cite[righttext={{, 339},{, 29-39}}][Woolf1997, Jones1997]. K rozvoji a oblibě konceptu romanizace přispěl nejen velký rozsah římské říše, ale i mnohonárodnostní složení jeho obyvatel a nutnost teoretického přístupu, který by vysvětloval velmi podobné změny v materiální kultuře obyvatel nově získaných území v přímé souvislosti s působením římského státního aparátu a zvýšenou vojenskou přítomností na daném území.

Společným rysem hellénizace a romanizace je jejich interpretace kulturní změny jako nevyhnutelné a přirozené reakce původních kultur na setkání s kulturně vyspělejšími civilizacemi \cite[righttext={{, 15},{, 336-337}}][Champion2005, Dietler1997]. Oba dva směry jsou primárně zaměřené na dominanci řecké a římské kultury nad pasivními kulturami původního obyvatelstva, jehož reakce na kontaktní situace jsou obecně v literatuře přehlíženy a recipročním změnám v řecké a římské kultuře není věnován dostatečný prostor. Oba dva směry jsou striktně dvoudimenzionální a soustředí se na kontakty mezi Řeky a Neřeky, Římany a Neřímany, kde řeckou, respektive římskou stranu považují kulturně, politicky i ekonomicky dominantní. Tento striktní dualismus a černobílé vidění obou přístupů je v posledních dvaceti letech kritizováno jako příliš zobecňující, opomíjející historickou realitu a jemné nuance a variace mezikulturních vztahů (Dietler 2005, 55-57; Mihailovič a Jankovič 2014, xv). Kritice čelí i omezenost hlavních proudů obou dvou směrů počátku 20. století na interakce převážně společenských elit a opomíjení nižších vrstev společnosti \cite[righttext={, 83}][Hingley1997]. Tento rys vychází z faktu, že primární pramenem studia obou směrů byly zejména písemné prameny, které zaznamenávají právě elitní vrstvy společnosti.

V neposlední řadě se setkáváme s nejednotným metodologickým přístupem k interpretaci materiálu. Práce zabývající se hellénizací pojímají tento termín poměrně vágně, či nemají potřebu definovat teoretický přístup k hellénizaci s předpokladem, že termín nese vysvětlení sám o sobě \cite[righttext={{, 324},{, 9}}][Dominguez1999, Vlassopoulos2013]. Práce zabývající se romanizací většinou předkládají detailnější metodologii, nicméně proces sám o sobě je taktéž chápán jako dostačující vysvětlení kulturní změny \cite[righttext={, 9-17}][Mattingly1997]. Ač oba dva koncepty spojuje do velké míry snaha postihnout změnu kultur v reakci na mezikulturní kontakt, prostředí, jež romanizace popisuje se od řeckého prostředí liší poměrně radikálně. Pojetí romanizace do velké míry ovlivnil charakter římského impéria, velký rozsah území a mnohonárodnostní složení, propracovaný státní aparát a dobře organizovaná armáda, což jsou prvky vyskytující se v řecké kultuře v menší míře, v nejednotné formě či dokonce vůbec. V řeckém prostředí totiž, na rozdíl od římské říše, neexistovala jednotná politická autorita, jednotná materiální kultura, a dokonce ani jednotný jazyk či kolektivní identita, což jsou zásadní podmínky pro úspěšnou aplikaci hellénizačního modelu v praxi. S tím souvisí i odlišné zaměření obou přístupů. Zatímco hellénizace se zaměřuje spíše na změny v kultuře a identitě obyvatel, romanizace se soustředí především na společensko-politické a ekonomické důsledky v různých regionech římského impéria s ohledem na projevy v tamní materiální kultuře \cite[righttext={, 27}][Freeman1997].

\subsubsection[nejednotný-charakter-řecké-kultury-a-identity]{Nejednotný charakter řecké kultury a identity}

Hellénizační model je založený na předpokladu, že neřecká kultura absorbuje a asimiluje řeckou materiální kulturu, řecký jazyk, způsob života a s tím spojenou i řeckou identitu \cite[righttext={{, 12-13},{, 21}}][Owen2005, Zacharia2008]. Aby tento model mohl fungovat, nutným předpokladem je nejen jednotný charakter řecké materiální kultury, ale také jednotný a statický charakter řecké identity, který je navíc úzce spojený s materiální kulturou. Archeologický výzkum v mnohém vyvrátil tvrzení, že je možné přímo spojovat etnickou identitu s materiální kulturou \cite[righttext={{, 106-127},{, 129}}][Jones1997, Antonaccio2001]. Jinými slovy, rozšíření tradičních archeologických kultur neodpovídá rozmístění jednotlivých etnických skupin a produkce určitého typu keramiky či dekorace je vázána spíše na individuální vkus a použitou technologii, než na etnicitu \cite[righttext={, 128-129}][Jones1997]. Archeologické výzkumy navíc dokládají, že v antice neexistovala jednotná a normativní řecká materiální kultura, ale jednalo se o lokální varianty, které se více či méně navzájem inspirovaly. Zpochybněn byl i argument na provázanost jednotlivých zvyklostí čistě s řeckou etnicitou, což mimo jiné dokazuje i absence jednotného pohřebního ritu v řeckých koloniích na pobřeží Černého moře \cite[Damyanov2012]. Dříve se usuzovalo, že se řecké etnikum vyznačovalo užitím kremace a místní Thrácké obyvatelstvo pohřbívalo inhumací, avšak paralelní existence obou ritů znemožnila etnickou identifikaci pouze na základě pohřebních zvyklostí. Stejně tak přítomnost objektů řecké provenience nedokazuje, že místo bylo obýváno Řeky, či obyvatelstvem, které se používáním řeckých předmětů stalo „hellénizovaným” a přijalo řeckou identitu \cite[righttext={, 41}][Vranič2014].\footnote{V nadsázce je možné tento přístup aplikovat i na současný svět: pokud by provenience používaných předmětů určovala etnicitu, více jak polovina moderního světa by byla obývána Číňany, vzhledem k tamní masové produkci předmětů denní potřeby.}

Druhým problematickým bodem je otázka řecké identity a její jednotnosti. François Hartog (1988 {[}1980{]})\footnote{1980 původní vydání, 1988 překlad do angličtiny.} a Edith Hall (1989) rozpoutali relativně bouřlivou akademickou debatu na téma řecké identity v antice a jejích projevech v kontaktu s jinými etnickými skupinami. Badatelé dnes všeobecně přijímají, že tzv. hellénicita, čili jednotná řecká etnicita, je konstruktem moderní společnosti a nejedná se o vrozenou vlastnost (tzv. {\em primordial ethnicity}). Naopak, většinově se dnes kloní k názoru, že etnicita tvořila a tvoří jen jednu ze složek identity, jejíž projevy se mohly mění dle aktuálních podmínek (tzv. {\em situational ethnicity}; Barth 1969; Hall 1997, 17-33; Jones 1997, 13; Demetriou 2012, 14; Mackil 2014, 270-271; Diaz-Andreu 2015, 102).

Namísto jednotné řecké etnicity je tak vhodné spíše mluvit o vědom řecké kolektivní identity, tzv. panhellénismu, která vznikala za velmi specifických podmínek a měla omezené trvání.\footnote{Vzhledem ke geografické členitosti řeckého území je velmi obtížné mluvit o řeckém národu v moderním slova smyslu \cite[Gellner1983]. Řecká komunita neměla jednotné politické ani kulturní centrum a přírodní podmínky a charakter individuálních řeckých obcí nenahrával vytvoření panhellénské identity, která by si uchovala permanentní charakter. Panhellénská identita se objevovala v řecké kultuře spíše epizodicky, jako reakce na setkávání s jinými kulturami, či jako důsledek všeřeckých náboženských slavností.} Dle antropologické teorie dochází k formování příslušnosti k určité skupině dvěma možnými způsoby: první tzv. opoziční identita se formuluje při setkání s jinou skupinou, např. v průběhu válečných konfliktů, diplomatického vyjednávání, či obchodních transakcí (Barth 1969; Malkin 2014). Druhá, tzv. aggregační identita vzniká uvnitř komunity afiliací s dalšími jejími členy, např. v průběhu panhellénských slavností, či v prostředí multikulturní společnosti řeckých {\em emporií} a {\em apoikií} (Morgan 2001; Demetriou 2012; Vlassopoulos 2013).

Panhellénská identita hrála v řecké společnosti relativně marginální roli, větší důležitost zaujímala spíše přináležitost s bezprostředním regionem, či mateřskou obcí, čemuž nasvědčuje i politické uspořádání řecké komunity. Irad Malkin (2011) správně přirovnal tehdejší situaci k „malým světům”, autonomním jednotkám, navzájem propojeným skrz moře a mořeplavbu. I přes tento fakt propojenosti Středozemního a Černého moře si řecké komunity uchovávaly lokální identity, které měly zásadní formativní charakter na vědomí kolektivní příslušnosti tehdejších obyvatel. Dokonce i řecký jazyk, jakožto jeden z prvků, které mají být společné všem Řekům, byl roztříštěn na lokální dialekty.\footnote{Nicméně na rozdíl od jazyků používaných místní populací, byly tyto lokální dialekty více či méně navzájem srozumitelné, ač tyto rozdíly reflektují sami mluvčí (Morpurgo Davies 2002, 161-168).}

Jak jsem již naznačila, úspěšné uplatnění hellénizačního modelu naráží na své limity z několika důvodů, jako například absence jednotné materiální kultury, která by se dala definovat jako řecká. Dále je to mnohotvárnost a proměnlivost řecké identity, která se lišila nejen v závislosti na čase, ale i místně dle aktuální historicko-politické situace \cite[righttext={, 118}][Woolf1994]. Globální použití hellénizačního modelu opomíjí celou řadu společenských interakcí a ignoruje aktivní roli neřecky mluvícího obyvatelstva v rámci celého procesu a nahlíží na neřeckou kulturu jako na nižší úrovni a primitivnější.

Ač se mnozí badatelé staví {\em a priori} negativně k hellénizaci s tím, že se jedná o překonaný model, který nenabízí nový úhel pohledu, a naopak mnohé skutečnosti zjednodušuje, i přesto je hellénizační rétorika natolik zakotvená v moderní společnosti, že je téměř nemožné tento model zcela ignorovat. Východiskem z dané situace může být přístup sledující jak celospolečenský kontext, tak i faktory a možné motivace ovlivňující mezikulturní kontakt a jeho důsledky, podobně jak navrhuje text {\em Memoranda pro studium akulturace} (Redfield {\em et al.} 1936). Tento alternativní přístup se v prvé řadě musí oprostit od kulturních předsudků a přistupovat k všem zúčastněným stranám nezaujatě, nikoliv pouze z pozice řecké společnosti, jak bylo běžné zcela v duchu tradičního pojetí hellénizace. Podobný trend začíná rezonovat i teoretickými archeologickými pracemi posledních 20 let (Dietler 1997; 2005; Mihailovič a Jankovič 2014, xv-xvi).

\subsubsection[hellénizace-v-thrákii]{Hellénizace v Thrákii}

Ačkoliv jsou akulturační modely dnes považovány za vágní a zastaralé, je nutné vzít v potaz, že koncept hellénizace ovlivnil velkou řadu badatelů ve 20. století a nadále si udržuje určitý vliv v široké veřejnosti a politických hnutích s národnostní tematikou, převážně v balkánských zemích s pozůstatky antických kultur. Určitá forma dualismu a předsudky o kulturní dominanci řecko-římské kultury se nevyhýbají ani badatelům zabývajícím se kontakty Thráků a Řeků (Asheri 1990; Xydopoulos 2004; 2007).

V průběhu 20. století byl koncept hellénizace často využíván k politickým účelům, a to například k ospravedlnění přináležitosti ke vznikajícím panevropským uskupením s poukazem na historii daných zemí a jejich propojenost s hellénským kulturním dědictvím Evropy (Owen 2005, 14; 18-21; Vranič 2014a, 39-42). Zároveň s tím ale na Balkáně existoval trend vyzdvihující předřecké kořeny moderních národů, které do jisté míry romantizoval a heroizoval v duchu moderního nacionalismu. Došlo tak k velmi zajímavé kombinaci těchto nacionalistických směrů poloviny 20. století, které přisuzovaly Thrákům poměrně aktivní roli v procesu hellénizace, a Řeky zmiňovaly mnohdy pouze okrajově (Vranič 2014b, 166-167; Fol a Marazov 1977; Fol 1997, 73).

Hellénizační model byl do nedávné minulosti používán mnoha badateli k popsání kontaktů mezi Řeky a neřecky mluvícími území známého jako Thrákie a důsledků těchto kontaktů (Danov 1976; Samsaris 1980; Boardman 1980; Fol 1997; Bouzek {\em et al.} 1996; 2002; 2007; Archibald 1998; Tsetshladze 2006; Tiverios 2008; Sharankov 2011).\footnote{Pro přehled užití hellénizačního přístupu v balkánské archeologii Vranič (2012, 29-31) a Vranič (2014b, 161-163).} Řekové jsou v rámci hellénizačního přístupu chápáni jako kolonizující nadřazený národ, který přináší kulturu a civilizaci primitivnějším Thrákům \cite[righttext={, 13}][Owen2005]. Hellénizace byla ztotožňována s nárůstem společenské komplexity, či jinými slovy příchodem civilizace, a zaměňována za jevy jako urbanizace, centralizace politické moci apod. Přítomnost řecké materiální kultury, nebo její imitace, byla v archeologických kruzích často interpretována jako důkaz hellénizace obyvatelstva, jejíž intenzita byla určována množstvím nalezených předmětů řecké provenience (Vranič 2014a, 36-37). Převládajícím motivem hellénizace byly ekonomické faktory a vzájemný obchod mezi Řeky a Thráky \cite[righttext={, 31-32}][Vranič2012].

Primárním interpretačním rámcem mnoha prací jsou řecké literární zdroje, které popisují první kontakty s Thráky, které líčí jako násilné a bojovné obyvatele úrodného území (např. Archilochos, Diehls frg. 2, 6 a 51; Owen 2005, 19).\footnote{Podrobněji o charakteru literárních pramenů pojednávám v kapitole 1.} Thrákové jsou následně prezentováni jako divoká stvoření s méně vyvinutým náboženským systémem, kteří ale ochotně přijímají řeckou materiální kulturu a spolu s ní i řeckou identitu \cite[righttext={, 133-150}][Asheri1990]. Jak ale poukazuji v kapitole 1, obraz Thráků v literárních pramenech je do značné míry ovlivněn literárním záměrem konkrétního autora a tehdejší politickou situací, a tudíž je nutné k informacím v nich dochovaným přistupovat nanejvýš kriticky a srovnávat s dalšími dostupnými prameny. Například archeologické práce posledních let dokazují, že první kontakty mezi Řeky a Thráky nebyly pouze násilné, ale že se jednalo o daleko komplexnější systém vztahů, než jak je popisují právě řecké literární zdroje. V některých případech bylo dokonce zaznamenáno soužití těchto dvou skupin pohromadě, bez archeologicky doložených známek násilí (Ilieva 2011; Damyanov 2012).

Mnozí historikové a archeologové považují vrchol hellénizace Balkánu dobu hellénismu, kdy došlo k intenzifikaci kontaktů mezi Řeky a Thráky, a tedy i k navýšení počtu kulturních výpůjček. Hlavní podíl na změnách společenské struktury měly dle Papazoglu (1980) místní elity, které se vědomě snažily imitovat hellénistické vladaře. Tato tendence je stále do jisté míry patrná i v současné literatuře, avšak badatelé se snaží reflektovat postkoloniální teoretické přístupy (Nankov 2008; 2009; 2012; Vranič 2014a; Vranič 2014b). Dle Nankova (2012) zásadní roli ve změně společnosti hráli navrátivší se vojáci, sloužící na dvorech a ve vojscích hellénistických vladařů, kteří se později usídlili v Thrákii (Hérakleia Sintská, Seuthopolis), v kombinaci s ekonomicko-politicky motivovanými snahami místních elit. Bouzek (Bouzek {\em et al.} 1996; 2002; 2007) a Archibald (1998; 2011) pak poukazují na aktivní zapojení řeckých obchodních kontaktů a Řeků sídlících přímo v Thrákii již od 5. st. př. n. l. a přisuzující hlavní roli na změně společenské struktury ekonomickým faktorům.

Hellénizace v thráckém prostředí je tradičně vnímána jako vliv řecké materiální a myšlenkové kultury na thrácké obyvatelstvo a s tím související proměny projevů materiální kultury, i uspořádání společnosti. Tento vliv je povětšinou jednostranně zaměřen a dochází k ovlivňování ze strany řecké kultury směrem ke kultuře thrácké. Práce zabývající se hellénizací v Thrákii se většinou zaměřují na dobu hellénismu a nereflektují návaznost na předcházející či následující období. Řecká i thrácká kultura jsou vnímány jako monolitické celky, bez reflexe regionálních variant, lokálních identit, či variability druhů interakcí. Thrácká společnost je považována, zcela v duchu řeckých literárních pramenů, za méně vyvinutou, a celkově je jí věnováno méně pozornosti.

\subsection[teorie-světových-systémů-a-její-návaznost-na-hellénizaci]{Teorie světových systémů a její návaznost na hellénizaci}

Jako alternativní teoretický směr, který do velké míry vycházel z podobných principů jako hellénizace, vznikla v 70. létech 20. století {\em teorie světových systémů} ({\em world-systems theory}). Teorie světových systémů se snažila vysvětlit změny ve společnosti na základě ekonomické teorie Immanuella Wallersteina (1974). Wallerstein s její pomocí vysvětloval rozdělení ekonomické síly a moci, pracovní síly, ekonomického potenciálu a pohybu zboží v moderním světě. Tato původně čistě ekonomická teorie se stala velmi oblíbenou mezi archeology, zabývajícími se mezikulturními kontakty a distribucí materiální kultury (Frankenstein a Rowlands 1978; Rowlands, Larsen a Kristiansen 1987; Bintliff 1996; Stein 1998; Champion 2005; Harding 2013).\footnote{Tato teorie se stala v archeologii též známá pod pojmem „centrum - periferie” ({\em core-periphery}) a sloužila primárně jako vysvětlení pohybu materiální kultury a dynamiky vzájemných kontaktů společností.}

Základním předpokladem teorie světových systémů je propojenost regionálního a nadregionálního ekonomického systému, jehož jednotlivé části se ovlivňují mezi sebou. Světové ekonomické systémy se skládají z center, periferií a semi-perifierií, přičemž v centrech se koncentruje většina ekonomické síly a politické moci. Tato moc je zajištěna existencí vojska, které zajišťuje hladký chod celého systému a dodává centru potřebnou autoritu. Z periferie do centra proudí pracovní síla a surový materiál, který je pak zpracován specialisty žijícími v centru \cite[righttext={, 4}][Rowlands1987]. Periferie je většinou v rámci teorie vnímána jako pasivní součást systému, která dodává potřebný materiál a levnou pracovní sílu, a je oproti centru ekonomicky i kulturně na nižší úrovni. Centra jsou oproti periferiím považována za společnost vyvinutější na škále společenské komplexity, jako např. stát vs. předstátní uskupení \cite[righttext={, 223-225}][Stein1998]. Systém center a periferií je tak poměrně nestabilní a založený na nerovné distribuci ekonomického kapitálu a moci. Systém center a periferií se vyvíjí v závislosti na vnějších a vnitřních okolnostech: centrum může postupem času degradovat na pouhou periferii, a naopak periferie se může proměnit na semi-periferii, či dokonce na centrum (Frankenstein a Rowlands 1978, 80-81).

V rámci archeologické aplikace je periferie na předstátní úrovni vedena skupinou mezi sebou soupeřících aristokratů, kteří organizují výměnu surového materiálu mezi periferií a centrem výměnou za vlastní prospěch a protislužby (Frankenstein a Rowlands 1978, 76).\footnote{Elity zajišťovaly přísun materiálu do centra, což v tomto případě představuje Středomoří, a zároveň si v periferii udržovaly výsadní postavení, které demonstrovaly právě vlastnictvím luxusních nádob pocházejícím z centra. Za odměnu elitám centrum poskytuje prestižní předměty zhotovené specialisty v centru. Většinou se mohlo jednat o nádoby, šperky či zbraně z drahého kovu, které aristokratům v rámci periferie sloužily jako prestižní předměty, poukazující na jejich vysoké postavení a společenský status (Frankenstein a Rowlands 1978, 76-77; Rowlands 1987, 5). Pouze pokud si elity zajistily kontrolu nad výměnou zboží s centrem, mohly úspěšně kontrolovat redistribuci importovaného luxusního zboží. Distribucí luxusního zboží si náčelníci zajišťovali dostatečný počet následovníků, a zavazovali si tak jejich přízeň a podporu do budoucnosti \cite[righttext={{, 288-296},{, 349-350}}][Sahlins1963, Whitley1991].} Archeologové tak za využití ekonomické teorie vysvětlují na tomto koloběhu materiálu a protislužeb přítomnost luxusních nádob v kmenově řízených společnostech v jihozápadním Německu rané doby železné.

Michael Dietler kritizoval tento model pro přílišné zjednodušování reality a popření aktivní role původního obyvatelstva (2005, 58-61). Dietler porovnával teorii světových systémů s hellénizačním modelem, podobně jako u hellénizace, je centrum prezentováno jako ekonomicky a kulturně jednotné a nadřazené periferii. Dalším bodem Dietlerovy kritiky je zaměření teorie pouze na ekonomické vysvětlení kontaktů a cirkulaci a rozmístění kapitálu, a nikoliv na kulturní a společenské následky vycházející ze setkání s cizí kulturou. Primárním prostředkem kontaktu je u teorie světových systémů obchod, což však nezahrnuje komplexnost kontaktů a složitost vztahů k nimž mohlo docházet.\footnote{Jen pro příklad Colin Renfrew (1986, 8) uvádí jako možné druh mezikulturních interakcí vojenské konflikty, diplomatické vztahy, průzkum nových oblastí, migraci, konkurenčního soupeření apod.} Teorie světových systémů pouze málokdy zahrnuje i jiné než obchodní motivace a způsoby šíření luxusní nádob, jako je například výměna darů v rámci budování sítě často nadregionálních společenských vztahů a kontaktů \cite[Mauss1966].\footnote{Tzv. {\em gift-giving society}, více Mauss (1966).}

Hlavním přínosem použití teorie světových systémů v archeologii bylo kladení většího důrazu na vzájemnou propojenost regionů, a zároveň snaha o vysvětlení nerovnoměrného rozdělení kapitálu na úrovni regionů \cite[righttext={, 379}][Harding2013]. Teorie světových systémů byla poměrně úspěšně aplikována na větší regionální a nadregionální celky, kde bylo možné definovat jedno ekonomické a politické centrum.\footnote{V případě starověkého světa se jednalo o říše rozkládající se na velké ploše jako např. Řím, Mezopotámie, Egypt, či pozdější Mayská říše \cite[righttext={, 5}][Rowlands1987].} Oproti tomu aplikace teorie světových systémů na řecký svět v sobě nese několik problémů: řecky mluvící svět nebyl po většinu své existence centralizovaný a nespadal pod jedno ekonomické a politické centrum \cite[righttext={, 226}][Stein1998].\footnote{John Bintliff se pokusil aplikovat teorii světových systémů na regionální celky v rámci antického Řecka (1996), avšak jeho model zahrnoval poměrně obecné ekonomické a produkční trendy a celková použitelnost tohoto modelu pro studium mezikulturních kontaktů byla velmi omezená.}

I přes na svou dobu inovativní postoj bývá teorie světových systémů kritizována pro svou aplikaci moderní ekonomické teorie na antickou ekonomiku, která jednak nedosahovala měřítek moderních ekonomik, ale pravděpodobně se řídila i jinými pravidly a měla odlišnou dynamiku, než moderní společnost \cite[righttext={, 5-7}][Hodos2006].

\subsubsection[teorie-světových-systémů-v-thrákii]{Teorie světových systémů v Thrákii}

Ekonomicky motivované interpretace archeologického materiálu v sobě spojují jednak teorii světového systému, ale nesou i prvky hellénizačního přístupu. Hlavní motivací mezikulturního kontaktu je obchod a výměna materiálu, případně absence surového materiálu v řeckých městech a jeho nadbytek v oblastech obývaných Thráky \cite[righttext={, 32-36}][Vranič2012]. Oblasti Thrákie bývá v nadregionálních studiích zabývajících se teorii světových systémů věnováno poměrně málo prostoru, většinou jako zmínka na okraji, či podpůrný argument. V rámci teorie světových systémů je Thrákie chápána jako periferie či jako semi-periferie, v závislosti na měnících se podmínkách a politické situaci, a řecká města na thráckém pobřeží jako centrum \cite[righttext={, 99-104}][Randsborg1994]. Thrákie bývá ve shodě s literárními prameny vnímána jako zdroj surových materiálů, zejména stříbra, zlata a surového dřeva na stavbu lodí, otroků a námezdných vojáků (Hdt. 1.64.1; 5.53.2; Sears 2013, 31; Tsiafakis 2000; Isaac 1986; 14-15; Lavelle 1992, 14-22).\footnote{Podrobněji v kapitole 1.} Lokální studie zaměřující se pouze na Thrákii interpretují hellénizovaná města a obchodní sídliště ({\em emporia}) jako lokální ekonomická centra, kde hellénizované elity či přímo řečtí obchodníci shromažďují a následně dodávají potřebné suroviny z thrácké periferie do řeckého světa (Bouzek 1996; 2002; 2007; Bouzek a Graninger 2015).

Přítomnost luxusních předmětů v thráckém vnitrozemí je vysvětlována jako důsledek výměny zboží a protislužeb mezi centrem a aristokraty z periferie. Z literárních pramenů jsou známy případy, kdy thráčtí aristokraté zprostředkovávali kontakt s řeckými městy, jako například jistý Nymfodóros z Abdéry, který se stal prostředníkem mezi Athénami a thráckými Odrysy (Hdt. 7.137; Thuc. 2.29; Sears 2013, 27). Můžeme jen odhadovat, že odměnou jim byly luxusní předměty řecké provenience, či služby řeckých specialistů, kteří pro ně pracovali přímo v Thrákii.\footnote{Archeologické výzkumy odhalují, že si thrácká aristokracie natolik považovala předmětů řecké provenience, že je ukládala do svých, mnohdy výstavních hrobek \cite[righttext={, 21-25}][Theodossiev2011].} Fenomén bohatých válečnických hrobek s množstvím importovaných předmětů, na nichž byly navršeny monumentální mohyly, je v duchu teorie světových systémů považován za jeden ze znaků kompetitivního charakteru thrácké společnosti, která byla ovládána místními kmenovými náčelníky. Tento fenomén se objevuje v Evropě v době železné, nejen v Thrákii, ale na dalších místech Evropy \cite[righttext={, 102}][Randsborg1994].

Pokud se podíváme na detailněji na povahu kontaktů mezi Řeky a Thráky, teorie světových systémů nemůže sloužit jako univerzální vysvětlení mezikulturních interakcí. Pro ilustraci uvádím rozložení ekonomické a politické moci v 5. st. př. n. l., které se zcela vymyká binárnímu rozdělení na centrum a periferii. O politické a ekonomické dominanci Řeků nad Thráky není možné mluvit zcela jednoznačně, zejména, když literární prameny vyjadřují spíše opak. Thúkýdidés si všímá, že řecká města platila thráckému králi nemalé obnosy a s největší pravděpodobností tudíž nad ním neměla ekonomickou ani politickou moc, ale spíše naopak (Thuc. 2.97; Graham 1992, 61-62). V neposlední řadě měli Thrákové nejen přístup k nerostnému bohatství, ale literární prameny na mnoha místech poukazují na fakt, že Thrákie měla silné a početné vojsko, které pravděpodobně svým počtem převyšovalo vojska jakéhokoliv z řeckých měst na pobřeží Thrákie.\footnote{Thúkýdidés zmiňuje, že v roce 431 př. n. l. měla armáda krále Seutha až 150 000 mužů (Thuc. 2.98).} Podobně tedy jako v případě hellénizace, ani teorie světových systémů nepostihuje mnohdy jemné nuance mezikulturního styku Thráků a Řeků.

\stopcomponent