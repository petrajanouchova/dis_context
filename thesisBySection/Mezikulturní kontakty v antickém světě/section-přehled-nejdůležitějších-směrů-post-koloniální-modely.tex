
\environment ../env_dis
\startcomponent section-přehled-nejdůležitějších-směrů-post-koloniální-modely
\section[přehled-nejdůležitějších-směrů-post-koloniální-modely]{Přehled nejdůležitějších směrů: post-koloniální modely}

V reakci na politický vývoj 20. století a snahu o vyrovnání se s rozpadem evropských koloniálních velmocí se i v akademickém světě objevila celá řada nových teorií a směrů. Viděno novou optikou, dříve aplikované akulturační teorie nevystihovaly komplexnost studií zabývajících se mezikulturními vztahy, a tak bylo nutné hledat nové alternativní přístupy.\footnote{Postkoloniální teoretické přístupy jsou charakterizovány svou vzájemnou provázaností a mnohostranností. Není tedy možné určit jeden převládající teoretický směr, ale jedná se spíše o amalgám několika přístupů, jehož složení se liší dle konkrétního badatele a jeho záměru. Následující výčet charakterizuje pouze nejvýznamnější z hlavních směrů současnosti. Odbornou literaturou poslední doby rezonuje proces formování místní identity jako reakce na vliv nadregionálních uskupení, jako je římské impérium \cite[Mattingly2010]. Podílem římské říše na globalizaci společnosti a následnými projevy na místní kulturu se zabývá směr taktéž známý jako tzv. {\em glokalizace} \cite[Pitts2008]. Dále sem patří směry zabývající se prolínání kultur a vytváření nových forem mezikomunitní komunikace a společenského uspořádání jako je hybridizace, kreolizace \cite[Liebmann2013] či {\em middle-ground theory} (White 1991; Antonaccio 2013). Formováním nových identit na základě kontaktů s jinými kulturami a reprezentací nově vzniklých identit v materiální kultuře se zabývá celá řada badatelů (Hodos 2006; 2010), stejně tak kontextualizací společenských role v rámci původní a nové kultury (Appadurai 1986; Dietler 1997).} Celková nespokojenost s teoretickými přístupy, které se snažily vysvětlit historické procesy jedním možným modelem, vyústila v rozvoj několika paralelních, navzájem se prolínajících teoretických směrů, jejichž zastánci se ve velké míře inspirovali v současné antropologické teorii, sociologii a v exaktních vědách. Hlavními charakteristikami těchto směrů je důraz, který kladou na multidimenzionalitu a variabilitu mezikulturních kontaktů. Více prostoru badatelé věnují národům a společnostem, které byly dříve vnímány pouze v dualistické opozici kolonizátor - kolonizovaný, a jejich různým reakcím na kontakty s jinými kulturami \cite[righttext={{, 25-26},{, 495}}][Dommelen1998, Silliman2013].\footnote{Silliman (2013, 495) popisuje jeden z postkoloniálních směrů, nicméně jeho definice může být použita jako definice celého teoretického směru: „{\em Hybridity in a postcolonial sense tends to be a direct critique of previous versions of colonial theory that considered the effects of colonialism on indigenous people to be those of assimilation, acculturation, or even the more neutrally termed culture change. Hybridity offers a counterclaim of cultural creativity and agency, and it lends more subversion, nuance, and ambiguity than traditional assessments of the effects of colonialism}.”} Vznikly tak zcela nové a unikátní modely a přístupy k archeologickému materiálu, zaměřující se jednak na dynamiku celého procesu, ale i na důsledky kulturních interakcí všech zúčastněných stran. Dřívější čistě ekonomická vysvětlení, či motivy kulturní převahy nejsou zcela zavrženy, ale figurují pouze jako jedno z možných paralelních vysvětlení.

\subsection[lokalizace-a-zaměření-na-místní-obyvatelstvo]{Lokalizace a zaměření na místní obyvatelstvo}

Postkoloniální teoretické přístupy podtrhují fluiditu mezikulturní výměny, hybnou sílu procesů a aktivní roli místní komunity v reakci na mezikulturní kontakty. Nově popsaným fenoménem je vznik tzv. smíšených společností, kdy nově vzniklá kultura v sobě nese prvky obou původních společností, která je neustále obnovována na základě vzájemných interakcí všech zúčastněných stran (White 1991; 2011). Nový pohled na mísení kultur tak nabízí zcela nové interpretace na interagující společnosti - obě kultury jsou nahlíženy jako sobě rovné, chybí prvek dominance jedné z nich a zcela zásadní je zde kreativní prvek, tedy vytváření nové kultury namísto přejímaní kultury dominantní společnosti.

Badatelé zabývající se mezikulturními vztahy hledali nové přístupy k popsání mnohdy velmi komplikovaných situací. Jedním z nejvlivnější přístupů je tzv. {\em middle-ground theory} amerického historika Richarda Whitea (1991; 2011). White, jakožto historik zabývající se kolonizací západní části amerického kontinentu a vzájemnými vztahy mezi Indiány a kolonizujícími bělochy, rozhodně netušil, jak dalekosáhlé důsledky a jak velké uplatnění jeho dílo bude mít i na poli středomořské archeologie. Základní myšlenkou je teorie o místě setkávání dvou kultur, které však nepatří ani do jedné z nich (tzv. {\em middle-ground}). White pojímal {\em middle-ground} jako reálné místo kontaktů, ale i přeneseně jako {\em ad hoc} vzniklý symbolický prostor s prvky z obou zúčastněných kultur, který byl ale plný vzájemných nedorozumění a nových významů (White 2011, xii).\footnote{V původním whiteovském použití {\em middle ground} vzniká jako reakce na obchodní kontakty mezi původními kmeny a příchozími francouzskými kolonizátory, které byly mnohdy plné násilí, vzájemného neporozumění, ale zároveň absence převahy jedné ze zúčastněných stran. Nicméně dokud se obě strany vzájemně potřebovaly, ať už z čistě obchodního hlediska, snažily se dosáhnout určité shody. Tím vznikl poměrně křehký stav, kde se obě strany snažily balancovat vzájemné vztahy a zároveň udržet života schopnou komunitu, např. zajištěním půdy, obživy atp.} {\em Middle-ground} je chápáno jako dočasný fenomén, který zaniká, pokud se jedna ze zúčastněných stran přestane podílet na společenské interakci, či získá dominanci nad druhou stranou \cite[righttext={, 132}][Bayman2010]. White tedy formování {\em middle-ground} chápal jako neustálý proces vytváření symbolického systému porozumění, typický pro daný čas a dané místo, který nelze dost dobře aplikovat na jiné situace (White 2011, xiii). Ač je tento teoretický přístup možné aplikovat pouze na velmi malé množství situací, otevírá zcela nový prostor k interpretacím a poukazuje na aktivní roli obou zúčastněných stran a na specifika mezikulturní výměny.

„{\em Middle-ground theory}” se stala velice oblíbeným interpretačním rámcem i pro prostředí středomořské archeologie a historie (Malkin 1998; 2011; Woolf 2009; Antonaccio 2013). V kontextu Středomoří badatelé interpretují {\em middle-ground} jako novou kulturu, která sice pochází z obou původních kultur, a uchovává si dlouhodobý, avšak proměnlivý, charakter (Malkin 1998; 2011).\footnote{Malkin aplikuje {\em middle-ground} v kombinaci s tzv. {\em network theory}, tedy teorii o decentralizované řecké společnosti, kde k hlavním interakcím dochází v místech setkávání - přístavech a nadregionálních svatyních (2011, 45-48). Pro Malkina je každé takovéto místo setkávání (angl. {\em node}, {\em cluster}) zároveň místem, kde konstantně dochází k formování nové kultury, tedy {\em middle-ground}. Malkinův koncept má spíše popisný než interpretační charakter, avšak i tak výrazně ovlivnil současnou akademickou debatu. Důraz se tak začal klást nejen na propojenost Středozemního prostoru (Malkin {\em et al.} 2009; Constantakopoulou 2007; Archibald 2013, 96-97), ale i na jeho decentralizaci a roli lokálních komunit.} Irad Malkin interpretuje {\em middle-ground} jako nově vznikající kulturu v místech kontaktu, tj. například v koloniích a jejich bezprostředním okolí, či v emporiích, kde docházelo k setkávání velkého množství skupin z odlišných kultur. Malkin, a po jeho vzoru i další badatelé, se uchýlili k tomuto modelu, protože stírá vzájemné odlišnosti a protiklady typu Řek vs. barbar, ale zároveň nechává dostatek prostoru pro různé druhy interakcí a vysvětluje vznik nových smíšených kultur \cite[righttext={, 239}][Antonaccio2013].\footnote{Alternativní pohled představuje Greg Woolf (2009, 224), který chápe {\em middle-ground} jako svět prezentovaný antickými geografy a etnografy, v čele s Hérodotem. Je to svět, který není ani jedním ze dvou světů a často vzniká ze vzájemného neporozumění, vytržení z původního kontextu. Jedná se o zcela odlišnou interpretaci termínu, než jak ho navrhoval White, nicméně Woolfovo pojetí vrhá nový úhel pohledu na naše vnímání antických etnografických textů a jejich relevanci pro popis dané kultury.}

Použitelnost konceptu {\em middle-ground} je v kontextu řecké kolonizace omezená jen na určitý druh situací a nedá se obecně aplikovat na veškeré mezikulturní kontakty. Zásadní přínos {\em middle-ground theory} je odklon o dřívějších binárních modelů, které předpokládaly jednosměrnou výměnu s jasnými výsledky v přijímající kultuře. Poukázání na fakt, že reakce místního obyvatelstva nebyla vždy jasně daná, a na jeho aktivní účast, zcela pozměnila paradigma současného přístupu k mezikulturním kontaktům.

\subsection[kreativní-síla-mezikulturních-kontaktů]{Kreativní síla mezikulturních kontaktů}

Dalším společným bodem postkoloniálních teoretických přístupů je zaměření na kreativní sílu mezikulturních kontaktů a vzájemné ovlivňování všech zúčastněných stran. Interakce mezi jednotlivými společnostmi jsou pojímány v kontextu dané situace a je dán větší prostor variabilitě motivů, které vedly ke setkávání kultur. Prolínání kultur je vnímáno jako dynamický proces, který má celou řadu projevů v obou zúčastněných společnostech. V průběhu posledních 20 let vzniklo několik konceptů, využívajících poznatky z přírodních věd, jejichž aplikace se stala velmi oblíbenou, až téměř módní záležitostí \cite[righttext={, 301-302}][VanValkenburgh2013]. Hybridizace\footnote{Jednotný přístup k {\em hybridizaci}, tedy vytváření hybridní materiální kultury a identit, neexistuje \cite[righttext={, 1-2}][Card2013]. V rámci současného diskurzu je termín {\em hybridizace} používán pro popis důsledků setkávání dvou či více kultur a je vnímán jako neustálý proces vytváření nových významů v rámci se setkávajících kultur, často probíhající na úrovni jednotlivce \cite[righttext={, 33-39}][Bhabha1994]. Termín se stal velmi oblíbeným mezi archeology, protože poskytuje prostředek analýzy materiální kultury, která je mnohdy směsí mnoha kulturních vlivů, ve formě prolínání stylistických a technologických trendů. Hlavní body kritiky {\em hybridizace} spočívají v silných biologických konotacích, které termín vyvolává, dále pak nadužívání termínu bez patřičného teoretického pozadí \cite[righttext={, 493}][Silliman2013], či jeho obsahová vyprázdněnost \cite[righttext={, 1-2}][Card2013].}, kreolizace\footnote{{\em Kreolizace} má svůj původ v lingvistice, kde popisuje proces kontaktu dvou sociolingvistických skupin. Ze specifických interakcí dvou skupin mluvícími rozdílnými jazyky vznikají zcela nová mínění a forma jazyka, zahrnující nejen slovní zásobu, ale i novou strukturu \cite[righttext={{, 2},{, 117}}][Stewart2007, Jourdan2015]. Typickým produktem kreolizace je nový jazyk, který čerpá prvky z různých jazyků ({\em creole} a {\em pidgin}), avšak udržuje si zcela unikátní charakter \cite[righttext={, 118}][Jourdan2015]. V rámci sociologie a antropologie je pak {\em kreolizace} vnímána jako proces vytváření nových významů kdy jedna strana má dominantní pozici, například v rámci nuceného přesídlení obyvatelstva či v diaspoře \cite[righttext={, 40}][Liebmann2013]. V mnoha ohledech je {\em kreolizace} velmi podobná {\em hybridizaci}, zejména s ohledem na vytváření nových významů a forem kultury. Hlavním rozdílem je zaměření {\em hybridizace} na výsledek kulturní výměny a na projevy v dané kultuře, zatímco {\em kreolizace} klade větší důraz na samotný proces změny \cite[righttext={, 119}][Jourdan2015].}, synkretismus\footnote{{\em Synkretismus} se jako termín poprvé objevil již v antice, kde označoval seskupení krétských komunit, které se často dostávaly do vzájemného konfliktu, ale v době nebezpečí se dokázali sjednotit. (Plut. {\em Mor.} 478a-490b.) V moderním pojetí je pojem používaný pro sloučení prvků z několika náboženství v rámci jednoho náboženského systému. V rámci studia mezikulturních kontaktů se {\em synkretismus} taktéž zaměřuje zejména na změny v rámci náboženství \cite[righttext={, 881-882}][Drooger2015]. V rámci moderní antropologie získal termín pejorativní význam, kdy popisoval změnu jako nechtěný důsledek kulturních interakcí, a tak došlo k jeho postupnému vymizení z odborné literatury \cite[righttext={, 28}][Liebmann2013].}, bricolage\footnote{Termín {\em bricolage} se poprvé objevil v díle Levi-Strausse, jako fenomén popisující kreativní přeměnu kulturních prvků způsobenou aktivitou jednotlivců v rámci jedné kultury \cite[righttext={, 29}][Liebmann2013]. Jean Comaroff (1985) rozšířila použití {\em bricolage} i na mezikulturní vztahy a koloniální kontext. {\em Bricolage} se soustředí spíše na vztah společenských struktur a jejich vliv na vytváření nové kultury, namísto archeology oblíbené hybné síly kulturní změny \cite[righttext={, 29-30}][Liebmann2013].} a mísení kultur představují jen výčet pojmů, používaných k popsání určitého druhu kontaktu dvou kultur, kdy dochází k adopci a adaptaci určitých prvků, často za neustálého vytváření a re-formulování kultury nové. Každý z těchto zmíněných termínů, a s nimi souvisejících směrů, se zaměřuje na jiný aspekt mísení kultur, avšak mnohé mají společného: důraz na aktivní zapojení zúčastněných společností, připisování kreativní role interagujícím skupinám a zdůrazňování podílu všech zúčastněných stran při vytváření nové kultury a identity.

Zásadním bodem kritiky těchto „kreativních směrů” je nemožnost určit původ konkrétních kultur, z čehož pramení neschopnost analyzovat jejich vzájemné ovlivňování \cite[righttext={, 119}][Jourdan2015]. Dle Jourdana je kultura sama o sobě neustále se proměňující systém znaků a symbolů, který reaguje na vnější podněty a vyvíjí se, a tudíž je velmi obtížné zpětně vysledovat vzájemné ovlivňování kultur.

Hlavním zastáncem kreativní role lokální komunity v rámci Středozemí je Peter Van Dommelen, který na příkladu kolonizace Sardinie dokazuje, že dlouhodobým působením a vzájemnými kontakty několika kultur docházelo k vytváření místních hybridních materiálních kultur (1998; 2005; Dommelen a Knapp 2010). Materiální kultura dle Dommelena projevuje velkou míru mísení prvků, které jsou používány v novém kontextu a jsou vědomou volbou členů místní komunity. Van Dommelen (2005, 134-138) se primárně dívá na proměny různých součástí materiální kultury, jako je například architektura, keramická produkce, ikonografie soch, související použité technologie, a na materiální projevy rituálů, jako jsou např. dedikační předměty, či svatyně. Velký prostor dává Van Dommelen i procesu formování lokálních identit a aktivnímu odporu místního obyvatelstva v rámci selektivního přijímání konceptů a strukturálních prvků jiných kultur (Van Dommelen 1998, 214-216).\footnote{Van Dommelen 2005, 117: „{\em Cultural hybridity is a concept that has been propagated particularly by Bhabha as a means to capture the “in-betweenness” of people and their actions in colonial situations and to signal that it is often a mixture of differences and similarities that relates many people to both colonial and indigenous backgrounds without equating them entirely with either} \cite[Bhabha1985]”.} V neposlední řadě se zaměřuje nejen na reakce původního obyvatelstva, ale i nově kriticky hodnotí role příchozí komunity v rámci nového přístupu a bez předsudků koloniálních modelů (Van Dommelen 2005, 118).

\subsection[kontextualizace-mezikulturních-kontaktů]{Kontextualizace mezikulturních kontaktů}

Směrem, který má velký ohlas v současných pracích zabývajícími se mezikulturními kontakty, je přístup navrhovaný Michaelem Dietlerem (1997; 1998; 2005). Zatímco Dietler odmítá monolitické modely jako hellénizace či teorie světových systémů pro svou přílišnou obecnost, Dietler se namísto velkých dualistních modelů zaměřuje na formování lokální identity a různé reakce obyvatelstva na kontaktní situace. Dietlerovou inovací je zdůraznění nutnosti sledovat daný fenomén v původním kontextu, s přihlédnutím k roli, jakou daný prvek měl v původní kultuře, a jaké místo zaujímá v kultuře nově formované. Dietler tak navazuje na antropologický směr přisuzující materiální kultuře aktivní roli při formování společenských struktur, jinak též znám jako tzv. {\em social life of things} \cite[Appadurai1986].

Dietlera zajímají zejména důsledky prvotních kontaktů a jejich projevy na materiální kulturu (1998, 218-219).\footnote{Dietler 1998, 218: „{\em Examination of the initial phase of the colonial encounter in Iron Age France is important precisely because it holds the promise of revealing the specific historical processes that resulted in the entanglement of indigenous and colonial societies and how the early experience of interaction established the cultural and social conditions from which other, often unanticipated, kinds of colonial relationship developed.}” Michael Dietler na příkladu archeologického materiálu rané doby železné z oblasti jižní Galie a Germánie podél toku řeky Rhôny sleduje projevy mezikulturní výměny na materiální kulturu. Na tomto území docházelo k setkávání velkého množství skupin, zahrnující např. Etrusky, Féničany, Řeky, Kelty a v neposlední řadě Římany.} Dietlerova interpretace materiálu může být charakterizována jako přístup „zdola”, tedy přistupující přímo k interpretaci materiálních pramenů, oproti obecně pojatým modelům, které k materiálu přistupují „shora”, tedy čtením dochovaných literárních pramenů. Hlavním zkoumaným materiálem je pro Deitlera keramika, zejména keramika používaná ke konzumaci alkoholu v aristokratických kontextech.\footnote{Téma konzumace se ostatně prolíná i Dietlerovou metodologií: Dietler sám používá termín {\em consumption} pro popis kontextu v jakém materiální kultura splňuje své poslání a je využívána ke svému účelu čili jakou roli zaujímá v rámci zkoumané společnosti, případně jak se vyvíjí s měnícími se společensko-politickými poměry (1998, 219-221).} Volba konkrétního stylu, materiálu, symbolického systému, a tedy i jejich případná změna, je pojímána jako vědomý akt, volba na úrovni jednotlivce, či místní komunity, které však může být motivována celospolečenskými jevy. Změna je podmíněna různými motivy, jejichž objasnění není vždy snadné, či dokonce možné. Příkladem může být nádoba považovaná v jedné kultuře za téměř bezcennou, zatímco v kultuře druhé za exotický předmět, jemuž je přisuzována zvláštní hodnota \cite[righttext={, 6-16}][Appadurai1986]. Role, jakou předmět v kultuře hraje, je do velké míry jeho společenskými a symbolickými konotacemi. Aby mohl být předmět vnímán jako hodnotný, a tedy i vhodný ke „konzumaci” v rámci dané komunity, musí splňovat požadavky dané konkrétním společensko-politickým uspořádáním a systémem hodnot. Tento systém hodnot bývá často interpretován jako vkus, či materializovaný projev životního stylu ({\em habitus} v bourdieovské teorii; Bourdieu 1977).

{\em Habitus}, tak jak ho definuje Bourdieu (1977), je souhrnem všech dispozic a předpokladů, které utvářejí životní styl a světonázor každého daného jedince. {\em Habitus} se nevědomě formuje v závislosti na společenských konvencích a struktuře společnosti, které člověka obklopuje, či obklopovala v minulosti. Nevyhnutelným projevem {\em habitu} každého člověka je materiální svět, který si sám vědomě okolo sebe vytváří v závislosti na vkusu a osobních preferencích \cite[righttext={, 173-5}][Bourdieu1984]. Lidé pocházející z podobného prostředí, mají zpravidla podobné zvyklosti a podobá se i materiální kultura, kterou se obklopují. Jakmile dojde ke změně jedné či více z okolností (kontextů), které formují {\em habitus}, musí časem dojít i k proměně materiální stopy, kterou po sobě člověk zanechává \cite[righttext={, 487}][Sapiro2015]. Může se jednat o změnu ekonomických podmínek, změnu společenského statutu, či se jedná o reakci na kontakt s cizí materiální kulturou, či zavedení nové technologie apod. Jinými slovy, plošné změny ve společenském uspořádání by se měly taktéž projevit na materiální kultuře a její produkci.

Kontextualizace, jak ji navrhuje Dietler, představuje poměrně komplexní analytický model, který vyžaduje systematické zhodnocení jak lokálních kontextů, definice role importovaných předmětů, tak i kvantitativní zhodnocení materiální kultury a její a časoprostorové rozmístění \cite[righttext={, 221}][Dietler1998]. Stejný přístup může být velice dobře použit i na studium nápisů a z nich plynoucích společenských změn. Nápisy, v daleko větší míře než keramické nádoby, poskytují informace o struktuře společnosti, místních identitách a proměnách vkusu dané komunity.

\stopcomponent
