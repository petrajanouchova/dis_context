
\environment ../env_dis
\startcomponent chapter-Nápisy a jejich hodnota pro studium antické společnosti
\chapter{Nápisy a jejich hodnota pro studium antické společnosti}
V této kapitole se zabývám výpovědní hodnotou a relevancí informací získaných z epigrafických památek a jejich využití ke studiu antické společnosti. Ton Derks velice příhodně poznamenal, že nápisy jsou často přehlížený druh historického pramene, jehož unikátní výpovědní hodnota bývá často využívána jen jako doplněk historického narativu či datační materiál, což vede k opomíjení jejich plného potenciálu (Derks 2009, 240). Při studiu nápisů vycházím z přesvědčení, že epigrafické památky jako produkt společnosti a jejich členů nesou nenahraditelné informace jak o jedincích, kteří se podíleli na vydávání nápisů, ale i o jejich proměňujícím se vkusu, komunikačních strategiích a postoji k jiným kulturám. Jsem přesvědčena, že studiem nápisů je možné získat jedinečné informace, jak o soukromém životě jednotlivců, tak i o kulturněspolečenském uspořádání a jeho proměnách v závislosti na čase a místě, za předpokladu, že jsou respektována specifika epigrafického materiálu.

\stopcomponent