
\environment ../env_dis
\startcomponent section-materializace-životního-stylu-na-nápisech
\section[materializace-životního-stylu-na-nápisech]{Materializace životního stylu na nápisech}

Historikové mnohdy vnímají nápisy zejména jako prameny pro jejich obsah, v neposlední řadě se též jedná o materiální projev společnosti, vycházející ze společenských tradic a hodnot (Dietler a Herbich 1998, 244-248). Podobně i nápisy jsou produktem společensko-kulturních norem, a ve způsobu jejich provedení, v jejich obsahu, či dokonce v jejich rozmístění a společenské funkci se odrážejí existující vzorce chování dané společnosti.

Podobně jako archeologické památky, či veškerý materiální svět, který nás obklopuje je v klasickém bourdieuovském pojetí možné chápat projevy epigrafické kultury jako jednu z materializací lidského {\em habitu} v okamžiku stvoření ({\em sensu} Bourdieu 1977, 15).\footnote{Je důležité nezaměňovat {\em epigraphic habit} a {\em habitus}. {\em Habitus} je dle klasické definice Pierra Bourdieu výsledkem působení vnějších společenských norem, které jedince obklopují, a stejně tak získaných zkušeností. Tyto vnější okolnosti ovlivňují lidské chování, vkus a následně ovlivňují i materiální kulturu, kterou si každý člověk vytváří okolo sebe \cite[righttext={, 173-5}][Bourdieu1984]. {\em Epigraphic habit} je materiálním a společenským projevem {\em habitu}, jinými slovy se jedná zvyk vydávat důležité zprávy na permanentním médiu a veřejně je vystavovat (MacMullen 1982; Meyer 1990; Bodel 2001, 12-13).} V textu a v materiálním provedení nápisu se odráží jednak životní postoj jednotlivce, ale také jeho vkus, a nepřímo i uspořádání okolní společnosti. Dle této teorie se člověk, který je členem epigraficky aktivní společnosti a sdílí stejné podmínky s ostatními producenty nápisů, jako např. finanční prostředky, přístup k materiálům a technologiím, podobné vzdělání, se s největší pravděpodobností také stane producentem nápisů. Dochází tak k přirozenému postupnému šíření tohoto zvyku jeho nápodobou, což vede k vytváření lokálních variant epigrafických monumentů, ovlivněných vkusem místních komunit. Změny v epigrafické kultuře a její produkci tak mohou reflektovat změny celospolečenského charakteru.

V rámci sledování změn epigrafické produkce sleduji proměňující se identifikaci a identitu jednotlivců, ale i celých skupin v závislosti na proměňujícím se politickém a kulturním kontextu. Analýzou měnících se trendů na poli formování identity se snažím zjistit, zda případná hellénizace, tedy šíření řecké kultury a stylu života, měla zásadní vliv na projevy a uspořádání společnosti, či se jednalo o kombinace faktorů nesouvisejících s řeckou kulturou.

\subsection[identita-a-identifikace-na-epigrafických-památkách]{Identita a identifikace na epigrafických památkách}

Způsob, jakým se jedinec rozhodne na nápisech vystupovat či jakým způsobem je reprezentován, vypovídá mnoho o vztahu, jaký zaujímá k svému nejbližšímu okolí a jak vnímá sám sebe v kontextu hierarchii dané společnosti. Díky tomu, že jsou nápisy „přímým pramenem”, lze se domnívat, že identifikace na epigrafických památkách patří mezi nejautentičtější vyjádření identity jednotlivců a skupin, jaké se nám z~antiky dochovala.

V pojetí identity a sebe-identifikace vycházím z díla sociologa Richarda Jenkinse v nichž je identita chápána jako základní prostředek lidí zasadit vlastní existenci do širšího rámce společnosti a je výsledkem přirozené lidské potřeby orientovat se v mezilidských vztazích (2008, 16-18). V Jenkinsově pojetí identita a reprezentace jednotlivce patří mezi nejzákladnější pojítko mezi člověkem a okolním světem. Kontakty s dalšími členy komunity lidé formují svou identitu. Identifikaci je pak možno chápat jako konkrétní projev identity, což je neustále probíhající obousměrný proces mezi jedincem a komunitou, který v sobě zahrnuje vědomou či nevědomou reflexi vzájemné pozice (Jenkins 2008, 13, 36-). Jinými slovy vědomá identifikace respektuje jedinečnost každého člověka, a zároveň ho napomáhá utvářet celospolečenské vztahy. Identita každého člověka je proměnlivá a vyvíjí se v závislosti na životní situaci. Tato interakce mezi jednotlivcem a komunitou je nikdy nekončící a stále se měnící proces, spolu s~tím, jak je jedinec konfrontován se změnami v~životním stylu a proměnami společnosti okolo něj (Jenkins 2008, 17, 36-48). Tím, že jedinec v rámci komunikace s dalšími lidmi zdůrazní určitou součást identity, dělá to proto, že je to pro něj v danou chvíli určitým způsobem výhodné, či se to ztotožňuje s jeho aktuálním světonázorem. Tím, že se jedinec situuje do určité komunity podobně se identifikujících lidí, si může zajistit bezpečí, ekonomickou soběstačnost, prestiž, nebo jen legitimizuje svou pozici ve společnosti.

Identita každého člověka se skládá z několika částí, z nichž každá složka může hrát důležitou roli v jiné životní situaci a více tak vystupuje na povrch. Každý člověk má potřebu se zároveň vůči společnosti vymezit, definovat svou vlastní identitu, a tím se zároveň zařadit do existujících komunitních struktur.\footnote{Ve své přelomové práci Fredrik Barth (1969) popisuje dynamiku fungování etnických skupin, a to jak vnitřní uspořádání, tak především interkomunitní společenské vztahy na principech stejnosti a odlišnosti. Barth tvrdí, že tyto principy fungují nejen mezi jednotlivými komunitami, ale utvářejí taktéž vnitřní dynamiku skupiny a v konečném důsledku napomáhají i identifikačním procesům jedince. Barthův model je možné použít nejen na popis fungování etnických skupin, ale jakéhokoliv kolektivu a formování identity obecně \cite[righttext={, 119}][Jenkins2008].} Jedinec se může vůči svému okolí vymezovat několika způsoby, jako například volbou použitého jazyka, dále jedinečným poznávacím znamením, jako je osobní jméno, nebo vymezením svého původu a vztahu k okolním komunitám za použití kolektivní identifikace.

\subsection[jazyková-identita]{Jazyková identita}

Nejniternější a základní součástí lidské identity je jazyk, jímž se vyjadřujeme. Volba jazyka, a s ním inherentně spojených kulturních zvyklostí vytváří most mezi interní identitou jednotlivce a strukturami společnosti okolo něj \cite[righttext={{, 143},{, 220-228}}][Jenkins2008, Bourdieu1991]. Jazyk a jeho znalost tvoří zcela jasně dané hranice komunity, která ale může být překračována v případě bilingvních i několikajazyčných mluvčí \cite[righttext={{, 22-25},{, 8}}][Barth1969, Adams2002]. Tito lidé jsou přecházením z jedné skupiny do druhé nuceni adaptovat svou identitu ve vztahu k společenským normám komunity daného jazyka. Přijetí nového jazyka a s ním často spojených společenských norem a zvyklostí, ať už v částečné či úplné podobě, představuje velký posun v rámci identity jedince. Proto se tak většinou nedělo bezdůvodně, ale v rámci adaptace na nové životní podmínky, snahy o zlepšení životní situace \cite[righttext={, 39-41}][Langslow2002] Volba epigrafického jazyka s sebou nesla jasnou zprávu o cílové skupině, které byl nápis určen, o identitě zhotovitele a často ze samotné volby jazyka vyplývá i povaha sdělení. Na příkladu Thrákie je možné vidět, že řecky psané texty určeny spíše místnímu obyvatelstvu, zatímco latinsky psané nápisy směřovaly úřednickému aparátu a státní samosprávě a členům armády středních a vyšších šarží, čemuž odpovídalo i zaměření nápisu.\footnote{Podrobněji se problematikou vztahem volby jazyka a identity v Thrákii zabývám v kapitole 5.} Jak podrobněji rozebírám v kapitole 5 v prostředí Thrákie je tedy volba řečtiny jako epigrafického jazyka spíše projevem komunikační strategie, než univerzálního vyjádření postoje mluvčích a znakem všeobecného přijetí řecké identity.

\subsection[osobní-jméno-a-identifikace-jednotlivce]{Osobní jméno a identifikace jednotlivce}

Osobní jméno je základním způsobem identifikace jednotlivce, pomáhající zasadit člověka do širšího kontextu dané komunity. Jméno může v širším kontextu sloužit jako ukazatel míry společenských tradic a konvencí a původu \cite[righttext={, 8-12}][Horsley1999].\footnote{Osobní jméno se obvykle sestávalo z vlastního jména, dále mohlo obsahovat přezdívku, zpravidla vycházející z charakteristické vlastnosti nositele, která sloužila pro lepší identifikace konkrétního člověka v rámci komunity \cite[righttext={{, 208},{, 47-64}}][Morgan2003, García-Ramón2007].} Jména si totiž většinou lidé sami nevybírají, ale je jim přiděleno nedlouho po narození na základě společenských zvyků dané komunity. Pomocí podoby osobního jména jsme schopni rozlišit pohlaví, případně i původ a společenské zařazení dané osoby \cite[righttext={, 20}][Morpurgo-Davies2000]. V průběhu života se nicméně může měnit jako reakce na důležité změny v životě nositele, jako např. přijmutí náboženství, udělení občanství, či sňatek.

V archaické a klasické době se jména většinou dědila v komunitách po dlouhé generace a nedocházelo k větším proměnám fondu jmen.\footnote{Nepsané pravidlo v řecké společnosti určovalo, že prvorozený syn obvykle dostával jméno po svém dědovi z otcovy strany, což by v praxi znamenalo střídavou linii jmen, např. Athénagorás - Hipparchos - Athénagorás - Hipparchos atd. \cite[righttext={{, 135},{, 76}}][Hornblower2000, McLean2002]. Druhorozený syn naopak dostával jméno po dědovi z matčiny strany.} V této době jména naznačovala na zařazení jeho nositele do širšího etno-kulturního rámce. Současní badatelé o antické prosopografii mají k dispozici obsáhlé databáze osobních jmen, které pomáhají určit, v jakých dobách a na jakém území se daná jména vyskytovala. Je tak možné alespoň přibližně určit, zda osobní jména spadají do řeckého kulturního okruhu, a tedy byla převážně používána lidmi řeckého původu, či alespoň lidmi, kteří se za Řeky označovali (jinak se též setkáváme s pojmem lingvistický původ jména, Sartre 2007, 199-201).

Od hellénismu docházelo k většímu prolínání jmen různých lingvistických původů jakožto důsledek většího pohybu řeckých, makedonských a v neposlední řadě také římských vojsk. Geografický výskyt jmen napovídá, ve kterých oblastech se uplatňoval vliv jaké skupiny, například jako důsledek kolonizace či bohatých obchodních kontaktů, které se po čase projevily i na výběru s výskytu osobních jmen (Parissaki 2007, 267-282; Habicht 2000, 119-121; McLean 2002, 87-90).

Pro římskou dobu je charakteristická větší fluktuace obyvatelstva, a to zejména vlivem pohybů římské armády, což mělo vliv i na mísení onomastických zvyků. Osobní jména již v této době není možné zcela jasně přisuzovat jednomu etno-kulturnímu rámci, vzhledem k tomu, že docházelo k jejich častému prolínání. I přesto osobní jméno samotné, případně jeho forma, napoví mnoho o jedinci samotném, o jeho kulturním pozadí, pohlaví, ale i o historii jeho rodu a zejména o jeho identitě. Za římské doby totiž lidé často přijímali nová jména jako důsledek udělení římského občanství.\footnote{Systém tří jmen se skládal z {\em praenomen}, osobního jméno, {\em nomen} gentilicium, rodového jméno, a {\em cognomen}, specifického přídavné jméno, a případně jména otce \cite[righttext={, 113-148}][McLean2002]. Tzv. systém tří jmen se rozšířil i mezi místní obyvatelstvo, které si ale většinou nechávalo svá původní jména a jen k nim přidávala jména současného císaře v případě udělení občanství či svobody, jako v případě Marka Ulpia Autolyka, jehož rodina získala občanství za císaře Trajána na nápise {\em I Aeg Thrace} 68 z Abdéry z 2. st. n. l.}

V dlouhodobém horizontu jména signalizují přístup komunit k tradičním hodnotám, jejich udržování a určitou míru konzervativismu, či naopak mohou reflektovat proměny společnosti související se zvýšeným kontaktem s jinými národnostmi a kulturami. Těmito dlouhodobými trendy se zabývám podrobně v kapitole 5 a 6 a zaměřuji se převážně na proměny tradičních onomastických zvyklostí v celé společnosti.

\subsection[původ-jako-prostředek-identifikace]{Původ jako prostředek identifikace}

Označení původu je po osobním jméně druhý nejčastější způsob identifikace jedince a slouží jako bezprostřední zasazení do nejbližší komunity, či naopak jeho vymezení se vůči ní. Původ je vlastní každému člověku, bez ohledu na jeho volbu, a provází ho od narození po celý život. Původ člověka do značné míry utváří jeho pohled na svět a ovlivňuje i projevy v rámci epigrafické produkce. Nejtypičtějším vyjádřením původu je odkaz na biologický či geografický původ dané osoby.

Nejjednodušší označení biologického původu je označení rodičů, případně prarodičů. K identifikaci osoby se v řeckém světě mohlo se však užívat i jméno jiného rodinného příslušníka, např. matky, manžela, bratra \cite[righttext={, 150}][Fraser2000]. Ženy většinou uváděly jméno svého otce a poté jméno manžela, případně jen jednoho z nich \cite[righttext={, 94}][McLean2002].\footnote{Areté, dcera Helléna, manželka Agathénóra, syna Artemidóra, {\em IG Bulg} 1,2 143, nedatovaný nápis z Odéssu.} Pokud to bylo důležité, mohla se přidávat ještě výjimečně informace o předchozí generaci pro zdůraznění rodové linie.\footnote{Makou, dcera Amyntóra, syna Hierónyma, {\em IG Bulg} 12 127, 2. - 3. st. n. l., z Odéssu. Dalším, poněkud méně rozšířeným, způsobem uvádění biologického původu je vypisování genealogických informací o předcházejících generacích, či o tzv. mýtických předcích a zakladatelích obcí, čím člověk legitimizoval své postavení v komunitě \cite[righttext={{, 25-29},{, 64-66}}][Hall2002, Malkin2005].} Specifikace biologického původu zároveň ale hraje roli při legitimizaci majetkoprávních nároků při případných dědických sporech, a proto se v epigrafických pramenech velmi často setkáváme s její veřejnou prezentací na nápisech v době římské.\footnote{Podrobně se biologickým původem na nápisech zabývám v kapitola 5.}

Geografický původ je taktéž nedílnou součástí určení identity jednotlivce, která se však projevuje pouze v určitých situacích. Vyjádření geografického původu se nemusí vztahovat jen k místu, odkud člověk pochází, kde se narodil, ale může naznačovat i místo, k němuž se člověk hlásí a přijal ho za své. Člověk tak může vyzdvihovat své rodné město, ale stejně tak město, ve kterém nyní žije a je pro něj důležité zdůraznit přináležitost ke svému současnému bydlišti.\footnote{Jedinec se může vztahovat ke svému rodnému místu, současnému či minulému bydlišti, místu, odkud pochází jeho předci, jako např. Métrodotos, syn Artémóna, původem z Mandry, {\em I Aeg Thrace} 164, nedatovaný nápis z lokality Mitriko v severním Řecku.} K udávání geografického původu většinou docházelo v kontextu kontaktu s jinou kulturou či společností, kde jedinec potřeboval zdůraznit svůj původ, jako např. cizinec žijící na území jiného státu.

Vyjádření původu na nápisech je jedním z důležitých měřítek o míře konzervativismu dané společnosti. V uzavřených společnostech se setkáváme s malou mírou udávání geografického původu, či lidí cizího původu je pouze minimum, ať už z obavy možných persekucí, či z velmi malé míry fluktuace obyvatelstva. Naopak u společností na vyšším stupni společenské komplexity se setkáváme s vyšším počtem vyjádření geografického původu, což poukazuje na větší pohyb obyvatelstva a mnohonárodnostní složení těchto uskupení.

\subsection[identifikace-s-komunitou]{Identifikace s komunitou}

V průběhu života se člověk nevyhne interakcím s mnohými skupinami lidí. Kolektivní identitu, tedy jakési povědomí o sounáležitosti s kolektivem lidí, lze charakterizovat jako vědomou reflexi vztahu jedince vůči dané komunitě (Jenkins 2008, 103, Cohen 1985, 118). Na rozdíl od jazykové příslušnosti, která je většinou daná jazykovým prostředím, do nějž se narodíme a v~němž žijeme, naše afiliace se zájmovými skupinami je věcí naší vlastní volby, a proto je daleko flexibilnější. Člověk může být členem více komunit najednou, může je opouštět, nebo naopak do nových vstupovat, podle toho k jaké skupině sám cítí přináležitost.

Ke změně kolektivní identity může docházet v případě, že podmínky původní skupiny již nevyhovují nové situaci a dotyčný se již zcela neztotožňuje se zájmy skupiny, či v případě, že nová skupina nabízí všeobecně výhodnější podmínky pro rozvoj jednotlivce \cite[righttext={, 21-25}][Barth1969]. Výhody nejrůznějšího druhu, plynoucí z účasti v těchto spolcích, jsou tedy hlavními důvody, proč se lidé sdružovali a stále sdružují do komunit a proč si přisuzují určitou kolektivní identitu \cite[righttext={{, 10-11},{, 133}}][Morgan2003, Barth1969].

Komunity hrají v životě jednotlivce různě důležitou roli. Existují skupiny, které jsou jednotlivci velice blízké a se kterými přichází do kontaktu každý den, jako např. rodina, přátelé, sousedé apod. Jejich podíl na vytváření identity je zcela zásadní, a proto se objevují na epigrafických záznamech nejčastěji. Skupiny vzdálené, často abstraktního charakteru, či skupiny, s nimiž je i jejich vlastní člen konfrontován zřídka, se na epigrafických památkách objevují v menším počtu. Do této kategorie by spadala např. politická či etnická příslušnost.

\subsubsection[kolektivní-identita-a-legitimizace-společenského-postavení]{Kolektivní identita a legitimizace společenského postavení}

Hlavním rysem kolektivní identifikace je snaha lidí zařadit se do již existující skupiny lidí. Ve snaze ospravedlnit svou pozici ve skupině, člověk přistupuje na její strukturu a hierarchii, do níž se snaží zařadit. Legitimizace vlastního postavení ve společnosti je jedním z klíčových faktorů při formování identity jedince i kolektivu, a jako taková hraje i zásadní roli na epigrafických památkách.

Nápisy měly sloužit jako trvalá připomínka na vykonané skutky a dosažené postavení zainteresovaných lidí a přirozená lidská vlastnost po uznání a zisku společenského postavení tvořila důležitou součást v podstatě každé komunity. Poměrně často se na nápisech setkáváme s prezentací dosažených životních úspěchů, jako jsou zastávané funkce v rámci státní organizace, či udělené pocty patří k nejčastějším z uváděných pozic na epigrafických monumentech, jako např. {\em búleutés}, {\em proxenos}, {\em stratégos} (Van Nijf 2015, 233-243; Heller 2015, 265-266). Vzhledem k četnosti výskytů bylo důležité uvádět i dosažené úspěchy na vojenském poli či vítězství v panhellénských závodech.\footnote{Většina z těchto úspěchů zároveň poukazovala postavení v rámci vyšších vrstev společnosti, protože například profesionálně se věnovat sportu si mohli dovolit jen aristokraté a bohatí lidé, kteří se nemuseli starat o obživu \cite[righttext={, 157-169}][Golden1998].} Lidé rádi zdůrazňovali svou pozici v rámci společenské hierarchie, a z ní vyplývající výhody, a proto se na nápisech často objevují plnoprávní občané, veteráni, či propuštění otroci.\footnote{U propuštěných otroků se většinou uváděl i způsob, jakým své nově nabyté pozice dosáhl \cite[righttext={, 263-272}][Zelnick-Abramovitz2005], kdo ho propustil a z jakého důvodu.}

Skupiny, které se ve starověku podílely na tvorbě kolektivní identity, a tudíž se i objevily na nápisech, tvořily široké spektrum uskupení, které je možné rozlišit do dvou základních směrů: komunity související s chodem a řízením politických uskupení a dále kultovní a náboženské komunity.

\subsubsection[politicko-administrativní-komunity]{Politicko-administrativní komunity}

Politické komunity sdružují lidi nějakým způsobem zapojené do chodu státního uspořádání. Proměny politické identity přímo reflektují strukturální změny tehdejší společnosti, jakožto reformy institucí, hierarchického uspořádání společnosti a rozdělení moci, jako jeden z projevů nárůstu či poklesu společenské komplexity. Stejně tak se na nápisech odráží i míra ztotožnění se se společenským uspořádáním, či naopak jeho odmítnutí.\footnote{Vyjádření politické příslušnosti, či participace na politickém životě komunity, nepřímo zaznamenává vnitřní přesvědčení mluvčího. Tím, že jedinec akceptuje společenské uspořádání, či dokonce se mnohdy podílí na chodu politické jednotky a souvisejících institucí, sám se ztotožňuje s principy a hodnotami, které tato politická autorita představuje.} Nositel se sám označuje za občana dané politické jednotky a hlásí se tak ke svým právům a povinnostem vyplývajícím z členství.\footnote{Na řeckých nápisech se nejčastěji setkáváme s uváděním politické příslušnosti ve formě identifikace s členskou základnou obyvatel, nikoliv s institucemi obce. Politickou autoritu tehdy spíše představovala komunita občanů, tedy konkrétní lidé, a nikoliv abstraktní instituce, budovy, území, aparát \cite[righttext={, 165-166}][Whitley2001].}

V řeckém prostředí se setkáváme se zařazením do struktur a institucí {\em polis} či do jednotlivých kmenových uskupení ({\em ethné}; Morgan 2003, 1). {\em Polis} však nebyla jedinou politickou organizací antického světa. Na nápisech se setkáváme se s afiliací s kmenovými státy ({\em ethné}), uskupeními několika států s centrální autoritou ({\em amfiktyoniemi}), či s menšími samosprávnými jednotkami, které existovaly nezávisle na státním zřízení, pod nějž spadaly např. jednotlivé vesnice ({\em kómai}). V prostředí římské říše je to pak uznání autority Říma, což bývá vyjadřováno nápisy vztyčovanými na počest císaře, ale i vyjmenováváním zastávaných funkcí v administrativním aparátu či armádě (Van Nijf 2015, 233-243). Vojenská identita, a zejména prestiž v rámci vojenské komunity byla velice ceněna, alespoň podle četnosti nápisů patřících členům římské armády \cite[righttext={, 240}][Derks2009].

\subsubsection[kultovní-a-náboženské-komunity]{Kultovní a náboženské komunity}

Kult a víra měly vždy podstatný vliv na formování identity, a nejinak tomu bylo i v antice. Vyjádřením příslušnosti k náboženské komunitě jedince zároveň vyjadřoval i své náboženské přesvědčení a kulturní přináležitost. Nápisy s projevy náboženské příslušnosti nepřímo reflektují rozvrstvení společnosti a přináležitost obyvatelstva do kulturního okruhu. Změny ve variabilitě náboženských projevů jsou pak jedním z nepřímých projevů nárůstu či poklesu společenské, a zejména kulturní komplexity.

Druh zvolené náboženské komunity napovídá mnoho o charakteru víry a vnitřním přesvědčení jednotlivce. Kulty místních božstev vycházejí většinou z lokální tradice, reagují na aktuální potřeby komunity a pomáhají utvářet lokální identitu a pocit sounáležitost na každodenní úrovni (Parker 2011, 25, 221). Kult {\em héroů} se stal velice důležitým sjednocujícím prvkem pro celou komunitu, která se mohla scházet u jeho svatyně. Do kultů místních božstev mohli vstupovat lidé urozeného původu, ale i lidé zcela neurození a chudí, někdy dokonce i ženy a otroci. Mýty a víra obecně osvětlovaly komunitě hierarchii vztahů, propůjčovaly legitimizaci nároků na půdu a území, vysvětlovaly její původ. V antickém světě téměř každá komunita měla svého {\em héroa}, od nějž mohla odvozovat svůj původ a nárokovat si jeho autoritu (Parker 2011, 116, 119). Příkladem mohou být tzv. {\em héroové} zakladatelé ({\em oikistai}), od nichž odvozovala svůj původ celá města svůj původ, jako např. Hagnón a Brásidás v Amfipoli (Thuc. 5.11; Malkin 1987, 228-232).

Vliv na formování identity mají jistě i velká kultovní centra, kde se scházeli lidé z nejrůznějších regionů a měst. Oproti lokálním kultům {\em héroů} měla totiž tato centra multi-komunitní charakter, kam přicházeli věřící z celého Středomoří a Pontu. Tato místa se stala centrem setkávání a kulturní výměny jednotlivých komunit a vzájemného atletického soupeření \cite[righttext={, 39-40}][Vlassopoulos2013]. Účast na těchto slavnostech však byla vzhledem ke své finanční náročnosti omezena spíše na bohatší vrstvy společnosti, a nikoliv na aktivity spojené s každodenními úkony víry. Stejně tak zastávání některých specializovaných funkcí v rámci kultu, jako např. kněží, bylo pouze záležitostí elit. S vykonáváním kultovních funkcí byla spojená prestiž, a někdy i finanční nákladnost, tudíž náboženské úřady často zastávali jen členové aristokracie \cite[righttext={, 51}][Parker2011]. Proto se často setkáme i s nápisy věnovanými kněžími, kteří tak jasně poukazují na své vysoké společenské postavení.

\stopcomponent