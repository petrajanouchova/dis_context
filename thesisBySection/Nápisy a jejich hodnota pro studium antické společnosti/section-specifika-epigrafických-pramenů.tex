
\environment ../env_dis
\startcomponent section-specifika-epigrafických-pramenů
\section[specifika-epigrafických-pramenů]{Specifika epigrafických pramenů}

Nápisy jakožto historický pramen se nacházejí na pomezí psaného pramenu a zároveň archeologického monumentu, což sebou nese mnoho specifických problémů a netradičních interpretačních přístupů, o nichž promluvím podrobněji v následujících sekcích.

\subsection[nápis-vs.-literární-pramen]{Nápis vs. literární pramen}

Nápisy si uchovávají vlastnosti typické jak pro literární tak archeologické prameny. Nápisy a archeologické prameny jsou často chápány jako přímý pramen pocházející přímo od členů společnosti, která jej vytvořila, zatímco literární prameny mohou pocházet od vnějších pozorovatelů (Hansen 2001, 331-343).\footnote{Přímý pramen má de Hansena stejnou formu a nese stejné informace jako v době svého vzniku, čímž se liší od nepřímých zdrojů, které představují literární prameny. Nepřímé prameny se dochovaly ve formě manuskriptů a byly v průběhu staletí již mnohokrát pozměňovány a upravovány. Hansen sám uznává, že toto dělení nepostihuje např. rozdíl mezi literárními a dokumentárními papyry, ale nenabízí přesvědčivou alternativu.}

Analogicky se v oblasti antropologie a lingvistiky setkáváme se dvěma pojmy, rozlišujícími zdroje na základě jejich původu a postoje autora (Pike 1954, 8-28; Cohen 2000, 5). První z nich, zdroj emický popisuje subjekt z pozice vnitřního pozorovatele, v případě člena komunity, který je obeznámen s kulturou, situací a disponuje nenahraditelnými informacemi o fungování společnosti. Oproti tomu etický zdroj je mnohdy nezúčastněný literární tvůrce popisující subjekt z pozice vnějšího pozorovatele. Výhodou informace pocházející z etického zdroje je větší nadhled a možnost srovnání s podobnými subjekty. Na druhou stranu je to přílišný odstup a neznalost prostředí, které mohou vézt k nepochopení situace a v konečném důsledku i ke značně zkreslenému svědectví. Řecky a latinsky psané literární prameny informující o Thrákii, nabízejí pohled na komunitu z venku. Často se tak děje z velmi odlišné perspektivy, která vypovídá více o kultuře autora, než o kultuře popisovaného subjektu. Avšak i přesto literární prameny představují další cenný druh informací o starověkém světě a postihují historické události, které samotné nápisy většinou nezaznamenávají (Bodel 2001, 41-45). Nápisy oproti tomu obsahují cenné a detailní informace o struktuře tehdejší společnosti, jejích proměnách a vzájemném prolínání kultur, které zaznamenali přímo členové dané komunity a je možné na ně nahlížet jako na emický zdroj infromací (McLean 2002, 2). Z toho plyne, že nápisy zprostředkovávají přímý pohled do společnosti, která je vytvořila, a představují tak nenahraditelný zdroj informací o vnitřním uspořádání komunit, identitě jedinců, povědomí o kolektivní sounáležitosti a o dynamice společenských vztahů, s nimiž pracuji podrobněji v kapitole 6 a 7.

\subsection[statický-a-selektivní-obraz-epigrafické-společnosti]{Statický a selektivní obraz epigrafické společnosti}

Ač epigrafické texty představují primární zdroj informací o antické společnosti, jedná se o zdroj značně selektivní a statický. Nápisy jsou produktem daného časového a místního kontextu, který se stává minulostí již v okamžiku publikování, a neslouží jako popis antické společnosti platný pro celou dobu jejího trvání, ale reagují na konkrétní životní či politickou situaci.

Texty nápisů se přímo nevyjadřují ke všem oblastem lidského života, ale drží se zavedených kategorií, jako jsou například funerální či dedikační texty, které mohou být do jisté míry stylizované a navzájem se prolínat (Bodel 2001, 46-47). Některé oblasti společenského života, např. styky s jinými kulturami, se na nápisech objevují jen výjimečně ve formě přímé zmínky.\footnote{To však neznamená, že by ke kontaktům nedocházelo, pouze se neprojevily na epigrafické produkci, či alespoň ne v podobě přímé zmínky v textu či reflexe dané skutečnosti.} Při studiu nápisů je tedy nutné respektovat charakteristické rysy jednotlivých kategorií, ale především se neomezovat na strohou interpretaci založenou na pouhém rozboru textu. Nápisy představují komplexní souhrn kulturních prvků, které jsou všechny materializovaným projevem téže kultury. Proto je vhodné nahlížet na epigrafický monument v jeho širším kontextu, aby se předešlo ztrátě podstatných informací či jejich zkreslení.

Se ztrátou informací souvisí i míra dochování nápisů, která byla ve velké části případů dílem náhody či sekundárního použití kamenných nápisů v novějších stavbách. Richard Duncan-Jones (1974, 360-362) odhaduje na souboru římských nápisů ze severní Afriky, že dnes dochované texty představují zhruba 5 \letterpercent{} původně existujících nápisů.\footnote{Pokud toto překvapivě nízké číslo aplikujeme na situaci v Thrákii s cca 4600 dochovanými nápisy, dostaneme se teoreticky až na 92000 původně existujících nápisů, což se však zdá až příliš nadhodnocené, vzhledem k povaze a množství současných nálezů. I tak zde analyzovaný soubor 4600 nápisů nepředstavuje všechny vytvořené nápisy, ale pouze jejich dochovaný zlomek.} Dochované nápisy nápisů tak zdaleka nepředstavují všechny v minulosti vytvořené nápisy, dokonce ani jejich reprezentativní vzorek, ale jedná se o náhodný soubor, který svému dochování vděčí spíše kvalitě použitého materiálu než obsahu textu.

Předpokládá se, že populace aktivně zapojená do produkce nápisů, tj. epigraficky aktivní část společnosti, představovala pouze malý výsek tehdejší společnosti. Vydávat nápisy na kameni a kovu si mohli dovolit lidé střední a vyšší třídy, a to vzhledem k relativně vysoké ceně materiálu. Alternativní materiály jako keramika či dřevo byly dostupnější, nicméně nápisy se nich hůře dochovávají (McLean 2002, 13-14). Z důvodů finanční náročnosti se příliš často nesetkáváme na nápisech s povoláními vykonávanými lidmi z nižší společenské a ekonomické vrstvy společnosti. Někteří autoři se snaží vypočítat na základě odhadů o velikosti populace a počtu dochovaných nápisů, jaký byl podíl epigraficky aktivních lidí v dané oblasti (Woolf 1998, 99-100; Beshevliev 1970, 64-65). Jednotlivé poměry se liší jak v závislosti na místě a čase a na stupni dochování nápisů, nicméně je možné souhrnně říci, že epigraficky aktivní obyvatelstvo tvořilo menšinu tehdejší populace a obsah a forma dochovaných nápisů poukazuje na zapojení střední až vyšší vrstvy společnosti.

Z výše řečeného plyne, že dochované nápisy představují pravděpodobně pouze zlomek z dříve existující produkce. Texty na kameni a kovu pocházejí převážně od středních a vyšších vrstev společnosti a jsou záznamem stavu v určitém časovém a místním kontextu. I přes tato omezení nesou celou řadu informací o proměnách tehdejší společnosti, jimiž se zabývám zejména v kapitole 6 a 7.

\stopcomponent