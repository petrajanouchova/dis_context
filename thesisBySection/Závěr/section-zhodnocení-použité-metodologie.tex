
\environment ../env_dis
\startcomponent section-zhodnocení-použité-metodologie
\section[zhodnocení-použité-metodologie]{Zhodnocení použité metodologie}

Použitá metodologie kvantitativní a kvalitativní studie epigrafické produkce v místě a čase může být aplikována i na jiná místa antického světa, kde docházelo k setkávání řecké, římské a místní kultury a není pevně vázána pouze na oblast Thrákie. Tato metodologie v sobě kombinuje tradiční přístup k epigrafickému materiálu s moderními teoretickými přístupy napříč příbuznými disciplínami. Syntetické práce srovnávající velké množství epigrafických dat na regionální úrovni nejsou příliš časté, a proto je použitá metodologie pokusem uchopit data získaná z nápisů se všemi jejich nedostatky, jako je například míra nejistoty časového zařazení či nejistota při určování provenience nápisu. Inspirací po metodologické stránce jsou podobné postupy užívané běžně v archeologii, které své uplatnění v historii a epigrafice teprve pomalu hledají. Použitá metodologie tak umožňuje zařadit velké množství nápisů do konkrétního kontextu a sledovat trendy celospolečenského charakteru a jejich vývoj v místě a čase. Podobné metody je možné použít nejen pro nápisy, ale i pro další součásti materiální kultury a případně navzájem tato data srovnávat.

Nespornou výhodou, kterou současná práce disponuje, je široký záběr moderních technologií, které umožňují nápisy analyzovat ve větším měřítku a zapojovat nové přístupy a metody, jako například statistické zhodnocení sledovaných jevů, srovnávání jednotlivých regionů či mapování změn společnosti na časové ose, tak i zapojení do kontextu krajiny a vztahu s lidským osídlením. Epigrafickou produkci je tak možno velmi srozumitelně prezentovat v sérii chronologických map, nebo se zaměřením na jednotlivé aspekty nápisů ve vztahu s dalšími lidskými aktivitami na území Thrákie.

Praktické využití pro spřízněné obory nabízí vytvořená elektronická databáze nápisů {\em Hellenization of Ancient Thrace}, která sjednocuje data z mnoha zdrojů a převádí je do jednotné koherentní formy{\em .} V rámci současných vědeckých principů jsou data z databáze v neupravené podobě k dispozici volně na internetu a případní zájemci je mohou použít v rámci vlastního výzkumu, či k ověření uváděných interpretací. Přímé praktické využití této databáze vidím v rámci jednotlivých archeologických projektů, které si tak mohou poměrně snadno opatřit data o epigrafických nálezech z blízkosti hledané lokality či regionu v již digitalizované podobě a připravené pro další analýzy. V takto kompletní podobě žádná jiná volně dostupná databáze neposkytuje data o místě nálezu nápisu, jeho obsahu a z něj vycházejících interpretacích, jako je datace, funkce a další relevantní informace.

\stopcomponent