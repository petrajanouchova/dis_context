\environment env_dis
\setuphead[chapter]      [style=\tfb]

\startcomponent[Abstrakt]

\title[abstractCZ]{Petra Janouchová - Hellénizace antické Thrákie ve světle epigrafických nálezů}

\setuppagenumber[number=0]

\noindentation 
\subject{\it Abstrakt:} \crlf
Z území antické Thrákie z oblasti jihovýchodního Balkánu pochází více než 4600 převážně řecky psaných nápisů. Tyto nápisy poskytují jedinečný zdroj demografických a sociologických informací o tehdejší populaci, umožňující hodnotit případnou proměnu vzorců chování v reakci na mezikulturní kontakt a vývoj společenské organizace. Řecky psané nápisy bývají považovány za jeden ze základních projevů hellénizace obyvatelstva antické Thrákie, tedy postupného a nevratného procesu adopce řecké kultury a identity. V této práci hodnotím na základě časoprostorové analýzy dochovaných nápisů relevanci hellénizace jako výchozího interpretačního rámce pro studium antické společnosti. Zároveň s tím uplatňuji alternativní přístup, který respektuje jednak specifika epigrafického materiálu, ale i poznatky současného bádání v oblasti mezikulturního kontaktu. Tento metodologický přístup umožňuje podrobně hodnotit produkci nápisů nejen v průřezu staletími, ale i navzájem srovnávat jednotlivé regiony a zapojení tamní populace. Z časoprostorové analýzy nápisů je zjevné, že rozvoj epigrafické produkce v Thrákii nemůže být spojován pouze s kulturním a politickým vlivem řeckých komunit, ale z velké části jde o jev úzce spojený s narůstající komplexitou politické organizace společnosti a s tím souvisejícími proměnami vzorců chování tehdejší populace. Tento jev je patrný zejména v době římské, kdy dochází k významnému rozvoji epigrafické produkce na celém území Thrákie, nikoliv pouze v okolí původně řeckých kolonií, a ve zvýšené míře i k zapojení thrácké populace do procesu publikace nápisů. Použitá metodologie je do velké míry inovativní kombinací řady moderních přístupů z příbuzných disciplín, za zachování tradičních principů epigrafické práce. Využití digitální technologie umožňuje studovat nápisy z nové perspektivy a umístit je do regionálního kontextu, což bylo dříve jen velmi obtížné. 

\noindentation 
\subject{\it Klíčová slova:} \crlf
nápisy; epigrafika; epigrafická produkce; Thrákie; hellénizace; kulturní kontakt; změny společnosti; komplexní společnost; digital humanities


\stoptext