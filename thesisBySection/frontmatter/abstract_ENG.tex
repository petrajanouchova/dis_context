\environment env_dis
\setuphead[chapter]      [style=\tfb]

\startcomponent[Abstrakt]

\title[abstractCZ]{Petra Janouchová - Hellenisation of Ancient Thrace based on epigraphical evidence}

\setuppagenumber[number=0]

\noindentation 
\subject{\it Abstract:} \crlf
More than 4600 inscriptions in the Greek language come from Thrace, the area located in the Southeastern Balkan Peninsula. These inscriptions provide socio-demographic data, allowing the study of changing behavioural patterns in reaction to cross-cultural interactions. Traditionally, one of the essential indications of the influence of the Greek culture on the population of ancient Thrace was the practice of commissioning inscriptions in the Greek language. By using quantitative and systematic analysis, the inscriptions can be studied from a new perspective that places them into broader regional context. I use this methodology to assess the concept of Hellenization as one of the possible interpretative frameworks for the study of ancient society. Using a spatiotemporal analysis of inscriptions, this research shows that epigraphic production cannot be solely linked with the cultural and political influence of Greek speaking communities. However, the phenomenon of epigraphic production is closely connected to the growth of social complexity and consequent changes in the behavioural patterns of the population. The growth in social complexity is followed by an increase of epigraphic production of public and private character alike; while at the time of socio-economic crisis and political unrest, the production of inscriptions significantly drops. The sudden change in the character of epigraphic production is obvious in the Roman period, where the production substantially intensifies as a result of growing role of sociopolitical organization. Moreover, the Thracian population becomes more involved in the whole process of commissioning inscriptions as a result of their participation in the civic and military service. The spatiotemporal analysis of inscriptions allows the discussion of the societal function of the epigraphic production over time, places them into broader regional context, and evaluates the degree of involvement of the population in the practice of commissioning of inscriptions. This research combines the specifics of epigraphic evidence and the current scholarship on crosscultural contact. This innovative methodological approach combines a range of modern theoretical concepts from related disciplines, such as archaeology and anthropology, while maintaining the epigraphic discipline best practices. 
 

\noindentation 
\subject{\it Keywords:} \crlf
inscriptions; epigraphy; epigraphic production; Thrace; hellenization; cultural contact; social change; complex society; digital humanities

\stoptext