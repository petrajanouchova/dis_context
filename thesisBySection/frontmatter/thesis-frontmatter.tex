{\em 2017 Mgr. Petra Janouchová}

{\bf Univerzita Karlova}

{\bf Filozofická fakulta}

Ústav řeckých a latinských studií

studijní program: Historické vědy

studijní obor: Dějiny antického starověku

Disertační práce

{\em Hellénizace antické Thrákie ve světle epigrafických nálezů}

{\em Hellenization of Ancient Thrace based on epigraphic evidence}

{\em PhDr. Jan Souček, CSc.}

{\em 2017 Mgr. Petra Janouchová}

{\bf Prohlášení:}

Prohlašuji, že jsem disertační práci napsal/a samostatně s využitím pouze uvedených a řádně citovaných pramenů a literatury a že práce nebyla využita v rámci jiného vysokoškolského studia či k získání jiného nebo stejného titulu.

V Praze, dne

\ldots{}\ldots{}\ldots{}\ldots{}\ldots{}\ldots{}\ldots{}\ldots{}\ldots{}\ldots{}\ldots{}...

Jméno a příjmení

{\bf Abstrakt:}

Z území antické Thrákie z oblasti jihovýchodního Balkánu pochází více než 4600 převážně řecky psaných nápisů. Tyto nápisy poskytují jedinečný zdroj demografických a sociologických informací o tehdejší populaci, umožňující hodnotit případnou proměnu vzorců chování v reakci na mezikulturní kontakt a vývoj společenské organizace. Řecky psané nápisy bývají považovány za jeden ze základních projevů hellénizace obyvatelstva antické Thrákie, tedy postupného a nevratného procesu adopce řecké kultury a identity. V této práci hodnotím na základě časoprostorové analýzy dochovaných nápisů relevanci hellénizace jako výchozího interpretačního rámce pro studium antické společnosti. Zároveň s tím uplatňuji alternativní přístup, který respektuje jednak specifika epigrafického materiálu, ale i poznatky současného bádání v oblasti mezikulturního kontaktu. Tento metodologický přístup umožňuje podrobně hodnotit produkci nápisů nejen v průřezu staletími, ale i navzájem srovnávat jednotlivé regiony a zapojení tamní populace. Z časoprostorové analýzy nápisů je zjevné, že rozvoj epigrafické produkce v Thrákii nemůže být spojován pouze s kulturním a politickým vlivem řeckých komunit, ale z velké části jde o jev úzce spojený s narůstající komplexitou politické organizace společnosti a následnými proměnami vzorců chování tehdejší populace. Tento jev je patrný zejména v době římské, kdy dochází k významnému rozvoji epigrafické produkce na celém území Thrákie, nikoliv pouze v okolí původně řeckých kolonií, a ve zvýšené míře i k zapojení thrácké populace do procesu publikace nápisů. Použitá metodologie je do velké míry inovativní kombinací řady moderních přístupů z příbuzných disciplín, avšak za zachování tradičních principů epigrafické práce. Zapojení digitální technologie umožňuje studovat nápisy z nové perspektivy a umístit je do regionálního kontextu.

{\bf Absctract:}

More than 4600 inscriptions in the Greek language come from Thrace, the area located in the Southeastern Balkan Peninsula. These inscriptions provide socio-demographic data, allowing the study of changing behavioural patterns in reaction to cross-cultural interactions. Traditionally, one of the essential indications of the influence of the Greek culture on the population of ancient Thrace was the practice of commissioning inscriptions in the Greek language. By using quantitative and systematic analysis, the inscriptions can be studied from a new perspective that places them into broader regional context. I use this methodology to assess the concept of Hellenization as one of the possible interpretative frameworks for the study of ancient society. Using a spatiotemporal analysis of inscriptions, this research shows that epigraphic production cannot be solely linked with the cultural and political influence of Greek speaking communities. However, the phenomenon of epigraphic production is closely connected to the growth of social complexity and consequent changes in the behavioural patterns of the population. The growth in social complexity is followed by an increase of epigraphic production of public and private character alike; while at the time of socio-economic crisis and political unrest, the production of inscriptions significantly drops. The sudden change in the character of epigraphic production is obvious in the Roman period, where the production substantially intensifies as a result of growing social complexity. Moreover, the Thracian population becomes more involved in the whole process of commissioning inscriptions as a result of their involvement in the civic and military service. The spatiotemporal analysis of inscriptions allows the discussion of the societal function of the epigraphic production over time, places them into broader regional context, and evaluates the degree of involvement of the population in the practice of commissioning of inscriptions. This research combines the specifics of epigraphic evidence and the current scholarship on crosscultural contact. This innovative methodological approach combines a range of modern theoretical concepts from related disciplines, such as archaeology and anthropology, while maintaining the epigraphic discipline best practices.

{\bf Klíčová slova:} nápisy; epigrafika; Thrákie; hellénizace; kulturní kontakt; digital humanities

{\bf Keywords:} inscriptions; epigraphy; Thrace; hellenization; cultural contact; digital humanities

{\bf Obsah}

\stopcomponent