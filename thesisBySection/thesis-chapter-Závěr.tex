
\chapter{Závěr}
Epigrafické památky sloužily v pracích zaměřených na hellénizaci neřecky mluvících obyvatel jako měřítko a zároveň prostředek naprostého pořečtění obyvatelstva, a to bez další analýzy. Kritické zhodnocení epigrafických památek za účelem identifikace společenských změn byla v~minulosti reflektována jen velice okrajově (Mihailov 1977, 343-344; Sharankov 2011, 135-145). Pouhá přítomnost řecky psaných textů na území Thrákie sloužila jako důkaz adopce řecké kultury, společenské organizace, a v souvislosti s tím i řecké identity. Hellénizace obyvatelstva byla vnímána jako nevyhnutelný proces, který začal s příchodem řeckých kolonistů v 7. a 6. st. př. n. l., pokračoval v hellénistické době s aktivitami místní aristokracie a nabyl ještě větší intenzity pod římskou nadvládou. Zvyk vydávat nápisy byl v tehdejší společnosti natolik zakořeněn, že nevymizel ani s oslabením politického a kulturního vlivu řeckých obcí v 1. st. př. n. l. a 1. st. n. l., ale naopak se změnou politického uspořádání za římské nadvlády v následujících stoletích ještě zesílil. Tento jev bývá někdy souhrnně nazýván termínem římská hellénizace, tedy jakési zintenzivnění hellénizačního procesu za pomoci infrastruktury římské říše (Vranič 2014, 39).

