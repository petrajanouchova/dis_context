
\section[alternativní-přístup-k-mezikulturnímu-kontaktu]{Alternativní přístup k mezikulturnímu kontaktu}

Archeologové, stejně jako historikové a epigrafici, do velké míry čelí omezenosti pramenů samotných: za prvé, nelze archeologicky prozkoumat naprosto vše, a proto se téměř vždy se jedná o generalizace založené na částečných znalostech minulé skutečnosti. Dalším problémem je náhodné dochování archeologických či epigrafických pramenů, s nímž je nutné se vyrovnat. Jinými slovy obraz minulosti, který archeologie skládá dohromady, nemusí být reprezentativní, ani přesný. Představené metody jsou snahou se s těmito nedostatky vyrovnat a zároveň reflektují i soudobé teoretické trendy a vývoj moderní společnosti, které v mnohém ovlivnily konečnou podobu diskutovaných teoretických směrů. Proto ani sebelepší teoretický přístup nemůže postihnout minulost v její úplnosti, ale může alespoň přiblížit jeden z aspektů lidských dějin.

Současné teoretické přístupy nabízejí jednak analytický přístup hodnotící kontaktní situace bez kulturně podmíněných předsudků, na rozdíl od přístupů založených na tradičně pojímaných akulturačních modelech. Velký přínos pro studium společnosti a dynamiky jejího fungování má tzv. postkoloniální teoretický rámec, který se štěpí do několika směrů zaměřujících se na konkrétní aspekt kulturních interakcí a změn.

Možným východiskem plurality přístupů k mezikulturnímu kontaktu a nejednoznačnosti materiální kultury je kombinace několika metod, které v sobě kombinují kritické zhodnocení rolí obou zúčastněných společností v daných kontextech, specifických pro jednu či druhou kulturu, spolu se systematickým studiem dochovaných materiálních pramenů (Dietler 1998; 2010). Přítomnost řeckých importů v thráckém vnitrozemí již není automaticky vysvětlována jako jasný znak hellénizace thráckého obyvatelstva a kulturní převahy řeckých komunit, ale badatelé kriticky přihlíží k společensko-historickým kontextům, které by vysvětlili přítomnost materiální kultury tradičně klasifikované jako „řecké” v neřeckém prostředí (Nankov 2012). Upouští se od přiřazovaní jednotlivých materiálních kultur a pohřebních rituálů jako určujících znaků etnické či jinak „kulturní” příslušnosti (Damyanov 2012), a autoři začínají plně uznávat komplexnost mezikulturních a mezikomunitních kontaktů.

V posledních dvou desetiletích se někteří badatelé odklonili od tradičních koloniálních přístupů, zdůrazňujících polaritu kulturních kontaktů, směrem k pluralitě možných interakcí obyvatel Thrákie různého kulturního původu. Zcela v duchu postkoloniálních přístupů badatelé přistupují k analýze materiálu bez kulturních předsudků (Dietler 2005). V reakci na postkoloniální přístupy přestávají literární prameny sloužit jako jediný interpretační rámec, podle nějž se archeologické prameny musí nutně řídit, ale jsou konzultovány spíše jako ilustrativní, sekundární zdroj informací, který navíc může v mnohém s archeologickými prameny nesouhlasit (Ilieva 2011, 25-43). Společenské uspořádání antické thrácké společnosti a případné změny v interní struktuře společnosti v reakci na kontakty s vnějším světem, ať už řeckým, makedonským, či později římským, se stávají centrálním tématem některých archeologických projektů, ač zatím pouze v regionálním měřítku (Sobotková 2013; Archibald 2015, 393-395). Na tyto a podobné projekty chci navázat v současné práci analýzou epigrafických pramenů, které, ač jsou produktem stejné společnosti, která produkovala archeologicky zkoumaný materiál, bývají mnohdy studovány odděleně, což vede ke ztrátě znatelné části jejich výpovědní hodnoty. Proto kulturně nezatížené zasazení dochovaných epigrafických památek do širšího archeologického a historického může být klíčem k pochopení dynamiky společnosti antické Thrákie.

