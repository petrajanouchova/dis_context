
\section[charakter-epigrafické-produkce-v-thrákii-v-předřímské-době]{Charakter epigrafické produkce v Thrákii v předřímské době}

Epigrafická kultura se v Thrákii v plné míře prosadila až v době římské, jako jeden z projevů rozvinuté komplexní společnosti a státní organizace. Do té doby můžeme sledovat dva zcela odlišné přístupy k nápisné kultuře, které do jisté míry vycházely z rozdílného kulturního pozadí pobřežních a vnitrozemských komunit. Na pobřeží Černého, Marmarského a Egejského moře byla epigrafická produkce ovlivněná přímo řeckou přítomností na tomto území a vykazovala rysy typické pro řeckou kulturu, zatímco obyvatelé thráckého vnitrozemí přistupovali k nápisné kultuře a písmu obecně velmi odlišně.

