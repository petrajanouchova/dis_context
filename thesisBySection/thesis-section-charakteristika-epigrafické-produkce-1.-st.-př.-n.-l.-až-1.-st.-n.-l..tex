
\section[charakteristika-epigrafické-produkce-1.-st.-př.-n.-l.-až-1.-st.-n.-l.]{Charakteristika epigrafické produkce 1. st. př. n. l. až 1. st. n. l.}

Nápisy datované do 1. st. př. n. l. až 1. st. n. l. se vyznačují narůstající otevřeností a multikulturalitou epigraficky aktivních komunit. Tomu nasvědčuje i prolínání onomastických tradic, rozvoj místního náboženství, ale i kultů nethráckého původu. Epigrafická aktivita je celkově nižší oproti předcházejícím stoletím, avšak narůstá celkový počet osob vystupujících na nápisech. Thrácká aristokracie se nicméně z epigrafických dokumentů vytrácí a stejně tak i ustávají epigrafické aktivity v thráckém vnitrozemí.

\placetable[none]{}
\starttable[|l|]
\HL
\NC {\em Celkem:} 56 nápisů

{\em Region měst na pobřeží:} Abdéra 3, Anchialos 1, Bizóné 1, Byzantion 30, Dionýsopolis 1, Maróneia 8, Mesámbria 1, Odéssos 3, Perinthos (Hérakleia) 2, Sélymbria 2, Séstos 1, Topeiros 1 (celkem 54 nápisů)

{\em Region měst ve vnitrozemí:} 0\footnote{Celkem dva nápisy nebyly nalezeny v rámci regionu známých měst, editoři korpusů udávají jejich polohu vzhledem k nejbližšímu modernímu sídlišti či muzeu, kde se v současnosti nacházejí.}

{\em Celkový počet individuálních lokalit}: 19

{\em Archeologický kontext nálezu:} sídelní 1, náboženský 1, sekundární 5, neznámý 49

{\em Materiál:} kámen 54 (mramor 52, neznámý 2), neznámý 2

{\em Dochování nosiče}: 100 \letterpercent{} 4, 75 \letterpercent{} 2, 50 \letterpercent{} 5, 25 \letterpercent{} 10, nemožno určit 35

{\em Objekt:} stéla 48, architektonický prvek 4, socha 2, neznámý 2

{\em Dekorace:} reliéf 39, bez dekorace 17; reliéfní dekorace figurální 32 nápisů (vyskytující se motiv: jezdec 3, sedící osoba 1, skupina lidí 1, zvíře 1, funerální scéna/symposion 7), architektonické prvky 8 nápisů (vyskytující se motiv: naiskos 3, sloup 2, báze sloupu či oltář 2, architektonický tvar/forma 1)

{\em Typologie nápisu:} soukromé 45, veřejné 9, neurčitelné 2

{\em Soukromé nápisy:} funerální 35, dedikační 9, neznámý 1

{\em Veřejné nápisy:} seznamy 1, honorifikační dekrety 6, státní dekrety 1, neznámý 1

{\em Délka:} aritm. průměr 4,14 řádku, medián 2, max. délka 48, min. délka 1

{\em Obsah:} dórský dialekt 2, latinský text 1 nápis, písmo římského typu 1; hledané termíny (administrativní termíny 10 - celkem 16 výskytů, epigrafické formule 5 - 11 výskytů, honorifikační 6 - 8 výskytů, náboženské 11 - 18 výskytů, epiteton 4 - počet výskytů 5)

{\em Identita:} řecká božstva 2, egyptská božstva 3, pojmenování míst a funkcí typických pro řecké náboženské prostředí, místní thrácká božstva, regionální epiteton 1, subregionální epiteton 3, kolektivní identita 14 termínů, celkem 14 výskytů - obyvatelé řeckých obcí z oblasti Thrákie 10, ale i mimo ni 2, kolektivní pojmenování Thráx 1, Rómaios 1; celkem 122 osob na nápisech, 31 nápisů s jednou osobou; max. 46 osob na nápis, aritm. průměr 2,18 osoby na nápis, medián 1; komunita multikulturního charakteru se zastoupením řeckého, římského a thráckého prvku, jména pouze řecká (33,9 \letterpercent{}), pouze thrácká (3,57 \letterpercent{}), pouze římská (8,92 \letterpercent{}), kombinace řeckého a thráckého (10,71 \letterpercent{}), kombinace řeckého a římského (7,14 \letterpercent{}), kombinace thráckého a římského (3,57 \letterpercent{}), kombinovaná řecká, thrácká a římská jména (5,35 \letterpercent{}), jména nejistého původu (12,49 \letterpercent{}), beze jména (14,28 \letterpercent{}); geografická jména z oblasti Thrákie 9, geografická jména mimo Thrákii 1;

\NC\AR
\HL
\HL
\stoptable

Oproti nápisům datovaným do 2. až 1. st. př. n. l. je u skupiny nápisů datovaných do 1 st. př. n. l. až 1. st. n. l. pozorovatelný pokles o 45 \letterpercent{} celkového počtu nápisů. Produkční centra se nachází výhradně na pobřeží, jak je možné vidět na mapě 6.06 v Apendixu 2. Hlavním produkčním centrem je i nadále Byzantion, odkud pochází přes polovinu všech nápisů. Pozici menšího produkčního centra si i nadále udržuje Maróneia s osmi nápisy, což představuje 14 \letterpercent{} celkového počtu nápisů. Materiálem, z nějž jsou nápisy zhotovovány, je výhradně kámen, většina nápisů má tvar stély a slouží jako funerální nápis. Zhruba 16 \letterpercent{} nápisů představují nápisy veřejné, což je nepatrně větší zastoupení než v předcházejícím období.

