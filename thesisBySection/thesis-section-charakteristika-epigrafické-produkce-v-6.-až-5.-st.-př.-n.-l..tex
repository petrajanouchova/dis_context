
\section[charakteristika-epigrafické-produkce-v-6.-až-5.-st.-př.-n.-l.]{Charakteristika epigrafické produkce v 6. až 5. st. př. n. l.}

Nápisy datované do 6. a 5. st. př. n. l. pocházejí rovněž z pobřežních oblastí Černého a Marmarského moře z komunit řeckého původu. Typologicky se taktéž jedná o soukromé nápisy funerálního charakteru, které si udržovaly původní charakter a nedocházelo k prolínání s thráckou kulturou.

\placetable[none]{}
\starttable[|l|]
\HL
\NC {\em Celkem:} 3

{\em Region měst na pobřeží:} Apollónia Pontská 1, Perinthos (Hérakleia) 2

{\em Region měst ve vnitrozemí:} 0

{\em Celkový počet individuálních lokalit:} 3

{\em Archeologický kontext nálezu:} funerální 1, neznámý 2

{\em Materiál:} kámen 3 (mramor 1, neznámý 2)

{\em Dochování nosiče}: 100 \letterpercent{} 1, nemožno určit 2

{\em Objekt:} stéla 3

{\em Dekorace:} reliéfní dekorace 3 nápisy; figurální dekorace 3 nápisy (vyskytující se motiv: stojící osoba 2, sedící osoba 1, zvíře 1); architektonická dekorace 1 nápis (vyskytující se motiv: naiskos 1)

{\em Typologie nápisu:} soukromé nápisy 3, veřejné 0

{\em Soukromé nápisy}: funerální 3

{\em Veřejné nápisy:} 0

{\em Délka:} aritm. průměr 3,3 řádky, medián 3, max. délka 4, min. délka 3

{\em Obsah:} bez hledaných termínů, náhrobní kameny vzpomínající na zemřelého

{\em Identita:} pouze řecká jména, krátké texty, 1 osoba na nápise

\NC\AR
\HL
\HL
\stoptable

Do 6. a 5. st. př. n. l. jsou datovány celkem tři nápisy, které pocházejí z regionu řeckých kolonií na pobřeží Marmarského a Černého moře, tedy do stejných oblastí jako skupina nápisů datovaných do 6. st. př. n. l., jak ilustruje Mapa 6.01 v Apendixu 2.

