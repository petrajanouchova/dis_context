
\section[charakteristika-epigrafické-produkce-v-6.-st.-př.-n.-l.]{Charakteristika epigrafické produkce v 6. st. př. n. l.\footnote{Dva potenciálně nejstarší nápisy z území Thrákie pocházejí z řeckého města Byzantion, které bylo založeno jako kolonie řecké Megary v roce 667 př. n. l. Dochované nápisy {\em IK Byzantion} 42 a 53a je možné datovat pouze velice široce: zhruba od poloviny 7. st. př. n. l. až do doby klasické, respektive hellénistické. Jejich datace je natolik široká, že je nelze přiřadit k jednomu či dvěma konkrétním stoletím. Protože se však může jednat o první řecky psané epigrafické památky nalezené na území Thrákie, považuji za nutné je zmínit. Typologicky se jednalo o označení váhové míry, a označení délky, v jednom případě s uvedenou příslušností ke komunitě obyvatel Byzantia. Pokud akceptujeme jejich ranou dataci, pak by se mohlo jednat o jeden z prvních projevů politické a ekonomické autority na území Thrákie, a to v rámci řecké komunity. Bohužel tyto dva nápisy jsou tak ojedinělé a nejistě datované, a tudíž jejich výpovědní hodnota je velmi nízká.}}

Nápisy ze 6. st. př. n. l. pocházejí výhradně z řeckých měst na pobřeží Černého, Marmarského a Egejského moře. Jedná se o krátké texty funerálního, a tedy soukromého charakteru. Soudě dle formy a obsahu, nápisy pocházejí čistě z řeckého kulturního prostředí.

\placetable[none]{}
\starttable[|l|]
\HL
\NC {\em Celkem:} 3

{\em Region měst na pobřeží:} Abdéra 1, Apollónia Pontská 1, Perinthos (Hérakleia) 1

{\em Region měst ve vnitrozemí:} 0

{\em Celkový počet individuálních lokalit:} 3

{\em Archeologický kontext nálezu:} neznámý 3

{\em Materiál:} kámen 3 (mramor 1, místní kámen 1 (póros\footnote{Druh porézního kamene s velkým množstvím zkamenělých ulit mořských živočichů.}), 1 neznámý);

{\em Dochování nosiče}: 100 \letterpercent{} 1, 75 \letterpercent{} 1, nemožno určit 1

Objekt: stéla 3

{\em Dekorace:} reliéfní dekorace 1 nápis (vyskytující se motiv: stojící osoba 1); architektonická dekorace 1 nápis (vyskytující se motiv: florální motiv 1, anthemion 1)\footnote{Výsledné číslo součtu nápisů nesoucích dekoraci může být vyšší než celkový počet nápisů, protože některé nápisy nesou více druhů dekorace, případně kombinace motivů.}

{\em Typologie nápisu:} soukromé nápisy 3, veřejné 0

{\em Soukromé nápisy:} funerální 3

{\em Veřejné nápisy:} 0

{\em Délka:} aritm. průměr 3, medián 3, max. délka 3, min. délka 3

{\em Obsah:} standardní epigrafické formule 1 (funerální)

{\em Identita:} pouze řecká jména, 2 ženy, 1 muž, krátké texty, celkem 3 osoby, 1 osoba na nápis

\NC\AR
\HL
\HL
\stoptable

Do 6. st. př. n. l. je možné zařadit tři nápisy s přesností datace na jedno století. Nápisy pocházejí z regionu řeckých kolonií na pobřeží Egejského, Marmarského a Černého moře; konkrétně z regionu obcí Abdéra, Apollónia Pontská a Perinthos, pozdější Hérakleia. Místa nálezu se nacházejí přímo na území daných obcí, a tedy i v bezprostřední blízkosti mořského pobřeží. Vzájemnou polohu nálezových míst ilustruje Mapa 6.01 v Apendixu 2.

Charakteristické datační prvky archaických nápisů se vyskytují na několika exemplářích.\footnote{Směr použitého písma je v některých případech bústrofédon, např. {\em IG Bulg} 1,2 404 z Apollónie Pontské, nebo je používána místní epichórická alfabéta z Thasu/Paru, např. {\em I Aeg Thrace} 30 z Abdéry (Petrova 2015, 9).} Objekty nesoucí nápis jsou vytvořeny z kamene převážně místního původu, jako je póros, a z části z mramoru, jehož původ není znám, ale dá se předpokládat jeho lokální zdroj. Vizuální podoba objektů nesoucích nápis se velmi podobá nápisům pocházejícím z ostatních řecky mluvících komunit té doby (Kurtz a Boardman 1971, 68-90, 121-127; Sourvinou-Inwood 1996, 277-297; Petrova 2015, 9-18).\footnote{Stély v Apollónii Pontské mají prostý tvar obdélníku, případně v Perinthu mají typický podlouhlý tvar s palmetovým ukončením, známým v literatuře jako {\em anthemion}.} Ze dvou třetin převládá reliéfní vyobrazení, které znázorňuje motivy typické pro náhrobní stély z řeckých kolonií té doby. Ve většině případů se jedná o vyobrazení nebožtíka s doplňkovými atributy jako je pes, psací tabulka, maková hlavice apod.\footnote{Z kontextu řeckých komunit v oblasti Thrákie se z 6. st. př. n. l. se dochovaly i další náhrobní stély podobného charakteru, které však nenesou žádný nápis, jako například stéla zobrazující stojícího muže se psem. Toto vyobrazení, které je obecně považováno za zpodobnění řeckého aristokrata, se do dnešní doby dochovalo celkem v 15 exemplářích po celém Černomoří (Petrova 2015, 11-18) a bývá datováno na přelom 6. a 5. st. př. n. l.} Není zcela jasné, zda se jedná o černomořskou produkci, či o stély importované z jiných částí řeckého světa, avšak jejich výskyt právě v Černomoří nasvědčuje na jejich původ v rámci místních řeckých komunit iónského původu.

