
\section[charakteristika-epigrafické-produkce-ve-2.-st.-n.-l.-až-3.-st.-n.-l.]{Charakteristika epigrafické produkce ve 2. st. n. l. až 3. st. n. l.}

Nápisy datované do 2. a 3. st. př. n. l. pocházejí ze dvou třetin z vnitrozemí s největší produkcí v okolí Augusty Traiany, středního toku řeky Strýmón a Filippopole. Poprvé dedikační nápisy převažují nad funerálními nápisy, což pravděpodobně souvisí i se zvyšujícím se počtem místních kultů a variabilitou vyobrazení božstev na nápisech. Složení společnosti je i nadále multikulturní, s dominantním římským prvkem, nicméně celkově dochází k většímu zapojení thrácké populace do epigrafické produkce.

\placetable[none]{}
\starttable[|l|]
\HL
\NC {\em Celkem:} 182 nápisů

{\em Region měst na pobřeží:} Anchialos 2, Byzantion 7, Dionýsopolis 2, Madytos 1, Maróneia 8, Mesámbria 3, Odéssos 9, Perinthos (Hérakleia) 18, Sélymbria 3, Zóné 1 (celkem 54 nápisů)

{\em Region měst ve vnitrozemí:} Augusta Traiana 49, Filippopolis 16, Hadriánopolis 2, Marcianopolis 5{\bf ,} Nicopolis ad Nestum 2, Nicopolis ad Istrum 11, Pautália 3, Plótinúpolis 4, Serdica 8, Traianúpolis 3, údolí středního toku řeky Strýmón 21 (celkem 124 nápisů)\footnote{Celkem dva nápisy nebyly nalezeny v rámci regionu známých měst, editoři korpusů udávají jejich polohu vzhledem k nejbližšímu modernímu sídlišti, či uvádí muzeum, v němž se nachází. Další dva nápisy pocházejí z neznámého místa v Bulharsku.}

{\em Celkový počet individuálních lokalit}: 61

{\em Archeologický kontext nálezu:} funerální 1, sídelní 14 (z toho obchodní 7), náboženský 36, sekundární 22, neznámý 109

{\em Materiál:} kámen 181 (mramor 138; vápenec 28, jiný 3, z toho syenit 2, varovik 1; neznámý 12), neznámý 1

{\em Dochování nosiče}: 100 \letterpercent{} 21, 75 \letterpercent{} 27, 50 \letterpercent{} 27, 25 \letterpercent{} 44, kresba 13, ztracený 2, nemožno určit 48

{\em Objekt:} stéla 136, architektonický prvek 36, socha 2, mozaika 1, jiný 5, neznámý 2

{\em Dekorace:} reliéf 120, malovaná 2, bez dekorace 58; reliéfní dekorace figurální 73 nápisů (vyskytující se motiv: jezdec 33, sedící osoba 1, stojící osoba 5, skupina lidí 1, zvíře 1, Artemis 3, Asklépios 2, Héraklés 2, Zeus a Héra 1, scéna lovu 2, funerální scéna/symposion 7, funerální portrét 8, jiný 8), architektonické prvky 48 nápisů (vyskytující se motiv: naiskos 8, sloup 5, báze sloupu či oltář 21, architektonický tvar/forma 13, florální motiv 6, věnec 4, jiný 7)

{\em Typologie nápisu:} soukromé 126, veřejné 46, neurčitelné 10

{\em Soukromé nápisy:} funerální 54, dedikační 76, jiný 1\footnote{Několik nápisů mělo vzhledem ke své nejednoznačnosti kombinovanou funkci, proto je součet nápisů obou typů vyšší než celkový počet soukromých nápisů.}

{\em Veřejné nápisy:} seznamy 3, honorifikační dekrety 30, státní dekrety 3, náboženský 4, jiný 3, neznámý 3

{\em Délka:} aritm. průměr 5,0 řádku, medián 4, max. délka 48, min. délka 1

{\em Obsah:} dórský dialekt 0, latinský text 4 nápisy, písmo římského typu 74; hledané termíny (administrativní termíny 31 - celkem 139 výskytů, epigrafické formule 24 - 128 výskytů, honorifikační 3 - 3 výskyty, náboženské 42 - 112 výskytů, epiteton 26 - počet výskytů 40)

{\em Identita:} řecká božstva 19, egyptská božstva 4, římská božstva 2, pojmenování míst a funkcí typických pro řecké náboženské prostředí, nárůst počtu lokálních kultů, regionální epiteton 12, subregionální epiteton 14, kolektivní identita 13 termínů, celkem 24 výskytů - obyvatelé řeckých obcí z oblasti Thrákie 7, mimo ni 2; kolektivní pojmenování etnik či kmenů (Thráx 2, Rómaios 3, Bíthýnos 1, Acháios 1); celkem 256 osob na nápisech, 85 nápisů s jednou osobou; max. 24 osob na nápis, aritm. průměr 1,41 osoby na nápis, medián 1; komunita multikulturního charakteru se zastoupením řeckého, římského a thráckého prvku, se silnou přítomností římského prvku, jména pouze řecká (11,53 \letterpercent{}), pouze thrácká (6,59 \letterpercent{}), pouze římská (24,72 \letterpercent{}), kombinace řeckého a thráckého (4,39 \letterpercent{}), kombinace řeckého a římského (12,08 \letterpercent{}), kombinace thráckého a římského (2,74 \letterpercent{}), kombinovaná řecká, thrácká a římská jména (4,94 \letterpercent{}), jména nejistého původu (6,02 \letterpercent{}), beze jména (26,92 \letterpercent{}); geografická jména z oblasti Thrákie 12, geografická jména mimo Thrákii 4;

\NC\AR
\HL
\HL
\stoptable

V případě skupiny 182 nápisů datovaných do 2. a 3. st. n. l. se jedná o nárůst o 93 \letterpercent{} oproti skupině nápisů z 1. až. 2. st. n. l. Nápisy pocházejí z většiny z thráckého vnitrozemí z okolí hlavních městských center a okolí hlavních komunikačních tepen, jak je patrné z mapy 6.08 v Apendixu 2. Největším producentem se stává Augusta Traiana, jakožto hlavní město provincie {\em Thracia}. Producenty střední velikosti jsou další velká centra městského typu, jako Perinthos, Nicopolis ad Istrum a oblast středního toku Strýmónu

Převážná část nosičů nápisů je z kamene, nicméně dva nápisy jsou zhotovené z kovu, a jeden na mozaice.\footnote{Nápisy na kovu představují vojenský diplom ({\em AE} 2007, 1259) a {\em instrumentum domesticum} čili předmět běžné denní potřeby ({\em AE} 2004, 1302, {\em instrumentum domesticum}, Chaniotis 2005, 92). Nápis {\em SEG} 54:652 se nachází na mozaice. Užitá osobní jména poukazují jak na řecký původ, tak na římské onomastické tradice u osob, které na nápisech figurují, ať už jako majitelé, či zhotovitelé. Nápisy pocházejí z okolí vojenského tábora v Kabylé a z vojenského tábora ve městě Nicopolis ad Istrum na Dunaji. Nápisy pocházejí nepřímo z kontextu spojeného s přítomností a projevy armádních struktur na území Thrákie. Vzhledem k jejich malému počtu a krátkému rozsahu však není možné vyvozovat nic dalšího.}

