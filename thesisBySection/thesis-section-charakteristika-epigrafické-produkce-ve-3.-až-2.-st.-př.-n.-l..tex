
\section[charakteristika-epigrafické-produkce-ve-3.-až-2.-st.-př.-n.-l.]{Charakteristika epigrafické produkce ve 3. až 2. st. př. n. l.}

Nápisy datované do 3. až 2. st. př. n. l. pocházejí převážně z řeckého kulturního prostředí. Dochází k nárůstu produkce veřejných nápisů, stejně tak k výskytu hledaných termínů. Prolínání řeckých a thráckých onomastických tradic je možné sledovat v omezené míře, dále i rozšíření lokálních, řeckých a egyptských náboženských tradic.

\placetable[none]{}
\starttable[|l|]
\HL
\NC {\em Celkem:} 59 nápisů

{\em Region měst na pobřeží:} Abdéra 5, Apollónia Pontská 4, Byzantion 12, Dionýsopolis 1, Lýsimacheia 1, Maróneia 10, Mesámbria 9, Odéssos 7, Perinthos (Hérakleia) 3 (celkem 52 nápisů)

{\em Region měst ve vnitrozemí:} Beroé (Augusta Traiana) 1, Filippopolis 1, údolí středního toku řeky Strýmónu 3\footnote{Celkem dva nápisy nebylo nalezeny v rámci regionu známých měst, editoři korpusů udávají jejich polohu vzhledem k nejbližšímu modernímu sídlišti (jedna lokalita s jedním nápisem), či uvádějí jejich původ jako blíže neznámé místo v Thrákii (jeden nápis).}

{\em Celkový počet individuálních lokalit}: 14

{\em Archeologický kontext nálezu:} funerální 4, sídelní 1, náboženský 1, sekundární 4, neznámý 48

{\em Materiál:} kámen 56 (mramor 48, jiný 2, neznámý 6), keramika 1, kov 1, neznámý 1

{\em Dochování nosiče}: 100 \letterpercent{} 3, 75 \letterpercent{} 10, 50 \letterpercent{} 8, 25 \letterpercent{} 14, kresba 1, ztracený 2, nemožno určit 21

{\em Objekt:} stéla 50, socha 1, nádoba 1, architektonický prvek 4, jiné 2, neznámý 1

{\em Dekorace:} reliéf 29, malovaná dekorace 1, bez dekorace 29; figurální 8 nápisů (vyskytující se motiv: skupina lidí 1, sedící postava 2, stojící postava 1, funerální scéna 2, jezdec 1), architektonické prvky 21 nápisů (vyskytující se motiv: naiskos 9, florální motiv 7, sloup 1, báze sloupu či oltář 3, věnec 1, architektonický tvar/forma 3)

{\em Typologie nápisu:} soukromé 37, veřejné 20, neurčitelné 2

{\em Soukromé nápisy:} funerální 30, dedikační 6, vlastnictví 1, jiné (jméno autora) 1\footnote{Vzhledem ke kumulaci typů je celkový součet vyšší než počet soukromých nápisů.}

{\em Veřejné nápisy:} náboženské 1, seznamy 1, honorifikační dekrety 5, státní dekrety 11, funerální na náklady obce 1, neznámý 1

{\em Délka:} aritm. průměr 5,08 řádku, medián 3, max. délka 25, min. délka 1

{\em Obsah:} dórský dialekt 7, graffiti 1; hledané termíny (administrativní termíny 11 - celkem 39 výskytů, epigrafické formule 7--20 výskytů, honorifikační 11--22 výskytů, náboženské 14--20 výskytů, epiteton 1 - počet výskytů 1)

{\em Identita:} řecká božstva, místní božstva, egyptská božstva, kolektivní identita 3 - obyvatelé řeckých obcí, z toho z oblasti Thrákie 1, mimo Thrákii 2, celkem 71 osob na nápisech, 34 nápisů s jednou osobou; max. 6 osob na nápis, aritm. průměr 1,2 osoby na nápis, medián 1; komunita převládajícího řeckého charakteru, jména pouze řecká (52 \letterpercent{} - celkem 31 nápisů), thrácká (3,38 \letterpercent{} - celkem 2 nápisy), kombinace řeckého a thráckého (5,08 \letterpercent{} - 3 nápisy), pouze římská jména (1,69 \letterpercent{} - 1 nápis), jména nejistého původu (16,94 \letterpercent{}); geografická jména míst v Thrákii 1;

\NC\AR
\HL
\HL
\stoptable

Celkem se dochovalo 59 nápisů, které i nadále pocházejí převážně z řeckých komunit na pobřeží a jen minimum nápisů je nalezeno v thráckém vnitrozemí, jak dokazuje mapa 6.04 v Apendixu 2. Nápisy z vnitrozemí pocházejí z okolí Kazanlackého údolí a středního toku Strýmónu, tedy oblastí spojovaných se zvýšenou přítomností makedonských či thráckých vojáků v řeckých službách (Nankov 2012; 2015).

Převládajícím materiálem je již tradičně kámen, nápisy na kovu a na keramice se dochovaly vždy po jednom exempláři. V obou případech nápisů na jiném nosiči, než na kameni se jedná o identifikaci autora předmětu či vnitřní dekorace hrobky v kombinaci se jménem majitele předmětu, tedy podobné použití jako v 5. až 3. st. př. n. l. Přes polovinu nápisů představují funerální nápisy, ale setkáváme se i s označením vlastnictví z prostředí thrácké aristokracie.\footnote{Do této kategorie spadá nápis {\em IG Bulg} 5 5638bis nalezený na amfoře v antickém městě Kabylé na středním toku řeky Tonzos. Kabylé bylo známo jako makedonské vojensko-obchodní osídlení založené na místě dřívějšího thráckého sídliště (Handzhijska a Lozanov 2010, 260-263). Nápis představuje pravděpodobně jméno majitele či objednatele amfory, Sadalás, syn Téreův. Obě jména jsou typicky thrácká, většinou patřící thrácké aristokracii (Dana 2014, 298-301; 355-361).}

