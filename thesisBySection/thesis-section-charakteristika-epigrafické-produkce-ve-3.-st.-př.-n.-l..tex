
\section[charakteristika-epigrafické-produkce-ve-3.-st.-př.-n.-l.]{Charakteristika epigrafické produkce ve 3. st. př. n. l.}

Nápisy v této době pocházejí převážně z pobřežních oblastí, nicméně nápisy se objevují i v kontextu thrácké aristokracie ve vnitrozemí stejně jako v 5. a 4. st. př. n. l. V řeckých komunitách na pobřeží dochází ke kontaktu s ostatními řecky mluvícími komunitami mimo Thrákii a v malé míře i s thráckým kulturním prostředím. Nárůst počtu veřejných nápisů a výskyt hledaných termínů nasvědčují nárůstu společenské komplexity, a to zejména v řeckém prostředí.

\placetable[none]{}
\starttable[|l|]
\HL
\NC {\em Celkem}: 117 nápisů

{\em Region měst na pobřeží}: Abdéra 5, Agathopolis 1, Anchialos 2, Apollónia Pontská 5, Byzantion 16, Dionýsopolis 6, Doriskos 1, Maróneia 21, Mesámbria 42, Naulochos 1, Odéssos 5, Perinthos (Hérakleia) 1 (celkem 106 nápisů)

{\em Region měst ve vnitrozemí}: Beroé (Augusta Traiana) 4, (Marcianopolis) 1, (Plótinúpolis) 1\footnote{Celkem pět nápisů nebylo nalezeno v rámci regionu známých měst, editoři korpusů udávají jejich polohu vzhledem k nejbližšímu modernímu sídlišti (jedna lokalita s jedním nápisem), či uvádějí jejich původ jako blíže neznámé místo v Thrákii (dva nápisy), či jako nápisy pocházející z území mimo Thrákii (dva nápisy).}

Celkový počet individuálních lokalit: 18

{\em Archeologický kontext nálezu}: funerální 10, sídelní 3, náboženský 2, sekundární 10, neznámý 92

{\em Materiál}: kámen 116 (mramor 103, z toho mramor z Thasu 1, vápenec 3, jiné 1; z čehož je syenit 1; neznámý 9), jiný materiál 1

{\em Dochování nosiče}: 100 \letterpercent{} 19, 75 \letterpercent{} 6, 50 \letterpercent{} 35, 25 \letterpercent{} 28, oklepek 1, kresba 2, nemožno určit 26

{\em Objekt}: stéla 110, architektonický prvek 5, socha 1, nástěnná malba 1

{\em Dekorace}: reliéf 61, malovaná dekorace 2, bez dekorace 54; reliéfní dekorace figurální 13 nápisů (vyskytující se motiv: jezdec 1, stojící osoba 3, sedící osoba 8, skupina lidí 4), architektonické prvky 44 nápisů (vyskytující se motiv: naiskos 22, sloup 5, báze sloupu či oltář 1, florální motiv 10, geometrický motiv 1, architektonický tvar/forma 5)

{\em Typologie nápisu}: soukromé 79, veřejné 36, neurčitelné 2

{\em Soukromé nápisy}: funerální 71, dedikační 5, jiný (jméno autora) 1, neznámý 2

{\em Veřejné nápisy}: nařízení 1, náboženské 2, seznamy 2, honor. dekrety 9, státní dekrety 23\footnote{V určitých případech může docházet ke kumulaci jednotlivých typů textů v rámci jednoho nápisu, či jejich nejednoznačnost neumožňuje rozlišit mezi několika typy a těmto nápisů jsou přiřazeny několikanásobné hodnoty (např. nápis náboženský a zároveň seznam). V těchto případech pak součet všech typů nápisů může přesahovat celkové číslo nápisů.}

{\em Délka}: aritm. průměr 5,56 řádku, medián 3, max. délka 37, min. délka 1

{\em Obsah}: dórský dialekt 26, graffiti 1; hledané termíny (administrativní termíny 25 - celkem 77 výskytů, epigrafické formule 11 - 46 výskytů, honorifikační 19 - 65 výskytů, náboženské 18 - 42 výskytů, epiteton 1 - počet výskytů 1)

{\em Identita}: řecká božstva, pojmenování míst a funkcí typických pro řecké náboženské prostředí, regionální epiteton 1, kolektivní identita 15 termínů, celkem 18 výskytů - obyvatelé řeckých obcí z oblasti Thrákie, ale i mimo ni, kolektivní pojmenování kmenové příslušnosti (Thessalos, Aitólos, Krés), celkem 166 osob na nápisech, 70 nápisů s jednou osobou; max. 12 osob na nápis, aritm. průměr 1,41 osoby na nápis, medián 2; komunita převládajícího řeckého charakteru, jména pouze řecká (75 \letterpercent{}), thrácká (3,41 \letterpercent{}), kombinace řeckého a thráckého (2,56 \letterpercent{}), jména nejistého původu (14,52 \letterpercent{}); geografická jména z oblasti Thrákie 5, geografická jména mimo Thrákii 3;

\NC\AR
\HL
\HL
\stoptable

Celkem se dochovalo 117 nápisů, zejména z řeckých měst či jejich bezprostředního okolí. Z vnitrozemí pocházejí nápisy z lokalit na největších řekách spojujících vnitrozemskou Thrákii s Egejskou oblastí, konkrétně z povodí řek Tonzos, Hebros a Strýmón. Mapa 6.04 v Apendixu 2 zachycuje konkrétní rozložení lokalit v nichž byly nalezeny nápisy. Apollónia Pontská přestává být hlavním producentem nápisů a na její místo nastupuje její jižní soused Mesámbria, odkud pochází 36 \letterpercent{} nápisů z daného období. Dalšími významnými producenty nápisů jsou Maróneia, Byzantion a Odéssos. Archeologický kontext nálezu nápisů většinou neznámý, nicméně zhruba u 14 nápisů byl tento kontext určen jako funerální, tj. pochází z pohřebiště či mohyly. Archeologické lokality dosvědčují náhlé změny poměrů, pravděpodobně související s příchodem keltských kmenů do Thrákie, které měly za následek destrukci některých lokalit, případně jejich úplný zánik, jako v případě {\em emporia} Pistiros (Bouzek {\em et al.} 2016).

Převládajícím materiálem je z 99 \letterpercent{} i nadále kámen: z téměř 91 \letterpercent{} mramor a dále kámen místního původu, jako je vápenec, syenit či varovik. Nápisy tesané do kamene pocházely z velké části z řeckých měst či jejich bezprostředního okolí. Výjimku tvoří několik nápisů pocházejících z thráckého vnitrozemí, a to údolí středního Strýmónu, dále oblast střední Thrákie s lokalitami Seuthopolis a Kabylé. V těchto místech se předpokládá buď přítomnost řeckého či makedonského obyvatelstva nebo při nejmenším velmi intenzivní kontakty thrácky a řecky mluvících komunit. Dochází zde k náznakům podobného využití písma pro regulaci společenských vztahů ve formě, na jakou jsme zvyklí z řeckých obcí na pobřeží.\footnote{V jednom případě se dochoval nápis na jiném materiále než na kameni, a to na fresce uvnitř hrobky z thráckého prostředí.} Podobně jako v předcházejících dvou stoletích se setkáváme s odlišným pojetím písma a jeho kulturně-společenské funkce v rámci řecké a thrácké komunity v malé míře i ve 3. st. př. n. l.

