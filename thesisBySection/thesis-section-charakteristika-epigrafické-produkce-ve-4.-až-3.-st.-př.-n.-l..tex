
\section[charakteristika-epigrafické-produkce-ve-4.-až-3.-st.-př.-n.-l.]{Charakteristika epigrafické produkce ve 4. až 3. st. př. n. l.}

Nápisy datované od 4. do 3. st. př. n. l. pocházejí převážně z řeckého prostředí na pobřeží. I nadále má největší počet nápisů funerální funkci, avšak začínají se objevovat i dedikace věnované božstvům řeckého původu. Nečetné nápisy z vnitrozemí i nadále slouží potřebám thrácké aristokracie a jejich společenská funkce se odlišuje od využití nápisů v řeckých městech. Komunity si i nadále udržují konzervativní charakter a k jejich prolínání dochází ve velmi omezené míře.

\placetable[none]{}
\starttable[|l|]
\HL
\NC {\em Celkem:} 25 nápisů

{\em Region měst na pobřeží:} Abdéra 2, Apollónia Pontská 4, Byzantion 3, Maróneia 3, Mesámbria 3, Perinthos (Hérakleia) 1, Zóné 1 (celkem 17 nápisů)

{\em Region měst ve vnitrozemí:} Beroé (Augusta Traiana) 1, Filippopolis 3, (Hadriánúpolis) 1\footnote{Celkem tři nápisy nebyly nalezeny v rámci regionu známých měst, editoři korpusů udávají jejich polohu vzhledem k nejbližšímu modernímu sídlišti (tři lokality ve vnitrozemí).}

{\em Celkový počet individuálních lokalit}: 14

{\em Archeologický kontext nálezu:} funerální 5, sídelní 3, sekundární 3, neznámý 14

{\em Materiál:} kámen 23 (mramor 16, vápenec 3, jiný 2, neznámý 4), jiný 1, neznámý 1

{\em Dochování nosiče}: 100 \letterpercent{} 3, 75 \letterpercent{} 1, 50 \letterpercent{} 6, 25 \letterpercent{} 2, kresba 4, nemožno určit 9

{\em Objekt:} stéla 23, nástěnná malba 1

{\em Dekorace:} reliéf 8, bez dekorace 17; figurální dekorace 0, architektonické prvky 10 nápisů (vyskytující se motiv: naiskos 7, florální motiv 2, neznámý 1)

{\em Typologie nápisu:} soukromé 20, veřejné 3, neurčitelné 2

{\em Soukromé nápisy:} funerální 14, dedikační 4, vlastnictví 1, jiný 1

{\em Veřejné nápisy:} honorifikační dekrety 1, státní dekrety 1, neznámý 1

{\em Délka:} aritm. průměr 4,24 řádku, medián 2, max. délka 20, min. délka 1

{\em Obsah:} dórský dialekt 1; stoichédon 1; hledané termíny (administrativní 4 - celkem 4 výskyty, epigrafické formule 6 - celkem 6 výskytů, honorifikační 7 - celkem 7 výskytů, náboženské 3 - celkem 4 výskyty, epiteton 0)

{\em Identita:} řecká božstva 2, kolektivní identita 2 - obyvatelé řeckých obcí z oblasti Thrákie, celkem 23 osob na nápisech, 17 nápisů s jednou osobou; max. 2 osob na nápis, aritm. průměr 1,15 osoby na nápis, medián 1; komunita převládajícího řeckého charakteru, jména pouze řecká (60 \letterpercent{} - celkem 15 nápisů), thrácká (8 \letterpercent{} - celkem 2 nápisy), kombinace řeckého a thráckého (0 \letterpercent{}), jména nejistého původu (12 \letterpercent{}); geografická jména 0;

\NC\AR
\HL
\HL
\stoptable

Nápisů datovaných do 4. až 3. st. př. n. l. se dochovalo 25, což představuje úbytek o 80 \letterpercent{} oproti podobné skupině nápisů datovaných do 5. až 4. st. př. n. l. Skupina nápisů ze 4. až 3. st. př. n. l. vykazuje mírný nárůst lokalit v thráckém vnitrozemí v okolí řeckých a makedonských sídel v Pistiru a Seuthopoli a podél hlavních toků, nicméně i přesto se většina epigrafické produkce nachází na pobřeží Černého, Marmarského a Egejského moře, jak dokazuje následující mapa 6.03 v Apendixu 2.

Nápisy datované do 4. až 3. st. př. n. l. pokračují v tradicích ustanovených v předcházejících obdobích a nedochází k zásadní kulturní změně zaznamenané na epigrafické produkci.\footnote{V 58 \letterpercent{} se jedná o funerální nápisy z okolí řeckých měst, ve 32 \letterpercent{} o nápisy dedikační a ve 24 \letterpercent{} o nápisy veřejné. Nosiče nápisů jsou převážně zhotoveny z kamene, sloužící soukromým účelům a veřejnému vystavení, nicméně dva nápisy pocházejí z interiéru hrobky thráckého aristokrata a typologicky odpovídají podobným utilitárním nápisům na cenných předmětech z 5. a 4. st. př. n. l. Nápis {\em SEG} 39:653 nespadá ani do kategorie funerálních či dedikačních nápisů, nicméně je důležitý vzhledem k tomu, že pochází z vnitrozemí, z regionu pozdější Hadriánopole. Nápis byl publikován pouze částečně, a je znám v podstatě pouze jeho text, který zní „{\em Hebryzelmis, syn Seutha, Prianeus}”. Hebryzelmis i Seuthés jsou thrácká aristokratická jména, nicméně badatelé si nejsou jisti původem etnického jména Prianeus, ale obecně ho považují za jméno thráckého kmene z egejské oblasti (Veligianni 1995, 158). Bohužel více informací o nápisu není dostupných, což znemožňuje jakékoliv další závěry.}

