
\section[charakteristika-epigrafické-produkce-ve-4.-st.-př.-n.-l.]{Charakteristika epigrafické produkce ve 4. st. př. n. l.}

Většina epigrafické produkce ve 4. st. př. n. l. pochází z řeckých komunit podél mořského pobřeží. I nadále převládají nápisy funerální, ale pozvolna narůstá i počet veřejných nápisů, sloužících převážně k organizaci a uplatnění politické svrchovanosti řeckých komunit, ale i jako prostředek vyjádření diplomatických vztahů mezi řeckými obcemi a thráckými aristokraty. Vyskytující se jména na nápisech nasvědčují na omezené mísení řecké a thrácké onomastické tradice v bezprostředním okolí řeckých měst. Nápisy se objevují ve velmi omezené míře i v thráckém vnitrozemí, kde slouží především potřebám thrácké aristokracie, stejně jako v 5. st. př. n. l.

\placetable[none]{}
\starttable[|l|]
\HL
\NC {\em Celkem:} 168 nápisů

{\em Region měst na pobřeží:} Abdéra 16, Apollónia Pontská 55, Byzantion 7, Chersonésos Molyvótés 1, Dionýsopolis 1, Maróneia 9, Mesámbria 11, Odéssos 5, Perinthos (Hérakleia) 2, Strýmé 29, Zóné 18 (celkem 154 nápisů)

{\em Region měst ve vnitrozemí:} Beroé (Augusta Traiana) 2, Pistiros 1\footnote{Celkem 11 nápisů nebylo nalezeno v rámci regionu známých měst, editoři korpusů udávají jejich polohu vzhledem k nejbližšímu modernímu sídlišti (čtyři lokality s celkem pěti nápisy), či uvádějí jejich původ jako blíže neznámé místo v Thrákii (šest nápisů).}

{\em Celkový počet individuálních lokalit}: 26

{\em Archeologický kontext nálezu:} funerální 58, sídelní 3, nábož. 2, sekundární 21, neznámý 84

{\em Materiál:} kámen 162 (mramor 92, z toho mramor z Thasu 3, vápenec 44, jiné 20, z čehož je pískovec 4, póros 2, vulkanický kámen 3; neznámý 6), kov 4 (stříbro 2, zlato 1, neznámý kov 1), jiný materiál 1, neznámý 1

{\em Dochování nosiče}: 100 \letterpercent{} 54, 75 \letterpercent{} 33, 50 \letterpercent{} 34, 25 \letterpercent{} 16, kresba 6, ztracený 1, nemožno určit 24

{\em Objekt:} stéla 142, architektonický prvek 18, nádoba 2, socha 1, nástěnná malba 1, jiný 2, neznámý 2

{\em Dekorace:} reliéf 67, malovaná dekorace 6, jiná dekorace 1, bez dekorace 94; reliéfní dekorace figurální 6 nápisů (vyskytující se motiv: jezdec 1, stojící osoba 2, sedící osoba 2, funerální scéna 1), architektonické prvky 61 nápisů (vyskytující se motiv: naiskos 24, sloup 4, báze sloupu či oltář 14, florální motiv 5, architektonický tvar/forma 19, jiný 1)

{\em Typologie nápisu:} soukromé 153, veřejné 9, neurčitelné 6

{\em Soukromé nápisy:} funerální 142, dedikační 5, vlastnictví 4, jiný 1, neznámý 1

{\em Veřejné nápisy:} nařízení 2, náboženské 1, seznamy 1, honorifikační dekrety 1, státní dekrety 2, jiný 1, neznámý 1\footnote{Součet nápisů jednotlivých typů je vyšší než počet veřejných nápisů vzhledem k možným kombinacím jednotlivých typů v rámci jednoho nápisu.}

{\em Délka:} průměr 2,96 řádku, medián 2, max. 46, min. 1

{\em Obsah:} dórský dialekt 12, iónsko-attický dialekt 3; stará attická alfabéta 2, stoichédon 3, graffiti 1; hledané termíny (administrativní 11 - celkem 16 výskytů, epigrafické formule 3 - celkem 3 výskyty, honorifikační 3 - celkem 4 výskyty, náboženské 8 - celkem 10 výskytů, epiteton 2)

{\em Identita:} řecká božstva 6, subregionální hérós 2, kolektivní identita 10 - obyvatelé řeckých obcí, Thrákové jako kolektivní pojmenování, celkem 174 osob na nápisech, 121 nápisů s jednou osobou; max. 7 osob na nápis, aritm. průměr 1,03 osoby na nápis, medián 1; komunita převládajícího řeckého charakteru, jména pouze řecká (72 \letterpercent{}), thrácká (2,97 \letterpercent{}), kombinace řeckého a thráckého (0,59 \letterpercent{}), jména nejistého původu (14,28 \letterpercent{}); geografická jména z oblasti Thrákie 4, geografická jména mimo Thrákii 1;

\NC\AR
\HL
\HL
\stoptable

Do 4. st. př. n. l. bylo datováno 168 nápisů, což znamená nárůst o 180 \letterpercent{} oproti skupině nápisů datovaných do 5. st. př. n. l. Ve 4. st. př. n. l. i nadále narůstá počet individuálních lokalit, v nichž byly nápisy nalezeny. Jak dokazuje mapa 6.03 v Apendixu 2, většina nápisů stále pochází z řeckých měst na pobřeží Černého, Marmarského a Egejského moře. Téměř jedna třetina nápisů pochází z řeckého města Apollónia Pontská, která v této době patří k hlavním kulturním a ekonomickým centrům regionu, ale zároveň také k nejlépe archeologicky prozkoumaným městům.\footnote{V posledních několika desetiletích zde byly nalezené nové nekropole s velkým počtem funerálních nápisů (Isaac 1986, 246; Avram, Hind a Tsetkhladze 2004, 931-932; Velkov 2005; Gyuzelev 2002, 2005, 2013). Oproti nápisům z 5. až 4. st. př. n. l. jsou nicméně celková čísla dochovaných nápisů z Apollónie nicméně zhruba o 40 \letterpercent{} nižší, což může být jednak důsledek lehkého útlumu postavení Apollónie, či výsledek do značné míry náhodných archeologických nálezů a široké datace nápisů.} Mezi producenty střední velikosti patří řecká města z pobřežních oblastí jako je Abdéra, Byzantion, Maróneia, Mesámbria, Odéssos, Strýmé a Zóné. Oproti 5. st. př. n. l. je možné zaznamenat objevení nových lokalit ve vnitrozemské Thrákii, a to zejména v okolí dvou hlavních toků, Hebros a Tonzos. Lokality sídelního a funerálního charakteru ve vnitrozemí jsou, dle obecně přijímaného konsenzu, jak řeckého, tak thráckého původu.\footnote{Lokalita Pistiros je archeology považována za řeckou obchodní stanici a říční přístav (Bouzek {\em et al.} 1996, 9). Lokalita Seuthopolis je interpretována jako sídlo thráckého panovníka Seutha III. (Dimitrov, Chichikova a Alexieva 1978, 3-5). Lokality Sborjanovo, Smjadovo, Alexandrovo, Kupino, Naip jsou považovány za hrobky bohatých thráckých aristokratů z kmene Odrysů (Stoyanov 2001, 207-218; Atanasov 2006, 6; Kitov 2001, 15-29; Gergova 1995, 385-392; Delemen 2006, 261-263). Archeologický kontext míst nálezu nápisů je bohužel z poloviny neznámý, dále z jedné třetiny funerální, což znamená, že nápisy byly nalezeny buď v konkrétní hrobce, její blízkosti, či v blízkosti pohřebiště. Ve třech případech pochází nápisy ze sídelního archeologického kontextu, a ve dvou případech bylo místo jejich nálezu určeno jako náboženského původu, např. u nápisů pocházejících ze svatyně. Sekundární archeologický kontext je znám celkem u 21 nápisů, což je zhruba osmina souboru.}

Podobně jako v 5. st. př. n. l. je možné sledovat odlišné charakteristiky nápisů zhotovených na kameni a na kovu či keramice. Jednak se jedná o jejich odlišnou kulturně-společenskou funkci, ale také i jejich výskyt v rámci prostorově odlišných komunit. nápisy na kameni (96 \letterpercent{}) se vyskytovaly převážně na pobřeží, nápisy na jiném médiu pocházejí převážně z vnitrozemí (3 \letterpercent{}).\footnote{U jednoho nápisu editor korpusu neudává materiál nosiče a je tedy neznámý.} Nápisy na kameni jsou datovány do 4. st. př. n. l. v počtu 162 nápisů, z nichž 59 \letterpercent{} nápisů je zhotoveno z mramoru, 28 \letterpercent{} z vápence a zbývající část z místně dostupného kamene.\footnote{V případě pobřeží Egejského moře zdroj mramoru pochází např. z lomů na Thasu.} Podobně jako v 5. st. př. n. l. dochází k využívání místních zdrojů a nemáme důkazy o tom, že by se nápisy staly předmětem dálkového obchodu. \footnote{Stejně jako v 5. st. př. n. l. má převážná část nápisů na kameni charakter soukromého nápisu, s podílem 91 \letterpercent{} nápisů z daného období. Dále jsou to nápisy veřejné se zastoupením 5 \letterpercent{} a nápisy, které nebylo možné určit, které představují zhruba 4 \letterpercent{}.} Oproti tomu nápisy z kovu a nápisy vyryté do stěny hrobky pocházejí z thráckého vnitrozemí či z oblastí na pobřeží Egejského moře tradičně ovládanými thráckým kmenem Odrysů v okolí hory Ganos, mezi řeckými městy Ainos a Perinthos (Archibald 1998, 109-111).\footnote{Stejně jako v případě nápisů na jiném materiálu než kameni, datovaných do 5. st př. n. l., i tato skupina čtyř nápisů pochází z kontextu bohatých pohřbů, patřícím pravděpodobně thrácké aristokracii, jako je Dalakova Mogila u Kazanlaku, lokalita Naip u hory Ganos na pobřeží Marmarského moře, neznámá mohyla u Kazanlaku (Delemen 2006, 251-273; Kitov 1995, 17; Kitov a Dimitrov 2008, 25-32). Jedná se jak o funerální nápisy primární, tak sekundárně umístěné.}

