
\section[charakteristika-epigrafické-produkce-ve-3.-st.-n.-l.]{Charakteristika epigrafické produkce ve 3. st. n. l.}

Nápisy datované do 3. st. n. l. pocházejí ze tří čtvrtin z vnitrozemských městských center, jako je Augusta Traiana, Filippopolis a Serdica. Zvýšený výskyt geografických termínů a kolektivních pojmenování z oblastí mimo Thrákii naznačuje otevírání společnosti a migraci lidí z dalších částí římské říše. I nadále převládají dedikační nápisy, často věnované božstvům lokálního charakteru. Veřejné nápisy představují téměř polovinu všech nápisů, poprvé se v nich objevuje i místní samospráva na úrovni vesnic, nikoliv pouze na úrovni měst.

\placetable[none]{}
\starttable[|l|]
\HL
\NC {\em Celkem:} 390 nápisů

{\em Region měst na pobřeží:} Abdéra 5, Anchialos 3, Bizóné 1, Byzantion 20, Caron Limen 2, Dionýsopolis 7, Ferai 1, Maróneia 21, Maximiánúpolis 1, Mesámbria 1, Odéssos 7, Perinthos (Hérakleia) 16, Sélymbria 2, Strýmé 1, Topeiros 9 (celkem 97 nápisů)

{\em Region měst ve vnitrozemí:} Augusta Traiana 58, Discoduraterae 7, Filippopolis 49, Hadriánopolis 1, Hérakleia Sintská 2, Marcianopolis 4, Neiné 3, Nicopolis ad Istrum 25, Pautália 7, Plótinúpolis 4, Serdica 91, Traianúpolis 1, údolí středního toku řeky Strýmón 29 (celkem 281 nápisů)\footnote{Celkem 12 nápisů nebylo nalezeno v rámci regionu známých měst, editoři korpusů udávají jejich polohu vzhledem k nejbližšímu modernímu sídlišti. Devět nápisů pochází z pobřeží Egejského moře a tři z bulharského vnitrozemí.}

{\em Celkový počet individuálních lokalit}: 115

{\em Archeologický kontext nálezu:} funerální 9, sídelní 47 (z toho obchodní 20), náboženský 79, sekundární 44, jiný 3, neznámý 208

{\em Materiál:} kámen 385 (mramor 237, z Prokonnésu 1; vápenec 104, jiný 17, z toho syenit 6, varovik 1, granit 1, tuf 1, břidlice 1; neznámý 27), keramika 2, neznámý 3

{\em Dochování nosiče}: 100 \letterpercent{} 45, 75 \letterpercent{} 47, 50 \letterpercent{} 74, 25 \letterpercent{} 60, oklepek 6, kresba 25, ztracený 8, nemožno určit 125

{\em Objekt:} stéla 215, architektonický prvek 145, socha 18, jiný 1, neznámý 11

{\em Dekorace:} reliéf 259, malovaná 1, bez dekorace 130; reliéfní dekorace figurální 115 nápisů (vyskytující se motiv: jezdec 59, sedící osoba 2, stojící osoba 7, skupina lidí 3, zvíře 1, Artemis 1, Asklépios 3, Héraklés 1, Zeus a Héra 1, Dionýsos 2, scéna lovu 2, funerální scéna/symposion 3, funerální portrét 13, socha 1, jiný 8), architektonické prvky 136 nápisů (vyskytující se motiv: naiskos 12, sloup 49, báze sloupu či oltář 70, architektonický tvar/forma 17, geometrický motiv 2, florální motiv 15, věnec 2, jiný 13)

{\em Typologie nápisu:} soukromé 203, veřejné 176, neurčitelné 11

{\em Soukromé nápisy:} funerální 76, dedikační 128, vlastnictví 2, jiný 5\footnote{Několik nápisů mělo vzhledem ke své nejednoznačnosti kombinovanou funkci, proto je součet nápisů obou typů vyšší než celkový počet soukromých nápisů.}

{\em Veřejné nápisy:} seznamy 7, honorifikační dekrety 97, státní dekrety 11, náboženský 1, jiný 52, neznámý 9

{\em Délka:} aritm. průměr 8,62 řádku, medián 7, max. délka 270, min. délka 1

{\em Obsah:} dórský dialekt 1, iónsko-attický 1, latinský text 12 nápisů, písmo římského typu 168; hledané termíny (administrativní termíny 53 - celkem 502 výskytů, epigrafické formule 26 - 315 výskytů, honorifikační 9 - 22 výskytů, náboženské 30 - 177 výskytů, epiteton 24 - počet výskytů 55)

{\em Identita:} řecká božstva 13, egyptská božstva 0, římská božstva 1, pojmenování míst a funkcí typických pro řecké náboženské prostředí, nárůst počtu lokálních kultů, regionální epiteton 11, subregionální epiteton 13, kolektivní identita 38 termínů, celkem 162 výskytů - obyvatelé řeckých obcí z oblasti Thrákie 20, mimo ni 3; kolektivní pojmenování etnik či kmenů (barbaros 3, Thráx 64, Rómaios 11, Kampános 1, Makedón 1), obyvatelé thráckých vesnic 10; celkem 1129 osob na nápisech, 134 nápisů s jednou osobou; max. 154 osob na nápis, aritm. průměr 2,89 osoby na nápis, medián 1; komunita multikulturního charakteru se zastoupením řeckého, římského a thráckého prvku, se nadpoloviční přítomností římského prvku, jména pouze řecká (6,66 \letterpercent{}), pouze thrácká (3,58 \letterpercent{}), pouze římská (27,43 \letterpercent{}), kombinace řeckého a thráckého (2,56 \letterpercent{}), kombinace řeckého a římského (19,23 \letterpercent{}), kombinace thráckého a římského (5,12 \letterpercent{}), kombinovaná řecká, thrácká a římská jména (10,25 \letterpercent{}), jména nejistého původu (8,18 \letterpercent{}), beze jména (16,92 \letterpercent{}){\bf ;} geografická jména z oblasti Thrákie 24, geografická jména mimo Thrákii 19;

\NC\AR
\HL
\HL
\stoptable

Ve 3. st. n. l. je úroveň epigrafické produkce přibližně o polovinu vyšší než v předcházejícím století. I nadále pocházejí tři čtvrtiny nápisů z vnitrozemských oblastí, a to zejména z lokalit nalézajících se podél římské cesty {\em Via Diagonalis,} jako je Augusta Traiana a Filippopolis. Tato cesta procházející ze severovýchodu na jihovýchod Thrákie spojovala Evropu s Malou Asií a sloužila k poměrně častému přesunu vojsk oběma směry (Madzharov 2009, 70-131). Zbývající čtvrtina epigrafické produkce i nadále pochází z tří regionů řeckých měst na pobřeží: z okolí Perinthu a Byzantia na pobřeží Marmarského moře, z okolí Abdéry a Maróneie na pobřeží Egejského moře, a dále z Odéssu a Dionýsopole na pobřeží Černého moře. Konkrétní polohu míst nálezů nápisů ilustruje mapa 6.09 v Apendixu 2.\footnote{Archeologický kontext je podobně jako u předcházejícího období z velké části neznámý, avšak zhruba u 18 \letterpercent{} lokalit je archeologický kontext náboženský a zhruba u 11 \letterpercent{} lokalit sídelní. Oproti předcházejícímu období dochází k mírnému nárůstu u obou kategorií, a téměř každý třetí nápis pochází buď z osídlení, či svatyně. Předpokládá se, že materiál na výrobu nápisů pocházel z místních zdrojů, jako např. z ostrova Prokonnésos v Marmarském moři. Zcela v tomto období chybí nápisy na kovových předmětech.}

Soukromé nápisy představují přes polovinu celého souboru. Podobně jako u nápisů datovaných do 2. až 3. st. n. l. převládají dedikační nápisy nad nápisy funerálními. Celý souboru doplňuje nebývale vysoký počet veřejných nápisů, které představují 45 \letterpercent{} všech nápisů z daného období, a dále 2,5 \letterpercent{} nápisů, jejichž typ nebylo možné přesněji určit.

