
\section[charakteristika-epigrafické-produkce-ve-4.-st.-n.-l.]{Charakteristika epigrafické produkce ve 4. st. n. l.}

Nápisy 4. st. n. l. následují trend prudkého poklesu produkce, který začal již na přelomu 3. a 4. st. n. l. Dochází k uzavírání komunit, snižuje se i variabilita forem i obsahu nápisů. Spolu s tím se objevují i jasné projevy křesťanské tematiky, a to zejména v oblasti Perinthu, Abdéry a Byzantia. Opět převládá funerální funkce nápisů, jejich obsah je však podstatně odlišný.

\placetable[none]{}
\starttable[|l|]
\HL
\NC {\em Celkem:} 23 nápisů

{\em Region měst na pobřeží:} Abdéra 2, Bizóné 1, Byzantion 5, Kallipolis 1, Mesámbria 2, Perinthos (Hérakleia) 7 (celkem 18 nápisů)

{\em Region měst ve vnitrozemí:} Augusta Traiana 3, Nicopolis ad Istrum 1, Traianúpolis 1, (celkem 5 nápisů)

{\em Celkový počet individuálních lokalit}: 11

{\em Archeologický kontext nálezu:} funerální 2, sídelní 4, náboženský 1, sekundární 2, neznámý 14

{\em Materiál:} kámen 22 (mramor 12; vápenec 3, neznámý 7), jiný 1

{\em Dochování nosiče}: 100 \letterpercent{} 3, 50 \letterpercent{} 3, 25 \letterpercent{} 3, nemožno určit 15

{\em Objekt:} stéla 17, architektonický prvek 4, nástěnná malba 1, jiný 1

{\em Dekorace:} reliéf 6, malovaná 1, bez dekorace 16; reliéfní dekorace figurální 1 nápis (vyskytující se motiv: zvíře 1), architektonické prvky 7 nápisů (vyskytující se motiv: naiskos 2, sloup 1, báze sloupu či oltář 1, geometrický motiv 1, florální motiv 1, jiný 2)

{\em Typologie nápisu:} soukromé 14, veřejné 6, neurčitelné 3

{\em Soukromé nápisy:} funerální 10, dedikační 1, jiný 3 (z toho popis sochy 2)

{\em Veřejné nápisy:} honorifikační dekrety 2, jiný 4 (z toho milník 2)

{\em Délka:} aritm. průměr 8,34 řádku, medián 6, max. délka 33, min. délka 1

{\em Obsah:} latinský text 2 nápisy, písmo římského typu 4; hledané termíny (administrativní termíny 13 - celkem 18 výskytů, epigrafické formule 11 - 22 výskytů, honorifikační 0 - 0 výskytů, náboženské 5 - 5 výskytů, epiteton 1 - počet výskytů 1)

{\em Identita:} řecká božstva 1, egyptská božstva 0, římská božstva 0, křesťanská tematika 9, pokles náboženské terminologie, včetně vymizení lokálních kultů z nápisů, regionální epiteton 1, subregionální epiteton 0, kolektivní identita 3 termíny, celkem 3 výskyty - obyvatelé řeckých obcí z oblasti Thrákie 3, mimo ni 0; celkem 34 osob na nápisech, 11 nápisů s jednou osobou; max. 8 osob na nápis, aritm. průměr 1,48 osoby na nápis, medián 1; komunita řeckého a římského charakteru, thrácký prvek zcela chybí, jména pouze řecká (26,08 \letterpercent{}), pouze thrácká (0 \letterpercent{}), pouze římská (4,34 \letterpercent{}), kombinace řeckého a thráckého (0 \letterpercent{}), kombinace řeckého a římského (39,13 \letterpercent{}), kombinace thráckého a římského (0 \letterpercent{}), kombinovaná řecká, thrácká a římská jména (0 \letterpercent{}), jména nejistého původu (4,34 \letterpercent{}), beze jména (26,08 \letterpercent{}); geografická jména z oblasti Thrákie 3, mimo Thrákii 0;

\NC\AR
\HL
\HL
\stoptable

Epigrafická produkce ve 4. st. n. l. podstupuje velké změny. V první řadě se jedná o celkový prudký úpadek produkce o 94 \letterpercent{} ze 390 na 23 nápisů. Příčiny tohoto procesu je možné sledovat již v druhé polovině 3. st. n. l. Tehdejší společensko-politická krize vyústila v nestabilitu říše, způsobenou jak vnitřními rozbroji, ekonomickou krizí, ale i hrozícím nebezpečím za hranicemi římského impéria (Lozanov 2015, 87-88). Tyto jevy se samozřejmě podepsaly i na úpadku epigrafické produkce, což vyústilo v poměrně radikální proměnu jejího charakteru. Většina nápisů pochází z pobřežních oblastí, s největší produkcí v Hérakleii, býv. Perinthu, jak je patrné na mapě 6.10 v Apendixu 2.

Nejčastější formou objektu nesoucí nápis je již tradičně mramorová stéla, na níž začínají převládat křesťanské motivy, jako je kříž či christogram.

