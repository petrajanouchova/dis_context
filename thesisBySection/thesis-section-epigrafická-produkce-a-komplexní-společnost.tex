
\section[epigrafická-produkce-a-komplexní-společnost]{Epigrafická produkce a komplexní společnost}

Produkce nápisů byla dříve považována za jedno z hlavních měřítek hellénizace, případně romanizace obyvatelstva (Woolf 1998, 78). V tomto pojetí bylo vydávání nápisů chápáno jako důsledek a přímý projev přijetí nové identity a s ní spojených kulturních zvyků. Perspektivou moderní sociologie, či archeologické teorie se spíše zdá, že k rozšíření zvyku publikovat nápisy došlo v důsledku zvýšených potřeb souvisejících s nárůstem společenské a kulturní komplexity.

Produkce nápisů ve větším měřítku, tedy větší než několik jednotlivých kusů, je proces plně provázaný s velkým počtem institucí, řemesel a dalších specializovaných povolání. K pravidelnému vytváření nápisů bylo nutné dosáhnout nejen určitého stupně kulturního vývoje, který umožnil ocenit výhody psaného slova, ale zejména vybudovat propracovanou infrastrukturu, nutnou k produkci epigrafických monumentů (McLean 2008, 4-14).\footnote{Tím se myslí nejen existence patřičných technologií, dostupnost nutného materiálu, ale zejména přítomnost lidí se specializovanými a krajně odbornými znalostmi, jako je znalost těžby a opracování kamene, doprava materiálu, znalost technologie tesání kamene, rytí, malby, výroby reliéfu atp. K vytvoření textu je dále nutná alespoň minimální gramotnost nutná objednavatele a tvůrce, a nakonec technologie nutná vytvoření samotného nápisu.} Tyto aktivity jsou poměrně finančně náročné a z hlediska strategie dlouhodobé udržitelnosti společnosti nijak nepřispívají k obživě obyvatelstva. Tvoří tedy jakousi kulturní a symbolickou nadstavbu, která není primární podmínkou existence komunit. Proto se s nápisy ve větším měřítku setkáváme pouze ve společnostech, které pro ně měly jasné využití a byly schopné ocenit výhody spojené s jejich produkcí, jako je například uchovávání informací či jejich předávání (Flannery 1972, 411).

Zajištění epigrafické produkce ve větším měřítku podmiňuje velká míra organizace pracovní síly a existence odborné specializace. To jsou typické znaky komplexní společnosti na úrovni raného centralizovaného státu tak, jak jí představuje Joseph Tainter v knize {\em The Collapse of Complex Societies} (1988). V Tainterově pojetí se lidé přirozeně sdružují do komunit, které jsou různě velké a mají různý stupeň provázanosti, tzv. komplexity. S narůstající velikostí komunity dochází i k nezbytnému nárůstu provázanosti: aby se společnost udržela v chodu dochází k postupné specializaci rolí. S tím je spojená i větší stratifikace společnosti, důsledkem čehož se centralizuje a odděluje vládnoucí vrstva od produkčních vrstev. Větší komunita má taktéž větší spotřební nároky a dochází tak nutně k zintenzivnění produkce, vytvoření specializovaných povolání, které se zaměřují pouze na jednu činnost za účelem dlouhodobého udržení společnosti. Všichni členové společnosti se navzájem doplňují, dochází k centrálně organizovaném přerozdělování materiálu tak, aby se společnost mohla dále rozvíjet, a aby jedinec byl za svou specializovanou činnost odměněn, a tudíž i nadále přispíval svou aktivitou k chodu společnosti (Tainter 1988, 22-36).

Tainter (1988, 24-31) definuje několik základních typů společností na základě jejich komplexity, které mohou mít různé varianty a stupně. Uskupení na úrovni kmenové společnosti bývají zpravidla méně početná a složení společnosti je relativně homogenní: autoritu představuje vládnoucí jedinec a skupina jeho následovníků, ale dosah jejich moci je omezený.\footnote{Tabulka 3.01 v Apendixu 1 poskytuje přehledné srovnání dvou základních stupňů vývoje společnosti na základě jejich provázanosti, jak je popisuje Tainter (1988).} Moc vládnoucí vrstvy není podmíněna silou, ale spíše vytvářenými společenskými pouty, které se pracně budují a udržují, ale mohou poměrně jednoduše zaniknout (Sahlins 1963, 295). Dále v kmenové společnosti existuje několik specializovaných povolání především ve zpracování a ve výrobě, ale převážná část obyvatelstva se zabývá zemědělskou produkcí.

Společnosti na úrovni raného státu oproti tomu vykazují větší míru společenské stratifikace a znatelný nárůst ekonomické specializace povolání: velkou roli v nich hraje narůstající byrokratický aparát, specialisté ve výrobě, ale i v jiných odvětvích, spojených s chodem společnosti (Johnson 1973, 3-4). Profesionalizovaný státní aparát má veškerou moc ve svých rukách a je schopen vyžadovat dodržování řádů a nařízení od svých členů, ať už pomocí zákonů nebo vojenské síly.\footnote{Tainter 1988, 29: „{\em The features that set states apart, abstracting from the previous discussion, are: territorial organization, differentiation by class and occupation rather than by kinship, monopoly of force, authority to mobilize resources and personnel, and legal jurisdiction.}”} Základním předpokladem dlouhodobého fungování státního uskupení bývá existence ohraničeného území, které má zásadní vliv na formování pocitu sounáležitosti se zbytkem komunity. Proto mnohdy státy vynakládají nemalé prostředky na vytyčení hranic a jejich udržení. Stejně tak vládnoucí vrstvy musí vynakládat nemalé prostředky k legitimizaci vlastního výsadního postavení a nezřídka tak vnikají centrálně organizované ideologie a náboženství. S narůstající provázaností společnosti dochází k jejímu demografickému růstu, který však nezbytně ústí v další nárůst komplexity. Tento cyklus však neomylně vede k dosažení maximálních kapacit daného politického uspořádání, které bývá zpravidla následováno kolapsem či přeměnou dané společnosti, jak dosvědčují mnohé případy vyspělých civilizací a říší (Tainter 1998, 4-18). Tehdy dochází k poměrně prudkému poklesu produkčních aktivit, decentralizaci moci a redistribuce, k úpadku byrokratického aparátu, snížení počtu specializovaných zaměstnání. V důsledku snížené role centrální autority může docházet k místním konfliktům a rozpadu na menší politické celky, které si dokáží zajistit bezpečnost a udržitelný rozvoj. Dochází též k vymizení vedlejších produktů komplexní společnosti, jako je monumentální architektura, umění a vzdělanost, ale například i nápisy (Tainter 1988, 4).

Z výše řečeného vyplývá, že pouze dobře organizovaná společnost s centrální mocí a fungujícími institucemi je schopna dlouhodobě zajistit produkci nápisů. Neznamená to nutně, že každý raný stát musel k udržení svého chodu vytvářet nápisy: v průběhu historie se setkáváme i s celou řadou anepigrafických státních zřízení, které dokázaly fungovat po dlouhou dobu i bez nápisů, jako např. říše Inků. Proč tedy mnohé komplexní společnosti produkci nápisů podporovaly, či ji dokonce organizovaly? Zde je třeba odlišit dvě skupiny nápisů, lišící se svou funkcí ve společnosti: jednak jsou to nápisy veřejné, související primárně s aktivitami státního řízení, kontroly, vedení byrokratického aparátu a legitimizace moci. Druhou skupinou jsou nápisy soukromého charakteru, které primárně nesouvisí s organizací státního zřízení a legitimizací politické moci, ale jsou tzv. vedlejším produktem komplexní společnosti.

První skupina nápisů je primárně podporována politickou autoritou, protože produkce veřejných nápisů splňuje své požadované poslání a přináší autoritě prospěch, jako např. dodržování norem, zvýšení autority státu, zvýšení finanční odpovědnosti občanů vůči státu, legitimizace moci a státní ideologie, předávání a uchovávání informací (Johnson 1973, 2-4). K pravidelnému a dlouhodobému vytváření takovýchto nápisů je potřeba existence celé řady specialistů a infrastruktur. Vytvoření a udržování této infrastruktury je však velmi nákladné a může si ji dovolit pouze stabilní společenské zřízení, které disponuje dostatkem prostředků. S objevením jednotné infrastruktury, organizované politickou autoritou, tak dochází i k jisté standardizaci procedur, a tedy i formy a obsahu nápisů, jakožto produktů či projevů těchto procedur (Johnson 1973, 3-4).

Vedlejší projevy existence této infrastruktury se po určité době projeví i soukromé sféře. Ač tyto soukromé aktivity nejsou přímo podporovány státem a nemají přímý vliv na chod státního aparátu, v komplexních společnostech dochází k poměrně rychlému nárůstu nápisů soukromé povahy. I v nich se nadále odráží společenská stratifikace a specializace povolání, která se může projevit např. popisy povolání či funkcí na funerálních a dedikačních nápisech (Tainter 1988, 111-126). Naopak s poklesem komplexity společnosti dochází i k poklesu epigrafické aktivity, a to zejména ve skupině veřejných nápisů, jejichž produkce již není nadále podporována centrální politickou autoritou. K poklesu dochází i v soukromé sféře, a to vzhledem k vymizení infrastruktury nutné k publikaci a celkovému poklesu vzdělanosti.\footnote{U soukromých nápisů však snadněji dojde k udržení jejich produkce, ač v omezené míře a v rámci menších komunit. V těchto případech je také možné sledovat přeměnu jejich funkce, jako reakci na měnící společenské uspořádání, podobně jak k tomu došlo například na konci antiky (Tainter 1988, 151-152).}

Uskupení fungující na principu rodové a kmenové pospolitosti jsou oproti společnostem na úrovni raného státu méně stabilní, a to jak na ekonomické, tak politické úrovni. Relativně nejistá pozice kmenového náčelníka a menší schopnost akumulovat majetek mají za následek, že nedochází k vytváření specializované vrstvy institucí, které mají na starosti publikaci nápisů. Pokud je primárním zájmem kmenového náčelníka takovouto infrastrukturu vytvořit, stane se tak za cenu nemalých nákladů, a její existence většinou nemá dlouhého trvání (Flannery 1972, 412). Nápisy většinou vznikají jako reakce na konkrétní události, a nedochází k jejich produkci ve větším měřítku. Nápisy zde často slouží v rámci upevnění autority a legitimizace výsadní pozice politické autority, a spíše než pro svou normativní a informativní funkci jsou nápisy oceňovány pro svou jedinečnost a symbolický význam (Earle 1989, 85). Vzhledem k nákladnosti pořízení nápisů a omezené existence vzdělané vrstvy v kmenové společnosti se zvyk vydávat nápisy nerozšiřuje do soukromé sféry nebo jen ve velmi omezené míře. Tomu odpovídá i míra produkce soukromých nápisů, která je velmi nízká a omezuje se na nápisy sloužící k prosazování ideologie vládnoucí vrstvy a jejího postavení. Tento druh specializované produkce prestižního zboží je typický právě pro kmenově uspořádané společnosti a liší se od výroby ve větším měřítku, kterou umožňují společnosti na vyšším stupni komplexity (Clark a Parry 1990, 319-323).

Míra provázanosti společenského uspořádání se projevuje i v tak na první pohled vzdálených a nečekaných součástech lidské historie, jako je epigrafická produkce. Zvyk publikovat nápisy se v antické společnosti stal jednak prostředkem udržování chodu komplexních společností, ale zároveň i vedlejším produktem těchto uskupení na úrovni raného státu. Instituce a ekonomický potenciál komplexní společnost stály za udržením tohoto zvyku po dlouhou dobu, v případě některých komunit i dlouhá staletí. Oproti tomu v případě společností fungujících na kmenovém principu zastávala epigrafická produkce pouze marginální roli a její praktické využití pro chod společnosti mělo krátké trvání. Z toho důvodu se například v Thrákii dochoval jen minimální počet nápisů, které je možné připsat thráckým kmenovým vůdcům.\footnote{Tj. pocházejících z území ovládaného thráckými kmeny a z doby, kdy kmenové uspořádání bylo jedinou formou politické organizace, jak je možné vidět v kapitole 6.} Naopak v řeckých městech na pobřeží máme doklady o dlouhotrvající státem podporované epigrafické aktivitě, svědčící o relativně stabilní politické a společenské organizaci.

