
\section[koloniální-teoretické-modely]{Koloniální teoretické modely}

Takzvané koloniální teoretické modely vycházejí z tradice evropského kolonialismu 19. a první poloviny 20. století, který vnímá západní kulturu jako kulturní a společenskou normu vzešlou z antické tradice (Dietler 2005, 33-47; Jones 1997, 33). Národy a společnosti, které se od této normy odlišují, jsou vnímány jako podřadné, a jako objekt vhodný k civilizování západní řecko-římskou kulturou.

Nejdříve se zaměřím na koncepty akulturace a hellénizace, které, ač koncepty 19. století, hluboce rezonovaly v archeologických a antropologických publikacích po většinu 20. století, a v případě Thrákie mají ze specifických důvodů vliv na interpretaci materiálu i dodnes. I přes svůj velký dopad na současné bádání mají tyto přístupy celou řadu nedostatků. Hlavními body kritiky je jejich jednostranné zaměření a interpretace mezikulturních kontaktů, dále pak monolitická neměnnost a opomíjení často bohatých kontextů a situací vznikajících z reality společenských a kulturních interakcí (Dietler 2005, 33-47).

