
\section[materializace-životního-stylu-na-nápisech]{Materializace životního stylu na nápisech}

Historikové mnohdy vnímají nápisy zejména jako prameny pro jejich obsah, v neposlední řadě se též jedná o materiální projev společnosti, vycházející ze společenských tradic a hodnot (Dietler a Herbich 1998, 244-248). Podobně i nápisy jsou produktem společensko-kulturních norem, a ve způsobu jejich provedení, v jejich obsahu, či dokonce v jejich rozmístění a společenské funkci se odrážejí existující vzorce chování dané společnosti.

Podobně jako archeologické památky, či veškerý materiální svět, který nás obklopuje je v klasickém bourdieuovském pojetí možné chápat projevy epigrafické kultury jako jednu z materializací lidského {\em habitu} v okamžiku stvoření ({\em sensu} Bourdieu 1977, 15).\footnote{Je důležité nezaměňovat {\em epigraphic habit} a {\em habitus}. {\em Habitus} je dle klasické definice Pierra Bourdieu výsledkem působení vnějších společenských norem, které jedince obklopují, a stejně tak získaných zkušeností. Tyto vnější okolnosti ovlivňují lidské chování, vkus a následně ovlivňují i materiální kulturu, kterou si každý člověk vytváří okolo sebe (Bourdieu 1984, 173-5). {\em Epigraphic habit} je materiálním a společenským projevem {\em habitu}, jinými slovy se jedná zvyk vydávat důležité zprávy na permanentním médiu a veřejně je vystavovat (MacMullen 1982; Meyer 1990; Bodel 2001, 12-13).} V textu a v materiálním provedení nápisu se odráží jednak životní postoj jednotlivce, ale také jeho vkus, a nepřímo i uspořádání okolní společnosti. Dle této teorie se člověk, který je členem epigraficky aktivní společnosti a sdílí stejné podmínky s ostatními producenty nápisů, jako např. finanční prostředky, přístup k materiálům a technologiím, podobné vzdělání, se s největší pravděpodobností také stane producentem nápisů. Dochází tak k přirozenému postupnému šíření tohoto zvyku jeho nápodobou, což vede k vytváření lokálních variant epigrafických monumentů, ovlivněných vkusem místních komunit. Změny v epigrafické kultuře a její produkci tak mohou reflektovat změny celospolečenského charakteru.

V rámci sledování změn epigrafické produkce sleduji proměňující se identifikaci a identitu jednotlivců, ale i celých skupin v závislosti na proměňujícím se politickém a kulturním kontextu. Analýzou měnících se trendů na poli formování identity se snažím zjistit, zda případná hellénizace, tedy šíření řecké kultury a stylu života, měla zásadní vliv na projevy a uspořádání společnosti, či se jednalo o kombinace faktorů nesouvisejících s řeckou kulturou.

