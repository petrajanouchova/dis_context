
\section[nápisy-z-6.-až-8.-st.-n.-l.]{Nápisy z 6. až 8. st. n. l.}

Vytvořená HAT databáze obsahuje celkem 12 nápisů ze 6. st. n. l., které odpovídají kritériím datace, tedy nápisů s koeficientem 1. Dále databáze obsahuje i dva nápisy ze 7. až 8. st. n. l., tedy nápisy s koeficientem 0,5. Tato skupina nápisů vykazuje stejné rysy jako nápisy z 5. a 5. až 6. st. n. l., a proto je zmíním alespoň ve stručnosti.

Nápisy pocházejí převážně pobřežních oblastí, pouze dva nápisy pocházejí z vnitrozemí. Produkční centra jsou prakticky stejná jako v 5. st. př. n. l., jak je patrné z Mapy 6.11 v Apendixu 2. Objekty nesoucí nápis jsou převážně z mramoru s vyobrazením křesťanského kříže a křesťanská tematika se odráží i v textu nápisů, které jsou převážně funerálního zaměření. Zbylé čtyři nápisy označují výrobce keramických cihel, nesoucí vyobrazení kříže a pořečtěnou verzi termínu {\em koubikoularios}, tedy eunuch sloužící v předpokojích římského a později byzantského císaře. Složení epigraficky aktivní společnosti je dle původu osobních jmen z třetiny řecké, s odkazy na místní komunity jako je Filippopolis či na obyvatele řeckých měst z Malé Asie z oblasti Galatie a Ankýry. Tato skupina nápisů tedy nijak nevybočuje z trendu epigrafické produkce nastavené již na konci 4. st. n. l., a zejména pak v průběhu 5. st. n. l., kde se křesťanství, a zejména jeho projevy v rámci funerálního ritu, staly jediným přeživším motivem epigrafické produkce v Thrákii.

