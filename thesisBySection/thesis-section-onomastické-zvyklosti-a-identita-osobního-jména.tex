
\section[onomastické-zvyklosti-a-identita-osobního-jména]{Onomastické zvyklosti a identita osobního jména}

Osobní jména, která se vyskytovala na nápisech, odhalují, že epigrafická produkce byla zaměřená především na mužskou část populace. Mužská individuální jména, tedy jména osoby, která vystupovala jako primární {\em agens} nápisu, ať už zemřelý, dedikant či honorovaný člověk, tvořila 81,04 \letterpercent{} všech jmen. V rámci onomastických zvyků se primární identita člověka skládala z několika individuálních jmen, dále ze jmen rodičů, případně prarodičů a partnerů. V rolích rodičů se z 92,32 \letterpercent{} vyskytovala mužská jména. Stejně tak v rolích prarodičů figurovala v 91,53 \letterpercent{} mužská jména. Tento fakt poukazuje na silnou tradici identifikace pomocí {\em patronymik} nebo {\em paponymik} v rámci epigraficky aktivní společnosti Thrákie. Co se týče jmen partnerů, pak se obě pohlaví vyskytují přibližně stejně často (42,98 \letterpercent{} oproti. 52,05 \letterpercent{}). Ženská jména se celkově vyskytovala na 12,26 \letterpercent{} nápisů a jako primární agens nápisu figurovala ve 13,84 \letterpercent{}, podrobněji v tabulce 5.05 v Apendixu 1.

