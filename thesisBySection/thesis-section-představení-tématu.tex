
\section[představení-tématu]{Představení tématu}

Oblast antické Thrákie sloužila jako důležitá spojnice na pomezí Evropy a Asie, díky čemuž hrála důležitou roli ve vývoji historických událostí formujících antickou společnost. V průběhu několika století se stala svědkem řecké kolonizace, makedonské expanze a nadvlády, a konečně mocenského vzestupu a krize římské říše. Thrácká společnost byla v přímém kontaktu s různorodými světy řeckých {\em poleis}, hellénistických vládců i římské říše, a to na bázi diplomatické, obchodní i vojenské. Vzájemná interakce obyvatel Thrákie různého kulturního původu ovlivňovala každodenní život aristokracie i běžných obyvatel, což vedlo k postupným změnám nejen ve struktuře společnosti, ale i v ideových a materiálních projevech tamní kultury.

V 19. a 20. století byla Thrákie v duchu tehdy oblíbených akulturačních teorií považována za sféru vlivu řecké kultury, což mělo za následek nevyhnutelnou {\em hellénizaci} místního obyvatelstva. Hellénizační přístup sloužil v té době jako univerzální vysvětlení kulturních změn, k nimž docházelo při kontaktu místního obyvatelstva s řeckým světem, který byl považován za civilizačně vyspělejší. Neřecké komunity pak v rámci postupného a nevratného procesu hellénizace přijímaly novou materiální kulturu a ideologické koncepty na úkor vlastní kultury i identity vlastní a přijetí identity pořečtěné (Dietler 2005; Vranič 2014). Stejný přístup byl aplikován i na projevy epigrafické produkce, aniž by byla vzata v potaz specifika epigrafického materiálu a zhodnocena výpovědní hodnota nápisů v rámci dané problematiky. Užívání řeckého písma a vydávání nápisů v řeckém jazyce bylo automaticky považováno za jeden z typických znaků pořečtění thráckého obyvatelstva a přijmutí řecké kultury za vlastní (Mihailov 1977, 343-344; Sharankov 2011, 135-145).

