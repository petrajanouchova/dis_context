
\section[přehled-analyzovaného-materiálu]{Přehled analyzovaného materiálu}

Soubor nápisů přestavuje jedinečný komparativní soubor, reflektující společensko-kulturní změny a vývoj společnosti antické Thrákie. Databáze HAT tak krom informací o nápisech samých obsahuje i data o 5464 epigrafických osobách, které figurují v textech nápisů, jejichž jméno a identita se skládají z celkem 3956 osobních jmen a 178 termínů vyjádření kolektivní identity. Dále databáze obsahuje informace o 678 místech, kde byly nápisy nalezeny, ať už na území Thrákie či z nejbližšího okolí. V neposlední řadě se zde nacházejí data vztahující se k obsahu nápisů: celkem databáze zaznamenává 134 zeměpisných jmen, 233 epitet jednotlivých božstev, 137 termínů z náboženské oblasti, 112 termínů z oblasti organizace společnosti, 49 termínů vztahujících se k epigrafickým zvyklostem, a 37 termínů vztahující se k udílení poct v rámci epigrafické kultury.\footnote{Podrobný komentář a vysvětlení termínů se nachází v kapitole 4, věnované metodologii práce.}

