
\section[rozsah-a-typologie-nápisů]{Rozsah a typologie nápisů}

Dochovaná délka textů nápisů se pohybuje v rozmezí 1 až 270 řádků, s průměrnou délkou nápisu 4,6 řádek (aritmetický průměr 4,6 řádek; medián 3 řádky).\footnote{Aritmetický průměr ({\em mean}) je součtem všech hodnot, vydělený celkovým počtem prvků. Medián udává střední hodnotu souboru vzestupně seřazených hodnot a dělí tak soubor na dvě stejně početné poloviny. Hodnota mediánu není zpravidla ovlivněna extrémními hodnotami, např. velmi se odlišujícími maximálními či minimálními hodnotami, na rozdíl od aritmetického průměru, a lépe tak poukazuje na střední hodnoty daného souboru. Pro srovnání udávám vždy aritmetický průměr a medián daného souboru.} U 784 nápisů se nepodařilo zjistit ani jejich přibližnou délku, a to zejména díky jejich extrémně špatnému stavu dochování. Z celkové analýzy délky dochovaného textu je zřejmé, že tři čtvrtiny nápisů mají délku do 5 řádků (3503 nápisů), z čehož jedna třetina nápisů obsahuje právě dva řádky textu (1206 nápisů). Skupina nápisů s rozsahem od 6 do 10 řádků obsahuje 777 textů, skupina nápisů s rozsahem 11-20 řádků 319 textů, skupina nápisů s rozsahem textu 21-50 řádků obsahuje 55 textů, a nad 51 řádků se celkem vyskytuje pouhých 11 nápisů. Celkově převládají texty krátkého charakteru a nápisy delší než 20 řádků se vyskytují spíše výjimečně.

Celkem 3609 nápisů bylo publikováno jednotlivými osobami či skupinami lidí pro soukromé účely, což představuje 77 \letterpercent{} všech nápisů. Nápisy veřejné povahy, vydané politickou autoritou či samosprávní jednotkou, tvoří zhruba 15 \letterpercent{} všech nápisů, zbytek nebylo možné přesně určit. Velký počet soukromých nápisů plně odpovídá převaze kratších textů, které ve velké většině mají charakter textů publikovaných pro osobní potřebu jednotlivců či skupin lidí. Jak plyne z dat uvedených v tabulce 5.02 v Apendixu 1, průměrná délka soukromého nápisu je 3,71 řádku (aritm. průměr). Oproti tomu veřejné nápisy jsou velmi často popisnějšího charakteru: jejich průměrná délka je 10,93 řádku (aritm. průměr).

Soukromé nápisy je možné podle jejich obsahu a primární funkce rozdělit na několik základních skupin. Jak je patrné z tabulky 5.03 v Apendixu 1, nejpočetnějšími dvěma skupinami soukromých nápisů jsou dedikační a funerální nápisy. Tyto dvě skupiny dohromady tvoří přes dvě třetiny všech nápisů (37 \letterpercent{} pro nápisy funerální a 38 \letterpercent{} pro nápisy dedikační) a přes 97 \letterpercent{} všech soukromých nápisů. Zbývající kategorie soukromých nápisů, jako jsou vlastnické nápisy, či soukromé nápisy nespadající do žádné z výše zmíněných kategorií, či nápisy, které nebylo možné do konkrétní kategorie přiřadit, tvoří zbývající 2,6 \letterpercent{} soukromých nápisů.

