
\section[shrnutí-1]{Shrnutí}

Současná epigrafika a s ní související metody výzkumu antické společnosti procházejí v posledních letech prudkým vývojem. Původně velmi konzervativní disciplína v současnosti využívá mnoho moderních přístupů ze sousedních disciplín, jako je archeologie, antropologie či sociologie a nově vzniklý obor tzv. {\em digital humanities}.

V této práci vycházím z tradičních principů epigrafické práce za využití potenciálu, jaký nám nabízí moderní technologie a digitální zpracování dat. Nesporným přínosem je možnost zpracovávat velká množství dat a navzájem je propojovat tak, že vznikají nové úhly pohledu a srovnání nejen jednotlivých regionů mezi sebou, ale i porovnání vývoje v různých časových obdobích.

Metodologie zkoumání společnosti na základě nápisů je do velké míry neprobádaným územím s velkým potenciálem do budoucnosti. Představované metodologické postupy jsou snahou vyrovnat se se specifiky epigrafického materiálu za zachování základních principů současné vědecké práce, jako je princip testovatelnosti analýz a ověřitelnosti jejich výsledků. Zcela v souladu s požadavky současného výzkumu jsou výsledná data dostupná komukoliv k ověření či vyvrácení hypotéz, případné modifikaci a pokračování výzkumu z jiné perspektivy.

