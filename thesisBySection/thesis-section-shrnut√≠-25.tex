
\section[shrnutí-25]{Shrnutí}

Nápisy v antické Thrákii pocházejí především z produkčních center umístěných ve městech a jejich bezprostředním okolí, což je trend podobný i v jiných částech antického světa, např. v římské provincii {\em Gallia}. Zvýšená koncentrace nápisů se nachází i podél cest, ať už z menších osídlení v jejich okolí či související se správou cesty.

Zásadní role politické autority a organizace společnosti je vidět i na rozmístění veřejných nápisů, které se koncentrují v okolí velkých administrativních center, případně podél cest v podobě milníků a stavebních nápisů. Na veřejné nápisy navazují nepřímo i nápisy soukromé, které vykazují podobné rysy: velký počet soukromých nápisů pochází z okolí měst, kde žila velká část epigraficky aktivních obyvatel v římské době, a dále z menších sídel v okolí cest, jak je patrné z rozmístění funerálních nápisů. Soukromé nápisy se, na rozdíl od nápisů veřejných, vyskytují i mimo dosah cest a často v těžko přístupném terénu, jak je možné vidět zejména v případě dedikačních nápisů.

Rozmístění funerálních nápisů poukazuje, že zvyk zřizovat náhrobky se rozšířil zejména v řecké komunitě na pobřeží, kde si udržel výsadní postavení jak v předřímské, tak i v římské době. Do vnitrozemí se tato zvyklost rozšířila pouze částečně, a to do původně makedonských hellénistických sídel v Héraklei Sintské či Filippopoli.

Dedikační nápisy oproti tomu pocházejí převážně z thráckého vnitrozemí. Vzhledem k jejich rozmístění v krajině se dá usuzovat, že převážně rurální charakter svatyní typický pro thrácké náboženství přetrvával i v době římské. Svatyně se nalézaly v blízkosti cest a jejich největší koncentrace pochází z podhůří Rodop v okolí města Filippopolis a horských oblastí okolo města Serdica. Nápisy pocházejí jak z malých venkovských svatyněk, tak i z lokalit až s téměř 200 nápisy. Božstva na nápisech nesou řecká jména, jako je např. Asklépios, Apollón či Nymfy, nicméně velmi často mají i místní jméno, které s největší pravděpodobností poukazuje na lokální tradici a náboženský synkretismus mezi thráckými a řeckými náboženskými představami.

Srovnání s dostupnými archeologickými daty nabízí zcela novou perspektivu a ukazuje, že přístup k publikaci nápisů nebyl vždy stejný. Obecně více nápisů pochází z řeckých komunit než z thráckého vnitrozemí, nicméně i v rámci řeckých měst se setkáváme zhruba s jednou třetinou lokalit bez epigrafických nálezů. Proto nelze zcela jednoznačně považovat zvyk publikovat nápisy za univerzální projev řecké kultury a společenského uspořádání, ale k vysvětlení důvodů rozšíření nápisné kultury je nutné se zaměřit i na jiné druhy motivací, jako je snaha využití nápisů k šíření státem podporované ideologie a řízení státního celku či osobní pohnutky spojené s projevem víry a dosažení společenské prestiže.

