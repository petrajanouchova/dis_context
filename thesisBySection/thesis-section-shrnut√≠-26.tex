Podoba epigrafické produkce a její rozšíření v antické Thrákii souvisí spíše než s civilizačním vlivem řecké kultury s rozvojem tehdejšího společenského uspořádání. Většina nápisů je sice psána řecky a užitá terminologie taktéž pochází z řeckého prostředí, ale charakter epigrafické produkce nasvědčuje, že spíše než k nevědomému přebírání kulturních zvyklostí a transformaci thrácké společnosti směrem ke společnosti řecké docházelo k adopci písma a epigrafických zvyklostí jako propracovaného systému vyjadřování za uchování kulturní integrity a identity jednotlivých komunit. V návaznosti na rozvoj politické organizace v době římské dochází k intenzifikaci epigrafické produkce, a to především v souvislosti s vytvořením potřebné infrastruktury a s tím souvisejícím větším zapojením obyvatelstva.
\section[shrnutí-26]{Shrnutí}

Podoba epigrafické produkce a její rozšíření v antické Thrákii souvisí spíše než s civilizačním vlivem řecké kultury s rozvojem tehdejšího společenského uspořádání. Většina nápisů je sice psána řecky a užitá terminologie taktéž pochází z řeckého prostředí, ale charakter epigrafické produkce nasvědčuje, že spíše než k nevědomému přebírání kulturních zvyklostí a transformaci thrácké společnosti směrem ke společnosti řecké docházelo k adopci písma a epigrafických zvyklostí jako propracovaného systému vyjadřování za uchování kulturní integrity a identity jednotlivých komunit. V návaznosti na rozvoj politické organizace v době římské dochází k intenzifikaci epigrafické produkce, a to především v souvislosti s vytvořením potřebné infrastruktury a s tím souvisejícím větším zapojením obyvatelstva.
