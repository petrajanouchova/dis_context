
\section[shrnutí]{Shrnutí}

Nápisy nabízejí jedinečný náhled do společnosti, která je vytvářela. V jejich textech, ale i v jejich provedení se zrcadlí nejen demografické složení epigraficky aktivní části populace, ale i vztah obyvatel vůči okolní komunitě, společensko-politickému uspořádání, ale i například postoj vůči cizím kulturám.

Ač byly nápisy produkovány relativně omezenou skupinou lidí, jejich charakter míra a trvání jejich produkce napovídá o struktuře společnosti více, než se na první pohled může zdát. Role, jakou nápisy hrály ve společnosti, souvisí do velké míry s vývojovým stupněm společnosti, která je vytvářela a mírou provázanosti jejích jednotlivých složek. V jednodušších a menších komunitách se nápisy objevovaly pouze zřídka a sloužily převážně legitimizaci moci a poukázání na společenskou prestiž vládnoucích vrstev. Naopak ve společnostech se složitější vnitřní strukturou mohlo docházet k organizované produkci nápisů za účelem udržení chodu celého aparátu a zajištění lepší efektivity předávání a zaznamenávání informací. Jako vedlejší produkt těchto aktivit vznikla produkce soukromého charakteru, která se začala obsahově rozvíjet nezávisle na produkci organizované politickou autoritou, ale i nadále s ní byla spojena zejména na technologické úrovni.

