
\section[statistické-zpracování-epigrafických-dat]{Statistické zpracování epigrafických dat}

Statistické zpracování většího počtu nápisů v sobě nese celou řadu metodologických problémů. Epigrafická data jsou svou podstatou nedokonalá, často nekoherentní a velmi specifická. Z tohoto důvodu dochází ke statistickému zpracování nápisů relativně málo často, a to zejména v římském období (MacMullen 1982; Meyer 1990; Saller a Shaw 1984; Prag 2002; Prag 2013; Korhonen 2011). Nápisy představují nejen výborný sociologický a antropologický materiál pro studium antické společnosti, ale pokud se k nim přistupuje komplexně, mohou odhalit nové pohledy na tehdejší společnost, které by bylo jen velmi obtížné získat studiem jednotlivých nápisů. I přes tento neoddiskutovatelný potenciál statistického zpracování nápisů existuje velké množství přístupů, které nereflektují problematickou a často nekoherentní povahu informací získaných z nápisů, a může tak docházet k jejich zkreslení, či nepochopení. V následující pasáži se nejdříve věnuji míře nejistoty datace, která bývá často opomíjená v rámci epigrafických studií, a dále metodě výběru statisticky signifikantního souboru nápisů vhodných k temporální analýze celospolečenských trendů. V neposlední řadě poukazuji na problém nadhodnocování celkových čísel epigrafické produkce při statistického zpracování nápisů v jednotlivých stoletích a řešení za využití metody normalizované datace.

