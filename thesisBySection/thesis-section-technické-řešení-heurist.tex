
\section[technické-řešení-heurist]{Technické řešení: Heurist}

V rámci studijního pobytu v Sydney v roce 2012 jsem se seznámila s platformou {\em Heurist Scholar}, tehdy vyvíjenou na University of Sydney a rozhodla jsem se jí využít pro sběr dat v digitální podobě.\footnote{Heurist vznikl v roce 2005 pod vedením Dr. Iana Johnsona z University of Sydney. \useURL[url2][http://heuristnetwork.org/][][{\em http://heuristnetwork.org/}]\from[url2]. Od května 2013 je projekt veden jako Open Source a zájemci ho mohou využívat pro výzkumné účely bezplatně. Heurist je online databáze specificky navržená pro potřeby digitálního zaznamenávání historických dat v akademickém prostředí. Nejedná se tedy o databázi v pravém smyslu slova (například při porovnání s MySQL, PostgreSQL atp.), ale spíše o platformu umožňující sbírat, analyzovat a zveřejňovat data bez předchozích programátorských znalostí. Hlavní výhodou je přístupnost a flexibilita: kdokoliv s přístupem na internet a přístupem k databázi může přes webové rozhraní zadávat a analyzovat data. Heurist navíc v sobě kombinuje i určité funkčnosti geomapovacího softwaru, což je jedna z prerekvizit mého výzkumu. V roce 2013 epigrafické databáze ještě nezaznamenávaly geografická data a nebyly propojeny s geoinformatickými systémy (GIS) do té míry, jako je tomu v roce 2017, a tudíž tato funkčnost Heuristu měla zásadní roli na mé rozhodnutí používat právě Heurist.} Databázi jsem vytvořila v programu Heurist 3.1.0 v červnu 2013. Po prvotní fází zadávání dat jsem provedla revizi a úpravu struktury v říjnu 2013. Hlavní fáze zadávání dat probíhala od června 2013 do září 2016.\footnote{Celková doba nutná k vytvoření databáze: zhruba dva měsíce studia principů vytváření databází, jeden měsíc věnovaný testování a vylepšování databáze. Celkové zadávání nápisů do databáze bylo rozděleno mezi tři uživatele v poměru Janouchová (48 \letterpercent{}, 2362 nápisů), Kobierská (27 \letterpercent{}, 1348 nápisů), Ctibor (25 \letterpercent{}, 1270 nápisů); celkem 293 nápisů má dva autory z důvodu např. jeho několikanásobné opakované publikace, či re-editace v epigrafickém korpusu. Průměrná doba zadávání jednoho nápisu je zhruba 15 minut, s tím, že komplexnější nápisy trvaly až několik hodin, jednodušší naopak kratší dobu. Prostým vynásobením 15 minut x počet nápisů (4667) se dostaneme na šest měsíců nepřetržitého zadávání. Zadávání bylo vzhledem k studijnímu a pracovními vytížení členů týmu rozloženo do dvou let (hlavní zadávací fáze červen 2013-prosinec 2014; březen 2016-červen 2016). Po ukončení zadávání jsem jakožto administrátor kontrolovala kvalitu záznamů a jejich přesnost, a případně nápisy editovala, kde bylo nutné, což zabralo zhruba třetinu času oproti zadávání, tedy zhruba dva měsíce času. Tato fáze skončila v září 2016.} K 9. listopadu 2016 databáze obsahuje 4665 nápisů které díky své digitalizaci získaly jednotnou strukturu, ač pochází z různých epigrafických zdrojů, různé kvality a detailnosti zpracování. Data jsou volně dostupná komukoliv na internetu v několika digitálních formátech.\footnote{Heurist poskytuje možnost data především prohlížet a filtrovat, vyhledávat v nich, nicméně pro komplexnější analýzu je nutné data vyexportovat a využívat další specializované programy. Data získaná zadáváním informací do Heuristu je možné exportovat, a následně analyzovat, několika možnými způsoby. Geografická data je možné exportovat ve formátu KML ({\em keyhole markup language}) či {\em shapefile}, který je možné zobrazit v geoinformačním softwaru (ArcGIS, QGIS) či v Google Earth. Geografická data je možné exportovat ve formátu CSV ({\em comma separated value}) a transformovat do požadovaného formátu přímo v geoinformačním softwaru. Tabulární data (data textového charakteru) je možné exportovat ve formátu CSV a dále zpracovávat pomocí analytických programů jako je MS Excel, Google Spreadsheet, R. Databázi je možné jako celek exportovat jako XML formát, MySQL, PostgreSQL avšak tento způsob je vhodný spíše pro zálohování dat, než pro jejich analýzu a je vhodný pro zkušené uživatele. Data v neupraveném formátu společně s výsledky analýz a mapami jsou volně dostupná na adrese: \useURL[url3][https://github.com/petrajanouchova/hat_project][][{\em https://github.com/petrajanouchova/hat_project}]\from[url3]. V současné době HAT databáze používá verzi Heurist 4.2.8. (od 14. září 2016) a po přihlášení je dostupná na adrese: \useURL[url4][http://heurist.sydney.edu.au/h4/?db=HAT_Hellenization_of_Ancient_Thrace][][{\em http://heurist.sydney.edu.au/h4/?db=HAT_Hellenization_of_Ancient_Thrace}]\from[url4].}

