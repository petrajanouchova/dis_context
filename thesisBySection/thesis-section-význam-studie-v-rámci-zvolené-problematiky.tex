
\section[význam-studie-v-rámci-zvolené-problematiky]{Význam studie v rámci zvolené problematiky}

Komplexní práce, která by hodnotila epigrafickou produkci antické Thrákie v celé její šíři a zároveň reflektovala teoretické přístupy, v současné době chybí. Podobné souhrnné práce vznikaly zejména v 80. létech 20. století (Mihailov 1977; Samsaris 1980; Samsaris 1989; Loukopoulou 1989), nicméně od té doby došlo nejen k proměně metodologického přístupu k materiálu, ale zejména k objevení stovek dalších nápisů a k vydání zhruba 1500 dosud nepublikovaných nápisů. Tato práce se snaží tento nedostatek napravit a pojednává o stavu dochované epigrafické produkce publikované do r. 2016 v co možná nejúplnější podobě, reflektující vývoj současné metodologie.

Široký geografický záběr této práce a snaha sjednotit zdroje různé povahy je v současné době ojedinělý. Jednotlivé epigrafické korpusy, které současná práce kombinuje, jsou především založeny na principu moderních národů a málokterý zdroj kombinuje nápisy pocházející z území dnešního Bulharska, Řecka a Turecka, na jejichž území se antická Thrákie rozkládala. Tyto korpusy jsou nejen omezeny na území moderních států, ale často jsou psány různými jazyky, což znesnadňuje přístup k datům a omezuje výzkum nápisů na menší oblasti a regiony (např. Mihailov 1977; Loukopoulou 1989). Tato práce se snaží reagovat na tuto roztříštěnost zdrojů, nejednotnost metodologických přístupů a omezenou dostupnost v digitální formě. Základem práce je převedení nápisů do jednotného digitálního formátu, který je vhodný pro další zpracování, a jejich uchování v podobě databáze. Sjednocením metodologických přístupů editorů jednotlivých korpusů došlo k převedení různorodých dat do koherentní a navzájem srovnatelné formy, což dříve nebylo možné. Zároveň další výhodou je jejich převedení do elektronické podoby, což usnadňuje budoucí využití celého souboru pro další projekty.

Zcela v duchu moderních principů vědecké práce jsou analyzovaná data volně dostupná na internetu a kdokoliv je může využít v rámci svého projektu, navázat na ně, či naopak zhodnotit platnost zde uváděných závěrů a interpretací.\footnote{\useURL[url1][https://github.com/petrajanouchova/hat_project][][{\em https://github.com/petrajanouchova/hat_project}]\from[url1]} Užitá metodologie je kombinací několika přístupů z oblasti epigrafiky, archeologie, sociologie a antropologie, a jejím hlavním cílem je sledovat proměny společnosti na základě systematického studia dochovaných nápisů. V současné práci je tento přístup použit na oblast Thrákie, nicméně stejně tak může být aplikován i na jiná místa, kde docházelo k setkávání dvou či více kultur, a odkud se dochoval dostatečný počet epigrafických památek.

