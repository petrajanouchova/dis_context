
\subsection[celospolečenské-trendy-a-složení-společnosti]{Celospolečenské trendy a složení společnosti}

Dále se při analýze nápisů zaměřuji na proměňující se složení populace, která se aktivně zapojovala do epigrafické produkce. V tomto ohledu mě zajímá především konkrétní způsob sebeidentifikace individuálních osob na médiu permanentního charakteru a její proměny. Zejména se zaměřuji na identifikaci s rodinou a nejbližší komunitou, dále s většími společensko-politickými celky, a případně s celky etnickými (Jenkins 2008; Derks 2009, 239-244; Schuler 2012, 63-67). V neposlední řadě mě zajímá, jaký byl vztah epigraficky aktivní populace k řeckému etniku a řeckým kulturním zvyklostech, zda docházelo k jejich přejímání či adaptaci na nové podmínky a ke vzniku zcela nového symbolického systému hodnot a dorozumívacích prostředků.

S tím velmi úzce souvisí i analýza norem chování v závislosti na společenském postavení, pohlaví a v neposlední řadě i původu. Zajímám se o konkrétní projevy a šíření inovací a kulturních prvků v pobřežních a vnitrozemských komunitách v průběhu jednotlivých století.\footnote{Podrobněji v kapitole 6 pro analýzu nápisů v jednotlivých stoletích a v kapitole 7 pro analýzu rozmístění epigrafických produkčních center.} Sleduji jednak proměny složení epigraficky aktivní populace, ale i proměny epigrafické produkce obecně. Dále se zaměřuji na výskyt konkrétních prvků společenské organizace a projevů kultury a náboženství, a zasazuji roli a význam daných prvků do kontextu sledované společnosti v daném časovém období (Dietler 2005, 66-68).

