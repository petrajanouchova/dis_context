
\subsection[dedikační-nápisy-1]{Dedikační nápisy}

Dedikačních nápisů je celkem pět a pocházejí z pobřeží Egejského moře: tři z Abdéry a dva ze Zóné. Nápisy jsou datovány do druhé poloviny 5. st. př. n. l. a jedná se výhradně o věnování řeckým božstvům, jako Afrodíté {\em Syria}, Hermés {\em Agoráios}, Pýthia a Hestiá. Věnování obsahují tradiční dedikační formule užívané v řecké epigrafické tradici, avšak pouze jeden nápis obsahuje výlučně řecká jména. Další nápisy obsahují jména velmi špatně dochovaná, u nichž bohužel není možné zjistit jejich původ. V jednom případě je nápis psán iónsko-attickým dialektem a v jednom případě je použita epichórická alfabéta z Thasu/Paru. Z těchto nepřímých důkazů je tak možné usuzovat, že dedikace pocházely převážně z řecky mluvících komunit v egejské oblasti. Dochované dedikační nápisy nasvědčují, že v 5. st. př. n. l. nedocházelo k prolínání řeckých a thráckých náboženských představ, ale tehdejší náboženství a jeho projevy na nápisech si udržovaly poměrně konzervativní přístup.

