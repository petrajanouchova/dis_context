
\subsection[dedikační-nápisy-10]{Dedikační nápisy}

Dedikační nápisy se pomalu začínají prosazovat i v rámci místních thráckých kultů, avšak řecká božstva mají stále převahu. Dedikačních nápisů se dochovalo celkem devět, z nichž tři jsou věnovány Apollónovi, který v jednom případě nesl lokální přízvisko {\em Eptaikenthos} a v jednom {\em Toronténos}, dále tři nápisy věnované {\em héróovi}, který dvakrát nesl přízvisko {\em Stomiános}, jednou {\em Perkón}. V jednom případě je nápis věnován neznámým božstvům ze Sélymbrie. Nápisy pocházejí převážně od osob nesoucí řecká jména, nicméně v případě nápisu {\em Perinthos-Herakleia} 51 věnovaného Apollónovi {\em Toronténovi} se setkáváme s dedikanty nesoucími thrácká jména. Římská jména se objevují pouze v jednom případě na nápise {\em I Aeg Thrace} 202 z Maróneie.

