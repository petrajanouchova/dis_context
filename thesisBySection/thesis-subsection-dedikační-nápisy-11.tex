
\subsection[dedikační-nápisy-11]{Dedikační nápisy}

Římská politická přítomnost v regionu se začíná projevovat i na měnícím se složení epigraficky aktivní části populace a dochází k opětovnému zapojení místních elit. Zvyk věnovat nápisy božstvům thráckého i řeckého původu se v této době rozšířil zejména mezi thráckou aristokracii, zastávající vysoké úřednické funkce v provinciální správě, a částečně i mezi veterány římské armády. Dedikačních nápisů se dochovalo celkem 10, z nichž tři pocházejí z vnitrozemí a sedm z pobřežních oblastí. Vnitrozemské dedikace jsou ve dvou případech věnovány stratégy nesoucí římská a thrácká jména, což poukazuje na jejich thrácký aristokratický původ\footnote{{\em SEG} 54:639, {\em IG Bulg} 4 2338.} a v jednom případě veteránem římské armády, který nese výhradně římská jména.\footnote{Nápis {\em IG Bulg} 3,1 1410 je psaný z větší čísti latinsky, nicméně jméno božstva a věnování je psáno řecky. Jméno veterána a jeho vojenské zařazení je psáno latinsky, stejně tak jako časové určení, které udává dobu vlády císaře Vespasiána a jeho sedmý konzulát, tedy rok 76 n. l.} Věnování těchto vnitrozemských dedikací patří Héře {\em Sonkéténé}, božstvu {\em Médyzeovi} a Artemidě Kyperské. Dedikace z pobřeží pak náleží císaři Vespasiánovi, Apollónovi {\em Karsénovi}, Diovi {\em Patróovi}, Diovi {\em Zbelsúrdovi}, Hygiei, Nymfám a Néreovi. Tato věnování pocházejí od stratégů nesoucího římská, řecká i thrácká jména, propuštěnce nesoucího řecká jméno a od nejvyššího kněze Dionýsova kultu nesoucího řecká jména.

Nápisy zmiňující stratégy pocházejí převážně z jihovýchodní části Thrákie, z okolí Perinthu a Anchialu, kde pravděpodobně docházelo ke kulturnímu transferu ve zvýšené míře, alespoň na úrovni nejvyšších představitelů politické organizace. Jména římsko-thrácká se vyskytují pouze na nápise {\em IG Bulg} 4 2338, kde figuruje Flavios Dizalas, syn Esbenida a jeho partnerka Reptaterkos, dcera Hérakleidova.\footnote{Nápis pochází z regionu města Nicopolis ad Nestum, datován do poslední čtvrtiny 1. st. n. l.} Jak prozrazuje text nápisu, thrácký Dizalás byl stratégem v Thrákii a občanství a s ním i právo nosit jméno Flavios si získal ve službách římské říši. Jedná se tak o jeden z prvních potvrzených případů, kdy Thrák získal římské jméno za své zásluhy, a nikoliv jako výsledek smíšených manželství.\footnote{Vzhledem k tomu, že Dizalás zastával úřad stratéga, jednalo se pravděpodobně o potomka thráckých aristokratů, který byl za svou loajalitu odměněn římským občanstvím, a tedy i právem nosit jméno Flavios.} Zvyklosti dedikovat nápisy se uplatňují v thrácké komunitě pouze v nejvyšších vrstvách aristokratů, případně veteránů, a stále nepronikly mezi běžné obyvatelstvo.

