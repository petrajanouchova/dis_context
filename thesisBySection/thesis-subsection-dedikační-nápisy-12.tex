
\subsection[dedikační-nápisy-12]{Dedikační nápisy}

Dedikačních nápisů se dochovalo celkem 19, z čehož 16 je soukromé povahy a tři jsou na pomezí soukromého a veřejného nápisu, což oproti minulému období představuje dvojnásobný nárůst. Polovina nápisů pochází z Perinthu a na dalších místech na pobřeží a ve vnitrozemí se vyskytují nápisy od jednoho do tří exemplářů. Nápisy jsou zhotoveny převážně z kamene, jen v jednom případě se jedná o zlatý amulet z Perinthu.

Věnování jsou určena jak řeckým, tak lokálním božstvům, která částečně využívají řecká jména pro božstva v kombinaci s místním epitetem. Dedikace jsou určeny Apollónovi {\em Iatrovi}, Asklépiovi {\em Zylmyzdriénovi}, Diovi s přízviskem {\em Lofeités}, Sarápidovi, Héře a anonymním božstvům. Osobní jména dedikantů poukazují na smíšené složení věřících. U nápisů věnovaných božstvům s místním (subregionálním) epitetem se vyskytují jak thrácká jména ve vyšším poměru než u ostatních dedikací, což naznačuje, že lokální kulty byly oblíbené především u místního thráckého obyvatelstva. Přítomnost jmen smíšených řecko-thráckých a římsko-řeckých však ukazuje, že lokální kulty byly přístupné i lidem mimo thráckou komunitu.

