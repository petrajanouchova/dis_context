
\subsection[dedikační-nápisy-13]{Dedikační nápisy}

Dedikačních nápisů se dochovalo celkem 49, z čehož 44 bylo soukromé povahy a pět bylo zhotoveno jako veřejný text, tj. dedikace císaři či text zhotovený na náklady obce. Nejvíce nápisů pochází z regionu Augusty Traiany, celkem s osmi nápisy, dále z údolí středního toku Strýmónu sedm nápisů, poté je Serdica s pěti nápisy.

Dedikace jsou věnovány jak božstvům nesoucím řecká jména, tak božstvům místním, případně božstvům, jež vznikla smíšením řeckých a místních tradic.\footnote{Objevuje se osm Asklépiovi, sedm Diovi, šest Dionýsovi a Artemidě, pět Apollónovi, jedna Hygiei, Sabaziovi, Sarápidovi, Héře, Athéně a Semelé.} Místní epiteta vznikla pravděpodobně ze jména místa, kde se vyskytovala svatyně.\footnote{Z místních božstev je to především hérós/theos {\em Karabasmos}, {\em Manimazos}, {\em Marón}, {\em Pyrmeroulas}, {\em Zbelsúrdos} a dále místní přízviska spojená s Apollónem a Asklépiem a Artemidou, jako např. {\em Aulariokos}, {\em Epékoos}, {\em Tadénos}, {\em Beeuchios}, {\em Patróos}, {\em Suidénos} (Janouchová 2017 k rozšíření přízviska {\em Patróos}).} Lokální thrácké kulty se vyskytovaly na více než jednom místě pouze v ojedinělých případech, většinou se jednalo kult pevně svázaný s daným místem.

Přes polovinu osobních jmen na dedikačních nápisech tvořila římská jména, třetinu jména řecká a zhruba 10 \letterpercent{} jména thrácká. Nápisy s thráckými jmény pocházely výhradně z vnitrozemských oblastí z okolí Augusty Traiany a středního toku Strýmónu a byly věnovány jak řeckým božstvům, tak řeckým božstvům nesoucím místní přízviska.\footnote{Příkladem nápisů věnovaných řeckému božstvu s místním přízviskem jsou dva nápisy z Kabylé, kde byl ve 2. st. n. l. umístěný vojenský tábor pomocných jednotek a pravděpodobně zde žili Thrákové společně s Řeky. Nápisy {\em Kabyle} 18 a 19 nesou věnování Apollónovi {\em Tadénovi}, provedené členy {\em cohors II Lucensium}, kteří sami nesli thrácká osobní jména (Velkov 1991, 23-24).} Zhruba polovina těchto dedikací byla věnována Thráky sloužícími v římské armádě, kteří zároveň nesli římská a thrácká jména.

