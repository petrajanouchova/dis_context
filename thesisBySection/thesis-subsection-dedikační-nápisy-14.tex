
\subsection[dedikační-nápisy-14]{Dedikační nápisy}

Dedikačních nápisů se dochovalo celkem 76, což představuje čtyřnásobný nárůst oproti nápisům z 1. až 2. st. n. l. Nejvíce nápisů pochází z regionu Augusty Traiany celkem se 33 nápisy, dále sem patří Serdica s osmi nápisy, obce z údolí středního toku Strýmónu se sedmi nápisy a Filippopolis s šesti nápisy.

Dedikace jsou věnovány jak božstvům nesoucím řecká jména, tak božstvům místním či smíšeným.\footnote{Celkem se dochovalo 24 dedikací Apollónovi, šest Asklépiovi, tři Diovi a Artemidě, dvě Hygiei a Sabaziovi, po jedné Héře a Sarápidovi, Ísidě, Anúbidovi, Harpokratiónovi, Músám, Horám, Athéně, Héfaistovi, Matce bohů, Hérakleovi, Déméter a Télesforovi. Z místních božstev je to především {\em hérós}/{\em theos} nesoucí lokální přízviska či božstvo s lokálním přízviskem, jako Maró{\em n, Poénos, Kersénos, Dabatopiénos, Zerdénos, Estrakénos, Raniskelénos, Zbelsúrdos, Aularkénos, Téradéenos} či {\em Kendreisos}.} Podle počtu výskytů na nápisech představuje Apollón nejpopulárnější božstvo, které bylo dle množství místních epitet rozšířeno do celé řady míst a do velké míry splynulo s místními kulty. Apollón se stal nejoblíbenějším božstvem a je mu určeno celkem 24 dedikací pocházejících z oblastí jižně od pohoří Haimos, zejména v okolí řeky Tonzos a Hebros, a dále v Sélymbrii a Serdice. Mezi jeho nejčastěji užívaná místní epiteta patří: {\em Foibos}, {\em Poénos}, {\em Kersénos}, {\em Zerdénos}, {\em Epékoos}, {\em Genikos}, {\em Geniakos}, {\em Patróos}, {\em Raniskelénos}, {\em Aularkénos}, {\em Lykios}, {\em Kendreisos} a {\em Téradéenos}, ale bývá oslovován též jako {\em hérós}, {\em theos} či {\em kyrios} (Goceva 1992; Bujukliev 1997). Největší počet dedikací pochází ze dvou velkých svatyní Apollóna s místním přízviskem {\em Téradéenos} a {\em Zerdénos}, nalézajících se u vesnice Kran v Kazanlackém údolí, nedaleko místa hellénistické Seuthopole (Tabakova 1959; Tabakova-Tsanova 1980).\footnote{Ač Seuthopole zanikla již na začátku 3. st. př. n. l., je možné, že určité povědomí o řeckém náboženství zůstalo zakořeněno v místní populaci a postupem času se transformovalo do podoby místních kultů, které se v římské době začínají objevovat i na nápisech.} Jak dokazují dochovaná osobní jména, kult Apollóna byl přístupný všem a relativně vysoký počet thráckých jmen může naopak naznačovat oblíbenost kultu Apollóna mezi místní populací.\footnote{Celkem se dochovala jména 12 mužů, z nichž čtyři nesli thrácká jména, čtyři pouze římská jména a tři řecká jména, jedno jméno nebylo možné určit vzhledem ke špatnému stavu dochování. Dedikanti byli ve dvou případech vojáci, v jednom případě členem {\em búlé} a v jednom případě se jednalo o dohlížitele nad agorou z Plotínopole.} Teorii o splynutí kultu Apollóna s lokálními kulty taktéž nahrává forma nosiče nápisů a jeho dekorace: 16 nápisů neslo reliéfní dekoraci zobrazující jezdce na koni, taktéž známého jako fenomén tzv. thráckého jezdce, který byl oblíben i mezi thráckými vojáky a veterány (Kazarow 1938; Dimitrova 2002; Boteva 2002; 2005; 2007; Oppermann 2006). Totéž platí i o dalších oblíbených kultech, jako je kult Asklépia.\footnote{Asklépiovi je věnováno celkem šest dedikací a božstvo je s oslovováno jako {\em kyrios}, {\em theos} či přízviskem {\em Liménios}. Asklépios je uctíván dohromady s Hygieí a Télesforem, a to lidmi se jmény řeckého i římského původu, nikoliv však thráckého. Nápisy pocházejí především ze svatyně u vesnice Slivnica v regionu města Serdica (Boteva 1985).}

Obecně je možné sledovat větší zapojení Thráků do publikace nápisů. Pokud srovnáme poměr osobních jmen na dedikačních nápisech vůči poměru nápisů na všech nápisech z daného období, je možné sledovat 16,5\letterpercent{} nárůst z 12,5\letterpercent{} zastoupení thráckých jmen na všech nápisech z 2. st. n. l. na 29 \letterpercent{} zastoupení na dedikačních nápisech.\footnote{U jmen řeckých je to pokles o 10 \letterpercent{} a u jmen římských pokles o 1,5 \letterpercent{}, nicméně i přesto římská jména představují přes 40 \letterpercent{} všech jmen na dedikačních nápisech a řecká jména jednu čtvrtinu.} Oproti jiným druhům nápisů byli Thrákové byli více epigraficky aktivní, pokud šlo o náboženské aktivity, čemuž by odpovídalo i narůstající množství lokálních kultů, které vznikly spojením místní tradice a řeckého náboženského systému.

