
\subsection[dedikační-nápisy-18]{Dedikační nápisy}

Ve své tradiční podobě se dedikační nápis nedochoval ani jeden, nicméně následující čtyři nápisy je možné zahrnout do široce pojaté kategorie dedikací: dva nápisy na podstavcích soch, jejichž text se s největší pravděpodobností vztahoval k nedochovaným sochám\footnote{Nápisy {\em SEG} 46:845 a 845,2 zmiňují bohyni Hekaté a pravděpodobně řečníka Aischína, a byly objeveny v rámci archeologických vykopávek na Istanbulském {\em hippodromu}.}, a dále dvě proklínací tabulky.\footnote{{\em SEG} 60:747 a 748, které jsou datované do 4. až 5. st. n. l. Jejich text se sestává z magických formulí a relativně běžného palindromu {\em ablanathanalba} (Gager 1999, 136). Účelem těchto proklínacích destiček bylo získat sílu, lásku, zdraví, peníze či moc, či naopak jejich magickou mocí uškodit nepříteli. Jejich text je často nesrozumitelný a obsahuje magické formule, které měly přimět nadpřirozenou sílu vykonat přání pisatele. Jejich výskyt je vcelku běžný v průběhu celé antiky, jejich dochování z oblasti Thrákie je však poměrně vzácné. Tyto dva texty ({\em SEG} 60:747 a 748) byly nalezeny v průběhu archeologických vykopávek v roce 2010 v moderním Istanbulu, a je možné, že v budoucnosti bude objeveno více podobných nálezů.}

