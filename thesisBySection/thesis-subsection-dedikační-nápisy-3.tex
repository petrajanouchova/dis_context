
\subsection[dedikační-nápisy-3]{Dedikační nápisy}

Celkový počet dedikačních nápisů se oproti 5. st. př. n. l. výrazně nezměnil. Dochovalo se celkem pět nápisů z oblastí na pobřeží Egejského a Černého moře. Nápisy obsahují tradiční formule dedikační nápisů ({\em epoiésen}, {\em euchén}) a byly věnovány převážně řeckým božstvům Démétér, Kybelé a {\em héróům}, s přízviskem {\em Epénór} a {\em Mesopolités}. Nic nenasvědčuje, že by se Thrákové aktivně podíleli na vydávání dedikačních nápisů či že by nápisy pocházely z čistě thráckého prostředí. Texty jsou krátké a psány řecky, gramaticky a ortograficky nevykazují žádné zvláštnosti oproti standardnímu užití řečtiny a obsahují pouze řecká jména. Osobní jména, jména božstev, forma a obsah nápisů navíc poukazují na řecký původ dedikantů a převážně řecký kontext komunit, z nichž nápisy pocházely.

