
\subsection[dedikační-nápisy-5]{Dedikační nápisy}

Dedikačních nápisů se dochovalo celkem pět a, až na jednu výjimku ze Seuthopole, pocházejí všechny z regionu řeckých měst na pobřeží. Věnování byla převážně řeckým božstvům jako je Zeus, Dionýsos a Démétér, případně tehdejším významným panovníkům.\footnote{V jednom případě byl nápis věnován Diovi a králi Filippovi, s titulem zachránce ({\em sótér}). Tento nápis {\em I Aeg Thrace} 186 pochází z Maróneie a králem může být myšlen jak makedonský Filip II., tak Filip V. (Loukopoulou {\em et al.} 2005, 372). Další podobná dedikace {\em IK Sestos} 39 určená králi Filippovi pochází z Lýsimachei na Thráckém Chersonésu, nicméně v tomto případě se jedná spíše o makedonského krále Filippa V. (Krauss 1980, 92).}

Výskyt osobních jmen poukazuje na převahu řeckého prvku: řeckých osobních jmen se vyskytlo celkem šest, zatímco thrácká jména se vyskytla pouze jedno, a to na nápise {\em IG Bulg} 3,2 1732 pocházejícím z vnitrozemské Seuthopole, které věnoval Amaistas, syn Medista, kněz Dionýsova kultu.\footnote{Nápisy {\em IG Bulg} 3,2 1731 a 1732 dosvědčují existenci řeckých kultů na území Seuthopole, nebo kultů nesoucí řecká jména. Epigraficky je zde doložen jak kult Dionýsa, tak svatyně Velkých božstev ze Samothráké, nicméně archeologicky se existenci těchto kultů nepodařilo zcela prokázat, ač zde byla nalezena terakotová soška Kybelé a sošky inspirované uměním východního Středomoří (Nankov 2007, 63; Barrett a Nankov 2010, 17).} Tento fakt svědčí o existenci relativně zavedených náboženských praktik, a to zejména na území řeckých měst, ale v případě Seuthopole i mimo ně. Dedikační nápisy se vyskytují převážně v kontextu řeckých komunit. Jedinou výjimkou je nápis pocházející z vnitrozemské Seuthopole, která je považována za původně thrácké osídlení se silnou přítomností řeckého či makedonského prvku (Nankov 2011, 120). Není tedy překvapivé, že tento v thráckém prostředí relativně ojedinělý dedikační nápis pochází z komunity, která nebyla čistě thrácká, ale docházelo zde ke kulturnímu a náboženskému synkretismu.

