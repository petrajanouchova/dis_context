
\subsection[dedikační-nápisy-8]{Dedikační nápisy}

Dedikačních nápisů se dochovalo celkem 11 a většina z nich pochází z pobřežních oblastí, až na jeden nápis z lokality Madara, která leží ve vnitrozemí zhruba ve vzdálenosti 70 km východně od Odéssu. Jména dedikantů jsou výhradně řeckého původu, až na jednu výjimku z regionu Topeiru, odkud pochází nápis {\em I Aeg Thrace} 105 se jménem pravděpodobně thráckého původu.

Vyskytující se epigrafické formule dosvědčují udržení řeckých tradic typických pro dedikační nápisy, avšak v menší míře než v předcházejícím období. Typické věnovací formule jako {\em charistérion} se objevuje pouze jednou, v případě formule {\em euchén} pouze dvakrát. V jednom případě se dochovalo {\em enkómion} {\em I Aeg Thrace 205} určené egyptské bohyně Ísidě s délkou přes 44 řádků a původem z Maróneie. Jedná se na svou dobu o neobvyklý nápis jak formou, tak obsahem a poukazuje na trvající vliv řeckého náboženství v jižních oblastech Thrákie (Loukopoulou {\em et al.} 2005, 385).

Poprvé se v nápisech objevují i místní thrácké kulty, a i nadále se rozšiřující vliv náboženství egyptského původu. Výskyt nápisů věnovaných místním božstvům je omezen na jeden region, či dokonce jednu svatyni v Thrákii, jako např. {\em hérós} {\em Karabasmos} či {\em Perkón} z Odéssu. Dále se zde objevují božstva řecká, jako např. Zeus {\em Hypsistos} či samothrácká božstva. Objevuje se opět i bohyně {\em Fosforos}, nejčastěji ztotožňovaná s Artemidou, Hekaté či Bendidou (Janouchová 2013, 103-104). Podobně jako v předcházejícím století se vyskytují i dedikace původně egyptským božstvům Sarápidovi, Ísidě, Anúbidovi a Harpokratiónovi, a dále zbožštělým egyptským vládcům Ptolemaiovi a Kleopatře. Obecně dochází k většímu prolínání náboženských systémů a představ a jejich zaznamenávání na permanentní médium nápisu, což může nasvědčovat na větší otevřenost společnosti, zvýšenou míru kulturního kontaktu a změny ve společnosti, které vyústily v proměnu náboženského systému, tedy obecně jedné z nejkonzervativnějších částí kultury.

