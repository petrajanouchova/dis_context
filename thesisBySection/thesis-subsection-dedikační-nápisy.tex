
\subsection[dedikační-nápisy]{Dedikační nápisy}

Dedikační nápisy představují nejčastější typ dochovaných nápisů s celkem 1777 exempláři. Dedikačními nápisy jsou označovány předměty nesoucí věnování určitému božstvu a zpravidla bývají nalézány ve společnosti dalších votivních předmětů na místě bývalé svatyně daného božstva (Janouchová 2013; 2016; v tisku 2017).\footnote{V určitých případech může docházet k cyklické kategorizaci, kdy je svatyně označena svatyní pouze na základě přítomnosti dedikačních nápisů, či naopak je nápis označen jako dedikační na základě nálezu v rámci archeologického kontextu svatyně. Abych tomuto předešla, přidržuji se informací poskytovaných editory jednotlivých epigrafických korpusů, avšak primárně přihlížím k obsahu a charakteristice nápisu, a automaticky neřadím jako dedikace nápisy pocházející z lokality charakterizované jako svatyně.} Dedikační nápisy věnují jak jednotlivci, tak i skupiny lidí konkrétnímu božstvu, či božstvům, s prosbou o pomoc či jako díky za již vykonané skutky a dobrodiní. Zhotovitelé nápisu často uvádějí nejen důvod věnování nápisu, ale i svou identitu, která v kontextu dané komunity hrála relevantní roli. Na nápisy tak zaznamenávají své osobní jméno, původ, zařazení do určité rodiny a komunity, ale mnohdy zdůrazňují i své životní úspěchy, případně společenský status.\footnote{Tato sdělení a jejich vývoj v rámci komunit podrobněji rozebírám v chronologickém přehledu jednotlivých století pak v kapitole 6.}

Celkem bylo možné na dedikačních nápisech dle osobních jmen identifikovat 1649 osob, což představuje průměrně 1,077 osoby na nápis. Dedikační nápisy byly spíše individuální záležitostí a se skupinovými dedikacemi se setkáváme méně často než například s rodinnými funerálními nápisy. Dedikant se na nápisech zpravidla identifikoval jedním až dvěma osobními jmény, nejčastěji svým jménem a jménem rodiče. Celkem se dochovalo 3112 osobních jmen, z čehož jména řeckého původu představují 29,5 \letterpercent{}, jména thráckého původu 24,5 \letterpercent{}, jména římského původu 36 \letterpercent{} a jmen nejistého původu 10 \letterpercent{}. Většinu dedikantů představovali muži, a to celkem v 88 \letterpercent{} případů. Ženy dedikovaly nápisy jen v 7 \letterpercent{} případů a u 5 \letterpercent{} osobních jmen nešlo jednoznačně určit pohlaví dedikanta.

Na dedikačních nápisech zaznamenáváme větší míru zapojení thráckého obyvatelstva, než je tomu u funerálních nápisů, což úzce souvisí i s povahou dedikací a zapojení procedury věnování nápisů mezi náboženské zvyklosti thráckých kultů římské doby. Dedikace byly z převážné většiny určeny božstvům nesoucím kombinovaná řecká jména a lokální epiteta, poukazující na udržování místních tradic zároveň s adaptací nových kulturních a náboženských prvků. Nejoblíbenějšími božstvy byl Asklépios s 249 dedikacemi, Apollón se 139, Zeus se 131, Héra se 72, a dále Dionýsos a Nymfy se 38 dedikacemi, Hygieia s 31, Héraklés s 28 a Artemis s 24. Mezi božstva z východního Středomoří patří Sarápis s 12 dedikacemi, Ísis s 10, Anúbis se třemi, Harpokratión se čtyřmi, Mithra s osmi, Kybelé s jednou, syrská bohyně se dvěma, Velká Matka se sedmi, Sabazios s pěti dedikacemi.

U 586 nápisů se dochovalo božské epiteton, které u dvou třetin nápisů poukazovalo na místní původ kultu. Mezi místní {\em héroy} nesoucí lokální přízvisko patřil {\em Zylmyzdriénos} se 47 nápisy\footnote{Dochované varianty jména jsou {\em Zylmyzdriénos, Zymdrénos, Zymedrénos, Zymlydrénos, Zymydrénos, Zymyzdros, Zysdrénos}.}, {\em Salsúsénos} s 32 nápisy\footnote{Dochované varianty jména jsou {\em Saldénos, Saldobysénos, Saldokelénos, Saldoúissénos, Saldoúsenos, Saldoússénos, Saldoúusénos, Saldúisénos, Saldúsénos, Saldúusénos, Salénos, Saltobysénos, Saltobyssénos, Saltúusénos}.}, {\em Keiladeénos} s 20 nápisy\footnote{Dochované varianty jména jsou {\em Keiladeénos, Kiladeénos, Deiladebénos, Keilaiskénos, K{[}eilade{]}énos}.}, {\em Karistorénos} se 17 nápisy, {\em Karabasmos} s 11 nápisy, {\em Zbelsúrdos} s devíti nápisy.\footnote{Dochované varianty jména jsou {\em Zbelsúrdos} a {\em Zbelthiúrdos}.} U těchto nápisů je nepatrně vyšší zastoupení jmen thráckého původu s 26,5 \letterpercent{}, zatímco jména řeckého původu mírně klesla na 25,5 \letterpercent{}, jména římská zůstávají na přibližně stejné úrovni 37 \letterpercent{} a neurčená jména na 11 \letterpercent{}.\footnote{Podobné srovnání pro oblast okolo Novae ukazují, že zhruba 24 \letterpercent{} nápisů je věnováno lidmi s řeckým jménem, 14 \letterpercent{} se jménem thráckého původu a 62 \letterpercent{} se jménem latinského původu čili římského (Tomas 2016, 84).}

Dedikační nápisy se staly velmi oblíbené ve vnitrozemských komunitách, kde můžeme sledovat vznik svatyní s velkými koncentracemi nápisů zejména ve 2. a 3. st. n. l. Z těchto svatyní pochází velká část dedikací, avšak některé lokality jsou známy i díky jedinému, někdy i náhodnému nálezu votivního předmětu či nápisu.\footnote{Více o dedikačních nápisech v proměnách času v kapitole 6 a o jejich rozmístění v krajině v kapitole 7.}

