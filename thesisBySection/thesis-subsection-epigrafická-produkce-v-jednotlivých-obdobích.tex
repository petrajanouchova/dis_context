
\subsection[epigrafická-produkce-v-jednotlivých-obdobích]{Epigrafická produkce v jednotlivých obdobích}

V zásadě je možné epigrafickou produkci rozdělit podle míry celkové epigrafické produkce a charakteru materiálu do několika období, v nichž dochází k zásadnějším proměnám společnosti, způsobených jak změnami vnitřního uspořádání společnosti, tak i ovlivněním nově příchozími kulturami.

Období 6. a 5. st. př. n. l. je charakterizováno soustředěním epigrafické produkce v řeckých městech na pobřeží, a pouze sporadickým užitím písma v thráckém aristokratickém kontextu ve vnitrozemí. Použití nápisů v thráckém kontextu je odlišné od použití v řecky mluvících komunitách na pobřeží a nic nenasvědčuje přenosu kulturních zvyklostí v počátečních staletích. Ke kontaktu dochází v omezené míře na diplomatické úrovni v řadách aristokratů a dále na úrovni obyvatelstva žijícího v bezprostřední blízkosti řeckých měst, avšak ani zde nedochází k významnějšímu přenosu kultury a zvyklostí. Výskyt thráckých osobních jmen v řeckém kontextu poukazuje na existenci smíšených svazků, ale vzhledem k tomu, že se nejedná se více než o 3 \letterpercent{} z celkového počtu dochovaných osobních jmen, pak i smíšené svazky byly záležitostí spíše výjimečnou. Tuto dobu je možné ohraničit objevením prvních epigrafických památek v 6. st. př. n. l. a polovinou 4. st. př. n. l., kdy dochází k nárůstu moci thrácké aristokracie spolu s výraznějším vstupem Makedonie na scénu.

Ve 4. st. př. n. l. pokračuje epigrafická produkce v řeckých městech na pobřeží beze změny, pouze se rozšiřuje do více míst a zvyšuje se i celková produkce. Převahu mají nápisy soukromé povahy, ale objevují se i veřejné nápisy, jejichž charakter je do velké míry ovlivněn tradicí a normou obvyklou pro daný typ nápisu. Epigrafická produkce se v omezené míře objevuje i v nově vzniklých smíšených makedonsko-thráckých osídleních ve vnitrozemí, nicméně ani zde nedochází k prolínání onomastických zvyklostí či náboženských představ a obě komunity si udržují spíše tradiční charakter, soudě dle epigrafické produkce. Podobný trend nárůstu celkové epigrafické produkce pokračuje i ve 3. a 2. st. př. n. l., kdy však dochází k poklesu produkce soukromých nápisů v řeckých městech na pobřeží ve 3. st. př. n. l., ale zároveň dochází k nárůstu produkce veřejných nápisů, což může svědčit o nárůstu institucionální regulace. Pobřežní komunity se taktéž v této době více otevírají multinárodnostnímu hellénistickému společenství, což vede k objevení neřeckých náboženství a osob s neřeckými jmény. Zároveň se v řeckých městech objevují i božstva thráckého původu, avšak podíl thráckých jmen se stále pohybuje pod úrovní 3 \letterpercent{} obyvatelstva. Thrácká aristokracie se na epigrafické produkci podílí zcela minimálně, a to zejména v okolí smíšených (řecko-)makedonsko-thráckých osídlení.

Skupina nápisů z 1. st. př. n. l. a 1. st. n. l. poukazuje na úpadek epigrafické produkce jako důsledek společensko-politické krize a nestálosti poměrů v oblasti. V této době začíná do politické situace silně zasahovat Řím, ať už přímo, či pomocí nepřímé intervence. Od poloviny 1. st. n. l. je oblast oficiálně připojena pod římskou říši jako provincie {\em Thracia} a {\em Moesia Inferior}, avšak nedochází k významnému kulturnímu předělu či k zásadním změnám v epigrafické produkci.

Významnější změny naopak přináší 2. st. n. l., kdy dochází k prudkému nárůstu jak soukromé, tak veřejné epigrafické produkce. Reformy administrativního uspořádání provincie, nárůst byrokratické zátěže a centrální organizace má za následek zintenzivnění epigrafické produkce, objevení se nových institucí a specializovaných profesí. Zapojení thráckého obyvatelstva v armádě a městské samosprávě s sebou nese i nárůst gramotnosti, povědomosti o epigrafických zvyklostech, a tedy i nárůst produkce soukromých nápisů.

Nápisy z 2., ale i z 3. st. n. l. nabízejí větší různorodost obsahu nápisů, která je odrazem smíšeného složení epigraficky aktivní společnosti. Zhruba jedna pětina epigraficky aktivní populace jsou osoby nesoucí thrácké jméno, kteří však v mnoha případech přijali i jména římská pravděpodobně jako důkaz dosaženého společenského postavení a vykonaných skutků. Spolu s rozšířením epigrafické produkce mezi thrácké obyvatelstvo se objevují ve větší míře i thrácké kulty, které se smísily s řeckými, ale i dalšími středozemními kulty. Přítomnost lidí z jiných částí římské říše poukazuje na zvýšený pohyb obyvatelstva, a to zejména z Malé Asie. Dedikační nápisy se ve 2. a 3. st. n. l. stanou vůbec nejhojnější skupinou nápisů, což svědčí o zařazení epigrafické produkce k náboženským zvyklostem. Dochází i k rozšíření nových zvyklostí, jako je udávání věku na funerálních nápisech či pravidelné uvádění skutků a společenského postavení, stejně tak jako nejbližší rodiny a přátel, kteří si tak pravděpodobně zajišťovali dědická práva. V závislosti na společenském uspořádání římské říše se proměňují i onomastické zvyklosti obyvatel, ač tyto trendy byly omezeně pozorovatelné už i v druhé polovině 1. st. n. l. Uplatňuje se systém tří jmen, z nichž alespoň jedno jméno má římský původ a poukazuje tak na společenské postavení svého nositele. Tento trend je však narušen po roce 212 n. l., kdy právo nosit římské jméno má každý obyvatel římské říše a v průběhu času tudíž ztrácí na původní prestiži.

Zároveň s narůstající variabilitou obsahu se však ve 2. a 3. st. n. l. projevuje i relativní ustálení formy, a to jak ve vnější podobě epigrafických monumentů, tak i v případě opakujících se formulí a ustálených slovních spojení. V případě veřejných nápisů je vidět jasně centralizovaná role římské administrativy, která do jisté míry předepisuje výslednou podobu a obsah nápisů, avšak za ponechání prostoru pro vytvoření lokálních variant místními samosprávami. V případě soukromých nápisů se spíše než o vliv předepsaných norem, jedná o věc osobního vkusu kombinovanou s nápodobou již existujících zvyklostí, která nepřímo vytváří unifikovanou podobu soukromých nápisů.

Následné 4. a 5. st. n. l. jsou znamením úpadku epigrafické produkce a přeměny struktury společnosti. Reformy organizace společnosti vedou k jiným formám uplatňování politické autority, v nichž vydávání nápisů nehraje žádnou roli a od tohoto zvyku se upouští. Nárůst významu křesťanství v životě soukromých osob je možné pozorovat téměř na všech dochovaných nápisech soukromého charakteru. Z nápisů zcela mizí náboženství jiného typu než křesťanství a stírají se jak etnické rozdíly, tak i zaniká původní hierarchie společnosti a vytváří se nová struktura, provázaná úzce s křesťanskou církví. V následujících stoletích je možné spatřovat přežívající pozůstatky epigrafické produkce, ale ve velmi omezeném množství a pouze v soukromé sféře.

