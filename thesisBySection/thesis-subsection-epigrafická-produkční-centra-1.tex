
\subsection[epigrafická-produkční-centra-1]{Epigrafická produkční centra}

Oblast Thrákie se počty dochovaných nápisů řadí mezi střední producenty, a to zejména od 1. st. n. l., kdy se stala součástí římské říše. Srovnáme-li území, které Thrákie zhruba zaujímala a celkový počet dochovaných nápisů dostaneme se na hodnoty epigrafické produkce srovnatelné s římskou provincií {\em Gallia} či {\em Germania} (Woolf 1998, 81-82).

Bereme-li průměrnou velikost území Thrákie v době římské zhruba 130,000 km\high{2}, a celkový počet dochovaných nápisů je 4665, hustota nápisů činí 3,58 nápisu na 100 km\high{2}. Toto číslo se samozřejmě liší v čase. V době římské sledujeme zhruba 2,6x vyšší hustotu nápisů než v předcházejících obdobích. Skupina nápisů od 6. do 1. st. př. n. l. na stejně velkém území zaujímá hustotu zhruba 0,75 nápisu na 100 km\high{2}. V případě nápisů od 1. do 5. st. n. l. se jedná o hustotu 1,98 nápisů na 100 km\high{2}, nedatované nápisy mají hodnotu 0,85 nápisu na 100 km\high{2}. Tato čísla poukazují na výrazný nárůst epigrafické produkce v době, kdy se Thrákie stala součástí římské říše. V žádném případě ale Thrákie nedosahovala úrovní provincií na Apeninském poloostrově, kde byla hustota nápisů 13 na 100 km\high{2} a v případě {\em Latia} i 55 nápisů na 100 km\high{2}. Thrákie v době římské se tak řadí spíše na úroveň provincie {\em Gallia Comata, Belgica} či {\em Germania Inferior} s hustotou dva nápisy na 100 km\high{2} (Woolf 1998, 81-83).

Nárůst epigrafické produkce však nebyl na všech místech rovnoměrný, ale je možné sledovat shlukování nápisů v okolí měst, pohřebišť, svatyní a dále v okolí komunikací. Vzhledem k tomu, že nápisy sloužily zejména pro místní trh a nestaly se komoditou dálkového obchodu ve větším měřítku, se dá předpokládat, že produkční centra se nacházela nedaleko od místa nálezu nápisu, zpravidla v řádu několika kilometrů. Na základě srovnání archeologických dat Bekker-Nielsen (1989, 30-32) došel k závěru, že vzdálenosti mezi centry a jejich ekonomicky a společensky závislými regiony se pohybují od 10 do 37 km dle zastávané funkce. Dle Bekker-Nielsena je průměrnou vzdáleností od města k hranicím závislého regionu 37 km, což průměrná délka maximálního denního pochodu v římské době, která odpovídala 25 římským mílím. Délka denního pochodu se stala se jednou ze základních délkových jednotek udávaných např. při přesunech římského vojska (Madzharov 2009, 51). Z těchto čísel odvozuje délku půl-denního pochodu, tedy vzdálenost 18,5 km, která by umožňovala obyvatelům žijícím v okolí centra cestu do města a návrat domů v témže dni. Vzdálenost 10 km pak používá jako oblast ze které se obyvatelé mohli uchylovat pod ochranu centra v případě nebezpečí, a to v řádu dvou hodin chůze či jedné hodiny jízdy.\footnote{Částečně vycházím i údajů projektu Orbis vytvořeného na Stanford University, \useURL[url22][http://orbis.stanford.edu/][][{\em http://orbis.stanford.edu/}]\from[url22]. Tento projekt modeluje a mapuje rychlost přepravy v římském světě na základě geografických dat a známých historických a literárních zdrojů, nejčastěji tzv. itinerářů, o způsobech a rychlostech přepravy v římské světě. Autoři projektu zahrnují různé způsoby přepravy a k nim doplňují i průměrnou vzdálenost jako bylo možné tímto druhem přepravy urazit. Pro pěší pochod počítají 30 km za den, pro přepravu pomocí vozu taženého oslem 12 km za den, pro přepravu s větším vozem 36 km za den, pro jízdu na koni 56 km za den, zrychlenou jízdu na koni až 250 km za den, na lodi apod. Autoři berou v úvahu i zrychlený přesun vojenských jednotek, kterému stanovili průměrnou hodnotu 60 km za den (Scheidel {\em et al.} 2012, \useURL[url23][http://orbis.stanford.edu/orbis2012/ORBIS_v1paper_20120501.pdf][][{\em http://orbis.stanford.edu/orbis2012/ORBIS_v1paper_20120501.pdf}]\from[url23]). Pro vypočtení konkrétní trasy je možné zadat celou řadu parametrů, které zohledňují jak roční a denní dobu, rychlost přepravy, způsob přepravy, zda se jednalo o vojenskou či soukromou přepravu. Na základě těchto parametrů pak Orbis spočítá nejen dobu trvání cesty, ale i její finanční nákladnost v římských denárech. Příkladem může být např. cesta z Byzantia do Filippopole, která v červenci konaná pěšky s rychlostí 30 km za den trvá 11,7 dní. Pokud použijeme rychlost zrychleného vojenského přesunu 60 km za den, dostaneme se na 6,2 dní. Cesta mezi městy Serdica a Byzantion trvala 16,5 dní pěšího pochodu o rychlosti 30 km za den a 8,6 dne zrychlené vojenského přesunu o rychlosti 60 km za den. Odéssos a Byzantion jsou ve vzdálenosti 2,3 dní plavby na moři.}

Vzhledem k tomu, že náročnost terénu se na území Thrákie navzájem liší, zaokrouhlila jsem vzdálenosti půldenního pochodu na 20 km a celodenního pochodu na 40 km. Území v okruhu do 20 km považuji za oblast spadající pod přímý ekonomicky a společensky vliv daného centra. Území do vzdálenosti do 40 km představuje oblast maximálního dosahu vlivu daného centra, avšak s nižší mírou interakce mezi centrem a regionem. Pro účely této práce se držím střední hodnoty 20 km pro vyznačení regionu daného centra s přímým ekonomickým a kulturním vlivem, což také odpovídá hodnotám 1 až 3 koeficientu určení přesnosti místa nálezu. Pro srovnání udávám i počty nápisů nalezených v širším regionu 40 km, abych srovnala bezprostřední okolí centra s jeho regionem, a z nich vyplývající trendy v rozmístění epigrafické produkce.

Na základě rozmístění zvýšených koncentrací nápisů v terénu je možné určit produkční centra, tedy místa, kde byly nápisy s největší pravděpodobností vytvářeny a určeny pro ekonomicky závislý region. Tato produkční centra byla v Thrákii nerovnoměrně rozmístěna v závislosti na geografických podmínkách, ale i na demografickém uspořádání oblasti.

