
\subsection[epigrafická-produkční-centra]{Epigrafická produkční centra}

Místa se zvýšenou epigrafickou aktivitou, tedy místa v jejichž blízkosti byly nápisy nalezeny, poukazují s největší pravděpodobností i na epigrafickou aktivitu v tamní komunitě.\footnote{Rozmístění nápisů a vztah k jednotlivým produkčním centrům podrobněji rozebírám v kapitole 7.} V rámci místně zaměřené studie mě zajímá vztah rozmístění nálezových míst nápisů v krajině, a zda je z nich možné vypozorovat obecně platné trendy. Zajímá mě například, zda se nápisy objevují pouze v okolí řeckých měst, či se nalézají i v čistě thráckém kontextu a zda je například možné pozorovat rozdíly mezi umístěním nápisů v krajině v závislosti na jejich společenské funkci. Dále mě zajímá, jaké je rozmístění nápisů vůči místům s lidskou aktivitou, jakou jsou např. města či cesty, a zda je možné pozorovat vliv této infrastruktury na rozmístění nápisů, případně na jejich zvýšené koncentrace na určitých místech v závislosti na demografii regionu (Woolf 1998, 77-105; Woolf 2004, 157-164; Bodel 2001, 80-82). Navzájem srovnávám jednotlivé oblasti Thrákie s cílem postihnout opakující se vzorce chování na základě rozmístění nápisů. V neposlední řadě srovnávám epigrafická produkční centra se známými archeologickými prameny. Interpolací a statistickým zhodnocením jednotlivých epigrafických nálezů z vybraných lokalit se snažím zhodnotit míru relevance a výpovědní hodnoty epigrafických památek pro studium antické společnosti jako celku.

