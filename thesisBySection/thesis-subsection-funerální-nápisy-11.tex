
\subsection[funerální-nápisy-11]{Funerální nápisy}

Celkem se dochovalo 25 funerálních nápisů, z nichž většina pochází z Byzantia, Maróneie a Perinthu. V této době se poprvé objevily funerální sarkofágy, které nesly nápis, a zároveň sloužily jako úložiště tělesných pozůstatků zemřelého. Sarkofágy pocházely z Istanbulu, Perinthu a Maróneie, tedy měst, v němž měl Řím silné politické postavení.

Převaha osobních jmen řeckého původu poukazuje na přetrvávající charakter produkčních center, avšak se silným římským vlivem na podobu osobních jmen. Celkem 27 osob neslo řecká jména, 18 jména římská, šest jmen thráckých a tři jména neznámého či cizího původu. K mísení řeckých a římských, případně thráckých jmen dochází zcela výjimečně, přijímání systému tří římských jmen není ještě rozšířené. Thrácká jména se vyskytují pouze na třech nápisech a jejich nositelé pocházejí z vyšších společenských vrstev.\footnote{Jako např. stratégos Rhoimetalkás, syn Diasenea se svou ženou Besoulou, dcerou Moukaporida, a dětmi Kaproubebou a Daroutourme na nápise {\em I Aeg Thrace} 387.} Poprvé se objevují osobní jména (Titos) Flavios/Flavia a (Titos) Klaudios.\footnote{{\em I Aeg Thrace} 317, {\em Perinthos-Herakleia} 72, {\em Perinthos-Herakleia} 128.} Přijímání jmen po rodových jménech římských císařů byla typická praxe při udílení římského občanství, kdy se přidávaly k původnímu jménu nositele jako ocenění za služby ve správě provincie či v armádě (Topalilov 2012, 13). V těchto konkrétních případech je jméno tvořeno i dalšími osobními jmény římského původu a nasvědčuje to spíše faktu, že se nejednalo o případy nově uděleného občanství, ale o osoby, které jména získali po svých předcích, a šlo tedy o Římany, nikoliv o Řeky či Thráky, kteří přijali nová jména.\footnote{Vyjádření kolektivní identity se vyskytují pouze jednou na podstavci, na němž byla pravděpodobně umístěna socha thráckého gladiátora. Nápis {\em I Aeg Thrace} 484 označuje etnický původ (Thráx) a označení druhu gladiátora ({\em mormillón}, lat. {\em mirmillo}).}

Jazyk funerálních nápisů si udržuje do jisté míry tradiční charakter, jak naznačují vyskytující se formule, ale dochází i k určitým inovacím ve funerálním ritu a jeho projevech v epigrafice.\footnote{Invokační formule {\em chaire} je použita třikrát, nebožtík je titulován jako {\em hérós} čtyřikrát, okolo jdoucí je oslovován jednou ({\em parodeita}).} Poprvé se setkáváme s formulí, která poskytuje právní ochranu hrobu, či spíše sarkofágu, před jeho dalším použitím.\footnote{{\em Perinthos-Herakleia} 88. Pokud by došlo k tomu, že budou do hrobu „{\em vloženy ostatky někoho jiného, zaplatí viník částku {[}x{]} městu/pokladníkovi}”, do jehož regionu nápis spadá. Výše částky se lišila v dalších stoletích město od města, stejně tak i přesné znění formule.} Kdo by toto nařízení překročil, hrozí mu finanční postih, který je vymahatelný autoritami města. Tento text je typicky na konci funerální nápisu, a vyskytuje se výhradně na nápisech z Perinthu, tedy místě se silnou římskou přítomností. V dalších stoletích dojde k rozšíření této formule i do jiných míst, nicméně její výskyt pravděpodobně souvisí s administrativní změnou a regulací funerálního ritu v oblasti jihovýchodní Thrákie.

