
\subsection[funerální-nápisy-13]{Funerální nápisy}

Funerálních nápisů se dochovalo celkem 84, z čehož 80 má povahu soukromého nápisu a čtyři byly zhotoveny na náklady města, konkrétně Maróneie. Oproti předcházejícímu období počet funerálních nápisů narostl více než třikrát, což je možné spojovat s nárůstem celkového počtu obyvatel, ale i proměnami přístupu obyvatelstva k zřizování funerálních nápisů. Nápisy pocházejí z celého území Thrákie, tedy jak pobřeží, tak vnitrozemí. Hlavními produkčními centry je údolí středního toku Strýmónu s městem Hérakleia Sintská, Neiné a Parthicopolis, odkud pochází 25 nápisů, dále Byzantion s 18 nápisy, Perinthos a Maróneia s 11 nápisy. Obecně funerální nápisy pocházejí z okolí velkých městských center, kde žilo nejvíce lidí.

Text funerálních nápisů poukazují na proměňující se složení společnosti a nárůst důležitosti římského vojska. Texty nesou standardní formule, podobně jako v předcházejících stoletích, \footnote{Invokační formule {\em chaire} se objevuje 16krát, oslovení okolojdoucího ({\em parodeita}) 11krát. termín označující hrob ({\em tymbos}) se objevuje jednou, termín {\em stélé} třikrát, termín {\em mnémé} 15krát, {\em mnémeion} čtyřikrát, termín pro sarkofág celkem třikrát ({\em soros}). Také se objevuje termín {\em chamosorion}, popisující pravděpodobně sarkofág umístěný na podstavci či přímo na zemi. Poprvé se čtyřikrát objevuje invokační formule vzývající podsvětní bohy v latině ({\em Dis Manibus}) a dvakrát v řečtině ({\em Theoi Katachthonioi}).} a jsou zhotovovány členy nejbližší rodiny a hrobky slouží k pohřbům více členů rodiny.\footnote{Celkem 22 nápisů vyjmenovává několik členů rodiny, po čtyřech nápisy zmiňují potomky a sourozence, v jednom případě rodiče a ve dvou přítele. Ve dvou případech se setkáváme s nápisem zhotoveným propuštěným otrokem pro svého bývalého pána.} Celkem 20 nápisů, pocházejících zejména z Byzantia a Perinthu, uvádí věk zemřelého, což je zvyk typický pro římské nápisy. Latinský text se objevil na šesti nápisech, většinou u nápisů patřícím vojákům či veteránům, kteří nesli římská jména. Vojáků se na nápisech objevuje celkem pět, od běžných vojáků až po legionáře a jezdce. Jiná, než vojenská povolání zmiňují funkce vždy po jednom výskytu kněžího ({\em hiereus}), člena městské rady ({\em búleutés}), člena {\em gerúsie} ({\em gerúsiastés}). Na nápisech figurují taktéž čtyři otroci, z čehož dva jsou propuštěnci. Sarkofágy, podobně jako v 1. st. n. l., pocházejí převážně z Perinthu (Hérakleii) a Byzantia a dalších pobřežních lokalit, ale dva pocházejí i z thráckého vnitrozemí. Z osmi sarkofágů z Perinthu (Hérakleie) a Byzantia pouze dva nesou ochrannou formuli, která zakazuje nové použití sarkofágu pod peněžní pokutou\footnote{Tato formule se objevila již na sarkofázích z Perinthu (Hérakleii) v 1. st. n. l. a tento zvyk se rozšířil do nedalekého Byzantia.}, ale téměř na všech se udává věk zemřelého, případně jeho původ a kdo nechal sarkofág zhotovit.

Geografický původ zemřelého poukazuje na zvýšenou migraci z oblastí Malé Asie, což je patrné zejména u nápisů pocházejících z Filippopole (Topalilov 2012, 13; Sharankov 2011, 143).\footnote{Celkem sedm nápisů udává geografický původ zemřelého, a to jako pocházejícího z Byzantia, Abdéry (dvakrát), Hérakleie, Filippopole, Kappadokie, Níkaie a Apameie v Bíthýnii.} Vyskytující se osobní jména i jsou nadále převážně řecká, nicméně je možné pozorovat nárůst jak jmen římských, tak především i jmen thráckých. Řecká jména představují 42 \letterpercent{}, thrácká jména představují 18 \letterpercent{}, římská jména představují necelou třetinu všech jmen. Thrácká jména se vyskytují jak samostatně, tak v malé míře i v kombinaci s řeckými i římskými jmény. Celé dvě třetiny nápisů s thráckým jménem pocházejí z údolí středního toku Strýmónu, kde se thrácká populace zapojovala do zvyku zhotovovat veřejně vystavené funerální nápisy zejména v první polovině 2. st. n. l.\footnote{Z celkem 22 nápisů v nichž se vyskytuje alespoň jedno thrácké jméno jich 14 pochází z okolí řeky Strýmón. Celkem se zde vyskytuje 40 osob, z nichž osm jsou ženy} Římská jména se vyskytují převážně samostatně, v menší míře v kombinaci s řeckými, a téměř minimálně v kombinaci s thráckými jmény. V kombinaci s thráckými jmény se vyskytují většinou u jedinců, kteří sloužili v římské armádě a přijali nové jméno, nicméně se i nadále odkazují na svůj thrácký původ.\footnote{Např. na nápise {\em IG Bulg} 5 5462 z Filippopole Ailios Polemón, {\em benefikarios}, věnoval svým předkům Beithytraleiovi, synovi Taséa(?) a Kouété, dceři Dydéa(?). Nápis Manov 2008 82 dokumentuje případ veterána se jmény Gaios Valerios Poudens, jehož otec nesl jméno Gaios Ourginios Poudens a matka jméno Moukasoké. Veteránova žena nesla jméno Severa, syn jméno Ioulios Maximos, a sestra Zaikaidenthé.} Kombinovaná jména, která by odkazovala na nedávné přijetí římského jména a s ním i spojených privilegií, se na rozdíl od 1. st. n. l. dochovala pouze v minimální míře. To může značit, že Thrákové nedostávali za svou službu právo nosit římská jména, či se plně adaptovali na systém tří jmen, zcela opustili od užívání původních thráckých jmen a stali se tak nerozlišitelnými od Římanů. Nicméně vzhledem k udržení thráckých jmen na nápisech ze 3. st. n. l. je toto vysvětlení nedostačující.

