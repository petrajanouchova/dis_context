
\subsection[funerální-nápisy-16]{Funerální nápisy}

Funerálních nápisů se dochovalo celkem devět, což představuje takřka šestinásobný propad v produkci oproti nápisům datovaným do 2. až 3. st. n. l. Celkem čtyři nápisy pocházejí z Maróneie, dva z Perinthu. V jednom případě se jedná o sarkofág, jehož neoprávněné použití bylo chráněno institucemi v Maróneii, podobně jako u stejných nápisů z 1. až 3. st. n. l.\footnote{Nápis {\em I Aeg Thrace} 312.} Většina dochovaných osobních jmen je řeckého a římského původu, pouze v údolí středního toku Strýmónu se dochovalo šest osob se jmény pravděpodobně thráckého původu.\footnote{Nápis Manov 2008 168.} Úbytek počtu nápisů je jedinou zásadní změnu, celkový charakter funerálních nápisů zůstává stejný jako v předcházejícím období, nakolik je možné usuzovat z omezeného vzorku nápisů.

