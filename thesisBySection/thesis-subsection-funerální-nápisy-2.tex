
\subsection[funerální-nápisy-2]{Funerální nápisy}

Skupina nápisů datovaných do 5. a 4. st. př. n. l. obsahuje 119 primárních funerálních nápisů tesaných do kamene. Až 95 z nich pochází z Apollónie Pontské, což je pravděpodobně důsledek nedávného archeologického výzkumu ve městě, při němž byly objeveny rozsáhlé nekropole Kalfata a Budžaka (Gyuzelev 2002, 2005, 2013; Velkov 2005).

Oproti předcházejícímu období je možné pozorovat jen velmi pozvolný nárůst výskytu thráckých jmen na funerálních nápisech. Řecká a thrácká jména se společně vyskytují na třech funerálních nápisech, což představuje pouhých 2,4 \letterpercent{} funerálních nápisů z daného období. Vyskytující se thrácká jména patří výhradně ženám a všechny tři nápisy pocházejí z nekropole Kalfata ve městě Apollónia Pontská, kde pravděpodobně docházelo ke kontaktu mezi Thráky a Řeky a k navazování partnerských vztahů mezi jedinci s odlišným původem.\footnote{Žena thráckého původu partnerka či dcera muže nesoucí jméno řeckého původu se jménem Paibiné Augé, partnerka/dcera Hermaia z nápisu {\em IG Bulg} 1,2 430; tak jako žena se jménem řeckého původu partnerka či dcera muže se jménem thráckého původu: Faniché, partnerka/dcera Kerzea (Gyuzelev 2002, 20), a dále Dioskoridé, partnerka/dcera Basstakilea z nápisu {\em IG Bulg} 1,2 440. Fakt, že dochované nápisy dokumentují vytváření příbuzenských kontaktů mezi Thráky a Řeky, znamená, že obě mezi oběma komunitami docházelo ke kontaktu již delší dobu a vznikly zde vztahy, které epigrafické prameny nepostihují vůbec, či nepřímo a se zpožděním i desítek let.}

Tzv. sekundárně funerální se dochoval pouze jeden nápis zhotovený na stříbrné nádobě. Tento nápis taktéž pochází z kontextu bohaté pohřební výbavy hrobky thráckého aristokrata, která je známá pod názvem Lešnikova Mogila a nachází se u moderního města Kazanlak ({\em SEG} 55:742; Kitov 1995, 19-21). Podobně jako u stejné skupiny nápisů datovaných do 5. st. př. n. l. se jedná o stříbrnou nádobu sloužící za života majitele, jehož jméno pravděpodobně nápis nese.\footnote{O přesné podobě jména a znění nápisu se badatelé nemohou shodnout, vzhledem k tomu, že se jedná o jediný výskyt tohoto jména. Dimitrov (2009, 31-32) navrhuje interpretaci „(nádoba) Dynta, syna Zeila”. Theodossiev (1997, 174) navrhuje znění „(fiálé) Dynta, syna Zemya” a Dana (2015, 247) navrhuje znění „(majetek) Dyntozelmia”.} Nápis sloužil v okruhu thráckých aristokratů jakožto předmět poukazující na společenskou prestiž majitele, podobně jako u nápisů na kovových nádobách datovaných do 5. st. př. n. l.

