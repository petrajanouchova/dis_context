
\subsection[funerální-nápisy-3]{Funerální nápisy}

Primární funerální nápisy si i ve 4. st. př. n. l. uchovávají stejný charakter jako v 5. st. př. n. l., doplněné o nápisy na vnitřní architektuře hrobek. Primární funerální nápisy i nadále pocházejí ze stejných či charakteristikou velmi podobných komunit na pobřeží, stejně tak jako sekundární funerální nápisy převážně pocházejí z kontextu thrácké aristokracie ve vnitrozemí.

\subsubsection[primární-funerální-nápisy-1]{Primární funerální nápisy}

Celkem se jedná o 142 nápisů tesaných do kamene, zhotovených nejčastěji ve tvaru stély.\footnote{Typický text funerálního nápisu představuje jméno zemřelého se jménem otce či partnera v genitivu. Pouze v jednom případě tento jednoduchý text doplňují okolnosti smrti a detaily z života zemřelého (Gyuzelev 2013, 20). Průměrná délka funerálního nápisu je 2,3 řádku, nicméně nejdelší text má až šest řádků.} Funerální nápisy pocházejí opět z řeckých komunit podél mořského pobřeží: největší skupina 55 funerálních nápisů pochází z Apollónie Pontské na pobřeží Černého moře. Další významnou skupinu tvoří 29 nápisů ze Strýmé na egejském pobřeží.\footnote{V sedmi případech je dokonce dochováno použití dórského dialektu v původně dórských koloniích Mesámbria s pěti nápisy, Byzantion s jedním; jeden nápis pochází z původně nedórského Odéssu.}

Osobní jména na funerálních nápisech jsou z 85 \letterpercent{} řeckého původu. V jednom případě nápisu {\em I Aeg Thrace} 153 ze Strýmé na egejském pobřeží je zaznamenáno mísení onomastických tradic, kdy jedna osoba nesoucí řecké jméno může mít thráckého otce.\footnote{Jméno otce Dadás není výhradně thrácké, tudíž není možné na jeho thrácký původ poukázat se 100 \letterpercent{} jistotou.} Vyjádření identity na funerálních nápisech na kameni poukazuje na čtyři případy příslušnosti k řecky mluvícím komunitám, naznačené užitím termínů Kyrenáios, Abdérítés, Amfipolítés a Kyzikénos, což poukazuje na stále relativně konzervativní společnost, kde imigrace nebyla častá, či nebylo nutné uvádět svůj geografický původ v rámci funerálních nápisů.\footnote{Geografická jména zmiňují pouze v jednom případě Olynthos, avšak text nápisu {\em I Aeg Thrace} 214 je příliš poškozen, abychom z něj mohli usuzovat něco dalšího.} Hledané termíny poukazují opět na převážně řeckou komunitu, která udržovala tradiční zvyky, nicméně tradiční formule náležící funerálním nápisům se vyskytla pouze jednou. Nesetkáváme se ani s individualizovanou prezentací jednotlivých zemřelých a poukazování na jejich společenský status, výjimkou je jeden nápis propuštěné otrokyně na nápise {\em IG Bulg} 1,2 334novies a.

Dva nápisy zhotovené na stěně vnitřní komory mohylové hrobky pocházejí z okolí hory Ganos\footnote{Text nápisu SEG 56:827bis: ΚΑΘΑΘΑ. Z téže hrobky dále pochází stříbrná nádoba {\em SEG} 56:828, označující majitele nádoby, thráckého aristokrata Térea (Delemen 2006, 261-262). Z historických pramenů víme, že jméno Térés tradičně patřilo panovníkům z rodu thráckých Odrysů, avšak nedokážeme s přesností posoudit, zda se jednalo o téhož Térea, majitele této hrobky (Hdt. 4.80, 7.137; Thuc. 2.29, 2.67, 2.95; Dem. 12.8).} na pobřeží Marmarského moře a z lokality u vesnice Smjadovo v severní části Thrákie.\footnote{Nápis {\em SEG} 52:712 ze Smjadova byl nalezen na překladovém kameni, který byl součástí vnitřní architektury hrobky, patřící pravděpodobně členovi či člence thrácké aristokracie. Vzhledem k tomu, že se jedná o nápis pocházející z vnitřního prostoru hrobky, jednalo se nejspíše o jméno majitelky hrobky: Gonimase(ze), ženy Seutha. Přesné znění nápisu {\em SEG} 52:712 není zcela jasné. Dimitrov (2009, 17-18) navrhuje překlad „Gonimaseze, Seuthova žena”, zatímco Dana (2015, 246-247) text čte následovně „Gonimase, Seuthova žena, (i nadále) žije!”.} Z uvedeného charakteru nápisů plyne, že přístup do samotných hrobek či k předmětům na kovových předmětech, měla velmi omezené skupina lidí z nejbližšího okolí majitele. Dle původu dochovaných jmen a charakteru pohřební výbavy je možné soudit, že se jednalo o členy thrácké aristokracie. Užití písma v tomto případě bylo omezeno převážně na utilitární funkci označení majitele či zhotovitele a obsah textu se nevázal k funerálnímu ritu samotnému. Svým charakterem se tak nápisy z vnitřní architektury hrobek řadí spíše k nápisům na kovových nádobách a jiných materiálech, nalezených v thráckém vnitrozemí v kontextu aristokratických hrobek v 5. až 3. st. př. n. l.

\subsubsection[sekundární-funerální-nápisy-1]{Sekundární funerální nápisy}

Do skupiny sekundárních funerálních nápisů patří celkem čtyři nápisy na kovových předmětech, zhotovené z drahých kovů: jeden zlatý pečetní prsten {\em SEG} 58:699 nese jméno a podobiznu pravděpodobně svého majitele Seutha, syna Térea. Dále do této skupiny patří stříbrná nádoba nesoucí nápis {\em SEG} 56:828 se jménem Térea, označující pravděpodobně majitele nádoby, zmíněná výše. Další nápis {\em SEG} 53:706 na stříbrné nádobě s textem „{\em Kotys, z Ergiské}”, který taktéž označuje majitele a geografický termín Ergiské označuje buď původ majitele či původ nádoby samotné. Bohužel u této nádoby není zcela možné určit její přibližné místo nálezu, protože pochází z aukce, avšak podobá se nádobám pocházejícím z území Thrákie (Loukopoulou 2008, 158-159). Poslední nápis {\em SEG} 46:851 na kovovém kratéru nese nápis označující pravděpodobně její obsah.\footnote{Nápis naznačuje, že je obsah jsou čtyři kyliky, tedy číše na víno.}

Ač byly všechny předměty nalezeny ve funerálním kontextu, obsah textu se nevtahuje specificky k pohřebnímu ritu. Nápisy představují součást pohřební výbavy uložené do hrobu společně se zemřelým a mají čistě utilitární funkci. Původ majitele, alespoň na kolik je možné soudit dle osobních jmen, byl thrácký.\footnote{Seuthés, Térés, Kotys jsou tradiční jména thrácké aristokracie, vyskytující se i v literárních pramenech (Thuc. 2. 29, 2.67, 2.95, 2. 97; Strabo 12.3.29; Tac. {\em Ann}. 2.64-65).} Naopak obsah nádoby je označen typicky řeckým způsobem, což může poukazovat na její řecký původ.\footnote{Písmeno delta označuje číslo 4, a slovo kylix pochází z řečtiny a označuje číši na pití vína.} Nicméně tento nápisy mohl vzniknout daleko dříve, než se nádoba dostala do thráckého prostředí, a její přítomnost dokládá pouze obchodní či diplomatické kontakty a výměnu zboží mezi thráckým vnitrozemím a řeckým pobřežím.

