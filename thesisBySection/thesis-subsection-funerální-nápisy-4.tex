
\subsection[funerální-nápisy-4]{Funerální nápisy}

Celkem se z této doby dochovalo 13 funerálních nápisů patřící do kategorie primárních funerálních nápisů na stélách. Těchto 13 nápisů pochází výhradně z řeckých měst na mořském pobřeží a vyskytují se na nich pouze řecká jména, podobně jako v předcházejících staletích.

Dva nápisy pocházejí z vnitřní architektury mohylových hrobek, které obě pravděpodobně patřily thráckým aristokratům, soudě dle bohaté pohřební výbavy a jejich umístění na území ovládaném kmenem Odrysů. Jeden nápis pochází z hrobky u moderní vesnice Alexandrovo na dolním toku řeky Hebros a druhý z hrobky u moderní vesnice Kupinovo u Veliko Turnova. Texty jsou vytesány či vyškrábány do stěn vnitřní komory hrobek. V případě hrobky z Alexandrova poukazují na majitele hrobky či zhotovitele hrobky či její výzdoby - nástěnných maleb zobrazující lov divokého kance, do nichž je právě vyryto zmíněné {\em graffito} {\em SEG} 54:628 (Kitov 2004, 45-46). Tento malíř je znám ještě z nápisu z hrobky v Kazanlaku, která bývá datována do 3. st. n. l. (Sharankov 2005, 29-35). Druhý nápis 46:852 se nachází na říčním kameni vestavěném do vnitřní konstrukce hrobky a nese nápis σκιάς nejasného významu. Z výše uvedeného plyne, že použití písma v kontextu thrácké aristokracie je identické s nápisy z 5. a 4. st. př. n. l. a nedochází k zásadním proměnám kulturních zvyklostí.

