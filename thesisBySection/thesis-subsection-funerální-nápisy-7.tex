
\subsection[funerální-nápisy-7]{Funerální nápisy}

Ze 2. st. př. n. l. pochází 80 funerálních nápisů, jejichž primární funkcí bylo sloužit jako náhrobní kámen a označovat hrob. Oproti předcházejícím obdobím se nedochovaly nápisy na předmětech osobní potřeby, které se staly součástí pohřební výbavy. Tento fakt může poukazovat na stav prozkoumání kulturních vrstev z této doby, či na upadající vliv thráckých aristokratů, a tudíž i na pokles epigrafických aktivit spojených a jejich aktivitami.

Celkově dochází k poklesu počtů funerálních nápisů napříč celou Thrákií, s výjimkou řeckého Byzantia, kde dochází k poměrně markantnímu nárůstu. Nápisy z pobřežních oblastí nesou ze 73 \letterpercent{} jména řeckého původu, nicméně zejména v oblasti Byzantia se setkáváme i s jmény thráckými, zastoupenými zhruba v 5 \letterpercent{}, a římskými, zastoupenými zhruba 7 \letterpercent{}, zbylých 15 \letterpercent{} jmen je cizího či nejistého původu.\footnote{Nápisů s pouze řeckými jmény je celkem 53, nápisy s řeckým a římským jménem jsou dva, a nápisy s řeckým a thráckým jménem jsou čtyři, všechny z Byzantia. Nápis {\em IK Byzantion} 214 jako jediný prokazatelně patří muži nesoucí thrácké jméno, jehož otec nese jméno řeckého původu: Mokaporis, syn Moscha. Thrácká jména se vyskytují celkem na šesti nápisech, z nichž pouze na jednom se setkáváme s kombinací čistě thráckých jmen. Nápis {\em IK Byzantion} 340 patřil Mokazoiré, dceři Dinea.} Necelých 30 \letterpercent{} nápisů je určeno ženám, které jsou dále identifikovány jako dcery či partnerky. Kombinace osobních jmen je možné interpretovat jako důsledek smíšených sňatků, či přejímání onomastických zvyků dané kultury. Tento jev je omezený na oblast Byzantia a nelze tedy hovořit o celospolečenském fenoménu.

Geografický původ je uváděn vždy po jednom případě zemřelých pocházejících z Galatie a Bíthýnie, Apameie, tedy z Malé Asie z oblastí sousedících s Thrákií, což poukazuje na přesuny obyvatel mezi Evropou a Asií, ač na omezené úrovni.\footnote{Nápis {\em IK Byzantion} 120 patřil Theodórovi, jehož otec pocházel z města Bíthýnion v Bíthýnii, ale sám Theodóros se cítil být občanem Byzantia, kde také byl pochován.} Nápisy jsou většinou krátké v rozsahu jednoho až tří řádků. Výjimku tvoří několik veršovaných nápisů z Maróneie a Byzantia o rozsahu až 13 řádek, které podrobně líčí životní osudy a vykonané skutky zemřelého tak, jak bývá obvyklé zejména v pozdějších dobách,\footnote{Ač jsou všechna jména na těchto delších nápisech až na jednu výjimku řeckého původu, jedná se spíše o zvyk, který je pozorovatelný v římské době.} nicméně vyskytující se epigrafické formule i nadále odpovídají funerálním nápisům tak, jak jsme je mohli vidět v řeckých komunitách v 5. - 3. st. př. n. l.\footnote{Celkem 11krát nápisy zdraví čtenáře. Místo pohřbu je v jednom případě označeno termínem {\em tafos}, jednou je použit termín {\em mnéma}, které označuje jak hrob, tak zároveň i náhrobní kámen.}

