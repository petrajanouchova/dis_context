
\subsection[funerální-nápisy-9]{Funerální nápisy}

Celkem se dochovalo 42 funerálních nápisů, z nichž všechny sloužily jako primární funerální nápisy, tedy k označení hrobu a předání informace o zemřelém jeho nejbližšímu okolí. Všechny nápisy pocházely z pobřežních oblastí, s největší koncentrací 33 funerálních nápisů pocházejících z Byzantia. Pravděpodobnosti v souvislosti s politickým a ekonomickým rozvojem se Byzantion stal již koncem 2. st. př. n. l. epigrafickým centrem Thrákie a tento trend pokračuje i v průběhu 1. st. př. n. l. a na přelomu letopočtu.

Text funerálních nápisů si udržuje i nadále tradiční podobu\footnote{Invokační formule {\em chaire} se objevuje 12krát, v jednom případě se text nápisu obrací na podsvětní božstva, Háda a Acherón. Rozsah většinou nepřesahuje tři řádky, pouze v jednom případě dosahuje maximální délky 10 řádků. Tento obsáhlejší nápis {\em IG Bulg} 1,2 344 z Mesámbrie mluví o dosažených úspěších jistého Aristóna, který udatně bojoval s nepřátelským kmenem Bessů. Bessové představovali jeden z thráckých kmenů, který se dostal do popředí zájmu pramenů zejména v 1. st. př. n. l. v souvislosti s politickým vývojem v regionu a nástupem aristokratů právě z kmene Bessů k moci.} a identita zemřelých a jejich nejbližších ukazuje na převážně řecký původ nápisů. Jedná se o společnost, v níž se potkává několik kulturních tradic, což se projevuje zejména na proměňujících se onomastických zvyklostech. Zhruba dvě třetiny jmen jsou původem řecká, nicméně 13 \letterpercent{} jmen je možné zařadit jako jména římská a 8 \letterpercent{} jako jména thrácká, zbývajících 16 \letterpercent{} představuje jména cizího či neznámého původu. V omezené míře dochází k mísení řeckých jmen se jmény římskými, což je pravděpodobně důsledek smíšených sňatků a zvýšené přítomnosti římských občanů na území Byzantia.\footnote{Příkladem může být nápis {\em IK Byzantion} 269, který představuje náhrobní kámen členů jedné rodiny, kde dochází ke smíšenému sňatku mezi mužem s řeckými jmény a ženou se jmény římského původu: Hadys, syn Poseidónia a jeho manželka Veigellia Katyla a Dionysis, syn Hadyův.} Poprvé se také začínají ve velmi malé míře objevovat řecká či thrácká jména v kombinaci s římskou nomenklaturou.\footnote{Např. Markos Antonios Dadas z nápisu {\em IK Byzantion} 198.}

