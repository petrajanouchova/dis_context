
\subsection[funerální-nápisy]{Funerální nápisy}

Funerální nápisy představují spolu s dedikačními nápisy nejčastější kategorii nápisů, a to nejen z Thrákie, ale i z celého antického světa (Bodel 2001, 30-35). V databázi je zaznamenaných funerálních 1740 nápisů, což představuje druhou nejpočetnější skupinu nápisů právě po dedikačních nápisech se 1777 exempláři. Do kategorie funerálních nápisů tak spadají nápisy vydávané pozůstalými členy rodiny, či blízkými přáteli zemřelého. Jejich primární funkcí bylo označit místo pohřbu a připomínat život zemřelé osoby v nejbližší komunitě (Sourvinou-Inwood 1996, 140-142). Funerální nápisy byly velmi často umístěny přímo nad místem pohřbu, ať už se jednalo o tzv. plochý hrob, či o vyvýšenou mohylu ve formě hliněného násypu, postaveného nad místem pohřbu a podobné hroby můžeme vidět v archaické době v Athénách (Kurtz a Boardman 1971, 68-90). U nápisů z Thrákie je jejich archeologický funerální kontext znám u 251 nápisů, z toho 146 nápisů je možné identifikovat s pohřební mohylou či jejím bezprostředním okolím, nicméně celkové počty funerálních nápisů jsou mnohonásobně vyšší, avšak bez známého kontextu nálezu.

První funerální nápisy se objevují již v 6. st. př. n. l. v prostředí řeckých kolonií na pobřeží, kde se tato tradice uchovala po celou dobu antiky. V římské době se náhrobní kameny rozšířily i do vnitrozemí, kde se nacházejí zejména v okolí měst a podél cest.\footnote{Více o vývoji funerálních nápisů v jednotlivých stoletích v kapitole 6. O rozmístění funerálních nápisů v krajině více v kapitole 7.} Obecně se soudí, že funerální nápisy na náhrobcích byly v rámci řecké a římské společnosti veřejně přístupné a každý, kdo uměl číst, se mohl dozvědět více o zemřelé osobě, ale i nejbližší komunitě (Kurtz a Boardman 1971, 86; Saller 2001, 97-107). Obsahem funerálních textů jsou nejen zmínky o nebožtíkovi, jeho původu, dosažených životních úspěších a společenské prestiži, ale podobný druh informací je možné z textu vyčíst i o nejbližších, kteří nechali nápis zhotovit. V určitých případech, a to zejména v rámci thrácké komunity, skupina funerálních nápisů mohla obsahovat i nápisy umístěné uvnitř hrobu samotného, ať už jako součást pohřební výbavy, nebo hrobové architektury. V takových to případech nápisy mohou nést výpovědní hodnotu i o průběhu pohřebního ritu samotného, jakožto o zesnulém a o komunitě, z níž pocházel.\footnote{Více o konkrétním obsahu a vývoji sdělení v rámci jednotlivých komunit v kapitole 6.}

Na funerálních nápisech se dochovaly záznamy o celkem 2294 osobách, a to v podobě osobního jména či kombinace osobních jmen, případně jmen označujících rodiče či partnery. Celkem je zaznamenáno 4678 osobních jmen, což znamená, že na jednom funerálním nápise figurovala v průměru 1,31 osoby a jednalo se tedy primárně o náhrobky patřící spíše jednotlivcům než rozvětveným rodinám. Tento poměr se proměňuje v jednotlivých stoletích, ale obecně je možné sledovat odklon od individualismu klasické a hellénistické doby směrem k uvádění většího počtu osob na náhrobcích v římské době. Tento jev je pozorovatelný napříč oblastmi římské říše, kdy bylo zvykem uvádět na nápisech i členy širší rodiny, případně přátele, a to pravděpodobně z dědických důvodů (Saller a Shaw 1984, 1445).

Na funerálních nápisech převládají mužská jména řeckého původu. V 71 \letterpercent{} případů se jedná o muže, ve 21 \letterpercent{} o ženu a zhruba 5 \letterpercent{} není možné pro svou nejednoznačnost přiřadit ani do jedné skupiny. Poměr původu osobních jmen vyznívá ve prospěch jmen řeckého původu. Řeckých jmen se dochovalo 60 \letterpercent{}, thráckých jmen 10 \letterpercent{}, římských jmen 21 \letterpercent{} a jmen nejasného původu bylo 9 \letterpercent{}. Kolektivní vyjádření identit na funerálních nápisech se dochovala u 60 textů. Převážně se jedná o uvedení původu zemřelého či člena rodiny. U 48 případů je to odkaz na město se silnou řecky mluvící populací převážně v Thrákii, či Malé Asii. Odkaz na region se objevuje pouze třikrát, na příslušnost ke konkrétnímu kmeni šestkrát, dále výraz {\em barbaros} pouze dvakrát a původ označovaný místní vesnicí pouze jednou. Na 57 nápisech se dochovaly dialektální znaky, což ve většině případů byly znaky dórského dialektu a tyto nápisy pocházely z původně dórských kolonií Mesámbria a Byzantion. Celkem 628 nápisů si uchovalo tradiční epigrafické formule spojené s funerálními nápisy, jako např. {\em mnémé}, {\em theois katachthoniois, chaire parodeita} téměř po celou dobu antiky. Pro popis místa pohřbu se používaly ustálené termíny {\em tymbos}, {\em tafos}, {\em bómos}, a {\em stélé} pro označení náhrobního kamene. V římské době se objevují sarkofágy a termíny {\em latomeion}, {\em soros.} Tyto znaky poukazují na kontinuitu jazyka užívaného pro účely funerálních nápisů, ale na proměnu v rámci ritu samotného, kdy hrobky na určitou dobu doplnily i sarkofágy, případně urny, jak dokazují i archeologické nálezy (Tomas 2016, 84-87).

