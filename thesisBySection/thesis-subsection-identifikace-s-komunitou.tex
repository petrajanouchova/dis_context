
\subsection[identifikace-s-komunitou]{Identifikace s komunitou}

V průběhu života se člověk nevyhne interakcím s mnohými skupinami lidí. Kolektivní identitu, tedy jakési povědomí o sounáležitosti s kolektivem lidí, lze charakterizovat jako vědomou reflexi vztahu jedince vůči dané komunitě (Jenkins 2008, 103, Cohen 1985, 118). Na rozdíl od jazykové příslušnosti, která je většinou daná jazykovým prostředím, do nějž se narodíme a v~němž žijeme, naše afiliace se zájmovými skupinami je věcí naší vlastní volby, a proto je daleko flexibilnější. Člověk může být členem více komunit najednou, může je opouštět, nebo naopak do nových vstupovat, podle toho k jaké skupině sám cítí přináležitost.

Ke změně kolektivní identity může docházet v případě, že podmínky původní skupiny již nevyhovují nové situaci a dotyčný se již zcela neztotožňuje se zájmy skupiny, či v případě, že nová skupina nabízí všeobecně výhodnější podmínky pro rozvoj jednotlivce (Barth 1969, 21-25). Výhody nejrůznějšího druhu, plynoucí z účasti v těchto spolcích, jsou tedy hlavními důvody, proč se lidé sdružovali a stále sdružují do komunit a proč si přisuzují určitou kolektivní identitu (Morgan 2003, 10-11; Barth 1969, 133).

Komunity hrají v životě jednotlivce různě důležitou roli. Existují skupiny, které jsou jednotlivci velice blízké a se kterými přichází do kontaktu každý den, jako např. rodina, přátelé, sousedé apod. Jejich podíl na vytváření identity je zcela zásadní, a proto se objevují na epigrafických záznamech nejčastěji. Skupiny vzdálené, často abstraktního charakteru, či skupiny, s nimiž je i jejich vlastní člen konfrontován zřídka, se na epigrafických památkách objevují v menším počtu. Do této kategorie by spadala např. politická či etnická příslušnost.

\subsubsection[kolektivní-identita-a-legitimizace-společenského-postavení]{Kolektivní identita a legitimizace společenského postavení}

Hlavním rysem kolektivní identifikace je snaha lidí zařadit se do již existující skupiny lidí. Ve snaze ospravedlnit svou pozici ve skupině, člověk přistupuje na její strukturu a hierarchii, do níž se snaží zařadit. Legitimizace vlastního postavení ve společnosti je jedním z klíčových faktorů při formování identity jedince i kolektivu, a jako taková hraje i zásadní roli na epigrafických památkách.

Nápisy měly sloužit jako trvalá připomínka na vykonané skutky a dosažené postavení zainteresovaných lidí a přirozená lidská vlastnost po uznání a zisku společenského postavení tvořila důležitou součást v podstatě každé komunity. Poměrně často se na nápisech setkáváme s prezentací dosažených životních úspěchů, jako jsou zastávané funkce v rámci státní organizace, či udělené pocty patří k nejčastějším z uváděných pozic na epigrafických monumentech, jako např. {\em búleutés}, {\em proxenos}, {\em stratégos} (Van Nijf 2015, 233-243; Heller 2015, 265-266). Vzhledem k četnosti výskytů bylo důležité uvádět i dosažené úspěchy na vojenském poli či vítězství v panhellénských závodech.\footnote{Většina z těchto úspěchů zároveň poukazovala postavení v rámci vyšších vrstev společnosti, protože například profesionálně se věnovat sportu si mohli dovolit jen aristokraté a bohatí lidé, kteří se nemuseli starat o obživu (Golden 1998, 157-169).} Lidé rádi zdůrazňovali svou pozici v rámci společenské hierarchie, a z ní vyplývající výhody, a proto se na nápisech často objevují plnoprávní občané, veteráni, či propuštění otroci.\footnote{U propuštěných otroků se většinou uváděl i způsob, jakým své nově nabyté pozice dosáhl (Zelnick-Abramovitz 2005, 263-272), kdo ho propustil a z jakého důvodu.}

Skupiny, které se ve starověku podílely na tvorbě kolektivní identity, a tudíž se i objevily na nápisech, tvořily široké spektrum uskupení, které je možné rozlišit do dvou základních směrů: komunity související s chodem a řízením politických uskupení a dále kultovní a náboženské komunity.

\subsubsection[politicko-administrativní-komunity]{Politicko-administrativní komunity}

Politické komunity sdružují lidi nějakým způsobem zapojené do chodu státního uspořádání. Proměny politické identity přímo reflektují strukturální změny tehdejší společnosti, jakožto reformy institucí, hierarchického uspořádání společnosti a rozdělení moci, jako jeden z projevů nárůstu či poklesu společenské komplexity. Stejně tak se na nápisech odráží i míra ztotožnění se se společenským uspořádáním, či naopak jeho odmítnutí.\footnote{Vyjádření politické příslušnosti, či participace na politickém životě komunity, nepřímo zaznamenává vnitřní přesvědčení mluvčího. Tím, že jedinec akceptuje společenské uspořádání, či dokonce se mnohdy podílí na chodu politické jednotky a souvisejících institucí, sám se ztotožňuje s principy a hodnotami, které tato politická autorita představuje.} Nositel se sám označuje za občana dané politické jednotky a hlásí se tak ke svým právům a povinnostem vyplývajícím z členství.\footnote{Na řeckých nápisech se nejčastěji setkáváme s uváděním politické příslušnosti ve formě identifikace s členskou základnou obyvatel, nikoliv s institucemi obce. Politickou autoritu tehdy spíše představovala komunita občanů, tedy konkrétní lidé, a nikoliv abstraktní instituce, budovy, území, aparát (Whitley 2001, 165-166).}

V řeckém prostředí se setkáváme se zařazením do struktur a institucí {\em polis} či do jednotlivých kmenových uskupení ({\em ethné}; Morgan 2003, 1). {\em Polis} však nebyla jedinou politickou organizací antického světa. Na nápisech se setkáváme se s afiliací s kmenovými státy ({\em ethné}), uskupeními několika států s centrální autoritou ({\em amfiktyoniemi}), či s menšími samosprávnými jednotkami, které existovaly nezávisle na státním zřízení, pod nějž spadaly např. jednotlivé vesnice ({\em kómai}). V prostředí římské říše je to pak uznání autority Říma, což bývá vyjadřováno nápisy vztyčovanými na počest císaře, ale i vyjmenováváním zastávaných funkcí v administrativním aparátu či armádě (Van Nijf 2015, 233-243). Vojenská identita, a zejména prestiž v rámci vojenské komunity byla velice ceněna, alespoň podle četnosti nápisů patřících členům římské armády (Derks 2009, 240).

\subsubsection[kultovní-a-náboženské-komunity]{Kultovní a náboženské komunity}

Kult a víra měly vždy podstatný vliv na formování identity, a nejinak tomu bylo i v antice. Vyjádřením příslušnosti k náboženské komunitě jedince zároveň vyjadřoval i své náboženské přesvědčení a kulturní přináležitost. Nápisy s projevy náboženské příslušnosti nepřímo reflektují rozvrstvení společnosti a přináležitost obyvatelstva do kulturního okruhu. Změny ve variabilitě náboženských projevů jsou pak jedním z nepřímých projevů nárůstu či poklesu společenské, a zejména kulturní komplexity.

Druh zvolené náboženské komunity napovídá mnoho o charakteru víry a vnitřním přesvědčení jednotlivce. Kulty místních božstev vycházejí většinou z lokální tradice, reagují na aktuální potřeby komunity a pomáhají utvářet lokální identitu a pocit sounáležitost na každodenní úrovni (Parker 2011, 25, 221). Kult {\em héroů} se stal velice důležitým sjednocujícím prvkem pro celou komunitu, která se mohla scházet u jeho svatyně. Do kultů místních božstev mohli vstupovat lidé urozeného původu, ale i lidé zcela neurození a chudí, někdy dokonce i ženy a otroci. Mýty a víra obecně osvětlovaly komunitě hierarchii vztahů, propůjčovaly legitimizaci nároků na půdu a území, vysvětlovaly její původ. V antickém světě téměř každá komunita měla svého {\em héroa}, od nějž mohla odvozovat svůj původ a nárokovat si jeho autoritu (Parker 2011, 116, 119). Příkladem mohou být tzv. {\em héroové} zakladatelé ({\em oikistai}), od nichž odvozovala svůj původ celá města svůj původ, jako např. Hagnón a Brásidás v Amfipoli (Thuc. 5.11; Malkin 1987, 228-232).

Vliv na formování identity mají jistě i velká kultovní centra, kde se scházeli lidé z nejrůznějších regionů a měst. Oproti lokálním kultům {\em héroů} měla totiž tato centra multi-komunitní charakter, kam přicházeli věřící z celého Středomoří a Pontu. Tato místa se stala centrem setkávání a kulturní výměny jednotlivých komunit a vzájemného atletického soupeření (Vlassopoulos 2013, 39-40). Účast na těchto slavnostech však byla vzhledem ke své finanční náročnosti omezena spíše na bohatší vrstvy společnosti, a nikoliv na aktivity spojené s každodenními úkony víry. Stejně tak zastávání některých specializovaných funkcí v rámci kultu, jako např. kněží, bylo pouze záležitostí elit. S vykonáváním kultovních funkcí byla spojená prestiž, a někdy i finanční nákladnost, tudíž náboženské úřady často zastávali jen členové aristokracie (Parker 2011, 51). Proto se často setkáme i s nápisy věnovanými kněžími, kteří tak jasně poukazují na své vysoké společenské postavení.

