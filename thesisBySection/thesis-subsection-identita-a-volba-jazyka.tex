
\subsection[identita-a-volba-jazyka]{Identita a volba jazyka}

Volba řečtiny jako publikačního jazyka nápisů byla do nedávné doby brána jako jeden ze znaků hellénizace thráckého obyvatelstva (Sharankov 2011, 139-141). Podobně pak volba latiny jako publikačního jazyka byla brána jako znak romanizace, která však na území Thrákie neproběhla v tak úspěšné míře, jako hellénizace, a to zejména vzhledem k celkovému většímu počtu dochovaných řecky psaných nápisů (Tomas 2016, 119-120). Musíme vzít v potaz, že místní thrácké obyvatelstvo bylo před příchodem řeckých osadníků na území Thrákii negramotné, a s největší pravděpodobností byla řecká alfabéta prvním písemný systémem, s nímž se setkalo a jemuž bylo vystaveno v dlouhodobém časovém horizontu. Převzetí písemného systému pak může v tomto případě představovat způsob komunikační strategie, a tedy snahu o nalezení společného dorozumívacího systému. Tzv. „thrácké” nápisy, tedy nápisy psané alfabétou, které dosud nebyly přesvědčivě interpretovány, mohou představovat mezistupeň vývoje, kdy Thrákové používali písmo ve velmi omezené podobě, avšak kontext použití byl specifický pro thrácké prostředí (Dana 2015, 244-245).

Hellénizace jakožto přijetí řečtiny jako jazyka písemné komunikace s sebou nese celou řadu problémů. Jedním z nich je fakt, že řečtinu není možné brát jako jednolitý jazyk, ale spíše jako komplex příbuzných dialektů, vycházejících, ze společného základu. To alespoň platí v archaické a klasické době, kdy se na nápisech objevují charakteristické prvky náležící např. dórskému či ionsko-attickému dialektu.\footnote{Podrobněji v kapitole 6 sekce věnované 6. až 4. st. př. n. l.} Pokud se na nápisech dochovaly konkrétní dialektální zvláštnosti, pak nápisy s největší pravděpodobností pocházely z řecké komunity, která tak chtěla poukázat na své vazby s mateřskou obcí, jak dosvědčují mnohé příklady z thráckého pobřeží. Celkem bylo v Thrákii zaznamenáno 97 nápisů nesoucích charakteristické znaky dórského dialektu, které pocházely převážně z řeckých kolonií založených dórskou Megarou jako je Mesámbria s 52 nápisy, Byzantion s 34 nápisy, a Sélymbria se 7 nápisy. Z dalších nedórských lokalit pocházejí pouze náhodné nálezy vždy po jednom nápise v Apollónii Pontské, Marcianopoli, Odéssu a lokalitě Karon Limen.\footnote{Lokalita Karon Limen, též známá jako Caron Limen či Portus Caria.} Většina (86,46 \letterpercent{}) nápisů psaných dórským dialektem pochází z 5. - 1. st. př. n. l., a největší část z nich pochází z 3. st. př. n. l. z černomořské Mesámbrie.\footnote{Z Mesámbrie 3. st. př. n. l. pochází dvojnásobně velký počet veřejných nápisů oproti ostatním obcím, a tyto texty jsou navíc psány dórským dialektem. Převážně se jedná o dekrety vydávané lidem Mesámbrie ({\em búlé} a {\em démos}), čili o nařízení regulující společenskou organizaci, dále o honorifikační dekrety a o seznamy občanů. Více v kapitole 6 v sekcích věnovaných 3. st. př. n. l.} Texty dialektálních nápisů si uchovávají velmi konzervativní charakter, a to nejen volbou dialektu mateřského města, ale i svou formou. Na těchto nápisech se objevuje velký počet standardních epigrafických formulí, stejně tak jako referencí tradičních institucí řeckých {\em poleis}, regionálních řeckých božstev. Co se týče osobních jmen a vyjádření identity, nápisy pocházejí převážně z čistě řeckého prostředí (68 nápisů, 70 \letterpercent{}), thrácká jména se vyskytují pouze na 4 nápisech (4,12 \letterpercent{}), a téměř vždy v kombinaci s řeckým jménem. Volba dórského dialektu na nápisech tak poukazuje na kontinuitu společensko-kulturních vazeb a norem i několik století po prvotním založení řeckých kolonií.

