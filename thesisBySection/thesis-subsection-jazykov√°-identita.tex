
\subsection[jazyková-identita]{Jazyková identita}

Nejniternější a základní součástí lidské identity je jazyk, jímž se vyjadřujeme. Volba jazyka, a s ním inherentně spojených kulturních zvyklostí vytváří most mezi interní identitou jednotlivce a strukturami společnosti okolo něj (Jenkins 2008, 143; Bourdieu 1991, 220-228). Jazyk a jeho znalost tvoří zcela jasně dané hranice komunity, která ale může být překračována v případě bilingvních i několikajazyčných mluvčí (Barth 1969, 22-25; Adams 2002, 8). Tito lidé jsou přecházením z jedné skupiny do druhé nuceni adaptovat svou identitu ve vztahu k společenským normám komunity daného jazyka. Přijetí nového jazyka a s ním často spojených společenských norem a zvyklostí, ať už v částečné či úplné podobě, představuje velký posun v rámci identity jedince. Proto se tak většinou nedělo bezdůvodně, ale v rámci adaptace na nové životní podmínky, snahy o zlepšení životní situace (Langslow 2002, 39-41) Volba epigrafického jazyka s sebou nesla jasnou zprávu o cílové skupině, které byl nápis určen, o identitě zhotovitele a často ze samotné volby jazyka vyplývá i povaha sdělení. Na příkladu Thrákie je možné vidět, že řecky psané texty určeny spíše místnímu obyvatelstvu, zatímco latinsky psané nápisy směřovaly úřednickému aparátu a státní samosprávě a členům armády středních a vyšších šarží, čemuž odpovídalo i zaměření nápisu.\footnote{Podrobněji se problematikou vztahem volby jazyka a identity v Thrákii zabývám v kapitole 5.} Jak podrobněji rozebírám v kapitole 5 v prostředí Thrákie je tedy volba řečtiny jako epigrafického jazyka spíše projevem komunikační strategie, než univerzálního vyjádření postoje mluvčích a znakem všeobecného přijetí řecké identity.

