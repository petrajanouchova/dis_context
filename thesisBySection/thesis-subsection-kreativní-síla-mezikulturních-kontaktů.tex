
\subsection[kreativní-síla-mezikulturních-kontaktů]{Kreativní síla mezikulturních kontaktů}

Dalším společným bodem postkoloniálních teoretických přístupů je zaměření na kreativní sílu mezikulturních kontaktů a vzájemné ovlivňování všech zúčastněných stran. Interakce mezi jednotlivými společnostmi jsou pojímány v kontextu dané situace a je dán větší prostor variabilitě motivů, které vedly ke setkávání kultur. Prolínání kultur je vnímáno jako dynamický proces, který má celou řadu projevů v obou zúčastněných společnostech. V průběhu posledních 20 let vzniklo několik konceptů, využívajících poznatky z přírodních věd, jejichž aplikace se stala velmi oblíbenou, až téměř módní záležitostí (VanValkenburgh 2013, 301-302). Hybridizace\footnote{Jednotný přístup k {\em hybridizaci}, tedy vytváření hybridní materiální kultury a identit, neexistuje (Card 2013, 1-2). V rámci současného diskurzu je termín {\em hybridizace} používán pro popis důsledků setkávání dvou či více kultur a je vnímán jako neustálý proces vytváření nových významů v rámci se setkávajících kultur, často probíhající na úrovni jednotlivce (Bhabha 1994, 33-39). Termín se stal velmi oblíbeným mezi archeology, protože poskytuje prostředek analýzy materiální kultury, která je mnohdy směsí mnoha kulturních vlivů, ve formě prolínání stylistických a technologických trendů. Hlavní body kritiky {\em hybridizace} spočívají v silných biologických konotacích, které termín vyvolává, dále pak nadužívání termínu bez patřičného teoretického pozadí (Silliman 2013, 493), či jeho obsahová vyprázdněnost (Card 2013, 1-2).}, kreolizace\footnote{{\em Kreolizace} má svůj původ v lingvistice, kde popisuje proces kontaktu dvou sociolingvistických skupin. Ze specifických interakcí dvou skupin mluvícími rozdílnými jazyky vznikají zcela nová mínění a forma jazyka, zahrnující nejen slovní zásobu, ale i novou strukturu (Stewart 2007, 2; Jourdan 2015, 117). Typickým produktem kreolizace je nový jazyk, který čerpá prvky z různých jazyků ({\em creole} a {\em pidgin}), avšak udržuje si zcela unikátní charakter (Jourdan 2015, 118). V rámci sociologie a antropologie je pak {\em kreolizace} vnímána jako proces vytváření nových významů kdy jedna strana má dominantní pozici, například v rámci nuceného přesídlení obyvatelstva či v diaspoře (Liebmann 2013, 40). V mnoha ohledech je {\em kreolizace} velmi podobná {\em hybridizaci}, zejména s ohledem na vytváření nových významů a forem kultury. Hlavním rozdílem je zaměření {\em hybridizace} na výsledek kulturní výměny a na projevy v dané kultuře, zatímco {\em kreolizace} klade větší důraz na samotný proces změny (Jourdan 2015, 119).}, synkretismus\footnote{{\em Synkretismus} se jako termín poprvé objevil již v antice, kde označoval seskupení krétských komunit, které se často dostávaly do vzájemného konfliktu, ale v době nebezpečí se dokázali sjednotit. (Plut. {\em Mor.} 478a-490b.) V moderním pojetí je pojem používaný pro sloučení prvků z několika náboženství v rámci jednoho náboženského systému. V rámci studia mezikulturních kontaktů se {\em synkretismus} taktéž zaměřuje zejména na změny v rámci náboženství (Drooger 2015, 881-882). V rámci moderní antropologie získal termín pejorativní význam, kdy popisoval změnu jako nechtěný důsledek kulturních interakcí, a tak došlo k jeho postupnému vymizení z odborné literatury (Liebmann 2013, 28).}, bricolage\footnote{Termín {\em bricolage} se poprvé objevil v díle Levi-Strausse, jako fenomén popisující kreativní přeměnu kulturních prvků způsobenou aktivitou jednotlivců v rámci jedné kultury (Liebmann 2013, 29). Jean Comaroff (1985) rozšířila použití {\em bricolage} i na mezikulturní vztahy a koloniální kontext. {\em Bricolage} se soustředí spíše na vztah společenských struktur a jejich vliv na vytváření nové kultury, namísto archeology oblíbené hybné síly kulturní změny (Liebmann 2013, 29-30).} a mísení kultur představují jen výčet pojmů, používaných k popsání určitého druhu kontaktu dvou kultur, kdy dochází k adopci a adaptaci určitých prvků, často za neustálého vytváření a re-formulování kultury nové. Každý z těchto zmíněných termínů, a s nimi souvisejících směrů, se zaměřuje na jiný aspekt mísení kultur, avšak mnohé mají společného: důraz na aktivní zapojení zúčastněných společností, připisování kreativní role interagujícím skupinám a zdůrazňování podílu všech zúčastněných stran při vytváření nové kultury a identity.

Zásadním bodem kritiky těchto „kreativních směrů” je nemožnost určit původ konkrétních kultur, z čehož pramení neschopnost analyzovat jejich vzájemné ovlivňování (Jourdan 2015, 119). Dle Jourdana je kultura sama o sobě neustále se proměňující systém znaků a symbolů, který reaguje na vnější podněty a vyvíjí se, a tudíž je velmi obtížné zpětně vysledovat vzájemné ovlivňování kultur.

Hlavním zastáncem kreativní role lokální komunity v rámci Středozemí je Peter Van Dommelen, který na příkladu kolonizace Sardinie dokazuje, že dlouhodobým působením a vzájemnými kontakty několika kultur docházelo k vytváření místních hybridních materiálních kultur (1998; 2005; Dommelen a Knapp 2010). Materiální kultura dle Dommelena projevuje velkou míru mísení prvků, které jsou používány v novém kontextu a jsou vědomou volbou členů místní komunity. Van Dommelen (2005, 134-138) se primárně dívá na proměny různých součástí materiální kultury, jako je například architektura, keramická produkce, ikonografie soch, související použité technologie, a na materiální projevy rituálů, jako jsou např. dedikační předměty, či svatyně. Velký prostor dává Van Dommelen i procesu formování lokálních identit a aktivnímu odporu místního obyvatelstva v rámci selektivního přijímání konceptů a strukturálních prvků jiných kultur (Van Dommelen 1998, 214-216).\footnote{Van Dommelen 2005, 117: „{\em Cultural hybridity is a concept that has been propagated particularly by Bhabha as a means to capture the “in-betweenness” of people and their actions in colonial situations and to signal that it is often a mixture of differences and similarities that relates many people to both colonial and indigenous backgrounds without equating them entirely with either} (Bhabha 1985)”.} V neposlední řadě se zaměřuje nejen na reakce původního obyvatelstva, ale i nově kriticky hodnotí role příchozí komunity v rámci nového přístupu a bez předsudků koloniálních modelů (Van Dommelen 2005, 118).

