
\subsection[krize-3.-a-4.-st.-n.-l.-a-transformace-společnosti]{Krize 3. a 4. st. n. l. a transformace společnosti}

Ve 3. st. n. l. musela římská říše čelit nájezdům nepřátelských kmenů, a nejinak tomu bylo i v Thrákii, která byla v druhé polovině století opakovaně stižena nájezdy Gótů. Za císaře Aureliána docházelo k poničení měst a infrastruktury, vyvrcholila vleklá ekonomická krize a došlo ke snížení počtu obyvatelstva. Zásluhou reforem a posílení vojenské přítomnosti na severních hranicích dochází k určitému oživení na konci 3. st. n. l., nicméně i přes tato opatření nájezdy Gótů probíhaly i v následujících staletích (Velkov 1977, 21-29; Poulter 2007, 29-36).

Koncem 3. st. n. l. docházelo za císaře Aureliána, ale zejména později za Diokleciána a Konstantina k reformám uspořádání římské říše a tyto změny se nevyhnuly ani Thrákii. Území Thrákie spadalo jednak pod nově vytvořenou diecézi {\em Thracia} a severovýchodní Thrákie s okolím města Serdica pod diecézi {\em Moesia} a od 4. st. n. l. pod diecézi {\em Dacia}.\footnote{Diecéze {\em Thracia} se dále dělila na šest provincií, z nichž každá měla své hlavní město: {\em Europa} s hl. m. Eudoxiopolí, dříve Sélymbrií, {\em Rhodope} s Ainem, {\em Haemimontus} s Hadriánopolí, {\em Moesia Inferior} s Marcianopolí, {\em Scythia Minor} s Tomidou a {\em Thracia} s Filippopolí (Velkov 1977, 61-62). Hlavou diecéze {\em Thracia} se stal pověřený {\em vicarius Thraciae} se sídlem v Konstantinopoli, který měl na starosti zejména civilní správu Thrákie.} Hlavním městem říše se v roce 330 n. l. stala Konstantinopol, bývalé Byzantion, což Thrákii posunulo z okrajové části římského impéria rovnou do jejího středu. S přesunutím hlavního města souvisí i zvýšení stavebních aktivit a udržování rozsáhlé sítě cest, které sloužily k přepravě zboží, ale zejména vojsk (Madzharov 2009, 63-65). Vojenské velení mělo propracovanou hierarchii, rozdělenou dle zeměpisného principu na jednotlivé regiony. Většina armádních jednotek byla umístěna podél Dunaje a také ve městech na pobřeží, nicméně vojenské tábory byly umístěny i ve vnitrozemí, avšak v menším počtu (Velkov 1977, 63-68).

Již v průběhu 3. a zejména na počátku 4. st. n. l. začíná veřejný život ovlivňovat rostoucí oblíbenost křesťanství. Nejvýznamnější křesťanské komunity byly ve Filippopoli, Konstantinopoli (býv. Byzantiu), Hérakleii (býv. Perinthu), Traianúpoli, Augustě Traianě a Odéssu. V Konstantinopoli bylo sídlo arcibiskupa a biskupství byla ve většině velkých městských center. Z tohoto důvodu docházelo k šíření křesťanské víry rychleji ve městech, než na thráckém venkově (Dumanov 2015, 92-94). V této době dochází také na okrajích měst ke stavbě kostelů a bazilik, zatímco na venkově se křesťanská architektura prosazuje až na počátku 6. st. n. l.

Byrokratický aparát koncem 3. st. n. l. a zejména v průběhu 4. st. n. l. narostl a stal se poměrně nákladným na udržování, což vyústilo v růst daňových povinností jednotlivých měst. V provinciích měli největší moc místodržící, kteří zpravidla sídlili v hlavním městě a obklopovali se početným úřednickým aparátem. Důležitým orgánem se staly městské sněmy ({\em concilium}, {\em koinon}) se zástupci městských samospráv, kde se projednávaly otázky regionálního charakteru bezprostředně spojené s chodem provincie. Postupem času tyto sněmy nabyly důležitosti a staly se prostředníkem mezi samosprávou jednotlivých měst a autoritou císaře. Menší osídlení, jako např. vesnice či venkovské usedlosti, administrativně a fiskálně spadala pod autoritu města, na jehož území se nacházela. V této době také dochází k přerozdělení půdy: území měst se zmenšují, a naopak se zvětšují soukromé pozemky bohatých jednotlivců, či území patřící církvi a armádě (Velkov 1977, 70-76). Tento trend omezení autority měst a nárůst moci bohatých soukromých osob, biskupů a vysokých vojenských úředníků je patrný již od 4. st. n. l. a trvá i v následujících stoletích. Vytváří se tak nová elitní vrstva, která disponuje mocí a ekonomickým kapitálem. Na této relativně malé skupině lidí nicméně leží i většinová daňová povinnost, která se s rostoucími náklady na udržení byrokratického aparátu a armády neustále zvyšovala, což vedlo k prohlubování ekonomické krize (Poulter 2007, 15-16; Dumanov 2015, 100).

Nájezdy Ostrogótů a Vizigótů na severní hranicích Thrákie pokračují i v průběhu 4. st. n. l. V roce 376 n. l. se dokonce část Vizigótů usídlila na území Thrákie, a to se svolením císaře Valenta, za podmínek že budou dodržovat římské právo a vyznávat křesťanskou víru. V té době však došlo i k příchodu dalších kmenů, což vyvolalo poměrně dramatický konflikt, v němž se místní venkovské obyvatelstvo postavilo na stranu Gótů a společně porazili římské vojsko v bitvě u Marcianopole a v roce 378 n. l. u Hadriánopole (Velkov 1977, 34-35). Tato porážka měla za následek destabilizaci regionu, přerušení dosavadních pořádků a ekonomickou krizi, způsobenou vleklými kampaněmi a drancováním země.

V polovině 5. st. n. l. došlo ještě k ničivějším nájezdům Ostrogótů a Hunů, které měly za důsledek pokračující ekonomický a společenský propad. To mělo za následek nejen omezení zemědělské a řemeslné produkce, ale především i pokles počtu obyvatel. Na přelomu 5. a 6. st. n. l. byla Thrákie napadena kmeny Bulharů, což vedlo k pokračují ekonomické a demografické krizi, zániku sídel a celkové transformaci společnosti směrem decentralizované vládě lokálních center (Dumanov 2015, 98-99). Tato nová centra byla menší, avšak silně opevněná a sloužila jako {\em refugium} pro okolní obyvatelstvo. Panovníci se snažili za velkých finančních nákladů udržet dunajské hranice, což však ještě více prohlubovalo ekonomickou krizi. Završením byly pak nájezdy Avarů v průběhu 6. st. n. l., vedoucí k rozpadu společenského pořádku a politické autority ve formě státu a k přerodu k novému uspořádání a pozdějšímu vytvoření bulharského království (Velkov 1977, 40-59).

