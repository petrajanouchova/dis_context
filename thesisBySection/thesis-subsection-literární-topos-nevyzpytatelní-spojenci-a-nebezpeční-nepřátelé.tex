
\subsection[literární-topos-nevyzpytatelní-spojenci-a-nebezpeční-nepřátelé]{Literární topos: nevyzpytatelní spojenci a nebezpeční nepřátelé}

Obraz Thráků v řecké literatuře se do značné míry vyvíjí v závislosti na současné politické situaci, avšak i přesto jsou mnohými autory chápáni jako blízcí a mocní sousedé, kteří však nedosahují stejné civilizační a kulturní úrovně jako Řekové, viz dříve. Ač se může zdát, že tento motiv se prolíná napříč celou řecky psanou literaturou, ne vždy se muselo nutně jednat o pouhý popis skutečnosti, ale spíše o literární {\em topos}, jakýsi všeobecně přijímaný názor, zvolený záměrně autorem daného díla. Tento dualismus se do velké míry přenesl i do literatury pozdějšího období a stal se tak typickým stereotypem zobrazování Thráků v literatuře, ale i v například v malířském umění a ikonografii (Tsiafakis 2000, 388-389). V moderní době se tento dualistický popis Thráků stal součástí interpretací vycházejících z hellénizačního teoretického přístupu.

\subsubsection[thrákové-a-literární-žánr-historiografie-5.-a-4.-st.-př.-n.-l.]{Thrákové a literární žánr historiografie 5. a 4. st. př. n. l.}

Hérodotos popisuje Thráky jako skupinu lidí relativně podobnou řeckému posluchači a záměrně vyzdvihuje zajímavosti a jasné odlišnosti jako je užívaný pohřební ritus, přítomnost tetování, zvláštnosti ve výchově potomků, postavení žen ve společnosti, víra v nesmrtelnost (Hdt. 5.4-6). Jedním z cílů Hérodotova vyprávění bylo posluchače pobavit a zaujmout novými a zvláštními fakty (Asheri 1990, 162). Vzhledem k tomu, že v Athénách, kde Hérodotos také působil, a i v jiných městech žilo poměrně velké množství thráckého obyvatelstva, nemusel své posluchače seznamovat se základními skutečnostmi o Thrácích, jako tomu třeba bylo v případě Aithiopů či jiných národů z vzdálenějších a exotičtějších končin (Sears 2013, 149).\footnote{Jedna z nejznámějších zmínek o thráckém obyvatelstvu v Athénách pochází z díla samotného Platóna. Platón na začátku Ústavy popisuje epizodu, kdy se Sókratés šel podívat do Peiraea na noční slavnost {\em Bendideí}, pořádanou místními Thráky (Plat. {\em Resp}. 1.327a). Bendis byla původně thrácká bohyně, která byla uvedena do oficiálního athénského pantheonu pravděpodobně v souvislosti se spojenectvím s Odrysy v r. 431 př. n. l. Její slavnosti se ve 4. st. př. n. l. každoročně odehrávaly v athénském přístavu za hojné účasti Thráků, kteří zde trvale žili jako {\em metoikové}, či jako pomocníci v domácnosti, případně otroci ve stříbrných dolech (Archibald 1999, 457-460; Janouchová 2013, 98; Sears 2013, 149-157).} Proto je Hérodotovo líčení Thráků a Thrákie značně selektivní a útržkovité, protože se u posluchačstva předpokládala předchozí znalost tématu (Asheri 1990, 162).

Thúkýdidés, jehož rodina pravděpodobně pocházela z Thrákie, je považován za věrohodný zdroj informací, avšak i on podává svůj výklad za účelem osvětlit politické dění a průběh peloponnéské války.\footnote{Sears 2013, 14: Thúkýdidés, syn Olorův, jehož rodina pocházela z Thrákie (Thuc. 4.104), se sám považoval za Athéňana (Thuc. 1.1). Sám přiznává, že jeho rodina měla právo dolovat drahé kovy v Thrákii (Thuc. 4.105), a tudíž byl obeznámen se situací v Thrákii detailněji, než většina Athéňanů.} Když Thúkýdidés mluví o obyvatelích Thrákie, používá společné pojmenování Thrákové, ale také rozeznává jednotlivé kmeny a jejich odlišný politický status, tj. autonomní kmeny vs. závislé kmeny, jejich postoj vůči Řekům a oblast, kterou obývají (Thuc. 2.96). Ve většině případů Thúkýdidés mluví specificky o aristokratech z kmene Odrysů, kteří se stali athénskými spojenci v roce 431 př. n. l. (2.29, 2.95-97). Ve svém líčení zdůrazňuje svrchovanost Odrysů nad ostatními thráckými kmeny, velikost a důležitost oblasti, které vládnou. Thúkýdidés zde specificky používá termín říše, aby poukázal na její významnost (ἀρχή, Thuc. 2.97). O Odrysech nevyjadřuje jako o barbarech, ale poukazuje na jejich bohatství a moc, zejména aby ospravedlnil důvody vedoucí k uzavření spojenectví v roce 431 př. n. l.\footnote{V jiném kontextu se zmiňuje o blíže nespecifikovaných kmenech obývajících Rodopy, kteří byli podle něj nejbojovnější ze svobodných thráckých kmenů, nespadali pod odryskou říši (Thuc. 2.98.4), avšak přišli Odrysům na pomoc v nouzi. Dodává tak vážnost odryské říši i v rámci nezávislých thráckých kmenů a poukazuje na sílu thráckého vojska.}

V případech kdy Thúkýdidés popisuje Thráky jako hrubé a bojovné, je to vždy, aby podtrhl jejich kvality coby válečníků či aby zdůraznil vážnost konkrétních historických situací. Thúkýdidés nazývá Thráky barbarskými a krvelačnými pouze ve specifických kontextech, kde thráčtí vojáci sehráli roli v rámci peloponnéské války. V jedné z těchto epizod popisuje thrácké žoldnéře z kmene Diů jako jedny z nejkrvelačnějších barbarů, avšak vztahuje tuto charakteristiku obecně na celé etnikum (Thuc. 7.29).\footnote{Jedná se o epizodu související s athénskou výpravou na Sicílii, kdy Athéňané povolali thrácké žoldnéře, aby se zúčastnili výpravy. Ti však přijeli do Athén příliš pozdě a byli posláni zpět bez jakékoliv finanční náhrady. Athénský generál Dieitrefés a thráčtí vojáci se rozhodli opatřit si slíbené peníze vypleněním boiótského Mykaléssu a povražděním jeho obyvatel a vypleněním chrámů. Tato akce byla vnímána velmi negativně, zejména kvůli krutosti počínání Thráků, ale i vzhledem k faktu, že vedení se ujal athénský generál, který Thráky využil k vlastnímu finančnímu zisku (Kallet 2002, 140-146).} Ve většině případů Thrákové vystupují v Thúkýdidově díle jako sousedé a spojenci, jejichž vojenské schopnosti by Athéňané rádi využili ve svůj prospěch, avšak divoká povaha a bojovnost Thráků v tom často brání.

Další historik Xenofón v šesté a sedmé knize {\em Anabasis} popisuje své vlastní zkušenosti s Thráky během tažení perského prince Kýra mladšího a služby pro thráckého panovníka Seutha II.\footnote{Xenofón, spolu s dalšími Řeky působil jako najatý žoldnéř ve službách Kýra, a když Kýros zemřel, Řekové se dali na ústup zpět do Evropy. Xenofón se sám cestou dostal do služeb thráckého prince Seutha, pozdějšího odryského panovníka Seutha II.} Xenofón popisuje Thráky jako skvělé válečníky a při mnoha příležitostech zdůrazňuje jejich divoký a nespoutaný charakter. Zejména se vyjadřuje o Seuthovi jako o krutém panovníkovi, který se v rámci udržení své autority neváhal uchýlit k pálení vesnic a zabíjení lidí (Xen. {\em Anab.} 7.4.1; 7.4.6). Ač s ním Xenofón uzavřel dohodu a přísahal na vzájemné přátelství, to však nemělo dlouhého trvání (Xen. {\em Anab.} 7.3.30). Konkrétně o Seuthovi Xenofón vždy hovoří jako o Thrákovi, nikoliv o příslušníkovi kmene Odrysů, ač si byl vědom i existence kmenové organizace mezi Thráky (Xen. {\em Anab.} 7.2.22). Specifika thráckých zvyklostí a společenského uspořádání zdůrazňoval zejména v případech, kdy se odlišovaly od řeckých či kdy se jejich neznalostí sám dostal do potíží (Xen. {\em Anab.} 7.3.19-21). Matthew Sears navrhuje (2013, 115), že se Xenofón takto ostře vymezoval vůči Seuthovi a ostatním Thrákům, aby vyvrátil případné domněnky o své náklonnosti vůči Thrákům a zejména jejich bohatství a zároveň aby ospravedlnil své činy v očích řeckého čtenáře. Seuthés totiž nabídl Xenofóntovi za své služby nejen peněžní odměnu, sídlo v Bisanthé na thráckém pobřeží, ale i svou vlastní dceru za ženu, což Xenofón poměrně ochotně přijal (Xen. {\em Anab.} 7.2.38). Po nějakém čase se ale přišlo na to, že Seuthés dle Xenofónta neplatil vojákům slíbené odměny, a tudíž k nim došlo k rozepři a ukončení spolupráce a přátelství (Xen. {\em Anab.} 7.7.48-54). Xenofón tak mohl, ve snaze se od Seutha distancovat, hodnotit některé Seuthovy akce negativněji, než by si bývaly zasloužily.

\subsubsection[thrákové-a-attická-komedie-a-řečnictví-5.-a-4.-st.-př.-n.-l.]{Thrákové a attická komedie a řečnictví 5. a 4. st. př. n. l.}

Attická komedie jednak poskytuje kritický obraz politických událostí, v nichž Thrákie hrála důležitou roli, a jednak přináší informace o tom, jak Athéňané obecně vnímali Thráky. Jak Matthew Sears (2013, 18) však správně podotýká, historická skutečnost je v komediích často překroucená tak, aby pobavila a lépe pasovala do děje, a není tudíž vhodné informace brát jako zcela odpovídající historické skutečnosti. Příkladem takového komentáře k aktuální politické situaci může být Aristofanova hra {\em Acharňané}, kde zmiňuje nedávno uzavřené spojenectví s Odrysy v roce 431 př. n. l. (Aristoph. {\em Ach.} 134-173).\footnote{Tato událost je známá i z jiných zdrojů (Thúkýdidés, epigrafické prameny, Janouchová 2013), a je možné ji tak považovat za věrohodný historický rámec, na němž Aristofanés založil epizodu týkající se Thrákie a Thráků.} Aristofanés v textu používá obecné pojmenování Thrákové pro popis etnika jako takového, ale rozlišuje i mezi jednotlivými thráckými kmeny. Thráky však v přímém protikladu k Athéňanům nazývá barbary a vyzdvihuje jejich negativní vlastnosti. Žoldnéři z kmene Odomantů jsou podle něj jedni z nejbojovnějších a nazývá je zhoubou pro Athény, požírající athénské bohatství.\footnote{Aristoph. Ach. 153: μαχιμώτατον Θρᾳκῶν ἔθνος.}

Poprvé se Thrákové objevují pod označením {\em barbaroi} v negativním slova smyslu až v attickém řečnictví, a to z politických důvodů (Xydopoulos 2004, 18-20). Ioannis Xydopoulos správně podotýká, že obraz Thráků jako barbarů v literatuře klasické doby velmi záležel na konkrétní politické situaci v Athénách a na politickém cíli, který si daný autor vytkl (Xydopoulos 2010, 214-221). Attické řečnictví 5. a 4. st. př. n. l. pak zejména poukazovalo na kulturně-společenskou převahu řeckých obcí a využívalo proti-thrácké a obecně proti-barbarské rétoriky pro politické účely.\footnote{Např. Ísokratés {\em Panegyrikos} 50.}

Démosthenés se vyjadřuje o Thrákii v souvislosti s tažením Filipa Makedonského do Thrákie. Varuje před nebezpečím v podobě makedonského vpádu a Thrákii používá jako odstrašující případ (8.43-45). O Thrákii hovoří o jako relativně chudé, zemědělské oblasti s obtížnými klimatickými podmínkami, a nejednotným vedením v podobě thráckých „králů” (23.8-11), kterou se i přesto Filip rozhodl získat, a je tedy pouze otázkou času, kdy se obrátí proti bohatým Athénám. Thráky popisuje většinou jako spojence Athén (18.25-27). Mluví o nich jako o barbarech, když popisuje zradu Kotya vůči Ífikratovi (23.130-132), nicméně je nepovažuje za krutější než samotné Řeky (23.169-170). Spojence z řad Thráků považuje za morálně na vyšší úrovni, než nepřátele Athén z řad Řeků, a to zejména v jednání se zajatci. Opačné názory zastává Démosthénův oponent Aischínés, který ve snaze vyvrátit Démosthenovu argumentaci zpochybňuje Thráky jako spojence Athéňanů a poukazuje na jejich slabé stránky a na jejich nespolehlivost coby spojenců, proto je jeho líčení Thráků velmi negativní (2. 81-86, 89-93).

\subsubsection[thrákové-a-pozdní-literární-zdroje]{Thrákové a pozdní literární zdroje}

Zhruba od poloviny 4. st. př. n. l. se Thrákie a Thrákové téměř vytrácí z literárních pramenů a objevují se znovu opět v 1. st. př. n. l. v řecky, tak i latinsky psaných pramenech. Prameny z římské doby (1. až 4. st. n. l.) mluví o Thrákii v rámci vývoje politické situace v římské říši či pro ilustraci pozadí významných událostí odehrávajících se právě v Thrákii či bezprostředním okolí.\footnote{Např. Tacitus 2.64-7; 3.38; 4.46-51; Suetonius {\em Aug}. 3-4; {\em Tib}. 37; Paus. 10.19.5-12; Dio. Cass. 51.23.2-27.2. Z pozdních autorů také Ammianus Marcellinus 22.8; 27.4. Pro kompletní přehled pramenů zmiňující Thrákii viz Katzarow 1949; Velkov {\em et al.} 1981.} Thrákové jsou většinou vnímáni jako odbojní poddaní římské říše, kteří se čas od času vzbouří. Thrákie je součástí římského impéria, která je nicméně po většinu času na pokraji politického zájmu. Literární autoři tedy v duchu této tradice většinou necítí potřebu vytvářet individuální pojednání o Thrákii a v mnohém vychází z předcházející literární tvorby. Zejména geografické a etnografické popisy Thrákie autorů působících v římské době do velké míry vychází z dříve píšících řeckých autorů jako je Hérodotos, Thúkýdidés, Xenofón, či Eforos.

Poměrně obsáhlé pojednání o zeměpise Thrákie a jejích obyvatelích sepsal v 1. st. n. l. Strabón (Strabo 7.3; 7.5-7.7). Strabón se věnuje obecnému popisu thráckých kmenů, zejména pak však Getů, jejich zvyků, a uvádí několik stručných výňatků z historie, vždy v souvislosti s historií řeckou. Další systematické, avšak stručné, pojednání o obyvatelích římské Thrákie poskytuje Pomponius Mela, taktéž autor 1. st. n. l. (2.2.16-33). Mela však do velké míry vycházel z Hérodota, ale i přesto se však jedná o jeden z nejucelenějších etnografických popisů římské Thrákie (Dimova 2014, 36). Dále se k Thrákii systematicky vyjadřuje v 1. st. n. l. i Plinius Starší (4. 11. 40-50), který částečně vychází právě z díla Pomponia Mely (Romer 1998, 27). Jeho deset odstavců dlouhý popis je součástí jeho monumentálního díla {\em Naturalis Historia} a, i přes malý rozsah, patří k nejinfomativnějším zdrojům o geografii a tehdejším uspořádání Thrákie. Dalším významným zdrojem je Diodóros Sicilský, který sepsal své historické dílo {\em Bibliothéké historiké} v 1. st. př. n. l., přičemž místy čerpal z dřívějších řecky píšících autorů, zejména historika Efora pro události 5. st. př. n. l.\footnote{I přesto je dílo Diodóra velmi cenným historickým pramenem, vzhledem k tomu, že mnohé události popisuje jako jediný dochovaný zdroj. Diodóros epizodicky zmiňuje historické události, kde Thrákové vystupují v rámci širšího kontextu, jako např. v době uzavření spojenectví Odrysů s Athénami (D. S. 12.50-51), Alkibiadovo spojenectví s odryskými panovníky (D. S. 13. 105), Xenofóntovo tažení skrz jihovýchodní Thrákii (D. S. 14.37), thrácké spojenectví s Thrasybúlem (D. S. 14.94), útok Thráků na Abdéru (D. S. 15.36). Jen ve fragmentech se dochoval popis makedonského tažení do Thrákie a následné vlády diadocha Lýsimacha, včetně jeho sporů s Thráky a jeho zajetí thráckým vládcem Dromichaitem (18.3; 18.14; 21.11-12).} Diodóros hovoří o Thrácích jako o spojencích - tehdy figurují jako rovnocenní partneři, kteří však mají odlišné kulturní zvyky a jsou na primitivnějším stupni vývoje společnosti (D. S. 14. 83; 15.36; 18.14), anebo jako nepřátelé a krvelační barbaři. V jeho díle se tak setkávají obě dvě tradice stereotypního popisu Thráků, jejichž užití závisí na konkrétní situaci.

Podobně jako dřívější řecké zdroje, i prameny doby římské se věnují spíše životu elit a jejich rodin, nikoliv běžných Thráků. Obyvatelstvo autoři charakterizují zcela v duchu stereotypního zobrazení jako drsné, divoké a bojovné (Pom. Mela 2.2.18; Strabo 7.3.7), a to obyvatelstvo nejen současné, ale i minulé (Arr. {\em Anab}. 1.3-6; Haynes 2011, 7-8). Thrákové jsou i v době římské vnímání jako dobří válečníci a vojáci, a to pravděpodobně vzhledem k faktu, že jich velký počet sloužil v římské armádě přímo v Thrákii či i mimo ni (Boyanov 2008; 2012, 251-269; Dana 2013, 219-269). Pojem barbaři se objevuje zejména ve spojení s vojenskými akcemi Thráků proti Římanům, ale nikoliv jako obecný pojem, který by charakterizoval všechny Thráky (Tac. {\em Ann.} 4.51).

