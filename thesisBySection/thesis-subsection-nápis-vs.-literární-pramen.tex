
\subsection[nápis-vs.-literární-pramen]{Nápis vs. literární pramen}

Nápisy stojí na pomezí literárního a archeologického pramenu a jako takové si uchovávají vlastnosti typické pro obě dvě skupiny. Nápisy a archeologické prameny jsou často chápány jako přímý pramen pocházejícím přímo od členů společnosti, která je vytvořila, zatímco literární prameny mohou pocházet od vnějších pozorovatelů (Hansen 2001, 331-343).\footnote{Přímý pramen má de Hansena stejnou formu a nese stejné informace jako v době svého vzniku, čímž se liší od nepřímých zdrojů, které představují literární prameny. Nepřímé prameny se dochovaly ve formě manuskriptů a byly v průběhu staletí již mnohokrát pozměňovány a upravovány. Hansenovo dělení na základě přímosti a nepřímosti dochování sdělené informace není dostačující. Sám uznává, že nepostihuje např. rozdíl mezi literárními a dokumentárními papyry, ale nenabízí přesvědčivou alternativu.}

Analogicky se v oblasti antropologie a lingvistiky setkáváme se dvěma pojmy, rozlišujícími zdroje na základě jejich původu a postoje autora (Pike 1954, 8-28; Cohen 2000, 5). První z nich, zdroj emický popisuje subjekt z pozice vnitřního pozorovatele, v případě komunity jejího člena, který je obeznámen s kulturou, situací a disponuje nenahraditelnými informacemi o fungování společnosti. Oproti tomu etický zdroj je mnohdy nezúčastněný literární tvůrce popisující subjekt z pozice vnějšího pozorovatele. Výhodou etického zdroje je větší nadhled a možnost srovnání s podobnými subjekty. Na druhou stranu je to přílišný odstup a neznalost prostředí, které mohou vézt k nepochopení situace a v konečném důsledku i ke značně zkreslenému svědectví. Literární prameny, zejména v případě Thrákie, naopak nabízejí pohled na komunitu z venku, často z velmi odlišné perspektivy, která mnohdy vypovídá více o kultuře autora, než o kultuře popisovaného subjektu. Nicméně i přesto literární prameny představují další cenný druh informací o starověkém světě a postihují historické události, které samotné nápisy většinou nezaznamenávají (Bodel 2001, 41-45). Nápisy jakožto emický zdroj oproti tomu obsahují cenné a detailní informace o struktuře tehdejší společnosti, jejích proměnách a vzájemném prolínání kultur, které zaznamenali přímo členové dané komunity a je možné na ně nahlížet jako na emický zdroj infromací (McLean 2002, 2). Z toho plyne, že nápisy, jakožto emický zdroj informací, zprostředkovávají přímý pohled do společnosti, která je vytvořila. Představují tak nenahraditelný zdroj informací o vnitřním uspořádání komunit, identitě jedinců, povědomí o kolektivní sounáležitosti a o dynamice společenských vztahů, s nimiž pracuji zejména v kapitole 6 a 7.

