
\subsection[onomastické-tradice]{Onomastické tradice}

Podíváme-li se na tradiční kulturně-společenský původ jmen, celková čísla ukazují, že většina jmen, která se na nápisech vyskytuje, pochází z řeckého kulturního prostředí (41,36 \letterpercent{}), následují římská jména (35,79 \letterpercent{}), a thrácká jména (13,57 \letterpercent{}). Tabulka 5.06 v Apendixu 1 poskytuje souhrnné počty jmen dle jejich původu ze všech časových období.\footnote{V kapitole 6 je rámci jednotlivých století vždy jednotlivě pojednáno o měnících se poměrech původu jmen a případných demografických změnách epigraficky aktivní populace.} Pokud obě dvě statistiky srovnáme dohromady, můžeme jasně vidět rozdíly mezi onomastickými zvyky jednotlivým kulturních prostředí. Jak je patrné z tabulky 5.07 v Apendixu 1, u řeckých jmen je to silná tradice uvádění {\em patronymik} (37,78 \letterpercent{}) a {\em paponymik} (0,61 \letterpercent{}) oproti římskému kulturnímu prostředí. V římském prostředí se užívání kombinace tří individuálních jmen pro jednotlivce ({\em tria nomina}) odráží ve vysokém počtu individuálních jmen (90,07 \letterpercent{}). Thrácké prostředí sleduje spíše řeckou onomastickou tradici, a to jak v užívání {\em patronymik} a {\em paponymik}. Z analýzy thráckých onomastických zvyků však víme, že dochází i k následování římské tradice, a to zejména v užívání kombinace několika individuálních jmen i v kombinaci se jménem římským, které se taktéž podílí na vysokém počtu římských individuálních jmen. Je tedy zřejmé, že osoby nesoucí thrácká osobní jména poměrně záhy přijaly zvyk uvádění osobního jména, doplněného jménem otce či prarodiče a tento způsob identifikace přetrval i v době římské. Naopak, v době římské byl tento zvyk doplněn zvykem novým, a to přidáním původně římská jména či několika jmen k jménu původně thráckému, za nímž i nadále následovalo jméno otce či prarodiče. Setkáváme se s kombinací několika onomastických tradic a vytvářením jedinečného systému identifikace osob thráckého původu.

