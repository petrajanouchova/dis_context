
\subsection[osobní-jméno-a-identifikace-jednotlivce]{Osobní jméno a identifikace jednotlivce}

Osobní jméno je základním způsobem identifikace jednotlivce, pomáhající zasadit člověka do širšího kontextu dané komunity. Jméno může v širším kontextu sloužit jako ukazatel míry společenských tradic a konvencí a původu.\footnote{Osobní jméno se obvykle sestávalo z vlastního jména, dále mohlo obsahovat přezdívku, zpravidla vycházející z charakteristické vlastnosti nositele, která sloužila pro lepší identifikace konkrétního člověka v rámci komunity (Morgan 2003, 208; García-Ramón 2007, 47-64).} Jména si totiž většinou lidé sami nevybírají, ale je jim přiděleno nedlouho po narození na základě společenských zvyků dané komunity. Pomocí podoby osobního jména jsme schopni rozlišit pohlaví, případně i původ a společenské zařazení dané osoby (Morpurgo-Davies 2000, 20). V průběhu života se nicméně může měnit jako reakce na důležité změny v životě nositele, jako např. přijmutí náboženství, udělení občanství, či sňatek (Horsley 1987).

V archaické a klasické době se jména většinou dědila v komunitách po dlouhé generace a nedocházelo k větším proměnám fondu jmen.\footnote{Nepsané pravidlo v řecké společnosti určovalo, že prvorozený syn obvykle dostával jméno po svém dědovi z otcovy strany, což by v praxi znamenalo střídavou linii jmen, např. Athénagorás -- Hipparchos -- Athénagorás -- Hipparchos atd. (Hornblower 2000, 135; McLean 2002, 76). Druhorozený syn naopak dostával jméno po dědovi z matčiny strany.} V této době jména naznačovala na zařazení jeho nositele do širšího etno-kulturního rámce. Současní badatelé o antické prosopografii mají k dispozici obsáhlé databáze osobních jmen, které pomáhají určit, v jakých dobách a na jakém území se daná jména vyskytovala. Je tak možné alespoň přibližně určit, zda osobní jména spadají do řeckého kulturního okruhu, a tedy byla převážně používána lidmi řeckého původu, či alespoň lidmi, kteří se za Řeky označovali (jinak se též setkáváme s pojmem lingvistický původ jména, Sartre 2007, 199-201).

Od hellénismu docházelo k většímu prolínání jmen různých lingvistických původů jakožto důsledek většího pohybu řeckých, makedonských a v neposlední řadě také římských vojsk. Geografický výskyt jmen napovídá, ve kterých oblastech se uplatňoval vliv jaké skupiny, například jako důsledek kolonizace či bohatých obchodních kontaktů, které se po čase projevily i na výběru s výskytu osobních jmen (Parissaki 2007, 267-282; Habicht 2000, 119-121; McLean 2002, 87-90).

Pro římskou dobu je charakteristická větší fluktuace obyvatelstva, a to zejména vlivem pohybů římské armády, což mělo vliv i na mísení onomastických zvyků. Osobní jména již v této době není možné zcela jasně přisuzovat jednomu etno-kulturnímu rámci, vzhledem k tomu, že docházelo k jejich častému prolínání. I přesto osobní jméno samotné, případně jeho forma, napoví mnoho o jedinci samotném, o jeho kulturním pozadí, pohlaví, ale i o historii jeho rodu a zejména o jeho identitě. Za římské doby totiž lidé často přijímali nová jména jako důsledek udělení římského občanství.\footnote{Systém tří jmen se skládal z {\em praenomen}, osobního jméno, {\em nomen} gentilicium, rodového jméno, a {\em cognomen}, specifického přídavné jméno, a případně jména otce (McLean 2002, 113-148). Tzv. systém tří jmen se rozšířil i mezi místní obyvatelstvo, které si ale většinou nechávalo svá původní jména a jen k nim přidávala jména současného císaře v případě udělení občanství či svobody, jako v případě Marka Ulpia Autolyka, jehož rodina získala občanství za císaře Trajána na nápise {\em I Aeg Thrace} 68 z Abdéry z 2. st. n. l.}

V dlouhodobém horizontu jména signalizují přístup komunit k tradičním hodnotám, jejich udržování a určitou míru konzervativismu, či naopak mohou reflektovat proměny společnosti související se zvýšeným kontaktem s jinými národnostmi a kulturami. Těmito dlouhodobými trendy se zabývám podrobně v kapitole 5 a 6 a zaměřuji se převážně na proměny tradičních onomastických zvyklostí v celé společnosti.

