
\subsection[peršané-a-vznik-dynastie-odrysů]{Peršané a vznik dynastie Odrysů}

V 6. a 5. st. př. n. l. se část Thrákie se na několik desítek let stala na součástí perské říše (Archibald 1998, 79-90; Boteva 2011, 738-749).\footnote{První zmínky o perské přítomnosti v Thrákii pocházejí z doby Dáreiova tažení proti Skýthům v roce 513 př. n. l. (Hdt. 4.83ff; 5.1-5.2). Po řecko-perských válkách byli Peršané postupně z Thrákie vytlačováni a poslední doklady o jejich přítomnosti pocházejí z území Thráckého Chersonésu z poloviny šedesátých let 5. st. př. n. l. (Plut. {\em Kim}. 14.1).} Bohužel nemáme přesné informace o tom, zda Peršané ovládli celou Thrákii a jakým způsobem řídili ovládnuté území (Zahrnt 2015, 38).\footnote{Tj. zda se jednalo o satrapii v pravém slova smyslu.} Hérodotos dokládá rozdělení ovládnutého území do několika oblastí, které byly spravovány ustanovenými místodržícími (Hdt. 7.106).

Perská přítomnost se projevila na proměňující se materiální kultuře a zvyklostech: Zofia Archibald však naznačuje, že právě vliv Peršanů mohl být hlavní příčinou objevení tzv. mincí thráckých králů na počátku 5. st. př. n. l. a zavedení nových technologických postupů při zpracování kovů (1998, 89-90). S obdobím perské nadvlády se taktéž spojuje zavádění nových technologií zpracování kovu a následné rozšíření luxusních nádob z drahých kovů. Ty se mnohdy staly součástí pohřební výbavy thrácké aristokracie v 5. a 4. st. př. n. l. a některé z nádob nesly nápisy odkazující na majitele (Archibald 1998, 85; Theodossiev 2011, 6; Loukopoulou 2008, 148).\footnote{K jednotlivým nádobám v kapitole 6 a sekcím věnovaným 5. a 4. st. př. n. l.} Peršané měli dále vliv na zvyky spojené s vytvářením nápisů a předmětů nesoucí nápisy. \footnote{Jak se dozvídáme z Hérodotova vyprávění, Peršané na území Thrákie vztyčili několik mramorových stél. Jednalo se o dvě stély s asyrským a řeckým písmem, zaznamenávající členy Dáreiovy výpravy, umístěné na evropské straně Bosporu a u pramenů řeky Teáru ve vnitrozemské Thrákii (Hdt. 4.87, 4.91). Hérodotos rovněž popisuje, že Byzantští si dvě zmíněné stély z Bosporu odvezli a sekundárně je použili v chrámě Artemidy {\em Orthósie}.} Ač se nám dochovaly starší nápisy přímo z řeckých měst na pobřeží Thrákie, tak se jedná o první reflexi epigrafické aktivity na území Thrákie a první zmínku o zvyku veřejně vystavovat nápisy tesané do kamene. Překvapivé je, že iniciátory vztyčení stél byli Peršané, a nikoliv Řekové, avšak i přesto dle slov Hérodota vztyčil Dáreios jednu ze stél psanou v řečtině, aby textu rozumělo i místní obyvatelstvo.

Po vyhnání Peršanů z oblasti se většina řeckých měst na pobřeží egejského moře připojila k athénskému námořnímu spolku, kam začala přispívat nemalou měrou. Podíl měst z thrácké oblasti činil zhruba jednu čtvrtinu všech příjmů spolku (Meritt, Wade-Gery a McGregor 1950, 52-57). Vzhledem ke strategické pozici a bohatství zdrojů nerostných surovin se Athény o oblast severní Egejdy začaly zajímat v prvních letech po vyhnání Peršanů z oblasti. Ekonomické a politické zájmy Athén v oblasti Thrákie vyústily nejen umístěním vojenských posádek na Thráckém Chersonésu, ale i opakovanou snahou založit vlastní kolonii na území egejské Thrákie. Zakládání nových kolonií v případě Brey či Amfipole narazilo na odpor místních obyvatel a bylo ztíženo probíhajícími konflikty znepřátelených řeckých obcí (IG I\high{3} 1, 46; Thuc. 1.98, 1.100, 4.102; Meiggs 1972, 158-159, 195).

Téměř současně ústupem perské nadvlády se zhruba před polovinou 5. st. př. n. l. na jihovýchodě Thrákie v okolí řek Hebros a Tonzos objevila první autonomní politická organizace, známá z řeckých historických pramenů jako odryský stát, či království.\footnote{Zofia Archibald (1998, 93) tvrdí, že se jednalo o stát se všemi jeho atributy: existovala zde společenská stratifikace, aristokracie ve svých rukách soustředila veškerou politickou moc a kontrolu nad přerozdělováním prostředků, docházelo zde k specializaci činností, a docházelo ke stavbám výstavných rezidencí a monumentálních pohřebních mohyl a v omezené míře se zde vyskytovaly i písemné památky. Dle dostupných výsledků archeologických pozemních sběrů z nedávné doby je však spíše pravděpodobné, že se jednalo o mezistupeň mezi kmenovým uspořádáním a raným státem, kde velkou roli hrála rodově uspořádaná aristokracie a specializace práce byla spíše záležitostí přítomnosti cizích umělců a specialistů, než reflexe skutečné stratifikace odryské společnosti a existence institucionálního uspořádání dlouhodobého charakteru (Sobotkova 2013, 133-142).} Odrysové si v průběhu 5. st. př. n. l. vybudovali takovou pozici, že se stali nejen spojenci Athén, ale dokonce si mohli dovolit od nich vybírat pravidelné peněžní i nepeněžní dávky ve výši až 800 talentů (Thuc. 2.29, 2.97; Archibald 1998, 145). Tyto platby mohly sloužit výměnou za ochranu řeckých měst na pobřeží i ve vnitrozemí, či jako jistá forma cla, poplatek za možnost obchodovat s thráckým vnitrozemím (Zahrnt 2015, 42).

Mezi nejznámější panovníky patří Sitalkés, za jehož vlády došlo jak k~ekonomické, tak územní expanzi odryské říše. V roce 431 př. n. l. dokonce uzavřel spojeneckou smlouvu s Athénami, kterou stvrdil významnou pozici odryského území (Thuc. 2.29; Archibald 1998, 118-120; Zahrnt 2015, 41-42). Jak víme z literárních pramenů, okolo r. 400 př. n. l. si odryský panovník Seuthés II. najal do svých služeb historika Xenofónta a další řecké žoldnéře, aby mu pomohli získat zpět území a výsadní pozici v rámci odryské aristokracie (Xen. {\em Anab.} 6-7; Zahrnt 2015, 43). Odrysové si udrželi výsadní pozici i po konci peloponnéské války a v první polovině 4. st. př. n. l., kdy i nadále vystupovali jako mocní spojenci Athén a okolních kmenů. V roce 383 př. n. l. se k moci dostal panovník Kotys I., který odryskou říši upevnil obratnou diplomatickou politikou. Pravděpodobně kvůli narůstající politické pozici byl v roce 359 př. n. l. zavražděn. Území Odrysů se rozpadlo na několik menších regionů a fakticky tak ztratilo svou výsadní pozici a stabilitu (Dem. 23.8; Archibald 1998, 218-222; Zahrnt 2015, 44-45). Cesta do Thrákie se tak uvolnila Athénám, ale i nejbližšímu rivalovi a sousedovi, Makedonii.

