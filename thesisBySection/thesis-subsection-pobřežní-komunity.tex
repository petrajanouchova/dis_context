
\subsection[pobřežní-komunity]{Pobřežní komunity}

V období od 7. do 1. st. př. n. l. se epigrafická produkce soustředila zejména v bezprostředním okolí řeckých měst na pobřeží Egejského, Černého a Marmarského moře a v menší míře v okolí řecky mluvících komunit ve vnitrozemí Thrákie v okolí řek Strýmón, Hebros a Tonzos. Většina nápisů z okolí těchto komunit byla soukromého charakteru a sloužila k označení místa pohřbu, v případě funerálních nápisů, a jako součást věnování božstvu v případě dedikačních nápisů. Charakter soukromých nápisů, dochovaná osobní jména a vyjádření identity poukazují na téměř výlučně řecký původ obyvatelstva, které se podílelo na epigrafické produkci. Thrácká osobní jména se na nápisech z pobřeží vyskytovala pouze ve 2 \letterpercent{}, což může poukazovat na velmi malou míru interakce a zapojování thrácké populace do chodu řeckých komunit, či se může jednat o výsledek nezájmu thráckého obyvatelstva podílet se na zvyklostech zhotovovat nápisy soukromého charakteru.

Veřejné nápisy z řecky mluvících komunit dokládají, že řecká města na pobřeží Thrákie byla organizována velmi podobně jako ve zbytku tehdejšího řeckého světa a zvyk vydávat veřejná nařízení a ustanovení byl v těchto komunitách plně rozvinutý. Veřejné nápisy taktéž dokazují existenci samosprávních institucí, stratifikaci společenských rolí a rozdělení moci, stejně tak fungující systém procedur a ustálených norem chování. Regionální varianty těchto prvků však poukazují na absenci sjednocující politické autority. Zásadní roli tedy hrála regionální centra jakožto autonomní politické jednotky, které vycházely z podobného kulturního pozadí, avšak jednaly v zájmů svých občanů.

