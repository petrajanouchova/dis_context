
\subsection[postavení-řečtiny-v-římské-době]{Postavení řečtiny v římské době}

V případě římské doby konkrétní volba jazyka představovala volbu mezi tradičním jazykem, který se v oblasti používal jako jazyk epigrafické publikace již několik století, či mezi oficiálním jazykem římského impéria, jeho byrokratického a vojenského aparátu (Zgusta 1980, 135-137; Gerov 1980, 155-164). Volbou publikačního jazyka zhotovitel situoval sdělení daného nápisu do konkrétního kulturního prostředí, které mělo předem stanovená pravidla a očekávání. Volba jazyka v sobě nesla očekávání čtenářské obce, konkrétní komunity, jíž bylo sdělení určeno. Obsah nápisu naznačuje, že se jednalo spíše o vědomou volbu než o projev nekritického přijímání určitého jazyka, což v určitých podmínkách naznačují právě hellénizační či romanizační teorie (Sharankov 2011, 139-141; Tomas 2016, 119-120).

Práce, která by srovnávala řecky a latinsky psanou epigrafickou produkci v jejich kompletním rozsahu, by jistě byla velmi záslužná, nicméně natolik časově náročná vzhledem k současnému stavu znalostí, že se svou povahou hodí spíše pro výzkumný tým. Proto v současné práci vycházím z několika místních studií, které však vykazují podobné rysy. Srovnání řeckých a latinských nápisů ve vztahu k demografii a funkci nápisu v několika regionech Thrákie ukazují, že nápisy, v nichž se vyskytuje latina, pocházejí z prostředí vojenské a politické samosprávy, úzce související s aktivitami římského impéria. Řečtina naopak převládá ve venkovských oblastech a zejména ve funkci dedikačních nápisů, ale setkáme se s ní i v rámci veřejných nápisů (Gerov 1980, 158; Sharankov 2011, 145; Tomas 2007, 44-45; Janouchová 2017, v tisku).

Zhruba 1 \letterpercent{} (40) všech analyzovaných nápisů v databázi jsou nápisy obsahující řecký a latinský text. Ve více jak polovině textů (24) se jedná o identický text v řečtině a v latině. V těchto případech nelze říci, že by jeden z jazyků měl přednost: oba dva figurují na pozici prvního uvedeného textu rovnoměrně, pravděpodobně dle preferencí zhotovitele a očekávané komunity čtenářů. V případě nápisů s odlišnými latinskými a řeckými texty (16) vidíme jasné rozdělení na komunity, jimž byl text primárně určen, případně z jaké komunity pocházel zhotovitel. Nápisy určené latinsky mluvící komunitě jsou psány primárně latinsky se sekundárním textem řecky. Sekundární text je jednak standardní epigrafická formule, jako např. invokační formule vzývající božstvo, či se může jednat o podpis zhotovitele předmětu nesoucí nápis. V neposlední řadě do této kategorie patří i dedikace římskému císaři psané latinsky doplněné označením místní komunity psaným taktéž řecky. Tyto sekundární texty poukazují na společensko-kulturní pozadí, které bylo pro zhotovitele obvyklé a cítil potřebu se na něj odkázat i v rámci nápisu, který byl jinak určen primárně latinské komunitě.\footnote{Např. {\em IG Bulg} 2 749; {\em Perinthos-Herakleia} 292.} Zhotovitelé nápisů jsou tak důkazem, že bylo možné přestupovat mezi jednotlivými jazykovými komunitami a panovala mezi nimi vzájemná kulturní tolerance. Malý počet těchto nápisů v celkovém korpusu však poukazuje na fakt, že většina epigraficky aktivní populace si zvolila jeden či druhý epigrafický jazyk. Dle dostupných dat se na většině území Thrákie jednalo o řečtinu, a to i v době, kdy se území stalo součástí římského impéria.

