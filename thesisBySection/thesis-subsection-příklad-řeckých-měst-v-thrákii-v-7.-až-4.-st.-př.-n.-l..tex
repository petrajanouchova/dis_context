
\subsection[příklad-řeckých-měst-v-thrákii-v-7.-až-4.-st.-př.-n.-l.]{Příklad řeckých měst v Thrákii v 7. až 4. st. př. n. l.}

Částečné srovnání poměru epigraficky aktivních komunit na úrovni měst a komunit bez epigrafické aktivity mezi 7. a 4. st. př. n. l. umožňují data získaná z nedávno publikovaných kompendií archeologických lokalit jako Hansen {\em et al.} (2004) {\em An Inventory of Archaic and Classical Poleis}, které zaznamenává především řecká města na pobřeží a Talbert (2000) {\em Barrington Atlas of the Greek and Roman World}, zaznamenávající archeologické lokality jak na pobřeží, tak ve vnitrozemí.\footnote{Hansen {\em et al.} (2004) se primárně zaměřuje jen na řecká města archaického a klasického období. Celkové počty řeckých kolonií udávaných kolektivem autorů nereprezentují celkový počet všech existujících lokalit v daném století, a to již z podstaty historiografických a archeologických dat, které obsahují velkou míru nejistoty a nekompletnosti. Nicméně i přesto se jedná o jeden ze současných nejucelenějších souborů informací, který lze použít pro celkové srovnání. Jako další kontrolní soubor používám lokality z {\em Barrington Atlas of Greek and Roman World} (Talbert 2000), který je však obecně považován za méně přesný než Hansen {\em et al.} (2004). Talbert (2000) uvádí jak řecká osídlení, tak i osídlení thráckého původu, avšak nerozlišuje mezi jednotlivými stoletími, ale lokality zasazuje dohromady v rámci období, a proto se celkové číslo jeví jako dvojnásobně vyšší v případě 5. a 4. st. př. n. l., tedy klasického období.} Počet lokalit z těchto dvou zdrojů je nicméně vhodné chápat nikoliv jako kompletní výčet, ale spíše měřítko aktivit v dané oblasti a indikátor objevujících se trendů.

Tabulka 7.01 v Apendixu 1 poskytuje přehled počtu lokalit udávaných Hansenem {\em et al.} (2004) a Talbertem (2000) v jednotlivých stoletích a srovnává je s celkovým počtem epigrafických lokalit. Odhlédneme-li od problematiky konečných čísel lokalit a zaměříme se spíše na obecné trendy, je možné sledovat, že a) narůstající počet sídlišť odpovídá i narůstajícímu počtu epigrafických lokalit, b) sídlišť bez epigrafické produkce je několikanásobně více než sídlišť s dochovanou epigrafickou produkcí, a to zejména v 7. a 6. st. př. n. l. a v kontextu vnitrozemských lokalit. Z tohoto srovnání plyne, že epigraficky aktivní byla v nejlepším případě jen třetina všech známých sídlišť a zhruba dvě třetiny měst řeckého původu. Oproti původnímu očekávání nelze zvyk publikovat nápisy automaticky považovat za jednotnou charakteristiku všech řeckých osídlení na území Thrákie, ale i v rámci řeckých {\em poleis} se setkáváme s odlišným přístupem k epigrafické produkci, který se měnil v průběhu staletí.

Jednou z hlavních myšlenek hellénizačního přístupu v tradičním slova smyslu je, že se hellénizace společnosti projevuje jako uniformní ve všech řecky mluvících komunitách a jejich okolí. Epigrafická produkce je tradičně považována za jeden z projevů „hellénství” a je vnímána jako nedílný projev společensko-kulturního uspořádání řecky mluvících komunit. Jak je patrné z výše uvedeného, epigrafická produkce se v případě řeckých komunit na území Thrákie objevovala zhruba ve dvou třetinách komunit existujících v období od 7. do 4. st př. n. l., a proto nelze tyto dva fenomény spojovat bez dalšího vysvětlení a zařazení místních specifických podmínek v jednotlivých obcích.\footnote{Podrobněji se vybranými regiony se zabývám v rámci sekcí věnovaným jednotlivým stoletím v kapitole 6.} Pro období 7. až 4. st. př. n. l. tak epigrafická produkce figuruje spíše jako doplňující zdroj informací o tehdejší společnosti a je nutné přihlížet k archeologickým poznatkům, případně k historiografickým pramenům.

I přesto, že nápisy představují značně selektivní zdroj informací, v celkovém pohledu nabízí jedinečné srovnání jednotlivých regionů, a to i v období od 7. do 4. st. př. n. l. (Woolf 1998, 80-82; Bodel 2001, 38-39). Regionální rozdíly v epigrafické produkci mohou značit jednak výrazně jiný přístup komunity k epigrafické kultuře, související se situací v mateřském městě, nestabilní politickou situaci, nedostatek materiálu, či pouze reflektují náhodnost dochování nápisů. Nápisy se zpravidla nedochovaly z prvních let po založení kolonie, ale pochází z následujících let, ne-li desetiletí, kdy došlo ke stabilizaci politické situace, zajištění základních potřeb a bezpečí obyvatelstva. Stabilizace podmínek, za nichž mohly vznikat nápisy ve větší míře, tak přímo může souviset s nárůstem společenské komplexity a míry společenské organizace. Pokud totiž dojde k upevnění politické moci, dochází i zpravidla k ustálení produkce a zajištění nutné infrastruktury (Tainter 1988, 106-118).\footnote{O roli institucí, byrokratického aparátu na rozšíření nápisů podrobněji hovořím v kapitole 3.} V období prosperity je tak možné sledovat nárůst celkového počtu soukromých nápisů, a naopak v době nestability narůstají ve snaze regulovat tehdejší společnost celkové počty nápisů veřejných.

Tabulka 7.02 v Apendixu 1 dokazuje, že 5. a 4. st. př. n. l. představovalo dle narůstajících počtů archeologických i epigrafických lokalit a celkových počtů epigrafické produkce dobu relativní stability, zatímco ve 3. st. př. n. l. došlo k poklesu celkové epigrafické produkce, markantnímu nárůstu veřejných nápisů na třetinu celkového počtu a poklesu soukromých nápisů. Stejně tak došlo i k poklesu epigrafických lokalit, což nasvědčuje na destabilizaci situace v regionu v průběhu 3. st. př. n. l. Z historických zdrojů víme, že 3. st. př. n. l. byla Thrákie svědkem několika vojenských tažení hellénistických panovníků, invaze Keltů ze střední Evropy, která s sebou nesla i destrukce archeologických lokalit a vedla k přeměně společenského uspořádání Thrákie. Ve 2. st. př. n. l. naopak dochází k určití stabilizaci poměrů, což se projevuje i na zvýšené epigrafické produkci, nárůstu počtu lokalit a nárůstu soukromých nápisů. 1. st. př. n. l. naopak zaznamenává propad epigrafické produkce, pokles počtu epigrafických lokalit a soukromých nápisů téměř na poloviční hodnoty, zatímco veřejné nápisy se udržují na podobné úrovni jako ve 3. st. př. n. l. a představují zhruba čtvrtinu všech nápisů. V Thrákii v této době dochází k přeměně společenské struktury směrem k vazalskému království v područí Říma, což s sebou neslo míru nejistoty, která se odrazila i na epigrafické produkci.

