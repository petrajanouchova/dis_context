
\subsection[shrnutí-11]{Shrnutí}

V průběhu 2. st. př. n. l. dochází k prvním projevům mísení řeckého a římského kulturního prostředí, avšak se znatelnou převahou řeckého elementu. Thrákové se do produkce soukromých nápisů zapojují minimálně, a to pouze v regionu Byzantia a na úrovni běžného obyvatelstva, nikoliv thrácké aristokracie, jak bylo zvykem v předcházejících stoletích. Obsah a forma veřejných nápisů naznačují objevení nových institucí, intenzifikaci diplomatických kontaktů mezi jednotlivými politickými autoritami a jejich kodifikaci v epigrafické produkci. Thráčtí králové jsou na veřejných nápisech pocházejících z řeckého kontextu vnímáni jako rovnocenní spojenci, nikoliv jako primitivní barbaři.

