
\subsection[shrnutí-12]{Shrnutí}

Nápisy 2. až 1. st. př. n. l. dokumentují omezení publikačních aktivit thrácké aristokracie pro potřeby udržení společenského postavení v rámci komunity. Do regionu poprvé vstupuje se vší silou Řím, což se projevuje jak na obsahu a formě veřejných nápisů, narůstající variabilitě náboženských systémů, ale i na proměňující se skladbě osobních jmen. Římané jsou členy kultů a jsou na území Thrákie pohřbíváni, zejména pak v okolí Byzantia. Nedochází však k mísení onomastických tradic, ale Římané a Řekové si i nadále udržují kulturní odstup.

