
\subsection[shrnutí-15]{Shrnutí}

Z dostupných nápisů je patrné, že připojení Thrákie k Římu s sebou neslo celou řadu změn. V reakci na zapojení thráckého obyvatelstva do římské armády se proměnily onomastické zvyky vojáků a veteránů. Epigrafická produkce se opět navrací do vnitrozemí, nicméně s příchodem autority římského císaře se proměňují projevy thrácké aristokracie. Projevy zvyšujícího vlivu římské administrativy jsou patrné zejména v druhé polovině 1. st. n. l., kdy dochází k budování provinciální infrastruktury a spolu s tím i k nárůstu počtu veřejných nápisů. Celkové počty dochovaných nápisů jsou však poměrně nízké, z čehož se dá usuzovat, že nárůst vlivu Říma byl postupný a reformy byly spíše dlouhodobého charakteru.

