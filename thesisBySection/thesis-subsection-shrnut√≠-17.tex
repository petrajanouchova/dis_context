
\subsection[shrnutí-17]{Shrnutí}

Na nápisech z 2. st. n. l. lze velmi dobře pozorovat rostoucí vliv Říma nejen na celkovou epigrafickou produkci, ale i na proměňující se strukturu společnosti. Zvyšující se epigrafická produkce souvisí s narůstající společenskou komplexitou, která se mimo jiné projevuje narůstající administrativou a institucionální zátěží spojenou s vedením velké říše. To se projevuje jak větším počtem veřejných nápisů regulujících společenské uspořádání římské provincie, ale i výsadní pozicí císaře, coby svrchované politické autority. Rostoucí role vojska je patrná jak na obsahu nápisů, často publikovaných vojáky či veterány, proměňujícími se zvyklostmi a přejímání nových vzorců chování, ale i samotným rozmístěním epigraficky aktivních komunit. Patrná je koncentrace epigrafické produkce v okolí městských center, podél vojenských cest a v okolí vojenských a polovojenských osídlení.

