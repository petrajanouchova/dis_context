
\subsection[shrnutí-18]{Shrnutí}

Ač epigraficky aktivní populace zůstává převážně řecká, dochází k nárůstu římského elementu, ale zejména i k sebeuvědomění a zapojení thráckého obyvatelstva. Thrákové se objevují zejména v souvislosti s vojenskou službou a nápisy využívají jako formu prezentace společenského statutu. Thrácký prvek se objevuje zejména v místních svatyních, které jsou ale přístupné všem obyvatelům. Nadále dochází i k upevňování zvyklostí a jejich epigrafických projevů, které se poprvé objevily s římskou přítomností, jako např. rozšíření funerálních sarkofágů či uvádění věku zemřelého. Veřejné nápisy pak poukazují na míru regionalismu a autonomie jednotlivých městských samospráv, avšak zaštítěného všudypřítomnou autoritou římského císaře a provinciálních institucí.

