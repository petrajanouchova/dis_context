
\subsection[shrnutí-19]{Shrnutí}

Dochované materiály ze 3. st. n. l. poukazují na výsadní roli římské administrativy a vojenské organizace, což se projevovalo i v rámci epigrafické produkce. Epigrafické záznamy dokládají nárůst stavebních aktivit a aktivit spojených s udržováním již existující infrastruktury, jako vojenských cest či táborů, ale i intenzivní rozvoj městských center, zejména ze začátku 3. st. n. l. Narůstají komplexita společnosti se projevuje i ve zvyšujícím se množství specializovaných funkcí, které se na nápisech objevují. Většina z nich je do větší či menší míry spojena s administrativním řízením provincie či s vojenskou službou.

Složení epigraficky aktivní společnosti je podobné jako ve 2. st. př. n. l., ale dochází k nárůstu výskytu osob thráckého původu na nápisech, zejména na nápisech soukromé povahy. Thrákové již zcela běžně slouží v římské armádě, kde zastávají nižší a střední posty. V rámci civilní samosprávy se podílejí na vedení provincie, a to dokonce i v roli vyšších úředníků. Přítomnost Thráků je ve zvýšené míře zaznamenána u nápisů pocházejících z lokálních svatyní ve vnitrozemí, nicméně i v těchto kultech nepřesahuje zastoupení osob s thráckými jmény jednu pětinu dochovaných jmen. Z toho plyne, že i místní kulty byly otevřeny i osobám jiného původu a nejednalo se o kulty přístupné výlučně Thrákům.

Ve 3. st. n. l. podobně jako v předcházejícím století spíše, než původ hraje roli status a postavení v rámci komunity: velký důraz je kladen na dosažené postavení, zastávané funkce, získané pozice v armádě, tak i na afiliaci s římskou říší v podobě proměněných onomastických zvyků, které poukazují na římské občanství a dosažený status nositele.

