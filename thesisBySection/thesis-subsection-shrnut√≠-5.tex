
\subsection[shrnutí-5]{Shrnutí}

Epigrafické památky z 5. st. př. n. l. pocházející z řeckých měst na egejském a černomořském pobřeží ukazují určitou míru konzervativismu vůči thráckým obyvatelům, alespoň dle obsahu dochovaných nápisů. Zcela odlišné pojetí funkce písma mezi řeckou a thráckou komunitou však nasvědčuje, že kontakty v 5. st. př. n. l. zůstávaly spíše na obchodní úrovni, a přenos kulturních zvyklostí byl i nadále minimální a omezoval se především na oblasti bezprostředně sousedící s řeckými městy na pobřeží. Nápisy v thráckém kontextu figurovaly pouze v elitních kruzích a jejich užití nasvědčuje o využití ryze pro soukromé účely prominentních jedinců. Oproti tomu v řeckém kontextu je epigrafická produkce rozšířena do větší části populace, nápisy jsou součástí života komunity a stejně tak tomu odpovídá i jejich obsah.

K vzájemným kontaktům Řeků a Thráků docházelo dle epigrafického materiálu ve velmi omezené míře v okolí Apollónie, která se nacházela v sousedství území thráckých kmenů a pravděpodobně sloužila jako středisko obchodní výměny mezi řeckým světem a thráckým obyvatelstvem. Není tedy vůbec překvapivé, že se thrácké obyvatelstvo objevuje i v onomastických záznamech pocházejících z Apollónie, což může dokazovat nově vznikající příbuzenské vztahy, či mísení onomastických tradicí obou komunit. Jedná se nicméně o první epigraficky postihnuté kontakty Thráků a Řeků na nearistokratické úrovni.

