
\subsection[shrnutí-7]{Shrnutí}

Z uvedeného je patrné, že nápisy soukromého charakteru ve 4. st. př. n. l. sledovaly podobné trendy jako nápisy v 5. st. př. n. l., a není zde patrný žádný společensko-kulturní předěl. Většina epigrafické produkce se koncentrovala na území řeckých komunit podél mořského pobřeží, nicméně se objevují nápisy i ve vnitrozemí v prostředí thrácké aristokracie.

Thrácká komunita využívala nápisy převážně utilitárně pro označení vlastníka, autora či v souvislosti s funkcí předmětu. Tyto nápisy byly zhotoveny pro velmi úzkou skupinu a jejich vlastnictví či vlastnictví předmětu nesoucí nápis poukazovalo na vysoký společenský status majitele. Jejich výsadní postavení v rámci společnosti nasvědčovalo i jejich uložení v hrobce jako součást pohřební výbavy. Oproti tomu nápisy pocházející z řeckých komunit poukazují na uchovávání kulturních tradic a zvyklostí a pouze omezený kontakt s thráckým obyvatelstvem. Převažují veřejně vystavené funerální nápisy, napomáhající vytváření povědomí o jednotné komunitě a kolektivní paměti.

V případě veřejných nápisů se setkáváme s mírným nárůstem jejich celkového počtu. Jednou z příčin rozšíření epigrafické produkce i mimo soukromou sféru může být jak nárůst společenské komplexity, jak v prostředí řeckých komunit, tak i nárůst politické moci kmene Odrysů. Veřejné nápisy v thráckém prostředí slouží jako prostředek komunikační strategie vůči Řekům, a pravděpodobně neslouží pro komunikaci uvnitř thrácké komunity.

