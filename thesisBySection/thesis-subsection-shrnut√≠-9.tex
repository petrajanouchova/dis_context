
\subsection[shrnutí-9]{Shrnutí}

Ve 3. st. př. n. l. i nadále většina epigrafické produkce pochází z řeckých měst na pobřeží a vykazuje všechny charakteristické rysy typické pro řecký svět té doby. Zvýšení celkového počtu veřejných nápisů z těchto komunit může nasvědčovat na zavádění nových procedur a institucí do měst, či alespoň větší míru využívání nápisů pro účely politické organizace a vedení administrativy. Naopak snížení počtu soukromých nápisů může poukazovat na jistou míru nejistoty, které museli obyvatelé čelit, což mohlo souviset s invazí keltských kmenů na počátku 3. st. př. n. l. Na poměrně bouřlivý charakter této doby poukazují i destrukční vrstvy z četných archeologických nalezišť, mimo jiné i z Pistiru.

Ve vnitrozemí se nápisy objevují pouze v kontextu aristokratických kruhů a slouží zejména k upevnění společenské pozice v rámci komunity, ale v ojedinělých případech slouží i jako prostředek komunikace s řeckými či makedonskými partnery. Výjimečnou roli zaujímají multikulturní komunity v Seuthopoli a Kabylé, které vycházejí z thráckých kořenů, avšak v určité míře přijímají i prvky tradičně označované jako řecké či makedonské. Dalším potenciálním místem kontaktu kultur je Hérakleia Sintská, původně makedonská vojensko-obchodní stanice u středního toku řeky Strýmónu. Epigrafické důkazy však i nadále poukazují spíše na uzavřený charakter komunit a přetrvávání tradičních kulturních hodnot v jejich rámci.

