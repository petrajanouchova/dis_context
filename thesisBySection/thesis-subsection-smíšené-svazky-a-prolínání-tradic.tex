
\subsection[smíšené-svazky-a-prolínání-tradic]{Smíšené svazky a prolínání tradic}

Údaje o smíšených svazcích mezi Řeky a Thráky se na nápisech dochovaly celkem ve 43 případech, a to v rámci identifikace jednotlivce ve formě jména primární osoby a jména partnera. Nejčastější formou jsou svazky mezi mužem z řeckého prostředí a ženou z thráckého prostředí (33 výskytů, 76,7 \letterpercent{}). Celkem je možné rozlišit šest variant v závislosti na původu jména a pohlaví osoby vystupující na nápisech:

\startitemize[A][stopper=.]
\item
  \startblockquote
  muž se jménem řeckého původu, partnerka se jménem thráckého původu (19 výskytů)
  \stopblockquote
\item
  \startblockquote
  žena se jménem řeckého původu, partner se jménem thráckého původu (4 výskyty)
  \stopblockquote
\item
  \startblockquote
  muž se jménem thráckého původu, partnerka se jménem řeckého původu (4 výskyty)
  \stopblockquote
\item
  \startblockquote
  žena se jménem thráckého původu, partner se jménem řeckého původu (13 výskytů)
  \stopblockquote
\item
  \startblockquote
  muž nesoucí jméno thráckého a římského původu, partnerka se jménem řeckého původu (2 výskyty)
  \stopblockquote
\item
  \startblockquote
  žena nesoucí jméno thráckého a římského původu, partner se jménem řeckého původu (1 výskyt).
  \stopblockquote
\stopitemize

Smíšené svazky se vyskytují zejména na nápisech pocházejících z regionu městských center a z oblastí, kde se dá předpokládat zvýšená míra kontaktů mezi thráckou a řeckou, případně římskou populací, tj. v okolí {\em emporií} či vojenských lokalit. Největší koncentrace nápisů se smíšenými svazky pochází z Odéssu s 23 nápisy, převážně z římské doby. Dále se nápisy se smíšenými svazky v době římské objevovaly v údolí středního toku řeky Strýmónu (10), a v menší míře ve městech Byzantion (3), Nicopolis ad Nestum (2), Pautália (2), Augusta Traiana (2), Filippopolis (1) a Seuthopolis (1). Nápisy jsou většinou římské, až na tři hellénistické nápisy, které pocházejí ve dvou exemplářích z Odéssu a jeden ze Seuthopole.\footnote{Právě hellénistický nápis {\em IG Bulg} 3,2 1731 ze Seuthopole dokumentuje aristokratický svazek thráckého krále Seutha III. a pravděpodobně řecké či makedonské Bereníké a jedná se o první zmínku o existenci smíšených svazků thrácké aristokracie (Tacheva 2000, 30-35; Calder 1996, 172-173). Ostatní nápisy dokumentují svazky mimo aristokratické kruhy, nakolik je možné tak soudit z mnohdy lakonického textu.} Údaje o smíšených svazcích pochází převážně z funerálních nápisů, u nichž se zejména v římské době předpokládá uvádění rodinných příslušníků a vzájemných vztahů, mimo jiné i z dědických důvodů (Saller a Shaw 1984, 145; Meyer 1990, 95).

