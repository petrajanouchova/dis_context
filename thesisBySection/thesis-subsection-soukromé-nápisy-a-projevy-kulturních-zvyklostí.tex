
\subsection[soukromé-nápisy-a-projevy-kulturních-zvyklostí]{Soukromé nápisy a projevy kulturních zvyklostí}

Rozmístění soukromých nápisů do velké míry sleduje podobné trendy jako veřejné nápisy: větší část nápisů se nachází ve vzdálenosti na úrovni jednodenního pochodu od měst, případně v lokalitách podél cest, avšak ve srovnání s veřejnými nápisy je to až o 15 \letterpercent{} nápisů méně. Ve vzdálenosti do 20 km od měst se totiž nachází 64 \letterpercent{} nápisů, zbývajících 36 \letterpercent{} je ve vzdálenosti větší než 20 km. Pokud bereme v potaz vzdálenost do 40 km, pak se v tomto rozsahu nachází 81 \letterpercent{} z 3440 soukromých nápisů a zbylých 19 \letterpercent{} je vzdáleno od měst více než 40 km.

Soukromé nápisy se objevují v úzkém pobřežním pásu do vzdálenosti zhruba 20 km na pobřeží Egejského a Marmarského moře, jak je patrné z mapy 7.09 v Apendixu 2. Na pobřeží Černého moře nápisy plynule přechází z pobřeží do vnitrozemí, což může být vysvětleno absencí pohoří, které by bránilo jejich rozšíření podobně jako v případě egejské oblasti. Ve vnitrozemí se nápisy nachází zejména v okolí {\em Via Diagonalis} a na ní ležících měst Filippopolis a Serdica, nicméně zvýšené koncentrace soukromých nápisů můžeme pozorovat téměř v okolí všech měst s výjimkou Bizyé. V případě Bizyé lze však absenci nápisů vysvětlit nedostatkem publikovaného materiálu, a nikoliv negativním postojem obyvatelstva vůči zhotovování nápisů pro soukromé účely. Na rozdíl od veřejných nápisů soukromé nápisy pocházejí i z venkovských oblastí, které nesousedí s městem a ani v jejich blízkosti neprochází řádná z hlavních římských cest. Přítomnost soukromých nápisů v rurálním kontextu je tak možné vysvětlit jako důsledek pohybu obyvatelstva či adopce epigrafických zvyklostí v omezené míře i ve venkovském prostředí.

Dvě nejčastější společenské funkce, jakou soukromé nápisy zastávaly byla funkce funerální a dedikační. Rozmístění soukromých nápisů dle jejich typologie přináší zajímavé poznatky o chování tehdejšího obyvatelstva a šíření kulturních zvyklostí mezi jednotlivými komunitami.

\subsubsection[funerální-nápisy-19]{Funerální nápisy}

Rozmístění funerálních nápisů reflektuje odlišné zvyklosti, respektive odlišnou demografickou strukturu obyvatelstva pobřežní a vnitrozemské Thrákie. Jak je patrné z mapy 7.10 v Apendixu 2, v případě pobřežní Thrákie se funerální nápisy nacházejí převážně přímo ve městech či v nejbližším okolí. Ve vnitrozemí pak nalézáme funerální nápisy jak ve městech, ale i v oblastech podél cest mimo region měst, případně na venkově mimo dosah cest. Z celkem 1631 funerálních nápisů se jich 87 \letterpercent{} nachází ve vzdálenosti do 20 km od měst a 13 \letterpercent{} ve vzdálenosti větší než 20 km. Do skupiny nápisů nalezených ve vzdálenosti od měst v délce denního pochodu, tedy 40 km, spadá 97 \letterpercent{} nápisů a jen 3 \letterpercent{} byla nalezena ve vzdálenosti větší než 40 km. Z toho plyne, že většina funerálních nápisů se nacházela v okolí měst či přímo ve městech. V případě vzdálenosti nálezových míst od trasy cest bylo 80 \letterpercent{} nápisů nalezeno ve vzdálenosti do 5 km od cesty, 20 \letterpercent{} nápisů ve vzdálenosti větší než 5 km. Pokud tuto vzdálenost změníme na 10 km od cesty, počet nápisů naroste na 84 \letterpercent{} všech nápisů nalezených do vzdálenosti 10 km a 16 \letterpercent{} ve vzdálenosti nad 10 km. V případě vzdálenosti 20 km tento poměr naroste na 97 \letterpercent{} nápisů ve vzdálenosti do 20 km a 3 \letterpercent{} ve vzdálenosti nad 20 km. Z toho plyne, že většina funerálních nápisů se našla v bezprostřední blízkosti cesty.

Pokud se podíváme na umístění funerálních nápisů v krajině, celkem 93 \letterpercent{} nápisů pochází z nížin do 273 m. n. m., čemuž odpovídá i průměrná nadmořská výška nálezových míst všech funerálních nápisů 81 m. n. m.\footnote{Pro srovnání průměrná nadmořská výška nálezových míst všech dedikačních nápisů je 311 m. n. m., viz níže.} Funerální nápisy tedy byly nalézány především v dobře přístupném terénu, v blízkosti měst a podél tras hlavních cest. Funerální nápisy zpravidla pocházejí z okolí lidských sídel, tedy z míst s vyšší mírou zalidnění. Lidé byli pohřbíváni mimo centrum měst, nejčastěji v bezprostředním okolí hlavních cest vedoucích směrem z města ven. Jedná se o zcela přirozený jev umisťovat pohřebiště v blízkosti lidských sídel, a nikoliv do odlehlých horských oblastí (Kurtz a Boardman 1973, 49-51; Damyanov 2010, 270).

Z četnosti výskytu funerálních nápisů na mapě 7.11 v Apendixu 2, zcela jasně vyplývá, že zvyk zhotovovat funerální nápisy převažuje v pobřežních oblastech, konkrétně ve městech Byzantion, Apollónia Pontská, Odéssos, Perinthos, Maróneia a Mesámbria, původně řecké kolonie. Z vnitrozemských center je to původně makedonské sídlo Hérakleia Sintská a Makedonci založená Filippopolis. Hustota výskytu funerálních nápisů ve vnitrozemí má však daleko nižší hodnoty, z čehož plyne, že zvyklost vytvářet a veřejně vystavovat nápisy patří spíše do řecké kulturní okruhu a pouze částečně se rozšířila z řeckých měst na pobřeží do vnitrozemské Thrákie. Tomuto faktu odpovídá i datace funerálních nápisů. Z 6. až 1. st. př. n. l. pocházejí funerální nápisy převážně z pobřežních oblastí či z oblasti Hérakleie Sintské a lokality Didymoteichon, zatímco z 1. až 5. st. n. l. pocházejí vnitrozemské nápisy jak z pobřeží, oblasti toku řeky Strýmón, tak i z oblasti centrální Thrákie.

\subsubsection[dedikační-nápisy-19]{Dedikační nápisy}

Dedikační nápisy jsou projevem náboženského přesvědčení jednotlivců i celých skupin a jejich rozmístění reflektuje normy chování v rámci jednotlivých komunit. Pobřežní Thrákie se vyznačuje velmi nízkým počtem dedikací, zatímco ve vnitrozemí se nachází několik oblastí s vysokou koncentrací dedikačních nápisů.

Jak je patrné z mapy 7.12 v Apendixu 2 dedikační nápisy se vyskytují jak v blízkosti měst a podél cest, podobně jako funerální nápisy. Ve vzdálenosti do 20 km od měst se nachází 43 \letterpercent{} nápisů, zbylých 57 \letterpercent{} nápisů pochází ze vzdálenosti vyšší než 20 km. Do skupiny nápisů nalezených ve vzdálenosti do 40 km od měst spadá 68 \letterpercent{} nápisů, zbylých 32 \letterpercent{} bylo nalezeno ve vzdálenosti větší než 40 km od měst. V porovnání s funerálními nápisy se dedikační nápisy vyskytují v okolí měst v menší míře a zejména pocházejí z venkovských a horských oblastí, které jsou vzdáleny od hlavních městských center.\footnote{Funerální nápisy mají poměr 87:13 u vzdálenosti 20 km (do 20 km vs. nad 20 km), a u vzdálenosti 40 km se tento poměr mění na 97:3.}

Největší koncentrace dedikací pocházejí z podhůří pohoří Rodopy a Pirin a Stara Planina v centrální Thrákii a z horských oblastí severozápadní Thrákie. Průměrná nadmořská výška místa nálezu dedikačního nápisu je 311 m. n. m., ale 21 nápisů bylo nalezeno dokonce ve výškách nad 764 m. n. m.\footnote{Pro srovnání průměrná nadmořská výška nálezových míst funerálních nápisů je 81 m. n. m., z čehož plyne, že dedikační nápisy byly nalézány ve vyšších horských polohách, případně na úpatí hor.} Téměř 80 \letterpercent{} nápisů pochází z oblastí do 509 m. n. m., představuje jednak oblast nížin, ale i na úpatí hor a v údolí řeky Strýmónu. Dedikace pocházející z horských oblastí s nadmořskou výškou vyšší než 510 m. n. m. pocházejí zejména z oblasti okolo měst Serdica a Pautália s pohořími Vitoša a Stara Planina.

Dedikační nápisy a jejich výskyt v horských oblastech a podhůří poukazuje na udržení tradičního charakteru thráckého náboženství, které bylo spojeno s přírodními silami a ve vztahu s okolní krajinou (Janouchová 2013, 10). Svatyně původně thráckých božstev byly nejčastěji umístěny ve volné přírodě a tento trend pokračoval i v době římských, z níž pochází většina dochovaných dedikačních nápisů.

