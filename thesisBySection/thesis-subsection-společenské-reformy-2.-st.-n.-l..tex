
\subsection[společenské-reformy-2.-st.-n.-l.]{Společenské reformy 2. st. n. l.}

Na začátku 2. st. n. l. za vlády Trajána a Hadriána došlo k poměrně zásadním reformám provinciální administrativy v Thrákii, které měly vliv na celou společnost. Nejen, že došlo ke změně statutu provincie na tzv. pretoriánskou provincii, ale i k organizačním změnám, které vedly k posílení politické moci místní samosprávy. To vedlo ve 2. st. n. l. ke zintenzivnění stavební aktivity, zakládání nových měst, či jejich rozšiřování a výstavbě veřejné infrastruktury, jako např. lázní, akvaduktů, divadel či cest.\footnote{Jména nově vzniklých či obnovených měst dokazují šíři těchto stavebních aktivit: Augusta Traiana, Trainúpolis, Hadriánúpolis, Plotinúpolis, {\em Ulpia Nicopolis ad Istrum}, {\em Ulpia Nicopolis ad Nestum}, {\em Ulpia Marcianopolis} (Jones 1971, 18-19).} Stejně tak narostl význam městských samospráv, jimž bylo uděleno více pravomocí na úkor rodové aristokracie. V Thrákii, ale např. i v sousední Bíthýnii se setkáváme s novou společenskou skupinou městských elit, které pocházely z místního prostředí, avšak dokázaly se plně adaptovat na novou společenskou strukturu (Fernoux 2004, 415-511). V této době také vznikaly nové úřady a politická uskupení, jako např. {\em koinon tón Thraikón}, což bylo uskupení vnitrozemských měst se sídlem ve Filippopoli (Lozanov 2015, 82-83).

Sídlo místodržícího se přesunulo z Perinthu na pobřeží do vnitrozemské Filippopole. Původně řecká města na pobřeží postupně ztrácela svůj autonomní status a stávala se součástí římské říše, a to včetně povinnosti platit daně. Města razila vlastní mince s vyobrazením císaře na aversu a charakteristickým symbolem daného města na reversu. Města byla navzájem propojena sítí silnic, které primárně sloužily pro přesuny armády, na jejichž stavbě a údržbě se podílela městská samospráva do jejíhož teritoria cesta spadala (Madzharov 2009, 29-30). Od konce 2. st. n. l. vznikaly podél cest stanice, které zajišťovaly bezpečnost a zásobování. K tomuto účelu byla ve vnitrozemí stavěna {\em emporia}, jejichž hlavním úkolem bylo zprostředkování obchodu mezi zemědělsky zaměřeným venkovem a městskými centry (Lozanov 2015, 84-85).\footnote{Příkladem může být např. {\em emporion} Pizos, či Discoduraterae, o nichž podobněji pojednávám v kapitole 6.}

Od 2. st. n. l. mohlo thrácké obyvatelstvo sloužit nejen v pomocných vojenských jednotkách, ale i vstupovat do legií, jak dosvědčují epigrafické památky z celého území římské říše (Samsaris 1980, 38-39). V průběhu 2. st. n. l. také došlo k výraznějšímu přesunu obyvatelstva z oblasti Bíthýnie do vnitrozemské Thrákie, jak dosvědčují mnohé nápisy a monumenty, což mělo za následek i proměny nejen ve složení obyvatelstva ale i nárůst počtu nových kultů východní provenience (Delchev 2013, 15-19; Raycheva a Delchev 2016; Lozanov 2015, 82; Tacheva-Hitova 1983).

