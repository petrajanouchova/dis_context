
\subsection[statický-a-selektivní-obraz-epigrafické-společnosti]{Statický a selektivní obraz epigrafické společnosti}

Ač epigrafické texty představují primární zdroj informací, jedná se o zdroj značně selektivní a statický. Nápisy jako produktem společnosti daného časového a místního kontextu, který se stává minulostí již v okamžiku publikování, a neslouží jako popis antické společnosti platný pro celou dobu jejího trvání.

Texty nápisů se přímo nevyjadřují ke všem oblastem lidského života, ale drží se zavedených kategorií, jako jsou například funerální či dedikační texty, které mohou být do jisté míry stylizované a navzájem se prolínat (Bodel 2001, 46-47). Některé oblasti společenského života, např. styky s jinými kulturami, se na nápisech objevují jen výjimečně ve formě přímé zmínky.\footnote{To však neznamená, že by ke kontaktům nedocházelo, pouze se neprojevily na epigrafické produkci, či alespoň ne v podobě přímé zmínky v textu či reflexe dané skutečnosti.} Při studiu nápisů je tedy nutné respektovat charakteristické rysy jednotlivých kategorií, které se ale mnohdy navzájem prolínají, ale především se neomezovat na strohou interpretaci založenou na pouhém rozboru textu. Nápisy představují komplexní souhrn kulturních prvků, které jsou všechny projevem téže kultury. Proto je vhodné nahlížet na objekt v jeho širším kontextu, aby se předešlo ztrátě podstatných informací či jejich zkreslení.

Se ztrátou informací souvisí i míra dochování nápisů, která byla ve velké části případů dílem náhody. Dochované nápisy nápisů tak zdaleka nepředstavují všechny v minulosti vytvořené nápisy, dokonce ani jejich reprezentativní vzorek, ale jedná se o náhodný soubor, který svému dochování vděčí spíše kvalitě použitého materiálu než obsahu textu. Richard Duncan-Jones (1974, 360-362) odhaduje na souboru římských nápisů ze severní Afriky, že dnes dochované texty představují zhruba 5 \letterpercent{} původně existujících nápisů.\footnote{Pokud toto překvapivě nízké číslo aplikujeme na situaci v Thrákii s cca 4600 dochovanými nápisy, dostaneme se teoreticky až na 92000 původně existujících nápisů, což se však zdá až příliš nadhodnocené, vzhledem k povaze a množství současných nálezů. I tak zde analyzovaný soubor 4600 nápisů nepředstavuje všechny vytvořené nápisy, ale pouze jejich dochovaný zlomek.}

Předpokládá se, že populace aktivně zapojená do produkce nápisů, tj. epigraficky aktivní část společnosti představovala pouze malý výsek tehdejší společnosti. Vydávat nápisy na kameni a kovu si mohli dovolit lidé střední a vyšší třídy, vzhledem k relativně vysoké ceně nápisů, zatímco materiály jako keramika či dřevo byly dostupnější, nicméně se hůře dochovávají (McLean 2002, 13-14). Z důvodů finanční náročnosti se příliš často nesetkáváme na nápisech s povoláními vykonávanými lidmi z nižší společenské a ekonomické vrstvy společnosti. Někteří autoři se snaží vypočítat na základě odhadů o velikosti populace a počtu dochovaných nápisů, jaký byl podíl epigraficky aktivních lidí v dané oblasti (Woolf 1998, 99-100; Beshevliev 1970, 64-65). Jednotlivé poměry se liší jak v závislosti na místě a čase a na stupni dochování nápisů, nicméně je možné souhrnně říci, že epigraficky aktivní obyvatelstvo tvořilo menšinu tehdejší populace a obsah a forma dochovaných nápisů poukazuje na zapojení spíše na střední až vyšší vrstvy společnosti.

Z výše řečeného plyne, že dochované nápisy představují pouze zlomek z dřívější produkce, pocházejí pravděpodobně pouze ze středních a vyšších vrstev společnosti a jsou záznamem stavu v určitém časovém a místním kontextu. I přes tato omezení nesou celou řadu informací o proměnách tehdejší společnosti, jimiž se zabývám zejména v kapitole 6 a 7.

