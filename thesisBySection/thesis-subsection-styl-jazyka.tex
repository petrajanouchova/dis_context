
\subsection[styl-jazyka]{Styl jazyka}

Kulturní a politické proměny tehdejší společnosti významnou měrou neovlivnily jazykovou formu a styl použitého jazyka. Zdá se, že styl epigrafického jazyka zachovával poměrně konzervativní podobu a vyplýval spíše ze společenské funkce nápisu, než že by podléhal aktuálním trendům (Petzl 2012, 52-58).

Celkem 121 textů nápisů, či alespoň jejich část, byla psána metricky. Jednalo se výhradně o nápisy soukromého charakteru: přes tři čtvrtiny, celkem 94 nápisů, mělo funkci funerálních nápisů. Dedikační nápisy se vyskytovaly v 17 případech, což představovalo zhruba 14 \letterpercent{}, zbytek nápisů nebylo možné určit. Poměr metricky psaných nápisů, u nichž byla známá datace, vůči celkovému počtu datovaných nápisů mírně narůstal od 1. st. př. n. l., a to zhruba o 0,5 \letterpercent{}.\footnote{Celkový počet datovaných nápisů ze 7. až 1. století př. n. l. (975,58; normalizovaná datace, více o metodě normalizované datace v kapitole 4) představuje 27,47 \letterpercent{} všech nápisů. Celkem 19 nápisů psaných metricky (normalizovaná datace) z téhož období, představuje 1,95 \letterpercent{} ze všech datovaných nápisů. U nápisů datovaných do 1. až 8. st. n. l. se jedná celkem o 2575,09 nápisů (normalizovaná datace), což představuje 72,51 \letterpercent{} všech datovaných nápisů. U nápisů psaných metricky a datovaných do 1. až 8. st. n. l. se jedná o 62 nápisů (normalizovaná datace), což představuje 2,41 \letterpercent{} všech datovaných nápisů.} Tento nárůst je statisticky nesignifikantní, a mohl být způsoben stavem a nahodilostí dochování nápisů. Celkově je tedy možné říci, že tendence publikovat metrické nápisy zůstává konstantní po celou dobu epigrafické produkce v Thrákii a nemění se v závislosti na proměnách složení populace, či v reakci na politicko-kulturní události. Metricky psané nápisy tvoří v průměru 2,2 \letterpercent{} celkové produkce, tedy její marginální část, která tvořila velmi konzervativní a neměnnou součást epigrafické produkce.

