
\subsection[teorie-světových-systémů-a-její-návaznost-na-hellénizaci]{Teorie světových systémů a její návaznost na hellénizaci}

Jako alternativní teoretický směr, který do velké míry vycházel z podobných principů jako hellénizace, vznikla v 70. létech 20. století {\em teorie světových systémů} ({\em world-systems theory}). Teorie světových systémů se snažila vysvětlit změny ve společnosti na základě ekonomické teorie Immanuella Wallersteina (1974). Wallerstein s její pomocí vysvětloval rozdělení ekonomické síly a moci, pracovní síly, ekonomického potenciálu a pohybu zboží v moderním světě. Tato původně čistě ekonomická teorie se stala velmi oblíbenou mezi archeology, zabývajícími se mezikulturními kontakty a distribucí materiální kultury (Frankenstein a Rowlands 1978; Rowlands, Larsen a Kristiansen 1987; Bintliff 1996; Stein 1998; Champion 2005; Harding 2013).\footnote{Tato teorie se stala v archeologii též známá pod pojmem „centrum - periferie” ({\em core-periphery}) a sloužila primárně jako vysvětlení pohybu materiální kultury a dynamiky vzájemných kontaktů společností.}

Základním předpokladem teorie světových systémů je propojenost regionálního a nadregionálního ekonomického systému, jehož jednotlivé části se ovlivňují mezi sebou. Světové ekonomické systémy se skládají z center, periferií a semi-perifierií, přičemž v centrech se koncentruje většina ekonomické síly a politické moci. Tato moc je zajištěna existencí vojska, které zajišťuje hladký chod celého systému a dodává centru potřebnou autoritu. Z periferie do centra proudí pracovní síla a surový materiál, který je pak zpracován specialisty žijícími v centru (Rowlands 1987, 4). Periferie je většinou v rámci teorie vnímána jako pasivní součást systému, která dodává potřebný materiál a levnou pracovní sílu, a je oproti centru ekonomicky i kulturně na nižší úrovni. Centra jsou oproti periferiím považována za společnost vyvinutější na škále společenské komplexity, jako např. stát vs. předstátní uskupení (Stein 1998, 223-225). Systém center a periferií je tak poměrně nestabilní a založený na nerovné distribuci ekonomického kapitálu a moci. Systém center a periferií se vyvíjí v závislosti na vnějších a vnitřních okolnostech: centrum může postupem času degradovat na pouhou periferii, a naopak periferie se může proměnit na semi-periferii, či dokonce na centrum (Frankenstein a Rowlands 1978, 80-81).

V rámci archeologické aplikace je periferie na předstátní úrovni vedena skupinou mezi sebou soupeřících aristokratů, kteří organizují výměnu surového materiálu mezi periferií a centrem výměnou za vlastní prospěch a protislužby (Frankenstein a Rowlands 1978, 76).\footnote{Elity zajišťovaly přísun materiálu do centra, což v tomto případě představuje Středomoří, a zároveň si v periferii udržovaly výsadní postavení, které demonstrovaly právě vlastnictvím luxusních nádob pocházejícím z centra. Za odměnu elitám centrum poskytuje prestižní předměty zhotovené specialisty v centru. Většinou se mohlo jednat o nádoby, šperky či zbraně z drahého kovu, které aristokratům v rámci periferie sloužily jako prestižní předměty, poukazující na jejich vysoké postavení a společenský status (Frankenstein a Rowlands 1978, 76-77; Rowlands 1987, 5). Pouze pokud si elity zajistily kontrolu nad výměnou zboží s centrem, mohly úspěšně kontrolovat redistribuci importovaného luxusního zboží. Distribucí luxusního zboží si náčelníci zajišťovali dostatečný počet následovníků, a zavazovali si tak jejich přízeň a podporu do budoucnosti (Sahlins 1963, 288-296; Whitley 1991, 349-350).} Archeologové tak za využití ekonomické teorie vysvětlují na tomto koloběhu materiálu a protislužeb přítomnost luxusních nádob v kmenově řízených společnostech v jihozápadním Německu rané doby železné.

Michael Dietler kritizoval tento model pro přílišné zjednodušování reality a popření aktivní role původního obyvatelstva (2005, 58-61). Dietler porovnával teorii světových systémů s hellénizačním modelem, podobně jako u hellénizace, je centrum prezentováno jako ekonomicky a kulturně jednotné a nadřazené periferii. Dalším bodem Dietlerovy kritiky je zaměření teorie pouze na ekonomické vysvětlení kontaktů a cirkulaci a rozmístění kapitálu, a nikoliv na kulturní a společenské následky vycházející ze setkání s cizí kulturou. Primárním prostředkem kontaktu je u teorie světových systémů obchod, což však nezahrnuje komplexnost kontaktů a složitost vztahů k nimž mohlo docházet.\footnote{Jen pro příklad Colin Renfrew (1986, 8) uvádí jako možné druh mezikulturních interakcí vojenské konflikty, diplomatické vztahy, průzkum nových oblastí, migraci, konkurenčního soupeření apod.} Teorie světových systémů pouze málokdy zahrnuje i jiné než obchodní motivace a způsoby šíření luxusní nádob, jako je například výměna darů v rámci budování sítě často nadregionálních společenských vztahů a kontaktů (Mauss 1966).\footnote{Tzv. {\em gift-giving society}, více Mauss (1966).}

Hlavním přínosem použití teorie světových systémů v archeologii bylo kladení většího důrazu na vzájemnou propojenost regionů, a zároveň snaha o vysvětlení nerovnoměrného rozdělení kapitálu na úrovni regionů (Harding 2013, 379). Teorie světových systémů byla poměrně úspěšně aplikována na větší regionální a nadregionální celky, kde bylo možné definovat jedno ekonomické a politické centrum.\footnote{V případě starověkého světa se jednalo o říše rozkládající se na velké ploše jako např. Řím, Mezopotámie, Egypt, či pozdější Mayská říše (Rowlands 1987, 5).} Oproti tomu aplikace teorie světových systémů na řecký svět v sobě nese několik problémů: řecky mluvící svět nebyl po většinu své existence centralizovaný a nespadal pod jedno ekonomické a politické centrum (Stein 1998, 226).\footnote{John Bintliff se pokusil aplikovat teorii světových systémů na regionální celky v rámci antického Řecka (1996), avšak jeho model zahrnoval poměrně obecné ekonomické a produkční trendy a celková použitelnost tohoto modelu pro studium mezikulturních kontaktů byla velmi omezená.}

I přes na svou dobu inovativní postoj bývá teorie světových systémů kritizována pro svou aplikaci moderní ekonomické teorie na antickou ekonomiku, která jednak nedosahovala měřítek moderních ekonomik, ale pravděpodobně se řídila i jinými pravidly a měla odlišnou dynamiku, než moderní společnost (Hodos 2006, 5-7).

\subsubsection[teorie-světových-systémů-v-thrákii]{Teorie světových systémů v Thrákii}

Ekonomicky motivované interpretace archeologického materiálu v sobě spojují jednak teorii světového systému, ale nesou i prvky hellénizačního přístupu. Hlavní motivací mezikulturního kontaktu je obchod a výměna materiálu, případně absence surového materiálu v řeckých městech a jeho nadbytek v oblastech obývaných Thráky (Vranič 2012, 32-36). Oblasti Thrákie bývá v nadregionálních studiích zabývajících se teorii světových systémů věnováno poměrně málo prostoru, většinou jako zmínka na okraji, či podpůrný argument. V rámci teorie světových systémů je Thrákie chápána jako periferie či jako semi-periferie, v závislosti na měnících se podmínkách a politické situaci, a řecká města na thráckém pobřeží jako centrum (Randsborg 1994, 99-104). Thrákie bývá ve shodě s literárními prameny vnímána jako zdroj surových materiálů, zejména stříbra, zlata a surového dřeva na stavbu lodí, otroků a námezdných vojáků (Hdt. 1.64.1; 5.53.2; Sears 2013, 31; Tsiafakis 2000; Isaac 1986; 14-15; Lavelle 1992, 14-22).\footnote{Podrobněji v kapitole 1.} Lokální studie zaměřující se pouze na Thrákii interpretují hellénizovaná města a obchodní sídliště ({\em emporia}) jako lokální ekonomická centra, kde hellénizované elity či přímo řečtí obchodníci shromažďují a následně dodávají potřebné suroviny z thrácké periferie do řeckého světa (Bouzek 1996; 2002; 2007; Bouzek a Graninger 2015).

Přítomnost luxusních předmětů v thráckém vnitrozemí je vysvětlována jako důsledek výměny zboží a protislužeb mezi centrem a aristokraty z periferie. Z literárních pramenů jsou známy případy, kdy thráčtí aristokraté zprostředkovávali kontakt s řeckými městy, jako například jistý Nymfodóros z Abdéry, který se stal prostředníkem mezi Athénami a thráckými Odrysy (Hdt. 7.137; Thuc. 2.29; Sears 2013, 27). Můžeme jen odhadovat, že odměnou jim byly luxusní předměty řecké provenience, či služby řeckých specialistů, kteří pro ně pracovali přímo v Thrákii.\footnote{Archeologické výzkumy odhalují, že si thrácká aristokracie natolik považovala předmětů řecké provenience, že je ukládala do svých, mnohdy výstavních hrobek (Theodossiev 2011, 21-25).} Fenomén bohatých válečnických hrobek s množstvím importovaných předmětů, na nichž byly navršeny monumentální mohyly, je v duchu teorie světových systémů považován za jeden ze znaků kompetitivního charakteru thrácké společnosti, která byla ovládána místními kmenovými náčelníky. Tento fenomén se objevuje v Evropě v době železné, nejen v Thrákii, ale na dalších místech Evropy (Randsborg 1994, 102).

Pokud se podíváme na detailněji na povahu kontaktů mezi Řeky a Thráky, teorie světových systémů nemůže sloužit jako univerzální vysvětlení mezikulturních interakcí. Pro ilustraci uvádím rozložení ekonomické a politické moci v 5. st. př. n. l., které se zcela vymyká binárnímu rozdělení na centrum a periferii. O politické a ekonomické dominanci Řeků nad Thráky není možné mluvit zcela jednoznačně, zejména, když literární prameny vyjadřují spíše opak. Thúkýdidés si všímá, že řecká města platila thráckému králi nemalé obnosy a s největší pravděpodobností tudíž nad ním neměla ekonomickou ani politickou moc, ale spíše naopak (Thuc. 2.97; Graham 1992, 61-62). V neposlední řadě měli Thrákové nejen přístup k nerostnému bohatství, ale literární prameny na mnoha místech poukazují na fakt, že Thrákie měla silné a početné vojsko, které pravděpodobně svým počtem převyšovalo vojska jakéhokoliv z řeckých měst na pobřeží Thrákie.\footnote{Thúkýdidés zmiňuje, že v roce 431 př. n. l. měla armáda krále Seutha až 150 000 mužů (Thuc. 2.98).} Podobně tedy jako v případě hellénizace, ani teorie světových systémů nepostihuje mnohdy jemné nuance mezikulturního styku Thráků a Řeků.

