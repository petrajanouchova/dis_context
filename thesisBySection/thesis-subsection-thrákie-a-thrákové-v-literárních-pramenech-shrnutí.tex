
\subsection[thrákie-a-thrákové-v-literárních-pramenech-shrnutí]{Thrákie a Thrákové v literárních pramenech: shrnutí}

Literární prameny se vyjadřují o Thrácích poměrně stereotypně po celou dobu antiky. Ač Thrákie nikdy nestála v popředí zájmu autorů, i přesto se objevovala poměrně často v narážkách, vedlejších příbězích či jako ilustrace širšího společensko-historického rámce. Nicméně z dochovaných četných zmínek o Thrákii a Thrácích vyplývá, že ke vzájemným kontaktům docházelo poměrně běžně, a to jak na úrovni diplomatických styků, tak i na úrovni obchodních kontaktů, a každodenních styků. K vzájemnému ovlivňování docházelo např. i na poli náboženství a kulturních zvyklostí, ač informace o vzájemném vlivu jsou zprostředkované skrze médium literární tvorby a často s odstupem několika století.

Historiografické prameny 5. a 4. st. př. n. l. daly základ všem pozdějším dílům. Právě z této doby pochází i dva základní stereotypy pevně svázané s charakterizací Thráků. Jako červená nit se vine literárními prameny zvěst o thrácké statečnosti, zálibě v boji a často až hrubosti, která byla jistě i v mnoha případech zasloužená. K Thrákům autoři přistupují dle aktuální politické situace a popisují je tak, aby dosáhli kýženého efektu. V dobách spojenectví jsou Thrákové líčeni jako rovní partneři, z trochu odlišnými zvyklostmi a vírou. V dobách, a v situacích, kdy jsou vnímáni jako nepřátelé, literární prameny mluví o barbarech, divokých a krutých stvořeních, kteří jsou kulturně primitivnější v porovnání s Řeky a Římany. Ani tento fakt však nebrání, aby Thrákové hráli roli rovnocenných partnerů a obávaných protivníku, a to jak v řeckém kulturním okruhu, tak i v prostředí římské říše.

Literární prameny rozlišují jednotlivé kmeny a uznávají odlišnost společenského uspořádání Thrákie. Řečtí autoři právě v Thrácích vidí svůj vlastní předobraz polomýtické minulosti, kdy společnost ovládali kmenoví vůdcové a hlavním měřítkem úspěšnosti byla společenská prestiž. Kontakty a kulturní výměnu s Thráky řecké prameny do určité míry reflektují, ale nemluví však o běžných obyvatelích, ale téměř výhradně o politických elitách. Navíc je popis thrácké společnosti značně nesystematický a selektivní, často přizpůsobený řeckému publiku. Proto je vhodné k literárním pramenům přistupovat obezřetně, pečlivě zvažovat kontext jejich vzniku a nepovažovat je automaticky za přesný obraz tehdejší společnosti. Pro doplnění a rozšíření je vhodné se obrátit na archeologické a epigrafické prameny, které mají tu výhodu, že pochází přímo od obyvatel Thrákie, a nikoliv od vnějších pozorovatelů. Jejich výpovědní hodnota není zprostředkovaná sítem řecké a římské literární tradice a dává nám tak nahlédnout do nitra komunity samotné.

