
\subsection[thrákie-v-1.-tis.-př.-n.-l.]{Thrákie v 1. tis. př. n. l.}

Materiální kultura první poloviny 1. tis. př. n. l. vykazuje velké množství lokálních variant, které poukazují na nejednotnost thráckého etnika, což odpovídá spíše regionálnímu rozdělení dle jednotlivých kmenových uskupení (Kostoglou 2010, 180-185; Graninger 2015, 25). Členění území dle kmenovému principu odpovídá i absence sídlišť městského typu, a naopak velké množství decentralizovaných menších sídel se zaměřením na zemědělskou produkci, doplněné opevněnými sídly kmenových vůdců a opevněných strážních sídel v horských oblastech (Theodossiev 2011, 15-17; Popov 2015, 111-114). Pro thráckou materiální kulturu jsou typické náhrobní mohyly, z hlíny navršené stavby překrývající hrobky, které zpravidla obsahovaly bohatou pohřební výbavu a usuzuje se, že patřily thráckým aristokratům (Theodossiev 2011, 20-21). Dalšími typickými stavbami jsou dolmeny a svatyně tesané do skal, které se nacházejí zejména v hornatých a hůře přístupných částech jihovýchodní Thrákie, a jejichž funkce souvisela pravděpodobně s rituálními aktivitami thráckého obyvatelstva (Nekhrizov 2015, 126-141).

Oblasti obývané Thráky se staly předmětem zájmu řeckých obcí zejména od 7. st. př. n. l., nicméně jak dokazují archeologické nálezy poslední doby, docházelo ke kontaktu s okolními oblastmi již v době před vznikem prvních řeckých kolonií na území Thrákie (Zahrnt 2015, 36). Tyto kontakty byly pravděpodobně obchodní povahy, ale bohužel archeologické doklady nejsou zcela jednoznačné, vzhledem k jejich mnoha možným interpretacím.

