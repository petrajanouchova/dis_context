
\subsection[typologie-nápisů-1]{Typologie nápisů}

Dochované nápisy z 6. až 5. st. př. n. l. jsou typologicky totožné se skupinou nápisů ze 6. st. př. n. l. Jedná se taktéž o funerální texty, vytvořené za účelem označení místa pohřbu a na památku zesnulého v nejbližší komunitě. Nosiče nápisu jsou zhotovené z místního zdroje kamene a tvarově zachovávají podobu stél s jednoduchou dekorací. Jazyk nápisu je řecký, stejně tak jako původ dochovaných jmen. Text nápisů je ve všech třech případech psán v první osobě a obrací se na čtenáře, kterého informuje o identitě zemřelého a jeho původu, případně vyzdvihuje jeho kladné vlastnosti. Nakolik můžeme z takto malého vzorku soudit, zvyk zhotovovat náhrobní kameny s nápisy se i nadále udržel pouze v rámci řecké komunity.

