
\subsection[veřejné-nápisy-1]{Veřejné nápisy}

Dochované tři veřejné nápisy reprezentují pouze 5 \letterpercent{} nápisů z daného souboru. Ve dvou případech se jedná o nápis vymezující hranice náboženského okrsku řeckých božstev Dia, Athény a {\em héróů} se jmény {\em Podalirios}, {\em Machaón} a {\em Periéstos}. Tyto nápisy pocházejí regionu řeckého města Strýmé na pobřeží Egejského moře. Nápis z Abdéry představuje velmi fragmentárně dochované nařízení vydané blíže neznámou politickou autoritou mezi lety 485 a 475 př. n. l. Na žádném z nápisů nebylo možné určit kontext komunity, z které nápis pocházel, vzhledem k chybějícím osobním jménům a vyjádřením identity. Místa nálezů pocházejí z regionu řeckých měst Abdéra, Strýmé, a tudíž se dá předpokládat, že byly vytvořeny zde žijícím řeckým obyvatelstvem pro interní potřeby řecké komunity. Použitý lokální materiál, jako je vápenec a mramor, více než nasvědčují omezení produkce veřejných nápisů pouze na řeckou komunitu v pobřežních oblastech.

