
\subsection[veřejné-nápisy-10]{Veřejné nápisy}

Veřejných nápisů se dochovalo celkem devět, z nichž šest představují honorifikační dekrety udělené institucemi řeckých měst významným jedincům.\footnote{Jako např. thrácký král Kotys na nápise {\em I Aeg Thrace} 207 z Maróneie,} Politickou autoritu představují jednotlivé instituce řeckých {\em poleis}, osoba thráckého krále a jednotliví stratégové, případně sama autorita Říma.\footnote{Příkladem je nápis {\em IG Bulg} 5 5011 z Dionýsopole; {\em IK Sestos} 1 ze Séstu.}

Osobní jména na veřejných nápisech dokazují stále ještě řecký charakter epigraficky aktivní populace: dochovalo se celkem 93 řeckých jmen, 11 thráckých a devět římských.\footnote{Nápis {\em IG Bulg} 1,2 46 představuje seznam kněžích nejmenovaného kultu z Odéssu a dochovalo se na něm celkem 50 jmen, z čehož 46 bylo řeckého původu a dvě byla identifikována jako jména thrácká a dvě římská. Celkem 46 kněžích byli výhradně muži a původ jejich jmen byl takřka výhradně řecký, což může nasvědčovat i jisté konzervativnosti nejmenovaného kultu.} Formule použité v honorifikačních nápisech dokumentují jisté přetrvání zvyklostí a procedur spojených s vystavením nápisů, podobně jako v předcházejících stoletích, avšak pokles výskytu tradičních formulí spojených s udílením poct značí proměnu vnitřního uspořádání politických autorit, která se odráží i v použitém jazyce veřejných nápisů.\footnote{Nápis {\em IG Bulg} 1,2 320 z Mesámbrie popisuje proceduru korunovace, což byla jedna z poct udílených v rámci řecké {\em polis}.}

