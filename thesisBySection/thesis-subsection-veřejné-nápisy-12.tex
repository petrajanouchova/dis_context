
\subsection[veřejné-nápisy-12]{Veřejné nápisy}

Celkem se dochovalo 15 veřejných nápisů: čtyři z Maróneie a Perinthu a po jednom nápise z ostatních měst na pobřeží. Politickou autoritu představují instituce řeckých {\em poleis}, jako {\em búlé} a {\em démos} v případě Maróneie, a {\em démos} v případě Odéssu a Perinthu, a dále římský císař a vysoce postavení úředníci římské říše. Thráčtí králové se na veřejných nápisech nevyskytují, stejně tak jako stratégové. Nápisy mají nejčastěji povahu honorifikačních dekretů či seznamů věřících a textů náboženské povahy. Tradiční forma honorifikačních dekretů tak, jak jí známe např. z 3. st. př. n. l. se nedochovala, z čehož se dá usuzovat, že došlo k proměně celé procedury udílení poct, a tedy i k proměně podoby honorifikačních textů, podobně jako např. v Malé Asii (Van Nijf 2015, 240).\footnote{Bohužel dochovaný soubor nápisů je natolik fragmentární, že není možné blíže popsat jednotlivé změny.}

