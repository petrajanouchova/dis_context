
\subsection[veřejné-nápisy-13]{Veřejné nápisy}

Poprvé větší část veřejných nápisů pochází z vnitrozemí, a to zejména z velkých městských center a jejich nejbližšího okolí. Tento fakt jistě souvisí s urbanizačními aktivitami a posílením politické moci a významu městské samosprávy za vlády Trajána a Hadriána (Topalilov 2012, 14-15). Celkem se dochovalo 117 veřejných nápisů, což představuje oproti 1. st. n. l. zhruba čtyřnásobný nárůst celkového počtu. Tento trend je pozorovatelný zejména ve oblasti vnitrozemské Thrákie v okolí měst Nicopolis ad Istrum a Filippopolis, odkud pochází 21 nápisů a Augusta Traiana se 14 nápisy. Nejvíce se dochovalo honorifikačních nápisů, ale objevují se ve větší míře i milníky a nápisy dokumentující stavební aktivity.

Honorifikační nápisy představují nejvýznamnější skupinu veřejných nápisů. Oproti předcházejícímu období se jedná o téměř osminásobný nárůst z osmi na 63 honorifikačních nápisů, což dokazuje oblíbenost tohoto druhu nápisů. Honorifikační nápisy pocházejí z měst jak na pobřeží, tak zejména ve vnitrozemí z regionu města Nicopolis ad Istrum, odkud pochází 18 textů, a z regionu Filippopole, kde bylo nalezeno 14 textů. Honorifikační dekrety jsou i nadále vydávány politickou autoritou města, a to pod patronátem římského císaře.\footnote{Nejčastějšími termíny reprezentujícími politickou autoritu měst jsou {\em búlé} s 24 výskyty, {\em démos} s 25, {\em polis} se 7 výskyty.} Tradiční forma honorifikačních nápisů, kdy se dobrodinci ({\em euergetés}) udělují konkrétní privilegia za vykonané dobrodiní, je nicméně vystřídána novou formou, kdy namísto významného jedince je hlavní pozornost věnována císaři, na jehož počet většina nápisů a s nimi spojených monumentů a budov vzniká.\footnote{Texty obsahují tradiční invokační formuli ve 27 případech ({\em agathé týché}) a jsou převážně psány řecky. Dva texty jsou psány latinsko-řecky, kde latinský text je uváděn první a řecký text je překladem latinského textu a hlavní autoritou těchto nápisů je vždy římský císař. Další dva nápisy jsou čistě latinské a jsou určeny přímo římskému císaři.} Tato proměna honorifikačních nápisů je patrná již v průběhu 1. st. n. l. i v jiných částech římské říše (Van Nijf 2015, 240). Důraz je kladen převážně na osobní kvality a zásluhy jedinců, jejich pověst a společenskou prestiž, což může odrážet i větší míru stratifikace tehdejší společnosti.

Milníky a nápisy dokumentující stavební aktivity organizované provinciální samosprávou poukazují na nárůst stavebních aktivit v 2. st. n. l., zejména po roce 116 n. l. Celkem devět milníků nápisů pochází jak z okolí významných měst ve vnitrozemí, tak na pobřeží. Silnice {\em Via Diagonalis} byla v této době doplněna dalšími menšími silnicemi spojujícími významná města právě s touto nejdůležitější dopravní tepnou celého Balkánu. Dochované milníky poukazují na existenci cest v druhé polovině 2. st. n. l. v okolí Marcianopole a Odéssu, dále i na opravy již zmíněné {\em Via Diagonalis} spojující Serdicu a Byzantion. Silnice využívala primárně římská armáda, nicméně z této infrastruktury těžilo i místní obyvatelstvo a silnice umožnily lepší propojení regionu a pohyb obyvatel. \footnote{Císař mohl udělit speciální práva a ochranu konkrétní skupině obyvatel, jak je tomu na nápise {\em I Aeg Thrace} 185 z Maróneie. Císař Hadrián tímto dekretem určeným obyvatelům Maróneie z roku 131 n. l. mimo jiné udělil právo k užívání a ochranu po dobu putování po cestě z Maróneii do Filipp, čímž se pravděpodobně myslela {\em Via Egnatia}, jedna z nejvýznamnějších římských cest. Na civilní využití menších cest a mostů v okolí Pautálie poukazuje nápis {\em SEG} 54:648 z poloviny 2. st. n. l.}

Další nápisy zaznamenávají stavbu akvaduktů v Odéssu, městského opevnění ve Filippopoli a Serdice, a lázní v Pautálii a Augustě Traianě či rekonstrukci divadla ve Filippopoli. Dva dochované nápisy označují hranice území Abdéry jako samosprávné jednotky pod autoritou císaře.\footnote{{\em I Aeg Thrace} 78 a 79. Předpokládá se, že hraniční kameny se vyskytovaly na pomezí území měst vcelku často, do dnešních dnů se jich dochovalo bohužel pouze několik exemplářů.} Stavební aktivity většinou financovali místodržící či vysocí úředníci na počest císaře a soudě dle osobní jmen, byli tito vysoce postavení muži zejména římského původu, ale vyskytovali se i jedinci nesoucí kombinaci thráckých a římských jmen.\footnote{Jako např. Titus Vitrasius Pollio z nápisu {\em IG Bulg} 1,2 59 a Poplios Ailios Aulouporis z nápisu {\em IG Bulg} 5 5334.}

Veřejným textům náboženského charakteru zcela dominuje císařský kult, nicméně v seznamech věřících převládají řecká a thrácká jména.\footnote{Na nápise {\em IG Bulg} 4 1925 se setkáváme se seznamem příslušnic mystérií Velké Matky a Attida, kde se vyskytují jak řecká, tak římská ženská jména, avšak text je poměrně špatně dochovaný.} Osoba císaře hraje zásadní roli i v rozložení politické moci.\footnote{Termíny označující autoritu císaře jsou s 30 výskyty {\em autokratór}, dále s 26 výskyty {\em kaisar}, s 12 {\em hypatos} a se 17 {\em hegémón}. Císař bývá někdy titulován i jako otec vlasti, {\em patér patridos}, a {\em archiereus}, což je řecká verze titulu {\em pontifex maximus} (Mason 1974, 115-116).} Autoritu císaře přímo v provincii reprezentuje místodržící a jeho zástupce, na nápisech často titulovaný jako {\em presbeutés} a {\em antistratégos}, latinsky {\em legatus Augusti pro praetore} (Mason 1974, 153-155), a to konkrétně ve 24 případech. Jako zástupce císařské moci vystupuje v několika případech i {\em thrakarchés}, tedy vysoký úředník s náboženským zaměřením, podobným nejvyššímu knězi celé provincie (Lozanov 2015, 82-83). Úřad {\em thrakarcha} obvykle zastupoval muže nesoucí kombinovaná thrácká a římská jména, což dokazuje jeho aristokratický thrácký původ, který nebyl překážkou kariéry v rámci provinciálních institucí (Sharankov 2005, 532).

S dobou vlády Trajána se taktéž spojují administrativní reformy, díky nimž narostla autorita měst ve vnitrozemí na úkor dřívějšího uspořádání založeného na kmenovém principu, což se projevilo i na výskytu termínů užívaných na nápisech.\footnote{Parissaki (2013, 83-84) se domnívá, že dříve fungující systém stratégií byl za doby vlády Trajána nahrazen novým administrativním systémem, v němž hrála zásadní roli městská samospráva.} Z jednotlivých úřadů či osob zastávajících funkce v rámci samosprávy se objevuje {\em búlé} ve 34 případech {\em démos} ve 42, {\em polis} ve 24, dále {\em archón} ve čtyřech nápisech, po dvou nápisech {\em fýlé} a {\em pontarchés}\footnote{Úřad {\em pontarcha} je znám z císařské doby a jedná se o nejvyššího úředníka uskupení šesti měst na pobřeží Černého moře, který zastupoval císařskou autoritu a zároveň vykonával úřad nejvyššího kněze (Cook 1987, 30). Toto uskupení černomořské Hexapole zahrnovalo pobřežní města převážně z provincie {\em Moesia Inferior} jako Histria, Tomis, Kallatis, Dionýsopolis, Odéssos a Mesámbria a usuzuje se, že vzniklo v době vlády císaře Trajána či Hadriána, tedy před polovinou 2. st. n. l. Vytvořením této organizace původně řecká města fakticky ztratila svou autonomii, či její pozůstatky, a spadala přímo pod autoritu císaře (Lozanov 2015, 83).} a po jednom nápise {\em agaranomos} a {\em gerúsia}. V rámci posílení pozice se města z Thrákie společně sdružovala v tzv. {\em koinon tón Thrákón}, s Filippopolí jako hlavním městem a sídlem spolkového sněmu ({\em métropolis}; Sharankov 2005, 518-531).\footnote{Přesná funkce tohoto uskupení měst není jasná, nicméně jeho vznik byl pravděpodobně inspirován vzorem ze sousední Bíthýnie pro usnadnění provinciální administrativy (Lozanov 2015, 82).} Politické centrum tehdejší provincie se tak přesunulo z Perinthu do vnitrozemské Filippopole, čemuž odpovídá i přesun jednoho z největších producentů veřejných, tak i soukromých nápisů.

