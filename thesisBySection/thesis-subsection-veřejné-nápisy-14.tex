
\subsection[veřejné-nápisy-14]{Veřejné nápisy}

Celkem se dochovalo 46 veřejných nápisů datovaných do 2. až 3. st. n. l., což představuje trojnásobný nárůst oproti skupině nápisů datovaných do 1. až 2. st. n. l. Nejvíce nápisů pochází z Augusty Traiany s 10 nápisy, z Perinthu a Nicopolis ad Istrum se sedmi nápisy a z Filippopole s pěti nápisy. Nejčastějším typem s 29 výskyty jsou i nadále honorifikační nápisy vydávané městskými institucemi. Dochází k pokračujícím změnám epigrafického jazyka a standardizaci veřejných nápisů, stejně tak k proměně formulí typických pro veřejné nápisy. I nadále dekrety vydává {\em búlé} a {\em démos} pod patronátem císaře\footnote{Císař je označován termíny {\em kaisar} sedmkrát a {\em autokratór} 17krát.}, nicméně nyní se setkáváme i s přídomky {\em kratistos, hieros, hierotatos} či {\em lamprotatos}, které jsou spojovány s jednotlivými městskými institucemi. Podobné označení se městským institucím odstávalo i v římských provinciích Malé Asie, a to přibližně ve stejné době, což je příznakem jisté standardizace epigrafického jazyka napříč římskými provinciemi (Heller 2015, 250-253). K určitému sjednocení formy nápisů napříč městy dochází i pokud jde o uvádění císařských titulů, císařovi rodiny a zařazení nápisů do daného roku císařovi vlády.\footnote{Např. na nápise {\em IG Bulg} 2 617 a 624 z Nicopolis ad Istrum.} Na nápisech jsou často uváděni i vysocí provinciální úředníci, za jejichž služby došlo k vydání nápisu, či došlo např. k stavebním aktivitám.\footnote{Termín {\em presbeutés} a {\em antistratégos} se vyskytuje sedmkrát, {\em thrakarchés} jednou, {\em búleutés} dvakrát, {\em efébarchos} jednou, {\em synedrion} jednou, {\em métropolis} jednou.}

Z dalších druhů nápisů se dochoval pouze jeden milník, stejně tak jako jeden nápis dokumentující stavební aktivity veřejného charakteru, či označení hranic regionu města.\footnote{{\em Perinthos-Herakleia} 38, 40 a {\em IG Bulg} 5 5540.} Z dalších dokumentů se dochovaly dva seznamy osob, které dobře zachycují onomastické zvyky ve vztahu ke kolektivní identitě. Na nápisech {\em IG Bulg} 1,2 50 a 51 představující seznam věřících boha Dionýsa a seznam efébů z Odéssu zcela převládají řecká a thrácká jména. To může být důsledek kulturně-společenských norem, které měly za následek, že osoby nesoucí římská jména do těchto skupin nevstupovali, případně si v daném společensko-kulturním kontextu vystupovali pod jinou součástí své identity, jejíž součástí nebylo římské jméno.

