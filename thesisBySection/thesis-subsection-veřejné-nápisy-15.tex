
\subsection[veřejné-nápisy-15]{Veřejné nápisy}

Oproti 2. st. n. l. je pozorovatelný další nárůst počtu veřejných nápisů zhruba o 50 \letterpercent{} na 176 exemplářů. Veřejné nápisy pocházejí většinou z thráckého vnitrozemí z bezprostředního okolí {\em Via Diagonalis} a městských center ležících na této významné spojnici. Mezi největší producenty veřejných nápisů patří Augusta Traiana s 38 nápisy, Filippopolis s 37 nápisy, Serdica s 27 nápisy, Nicopolis ad Istrum s 22 nápisy, Perinthos (Hérakleia) s 9 nápisy. Naopak veřejné nápisy téměř vymizely z Byzantia, odkud pocházejí pouze dva texty, či z Maróneie, kde byl nalezen jeden veřejný nápis.\footnote{Dekrety jsou stále nejčastějším typem dokumentu s 105 nápisy, z nichž 95 představuje dekrety honorifikační. Dále sem patří sedm seznamů a 51 nápisů jiného typu, z čehož 42 jsou milníky, tři hraniční kameny a jeden nápis dokumentují stavební aktivity.}

Honorifikační nápisy jsou i nadále nejpočetnější skupinou veřejných nápisů a pocházejí především z velkých městských center té doby s existujícími samosprávnými institucemi a velkým počtem obyvatelstva. Většina textů pochází z vnitrozemí: 23 nápisů pochází z Augusty Traiany, 21 nápisů z Nicopolis ad Istrum, 20 z Filippopole, 13 z Perinthu a pět z města Serdica. Nápisy jsou vydávány místní samosprávou, reprezentovanou {\em búlé} a {\em démem}, případně {\em gerúsií}, avšak zcela pod patronátem římského císaře, kterého zastupují místodržící a vysoce postavení úředníci daného města, např. {\em epimelúmenos} a {\em logistés}, lat. {\em curator}, {\em argyrotamiás}, lat. {\em questor}, či {\em thrakarchés} (Mason 1974, 25; 46). Mezi honorovanými jedinci se objevují členové {\em búlé}, {\em gymnasiarchové}, vojáci, kněží, nižší úředníci jako např. {\em grammateus}, či sportovci ({\em agonothétés}). Jak dokazují opakující se formule, na počest významných jedinců vztyčovány stély či jejich sochy byly umístěny na veřejných místech, což poukazuje na důležitou roli, jakou v tehdejší společnosti hrál společenský status (Van Nijf 2015, 233-243). Většina osobních jmen byla římského původu, což může naznačovat, že vysoké funkce zastávali římští občané italického původu, či místní Thrákové zcela upustili od tradice zachovávání thráckých jmen a přijali pouze jména římského původu.\footnote{Pokud by tomu tak bylo, ve většině případů je nedokážeme navzájem odlišit, pokud specificky neudávají geografický svůj původ, což se neděje.}

Významem zůstává obsah honorifikačních nápisů podobný, nicméně každé větší město používá specifické formule a slovní obraty, které se na jiných místech nevyskytují, či se vyskytují v jiné formě.\footnote{Příkladem může být Filippopolis, které je vždy označována jako {\em lamprotaté polis}, tedy nejslavnější z měst, a Perinthos, který v textech nápisů figuruje jako {\em lampra polis} čili jako slavné město.} Z variability formulací je možné soudit, že podoba honorifikačních nápisů nebyla striktně regulována vyšší autoritou, ale že se jednalo vždy o místní interpretaci sdělení. Je pravděpodobné, že se jednotlivé městské samosprávy navzájem inspirovaly, nicméně finální podoba nařízení byla ponechána zcela místním institucím. Jinak tomu však bylo v případě milníků, které vykazují shodný charakter včetně formy a obsahu napříč různými částmi římského impéria. To může poukazovat jednak snahu o zvýšení srozumitelnosti sdělení při přesunech mnohonárodnostního římského vojska, ale i jistý stupeň centralizovanosti byrokratického aparátu, který měl zřizování cest, a tedy i milníků, na starosti.

Dochované milníky dokumentují existenci fungující infrastruktury a stavebních aktivit v průběhu 3. st. n. l. Především koncem 2. st. a začátkem 3. st. n. l. dochází k zintenzivnění aktivit spojených s výstavbou a zejména opravou již existujících cest, které sloužily primárně pro přesuny římské armády, ale i k obchodu a zásobování (Madzharov 2009, 64). Milníky udávaly vzdálenost do nejbližšího města, které zároveň bylo i samosprávné jednotkou, na jehož území se milníky nacházely a které mělo za povinnost se starat o údržbu daného úseku cesty.\footnote{V deseti případech to byla Filippopolis, osmi případech Serdica, v sedmi Augusta Traiana, v pěti Pautália, ve dvou Perinthos (Hérakleia) a Hadrianúpolis, a v jednom Traianúpolis. Udávané vzdálenosti se pohybovaly v rozmezí dvou až 37 mil, tj. zhruba tří až 55 kilometrů.} Většina milníků pochází z trasy {\em Via Diagonalis}, která spolu s {\em Via Egnatia} spojovala evropské provincie s maloasijskými. Tato aktivita započatá za Severovců pokračuje i v průběhu celého 3. st. n. l., jak dokazuje počet dochovaných milníků.\footnote{Z doby severovské dynastie se dochovalo celkem 14 milníků, z období po roce 235 n. l. do konce století dalších 24, a do doby na přelomu 3. a 4. st. n. l. dalších pět.}

O rozsahu centrálně řízených stavebních aktivit a existující infrastruktury vypovídají dochované veřejné nápisy.\footnote{Zmínky na nápisech dokumentují stavbu agory v blíže neznámém městě v údolí řeky Strýmónu, jako je tomu v případě nápisu {\em IG Bulg} 4 2264 z moderního města Sandanski. Další nápis {\em I Aeg Thrace} 433 z Traiánúpole poukazuje na existující praxi vyměřování o vnitřní rozdělení a uspořádání území Traiánúpole na egejském pobřeží v roce 202 n. l. Velmi podobný nápis {\em I Aeg Thrace} 447, avšak hůře dochovaný, pochází i z nedaleké Alexandrúpole. Nápis datovaný taktéž do r. 202 n. l. nasvědčuje, že za Severovců došlo k novému vyměření území a jejich vnitřní struktury u nejméně dvou měst na egejském pobřeží.} O vnějším uspořádání měst a jejich území svědčí hraniční kameny ({\em horoi}), kterých se dochovalo celkem osm, což je největší počet ze všech předcházejících období. Města těmito kameny vymezovala rozsah svého území, a to jednak pro samosprávní účely, výběr daní, ale například i pro jasnější financování infrastrukturních projektů, jako byla např. údržba cest.\footnote{Hraniční kameny se našly na hranicích území dvou měst, jako je to v případě jednoho nápisu z Odéssu a dvou nápisů z Marcianopole, a dále na hranicích samosprávné jednotky na úrovni vesnice ({\em kómé, chórion}) či jiného osídlení neměstského typu jako v případě tří nápisů z Bendipary v blízkosti Filippopole, a jednoho nápisu patřící Eresénským, tedy obyvatelům Eresy(?) v blízkosti Maróneie. Hraniční kameny byly vydány pod hlavičkou tehdejšího císaře či místodržícího úředníka.}

Stavební aktivita se na konci 2. st. n. l. a na počátku 3. st. n. l. nevztahovala pouze na cesty, ale v epigrafických záznamech objevují nově vzniklá {\em emporia}: Discoruraterae, Pizos, dále sem patří i Pirentensium a Pautália (Lozanov 2015, 84-85). Tato {\em emporia} spadala pod samosprávu nejbližšího města, což byla Augusta Traiana, Nicopolis ad Istrum. Tato lokální centra obchodu zajišťovala městským celkům a vojenským jednotkám jednak zemědělské produkty, řemeslné výrobky, ale zároveň byla umístěna na důležitých vojenských, ale i obchodních cestách a umožňovala výměnu zboží mezi městem a venkovem. Nápis {\em IG Bulg} 3,1 1690 o délce 270 řádek zmiňuje {\em emporion} Pizos a podává výčet obyvatel vesnic z okolí, kteří se podíleli na zakládání {\em emporia}.\footnote{Pizos ležel na {\em Via Diagonalis} mezi přepřahacími stanicemi ({\em mutatio}) Ranilum, Arzus a Cillae a v blízkosti opevněného osídlení vojenského charakteru v Carasuře. Odbytiště pro produkty a výrobky z {\em emporia} zajišťovala z velké části římská armáda, a částečně i sama Augusta Traiana. Jedno z privilegií nového {\em emporia} bylo oproštění od dovozních cel do Augusty Traiany, což oproti místním vesnicím poskytovalo {\em emporiu} nespornou ekonomickou výhodu. {\em Emporion} ve 3. st. n. l. tak zároveň usnadňovalo výměnu zboží a zásobování mezi městskými centry a venkovem. Podobně tomu tak bylo i v případě dalších {\em emporií}, z nichž se nám však na nápisech dochovalo výrazně méně detailních informací o jejich fungování.} To vzniklo sestěhováním obyvatel nejméně devíti thráckých vesnic v době vlády Septimia Severa v roce 202 n. l.\footnote{Jména sestěhovaných vesnic jsou typicky thrácká: {\em Skedabria, Stratopara, Krasalopara, Skepte, Gelúpara, Kúrpisos, Bazopara, Strúneilos a Búsipara}. Zajímavým faktem je, že jedna skupina obyvatel měla sestěhování nařízeno od místodržícího provincie a další skupina se sestěhování účastnila dobrovolně (Boyanov 2014, 185).} V textu se vyskytují pouze mužská jména a mělo se jednat o první osadníky nově vzniklého {\em emporia}, které se nacházelo v regionu Augusty Traiany. Jména 154 sestěhovaných obyvatel jsou z více než dvou třetin thrácká, poukazující na thrácký původ jak nositelů samotných, tak jejich rodičů či sourozenců. Na jejich vyšší společenský status v rámci rurální společnosti poukazují použité funkce jako {\em toparchés} či {\em búleutés}.\footnote{Heller (2015, 266) na příkladech honorifikačních nápisů z Malé Asie dokazuje, že {\em búleutés} zaujímal spíše střední až nižší postavení v rámci hierarchie provinciálních úředníků, nicméně toto postavení bylo stále vyšší než u většiny běžné populace, a proto bylo na nápisech vyzdvihováno.}

Proměnu onomastických zvyklostí ve 3. st. n. l. v reakci nové politické uspořádání nejlépe ilustrují poměrně obsáhlé seznamy osob, kde lze velmi dobře sledovat proměňující se přístup k přijímání římských jmen v rámci vývoje širší společensko-politické situace v římské říši. Před rokem 212 n. l. bylo užití jmen jako Flavios, Iulios, Ulpios, Klaudios pouze záležitostí velmi úzké skupiny lidí, a to jak na pobřeží, tak i ve vnitrozemí, což zvyšovalo jejich prestiž (Parissaki 2007, 286). V dřívějších dobách bylo právo nosit římské jméno výsadou vysloužilých vojáků či vysokých úředníků jako odměna za jejich služby, avšak po roce 212 n. l. mohl toto jméno nosit každý svobodný obyvatel římské říše (Beshevliev 1970, 28-32). V roce 212 n. l. císař Caracalla ediktem známým jako {\em Constitutio Antoniniana} udělil římské občanství všem obyvatelům římské říše, a spolu s ním i právo nosit rodové jméno římského císaře (Beshevliev 1970, 31-32). Z této doby se dochovalo sedm nápisů obsahujících 437 osob a 658 osobních jmen\footnote{Nejvíce osob se nachází na již diskutovaném nápise {\em IG Bulg} 3,1 1690 o založení emporia Pizos, a to 154 osob. Dále sem patří tři nápisy se seznamy efébů z Odéssu ({\em IG Bulg} 1,2 47, 47bis, 48) a jeden z Dionýsopole ({\em IG Bulg} 1,2 14), jeden seznam věřících Dionýsova kultu z Cillae ({\em IG Bulg} 3,1 1517) a jeden seznam věřících blíže neznámého kultu z Augusty Traiany ({\em SEG} 58:679).} a seznamy vydané bezprostředně po roce 212 n. l. vykazují výrazně vyšší poměr přijatých římských jmen, především jména Aurelios, než nápisy např. z poloviny století.\footnote{Příkladem z Thrákie jsou nápisy se seznamy efébů {\em IG Bulg} 1,2 14 a 47 datovaných do doby krátce po r. 212 n. l., kde přes 95 \letterpercent{} osob přijalo ke svému jménu císařské jméno Aurelios. Na dalším seznamu efébů {\em IG Bulg} 1,2 47bis z roku 221 n. l. tento poměr klesl na 85 \letterpercent{}, a u nápisu {\em IG Bulg} 3,1 1517 datovaného do let 241-244 n. l. klesl na 80 \letterpercent{}. Na seznamech efébů se většinou vyskytují muži s kombinovanými řeckými a římskými jmény a zcela výjimečně jména thrácká v necelých 2,4 \letterpercent{}.} Tento trend se ve velké míře objevuje i v dalších částech římské říše, jak dokazuje nedávná studie z Malé Asie (Blanco-Perez 2016, 279). Jedno z možných vysvětlení může naznačovat, že s přibývajícím časovým odstupem od hromadného udělení římského občanství společenská hodnota a prestiž v rámci komunity klesala, a tudíž se jméno Aurelios, a vše co toto jméno představovalo, na nápisech objevovalo méně často.

Spolu s měnícími se onomastickými zvyklostmi je ve 3. st. n. l. taktéž pozorovatelná větší provázanost identity jednotlivce s institucemi a společenskou organizací římské říše, a to jak ve veřejných, tak na soukromých nápisech. Politická identita, respektive přináležitost do samosprávní jednotky na úrovni města, případně lokální samosprávy na úrovni vesnice, se stává důležitou součástí veřejných nápisů.\footnote{Výskyt kolektivního vyjádření identity s příslušností k městu se vyskytuje na soukromých i veřejných nápisech ze 3. st. n. l. celkem 55krát, s příslušností k vesnici sedmkrát.} Tato politická identita zcela nahrazuje etnickou příslušnost a její vyjádření na nápisech.\footnote{Kolektivní termíny Thrákové a Thrákie jsou zmiňovány na 66 soukromých a veřejných nápisech pouze v souvislostí s obyvateli provincie {\em Thracia}, či jako obecné pojmenování typu gladiátora. Zmínka o konkrétních thráckých kmenech je omezena pouze na Serdicu, o níž se na 17 nápisech hovoří jako o městu kmene Serdů ({\em hé Serdón polis}).}

