
\subsection[veřejné-nápisy-17]{Veřejné nápisy}

Celkem se dochovalo šest veřejných nápisů, z nichž dva jsou milníky\footnote{{\em Perinthos-Herakleia} 292 z Hérakleie a Velkov 2005 58 z Mesámbrie.}, dva stavební nápisy, dokumentující stavbu brány v Byzantiu (Konstantinopoli) a stavbu věže v Kaliakře\footnote{{\em SEG} 53:651, {\em IG Bulg} 1, 2 12bis.}, a dva honorifikační nápisy věnované tehdejším císařům vysokými městskými úředníky.\footnote{{\em SEG} 51:916 datovaný na poč. 4. st. n. l., a {\em SEG} 52:695 datovaný mezi roky 324-337 n. l., oba z Augusty Traiany.} Nápisy pocházejí z doby tetrarchie, zejména z vlády císaře Konstantina. Ač je jejich celkový počet velmi omezený, i přesto se jedná o tradiční projevy suverenity římské říše a císaře. Primární funkcí těchto nápisů bylo informovat o stavebních aktivitách státního aparátu, ale i šířit pověst a postavení římského císaře v rámci provincie. Malý počet veřejných nápisů může dosvědčovat, že státní aparát byl v této době v krizi, a tudíž i celková produkce nápisů byla velmi nízká, a omezené pouze do první poloviny 4. st. n. l.\footnote{V polovině 4. st. n. l. došlo k dokončení administrativních reforem, které začaly již za císaře Diokleciána: území Thrákie bylo přeměněno na diecéze {\em Thracia} a {\em Dacia}, z nichž Thracia se dále dělila na šest provincií se svými hlavními městy, která se staly administrativními centry nejbližších regionů (Dumanov 2015, 91).} Státní instituce do jisté míry i nadále fungovaly, což dokumentují dekrety pocházející z Augusty Traiany či milníky z Hérakleie (dříve Perinthu), obecně však produkce nápisů sloužících veřejnému zájmu končí zhruba po polovině 4. st. n. l. O této doby se vyskytují nápisy pouze pro soukromou potřebu jednotlivců či uzavřených komunit.

