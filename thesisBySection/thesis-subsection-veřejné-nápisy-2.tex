
\subsection[veřejné-nápisy-2]{Veřejné nápisy}

Z této skupiny nápisů se dochoval pouze jeden nápis {\em IG Bulg} 1,2 398, který Georgi Mihailov označuje jako {\em res sacrae}, ale svou povahou se nápis nachází na pomezí veřejného, dedikačního a stavebního textu (Mihailov 1970, 365-366). Tento nápis sloužil k označení místa v regionu Apollónie Pontské, kde stál chrám, {\em megaron}, řecké bohyně Gé {\em Chthonios}. Nejedná se tedy v pravém slova smyslu o nápis vytvořený politickou autoritou, který by souvisel s chodem státu či jiné politické organizace, ale spíše o nápis dokumentující rozčlenění půdy a existenci chrámu řecké bohyně na území řecké obce.

