
\subsection[veřejné-nápisy-3]{Veřejné nápisy}

Veřejných nápisů se dochovalo celkem devět, což představuje oproti 5. st. př. n. l. mírný nárůst. Narůstající číslo odpovídá i upevňování pozice politických autorit v regionu v důsledku stabilizace podmínek a nárůstu počtu nových osídlení.\footnote{Jedním z efektů dlouhodobé stabilizace politické situace může být i nárůst společenské komplexity. Tento jev se v rámci epigrafiky může projevit nejen nárůstem celkové epigrafické produkce, ale i vznikem nových funkcí a povolání, a tedy i jejich následnému objevení v textu nápisů.} Typologicky se jedná o šest dekretů, z čehož tři jsou honorifikační udílející pocty významným jedincům. Dále sem patří dvě nařízení, regulující obchodní výměnu, jeden seznam obsahující záznam pravděpodobně dlužných částek či vynaložených nákladů, jeden text náboženského charakteru na pomezí soukromého textu a dva nápis jiného či blíže neznámého charakteru.

Ve 4. st př. n. l. vidíme první náznaky využití nápisů v rámci regulace veřejného politického a ekonomického života: objevují se termíny jako je {\em polis}, {\em démos}, {\em búlé} apod. které poukazují na existenci samosprávných institucí. Tyto instituce dosud nebyly epigraficky potvrzeny z území Thrákie, nicméně pravděpodobně existovaly v daných městech již dříve. Termíny se objevují na nápisech pocházejících z řeckých komunit v Mesámbrii, Dionýsopoli, Zóné, Perinthu a Byzantiu, což svědčí o tradičním uspořádání těchto řeckých měst, o jejich politické autonomii a o fungujícím státním aparátu, schopném vydávat veřejná nařízení.\footnote{Příkladem může být nařízení {\em I Aeg Thrace} 3 z Abdéry datované do poloviny století regulující obchod s otroky a hospodářskými zvířaty.} O existenci diplomatických a ekonomických vztahů řeckých měst a thráckých aristokratů svědčí nápis {\em SEG} 49:911.\footnote{Tento nápis pochází pravděpodobně z prostředí řeckého {\em emporia} Pistiros v thráckého vnitrozemí a jedná se dohodu, v jejímž rámci se reguluje obchod mezi odryským panovníkem Kotyem a řeckými městy na pobřeží Egejského moře, a to konkrétně Maróneiou, Apollónií a Thasem. Dana (2015, 247-248) poukazuje na několik desítek {\em graffit} nalezených v okolí Pistiru, často nesoucí řecká jména či věnování řeckým božstvům, což může naznačovat trvalý pobyt osob řeckého původu (Domaradzka 1996, 2002, 2005, 2013). V okolí Pistiru se našlo několik dalších nápisů psaných osobami s řeckými jmény, mimo jiné i slavný dekret z Batkunu {\em IG Bulg} 3,1 1114, o němž hovořím níže. Vše tedy nasvědčuje, že {\em emporion} Pistiros bylo skutečně řeckou obchodní stanicí, jejíž část, nebo možná celá populace, byla řeckého původu.} Jedná se vůbec o první smlouvu ekonomického charakteru pocházející z území Thrákie, kde thrácký panovník vystupuje jako svrchovaná politická autorita, rovná autoritě řeckých měst na pobřeží. Dle současných interpretací šlo o regulaci již probíhajících ekonomických kontaktů, reagující na aktuální či minulé problémy ve snaze o nápravu a kodifikaci vzájemných vztahů (Velkov a Domaradzka 1996, 209-215; Bravo a Chankovski 1999, 279-290; Graninger 2013, 108-109). O vnitřním uspořádání území ovládaného Odrysy však z nápisu není bohužel možné vyčíst nic dalšího, ale vzhledem o ojedinělému výskytu podobného textu je možné říci, že thrácká aristokracie ve 4. st. n. l. nevyužívala nápisy k uplatnění politické moci a veřejné prezentaci svrchované autority stejným způsobem jako bylo obvyklé v řeckých městech na pobřeží. Naopak řecký charakter osídlení {\em emporia} Pistiros nasvědčuje, že thrácký panovník přistoupil na formu komunikace obvyklou pro řeckou komunitu, ať už pro usnadnění srozumitelnosti sdělení, či jako diplomatické gesto. Nápis je psán řecky, upravuje podmínky vzájemné interakce a zajišťuje ochranu řeckých obchodníků na území Thrákie, která je však garantována autoritou odryského panovníka. V případě tzv. pistirského nápisu se tedy jedná o vzájemnou dohodu řecké a thrácké politické autority, k jejíž prezentaci slouží výrazové a kulturně-společenské prvky pouze jedné ze zmíněných stran, a to strany řecké.

