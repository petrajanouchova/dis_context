
\subsection[veřejné-nápisy-4]{Veřejné nápisy}

Do této skupiny patří celkem tři nápisy, z nichž dva jsou honorifikační dekrety vydané autoritou řecké {\em polis}\footnote{Nápis {\em I Aeg Thrace} 400 pochází z antického města Drys na egejském pobřeží a jedná se o honorifikační dekret pro Polyárata, syna Histiáiova, kterému byla udělena privilegium proxénie, společně s udělením občanství, odpuštěním daní a právo volného vstupu do města. Obyvatele města Drys zastupují v tomto případě archónti, kteří reprezentují politickou autoritu daného města.} a jeden z nich pochází z thráckého vnitrozemí.\footnote{Honorifikační dekret {\em IG Bulg} 3,1 1114 pochází přímo z vnitrozemí z okolí moderní vesnice Batkun v regionu Filippopole, nedaleko řeckého {\em emporia} Pistiros. Text nápisu je znám bohužel jen částečně, nicméně z dostupného textu plyne, že občané neznámého města nechali neznámým bratrům postavit sochu a umístit jí ve svatyni Apollóna, a navíc se zavázali, že bratři budou vyznamenáni věncem při každé slavnosti konané pravděpodobně v dané svatyni. Nápis byl pravděpodobně nalezen u vesnice Batkun v kontextu venkovské svatyně věnované Asklépiovi {\em Zymdrénovi}, která se nachází na svazích pohoří Rodopy nedaleko Filippopole (Tsonchev 1941). Bravo a Chankowski (1990, 296-299) zpochybňují svatyni v Batkunu jako původní místo, kde nápis stál a jako alternativu navrhují kontext řecké komunity v nedalekém řecké {\em emporiu} Pistiros.} Není zcela jasné, kdo nápis vydal, ale usuzuje se, že se mohlo jednat o instituci reprezentující lid nedaleké Filippopole či Seuthopole (Archibald 2004, 886-889). Jedná se tak o první dochovaný dekret z vnitrozemí vydaný jinou politickou autoritou, než byl thrácký panovník, což může poukazovat na pomalu se měnící politické uspořádání v Thrákii.

