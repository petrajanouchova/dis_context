
\subsection[veřejné-nápisy-5]{Veřejné nápisy}

Ve 3. st. př. n. l. je možné sledovat narůstající využití psaného slova za účelem uplatňování autority existujících autonomních politických jednotek na území Thrákie. Celkem se dochovalo 36 veřejných nápisů, reprezentujících až 31 \letterpercent{} všech nápisů z daného období.\footnote{Oproti 4. st. př. n. l. je možné sledovat nárůst až na čtyřnásobek celkového počtu veřejných nápisů.} Veřejné nápisy až na tři výjimky pocházejí z řeckých měst na pobřeží, a z Mesámbrie na černomořském pobřeží dokonce až 18 exemplářů.\footnote{Tento vysoký poměr veřejných ale i soukromých nápisů z Mesámbrie z 3. st. př. n. l. může být dán jak stavem archeologických výzkumů, které se odehrávaly zejména v 60. až 80. létech 20. století, ale částečně i ekonomickým a politickým významem, který Mesámbria v průběhu 3. st. př. n. l. měla (Velkov 1969; 2005; Venedikov 1969; Ognenova-Marinova {\em et al.} 2005).} Dva nápisy z vnitrozemí pocházejí z původně thráckých osídlení, které hrály důležitou roli i za makedonských místodržících, konkrétně ze Seuthopole a Kabylé. Typologicky se jedná 31 dekretů vydaných politickou autoritou ({\em pséfisma}), z čehož devět bylo honorifikačních nápisů.

Dekrety jsou nejčastějších dochovaným typem veřejného nápisu, protože sloužily jako jeden z projevů suverenity politické autority, která je vydala.\footnote{Veřejné nápisy byly tesány do kamene a byly určeny pro veřejné vystavení. Jejich primárním účelem bylo jednak informovat o daném nařízení či usnesení politické autority, kterou v případě řeckých {\em poleis} byl lid, zastoupený termíny {\em démos} a {\em búlé,} a v případě thrácké aristokracie kmenový vůdce, označovaný termínem {\em basileus}. Předpokládá se, že každý si mohl nařízení přečíst, pokud toho byl schopen, a případně se na něj odvolat, pokud by docházelo k jeho porušování.} Zajišťovaly tak osobám, které spadaly pod vliv dané autority ochranu, základní práva a určitou prestiž v případě honorifikačních nápisů, výměnou za jejich loajalitu. Dále se částečně dochovalo jedno nařízení z Abdéry o platbách za dodávání pravdivých informací státu ({\em I Aeg Thrace} 2), jeden veřejný nápisy s náboženskou tematikou, seznam osob nejistého významu a jeden blíže neurčený nápis veřejného charakteru.

Ve zvýšené míře se na nápisech nachází administrativní termíny, a to celkem 25 termínů v 77 výskytech.\footnote{Předně se jedná o instituce a funkce zajišťující chod státního aparátu či symbolizující politickou autoritu samotnou jako je {\em démos}, {\em búlé}, {\em archón}, {\em stratégos}, {\em basileus} atd. Nejčastěji se opakujícím termínem byl {\em démos}, který se objevil 17krát, dále {\em polis} s 11 výskyty a {\em búlé} s šesti výskyty.} Častý výskyt tradičních termínů vychází z ustálené formy dekretů a formulí, které se používaly v celém řeckém světě ve velmi podobné formě. Texty nápisů jsou často velmi popisné a detailně zmiňují veškeré situace, v nichž je daný text nápisů platný a k čemu konkrétně dává pravomoci. Dále se zde vyskytuje celá řada specializovaných povolání, funkcí a referencí na existující společenskou hierarchii, což svědčí o narůstající komplexitě komunit, které nápisy vydávaly (Tainter 1988, 106-108).\footnote{Mezi objevující se nová povolání a funkce patří zejména funkce spojené s výkonem chodu státu a zajištění dodržování publikovaných nařízení, jako například {\em tamiás}, {\em argyrotamiás}, {\em gymnasiarchés}, {\em polítés}, {\em oikonomos}, {\em archón}, {\em stratégos}, {\em basileus}, {\em kéryx}, {\em theóros} či {\em presbys}.} Celá řada specializovaných termínů a formulí odkazuje na komplexní procedury spojené s pořizováním nápisů a jejich veřejným vystavováním. Jednalo se tedy pravděpodobně o ustanovenou proceduru, organizovanou obcí, podobně jak ji známe i z jiných řeckých komunit té doby. Zejména u honorifikačních nápisů z řeckých obcí je možné pozorovat výskyt ustálených formulí, které reflektují ustálené administrativní procedury a existující instituce pověřené epigrafickou produkcí. Ač je obsah formulí stejný, či velmi podobný, jejich konkrétní forma se liší město od města, často i v témže městě.\footnote{Text dekretů se typicky sestává z uvedení instituce udělující privilegia ({\em búlé}, {\em démos}, {\em polis} či kolektivní pojmenování občanů města), komu jsou udělena a proč, dále následuje výčet a specifika udělených privilegií a praktické informace o financování a veřejném vystavení nápisu. V textu je mnoho variant počínaje pořadím privilegií, udělenými poctami, způsobem veřejného vystavení, institucemi vydávajícím dekret, a dále i místní varianty použití určitých slovesných tvarů či infinitní větné konstrukce ({\em edoxe} vs. {\em dedochthai}, {\em edókan} vs. {\em dedosthai}). Příkladem může být udělení hostinného přátelství ({\em proxeniá}), které se objevilo celkem osmkrát.} Nejjednotnější formu mají nápisy z Odéssu, kde většina nápisů vycházela ze stejného vzoru, což poukazuje na zavedený systém byrokratických procedur v Odéssu. Nápisy z ostatních řeckých měst obsahují lehce pozměněné formulace, a lokální varianty, což nasvědčuje o velmi podobném politickém uspořádání a totožných procedurách v řeckých městech, avšak tento fakt poukazuje na absenci jednotné autority a normy, která by sjednocovala formální stránku nápisů. Řecká města tedy i ve 3. st. př. n. l. vystupovala a jednala jako autonomní politické jednotky, jejichž procedurální postupy byly do značné míry konzervativní.

Veřejné nápisy pocházející z thráckého kontextu přejímají jazyk a formu řeckých usnesení, ač jejich obsah nemá příliš paralel v celém antickém světě. Konkrétně se jedná o nápisy {\em IG Bulg} 3,2 1731 ze Seuthopole a {\em Kabyle} 2 z Kabylé, které byly v této době původně thráckými komunitami se silnou makedonskou přítomností, a to se odráží i na projevu politické autority na nápisech.\footnote{Nápis {\em IG Bulg} 3,2 1731 ze Seuthopole vydala vdova po thráckém panovníkovi Seuthovi III. Bereníké, která tak formou přísahy řeší nastalou situaci s paradynastou Spartokem z Kabylé, které se nachází zhruba 90 km po toku řeky Tonzos. Bereníké uzavírá se Spartokem dohodu, pravděpodobně za účelem uchování postavení a moci pro své syny Hebryzelma, Térea, Satoka a Sadala (Ognenova-Marinova 1980, 47-49; Velkov 1991, 7-11; Calder 1996, 172-173; Tacheva 2000; Archibald 2004. 886). Částečně dochovaný nápis {\em Kabyle} 2 (Velkov 1991, 11-12) byl vydán pravděpodobně v polovině 3. st. př. n. l. politickou autoritou řecké obce. Velkov navrhuje jako nejpravděpodobnější místo vzniku Mesámbrii a jednalo by i se tak o vzájemnou smlouvu mezi Kabylé a blízkou Mesámbrií na černomořském pobřeží. Po polovině století Kabylé i Mesámbria pravděpodobně spadaly pod vliv keltského panovníka Kavara, který v obou městech nechal razit mince pro svou potřebu (Draganov 1993, 75-86, 107).} Podobně jako v případě nápisu z Pistiru ze 4. st. př. n. l. se mohlo jednat o komunikační strategii thrácké aristokracie vůči řecky mluvícímu obyvatelstvu, či o snahu o zavedení zvyklostí typických pro řeckou komunitu, které však neměly dlouhého trvání. Vzhledem k ojedinělému užití nápisů ve veřejné funkci nejde příliš mluvit o dlouhodobém přejímání zvyků či organizace, ale spíše o dokumenty vzniklé jako reakce na aktuální politickou situaci a snahu se s ní vyrovnat způsobem obvyklým pro jednu zúčastněnou stranu. Absence institucí a úřadů typických pro řecké {\em poleis} poukazuje na specificitu thrácké politické autority a společenské organizace, kde hlavní roli hrál panovník a nejbližší okruh aristokratů, a nikoliv byrokratický aparát a státní instituce.

