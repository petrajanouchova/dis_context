
\subsection[veřejné-nápisy-7]{Veřejné nápisy}

Celkem 25 dochovaných veřejných nápisů, což oproti minulému století představuje mírný propad. Nápisy pocházejí výhradně z řeckých měst na pobřeží, případně byly nalezeny mimo oblast Thrákie, ale jejich obsah je zcela jasně s Thrákií spojuje. Nejvíce textů pochází z Maróneie s devíti exempláři. Celkem 80 \letterpercent{} veřejných nápisů představují dekrety vydané institucemi řecké {\em polis}, jako je {\em búlé} a {\em démos}.

Honorifikační nápisy jsou určeny význačným mužům řeckého, tak římského původu. Řím se objevuje v nápisech jako silný hráč na poli mezinárodní politiky, v mnoha případech vystupuje i jako spojenec řeckých států na egejském pobřeží, např. na nápise {\em I Aeg Thrace} 168. Pokud jde o zmínky o Thrácích, velmi záleží na konkrétním případě a záměru daného nápisu. Konkrétně na nápise {\em IK Sestos} 1 jsou Thrákové zmiňováni jako sousedé, kteří mohou ohrožovat bezpečí obyvatel Séstu. V dalším případě je thrácký panovník Kotys vnímán jako suverénní politická autorita {\em I Aeg Thrace} 5, stojící na podobné úrovni jako Řím a abdérský lid. V kontextu veřejných nápisů tedy nefigurují Thrákové jako barbaři či nepřátelé, ale spíše jako mocní spojenci a nevyzpytatelní sousedé autonomních řeckých měst.

Dochované administrativní termíny poukazují na nárůst vyskytujících se termínů na 35, které se dohromady objevují 117krát.\footnote{Nejpoužívanějšími termíny byly pojmy bezprostředně reprezentující politickou autoritu a vztahující k proceduře vydávání nařízení jako {\em démos}, {\em búlé}, {\em polis} a {\em pséfisma}. Objevují se nové funkce a instituce, které se nápisech dříve nevyskytovaly, jako například {\em synedrion}, {\em symmachiá}, {\em gymnasion}, {\em grammateus}, {\em chorériá}, {\em efébos} a {\em basilissa.} Existence {\em gymnasia} na území Thrákie je poprvé potvrzena epigraficky, a to v Apollónii a v Séstu, instituce {\em efébie} je potvrzena také v Séstu. Dále se zde vyskytují termíny s finanční a obchodní tematikou jako je {\em emporion}, {\em chrémata}, {\em analóma}, {\em trápeza}, {\em chóra}, {\em oikos} a {\em katoikia}. Zcela poprvé se objevuje termín {\em autokratór}, který je v následujících stoletích používán výhradně pro římského císaře, ale v tomto případě na nápise {\em IG Bulg} 1,2 388bis označuje velitele námořnictva z Istru, který pomohl Apollónii v době války mezi Apollónií a Mesámbrií.} Nárůst celkového počtu termínů a objevení nových institucí a funkcí související s organizací politické autority a jejím výkonem naznačuje, že docházelo k nárůstu společenské komplexity a organizace v rámci řeckých komunit na pobřeží. Nárůst dochovaných ujednání mezi politickými autoritami regionu naznačuje i nárůst diplomatických kontaktů, nebo alespoň jejich kodifikace a častější zaznamenávání na trvalé médium nápisu.

