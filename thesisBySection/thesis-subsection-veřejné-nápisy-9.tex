
\subsection[veřejné-nápisy-9]{Veřejné nápisy}

Veřejných nápisů se dochovalo celkem 17, což představuje mírný pokles oproti 2. st. př. n. l. Nápisy pocházejí výhradně z pobřežních komunit, nejvíce jich bylo nalezeno na černomořském a egejském pobřeží: čtyři v Odéssu a Maróneii, tři v Mesámbrii a dále vždy po jednom nápise. Místa s největším počtem veřejných nápisů jsou shodná s největšími producenty soukromých nápisů.

Nápisy mohou být nepřímým důkazem o politických událostech a o zvyšujícím se vlivu Říma v regionu, avšak stále za udržení tradičních zvyklostí a procedur spojených s vydáváním veřejných nápisů. Nápisy obsahují 14 hledaných termínů v 35 výskytech, což je představuje výrazný pokles oproti předcházejícímu období. Nejvíce se vyskytuje termín {\em démos} v 10 případech, dále {\em búlé} a {\em polis} se čtyřmi výskyty a {\em basileus} a {\em pséfisma} se třemi výskyty. Dochází i k poklesu celkového počtu termínů označujícího společenské funkce a instituce. Většinu veřejných nápisů představují dekrety vydávané politickou autoritou, což v tomto případě byly instituce řecké {\em polis}, nicméně nápisy zmiňují představitele římské říše a thrácké krále jako rovnocenné partnery a uznávají jejich autoritu.\footnote{Příkladem mohou být nápisy {\em IG Bulg} 1,2 13, {\em IG Bulg} 1,2 43, {\em IG Bulg} 1,2 314a. Jedním takovýmto příkladem je dekret {\em IG Bulg} 1,2 314 z Mesámbrie, díky němuž se dozvídáme, že makedonský místodržící M. Terentius Lucullus v roce 72/71 př. n. l. ustanovil vojenskou posádku na území Thrákie a upevnil tak římský vliv v oblasti (Lozanov 2015, 77). O délce trvání římské přítomnosti a o míře vlivu však nemáme žádné další informace, nicméně je to jeden z prvních náznaků římského zájmu v oblasti, který se projevil i v epigrafice.}

Thrácká aristokracie i nadále hraje důležitou politickou roli v regionu: na nápisech se vyskytují celkem čtyři muži, nesoucí označení král Thráků: Kotys, syn Rháskúporida, Rhoimetalkás, Sadalás a Burebista. Z historických zdrojů víme, že v 1. st. př. n. l. došlo ke sjednocení Thráků pod kmenem Sapaiů, a tedy i k upevnění pozice thráckého panovníka se sídlem v Bizyi v jihovýchodní Thrákii (Lozanov 2015, 78). Rozmístění nápisů odpovídá i přibližné rozmístění sféry vlivu jednotlivých dynastií: Sapaiové na jihovýchodě, Odrysové ve středu a Getové na severu Thrákie.\footnote{Do dynastie Sapaiů spadají Rhaskúporis a Kotys s nápisy pocházejícími z Bizóné, Maróneie a lokality Dydimoteichon. Sadalás pravděpodobně patří do dynastie Odrysů a vyskytuje se na nápise {\em IG Bulg} 1,2 43 z Odéssu a Burebista, zmíněný na nápise {\em IG Bulg} 1,2 13 z Dionýsopole pochází z kmene Getů.}

