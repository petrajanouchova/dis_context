
\subsection[veřejné-nápisy]{Veřejné nápisy}

Celkem se dochovalo 717 veřejných nápisů, což představuje 15 \letterpercent{} všech dochovaných nápisů. Veřejné nápisy, tedy nápisy zhotovené politickou a administrativní autoritou za účelem organizace společnosti a udržení společenského řádu, je taktéž možné rozdělit podle jejich obsahu do několika částečně se prolínajících skupin. Jak plyne z dat uvedených v tabulce 5.04 v Apendixu 1, nejpočetnější skupinou jsou honorifikační nápisy, tedy nápisy vydané v rámci samosprávné jednotky k poctě jednotlivce či skupiny lidí, které představují 41,56 \letterpercent{} veřejných nápisů a 6,39 \letterpercent{} všech nápisů. Další početnou skupinou jsou státní dekrety, tedy nařízení vydaná politickou autoritou, která mohou mít normativní charakter, či se jedná o veřejně vydaná ustanovení a smlouvy (2,35 \letterpercent{} všech nápisů). Dále do kategorie veřejných nápisů spadají i seznamy nejrůznějšího charakteru (1,37 \letterpercent{} všech nápisů) a náboženské texty (1,14 \letterpercent{} všech nápisů), veřejné nápisy nespadající do předchozích kategorií (2,94 \letterpercent{} všech nápisů) a veřejné nápisy, které nebylo možné přesněji určit (1,74 \letterpercent{} všech nápisů).

Jazyk veřejných nápisů je do značné míry konzervativní: ustálené formule a termíny se opakují v nápisech stejné funkce. Celkem 532 nápisů obsahuje hledané administrativní termíny, a to celkem v 1362 výskytech, což představuje zhruba 2,6 termínu na nápis. Mezi nejčastější termíny patří {\em démos} s 217 výskyty, {\em polis} se 158, {\em búlé} se 153, {\em autokratór} se 113, {\em kaisar} s 84, {\em presbeutés} a {\em antistratégos} se 79, {\em hégémón} se 41 a {\em hypatos} se 41 výskyty. V průběhu doby dochází k nárůstu používání těchto termínů, s čímž souvisí jednak narůstají epigrafická produkce, ale i zintenzivnění společenské organizace a objevení nových administrativních institucí v době římské.

I přes jistou ustálenost formy je možné sledovat vývoj regionálních variant textu, což by poukazovalo na míru autonomie regionů, a to zejména v předřímské době. Vliv jednotné autority, který by měl za následek sjednocování formy veřejný nápisů, není v této době patrný. Pokud docházelo k ovlivňování a výpůjčkám v rámci společenské organizace, dělo se tak místně a v relativně malém měřítku. Naopak v římské době dochází k určitému sjednocení obsahu i formy veřejných nápisů na celém území Thrákie, patrně pod vlivem jednotné administrativy a rozsáhlého byrokratického aparátu.\footnote{Podrobně se vývojem veřejných nápisů v jednotlivých stoletích zabývám v kapitole 6, o jejich rozmístění v krajině pak v kapitole 7.} Příkladem mohou být například milníky, které si zachovaly téměř identickou podobu nejen v Thrákii, ale například i v sousední Bíthýnii, a i jiných místech římské říše (pro Bíthýnii French 2013){\bf .}

