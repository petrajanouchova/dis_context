
\subsection[vnitrozemské-komunity]{Vnitrozemské komunity}

V případě thrácké komunity se v období od 7. do 1. st. př. n. l. jednalo o ojedinělé a omezené pokusy o zavedení řeckých epigrafických zvyklostí, typické pro komunity řízené autoritativními jedinci z řad aristokracie, jako v případě Seuthopole, či obchodní komunity se smíšeným obyvatelstvem, které však měly omezený vliv a trvání, jako v případě Pistiru či Hérakleie Sintské. Thráčtí aristokraté využívali písmo utilitárně, tedy k označení vlastnictví a účelu souvisejícího objektu či jako způsob komunikační strategie směrem k řecky mluvícím komunitám.

První skupina nápisů z vnitrozemí je spojena s aktivitami thrácké aristokracie a nachází se většinou na předmětech z drahých kovů či souvisejících s funerálním ritem. Thráčtí aristokraté nepřijali zvyk zhotovování a vystavování náhrobních stél tak, jak bylo běžné v řeckých komunitách na pobřeží, ale většina funerálních nápisů pocházela z interiéru hrobek, či z předmětů, které se staly součástí pohřební výbavy. Oproti řeckým komunitám byly tyto nápisy určeny jen nejbližšímu kruhu aristokratů, nebyly veřejně přístupné a funerální funkci získaly až druhotně. Tento fakt vypovídá jednak o odlišném společenském uspořádání thrácké a řecké společnosti, ale i o odlišném přístupu k písmu a zaznamenávání informací. Písmo zde nesloužilo k šíření informací mezi širokou veřejnost jako v případě nápisů z řeckých měst, ale bylo využíváno čistě pro potřeby úzkého okruhu aristokratů a sloužilo jako prostředek zvýšení společenského postavení.

Druhá skupina nápisů z thráckého vnitrozemí pochází z kontextů řeckých, makedonských či smíšených komunit, jako v případě řeckého {\em emporia} v Pistiru, makedonského {\em emporia} v Héraklei Sintské, hellénistické rezidence v Seuthopoli či hellénistického města v Kabylé. V těchto případech se užití písma podobá více zvyklostem zaznamenaných v řeckých komunitách na pobřeží. Vyskytují se zde nápisy na kamenných stélách, které byly veřejně vystavované a byly jak soukromého, tak veřejného charakteru. Navíc z těchto míst pochází i graffiti na keramice, dokazující určitý stupeň gramotnosti místního obyvatelstva. Osobní jména na nápisech jsou převážně řeckého původu či se jedná o thrácká jména pravděpodobně aristokratického původu. Charakter dochovaných nápisů naznačuje přejímání určitých zvyklostí typických pro řeckou kulturu a společenskou organizaci, avšak také naznačuje jejich adaptaci na místní podmínky. Epigrafická produkce tak v případě soukromých nápisů sloužila pro interní potřeby dané komunity a nedocházelo k rozšíření zvyklostí mimo bezprostřední sféru vlivu daného osídlení. V případě veřejných nápisů, které sloužily jako prostředek kodifikace diplomatických vztahů, je pak konkrétní forma ovlivněna zvolenou komunikační strategií všech zúčastněných stran a snahou nalézt společné vyjadřovací prostředky. Vzhledem k propracovanosti systému smluv a ustanovení v řeckém světě je pak zcela logické, že se zúčastněné strany rozhodly využít ustálenou formu a částečně i obsah těchto smluv. Z těchto důvodů tzv. pistirský či seuthopolský nápis vycházejí z řeckých vzorů, avšak reagují na konkrétní politickou situaci a tomu odpovídá i jejich unikátní obsah, který do značné míry zdůrazňuje autonomii a suverenitu thrácké aristokracie vůči řeckým komunitám.

Z podrobné analýzy dochovaných nápisů plyne, že epigrafická produkce v předřímské době pochází z řeckých komunit a ve velmi omezené míře i z kontextu thrácké aristokracie, která adoptovala prvky nápisné kultury pro své vlastní účely a dala jim unikátní charakter. Důsledkem kontaktu s řeckou kulturou bylo objevení nápisů v thráckém kontextu jakožto prostředku komunikace vně i uvnitř komunity, avšak za zachování tradičního charakteru thrácké společnosti a platných norem chování.

