
\subsection[zastoupení-pohlaví-v-závislosti-na-kulturním-prostředí]{Zastoupení pohlaví v závislosti na kulturním prostředí}

Pokud se podíváme na zastoupení pohlaví v jednotlivým kulturních prostředích, v řeckém prostředí je zastoupeno největší procento ženských jmen oproti mužským jménům (13,85 \letterpercent{}). Mužská jména však představují více jak 85 \letterpercent{} všech jmen jak v řeckém, římském, tak thráckém prostředí.

Jak je patrné z dat v tabulce 5.08 a 5.09 v Apendixu 1, ženská jména v řeckém prostředí figurují na místě rodičů častěji (1,67 \letterpercent{}), než v případě římského a thráckého prostředí. To poukazuje na větší důležitost mateřské linie v rámci řecké společnosti. Otcovská linie však i přesto bývá pro srovnání uváděna v rámci identifikace jednotlivce zhruba 20krát častěji, než je tomu v případě matky (36,26 \letterpercent{}). Naopak jako partnerská jména jsou v řeckém prostředí uváděna častěji mužská jména, než ženská (2,81 \letterpercent{} oproti 1,58 \letterpercent{}), což opět poukazuje na důležitější roli muže v rámci uspořádání řecké společnosti, která je nicméně minimální i v řeckém prostředí. Vysoký počet jmen rodičů poukazuje na fakt, že udávání {\em patronymika} bylo v řeckém prostředí velmi časté, vzhledem k tomu, že 78 \letterpercent{} všech individuálních jmen bylo doplněno právě jménem otce.

V římském prostředí roli při identifikaci jednotlivce hrají osobnost konkrétního člověka ({\em tria nomina}) a linie předků nemá tak význačnou pozici jako v řeckém prostředí (0,41 \letterpercent{} pro ženy a 7,26 \letterpercent{} pro muže). Role partnera při identifikace jednotlivce je v římském prostředí taktéž nižší, než v řeckém prostředí (1 \letterpercent{} pro ženy a 1,03 \letterpercent{} pro muže). Thrácké onomastické prostředí do velké míry přejímá zvyky jak řeckého, tak římského prostředí: poměry užívání individuálních jmen a jmen předků při identifikaci osoby mají v řeckém a thráckém prostředí velmi podobné hodnoty - jedinci nesoucí thrácká jména udávají svůj původ pomocí jmen otců a prarodičů. Stejně tak ale jedinci identifikují sama sebe pomocí kombinace tří jmen, z nichž jedno, případně dvě jsou římského původu. Výjimkou tvoří jména partnerů, kdy thrácká ženská jména jsou ze všech kulturních prostředí nejčastěji užívána v roli partnerek (2,38 \letterpercent{}), což může být důsledek smíšených partnerských svazků mezi Thrákyněmi a Řeky.

