
\subsection[zeměpisné-uspořádání-nápisů]{Zeměpisné uspořádání nápisů}

Nápisy pocházejí z oblasti jihovýchodní části dnešního Balkánského poloostrova, a to konkrétně z území dnešních států Bulharska, Řecka a Turecka. Velká část nápisů byla nalezena, či editory zařazena do oblastí ve vzdálenosti do 20 km od mořského pobřeží, a dále z vnitrozemských městských center, která se nacházejí poblíž toků řek Hebros, Strýmón a Tonzos, případně v blízkosti hlavních cest v nížinatých oblastech. Z hornatých oblastí vnitrozemské Thrákie pochází jen menší část nápisů, a to zejména z oblasti pohoří Stara Planina, Rodopy, Pirin a Rila.\footnote{Více o zeměpisném rozložení nápisů hovořím v kapitole 7.}

Dochované nápisy jsou do značné míry výsledkem náhodných nálezů v průběhu posledních dvou století a systematického archeologického výzkumu zhruba od poloviny 20. století. Archeologické výzkumy neprobíhaly na všech místech stejnou měrou a ve stejném rozsahu, a proto i počty dochovaných nápisů mnohdy odpovídají stavu archeologického výzkumu na území Thrákie. Příkladem může být dobrý stav poznání v černomořských řeckých koloniích, které bylo možné prozkoumat velmi detailně v posledních 50 letech zejména díky prudkému nárůstu turistického ruchu a s ním spojených záchranných výzkumů (Velkov 1969; Ognenova-Marinova {\em et al.} 2005; Baralis a Panayotova 2015). Opačným příkladem je oblast turecké Thrákie, která je prozkoumána jen velmi málo. Vzhledem k složité politické a ekonomické situaci jsou data z části evropského Turecka nedostupná, až na oblast okolo Perinthu, Byzantia a řeckých měst na Thráckém Chersonésu, které se podařilo získat díky výzkumu a spolupráci německých a tureckých vědců (Krauss 1980; Lajtar 2000; Sayar 1998). Ucelená publikace nápisů z egejské Thrákie je taktéž poměrně nedávnou záležitostí (Loukopoulou {\em et a}l. 2005), avšak zde jsou nálezy nápisů doplněny dlouhodobými archeologickými výzkumy.

Mapa 5.01 v Apendixu 2 vyznačuje oranžovou barvou oblast, kterou pokrývají v databázi zpracované nápisy. Naopak místa bílá jsou zároveň místy, pro něž nejsou dostupné systematicky zpracované korpusy nápisů. Tato oblast bez nápisů se vyznačuje poměrně nepřístupným terénem a pravděpodobně byla málo zalidněná jak v antice, tak dnes. Obecně se předpokládá, že z této části Thrákie pochází jen velmi malý počet nápisů, které by tak výrazněji neměly zasáhnout do celkového obrazu epigraficky aktivní společnosti antické Thrákie, pokud dojde v budoucnu k jejich publikování.

Nápisy se podařilo lokalizovat dle dat dostupných v epigrafických korpusech, a to s následující přesností: do 1 km 1540 nápisů (33 \letterpercent{})\footnote{Následující statistiky vycházejí z celkového počtu 4665 nápisů, což představuje 100 \letterpercent{} analyzovaného souboru. V určitých případech může dojít při konečném součtu všech položek k hodnotě větší či menší než 100 \letterpercent{}. Pokud je součet větší než 100 \letterpercent{}, daná položka mohla mít více hodnot, tj. nápis mohl být určen například jako funerální a honorifikační zároveň, pokud svými charakteristickými rysy spadal do obou kategorií, V tomto případě pak konečný součet má hodnotu více jak 100 \letterpercent{}, protože položky obsahují více než jednu hodnotu. V případě statistik, kde je konečná hodnota menší než 100 \letterpercent{}, pak tento rozdíl náleží položkám, které se nepodařilo určit, jak z důvodu jejich fragmentárnosti, nedostupnosti informace apod.}, s přesností do 5 km 2323 nápisů (50 \letterpercent{}), s přesností do 20 km 590 nápisů (13 \letterpercent{}), a s přesností nad 20 km 212 nápisů (4 \letterpercent{}). Udávaná poloha většiny nápisů byla editory označena jako místo primárního nálezu (4305 nápisů, tedy 92 \letterpercent{}), tedy místo, kde byl nápis umístěn, když sloužil své primární funkci. Archeologický kontext místa, v němž byl nápis nalezen, udávají autoři korpusů zhruba u třetiny nápisů: funerální 251 nápisů, sídelní 222 nápisů, z čehož 53 bylo nalezeno v rámci obchodního kontextu daného osídlení, např. na agoře, na foru, v emporiu, v přístavu apod.; jiný kontext 16, rituální/náboženský kontext 779 nápisů. Pouze u 380 nápisů (8 \letterpercent{}) udávají místo nálezu jako sekundární kontext, což znamená, že nápis byl použit při stavbě mladší stavby, či autor korpusu uvádí, že byl nápis přesunut z původního místa, a přestal sloužit primární funkci.

