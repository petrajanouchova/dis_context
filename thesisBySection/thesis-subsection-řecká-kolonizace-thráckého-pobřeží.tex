
\subsection[řecká-kolonizace-thráckého-pobřeží]{Řecká kolonizace thráckého pobřeží}

První doložená řecká osídlení trvalého charakteru se na thráckém pobřeží objevila v průběhu 7. st. n. l. a během dvou následujících století se řecké kolonie rozšířily téměř na celé pobřeží Egejského, Marmarského a Černého moře (Isaac 1986; Archibald 1998, 32-47). Kolonizační aktivity vycházely vždy z iniciativy jednotlivých obcí či ze spojeného úsilí několika obcí, ale v žádném případě se nejednalo o koordinovanou aktivitu, kterou by řídila jedna politická autorita na celém území pobřežní Thrákie.

Jako hlavní důvody řecké kolonizace literární prameny uvádějí bohatství Thrákie, zejména nerostné zdroje a úrodnou půdu (Hom. {\em Il}. 13.1-16; 10.484; Archilochos, Diehl frg. 2 a 51; Hdt. 1.64.1; 5.23.2; Isaac 186, 282-285; Tiverios 2008, 80). Řecká města byla zakládána v blízkosti mořského pobřeží, často nedaleko vodních toků a v blízkosti zdrojů nerostných surovin. Koncem 6. a v průběhu 5. st. př. n. l. řecká města začala razit zlaté a zejména stříbrné mince, na něž získávala materiál těžbou v místních dolech. Poměrně záhy začali mince razit i jednotliví thráčtí panovníci (Archibald 2015, 912). Mince byly raženy dle řeckých mincovních standardů, což usnadnilo kontakty ekonomického charakteru mezi řeckými městy a thráckými aristokraty (Tiverios 2008, 128). Aby dále podpořily obchodní aktivity, řecké kolonie zakládaly svá vlastní osídlení, {\em emporia}, jejichž primární funkcí bylo zajišťovat obchod s Thráky, případně obstarávat zemědělské produkty či se starat o těžbu nerostných surovin (Tiverios 2008, 86-91). Většinou se tato {\em emporia} nacházela v okolí nerostných zdrojů a řek. Nicméně osídlování neprobíhalo pouze na pobřeží, ale máme i archeologické důkazy o vznikajících obchodních stanicích ve vnitrozemí, které umožňovaly přímý kontakt a usnadňovaly obchod s pobřežními oblastmi, např. na řece Hebru ležící {\em emporion} Pistiros (Bouzek et al. 1996; 2002; 2007; 2010; 2013; 2016; Bouzek a Domaradzka 2011). Vnitrozemská Thrákie tak díky těmto stanicím snáze získávala předměty řecké provenience, jako např. keramika, transportní amfory, víno, olej, a to vše pravděpodobně výměnou za nerostné suroviny, jichž bylo v Thrákii dostatek.

Povaha kontaktů nově příchozího a původního obyvatelstva se velmi lišila dle konkrétní situace. V souvislosti se zakládáním nových sídlišť a obchodních kontaktů pravděpodobně docházelo i ke sporům o území a životní prostor. Řečtí kolonizátoři se často usazovali v místech již existujících thráckých osídlení, což jistě vyvolávalo nevoli thráckého obyvatelstva. Z řecky psané literatury víme konfliktech mezi Thráky a řeckými kolonizátory z Paru, jak je zmiňuje např. řecký básník Archilochos (Diehls frg. 2, 6 a 51) či o neúspěchu první kolonie v Abdéře způsobeném jak nepřátelsky nalazenými Thráky, tak i nepříznivými zeměpisnými podmínkami a šířením nemocí (Hdt. 1.168; Graham 1992, 46-48; Loukopoulou 1989, 185-190). Na jiných místech dochované archeologické nálezy nicméně nasvědčují spíše poklidnému soužití, či dokonce ekonomické kooperaci mezi Thráky a Řeky žijícími na pobřeží (Archibald 1998, 47, 73; Ilieva 2011, 25-43). V některých případech mohlo dokonce docházet k soužití, která se nevyhnutelně projevilo i na archeologickém materiálu. Epigrafická produkce, která by dokumentovala povahu raných kontaktů mezi Thráky a Řeky, buď vůbec nevznikla, nebo se do dnešní nedochovala, a veškeré nápisy pochází až z doby stabilních a fungujících řeckých komunit na thráckém území, tedy z doby několik desítek let od založení kolonií.\footnote{Podrobněji v kapitole 6.}

\subsubsection[geografický-rozsah-řecké-kolonizace]{Geografický rozsah řecké kolonizace}

Řecká kolonizace představovala aktivitu jednotlivých obcí, čemuž odpovídá i rozdělení sfér vlivu a silné regionální vazby nově vniklých osídlení. V egejské oblasti docházelo zejména k osídlení z přilehlého Thasu a Samothráké\footnote{Mezi nejaktivnější kolonizátory v egejské oblasti patřili osadníci z ostrova Paros a Thasos, kteří se zaměřili především na protilehlé thrácké pobřeží v okolí řeky Strýmón, kde se nacházely zdroje nerostných surovin (Zahrnt 2015, 36). Mezi nejvýznamnější thaské kolonie patří Neápolis, Oisímé a Strýmé-Molyvoti, které zajišťovaly jak obchodní výměnu, ale umožňovaly přístup k úrodné thrácké půdě (Tiverios 2008, 80-85).} a dále z iónských a aiolských maloasijských obcí. Oproti tomu v oblasti Propontidy a Černého moře zásadní roli zaujímaly kolonie dórské Megary a iónského Mílétu a Samu.

Nejvýznamnějším příkladem maloasijské iónské kolonizace je založení Abdéry na egejském pobřeží poblíž řeky Néstos. Abdéra byla založena zhruba v polovině 7. st. př. n. l. byla z maloasijských Klazomen, jejíž obyvatelé se potýkali s odporem místních thráckých kmenů a nemocemi, díky čemuž byla Abdéra pravděpodobně opuštěna. V polovině 6. st. př. n. l. došlo k její opětné kolonizaci z~Teu, a nakonec se jim podařilo Thráky přemoci a z Abdéry se později stala jedno z nejvýznamnějších ekonomických a kulturních center egejské Thrákie a jedním z prvních producentů nápisů (Hdt. 1.168; Isaac 1986, 73-89; Graham 1992, 46-53; Loukopoulou 2004, 872-875; Tiverios 2008, 91-99).\footnote{Dle odhadů mohla populace Abdéry dosahovat až 100 000. Jednalo se tak o jedno z největších měst z oblasti egejské Thrákie (Loukopoulou 2004, 873). Dimitri Samsaris (1980, 166-167) řadí Abdéru na stejnou úroveň jako města Amfipolis, Maróneiu, Byzantion, Perinthos a Thasos, tedy s populací o velikosti řádově několika desítek tisíc obyvatel.} Zhruba od r. 540 př. n. l. Abdéra razila vlastní mince z lokálních zdrojů stříbra, k nimž měla pravděpodobně neomezený přístup. Nálezy abdérských mincí v depotech mincí po celém Středomoří a na území Thrákie dosvědčují, že mince byly určeny jak pro dálkový, tak i pro místní obchod (May 1996, 1-2, 16-17, 59-66; Sheedy 2013, 40-41; Paunov 2015, 267). Vzhledem ke svému důležitému postavení v průběhu 5. st. př. n. l. hrála Abdéra poměrně zásadní roli při zprostředkování diplomatických kontaktů mezi Athénami a odryským panovníkem, které vyústily v uzavření spojenectví v roce 431 př. n. l. (Thuc. 2.29; 2.95-101; Janouchová 2013, 96-97; Sheedy 2013, 46). Abdéra hrála důležitou roli nejen při zprostředkování diplomatických kontaktů mezi Řeky a Thráky, ale na jejím území a v blízkém okolí docházelo i ke kontaktům běžné populace a prolínání náboženských představ.\footnote{Příkladem setkávání dvou tradic je kult řeckého Apollóna, který nese místní thrácké přízvisko {\em Derénos}. Tento kult je doložen nejen z Abdéry, ale v římské době se dokonce objevil v thráckém vnitrozemí (Graham 1992, 67-68; Janouchová 2016, 93-95). Abdéra tedy hrála poměrně zásadní roli zprostředkovávání kontaktů mezi Řeky a Thráky jak na politické, ekonomické, tak kulturní úrovni.}

Neméně významným regionálním centrem iónského původu byla Maróneia, kterou založil v polovině 7. st. př. n. l. Chios, pravděpodobně na místě dřívějšího thráckého osídlení. Maróneia a okolní oblast hory Ismaros se stala známou pro své víno a díky strategicky situovanému přístavu si zajistila výhodné ekonomické postavení a četné obchodní kontakty jak se Středozemím, tak s vnitrozemskými Thráky (Archilochos, Diehl frg. 2; Isaac 1986, 111-115; Tiverios 2008, 99-104).

Území na egejském pobřeží mezi řekami Néstos a Hebros spadalo do sféry vlivu Samothráké, která byla sama pravděpodobně kolonií iónského Samu a neznámého aiolského města. V 6. st. př. n. l. byla na protilehlé thrácké pevnině ze Samothráké založena města Drys a Zóné-Mesámbria.\footnote{Lokalita Mesámbria, či Mesémbria se vyskytuje jak na pobřeží Egejského, tak Černého moře. Mesámbria v egejské oblasti je známá z literárních zdrojů (Hdt. 7.108.2, Steph. Byz. 446.19-21) a bývá ztotožňována s další lokalitou Zóné, Orthagoria, či Drys (Loukopoulou 2004, 880). Lokalita Mesámbria z oblasti Černého moře je taktéž známá z literárních zdrojů (Hdt. 4.93, 6.33.2) a dnes se nachází pod moderním městem Nesebar v Bulharsku.} Mezi Thráky a Řeky v této oblasti docházelo k četným kontaktům, které dle dostupných archeologických pramenů nebyly násilného charakteru a měly za následek jak mísení obyvatelstva a náboženských zvyklostí, ale i například technologií zpracování kovu a výroby keramických nádob (Ilieva 2007, 221; 2011, 36-38; Tiverios 2008, 107-118; Kostoglou 2010, 180-185).

Východně od této oblasti ležel aiolský Ainos, který byl pravděpodobně založen na místě dřívějšího thráckého osídlení. Ainos se nacházel v ústí řeky Hebros, která spojovala egejské pobřeží s thráckým vnitrozemím, a stal se tak ekonomickým a kulturním centrem pro celý region s napojením na thráckou dynastii Odrysů (Isaac 1986, 140-156; Tiverios 2008, 118-120).

Oblast Marmarského moře se stala středem zájmu dórské Megary a iónského Samu. Megara založila již v 7. st. př. n. l. města Sélymbrii a Byzantion, která sehrála velmi důležitou roli v kulturně-politickém vývoji dalších století. Město Byzantion těžilo zejména ze své výhodné pozice, která mu umožňovala kontrolovat bosporskou úžinu, ale i přesto se nevyhnulo občasným útokům okolních thráckých kmenů (Isaac 1986, 230-231). Iónský Samos pak na začátku 6. st. př. n. l. na místě existujících thráckých osad založil Perinthos a Bisanthé, z nichž zejména Perinthos hrál velmi důležitou roli v pozdějších stoletích, kdy se stal prvním sídlem místodržícího římské provincie {\em Thracia} (Isaac 1986, 198-201).\footnote{Isaac 1986, 205: ve 3. či 4. st. n. l. byl Perinthos přejmenován na Hérakleiu. V epigrafických pramenech se vyskytují obě dvě jména, a proto se držím pojmenování Perinthos (Hérakleia), a od konce 3. st. n. l. Hérakleia (Perinthos).}

Oblast Thráckého Chersonésu byla pro svou strategickou pozici a úrodnou půdu osídlena nejpozději na konci 7. st. př. n. l. aiolskými osadníky. Jména nejvýznamnějších měst z této oblasti jsou Séstos, Alopekonnésos či Madytos (Loukopoulou 2004, 900). V 6. st. př. n. l. se o oblast Thráckého Chersonésu velmi zajímaly Athény, které sem na určitou dobu dosadily vojenské posádky a snažily se jak silou, tak spoluprací s místními thráckými kmeny ovládnout toto strategické území (Tiverios 121-124).

Na pobřeží Černého moře založil řecký Mílétos celou řadu kolonií již v 6. st. př. n. l., z nichž se poměrně záhy stala regionální ekonomická a kulturní centra, jako v případě Apollónie Pontské a později Odéssu. Nejvýznamnější dórskou kolonií v této oblasti se stala Mesámbria založená Megarou, která zaujímala výsadní postavení ve 3. st. př. n. l.\footnote{O thráckém původu Mesámbrie svědčí přípona -{\em bria}, označujícící osídlení v thráčtině (Venedikov 1997,76).} Černomořské kolonie byly pravděpodobně v úzkém kontaktu s Thráky, jak dokládají nejen thrácká jména vyskytující se na pohřebních stélách, ale i epigraficky potvrzené diplomatické kontakty mezi městy a Odrysy či přítomnost thrácké keramiky a kovových předmětů thrácké provenience na území řeckých měst.\footnote{Náboženství a kulty v černomořských koloniích do velké míry vycházely z thráckých vzorů, nicméně první doklady o tomto náboženském synkretismu máme až z hellénismu a římské doby (Isaac 1986, 241-257).}

Povaha kontaktů mezi Řeky a Thráky záležela vždy na konkrétní situaci a rozhodně nelze z dochovaných materiálních dokladů vyvozovat, že Thrákové k příchozím Řekům zaujímali automaticky nepřátelský postoj, jak by mohly naznačovat literární zmínky u básníka Archilocha.\footnote{Zejména Diehls frg. 2, 6 a 51.} Prvotní kontakty mohly být nepřátelské, zejména proto, že Řekové se většinou snažili založit kolonii v místě existujícího thráckého osídlení. Pravděpodobně také docházelo ke konfliktům o vlastnictví zemědělské půdy a zdrojů nerostných surovin, což je však velmi špatně doložitelné v materiálních pramenech. Pokud jde o období následující po primární kolonizaci, archeologické studie keramické a metalurgické produkce napovídají, že se vzájemné kontakty Řeků a Thráků nesly v duchu ekonomické spolupráce a relativně poklidného soužití, alespoň v nejbližším okolí řeckých měst (Ilieva 2007, 221; 2011, 36-38; Kostoglou 2010, 180-185).

