\environment ../env_dis
\startcomponent section-historický-nástin-vývoje-regionu
\section[historický-nástin-vývoje-regionu]{Historický nástin vývoje regionu}

Území Thrákie bylo v 1. tis. př. n. l. osídleno thráckými kmeny, jejichž jména známe často jen z řeckých literárních zdrojů. Thrákové bohužel nezanechali písemné prameny a z thráčtiny se do dnešní doby zachovaly jen místní názvy, osobní jména a několik nápisů nejasného charakteru (Dana 2015, 244-245). Proto se při výkladu thráckých dějin z této doby musíme spoléhat převážně na interpretaci archeologických pramenů a na občasné zmínky v \index{literární prameny}literárních pramenech.

\subsection[thrákie-v-1.-tis.-př.-n.-l.]{Thrákie v 1. tis. př. n. l.}

Materiální kultura první poloviny 1. tis. př. n. l. vykazuje velké množství lokálních variant, které poukazují na nejednotnost thráckého etnika, což odpovídá spíše regionálnímu rozdělení společnosti dle jednotlivých kmenových uskupení (Kostoglou 2010, 180-185; Graninger 2015, 25). Členění území dle kmenovému principu odpovídá i absence osídlení městského typu, a naopak velké množství decentralizovaných menších sídel se zaměřením na zemědělskou produkci, doplněné opevněnými sídly kmenových vůdců a opevněných strážních sídel v horských oblastech (Theodossiev 2011, 15-17; Popov 2015, 111-114). Pro thráckou materiální kulturu jsou typické náhrobní mohyly, z hlíny navršené stavby překrývající hrobky, které zpravidla obsahovaly bohatou pohřební výbavu a proto se usuzuje, že patřily thráckým aristokratům (Theodossiev 2011, 20-21). Dalšími typickými stavbami jsou dolmeny a svatyně tesané do skal, které se nacházejí zejména v hornatých a hůře přístupných částech jihovýchodní Thrákie. Jejich funkce souvisela pravděpodobně s rituálními aktivitami thráckého obyvatelstva, které byly úzce svázány s přírodními fenomény (Nekhrizov 2015, 126-141).

Oblasti obývané Thráky se staly předmětem zájmu řeckých obcí zejména od 7. st. př. n. l., nicméně jak dokazují archeologické nálezy poslední doby, docházelo ke kontaktu s okolními oblastmi již v době před vznikem prvních řeckých kolonií na území Thrákie (Zahrnt 2015, 36). Tyto kontakty byly pravděpodobně obchodní povahy, ale bohužel archeologické doklady nejsou vždy zcela jednoznačné a lze je interpretovat různými způsoby.

\subsection[řecká-kolonizace-thráckého-pobřeží]{Řecká kolonizace thráckého pobřeží}

První doložená řecká osídlení trvalého charakteru se na thráckém pobřeží objevila v průběhu 7. st. n. l. a během dvou následujících století se řecké kolonie rozšířily téměř na celé pobřeží Egejského, Marmarského a Černého moře (Isaac 1986; Archibald 1998, 32-47). Kolonizační aktivity vycházely vždy z iniciativy jednotlivých obcí či ze spojeného úsilí několika měst, ale v žádném případě se nejednalo o koordinovanou aktivitu, kterou by řídila jedna politická autorita na celém území pobřežní Thrákie.

Jako hlavní důvody řecké kolonizace \index{literární prameny}literární prameny uvádějí bohatství Thrákie, zejména nerostné zdroje a úrodnou půdu, (Archilochos, Diehl frg. 2 a 51; Hdt. 1.64.1; 5.23.2; srov. Isaac 186, 282-285; Tiverios 2008, 80).\footnote{Úrodnost thrácké půdy a existence bohatých nerostných zdrojů na území Trákie byly známé v řeckém světě již před obdobím kolonizace, jak dokazují zmínky u Homéra (Hom. {\em Il.} 13.1-16; 10.484).} Řecká města byla zakládána v blízkosti mořského pobřeží, často nedaleko vodních toků a v blízkosti zdrojů nerostných surovin. Koncem 6. a v průběhu 5. st. př. n. l. řecká města začala razit zlaté a zejména stříbrné mince, na něž získávala materiál těžbou v místních dolech. Poměrně záhy začali mince razit i jednotliví thráčtí panovníci (Archibald 2015, 912). Mince byly raženy dle řeckých mincovních standardů a nesly legendu s řeckým písmem, což usnadnilo kontakty ekonomického charakteru mezi řeckými městy a thráckými aristokraty (Tiverios 2008, 128). Aby dále podpořily obchodní aktivity, řecké kolonie zakládaly svá vlastní osídlení, {\em emporia}, jejichž primární funkcí bylo zajišťovat obchod s Thráky, případně obstarávat zemědělské produkty či se starat o těžbu nerostných surovin (Tiverios 2008, 86-91). Většinou se tato {\em emporia} nacházela v okolí nerostných zdrojů a řek. Nicméně osídlování neprobíhalo pouze na pobřeží, ale máme i archeologické důkazy o vznikajících obchodních stanicích ve vnitrozemí, které umožňovaly přímý kontakt a usnadňovaly obchod s pobřežními oblastmi, např. na řece Hebru ležící {\em emporion} Pistiros (Bouzek {\em et al}. 1996; 2002; 2007; 2010; 2013; 2016; Bouzek a Domaradzka 2011). Vnitrozemská Thrákie tak díky těmto stanicím snáze získávala předměty řecké provenience, jako např. keramiku, transportní amfory, víno, olej, a to vše pravděpodobně výměnou za nerostné suroviny, jichž byl v Thrákii dostatek.

Povaha kontaktů nově příchozího a původního obyvatelstva se velmi lišila dle konkrétní situace. V souvislosti se zakládáním nových sídel a rozvíjením obchodních kontaktů pravděpodobně docházelo i ke sporům o území a životní prostor. Řečtí osadníci se často usazovali v místech již existujících thráckých osídlení, což jistě vyvolávalo nevoli thráckého obyvatelstva. Z řecky psané literatury víme konfliktech mezi Thráky a řeckými kolonizátory z Paru, jak je zmiňuje např. řecký básník Archilochos (Diehls frg. 2, 6 a 51), či o neúspěchu první kolonie v Abdéře\index{Abdéra} způsobeném jak nepřátelsky nalazenými Thráky, tak i nepříznivými zeměpisnými podmínkami a šířením nemocí (Hdt. 1.168; srov. Graham 1992, 46-48; Loukopoulou 1989, 185-190). Na jiných místech dochované archeologické nálezy nicméně dokumentují spíše poklidné soužití, či dokonce ekonomickou kooperaci mezi Thráky a Řeky žijícími na pobřeží (Archibald 1998, 47, 73; Ilieva 2011, 25-43). Epigrafická produkce, která by dokumentovala povahu raných kontaktů mezi Thráky a Řeky, buď vůbec nevznikla, nebo se do dnešní doby nedochovala. Veškeré nápisy pochází až z doby stabilních a fungujících řeckých komunit na thráckém území, tedy z doby několik desítek let od založení kolonií (Loukopoulou 1989, 185-217).\footnote{Podrobněji v \in{kapitole}[chapter-Epigrafická-produkce-napříč-staletími:::chapter-Epigrafická-produkce-napříč-staletími].}

\subsubsection[geografický-rozsah-řecké-kolonizace]{Geografický rozsah řecké kolonizace}

\index{řecká kolonizace}Řecká kolonizace představovala většinou aktivitu jednotlivých obcí, čemuž odpovídá i rozdělení sfér vlivu a silné regionální vazby nově vniklých osídlení. V egejské oblasti docházelo zejména k osídlování z přilehlého Thasu a Samothráké\footnote{Mezi nejaktivnější kolonizátory v egejské oblasti patřili osadníci z ostrova Paros a Thasos, kteří se zaměřili především na protilehlé thrácké pobřeží v okolí řeky Strýmón, kde se nacházely zdroje nerostných surovin (Zahrnt 2015, 36). Mezi nejvýznamnější thaské kolonie patří Neápolis, Oisímé a Strýmé-Molyvoti, které zajišťovaly jak obchodní výměnu, ale umožňovaly přístup k úrodné thrácké půdě (Tiverios 2008, 80-85).} a dále z iónských a aiolských maloasijských obcí. Oproti tomu v oblasti Propontidy a Černého moře zásadní roli zaujímaly kolonie dórské Megary a iónského Mílétu a Samu.\footnote{Oblast Marmarského moře, též známá jako Propontis, se stala středem zájmu dórské Megary a iónského Samu. Megara založila již v 7. st. př. n. l. města Sélymbrii a Byzantion, která sehrála velmi důležitou roli v kulturním a politickém vývoji dalších století. Samos založil města Perinthos a Bisanthé, která hrála taktéž významnou roli v rámci regionu (Loukopoulou a Lajtar 2004, 912). Oblast západního pobřeží Černého moře se stala středem zájmu zejména iónského Mílétu, který založil např. Apollónii Pontskou či Odéssos, a dále dórské Megary, která založila město Mesámbriá (Avram, Hind a Tsetskhladze 2004, 926).}

Nejvýznamnějším příkladem maloasijské iónské \index{řecká kolonizace}kolonizace je založení Abdéry\index{Abdéra} na egejském pobřeží poblíž řeky Nestos. Abdéra byla založena zhruba v polovině 7. st. př. n. l. z maloasijských Klazomen. Její obyvatelé se potýkali s odporem místních thráckých kmenů a nemocemi, a kvůli tomu byla Abdéra pravděpodobně opuštěna. O několik desítek let později, zhruba v polovině 6. st. př. n. l. osadníci z Teu místo znovu osídlili a z Abdéry se stalo jedno z nejvýznamnějších ekonomických a kulturních center egejské Thrákie a jeden z prvních producentů nápisů v oblasti (Hdt. 1.168; srov. Isaac 1986, 73-89; Graham 1992, 46-53; Loukopoulou 2004, 872-875; Tiverios 2008, 91-99).\footnote{Dle odhadů mohla populace Abdéry dosahovat až 100 000. Jednalo se tak o jedno z největších měst z oblasti egejské Thrákie (Loukopoulou 2004, 873). Dimitri Samsaris (1980, 166-167) řadí Abdéru na stejnou úroveň jako města Amfipolis, Maróneiu, Byzantion, Perinthos a Thasos, tedy s populací o velikosti řádově několika desítek tisíc obyvatel.} Zhruba od r. 540 př. n. l. Abdéra taktéž razila vlastní mince z lokálních zdrojů stříbra, k nimž měla pravděpodobně přístup. Nálezy abdérských mincí v depotech po celém Středomoří a na území Thrákie dosvědčují, že mince byly určeny jak pro dálkový, tak i pro místní obchod (May 1996, 1-2, 16-17, 59-66; Sheedy 2013, 40-41; Paunov 2015, 267). Vzhledem ke svému důležitému postavení v průběhu 5. st. př. n. l. hrála Abdéra poměrně zásadní roli při zprostředkování diplomatických kontaktů mezi Athénami a odryským panovníkem Sitalkem, které vyústily v uzavření spojenectví v roce 431 př. n. l. (Thuc. 2.29; 2.95-101; srov. Janouchová 2013a, 96-97; Sheedy 2013, 46). Abdéra zaujímala důležitou prostředníka roli nejen při zprostředkování diplomatických kontaktů mezi Řeky a Thráky, ale na jejím území a v blízkém okolí docházelo i ke kontaktům běžné populace a prolínání náboženských představ.\footnote{Příkladem setkávání dvou tradic je kult řeckého Apollóna, který nese místní thrácké přízvisko {\em Derénos}. Tento kult je doložen nejen z Abdéry, ale v římské době se dokonce objevil v thráckém vnitrozemí (Graham 1992, 67-68; Janouchová 2016, 93-95).}

Neméně významným regionálním centrem iónského původu byla Maróneia\index{Maróneia}, kterou založil v polovině 7. st. př. n. l. Chios, pravděpodobně na místě dřívějšího thráckého osídlení. Maróneia a okolní oblast hory Ismaros se stala známou pro své víno a díky strategicky situovanému přístavu si zajistila výhodné ekonomické postavení a četné obchodní kontakty jak se Středozemím, tak s vnitrozemskými Thráky (Archilochos, Diehl frg. 2; srov. Isaac 1986, 111-115; Tiverios 2008, 99-104).

Území na egejském pobřeží mezi řekami Nestos a Hebros spadalo do sféry vlivu Samothráké, která byla sama pravděpodobně kolonií iónského Samu a neznámého aiolského města. V 6. st. př. n. l. byla na protilehlé thrácké pevnině ze Samothráké založena města Drýs a Zóné-Mesámbriá.\footnote{Lokalita Mesámbriá, či Mesémbriá se vyskytuje jak na pobřeží Egejského, tak Černého moře. Mesámbriá v egejské oblasti je známá z literárních zdrojů (Hdt. 7.108.2, Steph. Byz. 446.19-21) a bývá ztotožňována s další lokalitou Zóné, Orthagoria, či Drýs (Loukopoulou 2004, 880). Lokalita Mesámbriá z oblasti Černého moře je taktéž známá z literárních zdrojů (Hdt. 4.93, 6.33.2) a dnes se nachází pod moderním městem Nesebar v Bulharsku.} Mezi Thráky a Řeky v této oblasti docházelo k četným kontaktům, které dle dostupných archeologických pramenů nebyly násilného charakteru a měly za následek jak mísení obyvatelstva a náboženských zvyklostí, ale i například vzájemné ovlivňování technologií zpracování kovu a výroby keramických nádob (Ilieva 2007, 221; 2011, 36-38; Tiverios 2008, 107-118; Kostoglou 2010, 180-185).

Východně od této oblasti ležel aiolský Ainos, který byl pravděpodobně taktéž založen na místě dřívějšího thráckého osídlení. Ainos se nacházel v ústí řeky Hebros, která spojovala egejské pobřeží s thráckým vnitrozemím, a stal se tak ekonomickým a kulturním centrem pro celý region s napojením na thráckou dynastii Odrysů (Isaac 1986, 140-156; Tiverios 2008, 118-120). Dále na východ se nacházela oblast Propontidy s důležitými městy Byzantion a Perinthos. Dórská kolonie Byzantion těžila zejména ze své výhodné pozice, která jí umožňovala kontrolovat bosporskou úžinu, ale i přesto se nevyhnula občasným útokům okolních thráckých kmenů (Isaac 1986, 230-231). V téže oblasti založil iónský Samos na začátku 6. st. př. n. l. na místě existujících thráckých osad kolonie Perinthos a Bisanthé. Z nich zejména Perinthos hrál velmi důležitou roli v pozdějších stoletích, kdy se stal prvním sídlem místodržícího římské provincie {\em Thracia} (Isaac 1986, 198-201).\footnote{Isaac 1986, 205: ve 3. či 4. st. n. l. byl Perinthos přejmenován na Hérakleiu. V epigrafických pramenech se vyskytují obě dvě jména, a proto se držím pojmenování Perinthos (Hérakleia), a od konce 3. st. n. l. Hérakleia (Perinthos).}

Oblast Thráckého Chersonésu byla pro svou strategickou pozici a úrodnou půdu osídlena nejpozději na konci 7. st. př. n. l. aiolskými osadníky. Jména nejvýznamnějších měst z této oblasti jsou Séstos, Alópekonnésos či Madytos (Loukopoulou 2004, 900). V 6. st. př. n. l. se o oblast Thráckého Chersonésu velmi zajímaly Athény, které sem na určitou dobu dosadily vojenské posádky a snažily se jak silou, tak spoluprací s místními thráckými kmeny ovládnout toto strategické území (Loukopoulou 1989, 67-94; Tiverios 2008, 121-124).

Na pobřeží Černého moře založil řecký Mílétos již v 6. st. př. n. l. celou řadu kolonií, z nichž se poměrně záhy stala regionální ekonomická a kulturní centra, jako v případě Apollónie Pontské a později Odéssu. Nejvýznamnější dórskou kolonií v této oblasti se stala Mesámbriá založená Megarou, která zaujímala výsadní postavení ve 3. st. př. n. l.\footnote{O thráckém původu Mesámbrie svědčí přípona -{\em bria}, označujícící osídlení v thráčtině (Venedikov 1977, 76).} Černomořské kolonie byly pravděpodobně v úzkém kontaktu s Thráky, jak dokládají nejen thrácká jména vyskytující se na pohřebních stélách, ale i epigraficky potvrzené diplomatické kontakty mezi řeckými městy a odryskými aristokraty či přítomnost thrácké keramiky a kovových předmětů thrácké provenience na území řeckých měst.\footnote{Náboženství a kulty v černomořských koloniích do velké míry vycházely z thráckých vzorů, nicméně první doklady o tomto náboženském synkretismu máme až z hellénismu a římské doby (Isaac 1986, 241-257).}

Povaha kontaktů mezi Řeky a Thráky záležela vždy na konkrétní situaci a rozhodně nelze z dochovaných materiálních dokladů vyvozovat, že Thrákové k příchozím Řekům zaujímali automaticky nepřátelský postoj, jak by mohly naznačovat literární zmínky u básníka Archilocha.\footnote{Zejména Diehls frg. 2, 6 a 51.} Prvotní kontakty mohly být nepřátelské, zejména proto, že Řekové se většinou snažili založit kolonii v místě existujícího thráckého osídlení. Pravděpodobně také docházelo ke konfliktům o vlastnictví zemědělské půdy a zdrojů nerostných surovin, což je však velmi špatně doložitelné v materiálních pramenech. Pokud jde o období následující po \index{řecká kolonizace}primární kolonizaci, archeologické studie keramické a metalurgické produkce napovídají, že se vzájemné kontakty Řeků a Thráků nesly v duchu ekonomické spolupráce a relativně poklidného soužití, alespoň v nejbližším okolí řeckých měst na pobřeží (Ilieva 2007, 221; 2011, 36-38; Kostoglou 2010, 180-185).

\subsection[peršané-a-vznik-dynastie-odrysů]{Peršané a vznik dynastie Odrysů}

V 6. a 5. st. př. n. l. se část Thrákie na několik desítek let stala na součástí perské říše (Archibald 1998, 79-90; Boteva 2011, 738-749).\footnote{První zmínky o perské přítomnosti v Thrákii pocházejí z doby Dáreiova tažení proti Skýthům v roce 513 př. n. l. (Hdt. 4.83ff; 5.1-5.2). Po řecko-perských válkách byli Peršané postupně z Thrákie vytlačováni a poslední doklady o jejich přítomnosti pocházejí z území Thráckého Chersonésu z poloviny šedesátých let 5. st. př. n. l. (Plut. {\em Kim}. 14.1).} Bohužel nemáme přesné informace o tom, zda Peršané ovládli celou Thrákii a jakým způsobem řídili ovládnuté území (Zahrnt 2015, 38).\footnote{Tj. zda se jednalo o satrapii v pravém slova smyslu.} Hérodotos dokládá rozdělení ovládnutého území do několika oblastí, které byly spravovány ustanovenými místodržícími (Hdt. 7.106).

Perská přítomnost se projevila na proměňující se materiální kultuře a zvyklostech: Zofia Archibald navrhuje, že právě vliv Peršanů mohl být hlavní příčinou objevení tzv. mincí thráckých králů na počátku 5. st. př. n. l. (1998, 89-90). S obdobím perské nadvlády se taktéž spojuje zavádění nových technologií zpracování kovu a následné rozšíření luxusních nádob z drahých kovů. Ty se mnohdy staly součástí pohřební výbavy thrácké aristokracie v 5. a 4. st. př. n. l. a některé z nádob nesly nápisy odkazující na majitele (Archibald 1998, 85; Theodossiev 2011, 6; Loukopoulou 2008, 148).\footnote{K jednotlivým nádobám v \in{kapitole}[chapter-Epigrafická-produkce-napříč-staletími:::chapter-Epigrafická-produkce-napříč-staletími] v sekcích věnovaným nápisům datovaných do 5. a 4. st. př. n. l. \at{na straně}[charakteristika-epigrafické-produkce-v-5.-až-4.-st.-př.-n.-l.].} Peršané měli dále vliv na zvyky spojené s vytvářením nápisů a předmětů nesoucí nápisy. Jak se dozvídáme z Hérodotova vyprávění, Peršané na území Thrákie vztyčili několik mramorových stél, a ač se dochovaly starší nápisy přímo z řeckých měst na pobřeží Thrákie, Hérodotův popis tak představuje první reflexi epigrafické aktivity na území Thrákie, která paradoxně poukazuje ininiciativu.\footnote{Jednalo se o dvě stély s asyrským a řeckým písmem, zaznamenávající členy Dáreiovy výpravy, umístěné na evropské straně Bosporu a u pramenů řeky Teáru ve vnitrozemské Thrákii (Hdt. 4.87, 4.91). Hérodotos rovněž popisuje, že Byzantští si dvě zmíněné stély z Bosporu odvezli a sekundárně je použili v chrámě Artemidy {\em Orthósie}.} Překvapivé je, že iniciátory vztyčení stél byli v tomto případě Peršané, a nikoliv Řekové, avšak i přesto dle slov Hérodota nechal perský panovník Dáreios vztyčit jednu ze stél psanou v řečtině, aby textu rozumělo i místní obyvatelstvo.

Po vyhnání Peršanů z oblasti se většina řeckých měst na pobřeží egejského moře připojila k athénskému námořnímu spolku, kam začala přispívat nemalou měrou. Podíl měst z thrácké oblasti činil zhruba jednu čtvrtinu všech příjmů spolku (Meritt, Wade-Gery a McGregor 1950, 52-57). Vzhledem ke strategické pozici a bohatství zdrojů nerostných surovin se o oblast severní Egejdy Athény začaly zajímat v prvních letech po vyhnání Peršanů z oblasti. Ekonomické a politické zájmy Athén v oblasti Thrákie vyústily nejen umístěním vojenských posádek na Thráckém Chersonésu, ale i opakovanou snahou založit vlastní kolonii na území egejské Thrákie. Zakládání nových kolonií v případě Brey či Amfipole narazilo na odpor místních obyvatel a bylo ztíženo probíhajícími konflikty znepřátelených řeckých obcí (IG I\high{3} 1, 46; Thuc. 1.98, 1.100, 4.102; srov. Meiggs 1972, 158-159, 195).

Téměř současně s ústupem perské nadvlády se zhruba před polovinou 5. st. př. n. l. na jihovýchodě Thrákie v okolí řek Hebros a Tonzos objevila první autonomní politická organizace, známá z řeckých historických pramenů jako odryský stát, či království.\footnote{Zofia Archibald (1998, 93) tvrdí, že se jednalo o stát se všemi jeho atributy: existovala zde společenská stratifikace, aristokracie ve svých rukách soustředila veškerou politickou moc a kontrolu nad přerozdělováním prostředků, docházelo zde k specializaci činností, ke stavbám výstavných rezidencí a monumentálních pohřebních mohyl a v omezené míře se zde vyskytovaly i písemné památky. Dle dostupných výsledků archeologických pozemních sběrů z nedávné doby je však spíše pravděpodobné, že se jednalo o mezistupeň mezi kmenovým uspořádáním a raným státem, kde velkou roli hrála rodově uspořádaná aristokracie a specializace práce byla spíše záležitostí přítomnosti cizích umělců a specialistů, než reflexe skutečné stratifikace odryské společnosti a existence institucionální organizace dlouhodobého charakteru (Sobotkova 2013, 133-142; Sobotkova a Ross 2015, 5-8).} Odrysové si v průběhu 5. st. př. n. l. vybudovali takovou pozici, že se stali nejen spojenci Athén, ale dokonce si mohli dovolit od řeckých měst na thráckém pobřeží vybírat pravidelné peněžní i nepeněžní dávky ve výši až 800 talentů (Thuc. 2.29, 2.97; srov. Archibald 1998, 145). Tyto platby mohly sloužit výměnou za ochranu řeckých měst na pobřeží i ve vnitrozemí, či jako jistá forma cla, poplatek za možnost obchodovat s thráckým vnitrozemím (Zahrnt 2015, 42).

Mezi nejznámější panovníky patří Sitalkés, za jehož vlády došlo jak k~ekonomické, tak územní expanzi odryské říše. V roce 431 př. n. l. dokonce uzavřel spojeneckou smlouvu s Athénami, kterou stvrdil významnou pozici odryského území (Thuc. 2.29; srov. Archibald 1998, 118-120; Zahrnt 2015, 41-42). Jak víme z literárních pramenů, okolo r. 400 př. n. l. si odryský panovník Seuthés II. najal do svých služeb historika Xenofónta a další řecké žoldnéře, aby mu pomohli získat zpět území a výsadní pozici v rámci odryské aristokracie (Xen. {\em Anab.} 7.1.5-7.1.14; srov. Zahrnt 2015, 43). Odrysové si udrželi výsadní pozici i po konci peloponnéské války a v první polovině 4. st. př. n. l., kdy i nadále vystupovali jako mocní spojenci Athén a okolních kmenů. V roce 383 př. n. l. se k moci dostal panovník Kotys I., který odryskou říši upevnil obratnou diplomatickou politikou. Pravděpodobně kvůli narůstající politické moci Odrysů byl v roce 359 př. n. l. zavražděn a území kmene Odrysů se rozpadlo na několik menších částí a fakticky tak ztratilo svou výsadní pozici a stabilitu (Dem. 23.8; srov. Archibald 1998, 218-222; Zahrnt 2015, 44-45). Cesta do Thrákie se tak uvolnila Athénám, ale současně i nejbližšímu rivalovi a sousedovi, Makedonii.

\subsection[makedonská-kolonizace-a-období-hellénismu]{Makedonská kolonizace a období hellénismu}

Již před polovinou 4. st. př. n. l. Filip II. Makedonský podnikl několik výprav na pobřeží egejské Thrákie, kde ovládl řecká města, zmocnil se zdrojů nerostných surovin a umístil zde vojenské posádky\index{makedonská kolonizace}. O několik let později se vypravil do vnitrozemí Thrákie, aby si podrobil nebezpečné a výbojné thrácké kmeny a zabezpečil oblast bezprostředně sousedící s Makedonií (Worthington 2015, 76). Filip tak využil ve svůj prospěch nejednotnosti thráckých kmenů, k níž došlo po smrti Kotya I. a nakonec se v roce 340 př. n. l. zmocnil větší části území vnitrozemské a pobřežní Thrákie.\footnote{Makedonci podnikli výpravy nejen proti Odrysům, ale i proti Bessům či Maidům v údolí Strýmónu a nějakou dobu neúspěšně obléhali Perinthos a Byzantion a válčili s kmenem Triballů.} Thrákie se tak stala nedílnou součástí pozdější makedonské říše a odryští králové byli dočasně poraženi. Bohužel nemáme přesné informace o rozsahu makedonské moci v Thrákii, ale předpokládá se, že zhruba od roku 340 př. n. l. většina území plně spadala pod makedonskou administrativu, systém výběru daní a docházelo i k verbování thrácké populace do makedonské armády. Na území dohlížel ustanovený makedonský vojevůdce ({\em stratégos}; D. S. 17.62.5; Arr. {\em Anab.} 1.25.2; srov. Archibald 1998, 231-239; Delev 2015, 49-53; Worthington 2015, 76).

Během pobytu v thráckém vnitrozemí Filip II. založil několik vojensky zaměřených osídlení na místě existujících thráckých sídel (Dem. 8.44; 10.15). Tato místa se postupem času rozrostla na sídla městského charakteru, jak je patrné na příkladě Kabylé nebo původní thráckého sídla {\em Pulpudeva}, která je na Filippovu počest v této době přejmenována na Filippopolis, tedy Filippovo město. Podobným způsobem vznikla i Hérakleia Sintská na řece Strýmón, kdy se původně obchodní osada s vojenskou posádkou rozrostla až do podoby hellénistického města (Nankov 2015a, 26-27). V těchto prvních sídlech městského charakteru v thráckém vnitrozemí spolu pravděpodobně žili jak nově příchozí Makedonci, tak původní Thrákové. Jedinečný charakter těchto míst umožnil setkávání několika kultur na každodenní bázi, což mělo za následek i rozšíření epigrafické produkce do thráckého vnitrozemí.

Z literárních zdrojů víme o několika pokusech o osamostatnění se z řad makedonských vojevůdců, které však byly vesměs potlačeny (tzv. Memnónova revolta v roce 331/0 př. n. l., D. S. 17.62-63). Nicméně ani thráčtí aristokraté se nehodlali smířit s \index{makedonská kolonizace}makedonskou nadvládou a docházelo ke konfliktům mezi Makedonci a Odrysy. Příkladem thráckého odboje je konflikt mezi makedonským místodržícím makedonského původu Lýsimachem a thráckým kmenovým vůdcem Seuthem III., který pravděpodobně vyústil v určitou nezávislost Seuthova postavení (D. S. 18.14.2-4; srov. Tacheva 2000, 12-15; Delev 2015a, 53-55).\footnote{Lýsimachos nicméně i přes Seuthovo nezávislé postavení ovládal velkou část Thrákie a na Thráckém Chersonésu založil město Lýsimacheia (Jones 1971, 5). Bohužel se dochovalo jen velmi málo informací o povaze tehdejšího politického uspořádání a rozložení moci, nicméně dle existence staveb jako je Seuthopolis je možné soudit, že Seuthés si byl schopen minimálně na několik desetiletí zajistit minimálně semi-autonomní pozici.}

Odryský panovník Seuthés III., který žil na přelomu 4. a 3. st. př. n. l., je známý především díky objevu výstavné rezidence na řece Tonzos, která nesla jeho jméno, Seuthopolis, a díky objevu monumentálních mohylových hrobek v Kazanlackém údolí, patřících Seuthovi a jeho rodině (Dimitrov, Čičikova a Alexieva 1978; Dimitrova 2015; Delev 2015a, 53-54; Tzochev 2016, 779-783). Dle nalezeného archeologického a epigrafického materiálu se zdá, že Seuthopolis byla osídlena jak thráckým, tak řeckým či makedonským obyvatelstvem, které se po dobu existence rezidence podílelo na vzniku unikátní kombinace kultur a zvyklostí. Seuthopolis, ač se nachází uprostřed thráckého vnitrozemí, nesla všechny charakteristiky typické pro hellénistická sídla tehdejší doby, počínaje užitou architekturou s domy typu {\em pastas} a {\em prostas}, nalezenou keramikou řeckého a místního původu, precizně zhotovenou toreutikou, sochařskou a dekorativní výzdobou a zejména unikátními epigrafickými nálezy (Tacheva 2000, 25-35). Krátký časový horizont existence Seuthopole nicméně potvrzuje, že tamní unikátní společnost plně závisela na osobě panovníka Seutha III. a po jeho smrti na počátku 3. st. př. n. l. došlo k poměrně rychlému úpadku aktivit, vymizení jak materiální, tak epigrafické produkce a zániku tohoto jedinečného příkladu hellénistické kultury v samém středu thráckého území (Nankov 2012, 120; Janouchová 2018, v tisku).

Po smrti Alexandra Velikého a zejména ve 3. st. př. n. l. byla Thrákie svědkem válečných konfliktů diadochů, kteří se snažili ovládnout tuto významnou spojnici mezi Evropou a Asií. Asi nejvýznamnějším byl konflikt mezi Lýsimachem, původně místodržícím v Thrákii, který se později stal vládcem evropské části \index{makedonská kolonizace}makedonské říše a králem Thrákie, a Seleukem I. Lýsimachos v roce 281 př. n. l. nakonec Seleukovi podlehl a Thrákie se stala součástí seleukovské říše (Samsaris 1980, 33-34). V následujících letech pokračovaly konflikty následovnických rodů, které z větší či menší části zahrnovaly i území Thrákie. Z této doby pocházejí mince seleukovských panovníků ražené na území Thrákie, např. stříbrné mince z Kabylé, ale i drobné seleukovské mince nesoucí kontramarky Kabylé, určené pro lokální trh (Draganov 1991, 198-208; Draganov 1993, 87-99). Řecká města na pobřeží se v polovině 3. st. př. n. l. dostala do vlivu Ptolemaiovců, což vysvětluje výskyt původně egyptských božstev a motivů ve figurálním umění a na nápisech (Delev 2015b, 61).\footnote{Jones 1971, 6: Pod vliv Ptolemaiovců se dostal Ainos, Maróneia a část Chersonésu s Lýsimacheiou.}

V průběhu 3. st. př. n. l. do Thrákie v několika vlnách vpadly keltské kmeny, což s sebou neslo významné změny ve fungování mnoha významných osídlení či dokonce jejich zánik, jako v případě Seuthopole, či {\em emporia} Pistiros. Keltové se na území Thrákie na určitou dobu usadili, a dokonce si postavili nové hlavní město Tylis, které existovalo až do roku 213 př. n. l. (Theodossiev 2011, 15). Přítomnost Keltů měla vliv na thrácké umění, zejména na toreutiku, kde je možné sledovat výskyt nových motivů, ale i technologií. Keltové dokonce razili vlastní mince a využívali k tomu již existující infrastrukturu, čehož jsou důkazem např. mince keltského panovníka z Kabylé, Mesámbrie či Odéssu (Draganov 1993, 107).

Nedostatek pramenů neumožňuje podrobně rekonstruovat následující vývoj, a ve výkladu zůstává mnoho bílých míst. Thrákie zůstala ve sféře vlivu Makedonie až do poloviny 2. st. př. n. l. I nadále docházelo k politickým konfliktům na území Thrákie a ve větší míře zde operovaly makedonské armády, nicméně administrativa pravděpodobně fungovala bez větších změn. Během 2. st. př. n. l. se začal pomalu stupňovat politický vliv Říma na dění v Thrákii, který se zintenzivnil po roce 148-146 př. n. l., kdy došlo k ovládnutí Makedonie Římem (Eckstein 2010, 248). Thráčtí Odrysové využili situace a stali se spojenci a podporovatelé Říma (T. Liv. 45.42.6-12; srov. Delev 2015b, 63-68). Řecká města na pobřeží v této době z části spadala pod římskou provincii {\em Macedonia}, stejně tak jako vnitrozemská Hérakleia Sintská, a z části si udržela autonomní pozici, jako např. Abdéra, Maróneia a Ainos (T. Liv. 45.29.5-7).\footnote{Jones 1971, 7: města Ainos a Maróneia spolu s pobřežními oblastmi se opět stala součástí \index{makedonská kolonizace}Makedonie. V roce 168 př. n. l. se obě města stala autonomními politickými autoritami.} Byzantion v roce 148 př. n. l. uzavřel spojeneckou smlouvu, což vedlo k nárůstu politické moci Říma v regionu a zároveň i rozvoji Byzantia (Tac. {\em Ann.} 12.62-63; srov. Jones 1971, 7). Římská přítomnost v regionu byla patrná zejména podél cesty {\em Via Egnatia}, která spojovala oblast západního a východního balkánského poloostrova. Oblast Thráckého Chersonésu v té době spadala pod pergamské království a byla spravována vojenskými veliteli ({\em stratégy}), jak dokazují dochované nápisy (Delev 2015b, 68). Se zánikem pergamského království se v r. 133 př. n. l. Thrácký Chersonésos stal součástí římské provincie {\em Asia}. V první polovině 1. st. př. n. l. se část Thráků a řeckých měst přidala na stranu pontského krále Mithridata VI., což mělo za následek římské vojenské akce na území Thrákie ve snaze zajistit stabilitu v oblasti a eliminovat případný odpor místních kmenů (Lozanov 2015, 76-77). V této době se tak vytvořil prostor pro iniciativní jedince z řad thrácké aristokracie, kteří za svou podporu Říma získali výsadní postavení.

\subsection[období-vazalských-králů]{Období vazalských králů}

V 1. st. př. n. l. se prosadilo několik thráckých aristokratických dynastií, které získaly své postavení zejména spoluprací s Římem a došlo k vytvoření tzv. vazalských království, podobně jako např. v oblasti Británie či Judeje. Thrákové poskytovali Římu zejména vojenskou a materiální pomoc a výměnou se jim dostalo relativní autonomie a osobních výhod.\footnote{David Braund (1984, 23-29) uvádí, že vazalští králové často obdrželi jak římské občanství\index{občanství} a jiné posty, za které byli ochotni zaplatit nemalé finanční částky. Spolu s poctami často dostávali i luxusní dary, jako oděvy a znaky moci, které zvyšovaly jejich společenskou prestiž a upevňovaly postavení krále.} Mezi nejvýznamnější kmeny této doby patří Odrysové, Sapaiové a Astové (Lozanov 2015, 78).

Odryští králové byli u moci přibližně do poloviny 1. st. př. n. l. a z historických pramenů víme, že po roce 42 př. n. l. thrácký Rhéskúporis I. založil sapajskou dynastii se sídlem v Bizyi v jihovýchodní Thrákii a vládl v letech 48 - 41 př. n. l. (Strabo 7, frg. 47; 12.3.29; srov. Jones 1971, 9-10; Manov 2002, 627-631). Bezprostředně po něm není následnická linie zcela jasná, nicméně se sapajští králové udrželi u moci až do r. 46 n. l.\footnote{Lozanov 2015, (78-80); Manov 2002 (627-631): přesné roky vlády jednotlivých králů jsou i nadále předmětem akademické debaty. Z literárních pramenů známe jména a přibližnou dobu vlády následujících panovníků: Rhéskúporis I (48-42 př. n. l.), Kotys (42-18? př. n. l.), Rhéskúporis II. (18-13? př. n. l.), Rhoimétalkás I. (13 př. n. l. - 11 n. l.), Kotys (12-19 n. l.), Rhoimétalkás II. (18-38 n. l.) a Rhoimétalkás III (38-45 či 46 n. l.).} Posledním thráckým králem se stal Rhoimétalkás III., který vládl v letech 38 až 46 n. l. Ač byla dynastie Sapaiů teoreticky nezávislá, v praxi se jednalo o panovníky dosazené k moci Římem, který sledoval své vlastní mocenské zájmy. Z epigrafických a literárních zdrojů víme, že území Thrákie bylo rozděleno na samosprávní jednotky, tzv. stratégie, které spravovali stratégové, původem thráčtí aristokraté, či loajální Řekové. Rozdělení do stratégií mělo primárně usnadňovat administrativu a sloužilo i pro zvýšení vojenské kontroly území či verbování jednotek (Lozanov 2015, 78-79). Za vlády sapajské dynastie mnoho thráckých mužů vstoupilo do římské \index{římská armáda}armády jako příslušníci pomocných jednotek (Tac. {\em Ann.} 4.47). Nejenže Thrákové v této době sloužili v římské armádě, ale římská armáda operovala na území Thrákie.\footnote{Z literárních pramenů se dozvídáme, že místodržící v Makedonii Marcus Lucullus v roce 72/1 př. n. l. porazil kmen Bessů a dobyl nejen Kabylé, ale zmocnil se i měst na pobřeží Černého moře a založil zde vojenské posádky (Eutrop. {\em Breviarium} 6.10).}

Postupný nárůst moci Říma měl za následek transformaci vazalských království do podoby římské provincie. Sever území okolo řeky Istros se stal součástí římské říše pravděpodobně okolo r. 15. n. l. jako provincie {\em Moesia Inferior}. O 30 let později po smrti Rhoimétalka III. v roce 45 či 46 n. l. Řím využil příležitosti a chaosem zmítané území vnitrozemské Thrákie přeměnil na provincii {\em Thracia}, která od té doby plně spadala pod autoritu římského císaře (Lozanov 2015, 76-80).

\subsection[římské-provincie-thracia-a-moesia-inferior]{Římské provincie Thracia a Moesia Inferior}

Bezprostředně po vzniku nových provincií {\em Moesia Inferior} a {\em Thracia} nedošlo k zásadním proměnám společenského uspořádání a administrativy, ale spíše došlo k navázání na již existující infrastrukturu vytvořenou vazalskými panovníky. Systém stratégií přetrval pravděpodobně až do začátku 2. st. n. l. a výsadní pozici si udržovala i nadále thrácká aristokracie, která se dokázala adaptovat na nové podmínky.\footnote{Role stratégií zůstala pravděpodobně stejná, avšak s postupem času se snižoval jejich počet z původních 50 na 14 (Plin. {\em H. N.} 4.11.40; Ptolem. {\em Geogr.} 3.11.6; srov. Jones 1971, 10-15).} Odměnou za jejich loajální služby bylo udělení římského občanství\index{občanství} a vysoká pozice v provinciálním aparátu (Lozanov 2015, 80-82). Některá z řeckých měst si udržela nezávislost na systému stratégií a pravděpodobně si uchovala politickou a ekonomickou autonomii. Plinius zmiňuje specificky Abdéru, Maróneiu, Ainos a Byzantion (Plin. {\em H. N.} 4.42-43). Anchialos a Perinthos se stala hlavními městy stratégií, ale byla jim ponechána určitá autonomie (Jones 1971, 15).

Provincie {\em Thracia} byla územím bez trvale usídlené legie ({\em provincia inermis}) a existovalo zde pouze několik pravidelných jednotek, skládajících se mimo jiné i z místního obyvatelstva, jako jednotky usídlené v Kabylé či Germaneii. Oproti tomu {\em Moesia Inferior} byla provincií se silnou vojenskou přítomností, zejména podél řeky Dunaje, kde byla vytvořena soustava vojenských táborů a opevnění, což mělo vliv i na civilní obyvatelstvo (Haynes 2011, 7-9; Lozanov 2015, 80-82).

Předpokládá se také, že Thrákové se aktivně účastnili služby v římské armádě již v průběhu 1. st. př. n. l. a jejich nábor měl mít na starosti stratégos, který jednak znal nejlépe místní poměry, ale zároveň prokázal svou věrnost Římu (Lozanov 2015, 79-81). Z literárních a epigrafických pramenů pocházejících z celé římské říše víme, že již v 1. st. n. l. existovaly pomocné vojenské jednotky ({\em auxilia}), které nesly ve jméně thrácký původ. Tyto vojenské jednotky se skládaly jak z jezdců ({\em alae}), tak z pěšího vojska ({\em cohortes}) a sloužily po celém území římské říše (Jarrett 1969, 215). Jména pomocných jednotek nesla typicky odkaz na thrácký původ, např. {\em alae Thracorum}, či {\em alae Bessorum}.\footnote{Jednotné pojmenování jednotek za použití jména kmene Bessů bylo využíváno nikoliv proto, že by všichni členové jednotky pocházeli z kmene Bessů, ale jako zástupné pojmenování poukazující na původ vojáků v obecné rovině. Tento jev byl součástí verbovací strategie římských pomocných jednotek a je pozorovatelný i na jiných místech římského impéria, jako např. v Batávii (Derks 2009, 239-270). Thrákové naverbovaní do římské armády byli zařazeni do jednotky, která tak primárně nemusela odpovídat jejich kmenové příslušnosti, ale v očích římské administrativy byl zvolen jeden kmen, podle nějž se jednotky jmenovaly. Thrákové byli nuceni se vstupem do římského vojska adoptovat novou identitu, která se zcela nemusela shodovat s jejich původem. Jak je ale patrné z dochovaných nápisů a vojenských diplomů, Thrákové však na svůj původ nezanevřeli zcela a na vojenských diplomech často uvádí jednak svůj etnický původ, ale i konkrétní obci či vesnici, z níž pocházejí, což spolu s uchováním thráckých osobních jmen poukazuje na silný tradicionalismus a na vědomí vlastní identity (Dana 2013, 246).} Thrákové se stali jedním z nejpočetnějších etnik římské armády a po odsloužení 25 let vojenské služby se často vraceli jako veteráni zpět do vlasti, zakládali nová sídliště a podíleli se na správě provincie. Ke konci 1. st. n. l. byla zakládána nová města\index{římská kolonizace}, kde byli umísťováni veteráni římské armády, jako např. {\em Colonia Flavia Pacis Deultensium}, známá jako Deultum, či {\em Colonia Claudia Aprensis}, známá jako Apros (Lozanov 2015, 85). Veteráni tak měli zajišťovat bezpečí v nejbližším okolí strategicky umístěných měst výměnou za pozemky. Tento trend byl ještě patrnější severně od pohoří Haimos v {\em Moesii Inferior}, kde v okolí vojenských táborů vznikala civilní \index{urbanizace}osídlení, která zásobovala početné římské vojsko. Ve stejné době se zde rozvinul systém {\em vill}, tedy venkovských usedlostí zaměřených primárně na zemědělskou produkci, kde běžně docházelo ke kontaktu s místním obyvatelstvem.

\subsection[společenské-reformy-2.-st.-n.-l.]{Společenské reformy 2. st. n. l.}

Na začátku 2. st. n. l. za vlády Trajána a Hadriána došlo k poměrně zásadním reformám provinciální administrativy v Thrákii, které měly vliv na celou společnost. Nejen, že došlo ke změně statutu provincie na tzv. prétoriánskou provincii, ale i k organizačním změnám, které vedly k posílení politické moci místní samosprávy. Důsledkem reforem došlo ve 2. st. n. l. k zintenzivnění stavební aktivity, zakládání nových měst či rozšiřování starších osídlení a výstavbě veřejné infrastruktury, jako např. lázní, akvaduktů, divadel či silnic\index{římské silnice}.\footnote{\index{urbanizace}Jména nově vzniklých či obnovených měst dokazují šíři těchto stavebních aktivit: Augusta Traiana, Trainúpolis, Hadriánúpolis, Plotinúpolis, Ulpia Nicopolis ad Istrum, Ulpia Nicopolis ad Nestum, Ulpia Marcianopolis (Jones 1971, 18-19).} \index{centralizace}Stejně tak narostl význam městských samospráv, jimž bylo uděleno více pravomocí na úkor aristokracie. V Thrákii, ale např. i v sousední Bíthýnii, se setkáváme s novou společenskou skupinou městských elit, které pocházely z místního prostředí, avšak dokázaly se plně adaptovat na novou společenskou strukturu (Fernoux 2004, 415-511). V této době také vznikaly nové úřady a politická uskupení, jako např. {\em koinon tón Thraikón}, což bylo uskupení vnitrozemských měst se sídlem ve Filippopoli (Lozanov 2015, 82-83).

Sídlo místodržícího se přesunulo z Perinthu na pobřeží do vnitrozemské Filippopole. Původně řecká města na pobřeží postupně ztrácela svůj autonomní status a stávala se součástí římské říše, a to včetně povinnosti platit daně. Města razila vlastní mince s vyobrazením císaře na aversu a charakteristickým symbolem daného města na reversu. Města byla navzájem propojena sítí \index{římské silnice}silnic, které primárně sloužily pro přesuny armády. Na stavbě a údržbě silnic se podílela ta města, na jejichž území se silnice nacházely (Madzharov 2009, 29-30). Od konce 2. st. n. l. vznikaly podél silnic stanice, které zajišťovaly bezpečnost a zásobování. K tomuto účelu byla ve vnitrozemí stavěna {\em emporia}, jejichž hlavním úkolem bylo zprostředkování obchodu mezi zemědělsky zaměřeným venkovem a městskými centry ve větší míře než byl s to obsáhnout systém vill (Lozanov 2015, 84-85).\footnote{Příkladem může být např. {\em emporion} Pizos, či Discoduraterae, o nichž podobněji pojednávám v \in{kapitole}[chapter-Epigrafická-produkce-napříč-staletími:::chapter-Epigrafická-produkce-napříč-staletími].}

Od 2. st. n. l. mohlo thrácké obyvatelstvo sloužit nejen v pomocných vojenských jednotkách, ale vstupovat i do legií, což dosvědčují epigrafické památky z celého území římské říše (Samsaris 1980, 38-39). V průběhu 2. st. n. l. také došlo k výraznějšímu přesunu obyvatelstva z oblasti Bíthýnie do vnitrozemské Thrákie, jak dosvědčují mnohé nápisy a monumenty. To mělo za následek i proměny nejen ve složení obyvatelstva ale i nárůst počtu nových kultů východní provenience (Delchev 2013, 15-19; Raycheva a Delchev 2016; Lozanov 2015, 82; Tacheva-Hitova 1983).

\subsection[krize-3.-a-4.-st.-n.-l.-a-transformace-společnosti]{Krize 3. a 4. st. n. l. a transformace společnosti}

Ve 3. st. n. l. musela římská říše čelit nájezdům nepřátelských kmenů a nejinak tomu bylo i v Thrákii, která byla v druhé polovině století opakovaně postižena nájezdy Gótů. Za císaře Aureliána docházelo k poničení měst a infrastruktury, vyvrcholila vleklá ekonomická krize a došlo ke snížení počtu obyvatelstva. Zásluhou reforem a posílení vojenské přítomnosti na severních hranicích došlo k určitému oživení na konci 3. st. n. l., nicméně i přes tato opatření nájezdy Gótů probíhaly i v následujících staletích (Velkov 1977, 21-29; Poulter 2007, 29-36).

Koncem 3. st. n. l. za císaře Aureliána, ale zejména později za Diokleciána a Konstantina byly provedeny reformy uspořádání římské říše a tyto změny se nevyhnuly ani Thrákii. \footnote{Území Thrákie spadalo jednak pod nově vytvořenou diecézi {\em Thracia} a severovýchodní Thrákie s okolím města Serdica pod diecézi {\em Moesia} a od 4. st. n. l. pod diecézi {\em Dacia}. Diecéze {\em Thracia} se dále dělila na šest provincií, z nichž každá měla své hlavní město: {\em Europa} s hl. m. Eudoxiopolí, dříve Sélymbrií, {\em Rhodope} s Ainem, {\em Haemimontus} s Hadriánopolí, {\em Moesia Inferior} s Marcianopolí, {\em Scythia Minor} s Tomidou a {\em Thracia} s Filippopolí (Velkov 1977, 61-62). Hlavou diecéze {\em Thracia} se stal pověřený {\em vicarius Thraciae} se sídlem v Konstantinopoli, který měl na starosti zejména civilní správu Thrákie.} Hlavním městem říše se v roce 330 n. l. stala Konstantinopol, bývalé Byzantion, což Thrákii posunulo z okrajové části římského impéria rovnou do jejího středu. S přesunutím hlavního města souvisí i zvýšení stavebních aktivit a udržování rozsáhlé sítě silnic, které sloužily k přepravě zboží, ale zejména vojsk (Madzharov 2009, 63-65). Vojenské velení mělo propracovanou hierarchii, rozdělenou dle zeměpisného principu na jednotlivé regiony. Většina armádních jednotek byla umístěna podél Dunaje a také ve městech na pobřeží, avšak vojenské tábory se nacházely i ve vnitrozemí, ovšem v menším počtu (Velkov 1977, 63-68).

Již v průběhu 3. a zejména na počátku 4. st. n. l. začíná veřejný život ovlivňovat rostoucí oblíbenost křesťanství. Nejvýznamnější křesťanské komunity byly ve Filippopoli, Konstantinopoli (býv. Byzantiu), Hérakleii (býv. Perinthu), Traianúpoli, Augustě Traianě a Odéssu. V Konstantinopoli bylo sídlo arcibiskupa a biskupství byla ve většině velkých městských center. Z tohoto důvodu docházelo k šíření křesťanské víry rychleji ve městech, než na thráckém venkově (Dumanov 2015, 92-94). V této době dochází také na okrajích měst ke stavbě kostelů a bazilik, zatímco na venkově se křesťanská architektura prosazuje až na počátku 6. st. n. l.

Byrokratický aparát koncem 3. st. n. l. a zejména v průběhu 4. st. n. l. narostl a stal se poměrně nákladným na udržování, což vyústilo v růst daňových povinností jednotlivých měst. V provinciích měli největší moc místodržící, kteří zpravidla sídlili v hlavním městě a obklopovali se početným úřednickým aparátem. Důležitým orgánem se staly městské sněmy ({\em concilium}, {\em koinon}) se zástupci městských samospráv, kde se projednávaly otázky regionálního charakteru bezprostředně spojené s chodem provincie. Postupem času tyto sněmy nabyly důležitosti a staly se prostředníkem mezi samosprávou jednotlivých měst a autoritou císaře. Menší osídlení typu vesnice či venkovské usedlosti administrativně a fiskálně spadala pod autoritu města, na jehož území se nacházela. V této době také dochází k přerozdělení půdy: území měst se zmenšují, a naopak se zvětšují soukromé pozemky bohatých jednotlivců či území patřící církvi a armádě (Velkov 1977, 70-76). Tento trend omezení autority měst a nárůst moci bohatých soukromých osob, biskupů a vysokých vojenských úředníků je patrný již od 4. st. n. l. a trvá i v následujících stoletích. Vytváří se tak nová elitní vrstva, která disponuje mocí a ekonomickým kapitálem. Na této relativně malé skupině lidí nicméně leží i většinová daňová povinnost, která se s rostoucími náklady na udržení byrokratického aparátu a armády neustále zvyšovala, což vedlo k prohlubování ekonomické krize (Poulter 2007, 15-16; Dumanov 2015, 100).

Nájezdy Ostrogótů a Vizigótů na severní hranicích Thrákie pokračují i v průběhu 4. st. n. l. V roce 376 n. l. se dokonce část Vizigótů usídlila na území Thrákie, a to se svolením císaře Valenta za podmínek, že budou dodržovat římské právo a vyznávat křesťanskou víru. V té době však došlo i k příchodu dalších kmenů, což vyvolalo poměrně dramatický konflikt, v němž se místní venkovské obyvatelstvo postavilo na stranu Gótů a společně porazili římské vojsko v bitvě u Marcianopole a v roce 378 n. l. u Hadriánopole (Velkov 1977, 34-35). Tato porážka měla za následek destabilizaci regionu, přerušení dosavadních pořádků a ekonomickou krizi, způsobenou vleklými kampaněmi a drancováním země. 

V polovině 5. st. n. l. došlo k ještě ničivějším nájezdům Ostrogótů a Hunů, které měly za důsledek pokračující ekonomický a společenský propad. To mělo za následek nejen omezení zemědělské a řemeslné produkce, ale především i pokles počtu obyvatel. Na přelomu 5. a 6. st. n. l. byla Thrákie napadena kmeny Bulharů, což vedlo k pokračující ekonomické a demografické krizi, zániku sídel a celkové transformaci společnosti směrem k decentralizované vládě lokálních center (Dumanov 2015, 98-99). Tato nová centra byla menší, avšak silně opevněná a sloužila jako {\em refugium} pro okolní obyvatelstvo. Panovníci se snažili za velkých finančních nákladů udržet dunajské hranice, což však ještě více prohlubovalo ekonomickou krizi. Završením byly pak nájezdy Avarů v průběhu 6. st. n. l., vedoucí k rozpadu společenského pořádku a politické autority ve formě státu a k novému uspořádání a pozdějšímu vytvoření bulharského království (Velkov 1977, 40-59).

\subsection[thrákie-v-průběhu-dějin-shrnutí]{Thrákie v průběhu dějin: shrnutí}

Oblast Thrákie představovala důležitou spojnici mezi západem a východem, kde se střetávaly mocenské zájmy všech velmocí a v antice byla tato oblast svědkem mnohých historických zvratů. Politická intervence Řeků, Makedonců ale i Římanů s sebou však nesla i kulturní společenské změny, které měly dlouhodobé důsledky na obyvatele Thrákie, a to jak v pozitivním, tak negativním smyslu. Epigrafické památky představují zrcadlo tohoto politicko-kulturního vývoje a vypovídají mnohem více o tehdejší společnosti, než se může jevit na první pohled.

\stopcomponent