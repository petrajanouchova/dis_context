
\environment ../env_dis
\startcomponent section-literární-a-historiografické-prameny
\section[literární-a-historiografické-prameny]{Literární a historiografické prameny}

Literární zmínky o Thrákii a Thrácích jsou poměrně hojné, a to proto, že Thrákie se nacházela v přímém sousedství Řeky obývaného území světa, v němž měli Řekové a později Římané eminentní zájmy, a také proto, že mnoho Thráků žilo a sloužilo v řeckých městech (Xydopoulos 2010; Janouchová 2013a) a v římské armádě (Dana 2013; Boyanov 2008; Boyanov 2012, 251-269). Literární prameny se bohužel většinou nezaměřují výhradně na popis Thrákie a jejích obyvatel, ale o oblasti zmiňují se spíše v rámci širšího výkladu, pro ilustraci celkového historického děje v oblasti Středomoří. Nicméně i přes tento nedostatek sloužily antické literární prameny jako hlavní výchozí pramen v rámci tzv. hellénizačního přístupu, jednoho z nejčastěji používaných vysvětlení společenských změn jako důsledku kontaktu s řeckou kulturou (Dietler 2005, 33-47; Jones 1997, 33). Z tohoto důvodu věnuji obrazu Thrákie a Thráků v literárních pramenech poměrně velkou pozornost, ale zároveň se zabývám i relevancí informací v nich obsažených pro účely studia mezikulturních kontaktů a následných společenských změn.

\subsection[informativní-hodnota-literárních-pramenů]{Informativní hodnota literárních pramenů}

\index{literární prameny}Literární historiografické prameny jsou důležitým zdrojem informací, ale v případě Thrákie bohužel zdrojem jednostranným. Neexistence thrácké literární tradice má za následek, že veškeré dostupné historické a literární prameny pochází z nitra řecké či případně pozdější římské kultury, což do jisté míry omezuje jejich výpovědní hodnotu s ohledem na informace o thráckém obyvatelstvu. Antická literární díla byla zpravidla psaná autory žijícími mimo danou oblast, kteří se velkou část informací dozvídali přes prostředníky, z doslechu, či dokonce zprostředkovaně s odstupem několika století, což bezpochyby snižuje míru autenticity jejich sdělení. Mnohá literární díla měla navíc své zcela specifické zaměření a jejich obsah byl motivován politickými či jinými cíli, a ne vždy měl za cíl podávat nestranný výklad historické skutečnosti (Derks 2009, 240). Proto je k dochovaným literárním pramenům, které se zabývají Thrákií, vhodné přistupovat s jistou mírou kriticismu a nebrat je za popis skutečné historické reality, ale spíše za literární dílo, vyjadřující mnohdy autorovy soukromé postoje a názory.

Dalším neoddiskutovatelným faktem je i značný hellénocentrismus dochovaných \index{literární prameny}literárních památek, který se částečně přenáší z řeckých pramenů i do pozdější latinsky psané literatury. Fakt, že tato díla byla vytvořena pro řecky či latinsky mluvící publikum, nikoliv pro Thráky samotné, ovlivnil jejich obsah a došlo k přizpůsobení kultuře posluchače ({\em interpretatio Graeca} na poli náboženských představ). Dalším omezením literárních pramenů je zaměření jejich výkladu na horní vrstvy společnosti, a to zejména na politické, náboženské a vojenské elity, a naopak pouze okrajové zmínky o běžné populaci. O mezikulturních kontaktech na úrovni každodenního života běžných lidí se tedy mnoho nedozvídáme ani z literárních pramenů a musíme tento druh informací hledat mezi řádky či se obracet na jiné druhy pramenů, jako jsou archeologické památky a nápisy.

\subsection[thrákie-v-literárních-pramenech]{Thrákie v literárních pramenech}

\index{geografie}Thrákie patří k relativně dobře popsaným regionům antického světa, alespoň co se týká pobřežních oblastí, které byly v archaické době osídleny řecky mluvícími obyvateli.\footnote{Velmi užitečné informace o thráckém pobřeží, řeckých koloniích a obyvatelích poskytuje autor 4. st. př. n. l. Pseudo-Skylax ve svém {\em Periplu}, kap. 67, 1-10 (Shipley 2011, 69-71). Pseudo-Skylax uvádí, že po moři je možné obeplout thrácké pobřeží od řeky Strýmónu až k Séstu za dva dny a dvě noci, ze Séstu do Pontu pak za další dva dny a dvě noci. Odtud pak k řece Istru za tři dny a tři noci. Celkem je tedy dle něj možné obeplout Thrákii po moři za osm dní (67.10), {[}ač prostý součet je dní sedm, poznámka P. J.{]}. Dalším autorem popisujícím především řecká města na pobřeží je Pseudo-Skymnos (646-746), jehož dílo bývá datováno do 1. st. př. n. l. (Kazarow 1949, 143). Důležitým zdrojem z 2. st. n. l. je Klaudios Ptolemaios, který udává rozměry a uspořádání provincií {\em Thracia} a {\em Moesia Inferior} ({\em Geogr}. 3.10-11). Obecně avšak autoři římské doby udávají více detailů jako například vzdálenosti mezi městy (Arrián {\em Peripl. Ponti Euxini} 21-24).} Konkrétní hranice Thrákie se měnily v závislosti na politické situaci a na vnímání daného autora, nicméně pohoří a vodní plochy stanovily poměrně stabilní přírodní hranice, jak je dobře patrné na \in{Mapě}[Apendix2:::1.01a] v \in{Apendixu}[Apendix2:::Apendix2]. Tradičně dle antických pramenů území Thrákie začínalo již za makedonskou {\em Píérií}, ze západu bylo pak ohraničeno řekami {\em Strýmónem} a {\em Nestem}, z jihu Egejským mořem, Marmarským mořem a {\em Helléspontem}, z východu pak Černým mořem. Severní hranice Thrákie jsou méně jasně geograficky ohraničené, ale většinou se za severní hranici považuje řeka {\em Istros} (dn. Dunaj) a pohoří {\em Haimos} (dn. Stara Planina, též Balkán). Horské masivy Rodopy, Pirin a Rila tvořily relativně neprostupnou bariéru, oddělující pobřežní od vnitrozemské Thrákie, a tedy Řeky od Thráků. Spojnice vnitrozemí se Středomořím tvořily zejména řeky {\em Hebros} (dn. Marica-Evros), {\em Tonzos} (dn. Tundža-Tonzos), {\em Strýmón} (dn. Struma-Strymónas) a {\em Nestos} (dn. Mesta-Nestos) a několik cest skrze horské průsmyky.\footnote{Sears 2013, 7-8; Bouzek a Graninger 2015, 12-19; Theodossiev 2011, 2-4; Theodossiev 2014, 157. Někteří autoři zařazují do území obývané Thráky oblasti Bíthýnie v Malé Asii, ostrovy Thasos a Samothráké, a území na severu sahající až k řece Vardar-Morava \index{geografie}(Theodossiev 2014, 157).}

Thrákie je v \index{literární prameny}literárních pramenech z archaické a klasické doby známa pro své nerostné bohatství, úrodnou zemědělskou půdu, nevlídné a kruté zimy, vodnaté řeky, vysoké hory a široké planiny, na nichž se dařilo chovu koní, jak zmiňuje Homér.\footnote{Hom. {\em Il}. 13.1-16; 10.484; srov. Sears 2013, 31; Tsiafakis 2000, 365.} Od dob Archilocha je Thrákie proslulá svými silným vínem a existencí bohatých zlatých a stříbrných dolů, k nimž se jako první z Řeků pokoušeli dostat kolonisté z ostrova Paros.\footnote{Zejména Archilochos, Diehl frg. 2 a 51. V pozdějších dobách se o zdroj nerostných surovin začali zajímat i Athéňané (např. Peisistratos, Hdt. 1.64.1; srov. Isaac 1986; 14-15; Lavelle 1992, 14-22).} Hérodotos také zmiňuje i nadbytek dřeva vhodného ke stavbě lodí, jednu z hlavních komodit antického starověku.\footnote{Např. Hdt. 5.23.2; 7.112.} Celkově je možné říci, že Thrákie byla známá jako plodná, ale poměrně drsná krajina, a podobně řecké prameny líčily i její obyvatele.

\subsection[charakteristika-obyvatel-thrákie-v-literárních-pramenech]{Charakteristika obyvatel Thrákie v literárních pramenech}

\index{Thrákové}První zmínky o Thrácích jako o spojencích pocházejí již z homérských eposů, kde thráčtí králové vystupují po boku řeckých panovníků.\footnote{Nejslavnější polo-mýtický král Rhésos byl znám díky svým pověstným bílým koním a nádherné zbroji (Hom. {\em Il}. 10.435, 495), které se lstí podaří získat Odysseovi.} V archaické literatuře se zmínky o Thrákii týkají spíše její geografie a o jejích obyvatelích se mluví vždy v souvislosti s krajinou, kterou obývají.\footnote{Do této kategorie spadá např. zmínky v homérských eposech, homérských hymnech, Hésiodovi, Hekataiovi, Archilochovi, Simónidovi a v díle dalších archaických básníků (Xydopoulos 2004, 18-20).} Od klasické doby se pak setkáváme se dvěma hlavními tématy, která se prolínají téměř všemi dochovanými literárními prameny: Thrákové jsou vnímáni jednak jako blízcí sousedé a často i spojenci, kteří se podobají společenským uspořádáním a zvyklostmi dávné řecké minulosti, a dále jako krutí a bojovní válečníci, s nimiž není radno přijít do sporu.\footnote{Thuc. 2.96. 2, Hdt. 5.3; ; srov. Tsiafakis 2000, 365-366; Xydopoulos 2007, 600; Sears 2013, 148.} Z těchto důvodů thrácké kmeny\index{thrácké kmeny} často hráli nezanedbatelnou roli v politickém vývoji severního Egeidy jako spojenci Athén. Důležitou roli v 5. a 4. st. př. n. l. hráli zejména panovníci z thráckého kmen Odrysů, a to pro velikost jejich vojska, bohatství, kterým disponovali, a výhodnou geografickou pozici jimi ovládané části Thrákie.

\subsection[společenská-organizace-thráků-z-pohledu-vnějšího-pozorovatele]{Společenská organizace Thráků z pohledu vnějšího pozorovatele}

Dle Strabóna (7.7.4) bylo vnitrozemí směrem na východ od řeky Strýmónu obýváno thráckými kmeny, zatímco pobřeží bylo osídleno řecky mluvícími obyvateli, žijícími ve městech. Hlavním zdrojem obživy bylo pro Thráky zemědělství, chov koní a dobytka, válečné výpravy a do omezené míry zde fungoval i obchod a nepeněžní výměna se Středomořím (Strabo 7.3.7).

Hérodotos (5.3) uvádí, že Thrákové představují druhý nejpočetnější národ v tehdy známém světě hned po Indech.\footnote{Ač Hérodotos používá slovo národ či kmen (ἔθνος) a Thrákové byli v řeckém prostředí vnímáni a popisováni jako jedno etnikum, ve skutečnosti se jednalo o mnoho kmenů, které spadaly pod jednotné označení „thrácké”.} Sám Hérodotos uznává, že Thrákové jsou nejednotní, a tím přichází o značnou strategickou výhodu, jakou jim velké množství obyvatel přináší. Spojuje je však geografická blízkost obývaných území a, až na výjimky, podobné zvyklosti. Pozdější zdroj Plinius Starší uvádí jména 36 kmenů s relativní polohou oblasti, kterou obývají (Plin. {\em H. N.} 4.11). Zároveň ale zmiňuje existenci 50 stratégií, tedy administrativních jednotek, které pravděpodobně vycházely z kmenového principu (Theodossiev 2011, 8).\footnote{Plin. {\em H. N}. 4.11. 40: „{\em Thracia sequitur, inter validissimas Europae gentes, in strategias L divisa}.”} Strabón uvádí pouze 22 kmenů pro celou Thrákii (7 frg. 48; 7.3.2), avšak dá se předpokládat, že jich celkem bylo až dvakrát tolik. Fol a Spiridonov (1983, 21-61) celkem sesbírali na 50 jmen kmenů, které je možné označit jako thrácké a které se vyskytovaly v písemných pramenech do poloviny 3. st. př. n. l. Některé z kmenů prameny zmiňují jen jednou či ojediněle, o jiných víme poměrně hodně detailů, jako např. o Odrysech, Getech, Sapaích atp.\footnote{Sears (2013, 9-13) uvádí nejdůležitější thrácké kmeny\index{thrácké kmeny} vyskytující se v řeckých \index{literární prameny}literárních pramenech: Apsinthiové, Bessové, Bisaltové, Bíthýnové, Diové, Dolonkové, Édónové, Mygdóni, Thýnové, Odomanti, Odrysové a Satrové.}

Hérodotos dělí thrácké kmeny dle vztahu k Řekům a řecké kultuře jednak na civilizované kmeny, které jsou v kontaktu s řeckou komunitou a jsou do velké míry provázány společnými zájmy a žijí v dosahu pobřeží a řeckých kolonií,\footnote{Hdt. 7.110 a 7.115; Hdt. 6.34 např. kmen Dolonků.} a dále na kmeny žijící v horách, které jsou na Řecích nezávislé či jsou vůči nim nepřátelsky naladěné, mají bojovný charakter a udržují si své tradiční zvyky.\footnote{Hdt. 7.111 a 7.116, např. kmen Satrů či Hdt. 6.36 a 9.119 kmen Apsinthiů, Hdt. 5.124-126 kmen Édónů, Hdt. 6.45 kmen Brygů. V lidové tradici a mýtech je Thrákie je taktéž vnímána jako domov mýtického pěvce Orfea, a místo, kde se v odlehlých horách odehrávají divoké bakchické rituály (Pindar frg. 126.9; Paus. 6.20.18; srov. Archibald 1999, 460).} Postoj thráckých aristokratů vůči řecké komunitě se různí, od nepřátelských kontaktů až po snahu některých jedinců o adopci řeckého stylu života a vzdělání.\footnote{Hdt. 4.78-80: Hérodotos popisuje situaci u Skythů, přímých sousedů Thráků, u nichž se předpokládá velká podobnost chování a kulturních norem. Autor mluví o skýthském princi Skýlovi, který pocházel ze smíšeného svazku a uměl mluvit a psát řecky a byl nakloněn řeckému způsobu života, např. nosil řecký oděv, vyznával řecká božstva, a nakonec se usídlil v řeckém Borysthenu.} Ač jsou tito Thrákové řeckými prameny popisováni jako „hellénizovaní”, i nadále si udržují své typické zvyky a náboženství (Hdt. 5. 3-8).\footnote{Známá pasáž z Hérodota (Hdt 5. 7) je zářným příkladem: Hérodotos tvrdí, že Thrákové uctívají Área, Dionýsa, Artemidu a aristokraté ještě navíc Herma. Zůstává nadále otázkou, nakolik řecké prameny přizpůsobovaly reálie řeckému posluchači ({\em interpretatio Graeca}) a nakolik skutečně reflektovaly realitu. Nicméně srovnání s dostupnými archeologickými prameny poukazuje na uchování tradičního charakteru náboženství za užití řecké nomenklatury (Janouchová 2013a; Janouchová 2013b). Více k tomuto tématu v \in{kapitole}[chapter-Epigrafická-produkce-napříč-staletími:::chapter-Epigrafická-produkce-napříč-staletími].} Ke kontaktům a prolínání kulturních zvyklostí mezi Thráky a Řeky docházelo i dle Helláníka (frg. 71a), který nazývá původně thrácké obyvatele polovičními Řeky, avšak jeho popis se týká pouze thráckých kmenů žijících na pobřeží v těsné blízkosti řeckých měst.\footnote{Termín μιξέλληνες Helláníkos používá k popisu obyvatel thráckého pobřeží známého jako Kolpos Melas, mezi Thráckým Chersonésem a pevninou.}

Jednotlivým kmenům vládli kmenoví vůdci či dle řeckých pramenů králové, kteří ve svých rukou soustředili jak politickou, tak ekonomickou moc.\footnote{Thúkýdidés a Hérodotos používají termín οἱ βασιλέες, králové (Thuc. 2.97; Hdt. 5.7; 6.34); Diodóros pak při popisu odryského panovníka Sitalka používá termín ὁ τῶν Θρᾳκῶν βασιλεὺς, král Thráků (D. S. 12.50), srov. Archibald (2015, 912).} S řecky mluvícími městy udržovali kontakty právě tito thráčtí aristokraté, kteří pocházeli z kmenů žijících v blízkosti řeckých sídel na pobřeží. Zásadní roli mezi thráckou aristokracií hrálo společenské postavení a status, který si aristokraté pečlivě budovali. Hérodotos nás informuje o rozšířené praxi mnohoženství, které bylo známkou společenské prestiže, protože několik žen si mohl dovolit jen bohatý člověk (Hdt. 5.5; Strabo 7.3.4). Další známkou společenského postavení u Thráků byl zvyk přijímat, a nikoliv dávat bohaté dary, který popisuje Xenofón (Xen. {\em Anab.} 7.3.18-20). Čím vyšší bylo společenské postavení obdarovaného, tím větší a cennější dar se očekával. Úspěšný aristokrat si takto mohl opatřit poměrně velké bohatství, jak to Thúkýdidés popisuje v případě odryského panovníka Sitalka (2.97.3; D. S. 12.50). Bohatství pak Thrákové vystavovali na odiv v rámci společenských setkání, ale i zhotovováním nákladných hrobek a konáním pohřebních rituálů, které u nich byly ve velké oblibě (Hdt. 5.8).

Politické vztahy a vzájemné postavení řeckých měst a thráckých kmenů literární prameny přímo nezmiňují, nicméně Thúkýdidés v náznacích informuje o jistém druhu finanční nadřazenosti Thráků nad Řeky žijícími v Thrákii, když poukazuje na nemalé finanční částky, které kmen Odrysů vybíral právě od Řeků žijících na území Thrákie (Thuc. 2. 97). Kmen Odrysů si získal výsadní postavení v řecké historiografii, zejména pro svou důležitost v rámci athénské zahraniční politiky v 5. a ve 4. st. př. n. l., a tudíž nelze vztahovat vyjádření týkající se Odrysů obecně na všechny thrácké kmeny. Je však jisté, že Thrákové po většinu své historie netvořili jednotný stát, který by zahrnoval celou oblast Thrákie, ale spíše se jednalo o kmeny s podobnými zvyky a příbuznými dialekty (Theodossiev 2011, 2), které se více či méně dostávaly do kontaktu s řeckou komunitou, ale udržovaly si autonomní status. Pokud však vysoce postavení Thrákové vycítili politickou příležitost, dokázali využít situace za účelem naplnění vlastních mocenských ambicí, a to i za cenu ztráty autonomie v dlouhodobém měřítku (Haynes 2011, 7). To je případ thráckých panovníků z 1. st. př. n. l. a 1. st. n. l., kteří se vzdali politické autonomie a stali se vazaly Říma výměnou za vlastní prospěch (Tac. {\em Ann}. 4.46-47; srov. Lozanov 2015, 75).

\subsection[literární-topos-nevyzpytatelní-spojenci-a-nebezpeční-nepřátelé]{Literární topos: nevyzpytatelní spojenci a nebezpeční nepřátelé}

Obraz Thráků v řecké literatuře se do značné míry vyvíjí v závislosti na tehdejší politické situaci, v zásadě jsou mnohými \index{literární prameny}literárními autory chápáni jako blízcí a mocní sousedé, kteří však nedosahují stejné civilizační a kulturní úrovně jako Řekové. Ač se může zdát, že tento motiv se prolíná napříč celou řecky psanou literaturou, ne vždy se muselo nutně jednat o pouhý popis skutečnosti, ale spíše o literární {\em topos}, jakýsi všeobecně přijímaný názor, zvolený záměrně autorem daného díla. Tento dualismus se do velké míry přenesl i do literatury pozdějšího období a stal se tak typickým stereotypem zobrazování Thráků v literatuře, ale například i v malířském umění a ikonografii (Tsiafakis 2000, 388-389). V moderní době se tento dualistický popis Thráků stal základem interpretací vycházejících z hellénizačního teoretického přístupu.

\subsubsection[thrákové-a-literární-žánr-historiografie-5.-a-4.-st.-př.-n.-l.]{Thrákové a literární žánr historiografie 5. a 4. st. př. n. l.}

Hérodotos představuje jeden z nejinformativnějších etnografickcýh popisů Thráků, nicméně i přesto je Hérodotovo líčení Thráků a Thrákie značně selektivní a útržkovité, zejména protože se u posluchačstva předpokládala předchozí znalost tématu (Asheri 1990, 162). Jedním z cílů Hérodotova vyprávění bylo posluchače pobavit a zaujmout novými a zvláštními fakty (Asheri 1990, 162). Vzhledem k tomu, že v Athénách, kde Hérodotos také působil, a i v jiných městech žilo poměrně velké množství thráckého obyvatelstva, nemusel své posluchače seznamovat se základními skutečnostmi o Thrácích, jako tomu třeba bylo v případě Aithiopů či jiných národů ze vzdálenějších a exotičtějších končin (Sears 2013, 149).\footnote{Jedna z nejznámějších zmínek o thráckém obyvatelstvu v Athénách pochází z díla samotného Platóna. Platón na začátku Ústavy popisuje epizodu, kdy se Sókratés šel podívat do Peiraea na noční slavnost {\em Bendideí}, pořádanou místními Thráky (Plat. {\em Resp}. 1.327a). Bendis byla původně thrácká bohyně, která byla uvedena do oficiálního athénského pantheonu pravděpodobně v souvislosti se spojenectvím s Odrysy v r. 431 př. n. l. Její slavnosti se ve 4. st. př. n. l. každoročně odehrávaly v athénském přístavu za hojné účasti Thráků, kteří zde trvale žili jako {\em metoikové}, či jako pomocníci v domácnosti, případně otroci ve stříbrných dolech (Archibald 1999, 457-460; Janouchová 2013a, 98; Sears 2013, 149-157).} Proto Hérodotos popisuje Thráky jako skupinu lidí relativně podobnou řeckému posluchači a záměrně vyzdvihuje zajímavosti a jasné odlišnosti jako je užívaný pohřební ritus, přítomnost tetování, zvláštnosti ve výchově potomků, postavení žen ve společnosti, nebo jejich víra v nesmrtelnost (Hdt. 5.4-6).

Historik Thúkýdidés, jehož rodina pravděpodobně pocházela z Thrákie, je považován za věrohodný zdroj informací, avšak i on podává svůj výklad za účelem osvětlit politické dění a průběh peloponnéské války.\footnote{Sears 2013, 14: Thúkýdidés, syn Olorův, jehož rodina pocházela z Thrákie (Thuc. 4.104), se sám považoval za Athéňana (Thuc. 1.1). Sám přiznává, že jeho rodina měla právo dolovat drahé kovy v Thrákii (Thuc. 4.105), a tudíž byl pravděpodobně obeznámen se situací v Thrákii detailněji, než většina Athéňanů.} Když Thúkýdidés mluví o obyvatelích Thrákie, používá společné pojmenování Thrákové, ale také rozeznává jednotlivé kmeny a jejich odlišný politický status, tj. autonomní kmeny vs. závislé kmeny, jejich postoj vůči Řekům a oblast, kterou obývají (Thuc. 2.96). Ve většině případů Thúkýdidés mluví specificky o aristokratech z kmene Odrysů, kteří se stali athénskými spojenci v roce 431 př. n. l. (2.29, 2.95-97). Ve svém líčení zdůrazňuje svrchovanost Odrysů nad ostatními thráckými kmeny, velikost a důležitost oblasti, které vládnou. Thúkýdidés zde používá termín říše, aby poukázal na významnost spojenectví s Odrysy (ἀρχή, Thuc. 2.97). O Odrysech se nevyjadřuje jako o barbarech, ale zdůrazňuje jejich bohatství a moc, zejména aby ospravedlnil důvody vedoucí k uzavření spojenectví v roce 431 př. n. l.\footnote{V jiném kontextu se zmiňuje o blíže nespecifikovaných kmenech obývajících Rodopy, kteří byli podle něj nejbojovnější ze svobodných thráckých kmenů, nespadali pod odryskou říši (Thuc. 2.98.4), avšak přišli Odrysům na pomoc v nouzi. Dodává tak vážnost odryské říši i v rámci nezávislých thráckých kmenů a poukazuje na sílu thráckého vojska.}

Thúkýdidés nazývá Thráky barbarskými a krvelačnými pouze ve specifických kontextech, kde thráčtí vojáci sehráli roli v rámci peloponnéské války. V případech, kdy Thúkýdidés popisuje Thráky jako hrubé a bojovné, je to vždy, aby podtrhl jejich kvality coby válečníků či aby zdůraznil vážnost konkrétních historických situací. V jedné z těchto epizod popisuje thrácké žoldnéře z kmene Diů jako jedny z nejkrvelačnějších barbarů, avšak vztahuje tuto charakteristiku obecně na celé etnikum (Thuc. 7.29).\footnote{Jedná se o epizodu související s athénskou výpravou na Sicílii, kdy Athéňané povolali thrácké žoldnéře, aby se zúčastnili výpravy. Ti však přijeli do Athén příliš pozdě a byli posláni zpět bez jakékoliv finanční náhrady. Athénský generál Dieitrefés a thráčtí vojáci se rozhodli opatřit si slíbené peníze vypleněním boiótského Mykaléssu a povražděním jeho obyvatel a vypleněním chrámů. Tato akce byla vnímána velmi negativně, zejména kvůli krutosti počínání Thráků, ale i vzhledem k faktu, že vedení se ujal athénský generál, který Thráky využil k vlastnímu finančnímu zisku (Kallet 2002, 140-146).} Ve většině případů Thrákové vystupují v Thúkýdidově díle jako sousedé a spojenci, jejichž vojenské schopnosti by Athéňané rádi využili ve svůj prospěch, avšak divoká povaha a bojovnost Thráků v tom často brání.

Další historik Xenofón v šesté a sedmé knize {\em Anabasis} popisuje své vlastní zkušenosti s Thráky během tažení perského prince Kýra Mladšího a dále behěm služby pro thráckého panovníka Seutha II.\footnote{Xenofón, spolu s dalšími Řeky působil jako najatý žoldnéř ve službách Kýra, a když Kýros zemřel, Řekové se dali na ústup zpět do Evropy. Xenofón se sám cestou dostal do služeb thráckého prince Seutha, pozdějšího odryského panovníka Seutha II.} Xenofón popisuje Thráky jako skvělé válečníky a při mnoha příležitostech zdůrazňuje jejich divoký a nespoutaný charakter. O Seuthovi jako o krutém panovníkovi, který se v rámci udržení své autority neváhal uchýlit k pálení vesnic a zabíjení lidí (Xen. {\em Anab.} 7.4.1; 7.4.6). Xenofón s ním uzavřel dohodu a přísahal na vzájemné přátelství, to však nemělo dlouhého trvání (Xen. {\em Anab.} 7.3.30). Konkrétně o Seuthovi Xenofón vždy hovoří jako o Thrákovi, nikoliv o příslušníkovi kmene Odrysů, ač si byl vědom i existence kmenové organizace mezi Thráky (Xen. {\em Anab.} 7.2.22). Specifika thráckých zvyklostí a společenského uspořádání zdůrazňoval zejména v případech, kdy se odlišovaly od řeckých či kdy se jejich neznalostí sám dostal do potíží (Xen. {\em Anab.} 7.3.19-21). Matthew Sears navrhuje (2013, 115), že se Xenofón takto ostře vymezoval vůči Seuthovi a ostatním Thrákům, aby vyvrátil případné domněnky o své náklonnosti vůči Thrákům a zejména jejich bohatství a zároveň, aby ospravedlnil své činy v očích řeckého čtenáře. Seuthés totiž nabídl Xenofóntovi za své služby nejen peněžní odměnu, sídlo v Bisanthé na thráckém pobřeží, ale i svou vlastní dceru za ženu, což Xenofón poměrně ochotně přijal (Xen. {\em Anab.} 7.2.38). Po nějakém čase se ale přišlo na to, že Seuthés dle Xenofónta neplatil vojákům slíbené odměny, a tudíž mezi nimi došlo k rozepři a ukončení spolupráce a přátelství (Xen. {\em Anab.} 7.7.48-54). Xenofón tak mohl, ve snaze se od Seutha distancovat, hodnotit některé Seuthovy akce negativněji, než by si bývaly zasloužily.

\subsubsection[thrákové-a-attická-komedie-a-řečnictví-5.-a-4.-st.-př.-n.-l.]{Thrákové a attická komedie a řečnictví 5. a 4. st. př. n. l.}

Attická komedie poskytuje kritický obraz politických událostí, v nichž Thrákie hrála důležitou roli, a dále přináší informace o tom, jak Athéňané obecně vnímali Thráky. Jak správně podotýká Matthew Sears (2013, 18), historická skutečnost je v komediích často překroucená tak, aby pobavila a lépe navazovala na děj, a tudíž dochované narážky jsou spíše reflexí skutečných událostí, než jejich přímým a nestranným vyobrazením. Příkladem takového komentáře k aktuální politické situaci může být Aristofanova hra {\em Acharňané}, kde zmiňuje nedávno uzavřené spojenectví s Odrysy v roce 431 př. n. l. (Aristoph. {\em Ach.} 134-173).\footnote{Tato událost je známá i z jiných zdrojů (Thuc. 2.29; srov. Janouchová 2013a, 96-97; Sheedy 2013, 46 zmiňuje numismatické prameny, poukazující na spojenectví mezi Athénami a Thrákií okolo r. 430 př. n. l.), a je možné ji tak považovat za věrohodný historický rámec, na němž Aristofanés založil epizodu týkající se Thrákie a Thráků.} Aristofanés nazývá Thráky barbary, a to v přímém protikladu k Athéňanům, nicméně současně používá obecné pojmenování Thrákové pro popis etnika jako takového, ale rozlišuje i mezi jednotlivými thráckými kmeny a vyzdvihuje jejich negativní vlastnosti. Zejména pak žoldnéři z kmene Odomantů jsou podle něj jedni z nejbojovnějších válečníků a představují zhoubu pro Athény, požírající athénské bohatství.\footnote{Aristoph. {\em Ach}. 153: μαχιμώτατον Θρᾳκῶν ἔθνος.}

Poprvé se Thrákové objevují pod označením {\em barbaroi} v negativním slova smyslu až v attickém řečnictví, a to z politických důvodů (Xydopoulos 2004, 18-20). Ioannis Xydopoulos správně podotýká, že obraz Thráků jako barbarů v literatuře klasické doby velmi záležel na konkrétní politické situaci v Athénách a na politickém cíli, který si daný autor vytkl (Xydopoulos 2010, 214-221). Attické řečnictví 5. a 4. st. př. n. l. poukazovalo na kulturněspolečenskou převahu řeckých obcí a využívalo protibarbarské rétoriky výhradně pro politické účely (Souček 2011, 40-42).\footnote{Např. Ísokratés {\em Panegyrikos} 50; Démosthenés {\em De Chersoneso} (8.43-45) či {\em De Corona} (18.25-27).}

Řečník Démosthenés se vyjadřuje o Thrákii v souvislosti s tažením Filipa Makedonského do Thrákie. Řeky varuje před nebezpečím v podobě makedonského vpádu a Thrákii používá jako odstrašující případ (8.43-45). O Thrákii hovoří o jako relativně chudé, zemědělské oblasti s obtížnými klimatickými podmínkami a nejednotným vedením v podobě thráckých „králů” (23.8-11), kterou se i přesto Filip rozhodl získat, a je tedy pouze otázkou času, kdy se obrátí proti bohatým Athénám. O Thrácích hovoří jako o barbarech, ale zároveň jako o spojencích (18.25-27). Barbary Thráky nazývá, když popisuje zradu Kotya vůči Ífikratovi (23.130-132), nicméně je nepovažuje za krutější než samotné Řeky (23.169-170). Spojence z řad Thráků považuje za morálně na vyšší úrovni, než nepřátele Athén z řad Řeků, a to zejména v jednání se zajatci. Opačné názory zastává Démosthenův oponent Aischinés, který ve snaze vyvrátit Démosthenovu argumentaci zpochybňuje Thráky jako spojence Athéňanů a poukazuje na jejich slabé stránky a na jejich nespolehlivost coby spojenců, proto je jeho líčení Thráků velmi negativní (2. 81-86, 89-93).

\subsubsection[thrákové-a-pozdní-literární-prameny]{Thrákové a pozdní literární prameny}

Zhruba od poloviny 4. st. př. n. l. se Thrákie a Thrákové téměř vytrácejí z literárních pramenů a objevují se znovu opět v 1. st. př. n. l. v řecky i latinsky psaných pramenech. Prameny z římské doby, tj. z 1. až 4. st. n. l., mluví o Thrákii v rámci vývoje politické situace v římské říši či pro ilustraci pozadí významných událostí odehrávajících se právě v Thrákii či bezprostředním okolí.\footnote{Např. Tacitus 2.64-7; 3.38; 4.46-51; Suetonius {\em Aug}. 3-4; {\em Tib}. 37; Paus. 10.19.5-12; Dio. Cass. 51.23.2-27.2. Z pozdních autorů také Ammianus Marcellinus 22.8; 27.4. Pro kompletní přehled pramenů zmiňující Thrákii viz Kazarow 1949; Velkov {\em et al.} 1981.} Thrákové jsou většinou vnímáni jako odbojní poddaní římské říše, kteří se čas od času vzbouří proti autoritám. Thrákie je součástí římského impéria, nicméně se nachází po většinu času na pokraji politického zájmu. \index{literární prameny}Literární autoři tedy v duchu této tradice většinou necítí potřebu vytvářet individuální pojednání o Thrákii a v mnohém vychází z předcházející literární tvorby. Zejména geografické a etnografické popisy Thrákie autorů působících v římské době do velké míry vychází z dříve píšících řeckých autorů jako je Hérodotos, Thúkýdidés, Xenofón či Eforos, historik 4. st. př. n. l.

Poměrně obsáhlé pojednání o zeměpise Thrákie a jejích obyvatelích sepsal na přelomu 1. st. př. n. l. a 1. st. n. l. Strabón (7.3; 7.5-7.7), který se věnuje obecnému popisu thráckých kmenů, zejména pak však Getů, jejich zvyků, a uvádí několik stručných výňatků z historie, vždy v souvislosti s historií řeckou. Další systematické, avšak stručné, pojednání o obyvatelích římské Thrákie poskytuje Pomponius Mela, autor 1. st. n. l. (2.2.16-33). Mela do velké míry vycházel z Hérodota, ale i přesto se však jedná o jeden z nejucelenějších etnografických popisů římské Thrákie (Dimova 2014, 36). Dále se k Thrákii systematicky vyjadřuje v 1. st. n. l. i Plinius Starší (4. 11. 40-50), který částečně vychází právě z díla Pomponia Mely (Romer 1998, 27). Jeho deset odstavců dlouhý popis je součástí monumentálního díla {\em Naturalis Historia} a i přes malý rozsah patří k nejinfomativnějším zdrojům o geografii a tehdejším uspořádání Thrákie. Dalším významným zdrojem je Diodóros Sicilský, který sepsal své historické dílo {\em Bibliothéké historiké} v 1. st. př. n. l., přičemž místy čerpal z dřívějších řecky píšících autorů, zejména historika Efora pro události 5. st. př. n. l.\footnote{I přesto je dílo Diodóra velmi cenným historickým pramenem, vzhledem k tomu, že mnohé události popisuje jako jediný dochovaný zdroj. Diodóros epizodicky zmiňuje historické události, kde Thrákové vystupují v rámci širšího kontextu, jako např. v době uzavření spojenectví Odrysů s Athénami (D. S. 12.50-51), Alkibiadovo spojenectví s odryskými panovníky (D. S. 13. 105), Xenofóntovo tažení skrz jihovýchodní Thrákii (D. S. 14.37), thrácké spojenectví s Thrasybúlem (D. S. 14.94), útok Thráků na Abdéru (D. S. 15.36). Jen ve fragmentech se dochoval popis makedonského tažení do Thrákie a následné vlády diadocha Lýsimacha, včetně jeho sporů s Thráky a jeho zajetí thráckým vládcem Dromichaitem (18.3; 18.14; 21.11-12).} Diodóros hovoří o Thrácích jako o spojencích - tehdy figurují jako rovnocenní partneři, kteří však mají odlišné kulturní zvyky a jsou na primitivnějším stupni vývoje společnosti (D. S. 14. 83; 15.36; 18.14), anebo jako nepřátelé a krvelační barbaři. V jeho díle se tak setkávají obě dvě tradice stereotypního popisu Thráků, jejichž užití závisí na konkrétní situaci.

Podobně jako dřívější řecké zdroje, i prameny doby římské se věnují spíše životu elit a jejich rodin, nikoliv běžných Thráků. Obyvatelstvo autoři charakterizují zcela v duchu stereotypního zobrazení jako drsné, divoké a bojovné (Pom. Mela 2.2.18; Strabo 7.3.7), a to obyvatelstvo nejen současné, ale i minulé (Arr. {\em Anab}. 1.3-6; srov. Haynes 2011, 7-8). Thrákové jsou i v době římské vnímání jako dobří válečníci a vojáci, a to pravděpodobně vzhledem k faktu, že jich velký počet sloužil v římské armádě přímo v Thrákii či i mimo ni (Boyanov 2008; 2012, 251-269; Dana 2013, 219-269). Pojem barbaři se objevuje zejména ve spojení s vojenskými akcemi Thráků proti Římanům, ale nikoliv jako obecný pojem, který by charakterizoval všechny Thráky (Tac. {\em Ann.} 4.51).

\subsection[thrákie-a-thrákové-v-literárních-pramenech-shrnutí]{Thrákie a Thrákové v literárních pramenech: shrnutí}

Literární prameny se vyjadřují o Thrácích poměrně stereotypně po celou dobu antiky. Ač Thrákie nikdy nestála v popředí zájmu autorů, i přesto se objevovala poměrně často v narážkách, vedlejších příbězích či jako ilustrace širšího společensko-historického rámce. Z dochovaných četných zmínek o Thrákii a Thrácích vyplývá, že ke vzájemným kontaktům docházelo poměrně běžně, a to jak na úrovni diplomatických styků, tak i na úrovni obchodních kontaktů a každodenních styků. K vzájemnému ovlivňování docházelo např. i na poli náboženství a kulturních zvyklostí, ač informace o vzájemném vlivu jsou zprostředkované skrze médium literární tvorby a často s odstupem několika století.

Historiografické prameny 5. a 4. st. př. n. l. daly základ všem pozdějším dílům. Právě z této doby pochází i dva základní stereotypy pevně svázané s charakterizací Thráků. Jako červená nit se vine literárními prameny zvěst o thrácké statečnosti, zálibě v boji a často až hrubosti, která byla jistě i v mnoha případech zasloužená. K Thrákům autoři přistupují dle aktuální politické situace a popisují je tak, aby dosáhli kýženého efektu. V dobách spojenectví jsou Thrákové líčeni jako rovní partneři, z trochu odlišnými zvyklostmi a vírou. V dobách, a v situacích, kdy jsou vnímáni jako nepřátelé, literární prameny mluví o barbarech, divokých a krutých stvořeních, která jsou kulturně primitivnější v porovnání s Řeky a Římany. Ani tento fakt však nebrání, aby Thrákové hráli roli rovnocenných partnerů a obávaných protivníků, a to jak v řeckém kulturním okruhu, tak i v prostředí římské říše.

Literární prameny rozlišují jednotlivé kmeny a uznávají odlišnost společenského uspořádání Thrákie. Řečtí autoři právě v Thrácích vidí svůj vlastní předobraz polomýtické minulosti, kdy společnost ovládali kmenoví vůdcové a hlavním měřítkem úspěšnosti byla společenská prestiž. Kontakty a kulturní výměnu s Thráky řecké prameny do určité míry reflektují, ale nemluví však o běžných obyvatelích, ale téměř výhradně o politických elitách. Navíc je popis thrácké společnosti značně nesystematický a selektivní, často přizpůsobený řeckému publiku. Proto je vhodné k literárním pramenům přistupovat obezřetně, pečlivě zvažovat kontext jejich vzniku a nepovažovat je automaticky za přesný obraz tehdejší společnosti. Pro doplnění a rozšíření je vhodné se obrátit na archeologické a epigrafické prameny, které mají tu výhodu, že pochází přímo od obyvatel Thrákie, a nikoliv od vnějších pozorovatelů. Jejich výpovědní hodnota není zprostředkovaná sítem řecké a římské literární tradice a dává nám tak nahlédnout do nitra komunity samotné.

\stopcomponent