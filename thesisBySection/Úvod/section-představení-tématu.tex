
\environment ../env_dis
\startcomponent section-představení-tématu
\section[představení-tématu]{Představení tématu}

V 19. a 20. století byla Thrákie v duchu tehdy oblíbených akulturačních teorií považována za sféru vlivu řecké kultury, jejíž pouhá přítomnost v regionu měla za následek nevyhnutelnou {\em hellénizaci} místního obyvatelstva. Hellénizační přístup sloužil v té době jako univerzální vysvětlení kulturních změn, k nimž docházelo při kontaktu místního obyvatelstva s řeckým světem. Řecká kultura byla považována za civilizačně vyspělejší a neřecké komunity pak v rámci postupného a nevratného procesu hellénizace přijímaly novou materiální kulturu a ideologické koncepty na úkor vlastní kultury i identity vlastní a přijetí identity pořečtěné (Dietler 2005; Vranič 2014). Stejný přístup byl aplikován i na projevy epigrafické produkce, aniž by byla vzata v potaz specifika epigrafického materiálu a zhodnocena výpovědní hodnota nápisů v rámci dané problematiky. Užívání řeckého písma a vydávání nápisů v řeckém jazyce bylo automaticky považováno za jeden z typických znaků pořečtění thráckého obyvatelstva a přijetí řecké kultury za vlastní, za cenu ztráty identity původní. Tento proces v oblasti Thríkie pokračoval i když se stala součástí římského impéria, ba i dokonce nabral v určitých ohledech na intenzitě (Mihailov 1977, 343-344; Samsaris 1985, 235; Loukopoulou 1989, 198; Sharankov 2011, 135-145; Vranič 2014a, 39).

\stopcomponent