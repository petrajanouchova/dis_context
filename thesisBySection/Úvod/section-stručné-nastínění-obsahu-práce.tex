
\environment ../env_dis
\startcomponent section-stručné-nastínění-obsahu-práce
\section[stručné-nastínění-obsahu-práce]{Stručné nastínění obsahu práce}

V druhé části této úvodní kapitoly seznamuji čtenáře s geografickým rozsahem území Thrákie a charakterem jejich obyvatel tak, jak o nich hovoří dochované antické literární prameny, ale i moderní sekundární literatura. V \in{kapitole}[chapter-Mezikulturní-kontakty-v-antickém-světě:::chapter-Mezikulturní-kontakty-v-antickém-světě] pojednávám o teoretických přístupech problematiky mezikulturních kontaktů a jejich projevů v materiální kultuře z pohledu archeologie a historických věd. Hlavní pozornost věnuji revizi hellénizačního přístupu, jako jednoho v minulosti z nejčastěji používaných teoretických konceptů. Zároveň probírám současné teoretické přístupy zabývající se mezikulturním kontaktem a snažím se nabídnout alternativní teoretický směr, který by se věnoval všem zúčastněným stranám stejnou měrou. V \in{kapitole}[chapter-Nápisy-a-jejich-hodnota-pro-studium-antické-společnosti:::chapter-Nápisy-a-jejich-hodnota-pro-studium-antické-společnosti] se zabývám specifiky epigrafického materiálu, s důrazem na teoretické zhodnocení přínosu nápisů pro studium antické společnosti. Dále zde podrobněji představuji základní principy, s nimiž přistupuji ke studiu epigrafického materiálu. V \in{kapitole}[chapter-Metodologie-a-zvolená-metoda:::chapter-Metodologie-a-zvolená-metoda] se zabývám použitou metodologií a organizací práce, která vysvětluje uspořádání a obsah následujících analytických kapitol. \in{Kapitola}[chapter-Epigrafická-produkce-v-Thrákii:::chapter-Epigrafická-produkce-v-Thrákii] nahlíží na analyzovaný soubor nápisů jako na celek a představuje jeho základní charakteristické rysy. \in{Kapitola}[chapter-Epigrafická-produkce-napříč-staletími:::chapter-Epigrafická-produkce-napříč-staletími] představuje chronologický přehled datovaných nápisů a předkládá detailní náhled na epigrafickou produkci v jednotlivých stoletích. V \in{kapitole}[chapter-Mapování-kulturních-změn:::chapter-Mapování-kulturních-změn] se zabývám rozmístěním nápisů v krajině a vztahem mezi nárůstem společensko-politické organizace v Thrákii. V poslední \in{kapitole}[chapter-Závěr:::chapter-Závěr] shrnuji výsledky současné práce a nastiňuji další možné směřování zvolené problematiky. Na text práce samotné navazují přílohy, které obsahují detailní informace k organizaci databáze, a zejména pak tabulky a grafy s výsledky analýz a soubor map časoprostorového rozmístění nápisů.

\stopcomponent